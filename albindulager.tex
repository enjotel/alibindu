\chapter{Critical Edition \& Annotated Translation}
\clearpage
\begin{alignment}[
    texts=edition[class="edition"];
    translation[class="translation"],
  ]
\begin{edition}
 \ekddiv{type=ed}
    \centerline{\textrm{\small{[Introduction]}}}
    \bigskip
    \begin{prose}
%--------------------------
% śrī gaṇeśāya namaḥ /                                                     rājayogāntargataḥ //  binduyogaḥ   \E 
% śrī gaṇeśāya namaḥ /                                                     atha tattvabiṃduyogaprāraṃbhaḥ     \L
% śrī ṇe ya maḥ /                                                          atha rājayoga         liṣyate      \P
% \om                                                                                                         \B      
% śrī gaṇeśāya namaḥ // śrī gurave namaḥ //                                atha rājayogaprakāro  likhyate //  \N1
% śrī gaṇeśāya namaḥ //                                                //  atha rājayogaprakāro  likhyate //  \N2
% śrī gaṇeśāya namaḥ // śrī sarasvatyai namaḥ // śrī nirañjanāya namaḥ //  atha rājayogaprakāro  likhyate //  \D
% \om                                                                                                         \D2
% śrī gaṇeśāya namaḥ /  oṃ śrī niraṃjanāya //                              atha rājayogaprakāra  likhyate //  \U1
% śrī gaṇeśāya namaḥ /                                                     atha rājayoga         likhyate //  \U2
%--------------------------
%Homage to Śrī Gaṇeśa. Now the methods of rājayoga are laid down.
%--------------------------          
\noindent \app{\lem[wit={ceteri}]{śrī gaṇeśāya namaḥ}
        \rdg[wit={P}]{śrī ṇe ya maḥ}
        \rdg[wit={N1}]{śrī gaṇeśāya namaḥ || śrī gurave namaḥ ||}
        \rdg[wit={D}]{śrī gaṇeśāya namaḥ || śrī sarasvatyai namaḥ || śrī nirañjanāya namaḥ ||}
        \rdg[wit={U1}]{śrī gaṇeśāya namaḥ || oṃ śrī niraṃjanāya ||}}\dd{}
\app{\lem[wit={N1,N2,D}]{atha rājayogaprakāro likhyate}
        \rdg[wit={U1}]{atha rājayogaprakāra likhyate}
        \rdg[wit={E}]{rājayogāntargataḥ || binduyogaḥ}
        \rdg[wit={L}]{atha tattvabiṃduyogaprāraṃbhaḥ}
        \rdg[wit={P}]{atha rājayoga liṣyate}
        \rdg[wit={U2}]{atha rājayoga likhyate}}\dd{}
%-------------------------- 
% \om                        \E
% \om                        \L
% \om                        \B
% rājayogasyedaṃ phalaṃ      \P
% rājayogasya idaṃ phalaṃ    \N1
% rājayogasya idaṃ phalaṃ    \N2
% rājayogasya idaṃ phalaṃ // \D
% \om                        \D2
% rājayogasya idaṃ phalaṃ    \U1
% rājayogasyedaṃ phalaṃ /    \U2
%--------------------------    
\app{\lem[wit={P,U2}]{rājayogasyedaṃ phalaṃ}
  \rdg[wit={N1,N2,D}]{rājayogasya idaṃ phalaṃ}
  \rdg[wit={E,L}]{\om}}/
%--------------------------
%This is the result of \textit{rājayoga}:
%--------------------------
% \om                                                                                                                                                                         \E
% \om                                                                                                                                                                         \L
% \om                                                                                                                                                                         \B
% yena rājayogenāneka---rājyabhogasamaya   eva   anekapārthivavinodaprekṣaṇasamaya  eva   bahutarakālaṃ  śarīrasthitir  bhavati    sa eva  rājayogaḥ tasyaite     bhedāḥ      \P
% yena rājayogenāneka---rājyabhogasamaya   eva/  anekapārthivavinodaprekṣaṇasamaya  eva/  bahutarakālaṃ  śarīrasthitir  bhavati    sa eva  rājayogaḥ /  tasya ete bhedāḥ /    \N1
% yena rājayogena  anekarājyabhogasamaya   eva// anekapārthivavinodaprekṣaṇasamaya  eva   bahuttarakālaṃ śarīrasthitir  bhavati    sa eva  rājayogaḥ /  tasya ete bhedāḥ /    \N2
% yena rājayogena  anekarājyabhogasamaya   eva// anekapārthivavinodaprekṣaṇasamaya  eva// bahutarakālaṃ  śarīrasthitir  bhavati//  sa eva  rājayogaḥ // tasya ete bhedāḥ /    \D
% \om                                                                                                                                                                         \D2
% yena rājayogena  anekarājyabhogasamaya   eva// anekapārthivavinodaprekṣaṇasamaya  eva// bahutarakālaṃ  śarīrasthitir  bhavati    sa evaṃ rājayogaḥ    tasya ete bhedāḥ //   \U1 
% yena rājayogena  anekarājyabhogasamaya   eva// anekapārthivavinodaprekṣyaṇasamaya eva// bahutarakālaṃ  śarīrasthitir  bhavati//  sa eva  rājayogastaisyaite     bhedāḥ //   \U2
% --------------------------
%\textit{Rājayoga} is that by which longterm durability of the body arises even amongst manifold royal pleasures even amongst the manifold royal entertainments and spectacle. This truly is \textit{rājayoga}. Of this [\textit{rājayoga}] these are the varieties: \end{tlate}
%--------------------------
yena rāja\app{\lem[wit={P,N1}, alt={°yogenāneka°}]{yogenāneka}
  \rdg[wit={N2,D,U1,U2}]{°yogena aneka°}
}rājyabhogasamaya eva/ anekapārthivavinoda
      \app{\lem[wit={ceteri}]{prekṣaṇasamaya}
        \rdg[wit={U2}]{prekṣyaṇasamaya}}
      eva/ bahutarakālaṃ śarīrasthitir-bhavati/ sa
      \app{\lem[wit={ceteri}]{eva}
        \rdg[wit={U2}]{evaṃ}}
      \app{\lem[wit={ceteri}]{rājayogaḥ}
        \rdg[wit={U2}]{rājayogas}}/ 
      \app{\lem[wit={P,U2}]{tasyaite}
        \rdg[wit={ceteri}]{tasya ete}} bhedāḥ/
%-------------------------
%
% \om                                                                                                                                                                \E
% \om                                                                                                                                                                \L
% \om                                                                                                                                                                \B
% kriyāyogaḥ 1 jñānayogaḥ 2 caryāyogaḥ 3 haṭhayogaḥ 4 karmayogaḥ 5 layayogaḥ 6 dhyānayogaḥ 7 maṃtrayogaḥ 8 lakṣyayogaḥ 9 vāsanāyogaḥ 10 śivayogaḥ 11 brahmayogaḥ 12 advaitayogaḥ 13 siddhayogaḥ 14 rājayogaḥ 15 ete paṃcadaśayogāḥ \P
%
% kriyāyogaḥ / jñānayogaḥ / caryāyogaḥ / haṭhayogaḥ / karmayogaḥ / layayogaḥ / dhyānayogaḥ / maṃtrayogaḥ / lakṣyayogaḥ / vāsanāyogaḥ / śivayogaḥ / brahmayogaḥ / advaitayogaḥ / rājayogaḥ / siddhayogaḥ / ete paṃcadaśayogāḥ // \N1
%
% kriyāyogaḥ jñānayogaḥ caryāyogaḥ haṭhayogaḥ karmayogaḥ layayogaḥ dhyānayogaḥ maṃtrayogaḥ lakṣayogaḥ vāsanāyogaḥ śivayogaḥ brahmayogaḥ advaitayogaḥ rājayogaḥ siddhayogaḥ // ete paṃcadaśayogāḥ // \N2      
%      
% kriyāyogaḥ // jñānayogaḥ // caryāyogaḥ // haṭhayogaḥ // karmayogaḥ // layayogaḥ // dhyānayogaḥ // maṃtrayogaḥ // lakṣyayogaḥ // vāsanāyogaḥ // śivayogaḥ // brahmayogaḥ // advaitayogaḥ // rājayogaḥ // siddhayogaḥ // ete paṃcadaśayogāḥ // \D
% \om                                                                                                                                                                         \D2      
%
% kriyāyogaḥ // jñānayogaḥ // tvaryāyogaḥ // haṭhayogaḥ // karmayogaḥ // layayogaḥ // dhyānayogaḥ maṃtrayogaḥ  lakṣayogaḥ  vāsanāyogaḥ  śivayogaḥ  brahmayogaḥ  advaitayogaḥ  rājayogaḥ  siddhayogaḥ ete paṃcadaśayogāḥ  \U1
%
% kriyāyogaḥ // jñānayogaḥ // caryāyogaḥ // haṭhayogaḥ // karmayogaḥ // nayayogaḥ // dhyānayogaḥ // maṃtrayogaḥ // lakṣyayogaḥ // vāsanāyogaḥ // śivayogaḥ // brahmayogaḥ // advaitayogaḥ // siddhayogaḥ // rājayogaḥ // evaṃ paṃcadaśāyogā bhavaṃti // \U2
%-------------------------
         kriyāyogaḥ 1\dd{}
         jñānayogaḥ 2\dd{}
         \app{\lem[wit={ceteri}]{caryāyogaḥ}
          \rdg[wit={U1}]{tvaryāyogaḥ}} 3\dd{}
        haṭhayogaḥ 4\dd{}
        karmayogaḥ 5\dd{}
        \app{\lem[wit={ceteri}]{layayogaḥ}
          \rdg[wit={U2}]{nayayogaḥ}} 6\dd{}
        dhyānayogaḥ 7\dd{}
        mantrayogaḥ 8\dd{}
        \app{\lem[wit={ceteri}]{lakṣyayogaḥ}
          \rdg[wit={U1}]{lakṣayogaḥ}} 9\dd{}
        vāsanāyogaḥ 10\dd{}
        śivayogaḥ 11\dd{} 
        brahmayogaḥ 12\dd{}
        advaitayogaḥ 13\dd{} 
        \app{\lem[wit={P,U2}]{siddhayogaḥ}
          \rdg[wit={N1,N2,D,U1}]{rājayogaḥ}} 14\dd{}
        \app{\lem[wit={P,U2}]{rājayogaḥ}
          \rdg[wit={ceteri}]{siddhayogaḥ}} 15\dd{}     
        \note[type=philcomm, labelb=1, lem={rājayoga}]{The initial codification of 15 \textit{yoga}s appears in N\textsubscript{1}, N\textsubscript{2}, P, D, U\textsubscript{1} and U\textsubscript{2}. It is ommitted in E, L and B (missing folio). It is also absent in the \textit{Yogasaṃgraha}.}
        \note[type=source, labelb=2, lem={\textbf{Re}}]{YSV\textsuperscript{qcr \cdot PT} (Ed. p. 831): pañcadaśaprakāro 'yaṃ rājayogaḥ || kriyāyogo jñānayogaḥ karmayogo haṭhas tathā | dhyānayogo mantrayoga urayogaś ca vāsanā |  rājaty etad brahmavaśīva ebhiś ca pañcadaśadhā | idānīṃ lakṣaṇañ caiṣāṃ kathayāmi śṛṇu priye |}
        \note[type=testium, labelb=3, lem={\textbf{Ri}}]{\textit{Yogasiddāntacandrikā} (Ed. p. 2): nididhyāsanañcaika tānatādirūpo rājayogāparaparyāyaḥ samādhiḥ | tatsādhanaṃ tu kriyāyogaḥ, caryāyogaḥ, karmayogo, haṭhayogo, mantrayogo, jñānayogaḥ, advaitayogo, lakṣyayogo, brahmayogaḥ, śivayogaḥ, siddhiyogo, vāsanāyogo, layayogo, dhyānayogaḥ, premabhaktiyogaś ca |}
        \app{\lem[wit={P,N1,D,U1}]{ete pañcadaśayogāḥ}
          \rdg[wit={U2}]{evaṃ paṃcadaśāyogā bhavaṃti}}\dd{}
      \end{prose} \vfill
    \nolinenumbers    
\smallskip
\centerline{\textrm{\small{[Kriyāyoga]}}} \linenumbers
\bigskip
%--------------------------        
% \om                                      \E
% \om                                      \L
% \om                                      \B
% idānīṃ kriyāyogasya lakṣaṇaṃ kathyate/   \P
% idānīṃ kriyāyogasya lakṣaṇaṃ kathyate/   \N1
% idānī  kriyāyogasya lakṣaṇaṃ kathyate//  \N2
% idānīṃ kriyāyogasya lakṣaṇaṃ kathayate/  \D
% \om                                      \D2
% idānīṃ kriyāyogasya lakṣaṇaṃ kathyate/   \U1
% atha   kriyāyogas   lakṣaṇaṃ          // \U2
%--------------------------
%Now the characteristic of the Yoga of [mental] action (\textit{kriyāyoga}) described.
%--------------------------
\begin{prose}
        \app{\lem[wit={ceteri}]{idānīṃ}
            \rdg[wit={N2}]{idānī}
            \rdg[wit={U2}]{atha}}
          \app{\lem[wit={ceteri}]{kriyāyogasya}
            \rdg[wit={U2}]{kriyāyogas}} lakṣaṇaṃ
          \app{\lem[wit={ceteri}]{kathyate}
            \rdg[wit={D}]{kathayate}
            \rdg[wit={U2}]{\om}}/
\end{prose}
\begin{tlg}
%--------------------------   
% \om                                                    \E
% \om                                                    \L
% \om                                                    \B
% kriyāmuktir    ayaṃ yogaḥ    svapiṇḍe siddhidāyakaḥ    \P
% kriyāmuktir    ayaṃ yogaḥ /  svapiṇḍe siddhidāyakaḥ /  \N1
% kriyāmukti    layaṃ yogaḥ    svapiṇḍe siddhidāyakaḥ /  \N2
% kriyāmuktir    ayaṃ yogaḥ    svapiṇḍe siddhidāyakaḥ /  \D
% \om                                                    \D2
% kriyāyuktir    ayaṃ yogaḥ /  svapiṇḍe siddhidāyakaḥ /  \U1
% kriyāmuktiḥ // ayaṃ yogaḥ    svapiṇḍe siddhidāyakaṃ // \U2 
%--------------------------
%This Yoga is liberation through [mental] action, it bestows success(\textit{siddhi}) in ones own body.
%-------------------------- 
   \tl{\note[type=source, labelb=4, lem=\textbf{Cee}]{PT\textsuperscript{ccn \cdot YSV} (Ed. p. 831): kriyāmuktimayo (\textit{kriyāmuktir ayaṃ} YK\textsuperscript{ccn \cdot YSV} 1.209  Ed. p. 17) yogaḥ sapiṇḍisiddhidāyakaḥ (\textit{sapiṇḍe} YK\textsuperscript{ccn \cdot YSV} 1.210 Ed. p. 17) | yatkāromīti saṅkalpaṃ kāryārambhe manaḥ sadā ||}
     \app{\lem[wit={ceteri}, alt={kriyāmuktir}]{kriyāmukti\skp{r-a}}
    \rdg[wit={N2}]{kriyāmukti}
    \rdg[wit={U2}]{kriyāmuktiḥ ||}
}\app{\lem[wit={ceteri}, alt={ayaṃ}]{\skm{r-a}yaṃ}
  \rdg[wit={N2}]{layaṃ}}
\app{\lem[wit={ceteri}]{yogaḥ}
  \rdg[wit={N1,U1}]{yogaḥ |}} svapiṇḍe
\app{\lem[wit={ceteri}]{siddhidāyakaḥ}
  \rdg[wit={U2}]{siddhidāyakaṃ}}/}\\
%-------------------------
% \om                                                    \E
% \om                                                    \L
% \om                                                    \B
% yaṃ yaṃ karoti kallolaṃ kāryāraṃbhe manaḥ sadā         \P
% yaṃ yaṃ karoti kallolaṃ kāryāraṃbhe manaḥ sadā/        \N1
% yaṃ yaṃ karoti kallolaṃ kāryāraṃbhe manaḥ sadā//1//    \N2
% yaṃ yaṃ karoti kallolaṃ kāryāraṃbhe manaḥ sadā/        \D
% \om                                                    \D2
% yaṃ yaṃ karoti kallolaṃ kāryāraṃbhe manaḥ sadā/ 1      \U1
% yaṃ yaṃ karoti kallolaṃ kāryāraṃbhe manaḥ sadā/        \U2
%--------------------------
%Each wave the mind creates at the beginning of an action,
%-------------------------- 
\tl{yaṃ yaṃ karoti kallolaṃ kāryāraṃbhe manaḥ sadā/}\\
%--------------------------
% \om                                                        \E
% \om                                                        \L
% \om                                                        \B
% tattataḥ   kuñcanaṃ kurvan kriyāyogas tato bhavet            \P
% tattataḥ   kuñcanaṃ kurvan kriyāyogas ato bhava     //       \N1
% tattataḥ   kūrcanaṃ kurvan kriyāyogas ato bhava     //       \N2
% tattataḥ   kuñcanaṃ kurvan kriyāyogas ato bhava     //       \D
% \om                                                          \D2
% taṃkṛ taṃ  kuñcanaṃ kurvan kriyāyogas ato ?va      //1//    \U1
% tatastataḥ kuṃcanaṃ kurvan kriyāyogas tato bhavet //1//     \U2
%--------------------------
%of all those one shall withdraw oneself. Then \textit{kriyāyoga} arises.
%--------------------------
\tl{\note[type=source, labelb=5, lem=\textbf{Cee}]{PT\textsuperscript{ccn \cdot YSV} (Ed. p. 839): tatsāṅgācaraṇaṃ kurvan kriyāyogarato bhavet |}
  \app{\lem[wit={ceteri}]{tattataḥ}
    \rdg[wit={U2}]{tatas tataḥ}
    \rdg[wit={U1}]{taṃkṛ taṃ}}
  \app{\lem[wit={ceteri}]{kuñcanaṃ}
    \rdg[wit={N2}]{kūrcanaṃ}}
  kurvan-kriyāyoga\skp{s-ta}\app{\lem[wit={P,U2}, alt={tato bhavet}]{\skm{s-t}ato bhavet}
    \rdg[wit={N1,N2,D}]{ato bhava}
    \rdg[wit={U1}]{ato va}}\dd{}1\hskip-2pt\dd{}}
\end{tlg}
\vfill
\end{edition}
\begin{translation}
  \ekddiv{type=trans}
\centerline{\textrm{\small{[Introduction]}}}
\bigskip
\begin{tlate}
Homage to Śrī Gaṇeśa. Now the methods of Rājayoga are laid down. \\
\noindent This is the result of Rājayoga\footnote{This statement seems unconnected to the definition of rājayoga that follows.}: Rājayoga is that by which long-term durability of the body arises [and] even amongst manifold royal pleasures even amongst the manifold royal entertainments and spectacle. This truly is Rājayoga. These are the varieties of this Rājayoga:\\\\
\indent \textbf{1.} The Yoga of [mental] action (Kriyāyoga); \textbf{2.} the Yoga of knowledge (Jñānayoga); \textbf{3.} the Yoga of wandering (Caryāyoga);\footnote{The first three Yogas allude to the four \textit{pāda}s of the Śaiva \textit{āgama}s; namely \textit{kriyā}[\textit{pāda}], \textit{caryā}[\textit{pāda}], \textit{yoga}[\textit{padā}] and \textit{jñāna}[\textit{pāda}], see \citeauthor[2015: 77]{nishvasa2015}.} \textbf{4.} the Yoga of force (Haṭhayoga); \textbf{5.} the Yoga of deeds (Karmayoga); \textbf{6.} the Yoga of absorption (Layayoga); \textbf{7.} the Yoga of meditation (Dhyānayoga); \textbf{8.} the Yoga of Mantras (Mantrayoga); \textbf{9.} the Yoga of fixation objects (Lakṣyayoga); \textbf{10.} Yoga of mental residues (Vāsanāyoga); \textbf{11.} the Yoga of Śiva (Śivayoga); \textbf{12.} the Yoga of Brahman (Brahmayoga); \textbf{13.} the Yoga of non-duality (Advaitayoga); \textbf{14.} the Yoga of the Siddhas (Siddhayoga); \textbf{15.} the Yoga of kings (Rājayoga). These are the fifteen Yogas.\footnote{The authenticity of the list of the fifteen Yogas present at the beginning of the text is uncertain. It remains unclear whether the list is a subsequent addition by another scribe or if it is, in fact, a part of the original text composed by Rāmacandra. Despite the suggestion of a sequential arrangement of Yogas in the list, the text only loosely follows the order presented. This raises questions about the reliability of the list and its relationship to the rest of the text. A more detailled investigation of the 15 Yogas can be found at p. \pageref{15yogas}.}\\\\
      \bigskip 
      \centerline{\textrm{\small{[\textit{Kriyāyoga}]}}}
      \bigskip
\indent Now the characteristic of Kriyāyoga, the Yoga of [mental] action is described. \paragraph{1.} This Yoga is liberation through [mental] action. It bestows success(\textit{siddhi}) in one's own body. Each wave the mind creates at the beginning of an action, of all those, one shall withdraw oneself. Then Kriyāyoga arises.
\end{tlate}
 \vspace*{\fill}
 \ekdpb*{}
   \end{translation}
 \end{alignment}
%%%%%%%%%%%%%%%%%%%%%%%%%%%%%%%%%%%%%%%%%%
%%%%%%%%%%%%%%%%%%%%%%%%%%%%%%%%%%%%%%%%%%
%%%%%%%%PAGEBREAK%%%%%%%PAGEBREAK%%%%%%%%%
%%%%%%%%%%%%%%%%%%%%%%%%%%%%%%%%%%%%%%%%%%
%%%%%%%%%%%%%%%%PAGEBREAK%%%%%%%%%%%%%%%%%
%%%%%%%%%%%%%%%%%%%%%%%%%%%%%%%%%%%%%%%%%%
%%%%%%%%PAGEBREAK%%%%%%%PAGEBREAK%%%%%%%%%
%%%%%%%%%%%%%%%%%%%%%%%%%%%%%%%%%%%%%%%%%%
%%%%%%%%%%%%%%%%%%%%%%%%%%%%%%%%%%%%%%%%%%
%%%%%%%%%%%%%%%%%%%%%%%%%%%%%%%%%%%%%%%%%%
%%%%%%%%%%%%%%%%%%%%%%%%%%%%%%%%%%%%%%%%%%
%%%%%%%%PAGEBREAK%%%%%%%PAGEBREAK%%%%%%%%%
%%%%%%%%%%%%%%%%%%%%%%%%%%%%%%%%%%%%%%%%%%
%%%%%%%%%%%%%%%%PAGEBREAK%%%%%%%%%%%%%%%%%
%%%%%%%%%%%%%%%%%%%%%%%%%%%%%%%%%%%%%%%%%%
%%%%%%%%PAGEBREAK%%%%%%%PAGEBREAK%%%%%%%%%
%%%%%%%%%%%%%%%%%%%%%%%%%%%%%%%%%%%%%%%%%%
%%%%%%%%%%%%%%%%%%%%%%%%%%%%%%%%%%%%%%%%%%
%%%%%%%%%%%%%%%%%%%%%%%%%%%%%%%%%%%%%%%%%%
%%%%%%%%%%%%%%%%%%%%%%%%%%%%%%%%%%%%%%%%%%
%%%%%%%%PAGEBREAK%%%%%%%PAGEBREAK%%%%%%%%%
%%%%%%%%%%%%%%%%%%%%%%%%%%%%%%%%%%%%%%%%%%
%%%%%%%%%%%%%%%%PAGEBREAK%%%%%%%%%%%%%%%%%
%%%%%%%%%%%%%%%%%%%%%%%%%%%%%%%%%%%%%%%%%%
%%%%%%%%PAGEBREAK%%%%%%%PAGEBREAK%%%%%%%%%
%%%%%%%%%%%%%%%%%%%%%%%%%%%%%%%%%%%%%%%%%%
%%%%%%%%%%%%%%%%%%%%%%%%%%%%%%%%%%%%%%%%%%
\begin{alignment}[
    texts=edition[class="edition"];
    translation[class="translation"],
  ]
\begin{edition}
 \ekddiv{type=ed}
    \begin{tlg}
%--------------------------      
% \om                                                                                                 \B
% \om                                                                                                 \L
% kṣamā vivekaṃ vairāgyaṃ śāntiḥ santoṣaniṣpṛhā       etadyuktiyuto  yogī   kriyāyogī nigadyate       \E
% kṣamāvivekavairāgyaṃ    śāntiḥ santoṣanispṛhāḥ      etadyuktiyuto  yogī   kriyāyogī nigadyate       \P
% kṣamāvivekavairāgyaṃ    śāntiḥ santoṣanispṛhā       etat yuktiyuto yogī   kriyāyogī nigadyate       \N1
% kṣamāvivekavairāgyaṃ    śāntiḥ santoṣanispṛhā //2// etat yuktiyuto yo sau kriyāyogī nigadyate//     \N2
% kṣamāvivekavairāgyaṃ    śāntiḥ santoṣanispṛhaḥ      etat yuktiyuto yogī   kriyāyogī nigadyate       \D
% \om                                                                                            \D2
% kṣamāvivekavairāgya---- śāntisantoṣaniḥspṛhī        etad yuktiyuto  yo sau kriyāyogī nigadyate       \U1 
% kṣamā vivekaṃ vairāgyaṃ śāntisaṃtoṣaniṣpṛhāḥ //     etat muktiyuto yogī   kriyāyogī nigadyate //2// \U2
%--------------------------
%Patience, discrimination, equanimity, peace, modesty, desireless: He who is endowed with these means is said to be a \textit{kriyāyogī}.
%--------------------------
% The text of the Printed Edition starts here ---> 
%--------------------------
      \tl{\note[type=source, labelb=6, lem=\textbf{Cee}]{PT\textsuperscript{ccn \cdot YSV} (Ed. p. 831): kṣamāvivekavairāgyaśāntisantoṣanispṛhāḥ | etan muktiyuto yo 'sau (\textit{muktiyutaś cāsau} YK\textsuperscript{ccn \cdot YSV} 1.211 Ed. p. 17) kriyāyogo nigadyate |}
kṣamā\app{\lem[wit={ceteri}, alt={°viveka°}]{viveka}\rdg[wit={E,U2}]{vivekaṃ}}vairāgyaṃ\note[type=philcomm, labelb=8, lem={kṣamā°}]{E begins here.}śāntisantoṣa\app{\lem[wit={P},alt={°nispṛhāḥ}]{nispṛhāḥ}
          \rdg[wit={U2}]{°niṣpṛhāḥ ||}
          \rdg[wit={E,N1}]{°nispṛhā}
          \rdg[wit={N2}]{°niṣpṛhā ||2||}
          \rdg[wit={D}]{°nispṛhaḥ}
          \rdg[wit={U1}]{°niṣpṛhī}}/}\\
      \tl{\app{\lem[wit={E,P,U1},alt={etad}]{eta\skp{d-yu}}
          \rdg[wit={N1,N2,D,U2}]{etat}
}\app{\lem[wit={ceteri}, alt={yuktiyuto}]{\skm{d-yu}ktiyuto}  %%%SANDHI
    \rdg[wit={U2}]{muktiyuto}}
  \app{\lem[wit={N2,U1}]{yo 'sau}  
    \rdg[wit={E,P,N1,D,U2}]{yogī}}
kriyāyogī nigadyate\dd{}2\hskip-2pt\dd{}}
\end{tlg}
\begin{tlg}
%-----------------------
% \om                                                \B
% \om                                                \L
% mātsaryaṃ mamatā māyā hiṃsā ca   madagarvitā /     \E
% mātsarya  mamatā māyā hiṃsāśā    madagarvitāḥ      \P
% mātsarya  mamatā māyā hiṃsāḥ //  madagarvatā /     \N1    -> the hiṃsā---''ḥ//'' in \nepal looks like a śā -> indicator that the others copied from \nepal? 
% mātsarya  mamatā māyā hiṃsāśā    madagārvatā //3// \N2
% mātsarya  mamatā māyā hiṃsāśā    madagarvatā /     \D
% \om                                                \D2
% mātsaryaṃ mamatā māyā hiṃsāśā    madagarvatā /     \U1
% mātsaryaṃ mamatā māyā hiṃsāśā    madagarvatā /     \U2
%-----------------------
%Envy, selfishness, cheating, violence, desire and intoxication, pride,
%-----------------------
       \tl{\note[type=source, labelb=9, lem=\textbf{Ce}]{PT\textsuperscript{ccn \cdot YSV} (Ed. p. 831): mātsaryaṃ mamatā māyā hiṃsā ca madagarvitā | kāmaḥ krodho bhayaṃ lajjā lobho mohas tathā 'śuciḥ (\textit{śuciḥ} YK\textsuperscript{ccn \cdot YSV} 1.212 Ed. p. 17) ||}
         \app{\lem[wit={E,U1,U2}]{mātsaryaṃ}
           \rdg[wit={P,N1,D}]{mātsarya}}
         mamatā māyā
         \app{\lem[wit={E}]{hiṃsā ca}
           \rdg[wit={ceteri}]{hiṃsāśā}
           \rdg[wit={N1}]{hiṃsāḥ ||}}
         madagarvatā/}\\
%-----------------------
% \om                                                   \B
% \om                                                   \L
% kāmakrodhabhayaṃ   lajjā lobhamohau tathā śuciḥ //    \E
% kāmakrodhabhayaṃ   lajjā lobhamohau tathā 'śuciḥ      \P
% kāmakrodhabhayaṃ   lajjā lobhamohau tathā 'śuciḥ /    \N1    -> the hiṃsā---''ḥ//'' in \nepal looks like a śā -> indicator that the others copied from \nepal? 
% kāmakrodho bhayaṃ  lajjā lobhamohau tathā śuciḥ //    \N2
% kāmakrodho bhayaṃ  lajjā lobhamohau tathā 'śuciḥ //   \D
% \om                                                   \D2
% kāmakrodhau bhayaṃ lajjā lobhamohau tathā 'śuciḥ      \U1
% kāmakrodhau bhayaṃ lajjā lobhamohau tathā śuciḥ //3// \U2
% -----------------------
% lust, anger, fear, laziness, greed, error and impurity.
%-----------------------
       \tl{kāma\app{\lem[wit={U1,U2}, alt={°krodhau}]{krodhau}
           \rdg[wit={E,P,N1}]{krodha°}
           \rdg[wit={D}]{°krodho}}
         bhayaṃ lajjā lobhamohau tathā
         \app{\lem[wit={ceteri}]{'śuciḥ}
           \rdg[wit={E,N2,U2}]{śuciḥ}}\dd{}3\hskip-2pt\dd{}}    %%%AVAGRAHA
\end{tlg}
\begin{tlg}
%-----------------------
%  \om                                                           \B
%  atha dveṣo ghṛṇālasyaṃ bhrāṃtir   daṃbho kṣamā bhramaḥ //     \L
%  rāgadveṣau ghṛṇālasyaṃ bhrāntitvaṃ     mokṣamā bhramaḥ /      \E
%  rāgadveṣau ghṛṇālasyaṃ bhrāṃtir   ddaṃbhokaṣmā bhramaḥ        \P
%  rāgadveṣau ghṛṇālasyaṃ bhrāṃtir   daṃbho kṣamā bhramaḥ //4//  \N1
%  rāgadveṣau ghṛnālasyaṃ bhrāṃtir   daṃbho kṣamā bhramaḥ //4    \N2
%  rāgadveṣau ghṛṇālasyaṃ bhrāṃtir   debho  kṣamā bhramaḥ //     \D
% \om                                                            \D2
%  rāgadoṣau  ghṛṇālasyaṃ bhrāṃti    daṃbha kṣamī bhramaḥ 4      \U1
%  rāgadveṣau ghṛṇālasyaṃ bhrāṃtir   daṃbho kṣamā bhramaḥ //     \U2
%-----------------------
%Attachment and aversion, indignation and idleness, impatience and dizzyness
%-----------------------
        \tl{\note[type=source, labelb=10, lem=\textbf{Ce}]{PT\textsuperscript{ccn \cdot YSV} (Ed. p. 831): rāgadveṣau ghṛṇālasyaśrāntidambhakṣamābhramāḥ (\textit{ghṛṇālasyaṃ bhrāntir dambho 'kṣamā bhramaḥ} YK\textsuperscript{ccn \cdot YSV} 1.213 Ed. p. 17) | yasyaitāni na vidyante kriyāyogī sa ucyate ||}
          \app{\lem[wit={ceteri}]{rāgadveṣau}
            \rdg[wit={U1}]{rāgadoṣau}
            \rdg[wit={L}]{atha dveṣo}}\note[type=philcomm, labelb=11, lem={rāga°}]{L begins here.}
          \app{\lem[wit={ceteri},alt={ghṛṇā°}]{ghṛṇā}
            \rdg[wit={N2}]{ghṛnā°}}lasyaṃ 
          \app{\lem[wit={ceteri}, alt={bhraṃtir daṃbho}]{bhrantir-daṃbho}
            \rdg[wit={D}]{bhrāṃtir debho}
            \rdg[wit={E}]{bhrāntitvaṃ}
            \rdg[wit={U1}]{bhrāṃti daṃbha°}}
          \app{\lem[wit={ceteri}]{kṣamā bhramaḥ}
            \rdg[wit={E}]{mokṣam ābhramaḥ}
            \rdg[wit={U1}]{kṣamī bhramaḥ}}/}\\
%-----------------------
%  \om                                               \B
%  yasyai tāni na vidyaṃte kriyāyogī sa ucyate //    \L
%  yasyai tāni ca vidyante kriyāyogī sa ucyate 3     \E
%  yasyai tāni na vidyaṃte kriyāyogī sa ucyate       \P
%  yasyai tāni na vidyaṃte kriyāyogī sa ucyate //    \N1
%  yasyai tāni na vidyaṃte kriyāyogī sa ucyate //    \N2
%  yasyai tāni na vidyaṃte kriyāyogī sa ucyate //    \D
%  yasyai tāni na vidyaṃte kriyāyogī sa ucyate       \U1
%  yasyai tāni na vidyaṃte kriyāyogī sa ucyate //4// \U2 
%  -----------------------
% Whoever doesn't experience these is called a \textit{kriyāyogī}. 
%  -----------------------        
        \tl{
yasyai tāni \app{\lem[wit={ceteri}]{na}\rdg[wit={E}]{ca}} vidyante kriyāyogī sa ucyate\dd{}4\hskip-2pt\dd{}}
      \end{tlg}
      \bigskip
      \begin{prose}
%-----------------------
%  \om                                                                                          \B
%  yasyāntaḥkaraṇe kṣamāvivekavairāgyaśāntisantoṣādīny                         utpadyante //     \E
%  yasyāṃtaḥkaraṇe kṣamāvivekavairāgyaśāṃtisaṃtoṣa         ityādīny            utpādyaṃte        \P
%  tasyāṃtaḥkaraṇe kṣamāvivekavairāgyaśāṃtisaṃtoṣa         ityādīnotpādyaṃte                    \L
%  yasyāṃtaḥkaraṇe kṣamāḥ vivekavairāgya /    śāṃtisaṃtoṣa ityādīni            utpādyaṃte        \N1
%  yasyāṃtaḥkaraṇe kṣamā' vivekavairāgyā      śāṃtisaṃtoṣa ityādīni            utpādyaṃte /      \N2 %see Mss p3 recto vierte Zeile von unten  
%  yasyāṃtaḥkaraṇe kṣamā // vivekavairāgya // śāṃtisaṃtoṣa ityādīni            utpādyaṃte //     \D
%  yasyāṃtaḥkaraṇe kṣamāvivekavairāgyaśāṃtisaṃtoṣa         ityādīna niraṃtaram utyaṃte        \U1
%  yasyāṃtaḥkaraṇe kṣamāvivekavairāgyaśāṃtisaṃtoṣa         ityādayo niraṃtaraṃ utpādyaṃte       \U2
%  -----------------------
%  Patience, discrimination, equanimity, peace, contentment etc. are generated in his mind.
%  -----------------------        
        yasyāntaḥkaraṇe
        \app{\lem[wit={ceteri},alt={kṣamā°}]{kṣamā}
          \rdg[wit={N1}]{kṣamāḥ}
          \rdg[wit={N2}]{kṣamā '}
        }\app{\lem[wit={ceteri}]{vivekavairāgyaśānti}
          \rdg[wit={N1}]{kṣamāḥ vivekavairāgya | śāṃti°}
          \rdg[wit={N2}]{°vairāgyāśānti°}
          \rdg[wit={D}]{kṣamā || vivekavairāgya || śāṃti°}
        }\app{\lem[wit={ceteri}, alt={°santoṣa ityādīny}]{santoṣa ityādī\skp{ny-u}} %the°-problem
          \rdg[wit={E}]{°santoṣādīny}
          \rdg[wit={L}]{°santoṣa ity ādīno°}
          \rdg[wit={U1}]{°santoṣa ity ādīna niraṃtaram}
          \rdg[wit={U2}]{°santoṣa ity ādayo niraṃtaraṃ}
        }\app{\lem[wit={ceteri},alt={utpādyante}]{\skm{ny-u}tpādyante}
          \rdg[wit={E}]{utpadyante}
          \rdg[wit={L}]{°tpādyaṃte}
          \rdg[wit={U1}]{utyaṃte}}/
%-----------------------
% \om \oxford
%  sa eva bahukriyāyogī kathyate /      \E
%  sa eva bahukriyāyogī kathyate        \P
%  sa eva bahukriyāyogī kathyate //     \L
%  sa eva bahukriyāyogī kathyate /      \N1
%  sa eva bahukriyāyogī sa kathyate /   \N2
%  sa eva bahukriyāyogā sa kathyate //  \D
%  sa eva bahukriyāyogī kathyate /      \U1
%  sa eva bahukriyāyogī tkacyate /      \U2
%-----------------------
% He alone is called a \textit{yogī} of many actions (\textit{bahukriyāyogī}).
%-----------------------
        sa eva
        \app{\lem[wit={ceteri}]{bahukriyāyogī}
          \rdg[wit={D}]{bahukriyāyogā}}
        \app{\lem[wit={ceteri}]{kathyate}
          \rdg[wit={D,N2}]{sa kathyate}
          \rdg[wit={U2}]{tkacyate}}/\\
%-----------------------
% \om \B
%               kāpaṭyaṃ      vittaṃ   hiṃsā    tṛṣṇā    mātsaryam    ahaṃkāraḥ    roṣaḥ kṣayaṃ    lajjā lobhamohā      aśucitvaṃ                       pākhaṃḍatvaṃ       bhrāntiḥ indriyavikāraḥ kāmaḥ          ete yasya manasi pratidinaṃ vyunā bhavanti /    \E
%               kāpaṭyaṃ      vittaṃ   hiṃsā    tṛṣṇā    mātsaryaṃ    ahaṃkāraḥ    roṣo bhayaṃ     lajjā lobhaḥ mohaḥ   aśucitvaṃ rāgaḥ dveṣaḥ   ālasyaṃ pākhaṃḍitvaṃ       bhrāṃtiḥ indriyaṃ vikāraḥ kāmaḥ        ete yasya manasi pratidinaṃ nyunā bhavanti     \P
%               kāpayaṃ     //vitaṃ // hiṃsā // tṛṣṇā // mātsaryaṃ // ahaṃkāraḥ // roṣo bhayaṃ //  lajjā lobhaḥ // moha aśucitvaṃ // rājadveṣa  alasyaṃ // pākhaṃḍitvaṃ // bhrāṃtiḥ // itivikāraḥ // kāmaḥ        eta yasya manasi pratidinaṃ nyunā bhavaṃti//    \L
%yasyāṃtakaraṇe kapatyaṃ māyā vitvaṃ   hiṃsā    tṛṣṇā    mātsaryaṃ    ahaṃkāraḥ    roṣo bhayaṃ     lajjā // lobhamohā   asucitvaṃ rāgadveṣaḥ // alasyaṃ pāṣaṃḍitvaṃ      bhraṃtiḥ / iṃdriyaivikāraḥ / kāmaḥ       ete yasya manasi pratidinaṃ nyunā bhavaīti/     \N1
%               kāpaṭyaṃ māyā vitvaṃ   hiṃsā    tṛṣṇā    mātsaryaṃ    ahaṃkāraḥ    e?ṣo bhayaṃ     lajjā/ lobhamoha     asūcitvaṃ rāgadveṣaḥ    ālasyaṃ pārṣaḍitvaṃ        bhrāṃtiḥ iṃdriyavikāraḥ // kāma         ete yasya manasi pratidinaṃ nyunā bhavaṃti //  \N2      
%               kāpaṭyaṃ māya vitvaṃ   hiṃsā    tṛṣṇā    mātsarya     ahaṃkāraḥ    roṣo bhayaṃ     lajjā // lobhamohā   asucitvaṃ rāgadveṣaḥ // ālasyaṃ pāṣaṃḍitvaṃ        bhraṃtiḥ // iṃdriyavikāraḥ // kāmaḥ // ete yasya manasi pratidinaṃ nyunā bhavaṃti //   \D
%               kāpachaṃ yāya vitvaṃ   hiṃsā    tṛṣṇā    mātsarya     ahaṃkāraḥ    roṣaḥ bhayaṃ    lajā     lobhamohā   aśucitvaṃ rāgadveṣaḥ    ālasyaṃ pākhaṃḍitvaṃ       bhraṃtiḥ iṃdriyavīkāraḥ    kāmaḥ       rāte yasya manasi pratidinaṃ nyunā bhavaṃti //  \U1
%               kāpaṭyaṃ pāpā titaṃ    hiṃsā    tṛṣṇā    mātsaryaṃ // ahaṃkāraḥ    roṣo bhayaṃ     lajjā ----mohā       aśucitvaṃ rāgadveṣaḥ    ālasyaṃ pākhaṃḍitvaṃ //    bhraṃtiḥ iṃdriyavikāraḥ //-----        etate yasya manasi pratidinaṃ nyunā bhavaṃti // \U2
%-----------------------
%Fraud, illusion, property, violence, craving, envy, ego, anger, anxiety, shame, greed, error, impurity, attachment, aversion, idleness, heterodoxy, false view, affection of the senses, sexual desire: He who diminishes these from day to day in is mind,
%-----------------------              
\note[type=testium, labelb=12, lem={\textbf{Ci}}]{\textit{Yogasaṃgraha} IGNCA 30020 folio 1r. ll. 1-2: lobhamohau aśucitvaṃ rāgadveṣau ālasyaṃ pāṣaṃḍitvaṃ bhrāṃtiḥ iṃdryiavikāraḥ kāmaḥ ete yasya pratidinaṃ nyunā bhavaṃti}
        \app{\lem[wit={ceteri}]{kāpaṭyaṃ}
        \rdg[wit={N1}]{yasyāntaḥkaraṇe kapatyaṃ}
        \rdg[wit={L}]{kāpayaṃ}
        \rdg[wit={U1}]{kāpachaṃ}}\dd{}
      \app{\lem[wit={N1,N2}]{māyā}
        \rdg[wit={D}]{māya}
        \rdg[wit={U1}]{yāya}
        \rdg[wit={U2}]{pāpa}
        \rdg[wit={E,P,L}]{\om}}\dd{}
        %\rdg[wit={E,P,L}]{\textbf{omitted in}}}
      \app{\lem[wit={E,P}]{vittaṃ}
        \rdg[wit={L}]{vitaṃ}
        \rdg[wit={N1,N2,D,U1}]{vitvaṃ}
        \rdg[wit={U2}]{titaṃ}}\dd{}
      hiṃsā\dd{}
      tṛṣṇā\dd{}
      \app{\lem[wit={ceteri}]{mātsaryaṃ}
        \rdg[wit={E}]{mātsaryam}
        \rdg[wit={D,U1}]{mātsarya}}\dd{}
      ahaṃkāraḥ\dd{}
      \app{\lem[wit={E,U1}]{roṣaḥ}
        \rdg[wit={ceteri}]{roṣo}
        \rdg[wit={N2}]{eṣo}}\dd{}
      \app{\lem[wit={ceteri}]{bhayaṃ}
        \rdg[wit={E}]{kṣayaṃ}}\dd{}
      \app{\lem[wit={ceteri}]{lajjā}
        \rdg[wit={U1}]{lajā}}\dd{}
      \app{\lem[wit={P,L}]{lobhaḥ}
        \rdg[wit={ceteri}]{lobha°}
        \rdg[wit={U2}]{\om}}\dd{}
      \app{\lem[wit={P}]{mohaḥ}
        \rdg[wit={L,N2}]{moha}
        \rdg[wit={ceteri}]{mohā}}\dd{}        
      \app{\lem[wit={ceteri}]{aśucitvaṃ}  %%%Frage: vor daṇḍa wird m zu ṃ??? 
        \rdg[wit={N2}]{aśūcitvaṃ}}\dd{}
      \app{\lem[wit={P}]{rāgaḥ}
        \rdg[wit={ceteri}]{rāga°}
        \rdg[wit={L}]{rāja°}
        \rdg[wit={E}]{\om}}\dd{}
      \app{\lem[wit={ceteri}]{dveṣaḥ}
        \rdg[wit={L}]{dveṣa}
        \rdg[wit={E}]{\om}}\dd{}
      \app{\lem[wit={ceteri}]{ālasyaṃ}
        \rdg[wit={E}]{\om}}\dd{}
      \app{\lem[wit={ceteri}]{pākhaṃḍitvaṃ}
        \rdg[wit={D,N1}]{pāṣaṃḍitvaṃ}
        \rdg[wit={E}]{pākhaṃḍatvaṃ}
        \rdg[wit={N2}]{pārṣaḍitvaṃ}}\dd{}
     bhrāntiḥ\dd{}
     \app{\lem[wit={ceteri}]{indriyavikāraḥ}
        \rdg[wit={P}]{iṃdriyaṃ vīkāraḥ}
        \rdg[wit={L}]{itivikāraḥ}}\dd{}
      \app{\lem[wit={ceteri}]{kāmaḥ}
        \rdg[wit={N2}]{kāma}
        \rdg[wit={U2}]{\om}}\dd{}
      \app{\lem[wit={ceteri}]{ete}
        \rdg[wit={L}]{eta}
        \rdg[wit={U1}]{rāte}
        \rdg[wit={U2}]{etate}}
      yasya manasi pradidinaṃ nyūna
      \app{\lem[wit={ceteri}]{bhavanti}
        \rdg[wit={N1}]{bhavaīti}}/ 
%-----------------------       
%sa eva bahukriyāyogī kathyate// \E
%sa eva bahukriyāyogī kathyate// \P
%sa eva bahukriyāyogī kathyate// \L
%sa eva bahukriyāyogī kathyate// \N1
%sa eva bahukriyāyogī kathyate// \N2
%sa eva bahukiyāyogī  kathyate//  \D
%sa eva bahukiyāyogī  kathyaṃte// \U1
%sa eva bahukiyāyogī  kathyaṃte// \U2
%-----------------------
%he alone is called a yogī of many actions (\textit{bahukriyāyogī})
%-----------------------
\note[type=testium, labelb=13, lem={\textbf{Cie}}]{\textit{Yogasaṃgraha} IGNCA 30020 folio 1r. ll. 2: sa eva kriyāyogī kathyate ||}
%\note[type=philcomm, labelb=14, lem={bahukriyāyogī}]{The term \textit{bahukriyāyogī} currently seems to be unique in Sanskrit literature. The elaborations of Rāmacandra on Kriyāyoga after the quotes from the YSV are either taken from an unknown source or his own creation.}
sa eva \app{\lem[wit={ceteri}]{bahukriyāyogī}
e  \rdg[wit={D,U1,U2}]{bahukiyāyogī}}
      \app{\lem[wit={ceteri}]{kathyate}
        \rdg[wit={U1,U2}]{kathyaṃte}}\dd{}
\end{prose}
\end{edition}
\begin{translation}
\ekddiv{type=trans}
\begin{tlate}
\paragraph{2.} Patience, discrimination, equanimity, peace, modesty, desireless: the one who is endowed with these means is said to be a Kriyāyogī. 
\paragraph{3.} Envy, selfishness, cheating, violence, desire and intoxication, pride, lust, anger, fear, laziness, greed, error and impurity. 
\paragraph{4.} Attachment and aversion, indignation and idleness, impatience and dizzyness: Whoever does not experience these is called a Kriyāyogī.\footnote{All four verses on Kriyāyoga were taken from the \textit{Yogsavarodaya} as quotations in the \textit{Prāṇatoṣinī} and \textit{Yogakarṇikā}. No sources for the following prose section can be identified.}\\\\
Patience, discrimination, equanimity, peace, contentment etc., are generated in his mind. He alone is called a Yogī of many actions (\textit{bahukriyāyogī})\footnote{The term \textit{bahukriyāyogī} is only found in the \textit{Yogatattvabindu}. It seems to be a neologism of Rāmacandra since the \textit{Yogasvarodaya} and \textit{Yogasaṃgraha} only use the word \textit{kriyāyogī} in its passage on Kriyāyoga to denote its practitioner.}. Fraud, illusion, property, violence, craving, envy, ego, anger, anxiety, shame, greed, error, impurity, attachment, aversion, idleness, heterodoxy, false view, affection of the senses, sexual desire: He who diminishes these from day to day in his mind, he alone is called a Yogī of many actions (\textit{bahukriyāyogī}).\footnote{The most notable mention of the term \textit{kriyāyoga} appears in \textit{Pātañjalayogaśāstra} or \textit{Yogasūtra} 2.1 where is is defined as  
\begin{quote}
tapaḥsvādhyāyeśvarapraṇidhānāni kriyāyogaḥ || 2.1 || (\citeauthor[1983:113]{yogasutra})
\end{quote}
According to the introduction of this \textit{sūtra} in the \textit{Vyāsabhāṣya}, Kriyāyoga is introduced as a means how someone with a distracted mind can also attain Yoga (\textit{vyutthitacitto 'pi yogayuktaḥ}). Yoga, which for Patañjali is \textit{samādhi}, shall be achieved by the three elements of Kriyāyoga, namely mental, moral and physical austerity (\textit{tapas}), repetition of \textit{mantra}s or study of sacred literature (\textit{svadhyāya}) and surrender to god (\textit{īśvarapraṇidhāna}). This trinity of means is supposed to destroy the impurities (\textit{kleśa}s) of \textit{citta}. These are given in \textit{Pātanjalayogaśāstra} 2.3 as ignorance (\textit{avidyā}), egoism (\textit{asmitā}), attachment (\textit{rāga}), aversion (\textit{dveṣa}) and fear of death (\textit{abhiniveśa}), see (\citeauthor[1983:116]{yogasutra}). All three terms of Patañjali's Kriyāyoga are absent in the \textit{Yogatattvabindu}. Nevertheless, the individual elements of the \textit{kleśa}s, along with the aim to reduce these in the yogi's mind, can also be found in the \textit{Yogatattvabindu}. Nārāyaṇatīrtha in this commentary on the \textit{Pātanjalayogaśāstra} titled \textit{Yogasiddhāntacandrikā}, who, like Rāmacandra uses a very similar list of 15 Yogas (possible source for Rāmacandras 15 Yogas), presents Kriyāyoga as the first item of his list and explains its purpose as the generation of \textit{samādhi} and the reduction of \textit{kleśas}, see (\citeauthor[2000:71]{yogacandrika}).
Rāmacandra and Nārāyaṇatīrtha both present their Kriyāyoga as a means to achieve their final goal. However, the goal is significantly different. While Nārāyaṇatīrtha's Kriyāyoga is said to lead to \textit{samādhi}, so the Kriyāyoga of Rāmacandra is said to lead to Rājayoga, which he conceptualizes as bringing about the steadiness of the body.}
\end{tlate}
 \ekdpb*{}
\end{translation}
\end{alignment}
%%%%%%%%%%%%%%%%%%%%%%%%%%%%%%%%%%%%%%%%%%
%%%%%%%%%%%%%%%%%%%%%%%%%%%%%%%%%%%%%%%%%%
%%%%%%%%PAGEBREAK%%%%%%%PAGEBREAK%%%%%%%%%
%%%%%%%%%%%%%%%%%%%%%%%%%%%%%%%%%%%%%%%%%%
%%%%%%%%%%%%%%%%PAGEBREAK%%%%%%%%%%%%%%%%%
%%%%%%%%%%%%%%%%%%%%%%%%%%%%%%%%%%%%%%%%%%
%%%%%%%%PAGEBREAK%%%%%%%PAGEBREAK%%%%%%%%%
%%%%%%%%%%%%%%%%%%%%%%%%%%%%%%%%%%%%%%%%%%
%%%%%%%%%%%%%%%%%%%%%%%%%%%%%%%%%%%%%%%%%%
%%%%%%%%%%%%%%%%%%%%%%%%%%%%%%%%%%%%%%%%%%
%%%%%%%%%%%%%%%%%%%%%%%%%%%%%%%%%%%%%%%%%%
%%%%%%%%PAGEBREAK%%%%%%%PAGEBREAK%%%%%%%%%
%%%%%%%%%%%%%%%%%%%%%%%%%%%%%%%%%%%%%%%%%%
%%%%%%%%%%%%%%%%PAGEBREAK%%%%%%%%%%%%%%%%%
%%%%%%%%%%%%%%%%%%%%%%%%%%%%%%%%%%%%%%%%%%
%%%%%%%%PAGEBREAK%%%%%%%PAGEBREAK%%%%%%%%%
%%%%%%%%%%%%%%%%%%%%%%%%%%%%%%%%%%%%%%%%%%
%%%%%%%%%%%%%%%%%%%%%%%%%%%%%%%%%%%%%%%%%%
%%%%%%%%%%%%%%%%%%%%%%%%%%%%%%%%%%%%%%%%%%
%%%%%%%%%%%%%%%%%%%%%%%%%%%%%%%%%%%%%%%%%%
%%%%%%%%PAGEBREAK%%%%%%%PAGEBREAK%%%%%%%%%
%%%%%%%%%%%%%%%%%%%%%%%%%%%%%%%%%%%%%%%%%%
%%%%%%%%%%%%%%%%PAGEBREAK%%%%%%%%%%%%%%%%%
%%%%%%%%%%%%%%%%%%%%%%%%%%%%%%%%%%%%%%%%%%
%%%%%%%%PAGEBREAK%%%%%%%PAGEBREAK%%%%%%%%%
%%%%%%%%%%%%%%%%%%%%%%%%%%%%%%%%%%%%%%%%%%
%%%%%%%%%%%%%%%%%%%%%%%%%%%%%%%%%%%%%%%%%%
\begin{alignment}[
    texts=edition[class="edition"];
    translation[class="translation"],
  ]
\begin{edition}
 \ekddiv{type=ed}
    \centerline{\textrm{\small{[Siddhakuṇḍalinīyoga and Mantrayoga]}}}
    \bigskip
    \begin{prose}
%-----------------------   
% \om                                   \B
%idānīṃ rājayogasya bhedāḥ kathyante // \E
%idānīṃ rājayogasya bhedāḥ kathyaṃte    \P
%idānīṃ rājayogasya bhedāḥ              \L
%idānīṃ rājayogasya bhedāḥ kathyaṃte    \N1
%idānīṃ rājayogasya bhedā  kathyate//    \N2
%idānīṃ rājayogasya bhedāḥ kathyaṃte // \D
% \om                                   \U1
%idānīṃ rājayogasya bhedāḥ kathyaṃte // \U2
%-----------------------
%Now varieties of \textit{rājayoga} will be described.
%-----------------------
      \noindent idānīṃ rājayogasya
      \note[type=testium, labelb=15, lem={\textbf{Ci}}]{\textit{Yogasaṃgraha} IGNCA 30020 folio 1r. ll. 2-3:  atha rājayogasya bhedau kathyete ||}
       \app{\lem[wit={ceteri}]{bhedāḥ}
         \rdg[wit={N2}]{bhedā}}
       \app{\lem[wit={ceteri}]{kathyante}
         \rdg[wit={N2}]{kathyate}
         \rdg[wit={L}]{\om}}/
       \note[type=philcomm, labelb=16, lem={kathyante}]{The whole sentence is \om in U\textsubscript{1}.}     
%-----------------------
%te ke     \E
%te ke     \P
%te ke     \L
%ke te //  \D
%ke te /   \N1
%kriyate// \N2       
%ke te     \U1
%te ke     \U2
%-----------------------
%Which are these?
%-----------------------       
\app{\lem[wit={D,N1,U1}]{ke te}
         \rdg[wit={ceteri}]{te ke}
         \rdg[wit={N2}]{kriyate}}/ 
%-----------------------
%\om                                       \B
%ekaḥ siddhakuṇḍalinīyogaḥ / mantrayogaḥ / \E
%ekaḥ siddhakuṃḍaṃliṃ yogaḥ maṃtrayogaḥ    \P
%ekaḥ siddhakuṇḍalanīyoga /                \L 
%ekaḥ siddhakuṇḍalinīyogaḥ maṃtrayogaḥ /   \N1
%ekaḥ siddhakuṇḍalanīyogaḥ maṃtrayogaḥ //  \N2
%ekaḥ siddhakuṃḍalanīyogaḥ mantrayogaḥ //  \D
%ekaḥ siddhakuṇḍaliniyogaḥ mantrayogaḥ     \U1
%ekaḥ siddhakuṇḍalinīyoga // mantrayogaḥ   \U2
%-----------------------
%One is \textit{siddhakuṇḍalinīyoga} [and one] is \textit{mantrayoga}.       
%-----------------------
\note[type=testium, labelb=17, lem={\textbf{Ci}}]{\textit{Yogasaṃgraha} IGNCA 30020 folio 1r. ll. 3: siddhakuṃḍaliyogaḥ mantrayogaś ceti |}
ekaḥ
\app{\lem[wit={E,N1}]{siddhakuṇḍalinīyogaḥ |}
   \rdg[wit={U1}]{siddhakuṇḍalinīyogaḥ}
   \rdg[wit={U2}]{siddhakuṇḍalinīyoga ||}
   \rdg[wit={L}]{siddhakuṇḍalanīyoga |}
   \rdg[wit={N2,D}]{siddhakuṃḍalanīyogaḥ}
   \rdg[wit={P}]{siddhakuṃḍaṃliṃ yogaḥ}}
 \app{\lem[wit={ceteri}]{mantrayogaḥ}
   \rdg[wit={L}]{\om}}
       \note[type=source, labelb=19, lem={\textbf{Re}}]{PT\textsuperscript{ccn \cdot YSV} (Ed. p. 831): jñānayogaṃ pravakṣyāmi tajjñānī śivatāṃ vrajet | paṭhanāt smaraṇād vyānān maṇḍanāt brahmasādhakaḥ | tad bhedasyaikasandhānam aṣṭaiśvaryamayo bhavet | tritīrthaṃ yatra nāḍī ca tripuṇyaṃ parameśvari | \ldots eṣo 'sya viśvarūpasya rājayogo mato budhaiḥ | viśeṣaṃ kathayiṣyāmi śṛṇu caikamanāḥ sati |}
%-----------------------
% \om                         \B
%astu rājayogaḥ kathyate/    \E
%amū  rājayogau kathyete       \P
%amū  rājayogau kathyate//    \L
%amū  rājayogau kathyate       \N1
%amū  rājayogau kathyate//     \N2  %%%p3verso
%amū  rājayogau kathyate//    \D
%amū  rājayogau kathyate       \U1
%amū  rājayogau kathyaṃte//   \U2
%-----------------------
%These two rājayogas are described [in the following].
%-----------------------
       \app{\lem[wit={ceteri}]{amū}
         \rdg[wit={E}]{astu}}
       \app{\lem[wit={ceteri}]{rājayogau}
         \rdg[wit={E}]{rājayogaḥ}}\\
       \app{\lem[wit={P}]{kathyete}
         \rdg[wit={ceteri}]{kathyate}
         \rdg[wit={U2}]{kathyaṃte}}/
%-----------------------
% \om                                                              \B
%mūlakandasthāne    ekā tejorūpā    mahānāḍī varttate /            \E
%mūlaṃ kaṃdasthāne  ekā tejorūpā    mahānāḍī varttate              \P
%mūlakaṃdasthāne    ekā tejorūpā    mahānāḍī vartate               \L
%mūlakaṃdasthāne    eka tejorūpā    mahānāḍī varttate /            \N1
%mūlakaṃdasthāne    eka tejorūpā    mahānāḍī varttate /            \N2
%mūlakaṃdasthāne    ekā tejorūpā    mahānāḍī varttate //           \D
%mūlakaṃdasthāne    ekā tejorūpā    mahānāḍī vartate /             \U1
%mūlakaṃdasthāne // ekā tejorūpā // mahānāḍī pravarttate /         \U2
%-----------------------
%At the location of the root-bulb exists one major vessel in the form of energy.
%-----------------------       
\note[type=testium, labelb=20, lem={\textbf{Ci}}]{\textit{Yogasaṃgraha} IGNCA 30020 folio 1r. ll. 3-4: mūlakandasthāne ekā tejomayā mahānāḍī vartate |}
       \app{\lem[wit={ceteri}]{mūlakandasthāne}
         \rdg[wit={U2}]{mūlakaṃdasthāne ||}
         \rdg[wit={P}]{mūlaṃ kaṃdasthāne}}
       \app{\lem[wit={ceteri}]{ekā}
         \rdg[wit={N1,N2}]{eka}}
       \app{\lem[wit={ceteri}]{tejorūpā}
         \rdg[wit={U2}]{tejorūpā ||}}
       mahānāḍī
       \app{\lem[wit={ceteri}]{vartate}
         \rdg[wit={U2}]{pravartate}}/
       \note[type=source, labelb=21, lem={\textbf{Re}}]{PT\textsuperscript{ccn \cdot YSV} (Ed. p. 831-832): mūlakande sthale caikā nāḍī tejasvatī parā (\textit{tejasvitāparā} YK\textsuperscript{ccn \cdot YSV} 1.246 Ed. p. 20) |}
%-----------------------
% \om                                                            \B
%iyam ekanāḍī /  iḍāpiṃgalāsuṣumṇā      etān bhedān prāpnoti /    \E
%iyaṃ ekanāḍī    iḍāpiṃgalāsuṣumṇā      etān bhedān prāpnoti      \P
%trayaṃ kā nāḍī  iḍāpiṃgalāsuṣumnā //   etān bhedān prāpnoti      \L
%iyaṃ ekā nāḍī   iḍāpiṃgalāsuṣumnān /   ete  bhedān prāpnoti      \N1
%iyaṃ ekā nāḍī   iḍāpiṃgalāsuṣumnān//   ete  bhedān prāpnoti/     \N2
%iyaṃ ekā nāḍī   iḍāpiṃgalasuṣumnān //  ete  bhedān prāpnoti      \D    
%iyaṃ ekā nāḍī   iḍāpiṃgalāsuṣumnā      etān bhedān prāpnoti      \U1
%iyaṃ eka nāḍī   iḍāpiṃgalāsuṣumṇā      etān bhegān prāpnoti      \U2
%-----------------------
%This single vessel reaches to these openings which are \textit{iḍā}, \textit{piṅgalā} and \textit{suṣumnā}.
%-----------------------       
\note[type=testium, labelb=23, lem={\textbf{Ci}}]{\textit{Yogasaṃgraha} IGNCA 30020 folio 1r. ll. 4: iyaṃ iḍāpiṃgalasuṣumnā bhedā tridhā |}
\app{\lem[wit={E},alt={iyam}]{iya\skm{m-e}}
         \rdg[wit={ceteri}]{iyaṃ}
         \rdg[wit={L}]{trayaṃ}
}\app{\lem[wit={ceteri}, alt={ekā}]{\skp{m-e}kā}
         \rdg[wit={E}]{eka |}
         \rdg[wit={P}]{eka}
         \rdg[wit={L}]{kā}}
nāḍī iḍāpiṅgalā\app{\lem[wit={N1,N2,D},alt={°suṣumṇān}]{suṣumṇān}
         \rdg[wit={L}]{suṣumnā}
         \rdg[wit={ceteri}]{°suṣumṇā}}\dd{}
       \app{\lem[wit={ceteri}]{etān}
         \rdg[wit={N1,N2,D}]{ete}} bhedān prāpnoti/
%-----------------------
%\om                                           \B
%vāmabhāge candrarūpā iḍā nāḍī varttate /      \E
%vāmabhāge caṃdrarūpā iḍā nāḍī varttate        \P
%vāmabhāge caṃdrarūpā iḍā nāḍī varttate //     \L
%vāmabhāge caṃdrarūpā iḍā nāḍī varttate /      \N1
%vāmabhāge caṃdrarūpā iḍā nāḍī varttate //     \N2
%vāmabhāge caṃdrarūpā iḍā nāḍī varttate /      \D
%vāmabhāge caṃdrarūpā iḍā nāḍī vartate         \U1
%vāmabhāge caṃdrarūpā     nāḍī pravarttate //  \U2
%-----------------------
%On the left side is the \textit{iḍā}-channel, being a resemblence of the moon.
%-----------------------        
\note[type=testium, labelb=24, lem={\textbf{Ci}}]{\textit{Yogasaṃgraha} IGNCA 30020 folio 1r. ll. 4: vāmabhāge caṃdrarūpā iḍā |}
vāmabhāge candrarūpā
        \app{\lem[wit={ceteri}]{iḍā}
          \rdg[wit={U2}]{\om}}
        \app{\lem[wit={ceteri}]{nāḍī}}
        \app{\lem[wit={ceteri}]{vartate}
          \rdg[wit={U2}]{pravarttate}}/
          \note[type=source, labelb=24x, lem={\textbf{Re}}]{PT\textsuperscript{ccn \cdot YSV} (Ed. p. 832): gudorddhe (\textit{gudordhve} YK\textsuperscript{ccn \cdot YSV} 1.247 Ed. p. 20) sā tribhāgābhūdiḍā (\textit{tridhā bhūyādiḍāvāme} YK\textsuperscript{ccn \cdot YSV} 1.247 Ed. p. 20) nāma śaśiprabhā | śaktirūpā mahānāḍī dhyānāt sarvārthadāyinī | dakṣiṇe 'pi kulākhyeti (\textit{piṅgalākhyeti} YK\textsuperscript{ccn \cdot YSV} 1.248 Ed. p. 20) puṃrūpā sūryavigrahā |}
%-----------------------
% \om                                                \B
%dakṣiṇabhāge  sūryarūpā piṅgalā  nāḍī    varttate /  \E
%dakṣiṇabhāge  sūryarūpā piṃgalā  nāḍī    varttate    \P
%dakṣiṇabhāge  sūryarūpā piṃgalā  nāḍī    varttate // \L
%dakṣiṇabhāge  sūryarūpā piṃgalā  nāḍī    varttate // \N1
%dakṣiṇabhāge  sūryarūpā piṃgalā  nāḍī    varttate/   \N2
%dakṣiṇabhāge  sūryarūpā piṃgalā  nāḍī    varttate // \D       
%dakṣiṇe bhāge sūryarūpā piṃgalā  nāḍī    vartate     \U1
%dakṣiṇabhāge  sūryarūpā piṃgalā  nāḍī pravartate //  \U2
%-----------------------
%On the right side exists the \textit{piṅgalā}-channel, being a resemblence of the sun.        
%-----------------------
\note[type=testium, labelb=25, lem={\textbf{Ci}}]{\textit{Yogasaṃgraha} IGNCA 30020 folio 1r. ll. 4: dakṣiṇabhāge sūryarūpā piṃgalā |}
        \app{\lem[wit={ceteri}]{dakṣiṇabhāge}
          \rdg[wit={U1}]{dakṣiṇe bhāge}}
        sūryarūpā piṅgalā nāḍī
        \app{\lem[wit={ceteri}]{vartate}
          \rdg[wit={U2}]{pravarttate}}/
%-----------------------
% \om                                                                   \B
%madhyamārge `tisūkṣmā padminī taṃtusamākārā  koṭividyutsamaprabhā      \E
%madhyamārge `tisūkṣmā padmanī taṃtusamākāra  koṭividyutsamaprabhā      \P
%madhyamārge `tisūkṣmā padmanī taṃtusamākārā  koṭividyutsamaprabhā      \L
%madhyamārge atisūkṣmā padmanī taṃtusamākārā  koṭividyutsamaprabhā //   \N1
%madhyamārge atisūkṣmā padmanī taṃtusamākārā  koṭividyutsamaprabhā //   \N2
%madhyarge   atisūkṣmā padminī taṃtusamākārā  koṭividyutsamaprabhā //   \D
%madhyamārge atisūkṣmā padminī taṃtusamākārā  koṭividyutsamaprabaḥ      \U1
%madhyamārge  tisūkṣmā padminī taṃtusamākārā  koṭividyutsamaprabhā //   \U2
%-----------------------
%Within the middle path is a lotuspond being very subtle. [It is] made from a web of light [and it] shines like a thousand lightnings.
%----------------------- 
\note[type=testium, labelb=26, lem={\textbf{Ci}}]{\textit{Yogasaṃgraha} IGNCA 30020 folio 1r. ll. 5: madhyamārge atisūkṣmā visataṃtusamākārā koṭividyutprabhā}
\note[type=source, labelb=26x, lem={\textbf{Re}}]{PT\textsuperscript{ccn \cdot YSV} (Ed. p. 832): madhyabhāge suṣumnākhyā brahmaviṣṇuśivātmikā | śuddhacittena sā vijñā vidyutkoṭisamaprabhā | bhuktimuktipradā dhyānād aṇimādiguṇapradā|}
        \app{\lem[wit={ceteri}]{madhyamārge}
          \rdg[wit={D}]{madhyarge}}
        \app{\lem[wit={E,P,L,U2}]{'tisūkṣmā}
          \rdg[wit={D,N1,N2,U1}]{atisūkṣmā}}
        \app{\lem[wit={ceteri}]{padminī}
          \rdg[wit={P,L,N1,N2}]{padmanī}}/
        \app{\lem[wit={ceteri}]{tantusamākārā}
          \rdg[wit={P}]{taṃtusamākāra°}}
      koṭividyutsama\app{\lem[wit={ceteri},alt={°prabhā}]{prabhā}
        \rdg[wit={U1}]{°prabhaḥ}}/
      \note[type=testium, labelb=27, lem={\textbf{Re}}]{\textit{Siddhasiddāntapaddhati} 2.26 (Ed. p. 38): mūlakandād daṇḍalagnāṃ brahmanāḍīṃ śvetavarṇāṃ brahmarandhraparyantaṃ gatāṃ saṃsmaret | tanmadhye kamalatantunibhāṃ vidyutkoṭiprabhām ūrdhvagāminīṃ tāṃ mūrtiṃ manasā lakṣayet | sarvasiddhipradā bhavati |}
%-----------------------
%\om                                                                                                                                                                 \B
%bhuktimuktipradā                                     'syā jñānotpattau satyaṃ puruṣaḥ sarvajño  bhavati   \E
%bhuktimuktidā                                        asyā jñānotpattau satyāṃ puruṣaḥ sarvajño  bhavati   \P
%bhuktimuktipradā //                                  asyā jñānotpattau satyāṃ puruṣaḥ sarvajño  bhavati   \L
%bhuktimukti--------------------------------------------------dotpanne  sati---puruṣaḥ sarrvajño bhavati   \N1
%bhuktimukti--------------------------------------------------dotpanne  sati---puruṣaḥ sarrvajño bhavati   \N2
%bhuktimukti--------------------------------------------------dotpanne  sati---puruṣaḥ sarrvajño bhavati   \D1 
%bhuktimukti--------------------------------------------------dotpanne  sati---puruṣaḥ sarrvajño bhavati   \U1
%bhuktimuktidā śivarūpiṇī suṣumṇā nāḍī pravarttate // asyā jñānotpattau satyāṃ puruṣa--sarvajño  bhavati   \U2
%-----------------------
%She \extra{emerges as the central channel, assuming the form of benevolence (\textit{śiva}),} is the bestower of enjoyment and liberation. While abiding in (\textit{satyāṃ}) her (\textit{asyāṃ}) knowledge arises [to the point of which]%the person becomes all-knowing.
%-----------------------
\note[type=testium, labelb=27a, lem={\textbf{Ci}}]{\textit{Yogasaṃgraha} IGNCA 30020 folio 1r. ll. 5-6: bhuktimuktipradā suṣumnā nāḍī vartate | yasyāḥ jñāne purusaḥ sarvajño bhavati |}
  \app{\lem[wit={P,U2}]{bhuktimuktidā}
  \rdg[wit={ceteri}]{bhuktimuktido°}
  \rdg[wit={E,L}]{bhuktimuktipradā}}
   %\rdg[wit={U2}]{bhuktimuktidā śivarūpiṇī suṣumṇā nāḍī pravarttate}} %Lesart oder einfach zusätzliches Material? 
   %\textcolor{red}{śivarūpiṇī suṣumṇā nāḍī pravarttate/}
\extra{\app{\lem[wit={U2}]{śivarūpiṇī suṣumṇā nāḍī pravarttate}
    \rdg[wit={ceteri}]{\om}}/}
\note[type=philcomm, labelb=27b, lem={śivarūpiṇī}]{Sentences unlikely
  to be authorial, but enriching, are included within the edition and marked in blue.}
\app{\lem[resp=egoscr, type=emendation]{'syāṃ}
      \rdg[wit={E}]{'syā}
      \rdg[wit={P,L,U2}]{asyā}
      \rdg[wit={ceteri}]{\om}}
    \app{\lem[wit={E,P,L,U2}]{jñānotpattau}
      \rdg[wit={ceteri}]{°tpanne}}
    \app{\lem[wit={P,L,U2}]{satyāṃ}
      \rdg[wit={E}]{satyaṃ}
      \rdg[wit={ceteri}]{sati}}
    sarvajño bhavati\dd{}    
    \end{prose}
    % \vspace*{\fill}
\ekdpb*{}
\end{edition}
\begin{translation}
  \ekddiv{type=trans}
   \bigskip
    \centerline{\textrm{\small{[Siddhakuṇḍalinīyoga and Mantrayoga]}}}
    \bigskip
    \begin{tlate}
      Now varieties of Rājayoga will be described. \\\\
      \indent Which are these? One is Siddhakuṇḍalinīyoga\footnote{It is surprising to note the use of the term \textit{siddhakuṇḍalinīyoga} instead of \textit{siddhayoga} as listed initially. Furthermore, it is intriguing that this type of Yoga, which was listed as the second-last item in the Yoga taxonomy, is introduced as the second type right after Kriyāyoga, which was the first item in both the initial list and in the subsequent material. This raises further questions as the term \textit{kuṇḍaliṇī} is not mentioned at all in the subsequent description of this type of Yoga, adding to the mysterious nature of this terminology.} [and one\footnote{The distinction between "Siddhakuṇḍalinīyoga" and "Mantrayoga" is not entirely clear. The witness U\textsubscript{2} provides some description of Mantrayoga, but this raises questions about the relationship between the two. Based on the information from U\textsubscript{2} alone, it could be translated as "Siddhakuṇḍalinīyoga being Mantrayoga." However, given the lack of information from the other witnesses, the contents of this passage remain unclear.}] is Mantrayoga\footnote{The sudden appearance of the term \textit{mantrayoga} in this section is peculiar as the subsequent section does not mention the practice of mantras at all. This discrepancy may be the result of an early scribe's mistake that was subsequently copied by many of the manuscripts. All witnesses except L (L omits the term \textit{mantraygoa}) preserve this reading, and the following sentence supports the reading of \textit{mantrayoga} through the use of dual forms. The structure and content of \textit{Yogatattvabindu} closely follow \textit{Yogasvarodaya}, as quoted in \textit{Prāṇatoṣiṇī} and \textit{Yogakarṇikā}. However, the Yoga introduced in \textit{Yogasvarodaya} at this point is \textit{jñānayoga}, which is taken up by \textit{Yogatattvabindu}. It is also possible that, in the early transmission of the text, folios were lost or became confused, leading to a diffuse arrangement of the five types of Lakṣyayoga and missing Yogas. This issue cannot currently be resolved. Only the additional passages of witness U\textsubscript{2}, highlighted in blue, indirectly allude to a practice of mantra. U\textsubscript{2} prescribes the \textit{japājapa} of \textit{so 'haṃ} during meditation for almost each \textit{cakra}.}. These two Rājayogas are described [in the following]. \\\\
      \indent At the location of the root-bulb exists one major vessel in the form of energy. This single vessel reaches to these openings which are \textit{iḍā}, \textit{piṅgalā} and \textit{suṣumnā}. On the left side is the \textit{iḍā}-channel, being a resemblence of the moon. On the right side exists the \textit{piṅgalā}-channel, being a resemblence of the sun. Within the middle path is a lotuspond being very subtle. [It is] made from a web of light [and it] shines like a thousand lightnings.\\\\
\indent She \extra{emerges as the central channel assuming the form of benevolence (\textit{śiva}), [and]} is the bestower of enjoyment and liberation. While abiding in (\textit{satyāṃ}) her (\textit{asyāṃ}) knowledge arises. The person becomes all-knowing. \vspace*{\fill}
    \end{tlate}
    \ekdpb*{}
   \end{translation}
 \end{alignment}
%%%%%%%%%%%%%%%%%%%%%%%%%%%%%%%%%%%%%%%%%%
%%%%%%%%%%%%%%%%%%%%%%%%%%%%%%%%%%%%%%%%%%
%%%%%%%%PAGEBREAK%%%%%%%PAGEBREAK%%%%%%%%%
%%%%%%%%%%%%%%%%%%%%%%%%%%%%%%%%%%%%%%%%%%
%%%%%%%%%%%%%%%%PAGEBREAK%%%%%%%%%%%%%%%%%
%%%%%%%%%%%%%%%%%%%%%%%%%%%%%%%%%%%%%%%%%%
%%%%%%%%PAGEBREAK%%%%%%%PAGEBREAK%%%%%%%%%
%%%%%%%%%%%%%%%%%%%%%%%%%%%%%%%%%%%%%%%%%%
%%%%%%%%%%%%%%%%%%%%%%%%%%%%%%%%%%%%%%%%%%
%%%%%%%%%%%%%%%%%%%%%%%%%%%%%%%%%%%%%%%%%%
%%%%%%%%%%%%%%%%%%%%%%%%%%%%%%%%%%%%%%%%%%
%%%%%%%%PAGEBREAK%%%%%%%PAGEBREAK%%%%%%%%%
%%%%%%%%%%%%%%%%%%%%%%%%%%%%%%%%%%%%%%%%%%
%%%%%%%%%%%%%%%%PAGEBREAK%%%%%%%%%%%%%%%%%
%%%%%%%%%%%%%%%%%%%%%%%%%%%%%%%%%%%%%%%%%%
%%%%%%%%PAGEBREAK%%%%%%%PAGEBREAK%%%%%%%%%
%%%%%%%%%%%%%%%%%%%%%%%%%%%%%%%%%%%%%%%%%%
%%%%%%%%%%%%%%%%%%%%%%%%%%%%%%%%%%%%%%%%%%
%%%%%%%%%%%%%%%%%%%%%%%%%%%%%%%%%%%%%%%%%%
%%%%%%%%%%%%%%%%%%%%%%%%%%%%%%%%%%%%%%%%%%
%%%%%%%%PAGEBREAK%%%%%%%PAGEBREAK%%%%%%%%%
%%%%%%%%%%%%%%%%%%%%%%%%%%%%%%%%%%%%%%%%%%
%%%%%%%%%%%%%%%%PAGEBREAK%%%%%%%%%%%%%%%%%
%%%%%%%%%%%%%%%%%%%%%%%%%%%%%%%%%%%%%%%%%%
%%%%%%%%PAGEBREAK%%%%%%%PAGEBREAK%%%%%%%%%
%%%%%%%%%%%%%%%%%%%%%%%%%%%%%%%%%%%%%%%%%%
%%%%%%%%%%%%%%%%%%%%%%%%%%%%%%%%%%%%%%%%%%