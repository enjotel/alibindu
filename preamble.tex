\documentclass[10pt]{memoir}
\setstocksize{220mm}{155mm} 	        
\settrimmedsize{220mm}{155mm}{*}	
\settypeblocksize{170mm}{116mm}{*}	
\setlrmargins{18mm}{*}{*}
\setulmargins{*}{*}{1.2}
%\setlength{\headheight}{5pt}
\checkandfixthelayout[lines]
\linespread{1.16}

%%% Hyphenation settings
\usepackage[htt]{hyphenat}
\hyphenation{he-lio-trope opos-sum}
\tracingparagraphs=1
%Hyphenation in Devanāgarī of the edition still missing? Probably this needs to be modified in babel-iast package? 

%%% babel
\usepackage[english]{babel}
\usepackage{babel-iast/babel-iast}

%%maxens extrabefehle 
%\usepackage[sanskrit]{babel}
%\babelprovide[transforms = transliteration.iast]{sanskrit}

\babelfont[iast]{rm}[Renderer=Harfbuzz, Scale=1.3]{AdishilaSan}%AdishilaSan}
\babelfont[english]{rm}{Adobe Text Pro}

%\tolerance=9999
%\emergencystretch=10pt
%\hyphenpenalty=10000
%\exhyphenpenalty=100


%%% footnote without a marker
%\newcommand\blfootnote[1]{%
%\begingroup
%\renewcommand\thefootnote{}\footnote{\noindent #1}%
%\addtocounter{footnote}{-1}%
%\endgroup
%}
%%%%%%%%%%%%%%%%%%%%%%%%%%%
\makeatletter
\def\blfootnote{\gdef\@thefnmark{}\@footnotetext}
%\renewcommand*{\@makefntext}[1]{%
%  \noindent\makebox[0pt][r]{\@makefnmark}#1%
%}
\makeatother
%%%%%%%%%%%%%%%%%%%%%%%%%

%\makeatletter
%5\def@makefntext#1{%
%\noindent
%\makebox[0.75em][l]{@thefnmark.}%
%\hspace{0.25em}#1}
%\makeatother

%\usepackage{etoolbox}
%\newcounter{blfootnote}
%\newenvironment{blfootnote}[1][]
%{\refstepcounter{blfootnote}%
%\begingroup%
%\def\thefootnote{\arabic{blfootnote}}%
%\ifstrempty{#1}{}{\noindent\hangindent=0.75em\hangafter=1 #1}\footnotetext}
%{\endgroup}
  
%%% more functionality
\PassOptionsToPackage{hyphens}{url}
\usepackage{hyperref}
\usepackage{pdflscape}
\usepackage{cleveref}
\usepackage{url}
\usepackage{cleveref}
\usepackage{microtype}
\usepackage{lineno}
%\usepackage[none]{hyphenat}
%\newcommand{\dd}[1]{#1\nobreak\hspace{0pt plus 1pt}\nobreak\ignorespaces}

%\usepackage{bigfoot}
%%% more functions
\usepackage[dvipsnames]{xcolor}
% \usepackage[flushmargin]{footmisc}
%\usepackage[para,perpage]{footmisc}

%%%für den Counter von Kapiteln und Sätzen! 
\newcommand{\uproman}[1]{\uppercase\expandafter{\romannumeral#1}}
\newcommand{\lowroman}[1]{\romannumeral#1\relax}

\makeindex
\newfontfamily\sanskritfont[Script=Devanagari,Mapping=RomDev,Scale=1.1]{Sanskrit2003}
\usepackage{pifont,fourier-orns,lettrine,psvectorian,paralist,enumitem,pdfpages,wrapfig,tabulary,lettrine,longtable}
\setlist[enumerate]{itemsep=0mm}
\usepackage[autostyle]{csquotes}
\usepackage[defaultlines=2,all]{nowidow}
\usepackage{ellipsis,adforn,booktabs,longtable,url,tikz}
\lineskiplimit=-3pt          

\makechapterstyle{IeT}{%
  \chapterstyle{default}
  \renewcommand*{\printchapternonum}{\centering}
  \renewcommand*{\clearforchapter}{\cleartorecto} 
  \aliaspagestyle{chapter}{empty}}
\chapterstyle{IeT}
\setsecnumdepth{none}  \openright  \nouppercaseheads
\settocdepth{subsubsection}

%%%% test better pagebreaks
\def\fussy{%
  \emergencystretch\z@
  \tolerance 200%
  \hfuzz .1\p@
  \vfuzz\hfuzz}

\interfootnotelinepenalty=10000\relax

\usepackage[maxfloats=256]{morefloats}

\maxdeadcycles=500

\raggedbottomsectiontrue
\checkandfixthelayout

%%%
%\setlength{\parskip}{2cm plus0.5cm minus0.5cm}

%%%%%%%  biblatex
\newcommand{\noun}[1]{\textsc{#1}}    %  philosophy-verbose
\usepackage[backend=biber, sorting=nyt, style=verbose]{biblatex} %%%%ORIGINAL TiE
\renewcommand*{\mkbibnamefamily}[1]{\textsc{#1}}
% scauthors=all, publocformat=locpubyear,
% inbeforejournal=true, volnumformat=plain,relatedformat=parens
%\renewcommand*{\volnumpunct}{.}
%\addbibresource{GYS-Iet.bib}

\DeclareFieldFormat{url}{%
  \mkbibacro{URL}\addcolon\space
  \href{#1}{\nolinkurl{\thefield{urlraw}}}}

\DeclareFieldFormat{citeurl}{%
  \href{#1}{\nolinkurl{\thefield{urlraw}}}} 


%Index
%\usepackage[backend=biber, bibstyle=verbose, citestyle=authoryear]{biblatex}
%
%\DefineBibliographyStrings{german}{
%  references = {Literaturverzeichnis},
%  bibliography = {Bibliographie},
%  shorthands = {Abkürzungen der Zeugen des kritischen Apparatus},
%}

%\makeatletter
%\renewcommand*{\mkbibnamefamily}[1]{%
%  \ifdefstring{\blx@delimcontext}{parencite}
%    {\textsc{#1}}
%    {#1}}
%\makeatother

\DeclareFieldFormat{postnote}{#1}
\renewcommand{\postnotedelim}{, }
\addbibresource{bindu.bib}

%%% ekdosis
\usepackage[teiexport=tidy,parnotes=true]{ekdosis}% =tidy cleans up HTML and XML documents by fixing markup errors and upgrading legacy code to modern standards. parnotes=footnotes below or above critical apparatus

\SetLineation{lineation=page} %lineation=page sets thenumbering to start afresh at the top of each page. =modulo makes every fifth line numbered. 

\renewcommand{\linenumberfont}{\selectlanguage{english}\footnotesize} %sets language of lines to English

\SetTEIxmlExport{autopar=false} %autopar=falseinstructs ekdosis to ignore blank lines in the.tex sourcefile as markers for paragraph boundaries. As a result, each paragraph of the edition must be found within an environment associated with the xml <p> element

\SetHooks{
  lemmastyle=\bfseries,
  %refnumstyle=\selectlanguage{english}\bfseries,
  refnumstyle=\selectlanguage{english}\color{blue}\bfseries,
  appheight=0.8\textheight,
}

\newif\ifinapparatus
\DeclareApparatus{source}[
%bhook=\inapparatustrue,
lang=english,
notelang=english,
% bhook=\selectlanguage{english},
bhook=\selectlanguage{english}\textbf{Sources:},%
%maxentries=4, 
%ehook=.]
%sep={] },
%nosep,
]

\newif\ifinapparatus
\DeclareApparatus{testium}[
%bhook=\inapparatustrue,
lang=english,
notelang=english,
% bhook=\selectlanguage{english},
bhook=\selectlanguage{english}\textbf{Testimonia:},
%maxentries=4, 
%ehook=.]
%nosep, 
]

% Declare \ifinapparatus and set \inapparatustrue at the beginning of
% the apparatus criticus block. Also set the language.  
\newif\ifinapparatus
  \DeclareApparatus{default}[
  %bhook=\inapparatustrue, 
  lang=english,
  %maxentries=33,
  %bhook=\selectlanguage{english},
  sep = {] },
  delim=\hskip 0.75em,
  rule=\rule{0.7in}{0.4pt},
]

\newif\ifinapparatus
\DeclareApparatus{philcomm}[
%bhook=\inapparatustrue,
lang=english,
notelang=english,
bhook=\selectlanguage{english}\textbf{Philological Commentary:},
%bhook=\selectlanguage{english},
sep={: },
]

\ekdsetup{
showpagebreaks,
spbmk = \textcolor{blue}{spb},
hpbmk = \textcolor{red}{hpb}
}

% Macros and Definitions for the Print of Sigla
\def\acpc#1#2#3{{#1}\rlap{\textrm{\textsuperscript{#3}}}\textsubscript{\textrm{#2}}\space}
\def\sigl#1#2{{{#1}}\textsubscript{\textrm{#2}}}
\def\None{{\sigl{N}{1}}} \def\Noneac{\acpc{N}{1}{ac}\,} \def\Nonepc{\acpc{N}{1}{pc}\,}
\def\Ntwo{{\sigl{N}{2}}} \def\Noneac{\acpc{N}{2}{ac}\,} \def\Nonepc{\acpc{N}{2}{pc}\,}
\def\Done{{\sigl{D}{1}}} \def\Doneac{\acpc{D}{1}{ac}\,} \def\Donepc{\acpc{D}{1}{pc}\,}
\def\Dtwo{{\sigl{D}{2}}} \def\Dtwoac{\acpc{D}{2}{ac}\,} \def\Dtwopc{\acpc{D}{2}{pc}\,}
\def\Uone{{\sigl{U}{1}}} \def\Uoneac{\acpc{U}{1}{ac}\,} \def\Uonepc{\acpc{U}{1}{pc}\,}                 
\def\Utwo{{\sigl{U}{2}}} \def\Utwoac{\acpc{U}{2}{ac}\,} \def\Utwopc{\acpc{U}{2}{pc}\,}

%%%%%%%%%%%%%% Tattvabinduyoga - List of Witnesses   %%%%%%%%%%%%%%%%%%%
\DeclareWitness{ceteri}{\selectlanguage{english}cett.}{ceteri}[]   
\DeclareWitness{E}{\selectlanguage{english}E}{Printed Edition}[]    
\DeclareWitness{P}{\selectlanguage{english}P}{Pune BORI 664}[]  
\DeclareWitness{B}{\selectlanguage{english}B}{Bodleian 485}[]       
\DeclareWitness{N1}{\selectlanguage{english}N\textsubscript{1}}{NGMPP 38/31}[]
\DeclareWitness{N2}{\selectlanguage{english}N\textsubscript{2}}{NGMPP B 38/35}[]
\DeclareWitness{L}{\selectlanguage{english}L}{LALCHAND 5876}[]  
\DeclareWitness{D}{\selectlanguage{english}D}{IGNCA 30019}[] 
%\DeclareWitness{D2}{\selectlanguage{english}D\textsubscript{2}}{IGNCA 30020}[]  
\DeclareWitness{U1}{\selectlanguage{english}U\textsubscript{1}}{SORI 1574}[] 
\DeclareWitness{U2}{\selectlanguage{english}U\textsubscript{2}}{SORI 6082}[]
%%%%%%%%%%%%%% Tattvabinduyoga - Groups of Witnesses   %%%%%%%%%%%%%%%%%%%
\DeclareWitness{X}{\selectlanguage{english}\alpha}{Alpha Group: D,N1,N2,U1}[]
\DeclareWitness{Y}{\selectlanguage{english}\beta}{Beta Group: B,E,L,P,U2}[]
%%%%%%%%%%%%% Testimonia
\DeclareWitness{Ysv}{\selectlanguage{english}Ysv}{Yogasvarodaya}[] %%%add infos!  

%%%%%%%%%%%%%%%%%%%%%%%%%%%%%%%%%%%%%%%%%%%
% Macro for Editing Abbrevs.
\def\om{\textrm{\footnotesize \textit{om.}\ }} %prints om. for omitted in apparatus
\def\korr{\textrm{\footnotesize \textit{em.}\ }} %prints em. for emended in apparatus
\def\conj{\textrm{\footnotesize \textit{conj.}\ }} %prints conj. for conjectured in apparatus

% \supplied{text} EDITORIAL ADDITION -> Within \lem oder \rdg
% \surplus{text} EDITORIAL DELETION -> Within \lem oder \rdg
% \sic{text} CRUX
% \gap{text} LACUNAE -> [reason=??, unit=??, quantity=??, extent=??]


%%%%%%%%%%%%%%%%%%%%%%%%%%%%%%%%%%%%%%%%%%% All macros of this list can be used in 
% Macro for Editing Abbrevs.
\def\eyeskip{\textrm{{ab.\,oc. }}}
\def\aberratio{\textrm{{ab.\,oc. }}}
\def\ad{\textrm{{ad}}}
\def\add{\textrm{{add.\ }}}
\def\ann{\textrm{{ann.\ }}}
\def\ante{\textrm{{ante }}} 
\def\post{\textrm{{post }}}
%\def\ceteri{cett.\,}                   
\def\codd{\textrm{{codd.\ }}}

\def\coni{\textrm{{coni.\ }}}
\def\contin{\textrm{{contin.\ }}}
\def\corr{\textrm{{corr.\ }}}
\def\del{\textrm{{del.\ }}}
\def\dub{\textrm{{ dub.\ }}}

\def\expl{\textrm{{explic.\ }}} 
\def\explica t{\textrm{{explic.\ }}}
\def\fol{\textrm{{fol.\ }}}
\def\foll{\textrm{{foll.\ }}}
\def\gloss{\textrm{{glossa ad }}}
\def\ins{\textrm{{ins.\ }}}      
\def\inseruit{\textrm{{ins.\ }}} 
\def\im{{\kern-.7pt\lower-1ex\hbox{\textrm{\tiny{\emph{i.m.}}}\kern0pt}}} %\textrm{\scriptsize{i.m.\ }}}      
\def\inmargine{{\kern-.7pt\lower-.7ex\hbox{\textrm{\tiny{\emph{i.m.}}}\kern0pt}}}%\textrm{\scriptsize{i.m.\ }}}      
\def\intextu{{\kern-.7pt\lower-.95ex\hbox{\textrm{\tiny{\emph{i.t.}}}\kern0pt}}}%\textrm{\scriptsize{i.t.\ }}}           
\def\indist{\textrm{{indis.\ }}}  
\def\indis{\textrm{{indis.\ }}}
\def\iteravit{\textrm{{iter.\ }}} 
\def\iter{\textrm{{iter.\ }}}
\def\lectio{\textrm{{lect.\ }}}   
\def\lec{\textrm{{lect.\ }}}
\def\leginequit{\textrm{{l.n. }}} 
\def\legn{\textrm{{l.n. }}}
\def\illeg{\textrm{{l.n. }}}

\def\primman{\textrm{{pr.m.}}}
\def\prob{\textrm{{prob.}}}
\def\rep{\textrm{{repetitio }}}
\def\secundamanu{\textrm{\scriptsize{s.m.}}}            \def\secm{{\kern-.6pt\lower-.91ex\hbox{\textrm{\tiny{\emph{s.m.}}}\kern0pt}}}%   \textrm{\scriptsize{s.m.}}}
\def\sequentia{\textrm{{seq.\,inv.\ }}}  
\def\seqinv{\textrm{{seq.\,inv.\ }}}
\def\order{\textrm{{seq.\,inv.\ }}}
\def\supralineam{{\kern-.7pt\lower-.91ex\hbox{\textrm{\tiny{\emph{s.l.}}}\kern0pt}}} %\textrm{\scriptsize{s.l.}}}
\def\interlineam{{\kern-.7pt\lower-.91ex\hbox{\textrm{\tiny{\emph{s.l.}}}\kern0pt}}}   %\textrm{\scriptsize{s.l.}}}
\def\vl{\textrm{v.l.}}   \def\varlec{\textrm{v.l.}} \def\varialectio{\textrm{v.l.}}
\def\vide{\textrm{{cf.\ }}}
\def\cf{\textrm{{cf.\ }}} 
\def\videtur{\textrm{{vid.\,ut}}}
\def\crux{\textup{[\ldots]} }
\def\cruxx{\textup{[\ldots]}}
\def\unm{\textit{unm.}}
%%%%%%%%%%%%%%%%%%%%%%%%%%%%%%%%%%%%

% List of Scholars
\DeclareScholar{ego}{ego}[
forename=Nils Jacob,
surname=Liersch]

% Persons:14\DeclareScholar{ego}{ego}[15forename=Robert,16surname=Alessi]17% Useful shorthands:18\DeclareShorthand{codd}{codd.}{V,I,R,H}19\DeclareShorthand{edd}{edd.}{Lit,Erm,Sm}20\DeclareShorthand{egoscr}{\emph{scripsi}}{ego}

%Useful shorthands:
%\DeclareShorthand{codd}{codd.}{V,I,R,H}
%\DeclareShorthand{edd}{edd.}{Lit,Erm,Sm}
\DeclareShorthand{egoscr}{em.}{ego}
\DeclareShorthand{egoscrconj}{conj.}{ego}
\DeclareShorthand{egomute}{\unskip}{ego}

\usepackage{xparse}

\NewDocumentEnvironment{tlg}{O{}O{}}{\setlength{\leftskip}{0pt}\vspace{-1ex}\begin{quotation}}{\hfill #1\ \vspace{-1ex}\end{quotation}\vspace{-1ex}} %verse environment
%\NewDocumentEnvironment{tlg}{O{}O{}}{\begin{verse}}{॥#1\hskip-4pt ॥\\ \end{verse}}
\NewDocumentCommand{\tl}{m}{{\selectlanguage{iast} #1}}

\NewDocumentCommand{\extra}{m}{{\textcolor{gray}{#1}}} %command for additions to U2
\NewDocumentCommand{\crazy}{m}{{\textcolor{red}{#1}}} %totally corrupted passage
\NewDocumentCommand{\coro}{m}{{\textcolor{violet}{#1}}} %colour for sentence counter! 

\NewDocumentEnvironment{prose}{O{}}{\begin{otherlanguage}{iast}}{\end{otherlanguage}}
% \NewDocumentEnvironment{padd}{O{}}{\begin{otherlanguage}{iast}}{\end{otherlanguage}}
\NewDocumentEnvironment{tlate}{O{}}
%\NewDocumentEnvironment{tadd}{O{}}

%Define two commands: \skp ("sanskrit plus"), to be ignored by TeX in
%the edition text, but processed in the TEI output. Conversely, \skm
%("sanskrit minus") is to be processed in the edition text, but
%ignored if found in the apparatus criticus and in the TEI output:

\NewDocumentCommand{\skp}{m}{}
\TeXtoTEIPat{\skp {#1}}{#1}

%\NewDocumentCommand{\skpp}{m}{}
%\TeXtoTEIPat{\skpp {#1}}{#1}

\NewDocumentCommand{\skm}{m}{\unless\ifinapparatus#1-\fi}
\TeXtoTEIPat{\skm {#1}}{}

% \NewDocumentCommand{\dd}{}{/\hskip-4pt/}
\NewDocumentCommand{\dd}{}{\mbox{/\hskip-4pt/}}
\TeXtoTEIPat{\dd {}}{//}


%%% modify environments and commands
%%% TEI mapping
\TeXtoTEIPat{\begin {tlg}}{<lg>} %lg=(Group of verse (s)) contains one or more verses or lines of verse that together form a formal unit (e.g. stanza, chorus).
\TeXtoTEIPat{\end {tlg}}{</lg>}

\TeXtoTEIPat{\begin {prose}}{<p>}
\TeXtoTEIPat{\end {prose}}{</p>}

\TeXtoTEIPat{\begin {tlate}}{<p>}
\TeXtoTEIPat{\end {tlate}}{</p>}

\TeXtoTEIPat{\\}{}
\TeXtoTEIPat{\linebreak}{<br/>}
\TeXtoTEIPat{\noindent}{}
%\TeXtoTEI{tl}{l}
\TeXtoTEI{emph}{hi}
\TeXtoTEI{bigskip}{}
\TeXtoTEI{None}{N1}
\TeXtoTEI{Ntwo}{N2}
\TeXtoTEI{Done}{D1}
\TeXtoTEI{Dtwo}{D2}
\TeXtoTEI{Uone}{U1}
\TeXtoTEI{Utwo}{U2}
%\TeXtoTEIPat{/}{ |}
%\TeXtoTEI{//}{ ||}
\TeXtoTEIPat{\korr}{em. }
\TeXtoTEIPat{\conj}{conj.}
\TeXtoTEIPat{\om}{om.}
\TeXtoTEIPat{english}{}
\TeXtoTEIPat{\hskip}{}
\TeXtoTEIPat{\hskip-4pt}{}
\TeXtoTEIPat{\hskip-2pt}{}
\TeXtoTEIPat{-}{ }
\TeXtoTEIPat{4pt}{}
\TeXtoTEIPat{2pt}{}
\TeXtoTEIPat{\textcolor {#1}{#2}}{<hi rend="#1">#2</hi>} 

% Nullify \selectlanguage in TEI as it has been used in
% \DeclareWitness but should be ignored in TEI.
\TeXtoTEI{selectlanguage}{}


