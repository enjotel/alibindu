\chapter{Introduction}
\mainmatter

\begin{quote}
nādakoṭisahasrāṇi bindukoṭiśatāni ca/
sarve tatra layaṃ yānti yatra devo nirañjanaḥ//
\end{quote}

Thousands of crores of resonances and hundreds of crores of visual focal points, all dissolve into the place where the unadorned god is.

\citetitle{hathapradipika2024}

\chapter{The List of the 15 Yogas}
\label{yogas_list}
The authenticity of the list specifying the fifteen Yogas at the beginning of the text is ambiguous. This is due to the discrepancy between the structure of the Yogas presented in the text and the order presented in the list. For example, the text commences with a description of \textit{kriyāyoga} and goes on to describe \textit{siddhakuṇḍaliniyoga} and then mentions \textit{mantrayoga} without adhering to the order presented in the list. This incongruity raises questions as to why the text structure deviates from the list. However, the reference to \textit{jñānotpattav upāyaḥ} may provide some insight into why \textit{jñānayoga} is included as the second \textit{yoga} in the list. To reconcile these apparent inconsistencies, there are several possible explanations: 1) The text is severely corrupted. 2) The list was added by a different hand at a later time. 3) The term \textit{jñānayoga} is included as a result of the practice of \textit{siddhakuṇḍalinīyoga}, which is said to generate knowledge through the central channel, as stated in the text. These explanations may be combined to provide a comprehensive understanding of the situation.

\section{Lakṣyayoga}

\begin{itemize}
\item origin tantric Traditions -> e.g. Netratantra
\item also check Mālinivijayottara 2004 Vasudeva pp. 256-257
\item also \citetitle{birch2013} 2.10 Śāmbhavī Mudrā
  \end{itemize} 

\chapter{Sources}
\section{The Additions of  SORI 6082 - U\textsubscript{2}}
\label{discussionu2}
Analyse the additions of U\textsubscript{2} and present the \textit{cakra}s and their attriubutes in a table .
\begin{itemize}
\item  Muktabodha-Texte sehe ich 3 Belege für bahiśśakti Muktabodha/krīyakramādyotikā.html 2938 suṣirānte bahiśśaktiṃ vinyasedvyomarūpiṇīm | tasyā madhye tu Muktabodha/sakalāgamasārasaṅgraha.html 2186 suṣirāntabahiśśaktiṃ vyāpinīṃ cintayet tataḥ || Muktabodha/kriyakramadyotikavyākhyā.html 1846 tanmadhye ca bahiśśaktiṃ sudhābindu parisrutim
  \item  Parā\footnote{Im Kaśm. Śiv. °das ewige Wort, in welchem potentiell alle Begriffe und Worte ruhen; vgl. das śabdabrahma des Vyākaraṇa. [B.]― Schmidt S. 246}.
  \end{itemize}

\chapter{Conventions in the Critical Apparatus}
\section{Sigla in the Critical Apparatus}

\begin{itemize}
\item E : Printed Edition
\item P : Pune BORI 664
\item L : Lalchand Research Library LRL5876
\item B : Bodleian Oxford D 4587
\item \None : NGMPP B 38-31
\item \Ntwo : NGMPP B 38-35 / A 1327-14
\item \Done : IGNCA 30019
\item \Uone : SORI 1574
\item \Utwo: SORI 6082
\end{itemize}

The order of the readings in the critical apparatus is arranged according to the quality of readings in decending order. The critical apparatus is positive. Gemitation is not recorded. 

\section{Abbreviations}
\begin{itemize}
  \item qcr: quote cum referencia (quoted with reference)
  \end{itemize}

\section{Marking the Reliability of Sources and Testimonia in the Critical Apparatus}
\label{kennz}

To accurately depict information about the textual relationship and estimated degree of relatedness of a passage from the \textit{Yogatattvabindu} in the layers for sources and testimonia of the critical apparatus, a system of sigla was introduced.\footnote{This type of identification system is based on the use of the critical apparatus in \parencite[lii-liii]{steinkellner2005}. It was modified for the text-critical work on the \textit{Yogatattvabindu}.} The sigla are meaningful when a passage is corrupted in all witnesses and can only be reconstructed by means of other texts. The layers of the critical apparatus for sources and testimonia use the following sigla:

\begin{enumerate}
\item[\textbf{Ce}] \textit{citatum ex alio} / quotation from another (text).\footnote{The sigla \textbf{Ce} indicates an identical or largely identical content in the lesser witness and only allows for minor deviations in the wording of the passage.}
\item[\textbf{Cee}] \textit{citatum ex alio modo edendi} / quotation from another (text) with editorial changes.\footnote{The sigla \textbf{Cee} identifies passages with noticeable deviations in the lesser witness.}
\item[\textbf{Ci}] \textit{citatum in alio} / quotation in another (text).\footnote{The sigla \textbf{Ci} indicates an identical or largely identical content in the lesser witness and only allows for minor deviations in the wording of the passage.}
\item[\textbf{Cie}] \textit{citatum in alio modo edendi} / quotation in another (text) with editorial changes.\footnote{The sigla \textbf{Cie} identifies passages in the lesser witness with noticeable deviations that have the intended character of the composer.}
\item[\textbf{Re}] \textit{relatum ex alio} / (content), attested from another text.\footnote{The sigla \textbf{Re} identifies content parallels in the lesser witness that are relevant to the constitution of the critical text. It further indicates in certain cases that the composer might have used this source when composing his text.}
\item[\textbf{Ri}] \textit{relatum in alio} / (content), attested in another text.\footnote{The sigla \textbf{Ri} identifies content parallels in the lesser witness that are relevant to the constitution of the critical text.}
\end{enumerate}

The following acronyms refer to passages that originated from texts that the author of the \textit{Yogatattvabindu} utilized in compiling his work: \textbf{Ce}, \textbf{Cee}, \textbf{Re}. These texts must predate the \textit{Yogatattvabindu}. The other acronyms, such as \textbf{Ci}, \textbf{Cie}, and \textbf{Ri}, are texts that have adopted passages from the \textit{Yogatattvabindu}, or verses or passages that share similar content with the \textit{Yogatattvabindu}, but their relation is given literally, making it impossible to determine who adopted from whom. \textbf{Re} and \textbf{Ri} each refer to passages that are so closely related in content to those of the \textit{Yogatattvabindu} that they are significant in reconstructing a passage.\footnote{\textbf{Ce} and \textbf{Cee} have the highest degree of reliability, \textbf{Ci} and \textbf{Cie} have a moderate degree, and \textbf{Re} and \textbf{Ri} have the lowest.}

\section{Punctuation}

The inconsistent use of punctuation marks in the available witnesses necessitates standardization. Upon close examination, it appears that punctuation has frequently been dropped or added during the transmission of the texts. The neglect or improper handling of punctuation by the copists has resulted in different versions of lists with and without punctuation. In many instances, missing punctuation has led to the addition of case endings, alteration of the text, and the combination of list items into compound formations that were not present in the original text. Although punctuation plays an important role, deviations in punctuation at the end of sentences, lists, and verse-numbering will only be extensively documented in the critical apparatus of the printed edition. This means that emendations of obvious punctuation mistakes will not be recorded in the critical apparatus. However, the digital edition of this work provides a more detailed documentation of deviations in punctuation through diplomatic transcripts of each witness, and even has a function to display sentences cumulatively.

In the printed edition of the \textit{Yogatattvabindu}, standard conventions of punctuation are followed. In verse poetry, a \textit{daṇḍa} (|) marks the end of a half-verse or half of the \textit{śloka}, and a double \textit{daṇḍa} (||) marks the end of a verse. In prose, a single \textit{daṇḍa} indicates the end of a sentence, and a double \textit{daṇḍa} marks the end of a paragraph. Variations in the use of \textit{avagraha} will be recorded, and items in lists will be separated by a double-\textit{daṇḍa}.

\section{Sandhi}

Among the witnesses we see deviating and inconsistent application of \textit{sandhi}. There is no clear evidence that originally \textit{sandhi} was intentionally not applied. This edition will therefore apply \textit{sandhi} consistently throughout the constituted text to provide a readable text sticking to contemporary conventions in Sanskrit. The variant readings concerning \textit{sandhi} are recorded consistently in the apparatus criticus. This is due to various textcritical problems arising from the inconsistent usage of punctuation which results in application or non-application of \textit{sandhi} wheter the respective witness applied a \textit{daṇḍa} or not. This is particularly the case within lists, which frequently occur in our compilation. Items were most likely originally separated by \textit{daṇḍa}. 


\section{Class Nasals}

Due to inconsistent use of class nasals among the witnesses \textit{anusvāra}s have been substituted with the respective class nasals throughout the edition.

\section{Lists}

Lists are a frequent feature in the \textit{Yogatattvabindu}. The text opens with a list of 15 Yogas and there are many more lists utilized throughout its content. To produce a consistent and easily readable edition, all lists have been identified, normalized to the Nominative Singular or Nominative Plural form of the respective item, or in the case of explanatory lists, to the Ablative Singular or Plural. The items are separated by a double \textit{daṇḍa}. Differences in punctuation and simple punctuation emendations, unless they are text-critically or systematically significant, will not be recorded in the apparatus criticus.
\clearpage

\section{Structural Issues of the Yogatattvabindu}
\label{structure}
\chapter{Related Texts}

\section{Yogasvarodaya}

Note: Mention the parallels to \citetitle{sarada} and how here \textit{svarodaya} plays an important role in the system of yoga. Also there seems to be some distant influence. I think originally there might have been was a larger section of svarodaya or even a chapter in the Yogasvarodaya which was not quoted in PT and YK!! 

\section{Śivayogapradīpikā}
In the \citetitle{shivayogapradipika} 4.41cd-47ab we find descriptions closely resembling those of \citetitle{advaya}:
\begin{quote}
antarlakṣyam iti jñeyaṃ bahirlakṣyam atha śṛṇu ||41||\\
nāsāgradeśāc caturaḥ ṣaḍ aṣṭau tathā daśa dvādaśa saṃkhyayāṅguliḥ |\\
bahiḥ smaren nīlasudhūmraraktataraṅgapītābhasutattvapañcakam ||42||\\
athavā sanmukhākāśaṃ sthiradṛṣṭyā vilakṣayet |\\
jyotirmayūkhā dṛśyante yogibhir dhīramānasaiḥ ||43||\\
dṛṣṭyagre vāpy apāṅge vā taptakāñcanasaṃnibham | \\
bhūmiṃ saṃlakṣayed dṛṣṭiḥ sthirā bhavati yoginaḥ ||44||\\
athavā śirasaś cordhve dvādaśāṅgulasaṃmite |\\
jyotiḥpuñjaṃ nirākāraṃ lakṣayen muktidaṃ bhavet ||45||\\
yatra yatrārthavān yogī tatra tatra vilakṣayet |\\
ākāśam eva yas tasya cittaṃ bhavati tādṛśam ||46||\\
ity anekavidhākāraṃ bahirlakṣyam udīritam |\\
\end{quote}

Revise translation! see Powell 2023! 

``(41cd) That was the inner fixation. Now hear the external fixation that needs to be understood.(42) From the tip of the nose, counting with four, six, ten, and twelve, using the numerical system of the fingers. The five elements in [the colours of] outdoor blue, intense grey, wave of red and yellow mystery. (43) Alternatively, one may gaze steadily towards the space [directly] in front of [the face]. Luminous rays are perceived by steadfast-minded yogins. (44) In front of the gaze or at the outer corner of the eye space, resembling the shine of molten gold, the gaze should be fixed on the ground - [thus] stability arises for the yogin. (45) Alternatively, above the head, with a [distance of] twelve finger-breadths, one should fixate the formless cluster of light, which bestows liberation. (46) Wherever the yogin is suitable to the object, there he should fixate only space, in order for his mind to becomes as such. (47ab) Thus, various external fixations have been mentioned.''
\section{Netratantra}

Netratantra

Gavin Flood, Bjarne Wernicke-Olesen and Rajan Khatiwoda
Consultants: Alexis Sanderson, Diwakar Acharya

The Netratantra (NT), the ‘Tantra of the Eye’, is an important text in Kashmir and Nepal, dating from around the early ninth century, and widely disseminated during the eleventh and probably tenth centuries. The text takes its name from Śiva as Netranātha or ‘Lord of the Eye’. It was commented on by the Pratyabhijñā philosopher Kṣemarāja (c. 1000-1050) in his extant Netratantroddyota, that itself bears witness to its importance in his desire to bring the text into the orbit of his non-dualist metaphysics. The project will edit, translate, and describe its traditions as borne witness to in the Nepalese recension of the text. Alexis Sanderson has shown how the Netratantra was connected with royalty and used in the courts by Śaiva officiants in the role of royal priest or rājapurohita. That Śaiva and Mahāyāna gurus performed ‘apatropaic, restorative and aggressive Mantra rituals’ for the protection of king and kingdom is well attested in the kingdoms of south and south-east Asia from the ninth to eleventh century and the Netratantra is a text that bears witness to Śaiva gurus in the service of kings.[1] The principle use of the text would have been the protection of the king and his family through the propagation of its ritual procedures and particularly the recitation of the netra mantra (OṂ JUṂ SAḤ in the short version). Thus, the text is a ‘universal’ (sarvasāmānya-) tantra, which ‘overrides the distinctions between the various branches of the Mantramārga […] and that between the Mantramārga and the Kulamārga by propagating a form of worship for use by royal officiants that can be inflected as required to take on the character of any of these divisions and indeed of others outside Śaivism.’[2]

The text was first brought to our attention by Hélène Brunner who describes each chapter in some detail in her 1974 paper;[3] an extremely useful source for not only the contents of the text, but for her comments on its structure and relation to other texts, and has been researched by André Padoux in his studies of the correspondences between cosmos, sound, and body[3] and of the way the netramantra is formed. Somadeva Vasudeva has done research on yoga in the text, particularly the subtle visualization and subtle body of chapter seven,[5] as has James Mallinson.[6]

It is probable that the Netratantra was composed over a long period of time and the redactor is bringing together diverse elements into a whole. There are parallels between the Netra and the Svacchandatantra although more work on the parallels and influence of the Svacchanda needs to be done.[7] David White argues that the oldest or original section of the work is the material concerned with possession and exorcism[8] and this systematic treatment of possession is indeed a notable feature of it, akin to similar treatment in the Īśānaśivagurudevapaddhati Mantrapāda chapter 42.

The central deity of the Netratantra is Amṛteśvara, called Amṛtīśa in the Nepalese recension, also known as Amṛteśabhairava, Mṛtyunjit, and Mṛtyuñjaya, whose consort is Lakṣmī/Śrī called Amṛtalakṣmī in ritual manuals based on the text.[9] After an initial chapter in which Amṛteśvara, referred to as Bhairava, responds to the questions of the Goddess by extolling the virtues and powers of Śiva’s eye, the text presents a number of visualisations of a number of deities, catholic in its range, not only from the systems of the Mantramārga but from Vaiṣṇava traditions as well.[10] Furthermore, a strong Śākta influence is evident in the text with its many references to deities and practices characteristic of the Kulamārga (e.g. chapter 7 on the subtle visualising meditation and chapter 20 on yoginīs).

The project to study the text will especially focus on the theme of models of the person or self that the text entails. Based on close philological reading, we hope to account for different understandings of the person implicit in the text. Chapters on ritual and meditation reflect the understandings of the person in the wider community of which the text is an index. In particular, three chapters, six, seven, and eight, that the text calls the mundane or gross meditation (sthūladhyānam), the subtle meditation (sūkṣmadhyānam), and the supreme meditation (para­dhyānam), correspond to three types or levels of the body, gross, subtle and supreme.[11] It seems that this threefold hierarchical structure is an attempt to order a range of practices that the Netra is incorporating and it does so with some coherence. The lowest level of meditation practice is concerned with magical protection (primarily of the king [6.35] and his family) from demonic beings. This involves the practitioner, the Sādhaka or Mantrin, constructing diagrams within which the name of the person to be protected is written along with other rites of appeasement (śāntiḥ) and prosperity (puṣṭiḥ). The subtle level concerns the visualisation of the body and the powers moving within it. The subtle meditation is especially interesting because it presents two different systems of visualisation, one in which subtle energy rises up through the body, piercing the levels to the location of Śiva at the crown of the head and a second in which that same power rising through the body releases nectar at the crown of the head that then floods the body.[12] In his commentary Kṣemarāja calls these the tantra-prakrīyā and the kula-prakrīyā respectively, the latter being an index of the Śākta kulamārga. Finally, the supreme meditation is principally a reinterpretation of the ‘limbs’ of classical yoga from the perspective of supreme reality, the level of Śiva.[13] All of these entail distinct understandings of what a person is (e.g. a permeable self in ch. 6 and 19, a processual self in ch. 7 and a gnostic self in ch. 8).

There are two major recensions of the text, one in Kashmir (where four manuscripts exist to our knowledge) and one in Nepal where again there are four manuscripts (to be described presently). These have been preserved by the Nepal-German Manuscript Preservation Project (NGMCP). The Nepalese manuscripts probably represent an older recension of the text, a judgement based on its slightly less polished language, which the Kashmiris have amended at times in the interests of producing a better text although Sanderson argues for the Kashmir origin of the text between 700 and 850 AD.[14] Of the four Nepalese witnesses, the oldest is a palm leaf manuscript (N1) of which there is a much more recent (19th century?) devanāgarī apograph (N2). N1 is dated to February or March 1200, the copying being done by Pandit Kīrttidhara, commissioned by the author of a ritual manual Viśveśvara, and completed during Caitra in saṃvat 320 (= 1200 AD).[15] Often the Kashmir reading is better semantically and grammatically, but we intend to preserve the text as it stands while noting the Kashmir variants.

Project output:
A full annotated translation of the Netratantra with an introduction in two volumes in the Routledge Studies in Tantric Traditions series.

[1] Alexis Sanderson, ‘Religion and the State: Śaiva Officiants in the Territory of the King’s brahmanical Chaplain,’ p. 238, Indo-Iranian Journal vol. 47, 2004, pp. 229-300. This is corroborated by texts such as the Amṛteśadīkṣāvidhi that prescribe initiation and ritual for the royal family (p. 241).
[2] Alexis Sanderson, ‘The Śaiva Literature,’ p. 30, Journal of Indological Studies, Nos. 24 \& 25 (2012–2013), pp. 1-113.
[3] Hélène Brunner, ‘Un Tantra du Nord: le Netra Tantra’, Bulletin l’École Français d’Extreme Orient, vol. 61, 1974, pp. 125-97.
[4] André Padoux, Vāc: A Study of the Word in Selected Hindu Tantras, trans. J. Gontier (Albany: SUNY Press, 1991). Also, his useful and lucid paper ‘Corps et cosmos: l’image du corps du yogin tantrique,’ in V. Boullier and Gilles Tarabout (eds.), Images du corps dans le monde hindou (Paris: CNRS, 2002), pp. 163-87. See also Gavin Flood, ‘Body, Breath, and Representation in Śaiva Tantrism,’ in Axel Michaels and Christoph Wulf (eds.), Images of the Body in India (London: Routledge, 2011), pp. 70-83.
[5] Somadeva Vasudeva, ‘The Śaiva Yogas and their Relation to Other Systems of Yoga,’ pp. 7-8, RINDAS Series of Working Papers, Traditional Indian Thought 26, 2017, pp. 1-16.
[6] James Mallinson and Mark Singleton, The Roots of Yoga (London: Penguin, 2017), ch 5.
[7] André Padoux, Tantric Mantras (London: Routledge, 2011), pp. 90. 95.
[8] David White, ‘Netra Tantra at the Crossroads of the Demonological Cosmopolis,’ Journal of Hindu Studies, vol. 5, 2012, pp. 145-71.
[9] Sanderson, ‘Religion and the State,’ p. 239, n. 18.
[10] For example, it describes Viṣṇu as a sixteen-year old, ityphallic youth seated on a ram (13.10-13b), as well as visualisations of Tumburu and his sisters (chapter 11).
[11] Padoux (2002, p. 172) cites Kṣemarāja’s commentary on the Śivasūtra 3.4 where a triple body is related to the cosmic hierarchy.
[12] Bjarne Wenicke-Olesen has referred to the latter as being a ‘Śākta anthropology’ that can be contrasted with the earlier idea of the retention of semen (bindu) in the head. In an article with Silje Lyngar Einarsen he writes: ‘Es zeigt sich, daß eine ursprüngliche oder frühe Binduyoga-Anthropologie, die auf das Zurückhalten des Samens (bindhudhāraṇa) ausgerichtet war, von einem mit dem Kuṇḍalinī-System verknüpften Śākta-Anthropologie ersetzt wird, die auf die Überströmung des Körpers mit Unsterblichkeitselexir (amṛtaplavana) ausgerichtet ist’ (Wernicke-Olesen, B. and S. L. Einarsen. 2018. ’Übungswissen in Yoga, Tantra und Asketismus des frühen indischen Mittelalters’, in A.-B. Renger and A. Stellmacher (eds), Übungswissen in Religion und Philosophie: Produktion, Weitergabe, Wandel, pp. 241-257. Berlin: LIT Verlag). Also see James Mallinson, ‘Śāktism and Haṭha Yoga’ in B. Wernicke-Olesen (ed.), Goddess Traditions in Tantric Hinduism: History, Practice and Doctrine (London: Routledge, 2015), pp. 109-40.
[13] Vasudeva has written on the six ancillaries of yoga. Concerning those in the Netratantra he observes that ‘it may actually be more appropriate to compare the eight ancillaries of the Netratantra with the formulaic dhāraṇās taught in the Vijñānabhairava, which show an even greater tendency towards the transcendence of the inherited complex of ritual and yogic procedures’ (Vasudeva 2004, p. 382).
[14] Sanderson, ‘Religion and the State,’ p. 242.
[15] N1 folio 49. Amṛteśatantra, NAK MS 1-285, NGMPP Reel No. B 25/5. Palm Leaf; Nepalese variant of proto-Bengali script, 1200 AD (= Saṃvat 320). NAK 5-4866, NGMPP Reel No. A 171/12.

Link to chapter 7: Netratantra VII: Subtle Visualisation (sample chapter)
The Lord of Immortality: An Introduction, Critical Edition, and Translation of the Netra Tantra, chapter 7. Critically edited, translated and introduced by Gavin Flood, Bjarne Wernicke-Olesen, Rajan Khatiwoda (Oxford: OCHS 2019).
https://saktatraditions.org/netratantra/

\section{4.9.6 The Śivatattvaratnākara}
The Śivatattvaratnākara is a large compendium attributed to a king named Keḷadi Basavabhūpāla (also
known as Basavarāja, Basavāppa Nāyaka I) who reigned from 1696–1714 in Ikkeri, Karnataka. In the
seventh chapter of the Śivatattvaratnākara, in a section providing instructions on yoga for the king, a
large portion of the Śivayogapradīpikā is quoted. 338 The Śivatattvaratnākara also at times provides
further details or interpretations of the verses, for example, supplying the mantras referred to in
Śivayogapradīpikā 1.5. 339 The text thus provides an intriguing early modern example of the adapation of
yoga in a non-ascetic and courtly environment. page 146 in Powell 2023

\chapter{notes}
4.9.6 The Śivatattvaratnākara
The Śivatattvaratnākara is a large compendium attributed to a king named Keḷadi Basavabhūpāla (also
known as Basavarāja, Basavāppa Nāyaka I) who reigned from 1696–1714 in Ikkeri, Karnataka. In the
seventh chapter of the Śivatattvaratnākara, in a section providing instructions on yoga for the king, a
large portion of the Śivayogapradīpikā is quoted. 338 The Śivatattvaratnākara also at times provides
further details or interpretations of the verses, for example, supplying the mantras referred to in
Śivayogapradīpikā 1.5. 339 \textbf{The text thus provides an intriguing early modern example of the adapation of
yoga in a non-ascetic and courtly environment.}

Powell 2024:146

%%% Local Variables:
%%% mode: latex
%%% TeX-master: t
%%% End:
