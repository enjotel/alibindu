%Ultimatives Tool zur Datierung:
%https://www.cc.kyoto-su.ac.jp/~yanom/pancanga/
%skp = ignored in edition
%skm = ignored in xml
\input{preamble.tex}
\FormatDiv{1}{\begin{center}\Large}{\end{center}}
\FormatDiv{2}{\begin{center}\small}{\end{center}}
\FormatDiv{3}{\bfseries}{.}
\title{Tattvayogabindu of Rāmacandra\\ A Critical Edition and Annotated Translation\\ and a Comparative Analysis of the \\Complex Early Modern Yoga Yaxonomies }
\date{\today}
\parindent=15pt

\begin{document}

\frontmatter
\thispagestyle{empty} % Verhindert Seitenzahl auf der Seite
\begin{center}

%\vspace{0.5in}

%\begin{otherlanguage}{iast}
%   \large\sanskritfont{Tattvayogabindu}\\
%\end{otherlanguage}

\vspace{0.25in}


\huge\textbf{\MakeUppercase{The Tattvayogabindu \\of Rāmacandra}}\\

\vspace{0.2in}

\Large  Critical Edition and Annotated Translation of an Early Modern Text on Rājayoga, with a Comparative Analysis of the Complex Yoga Taxonomies from the Same Period\\ 

\vspace{0.45in}

\thispagestyle{empty}
\end{center}
%\newpage
%\thispagestyle{empty}
%\mbox{}
%\newpage

\newpage

  \thispagestyle{empty}
  \begin{figure}[p]
    \centering
    \includegraphics[width=0.25\textwidth]{pics/purna.jpg}
  \end{figure}
  
\newpage

\begin{landscape}
\thispagestyle{empty}
  \begin{figure}[p]
	\centering
  \includegraphics[width=1.5\textwidth]{pics/folio1.jpg}
	\caption{Folio 1v of Ms. \getsiglum{N1}.}
	 \phantomsection\label{fig_folio1}
\end{figure}
\end{landscape}

\cleardoublepage
\tableofcontents
\thispagestyle{empty}
\newpage 
\listoffigures
\thispagestyle{empty}
\newpage
\listoftables
\thispagestyle{empty}
\newpage

\mainmatter
\pagestyle{defaultstyle}
\counterwithout{footnote}{chapter}
\counterwithout{figure}{chapter}
\counterwithout{table}{chapter}
\renewcommand{\thetable}{\arabic{table}}
%%%tables 
\setsecnumdepth{section}
\maxsecnumdepth{subsubsection}
\newpage
\chapter{Introduction}
\cleardoublepage

\section{General remarks}
 \phantomsection\label{generalremarks}
 \lettrine{T}{he} \textit{Tattvayogabindu} of Rāmacandra\footnote{A discussion about the author Rāmacandra is found on p. \pageref{ramarama}.} is an early modern Sanskrit text on Rājayoga that was written in the first half of the seventeenth century\footnote{The dating of the text is discussed on p. \pageref{dating}.} in northern India.\footnote{The detailed discussion of the place of origin is found on p. \pageref{riversrivers}, n. \ref{riversrivers}.} The most salient feature of the work that makes it historically significant is its highly differentiated taxonomy of types of yoga.\footnote{This is a remarkable increase in the number of declared yogas compared to the standard medieval tetrad of Mantra, Laya, Haṭha and Rājayoga.} In the \textit{Tattvayogabindu}'s introduction, most manuscripts name fifteen types of yoga, presented as methods of Rājayoga. These are 1. Kriyāyoga, 2. Jñānayoga, 3. Caryāyoga, 4. Haṭhayoga, 5. Karmayoga, 6. Layayoga, 7. Dhyānayoga, 8. Mantrayoga, 9. Lakṣyayoga, 10. Vāsanāyoga, 11. Śivayoga, 12. Brahmayoga, 13. Advaitayoga, 14. Siddhayoga, and 15. Rājayoga itself. The text is a yogic compendium written in a mix of mainly prose and 47 verses in textbook-style, where its 59 topics are introduced in sections most of the time launched by recognizable phrases. The sections deal with the methods of Rājayoga and their effects, but others also cover topics like yogic physiology, the Avadhūta, the importance of the guru, cosmogony, and a \textit{yogaśāstrarahasya}.  

The \textit{Tattvayogabindu} has not been discussed comprehensively or considered in the secondary literature on yoga. The only exception is \citeauthor{birch2014} (2014: 415–416) who briefly described its list of fifteen yogas in the context of the ``fifteen medieval yogas'' and noted that a similar taxonomy occurs in Nārāyaṇatīrtha’s \textit{Yogasiddhāntacandrikā} (17th century), a commentary on the \textit{Pātañjalayogaśāstra} that integrates fifteen medieval yogas within its \textit{aṣṭāṅga} format. An incomplete account of the fifteen yogas is found within the Sanskrit yoga text \textit{Yogasvarodaya}, which is known only through quotations in the \textit{Prāṇatoṣinī}, the \textit{Yogakarṇikā} and the \emph{Śabdakalpadruma}.\footnote{Manuscripts under the name of \textit{Yogasvarodaya} seem to be lost. I was not able to locate the manuscripts of the text in any manuscript catalogue at hand.} The \textit{Yogasvarodaya} announces a total of fifteen yogas but names only eight of them in its introductory \textit{śloka}s. It is the primary source and template for the compilation of the \textit{Tattvayogabindu}. Besides several passages, Rāmacandra, in many instances, follows its content and structure by rewriting the \textit{Yogasvarodaya}’s \textit{śloka}s into prose or quoting them directly without attribution. Due to the incomplete transmission of the \textit{Yogasvarodaya}, Rāmacandra’s \textit{Tattvayogabindu} is a natural and valuable starting point for an unprecedented in-depth study of the complex early modern yoga taxonomies, a phenomenon that can be narrowed down precisely in terms of time and as I will show regarding its localisation. The other source text that Rāmacandra used is the \textit{Siddhasiddhāntapaddhati} whose content he draws on, particularly in the second half of his composition. Another text that includes an almost similar taxonomy of twelve yogas divided into three tetrads\footnote{See p.\pageref{sarvasarva} for a detailed discussion of the \textit{Sarvāṅgayogapradīpikā}.} is Sundardās’s \textit{Brajbhāṣā} yoga text named \textit{Sarvāṅgayogapradīpikā} which not just shares most of the types of yogas but also provides a different and valuable perspective on the addressed yoga categories.\footnote{For a comparative table of the complex early modern yoga taxonomies see table \ref{tab:complextaxonomies} on p. \pageref{tab:complextaxonomies}.}

These complex taxonomies that emerged during the 17th century crossed sectarian divides and were adapted to the specific needs of different authors and traditions. The \textit{Tattvayogabindu} thus encapsulates a large proportion of the diversity of yoga types and teachings after the \textit{Haṭhapradīpikā} (15th century) that were adopted and practised by a broad spectrum of religious traditions and strata of Indian society. In the particular case of the \textit{Tattvayogabindu}, there are various statements throughout the text that reveal a strategy to detach yoga from its ascetic and renunciate connotations and to stylise Rājayoga as a practice that can bring the desired soteriological benefits even to practitioners who enjoy worldly pleasures and expensive lifestyles. Textual evidence suggests that the \textit{Tattvayogabindu} is an important example of a text that provides an early modern adaptation of Rājayoga for \textit{kṣatriya}s in a courtly environment.

One printed edition of the \textit{Tattvayogabindu} was published in 1905 with a Hindi translation and based on (an) unknown manuscript(s).\footnote{\emph{Binduyoga}. \textit{Binduyogaḥ with Bhāṣaṭīkā}. Ed. by Jvālāprasāda Miśra. Mumbai, 1905.} This publication has the title ``\textit{Binduyoga}'' confirmed by the printed text’s colophon. However, as I will discuss in the introduction, the text was originally known as \textit{Tattvayogabindu}. The consulted manuscripts contain significant discrepancies, structural differences and variant readings between them and the printed edition.\footnote{For example, the printed edition does not contain the complex yoga taxonomy presented in the manuscripts of the \emph{Tattvayogabindu}.} Furthermore, the manuscripts are scattered over the northern half of the Indian subcontinent and Nepal, which suggests that the text was widely transmitted at some point. Lengthy passages of the \textit{Tattvayogabindu} are quoted without attribution in a text called \textit{Yogasaṃgraha} and Sundaradeva’s \textit{Haṭhasaṅketacandrikā}.

The first chapter of this dissertation contains a general introduction to Rāmacandra's \textit{Tattvayogabindu}. The chapter gives a brief overview of the content of the text and discusses its origin, the author and the author's intended audience. Subsequently, the textual witnesses, source texts and testimonies of the \textit{Tattvayogabindu} are described. A stemmatic analysis of the text is then presented, based on manual philological observation and computer-assisted stemmatics to present a \textit{stemma codicum}. The chapter concludes with a presentation of the editorial policies, which form the basis for the second chapter of this thesis.
The second chapter, the core of this dissertation, is a critical edition and annotated translation of the \textit{Tattvayogabindu}. The critical edition significantly improves the text and sheds new light on its historical significance.
The third chapter contains a comparative analysis of the complex early modern yoga taxonomies based on hermeneutics of difference.\footnote{The conceptof hermeneutics of difference is discussed on p. \pageref{hermeneutics}, n. \ref{hemerneutics}.}  Using the new critical edition of the \textit{Tattvayogabindu} and the texts mentioned above, \emph{Yogasvarodaya}, \emph{Yogasiddhāntacandrikā} and \emph{Sarvāṅgayogapradīpikā}, the complex yogic taxonomies of the four texts are compared in detail. Based on this comparative analysis, a differentiated hypothesis on the emergence of the complex yoga taxonomies was developed, and the complex yoga taxonomies were located und explained in the broader context of the historical development of the yoga traditions. The comparison includes a nuanced description of each yoga category used by the authors of the texts with complex yoga taxonomies. While the authors of the four texts often operate with identical terms for the individual yoga categories, they interpret these categories according to their religious backgrounds and agendas, with intriguing and exciting differences. Contrasting the comparanda, i.e. the authors, the texts, the yoga taxonomies and the yoga categories, therefore provides a deep insight into the discursive negotiation processes of the Indian yoga traditions of the 17th century.


\chapter{Conventions in the Critical Apparatus}
\section{Sigla in the Critical Apparatus}

\begin{itemize}
\item \beta : \getsiglum{D}, \getsiglum{J}, \getsiglum{K1}, \getsiglum{N1}, \getsiglum{N2}, \getsiglum{U1}
\item \gamma : \getsiglum{B}, \getsiglum{E}, \getsiglum{L}, \getsiglum{P}, \getsiglum{U2}
\item B : Bodleian Oxford D 4587
\item C : \emph{Haṭhasaṅketacandrikā} GOML Ms. No. R 3239
\item C\textsubscript{pc} : \emph{Haṭhasaṅketacandrikā} GOML Ms. No. R 3239
\item cett.: ceteri (all manuscripts except the ones mentioned in the lemma)
\item \Done : IGNCA 30019
\item E : Printed Edition
\item J : JNUL Ms. No. 55769
\item Jo : \emph{Haṭhasaṅketacandrikā} MMPP MS. No. 2244
\item \Kone : AS G 11019
\item L : Lalchand Research Library LRL5876
\item M : \emph{Haṭhasaṅketacandrikā} ORI Ms. No. B 220
\item \Ntwo : NGMPP B 38-35 / A 1327-14
\item \None : NGMPP B 38-31
\item P : Pune BORI 664
\item PT : \emph{Prāṇatoṣiṇī}
\item \Uone : SORI 1574
\item \Utwo : SORI 6082
\item V : OI MSU 10558
\item YK : \emph{Yogakarṇikā}% 
\item YSv : \emph{Yogasvarodaya}
\end{itemize}
\newpage

\chapter[Critical Edition \& Annotated Translation of the \emph{Tattvayogabindu}]{The \emph{Tattvayogabindu} of Rāmacandra \\ \huge  
  Critical Edition \& Annotated Translation}
\pagestyle{chapter2style}
\newpage
\begin{alignment}[
  texts=edition[class="edition"];
  translation[class="translation"],
  ]
  \begin{edition}
    \begin{prose}[p12_03]
      \noindent
 \note[type=source, labelb=_34b, labele=_34e, nosep]{cf. YSv (PT, p. 833): anantaparamānandasthānaṃ jñeyaṃ tadūrddhvataḥ (\textit{tadarddhataḥ} YK 1.278) | ūrddhvagatakalā tatra tasya dhyānād bhaved iti | iti siddhirājayogaṃ strīṇāṃ bhogaṃ mahāsukham | gītavādyavinodādi saśivaṃ varddhate kṣitau | dhyānaṃ nirantarañ cāsya puṇyapāpe sthire (\textit{sthirau } YK 1.280) na hi | nijarūpasya dṛṣṭiḥ syād dūrasyārthañ ca paśyati |}
 \note[type=source, labelb=_34b, labele=_34e, nosep]{cf. SSP 2.9 (Ed p. 32): navamam ākāśacakraṃ ṣodaśadalakamalam ūrdhvamukham | tanmadhye karṇikāyāṃ trikūṭākārāṃ tadūrdhvaśaktiṃ tāṃ paramaśūnyāṃ dhyāyet | tatraiva pūrṇagiripīṭhaṃ sarveṣṭasiddhipradaṃ bhavati |}
  \note[type=analogia, labelb=_34b, labele=_34e, nosep]{cf. \citetitle{saubhagya}: navamamākāśacakram | tatra ṣoḍaśadalapadmam ūrdhvamukhaṃ tanmadhyakarṇikātrikūṭākāram | tanmadhye ūrdhvaśaktiḥ | tāṃ paśyan dhyāyet | tatraiva pūrṇagiripīṭhaṃ sarvecchāsiddhisādhanaṃ bhavati |}
%---------------------
%tadupari anaṃtaparamānandasya sthānam / \E
%tadupari anaṃtaparamānandasya sthānaṃ   \P
%tadupari anantaparamānaṃdasya sthānam / \N1
%tadupari anantaparamānaṃdasya sthānam / \N2
%tadupari anantaparamānaṃdasya sthānaṃ / \D 
%tadupari anantaparamānaṃdasya sthānam vartate/ \B
%tadupari anaṃtaparamānaṃdasya sthānam vartate/ \L
%tadupari alakṣaparamānaṃdasya sthānam   \U1
%tadupari alakṣaparamānaṃdasya sthānaṃ // \J
%tadupari anaṃtaparamānaṃdasya sthānaṃ / \K1  
%tadupari anaṃtaparamānaṃdasya sthānaṃ// U2
%---------------------
%Above that is the place of infinite supreme bliss.
%---------------------
tadupari
\app{\lem[wit={ceteri}, alt={ananta°}]{ananta}\phantomsection\label{cakra9kala}\linelabel{_34b}
  \rdg[wit={J,U1}]{alakṣa°}}paramānandasya
\app{\lem[wit={ceteri}]{sthānam}
  \rdg[wit={D,U2}]{stānaṃ}
  \rdg[wit={B,L}]{sthānam vartate}}/      
%---------------------
%tatrordhvaśaktiḥ / \E
%tatordhvaśaktiḥ \P
%rdhaśakti ardhaśakti \B
%rdhaśakti ardhaśakti \L
%tatrordhvaśaktiḥ / \N1
%tatra ūrdhva śaktiḥ / \D
%tatra ūrdhva śaktiḥ / \K1
%tata ūrdhva śaktir \J
%tatra ūrdhva śakti / \N2
%urdhvaśaktir         \U1
%tatrordhvaśaktiḥ// \U2
%---------------------
%There above is \textit{śakti},
%------------------------------
\app{\lem[wit={E,K1,N1,U2}]{tatrordhvaśaktiḥ}
  \rdg[wit={P}]{tatordhvaśaktiḥ}
  \rdg[wit={U1}]{urdhvaśaktir}
  \rdg[wit={D}]{tatra ūrdhva śaktiḥ}
  \rdg[wit={J}]{tata ūrdhva śaktir}
  \rdg[wit={N2}]{tatra ūrdhva śakti}
  \rdg[wit={B,L}]{rdhaśakti ardhaśakti}}/
%------------------------------
%etādṛśī  saṃjñā   ekā kalā vartate / \E
%ekādaśā  saṃjñā   ekā kalā vartate   \P
%etādṛśī  saṃjñā   ekā kalā vartate /  \N1
%etādṛśī  saṃjñā   ekā kalā varttate / \N2
%etādṛsaṃ saṃjñā   ekā kalā vartate / \D
%etādṛsa  saṃjñā   ekā kalā vartatte / \K1
%etādṛsaṃ saṃjñākā//   ekā kalā vartate / \J
%ekādaśā  saṃjñā   ekā kalā vartate / \B
%ekādaśā  saṃjñā   ekā kalā vartate / \L
%etādṛśī  saṃjñakā ekā kalā vartate /  \U1
%etādṛśā  saṃjñā   ekā kalā vartate/ \U2 
%---------------------
%Being designated as such she is one single digit. 
%------------------------------
\app{\lem[wit={ceteri}]{etādṛśī}
  \rdg[wit={U2}]{etādṛśā}
  \rdg[wit={J}]{etādṛsa}
  \rdg[wit={D}]{etādṛsaṃ}
  \rdg[wit={B,L,P}]{ekādaśā}}\app{\lem[wit={ceteri}]{saṃjñā}
  \rdg[wit={J,U1}]{saṃjñakā}}
ekā kalā vartate/
%------------------------------
%asyāḥ  kalāyā   dhyānakāraṇāt     puruṣo yad icchati / \E
%asyāḥ  kalāyā   dhyānakāraṇāt     puruṣo yad icchati ?Zeichen? \P
%asyāḥ  kalāyā   dhyānakāraṇāt     puruṣo yad icchati  tad bhavati \N1
%tasyāḥ kalāyāḥ  dhyānakāraṇāt     puruṣo yad icchati  tad bhavati \N2
%asyāḥ  kalāyā   dhyānakāraṇā      puruṣo yad icchati  tad bhavati \D
%asyāḥ  kalāyā   dhyānakāraṇāt//   puruṣo yad icchati  tad bhavati// cha //  \J
%tasyāḥ  kalāyāḥ   dhyānakāraṇāt   puruṣo yad icchati  tad bhavati// \K1
%asyāḥ  kalāyā   dhyānakāraṇāt /   puruṣo yad icchati / \B
%asyāḥ  kalāyā   dhyānakāraṇāt /   puruṣo yad icchati / \L
%asyā   kalāyā   dhyānakāraṇāt     puruṣo yad icchati tad bhavati vā \U1
%asyāḥ  kalāyāḥ  dhyānakāraṇāt //  puruṣo yad icchati // \U2
%---------------------
%Due to the exercise of meditation on this part the person manifests whatever he wishes for.
%------------------------------
\app{\lem[wit={ceteri}]{asyāḥ}
  \rdg[wit={U1}]{asyā}
  \rdg[wit={K1,N2}]{tasyāḥ}}
\app{\lem[wit={ceteri}]{kalāyā}
  \rdg[wit={K1,N2,U2}]{kalāyāḥ}}
\app{\lem[wit={ceteri}]{dhyānakāraṇāt}
  \rdg[wit={D}]{dhyānakāraṇā}}
puruṣo yad-icchati
\app{\lem[wit={D,K1,N1,N2}, alt={tad bhavati}]{tad\skp{-}bhavati}
  \rdg[wit={J}]{tad bhavati ||cha||}
  \rdg[wit={U1}]{tad bhavati vā}
  \rdg[wit={Y}]{\om}}/ 
%------------------------------
%tasya sukhabhogavataḥ / \E
%tasya sukhabhogavataḥ \P
%rājya-sukhabhogavataḥ \N1
%rājya-sukhabhogavataḥ \N2
%rājya-sukhabhogavṛtaḥ \D !!!
%rājya-sukhabhogavṛtaḥ \K1
%rājya-sukhabhogavataḥ \J (P7/34)
%tasya-khaṃ bhogavataṃ / \B
%tasya-sukhaṃ bhogavaṃtaṃ / \L
%rājya-sukhabhogavataḥ \U1
%tasya-sukhabhogavataḥ / \U2
%---------------------
%He is furnished with royal pleasure and enjoyment. 
%------------------------------
\note[type=testium, labelb=_35b, labele=_35e, nosep]{ \approx  \textit{Yogasaṃgraha} (IGNCA 30020 f. 4r. ll. 1-4): rājyasukhabhogavataḥ strī vilāsavataḥ saṃgītavinoda prekṣāvato pi sādhakasya śuklapakṣacaṃdravat pratidinaṃ tejaso vapuṣaś ca vṛddiḥ puṇyapāpasya śārbhāvaḥ nijasva rūpaprakāśasāmarthaṃ dūrasthapy arthasya samīpastham iva darśanaṃ ca bhavati | cha | tad uktaṃ tattvajñānapradīpikāyāṃ ||}
\app{\lem[wit={J,N1,N2,U1}, alt={rājyasukhabhogavataḥ}]{rājyasukhabho:\\gavataḥ}\linelabel{_35b}
  \rdg[wit={D,K1}]{rājyasukhabhogavṛtaḥ}
  \rdg[wit={E,P,U2}]{tasya sukhabhogavataḥ}
  \rdg[wit={B}]{tasya khaṃ bhogavataṃ}
  \rdg[wit={L}]{tasya sukhaṃ bhogavaṃtaṃ}}
%------------------------------
%strīmadhye     vilāsavataḥ    saṃgītavilāsavataḥ vinodaprekṣāvataḥ        puruṣasya pratidinaṃ śuklapakṣe candrakalāvat   kalā     vardhate/   \E
%strīmadhye     vilāsavataḥ    saṃgītavinodaprekṣāvataḥ              eva   puruṣasya pratidinaṃ śuklapakṣe candrakalāvat   kalā     vardhate /  \P
%strīmadhye     vilāsavaṃtaṃ   saṃgītaṃ prekṣāvatāḥ //               evaṃ  puruṣasya pratidinaṃ śuklapakṣe caṃdrakalāvat / kalā     vartate /   \L
%strīmadhye     vilāsavaṃtaṃ   saṃgītaṃ vinodavaṃtaṃ prekṣāvaṃtāḥ // eva   puruṣasya pratidinaṃ śuklapakṣe caṃdrakalāvat / kalā     vartate /   \B
%strīmadhye     vilāsavataḥ    saṃgītavinodaprekṣyāvataḥ             evaṃ  puruṣasya pratidinaṃ śuklapakṣe candrakalā vṛddhivato?   vardhate / \N1
%śrī strīmadhye vilāsavataḥ    saṃgītavinodaprekṣāvataḥ              evaṃ  puruṣasya pratidinaṃ śuklapakṣa candrakalā vṛddhi vaṃto  varttate /  \N2
%strīmadhye     vilāsavataḥ // saṃgītavinodaprekṣyāvataḥ //          evaṃ  puruṣasya pratidinaṃ śuklapakṣe candrakalā vṛddhivato    vardhate / \D
%strīmadhye     vilāsavataḥ    saṃgītavinodaprekṣyāvataḥ //          evaṃ  puruṣasya pratidinaṃ śuklapakṣe candrakalā vṛddhivata    vardhate / \K1
%strīmadhye     vilāsavataḥ // saṃgītavinodaprekṣyāvataḥ //          evaṃ  puruṣasya pratidinaṃ śuklapakṣe candrakalā vṛddhivato    vartate // \J
%strīmadhye     vilāśavataḥ    saṃgītavinodaprekṣyāvataḥ             eka   puruṣasya pratidinaṃ śuklapakṣe caṃdrakalā vṛddhir       varddhate / \U1
%strīmadhye     vilāsavata     saṃgītavinodaprekṣāvata//             evaṃ  puruṣasya pratidinaṃ śuklapakṣe candrakalāvat   kalā     varttate/   \U2
%---------------------
%(Selbst) bei einem Menschen, der sich inmitten von Frauen vergnügt, (und) ein Musikvergnügen
%ansieht, wächst täglich die Kraft (kalā = śakti?) wie die "kalā" (Phase) des Mondes in der hellen Monatshälfte.
%The \textit{kalā} of a person grows daily, like the \textit{kalā} of the moon in the bright half of the month, even amusing oneself amongst women and watching a musical pleasure.
%(Even) amusing oneself amongst women, and watching musical pleasures, the \textit{kāla} of the person grows daily like the \textit{kalā} of the moon in the bright half of the month. 
%------------------------------
\app{\lem[wit={ceteri}]{strīmadhye}
  \rdg[wit={N2}]{śrī strīmadhye}}
\app{\lem[wit={ceteri}]{vilāsavataḥ}
  \rdg[wit={U2}]{vilāsavata°}
  \rdg[wit={L,B}]{vilāsavaṃtaṃ}} 
saṃgīta\app{\lem[wit={D,J,K1,N1,U1},alt={°vinodaprekṣyāvataḥ}]{vinodaprekṣyāvataḥ}
  \rdg[wit={P,N2}]{°vinodaprekṣāvataḥ}
  \rdg[wit={U2}]{°vinodaprekṣāvata}
  \rdg[wit={B}]{°ṃ vinodavaṃtaṃ prekṣāvaṃtāḥ}
  \rdg[wit={E}]{°vilāsavataḥ vinodaprekṣāvataḥ}
  \rdg[wit={L}]{°ṃ prekṣāvatāḥ}}
 \app{\lem[wit={B,P}]{eva}
  \rdg[wit={ceteri}]{evaṃ}
  \rdg[wit={U1}]{eka}}
puruṣasya pratidinaṃ śuklapakṣe
candrakalā\app{\lem[wit={Y},alt={°vat kalā}]{vat kalā}
  \rdg[wit={D,J,N1}]{vṛddhivato}
  \rdg[wit={K1}]{vṛddhivata}
  \rdg[wit={N2}]{vṛddhi vaṃto}
  \rdg[wit={U1}]{vṛddhir}}
\app{\lem[wit={D,E,P,N1,U1}]{vardhate}
  \rdg[wit={ceteri}]{vartate}}/
%------------------------------
%puṇyapāpe  'sya śarīraṃ   na spṛśataḥ /    \E
%\om                                     \P
%puṇyapāpe  asya śarīrena     spṛśataḥ /      \N1
%puṇyapāpe  asya śarīrena     spṛśataḥ /      \N2
%puṇyapāpe  asya śarīrena     spṛśataḥ /      \D
%puṇyapāpe  asya śarīrena     spṛśataḥ       \K1
%puṇyapāpe  asya śarīrena     spṛśataḥ //      \J
%puṇyapāpe  asya śarīrasya na spṛśataḥ // \B
%puṇyapāpe  asya śarīrasya na spṛśataḥ // \L
%puṇyapāpau asya śarīrena     spṛśāt         \U1
%puṇyapāpe  asya śarīraṃ   na spṛśataḥ // \U2
%---------------------
%puṇyapāpe\varc{puṇyapāpe \edprint \lalchand \oxford \nepal \dehlia}{\om \pune} 'sya\varc{'sya \edprint}{asya \nepal \dehlia \oxford \lalchand \om \pune} śarīrasya\varc{śarīrasya \lalchand \oxford}{śarīraṃ \edprint śarīrena \nepal \dehlia \om \pune} na\varc{na \edprint \oxford \lalchand}{\om \nepal \dehlia \pune} spṛśataḥ\varc{spṛśataḥ \edprint \lalchand \oxford \nepal \dehlia}{\om \pune} /
%---------------------
%His body is not affected by merit and sin. 
%------------------------------
\app{\lem[wit={ceteri}]{puṇyapāpe}
  \rdg[wit={U1}]{puṇyapāpau}
\rdg[wit={P}]{\om}}\app{\lem[wit={E}]{'sya}
  \rdg[wit={P}]{\om}
  \rdg[wit={ceteri}]{asya}}  
\app{\lem[wit={B,L},alt={śarīrasya}]{śarīrasya}
  \rdg[wit={X}]{śarīrena}
  \rdg[wit={E,U2}]{śarīraṃ}
  \rdg[wit={P}]{\om}}
\app{\lem[wit={B,E,L,U2}]{na}
  \rdg[wit={X,P}]{\om}}
\app{\lem[wit={ceteri},alt={°śataḥ}]{spṛśataḥ}
  \rdg[wit={U1}]{spṛśāt}}/
%------------------------------
%                          nirantaradhyānakaraṇāt     nijasvarūpaṃ prakāśanasāmarthyaṃ bhavati / \E
%                          \om until .....            nijasvarūpaprakāśasāmarthyaṃ     bhavati / \P
%                          niraṃtaraṃ dhyānakaraṇāt   nijasvarūpaprakāśasāmarthyaṃ     bhavati / \B
%                          niraṃtaraṃ dhyānakaraṇāt// nijasvarūpaprakāśasāmarthyaṃ     bhavati / \L
%                          nirantaradhyānakaraṇāt /   nijasvarūpaprakāśasāmarthyaṃ     bhavati / \N1 <-----
%                          niraṃtaradhyānakaraṇāt /   nijasvarūpaprakāśasāmarthyaṃ     bhavati // \N2
%                          nirantaradhyānakaraṇāt /   nijasvarūpaprakāśasāmarthyaṃ     bhavati / \D
%                          nirantaradhyānakaraṇāt /   nijasvarūpaprakāśasāmarthyaṃ     bhavati // \J
%                          nirantaradhyānakaraṇāt //  nijasvarūpaprakāśasāmarthyaṃ     bhavati // \K1
%                          nirantaradhyānakaraṇāt /   nijasvarūpaprakāśasāmarthyaṃ     bhavati    \U1
%evaṃ puruṣasya pratidinaṃ niraṃtaraṃ dhyānakaraṇāt   nijasvarūpaṃ prakāśanasāmarthyaṃ bhavati// \U2 
%---------------------
%Due to uninterrupted meditation the power of the light of the innate nature arises. 
%------------------------------
\app{\lem[type=emendation, resp=egoscr, alt={nirantaradhyānakaraṇān}]{nirantaradhyānaka:\\raṇā\skp{n-ni}}
  \rdg[wit={E,X}]{nirantaradhyānakaraṇāt}
  \rdg[wit={B,L}]{niraṃtaraṃ dhyānakaraṇāt}
  \rdg[wit={U2}]{evaṃ puruṣasya pratidinaṃ niraṃtaraṃ dhyānakaraṇāt}
  \rdg[wit={P}]{\om}
}\skm{n-ni}jasvarūpa\app{\lem[wit={ceteri},alt={°prakāśa°}]{prakāśa}
  \rdg[wit={E,U2}]{°ṃ prakāśana°}
}sāmarthyaṃ bhavati/
%------------------------------
%dūrasthopi ca dūrasthavastu                   samīpa iva   paśyati // \E
%dūrasthamapi                                  samīpam iva  paśyati // \N1
%dūrasthamapi                                  samīpaṃ iva  paśyati // \N2
%dūrasthamapy-arthaṃ                           samīpa iva   paśyati // \D
%dūrasthamapy-arthaṃ                           samīpa iva   paśyati // \K1
%dūrasthamapy arthaṃ                           samīpam eva   paśyati // \J
%dūrasthamapi padārthaṃ                        samīpa iva   paśyati // \B
%dūrasthamapi parārthaṃ                        samīpa iva   paśyati // \L
%dūrasthamapi padārthaṃ                        samīpa iva   paśyati // \P
%dūrasthamapy-arthaṃ                           samīpam eva  paśyati // \U1
%dūrasthamapi bhavati //dūrasthamapi padārthaṃ samīpa iva   paśyati // \U2
%------------------------------
%dūrasthamapyarthaṃ\varc{dūrasthamapyarthaṃ \dehlia}{dūrasthamapi padārthaṃ \oxford \pune durasthamapi parārthaṃ \lalchand sūrastamapi \nepal ca dūrasthavastu \edprint} samīpa\varc{samīpa \dehlia \edprint \lalchand \oxford \pune}{samīpam \nepal} iva paśyati //
%------------------------------
%He sees remotely located objects as if they'd be near.
%------------------------------
dūra\app{\lem[wit={D,J,K1,U1},alt={°stham apy arthaṃ}]{stham-apy-arthaṃ}
  \rdg[wit={B,P}]{°stham api padārthaṃ}
  \rdg[wit={L}]{°stham api parārthaṃ}
  \rdg[wit={E}]{°sthopi ca dūrasthavastu}
  \rdg[wit={N1,N2}]{°stham api}
  \rdg[wit={U2}]{°stham api bhavati || dūrastham api padārthaṃ}}
\app{\lem[wit={ceteri}]{samīpa}
  \rdg[wit={J,N1,U1}]{samīpam}
  \rdg[wit={N2}]{samīpaṃ}}
\app{\lem[wit={ceteri}]{iva}
  \rdg[wit={J,U1}]{eva}} 
paśyati\dd{}\linelabel{_34e}\linelabel{_35e}
\phantomsection\label{cakra9end}
    \end{prose}
  \end{edition}
  \begin{translation}
    \begin{tlate}[p12_03]
      \noindent
      Above that is the place of infinite supreme bliss. In that [place], there exists the upper power (\textit{ūrdhvaśakti})\footnote{Examining the primary source for the composition of Rāmacandra's account of the ninth \textit{cakra}, found in the \emph{Yogasvarodaya} (cf. sources), one might infer that the term \textit{ūrdhvaśakti} is a corruption of the term \textit{tadūrddhvataḥ}. However, the term and concept are validated by its occurrence in \emph{Siddhasiddhāntapaddhati} 2.9 (cf. sources), which Rāmacandra must have consulted as well. The association of \textit{ūrdhvaśakti} with a \textit{kalā} is likely an innovation by Rāmacandra, as the \textit{ūrdhvaśakti} of the \emph{Siddhasiddhāntapaddhati}  is not explicitly identified as a \textit{kalā}. Moreover, the \textit{kalā} mentioned in the YSv is not designated as a \textit{śakti}. Rāmacandra, therefore, amalgamates assertions from both sources in his interpretation.} as a unique digit which has such a designation. As a result of meditation on this digit, whatever the person wants arises. Even though [one is] enjoying royal pleasures, amusing oneself amongst women and watching musical performances, the person's digit grows daily like the digit of the moon in the bright half of the month. His body is not affected by merit and sin. As a result of uninterrupted meditation [onto this digit], the ability to illuminate one's own nature arises. He sees even remote objects as if they were near.\footnote{The final testimony from \textit{Yogasaṃgraha} IGNCA 30020 references its source as \textit{Tattvajñānapradīpikā} (cf. testimonia). After that, the manuscript paraphrases and quotes other yoga texts such as the \emph{Uttaragītā}.}
\flushpage 
      \end{tlate}
    \end{translation}
  \end{alignment}
  \pagebreak % after pp. 39-40
\cleardoublepage
\selectlanguage{english}
\chapter{Appendix}
\section{Figures}
 
% \begin{landscape}
\clearpage

  \begin{figure}[ht]
	\centering
  \includegraphics[width=1\textwidth]{pics/Wolpertinger.png}
\caption[The \textit{dehasvarūpa} of \textit{ajapāgāyatrī}]{The \textit{dehasvarūpa} of \textit{ajapāgāyatrī}. The image, reminiscent of a hippogriff, is part of an illustrated Sanskrit manuscript written in the Śāradā script. Preserved as a single large scroll under Acc. No. 1334 at the Oriental Institute in Srinagar (Kashmir), it is entitled \textit{Nāḍīcakra}. The manuscript contains a depiction of the yogic body’s \textit{cakra}s and \textit{nāḍī}s. The text surrounding the figure closely corresponds to the additional material found in manuscript \getsiglum{U2} of the \textit{Tattvayogabindu}. The manuscript reads (diplomatic transcription): \textit{oṃ daśame pūrṇagiripīṭhe lalāṭamaṇḍale candro devatā amṛtāśaktiḥ paramātmā ṛṣiḥ dvāviṃśaddalāni amṛtavāsinikalā 4: ambikā 1 lambikā 2 gha(ṃ)ṭkā 3 tālikā 4 dehasvarūpaṃ kākamukhaṃ 1 naranetraṃ 2 gośṛṅgaṃ 3 lalāṭabrahmapara 4 hayagrīvā 5 mayūramuśchaṃ 6 haṃsacārītani 7 sthāna.}}
	\phantomsection\label{fig_wolpertinger}
      \end{figure}

      \clearpage

  \begin{figure}[ht]
	\centering
  \includegraphics[width=1\textwidth]{pics/Vishnu_Vishvarupa_cropped.jpg}
	\caption{Viṣṇu Viśvarūpa, India, Rajasthan, Jaipur, ca. 1800–1820, Opaque watercolor and gold on paper, 38.5 × 28 cm, Victoria and Albert Museum, London, Given by Mrs. Gerald Clark.}
	\label{fig1}
      \end{figure}
\clearpage
  \begin{figure}[ht]
	\centering
  \includegraphics[width=0.5\textwidth]{pics/The_Equivalence_of_Self_and_Universe_(detail),_folio_6_from_the_Siddha_Siddhanta_Paddhati,_(Bulaki),_1824_(Samvat_1881);_122_x_46_cm._Mehrangarh_Museum_Trust..jpg}
	\caption{The Equivalence of Self and Universe (detail), folio 6 from the \textit{Siddhasiddhāntapaddhati} (Bulaki), India, Rajasthan, Jodhpur, 1824 (Samvat 1881), 122 x 46 cm, RJS 2378, Mehragarh Museum Trust.}
	\label{fig2}
      \end{figure}
      % \end{landscape}

      \newpage
      \cleardoublepage
\chapter{Bibliography}
 \label{sec:bibli}
\clearpage
\newpage 
\thispagestyle{empty}
\quad  \addtocounter{page}{-1}

\newrefcontext[sorting=tixel]
\printbibliography[heading=subbibintoc, title=Primary Sources, keyword=primary]

\newrefcontext[sorting=nyt]
\printbibliography[heading=subbibintoc, title=Secondary Literature, keyword=seclit]

\printbibliography[heading=subbibintoc, title=Catalogues, keyword=catalogues]

\printbibliography[heading=subbibintoc, title=Online Sources, keyword=onlinesource]

\end{document}


%%% Local Variables:
%%% mode: latex
%%% TeX-master: t
%%% End:
