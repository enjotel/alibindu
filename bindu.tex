\input{preamble.tex}
\FormatDiv{1}{\begin{center}\Large}{\end{center}}
\FormatDiv{2}{\begin{center}\small}{\end{center}}
\FormatDiv{3}{\bfseries}{.}
\title{Yogatattvabindu of Rāmacandra\\ A Critical Edition and Annotated Translation}
\date{\today}

\parindent=15pt
\begin{document}

% Zitiermöglichkeiten:
%\footcite[See][p.\,1]{goldstein01:_tibet_englis_diction_moder_tibet}
%\footnote{\cite{goldstein01:_tibet_englis_diction_moder_tibet}.}

\frontmatter
\thispagestyle{empty}
\begin{center}
  {\Large \emph{The Yogatattvabindu}}\\[3mm]
\end{center}



\newpage

\

\thispagestyle{empty}



\normalsize


\newpage


\begin{center}
\thispagestyle{empty}

\

\vskip 2mm

\begin{otherlanguage}{iast}
\LARGE \sanskritfont{Yogatattvabindu}
\end{otherlanguage}

\vskip .4cm

\Huge Yogatattvabindu \\[7mm]
\Large Critical Edition\\
with annotated Translation


\large

\vspace{3cm}

Von

Nils Jacob Liersch
\small
\vfill

\vfill

Indica et Tibetica Verlag \\ % $\cdot$ 
Marburg 2024

\vskip 6mm

\end{center}

\newpage
\newpage \ \thispagestyle{empty}
\small  \

\noindent

\
\vfill


\small
\noindent \textbf{Bibliographische Information Der Deutschen Bibliothek}

\noindent
Die Deutsche Bibliothek verzeichnet diese Publikation in der Deutschen Nationalbibliographie;
detaillierte bibliographische Informationen sind im Internet über http://dnb.ddb.de abrufbar.

\noindent
\textbf{Bibliographic information published by Die Deutschen Bibliothek}

\noindent
Die Deutsche Bibliothek lists this publication in the Deutsche Nationalbibliographie; detailed
bibliographic data is available in the Internet at http://dnb.ddb.de.  


\vskip 1cm

\noindent
\copyright\ Indica et Tibetica Verlag, Marburg 2024

\medskip

\noindent
Alle Rechte vorbehalten / All rights reserved

\medskip

\noindent
Ohne ausdrückliche Genehmigung des Verlages ist es nicht gestattet, das Werk oder einzelne Teile
daraus nachzudrucken, zu vervielfältigen oder auf Datenträger zu speichern.

\smallskip

\noindent
Apart from any fair dealing for the purpose of private study, research, criticism or review, no
part of this book may be reproduced or translated in any form, by print, photo form, microfilm, or
any other means without written permission. Enquiries should be made to the publishers.

\bigskip

\noindent
Satz: \ \ Nils Jacob Liersch \\
Herstellung: \ \ BoD – Books on Demand GmbH, Norderstedt  \\

\bigskip

\noindent
%\ISBN     

\normalsize

\newpage

%\maketitle
\clearpage
\tableofcontents
\addtocounter{page}{-1}
\thispagestyle{empty}
\clearpage


\mainmatter

\chapter{Conventions in the Critical Apparatus}
\section{Sigla in the Critical Apparatus}

\begin{itemize}
\item E : Printed Edition
\item P : Pune BORI 664
\item L : Lalchand Research Library LRL5876
\item B : Bodleian Oxford D 4587
\item \None : NGMPP B 38-31
\item \Ntwo : NGMPP B 38-35 / A 1327-14
\item \Done : IGNCA 30019
\item \Uone : SORI 1574
\item \Utwo: SORI 6082
\end{itemize}

\chapter{Critical Edition \& Annotated Translation}
\cleardoublepage
\begin{alignment}[
  texts=edition[class="edition"];
  translation[class="translation"],
  ]
  \begin{edition}
    \begin{prose}[p17_02]
      \noindent
%------------------------------
%navīnāni         paṭṭasūtramaya     dhṛtāni vastrāṇi   \E
%navīnāni         paṭasūtramayāni    dhṛtāni vastrāṇi   \P
%navinīnīśpī      paṭṭasūtramayāni   dhṛtāni vastrāṇi// \L
%navinīnīr api    paṭṭasūtramayāni   dhṛtāni vastrāṇi// \B
%navīnāni         paṭasūtramayāni    dhṛtāni vastrāṇi/  \N1
%navīnāni         paṭasūtramayāni    dhṛtāni vastrāṇi// \D
%navīnāni         paṭasūtramayāni    dhṛtāni vastrāṇi/  \N2
%navīnāni         padasūtramayāni       tāni vastrāṇi   \U1
%navīnāni      paṭ(h)asūtramayāni    dhṛtāni            \U2
%------------------------------
%New durable clothes made of silk,  
%------------------------------
\app{\lem[wit={ceteri}]{navīnāni}
  \rdg[wit={B}]{navinīnīr api}
  \rdg[wit={L}]{navīnīnīś pī}}
\app{\lem[wit={B,E,L}, alt={paṭṭa°}]{paṭṭa}
  \rdg[wit={D,P,N1,N2,U2}]{paṭa°}
  \rdg[wit={U1}]{pada°}
}sūtra\app{\lem[wit={ceteri},alt={°mayāni}]{mayāni}
  \rdg[wit={E}]{°maya}}
\app{\lem[wit={ceteri}]{dhṛtāni}
  \rdg[wit={U1}]{tāni}}
\app{\lem[wit={ceteri}]{vastrāṇi}
  \rdg[wit={U2}]{\om}}
%------------------------------ %%%%KOLLOQUIUM: was hier tun? kastūrī/kasturikā = gleichwertig 
%athavā jīrṇāni chidrāṇi    dhṛtāni    kastūrīcandanalepair   vā  kardamalepena   yasya manasi harṣaśokau  na staḥ/ \E
%athavā jīrṇāni sachadrāṇi  dhūtāni    kastūrīcaṃdanalepo     vā  karddamalepo vā yasya manasi harṣaśokau na staḥ/ \P
%athavā jīrṇāni svachidrāṇi dhṛtāni    kasturīcaṃdanalepo     cā  kardamalepo  vā yasya manasi harṣaśokau na sthaḥ// \L
%athavā jīrṇāni svachidrāṇi dhṛtāni    kastūrīcaṃdanalepo     vā  kardamalepo  vā yasya manasi harṣaśokau na sthaḥ// \B
%athavā jīrṇāni sacchidrāṇi dhṛtāni/   kasturikā caṃdanalepo vā/ kardamalepo  vā yasya manasi harṣaśoko  na sthaḥ  \N1
%athavā jīrṇāni sacchidrāṇi dhṛtāni//  kasturikā caṃdanalepo vā/ kardamalepo  vā yasya manasi harṣaśoko  na sthaḥ  \D
%athavā jīrṇāni sacchidrāṇi dhṛtāni // kasturikā caṃdanalepo vā/ kardamalepo  vā yasya manasi harṣaśoka  na sthāḥ \N2
%athavā jīrṇāni sachidrāṇi  dhvatāni   kasturikā caṃdanalepo vā  kardamalepo  vā yasya manasi harṣaśokau na sthāḥ \U1 %%272.jg
%                                       kastūrīcaṃdanalepo     vā                  yasya manasi harṣaśoko  na sta// \U2
%------------------------------
%or however, old, worn (clothes) with holes smeared with sandalwood and musk, or smeared with mud. In whose mind joy and sorrow are not situated,
%------------------------------
athavā jīrṇāni
\app{\lem[wit={D,N1,N2}]{sacchidrāṇi}
  \rdg[wit={U2}]{sachidrāṇi}
  \rdg[wit={P}]{sachadrāṇi}
  \rdg[wit={B,L}]{svachidrāṇi}
  \rdg[wit={E}]{chidrāṇi}}
\app{\lem[wit={ceteri}]{dhṛtāni}
  \rdg[wit={U2}]{dhvātāni}
  \rdg[wit={P}]{dhūtāni}}
      \app{\lem[wit={X}]{kasturikā}
  \rdg[wit={B,E,P,U2}]{kastūrī}
  \rdg[wit={L}]{kasturī}
}candana\app{\lem[wit={ceteri}]{lepo}
  \rdg[wit={E},alt={lepair}]{lepair}}
\app{\lem[wit={ceteri},alt={vā}]{vā}
  \rdg[wit={L}]{cā}}
\app{\lem[wit={ceteri}]{kardamalepo}
  \rdg[wit={E}]{kardamalepena}}
\app{\lem[wit={ceteri}]{vā}
  \rdg[wit={E}]{\om}}/
yasya manasi
harṣa\app{\lem[wit={ceteri},alt={°śokau}]{śokau}
  \rdg[wit={D,N1,U2}]{°śoko}
  \rdg[wit={N2}]{°śoka}}
na
\app{\lem[type=emendation, resp=egoscr]{sthau}
  \rdg[wit={ceteri}]{sthaḥ}
  \rdg[wit={N2,U1}]{sthā}
  \rdg[wit={U2}]{sta}}
%------------------------------
%sa evātra tiṣṭhati/         \E
%sa eva rājayogaḥ            \P
%sa eva rājayogaḥ// idānīṃ// \L
%sa eva rājayogaḥ// idānīṃ// \B
%sa eva rājayogaḥ//          \N1
%sa eva rājayogaḥ//          \D
%sa eva rājayogaḥ//          \N2
%sa eva rājayogaḥ            \U1
%sa eva rājayoga             \U2
%------------------------------
%Only this is Rājayoga. 
%------------------------------
%yasya janmamaraṇe na staḥ sukhaṃ na bhavati/ kulaṃ na bhavati śīlaṃ na bhavati/ sthānaṃ na bhavati/ \E
%\om \P
%\om \L
%\om \B
%\om \N1
%\om \D
%\om \N2
%\om \U1
%\om \U2
%------------------------------
%One who is not situated in birth and death has no happiness, has no family, and cold does not arise, place does not arise.?!?!!?
%----------------------------
\app{\lem[wit={ceteri}]{sa eva}
  \rdg[wit={E}]{sa evātra}}
\app{\lem[wit={ceteri}]{rājayogaḥ}
  \rdg[wit={U2}]{rājayoga}
  \rdg[wit={B,L}]{rājayogaḥ || idānīṃ ||}
  \rdg[wit={E}]{tiṣṭhati | yasya janmamaraṇe na staḥ sukhaṃ na bhavati | kulaṃ na bhavati śīlaṃ na bhavati | sthānaṃ na bhavati |}}/
%---------
%rājayogaḥ naramadhye      atha ca vanamadhye             yuddhe saṃgrāmamadhye                        vā yasya manaḥ        bhayapūrṇaṃ vā  na bhavati/  so pi rājayogaḥ kathyate// \E
%          nagaramadhye    'tha ca vanamadhye                  utasaṃgrāmamadhye                       vā yasya mana      ūnaṃ    pūrṇaṃ vāṃ na bhavati   so pi rājayogaḥ            \P
%          nagaramadhye     tha ca vanamadhye                 udvastagrāmamadhye                       vā yasya manaḥ     unaṃ    pūrṇaṃ vā  na bhavati   so pi rājayogaḥ//          \L
%          nagaramadhye  (')tha ca vanamadhye                udvastagrāmaṃmadhye                       vā yasya manaḥ     unaṃ    pūrṇaṃ vā  na bhavatī   so pi rājayogaḥ//          \B
%          nagaramadhye    atha ca vanamadhye/                 udvesūgrāmamadhye .. ..pūrṇagrāmamadhye vā yasya manaḥ     ūnaṃ na pūrṇaṃ vā  na bhavati// so pi rājayogaḥ//          \N1
%          ṣagaramadhye    atha ca vanamadhye//                udvesūgrāmamadhye svetapūrṇagrāmamadhye vā yasya manaḥ     ūnan na pūrṇaṃ vā  na bhavati/  so pi rājayogaḥ//          \D
%          nagaramadhye    atha ca vanamadhye//                udvesūgrāmamadhye svetapūrṇagrāmamadhye vā yasya manaḥ     ūnan na pūrṇaṃ vā  na bhavati/  so pi rājayogaḥ//          \N2
%       vā nagaramadhye    atha ca vanamadhye                 udassaṃgrāmamadhye  lokapūrṇagrāmamadhye vā yasya manaḥ     unaṃ    pūrṇaṃ     na bhavati   so pi rājayogaḥ            \U1
%          nagaramadhye    'tha vā vanamadhye                  udvasagrāmamadhye                       vā yasya mana      ūnaṃ    pūrṇaṃ vāṃ na bhavati   so pi rājayogaḥ            \U2
%------------------------------
%Just he is in the state of Rājayoga for whom the mind is neither in abundance nor in lack, being located in a city, a forest, an uninhabited village or a village full of people. 
%----------------------------
\note[type=source, labelb=127b, nosep]{Cf. YSv (PT p. 835): harṣaśokau na jātv eṣāṃ nodvego lokasaṅgame | nityollāse nirākāre nirāsane nirātmani | manasā niścalo bhūtvā sadā tiṣṭhet samo 'pi ca |}
%\note[type=philcomm, labelb=127c, lem={nagaramadhye \ldots}]{Corresponding prose version of the original with extensive editorial changes in \uproman{20}.\lowroman{13}-\lowroman{15}.}
\app{\lem[wit={ceteri}]{nagaramadhye}
  \rdg[wit={E}]{rājayogaḥ nagaramadhye}
  \rdg[wit={D}]{ṣagaramadhye}
  \rdg[wit={U1}]{vā nagaramadhye}
}\app{\lem[wit={P,L,B,U2}]{'tha ca}
  \rdg[wit={D,E,N1,N2,U1}]{atha ca}}
vanamadhye
\app{\lem[wit={U2},alt={udvasa°}]{udvasa}
  \rdg[wit={E}]{yuddhe saṃ°}
  \rdg[wit={P}]{utasaṃ°}
  \rdg[wit={B,L}]{udvasta°}
  \rdg[wit={D,N1,N2}]{udvesū°}
  \rdg[wit={U1}]{udassaṃ°}
}\app{\lem[wit={ceteri}]{grāmamadhye}
  \rdg[wit={B}]{grāmaṃ madhye}}
\app{\lem[wit={U1}]{lokapūrṇagrāmamadhye}
  \rdg[wit={N1}]{....pūrṇagrāmamadhye}
  \rdg[wit={D,N2}]{svetapūrṇagrāmamadhye}}
vā yasya
\app{\lem[wit={P,U2}]{mana}
  \rdg[wit={ceteri}]{manaḥ}}
\app{\lem[wit={P,N1,N2,U2}]{ūnaṃ}
  \rdg[wit={D,N2}]{ūnan}
  \rdg[wit={B,L,U1}]{unaṃ}
  \rdg[wit={E}]{bhaya°}}
\app{\lem[wit={D,N1,N2}]{na}
  \rdg[wit={ceteri}]{\om}}
pūrṇaṃ
\app{\lem[wit={ceteri}]{vā}
  \rdg[wit={P,U2}]{vāṃ}
  \rdg[wit={U1}]{\om}}
na bhavati/ so\app{\lem[type=emendation, resp=egoscr]{'pi}
  \rdg[wit={ceteri}]{pi}}
\app{\lem[wit={ceteri}]{rājayogaḥ}
  \rdg[wit={E}]{rājayogaḥ kathyate}}\dd{}
\end{prose}
               \ekddiv{
                 head={[\uproman{18}. \textbf{caryāyogaḥ}]},
                 type=section,
                 depth=2, 
                 n=XVIII
                 }
                  \xmlhead[h18]{[XVIII. caryāyogaḥ]}
                \label{caryayoga}
\begin{prose}[p18_01]
    \noindent
%----------------------------
%idānīṃ      yogaḥ  kathyate/ \E
%idānīṃ caryāyogaḥ  kathyate   \P
%idānīṃ caryāyogaḥ  kathyate// \L
%idānīṃ caryāyogaḥ  kathyate// \B
%idānīṃ caryāyoga   kathyate// \N1
%idānīṃ caryāyogaḥ  kathyate// \D [S.7, Z.7]
%idānīṃ caryāyoga   kathyate// \N2
%idānīṃ tvaryāyogaḥ kathyate \U1
%idānīṃ caryāyoga   kathyate// \U2
%------------------------------
%Now \textit{caryāyogaḥ}, the Yoga of wandering is explained.
%----------------------------
idānīṃ
\app{\lem[wit={ceteri}]{caryāyogaḥ}
     \rdg[wit={U1}]{tvaryāyogaḥ}
     \rdg[wit={E}]{yogaḥ}} kathyate/
%----------------------------
%nirākāro         nityo 'bhedyaḥ    sa etādṛśaḥ ātmani                  mano   yasya  niścalaṃ tiṣṭhati/  \E
%nirākāro  'calo  nityo  bhedhyaḥ   sa etādṛa   ātmā    etādṛśo  ātmani mano   yasya  niścala  tiṣṭhati   \P %%%7639.jpg
%nirākāro  calo   nityo  bhedhyaḥ   sa etādṛa   ātmā sa etādṛśe  ātmani               niścala  tiṣṭhati/  \L     %daṇḍa nach ātmā besser -> emend? oder in weiteren Hss?
%nirākāro  calo   nityo  bhedhyaḥ   sa etādṛa   ātmā sa etādṛśye ātmani               niścalaṃ tiṣṭhati/  \B
%nirākālo  nityo   calo 'bhedhyaḥ/  sa etādṛśaḥ ātmā    etādṛśe  ātmani manaḥ  yasya  niścalaṃ tiṣṭhati   \N1
%nirākālo  nityo   calo 'bhedhyaḥ// sa etādṛśaḥ ātmā    etādṛśe  ātmani manaḥ  yasya  niścalaṃ tiṣṭhati   \D
%nirākālo  nityo   calo 'bhedhyaḥ   sa etādṛśaḥ ātmā    etādṛśa  ātmani manaḥ  yasya  niścala  tiṣṭhati/  \N2
%nirākāro  nityo   calo abhedhyaḥ   sa etādṛśaḥ ātmā    etādṛśo  ātmani mano   yasya  niścalaṃ bhavati    \U1
%nirvikāro  'calo nityo 'bhedhya    sa etādṛśā  ātmani                  mano   yasya  niścalaṃ tiṣṭhati// \U2
%------------------------------
%Shapeless, unchangeable, permanent [and] unsplitable - such is the self. Such is the mind of him who remains motionless like this in the self. 
%------------------------------
   \note[type=source, labelb=128, nosep]{Cf. YSv (PT p. 835): harṣaśokau na jātveṣāṃ nodvego lokasaṅgame | nityollāse nirākāre nirāsane nirātmani | manasā niścalo bhūtvā sadā tiṣṭhet samo 'pi ca |}
\app{\lem[wit={B,E,L,P,U1}]{nirākāro}
  \rdg[wit={D,N1,N2}]{nirākālo}
  \rdg[wit={U2}]{nirvikāro}}
\app{\lem[wit={X}]{nityo}
  \rdg[wit={P,U2}]{'calo}
  \rdg[wit={B,L}]{calo}
  \rdg[wit={E}]{\om}
}\app{\lem[wit={X}]{'calo}
  \rdg[wit={Y}]{nityo}
}\app{\lem[wit={D,E,N1,N2}]{'bhedyaḥ}
  \rdg[wit={B,L,P}]{bhedhyaḥ}
  \rdg[wit={U1}]{abhedhyaḥ}
  \rdg[wit={U2}]{'bhedyha}}
   sa
\app{\lem[wit={B,L,P}]{etādṛśa}
  \rdg[wit={D,E,N1,N2,U1}]{etādṛśaḥ}
  \rdg[wit={U2}]{etādṛśā}}
\app{\lem[wit={ceteri}]{ātmā}
  \rdg[wit={E,U2}]{ātmani}}/
\app{\lem[wit={D,N1}]{etādṛśe}
  \rdg[wit={B}]{sa etādṛśye}
  \rdg[wit={L}]{sa etādṛśe}
  \rdg[wit={N2}]{etādṛśa}
  \rdg[wit={P,U1}]{etādṛśo}
  \rdg[wit={E,U2}]{\om}}
\app{\lem[wit={ceteri}]{ātmani}
  \rdg[wit={E,U2}]{\om}}
\app{\lem[wit={E,P,U1,U2}]{mano}
  \rdg[wit={D,N1,N2}]{manaḥ}
  \rdg[wit={B,L}]{\om}}
\app{\lem[wit={ceteri}]{yasya}
  \rdg[wit={B,L}]{\om}}
\app{\lem[wit={ceteri}]{niścalaṃ}
  \rdg[wit={P,L,N2}]{niścala}}
\app{\lem[wit={ceteri}]{tiṣṭhati}
  \rdg[wit={U1}]{bhavati}}
%------------------------------
%tasyātmanaḥ puṇyapāpasparśo na bhavati/ \E
%tasyātmanaḥ puṇyapāpasparśo na bhavati  \P
%tasyātmanaḥ puṇyapāpasparśo na bhavati/ \L
%tasyātmanaḥ puṇyapāpasparśo na bhavatī/ \B
%tasyātmanaḥ punyapāpasparśo na bhavati/  \N1
%tasyātmanaḥ punyapāpasparśo na bhavati// \D
%tasyātmanaḥ puṇyapāpasparśo na bhavati/ \N2
%tasya ātmanaḥ puṇyapāsya sparśo na bhavati  \U1
%tasya ātmanaḥ puṇyapāsya sparśo na bhavati//  \U2
%------------------------------
%His self is not touched by sin and merit. 
%------------------------------
\app{\lem[wit={ceteri}]{tasyātmanaḥ}
  \rdg[wit={U1,U2}]{tasya ātmanaḥ}}
\app{\lem[wit={ceteri}]{puṇyapāpasparśo}
  \rdg[wit={U1,U2}]{puṇyapāsya sparśo}}
na bhavati/
%------------------------------
%udakamadhye sthitasya padmapatre       yathodakasya sparśo    bhavati/  tathaivātmani   \E
%udakamadhye sthitasya padmanī patrasya yathodakasya sparśo na bhavati   tathaivātmani   \P
%udakamadhye sthitasya padmanī patrasya yathodakasya sparśo na bhavati/  tathaivātmani   \L
%udakamadhye sthitasya padmanī patrasya yathodakasya sparśā na bhavatī/  tathaivātmani   \B
%udakamadhye sthitasya padminī patrasya yathā/ udakasparśo  na bhavati/  tathaivātmani   \N1
%udakamadhye sthitasya padminī patrasya yathā  udakasparśo  na bhavati// tathaivātmani   \D
%udakamadhye sthitasya padminī patrasya yathā  udakasparśo  na bhavati/  tathaivātmani   \N2
%udakamadhye sthitasya padminī patrasya yathā  udakasparśo  na bhavati   tathaivātmani   \U1
%udakamadhye sthitasya padminī patrasya yathodakasparśo     na bhavati// tathaivātmani   \U2
%------------------------------
%Just as touch with water does not arise for the lotusleaf situated in water; likewise in the self [touch with sin and merit does not arise].
%------------------------------
udakamadhye sthitasya
\app{\lem[wit={ceteri}, alt={padminīpatrasya}]{padminīpatra:\\sya}
  \rdg[wit={B,L,P}]{padmanīpatrasya}
  \rdg[wit={E}]{padmapatre}}
\app{\lem[wit={U2}]{yathodakasparśo}
  \rdg[wit={X}]{yathā udakasparśo}
  \rdg[wit={E,P,L}]{yathodakasya sparśo}
  \rdg[wit={B}]{yathodakasya sparśā}}
na
\app{\lem[wit={ceteri}]{bhavati}
  \rdg[wit={B}]{bhavatī}}
tathaivātmani/
%------------------------------
%yathākāśamadhye   pavanaḥ svecchayā bhramati/ \E
%yathākāśamadhye   pavanaḥ svechayā  bhramati \P
%yathā ākāśamadhye pavanaḥ svechayā  bhramati/ \L
%yathā ākāśamadhye pavanaḥ svechayā  bhramatī/ \B
%yathā ākāśamadhye pavanasvachayā    bhramati/ \N1
%yathā ākāśamadhye pavanasvachayā    bhramati \D
%yathā ākāśamadhye pavanasvachayā    bhramati/ \N2
%yathā ākāśamadhye pavanaḥ svechayā  bhramayati \U1
%yathā 'kāśamadhye pavanaḥ svechayā  bhramati// \U2
%------------------------------
%Just as the wind wanders according to its own will in space,...  
%------------------------------
\note[type=source, labelb=130, nosep]{cf. YSv (PT p. 835): yathākāśe bhraman vāyur ākāśaṃ vrajate svayam | tathākāśe mano līnaṃ rājayogakriyā matā | jagatsaṃsarganirlepaṃ padmapatrajalaṃ yathā |}
\app{\lem[wit={E,P}]{yathākāśamadhye}
  \rdg[wit={U2}]{yathā 'kāśamadhye}
  \rdg[wit={ceteri}]{yathā ākāśamadhye}}
\app{\lem[wit={ceteri}]{pavanaḥ svechayā}
  \rdg[wit={D,N1,N2}]{pavanasvachayā}}
\app{\lem[wit={ceteri}]{bhramati}
  \rdg[wit={U1}]{brahmayati}}
%------------------------------
%tathā yasya manaḥ nirākāramadhye līnaṃ bhavati/  sa eva caryāyogaḥ// \E
%tathā yasya manaḥ nirākāramadhye līnaṃ bhavati   sa eva caryāyogaḥ   \P
%tathā yasya manaḥ nirākāramadhye līnaṃ bhavati   sa eva caryāyogaḥ// \L
%tathā yasya manaḥ nirākāramadhye līnaṃ bhavatī   sa eva caryāyogaḥ// \B
%tathā yamanaḥ     nirākāramadhye līnaṃ bhavati/  sa eva kriyāyogaḥ// \N1
%tathā yasya manaḥ nirākāramadhye līnaṃ bhavati/  sa eva kriyāyogaḥ// \D !!!!!Stemma point!!!!!!
%tathā       pavananirākāramadhye līnaṃ bhavati/  sa eva kriyāyogaḥ// \N2
%tathā yasya manaḥ nirākāramadhye līnaṃ bhavati   sa eva kriyāyogaḥ   \U1 
%tathā yasya manaḥ nirākāramadhye līnaṃ bhavati// sa eva caryāyogaḥ// \U2
%------------------------------
%Likewise he whose mind of is absorbed into the universal spirit [wanders according to its own will in space]. This is \textit{\caryāyoga}.  
%------------------------------
tathā
\app{\lem[wit={ceteri}]{yasya manaḥ}
  \rdg[wit={D}]{yamanaḥ}
  \rdg[wit={N2}]{pavana°}}
nirākāramadhye līnaṃ
\app{\lem[wit={ceteri}]{bhavati}
  \rdg[wit={B}]{bhavatī}}
sa eva
\app{\lem[wit={Y}]{caryāyogaḥ}
  \rdg[wit={X}]{kriyāyogaḥ}}\dd{}
\end{prose}
\end{edition}
  \begin{translation}
    \begin{tlate}[p17_02]
      \noindent
Whether [one has] new clothes made of silk, or old, worn [clothes] with holes, whether [one is] smeared with sandalwood and musk, or smeared with mud - when delight and grief do not reside within the mind, it is that which is Rājayoga. When the mind is neither bored nor overwhelmed situated in a city, a forest, an uninhabited village or a village full of people, also this is Rājayoga. 
\end{tlate}
               \ekddiv{
                 head={[\uproman{18}. \textbf{Caryāyoga}]},
                 type=section,
                 depth=2, 
                 n=XVIII.1
               }
               \xmlhead[h18]{[XVIII. Caryāyoga]}
      \label{caryayogatrans}
     \begin{tlate}[p18_01]
      Now, Caryāyoga is explained.\footnote{Caryāyoga is not mentioned in YSv (PT and YK). The term is absent in the text and the initial list of fifteen Yogas. Rāmacandra, however, utilizes a passage that in YSv still belongs to the section on Rājayoga to construe this new type of Yoga. Due to its brevity, it might be an attempt to do justice to the list of Yogas provided in the beginning (cf. PT p. 835 (\textit{harṣaśokau} \ldots \textit{samo 'pi ca} |)). The passage's content does not explain why Rāmacandra uses the term \textit{caryā°} to specify this type of Yoga. The introduction of Caryāyoga into the list of fifteen yogas is based on the respective \textit{pāda} among the four \textit{pāda}s of the śaivaite Āgamas, which bear the same names as the first four Yogas in Rāmacandra's list of fifteen Yogas (\textit{kriyā}-, \textit{jñāna}-, \textit{caryā}- and \textit{yogapāda}). Perhaps, in this context, the concept of \textit{caryā°} = √\textit{car} + \textit{kṛt}-suffix \textit{-yā} f. might express the action, which refers to the meaning ``wandering, roaming'' of the verbal root √\textit{car}, which Rāmacandra brings up in his description. There is no connection to ritual conduct/discipline of śaivite practices. Since this is mere speculation, I refrain from attempting to translate it.} Formless, permanent, immovable [and] unsplittable - such is the self. For whose mind remains steady in such a self, his self does not come into contact with sin and merit. Just as contact with water does not arise for the lotus leaf situated in water, likewise in the [case of the] self. When the mind is absorbed into the formless,\footnote{The term \textit{nirākāra} was already used in the second sentence of this section as an adjective qualifying the self (\textit{ātman}). Here, it is a noun and probably synonymous with the self.} in the same way as the wind wanders according to its own will in space, only that is Caryāyoga.\footnote{Parallels to Rāmacandra's innovative Caryāyoga can be identified in the texts with similar taxonomies. In \citetitle{yogacandrika} (ed. pp. 2, 52-53, 100-101, 150) Nārāyaṇatīrtha presents Caryāyoga in the context of Yogasūtra 1.33. According to Nārāyaṇatīrtha's commentary, the practice of this involves the cultivation of specific mental attitudes, such as \textit{maitrī} (loving-kindness), \textit{karuṇā} (compassion), \textit{muditā} (compassionate joy) and \textit{upekṣā} (equanimity), towards various objects or situations, such as happiness, suffering, merit and demerit. Sundardās, in his \citetitle{sarvangayoga} (2.40-51, ed. pp. 96-98), describes the similar sounding Cārcāyog as a type of \textit{bhaktiyog} that is \textit{bhakti} towards the unmanifest consciousness (\textit{avyakta puruṣa}) in rapturous devotion. According to Sundardās, the unmanifest consciousness (\textit{avyakta puruṣa}) is formless, eternal, etc. (40). However, in Sanskrit and \textit{brājbhāṣā} the term means ``discussion''. It has nothing to do with \textit{caryā}, and we thus must assume that both types are unrelated. A detailed discussion of Caryāyoga can be examined on p.\pageref{caryayogaintro}.} 
       \flushpage
     \end{tlate}
   \end{translation}
 \end{alignment}
 \pagebreak %after pp. 37-38
%%%%%%%%%%%%%%%%%%%%%%%%%%%%%%%%%%%%%%%%%%
%%%%%%%%%%%%%%%%%%%%%%%%%%%%%%%%%%%%%%%%%%
%%%%%%%%PAGEBREAK%%%%%%%PAGEBREAK%%%%%%%%%
%%%%%%%%%%%%%%%%%%%%%%%%%%%%%%%%%%%%%%%%%%
%%%%%%%%%%%%%%%%PAGEBREAK%%%%%%%%%%%%%%%%%
%%%%%%%%%%%%%%%%%%%%%%%%%%%%%%%%%%%%%%%%%%
%%%%%%%%PAGEBREAK%%%%%%%PAGEBREAK%%%%%%%%%
%%%%%%%%%%%%%%%%%%%%%%%%%%%%%%%%%%%%%%%%%%
%%%%%%%%%%%%%%%%%%%%%%%%%%%%%%%%%%%%%%%%%%
%%%%%%%%%%%%%%%%%%%%%%%%%%%%%%%%%%%%%%%%%%
%%%%%%%%%%%%%%%%%%%%%%%%%%%%%%%%%%%%%%%%%%
%%%%%%%%PAGEBREAK%%%%%%%PAGEBREAK%%%%%%%%%
%%%%%%%%%%%%%%%%%%%%%%%%%%%%%%%%%%%%%%%%%%
%%%%%%%%%%%%%%%%PAGEBREAK%%%%%%%%%%%%%%%%%
%%%%%%%%%%%%%%%%%%%%%%%%%%%%%%%%%%%%%%%%%%
%%%%%%%%PAGEBREAK%%%%%%%PAGEBREAK%%%%%%%%%
%%%%%%%%%%%%%%%%%%%%%%%%%%%%%%%%%%%%%%%%%%
%%%%%%%%%%%%%%%%%%%%%%%%%%%%%%%%%%%%%%%%%%
%%%%%%%%%%%%%%%%%%%%%%%%%%%%%%%%%%%%%%%%%%
%%%%%%%%%%%%%%%%%%%%%%%%%%%%%%%%%%%%%%%%%%
%%%%%%%%PAGEBREAK%%%%%%%PAGEBREAK%%%%%%%%%
%%%%%%%%%%%%%%%%%%%%%%%%%%%%%%%%%%%%%%%%%%
%%%%%%%%%%%%%%%%PAGEBREAK%%%%%%%%%%%%%%%%%
%%%%%%%%%%%%%%%%%%%%%%%%%%%%%%%%%%%%%%%%%%
%%%%%%%%PAGEBREAK%%%%%%%PAGEBREAK%%%%%%%%%
%%%%%%%%%%%%%%%%%%%%%%%%%%%%%%%%%%%%%%%%%%
%%%%%%%%%%%%%%%%%%%%%%%%%%%%%%%%%%%%%%%%%%
\begin{alignment}[
  texts=edition[class="edition"];
  translation[class="translation"],
  ]
  \begin{edition}
    \ekddiv{
      head={[\uproman{19}. \textbf{haṭhayogaḥ}]},
      type=section,
      depth=2, 
      n=XIV
    }
    \xmlhead[h19]{[XIX. haṭhayogaḥ]}
\label{hathayoga}
    \begin{prose}[p19_01]
      \noindent
%------------------------------
%idānīṃ grahayogaḥ kathyate/  \E %[p.23]
%idānīṃ haṭhayogaḥ kathyate   \P
%idānīṃ haṭhayogaḥ kathyate/  \L
%idānīṃ haṭayoga   kathyate/  \B
%idānīṃ haṭhayogaḥ kathyate//  \N1
%idānīṃ haṭhayogaḥ kathyate/  \D
%idānīṃ haṭhayoga  kathyate// \N2
%idānīṃ haṭhayogaḥ kathyate   \U1
%idānīṃ haṭhayoga  kathyate   \U2
%------------------------------
%Now \textit{haṭhayoga} is explained. 
%------------------------------
idānīṃ
\app{\lem[wit={D,L,P,N1,U1}]{haṭhayogaḥ}
     \rdg[wit={B}]{haṭayoga}
     \rdg[wit={E}]{grahayogaḥ}
     \rdg[wit={U2}]{haṭhayoga}} kathyate/
\note[type=source, labelb=131, labele=_131e, nosep]{cf. YSv (PT p. 835): idānīṃ haṭhayogas tu kathyate haṭhasiddhidaḥ | kṛtvāsanaṃ pavanāśaṃ śarīre rogahārakam | pūrakaṃ kumbhakañcaiva recakaṃ vāyunā bhajet | itthaṃ kramotkramaṃ jñātvā pavanaṃ sādhayet sadā | dhauty ādikarmaṣaṭkañ ca prakuryād haṭhasādhakaḥ | etan nāḍyān tu deveśi vāyupūrṇaṃ pratiṣṭhitam | tato mano niścalaṃ syāt tata ānanda eva hi | haṭhayogān na kālaḥ syān manonāśo bhaved yadi |}
%------------------------------
%recakapūrakakumbhaka  ity ādiprakāreṇa   pavanasādhanaṃ     kartavyam/ \E
%recakapūrakakuṃbhaka  ity ādiprakāreṇa   pavanasādhanaṃ     karttavyaṃ \P
%recakapūrakakumbhaka  ity ādiprakāreṇa   pavanasya sādhanaṃ kartavyam// \L
%recakapūrakakuṃbhaka  ity ādiprakāreṇa// pavanasya sādhanaṃ kartavyam \B
%recakapūrakakuṃbhaka/ ity ādiprakāreṇa   pavanasya sādhanaṃ kartavyaṃ/ \N1
%recakapūrakakuṃbhaka  ity ādiprakāreṇa   pavanasya sādhanaṃ kartavyaṃ// \D
%recakapūrakakuṃbhaka  ity ādhiprakāreṇa  pavanasya sādhanaṃ kartavyaṃ// \N2
%recakapūrakakuṃbhaka  ity ādiprakāreṇa   pavanasya sādhanaṃ kartavyaṃ \U1
%recakapūrakakuṃbhaka  ity ādiprakāreṇa   pavanasya sādhanaṃ kartavyaṃ// \U2
%------------------------------
%The practice of breath shall be done in this manner: "Exhalation, Inhalation [and] Retention etc.
%------------------------------        
 recakapūrakakuṃbhaka
        \app{\lem[wit={ceteri}, alt={ity ādi°}]{ityādi}
          \rdg[wit={N2}]{ity ādhi°}
        }prakāreṇa
        \app{\lem[wit={ceteri}]{pavanasya sādhanaṃ}
          \rdg[wit={E,P}]{pavanasādhanaṃ}}
 \app{\lem[wit={B,E,L}]{kartavyam}
   \rdg[wit={ceteri}]{kartavyaṃ}}/
%------------------------------
%atha ca dhautyādiṣaṭkarmakāraṇāt   śarīrasya śuddhir bhavati/ \E
%atha ca dhautyādiṣaṭkarmakāraṇāt   śarīrasya śuddhir bhavati \P
%atha ca dhautyādiṣaṭkarmakāraṇāt// śarīrasya śuddhir bhavati \L
%atha ca  dhotyādiṣaṭkarmakaraṇāt// śarīrasya śuddhir bhavatī \B
%atha ca dhautyādiṣaṭkarmakaraṇāt/  śarīrasya śuddhir bhavati/ \N1
%atha ca dhautyādiṣaṭkarmakaraṇāt   śarīrasya śuddhir bhavati// \D
%atha ca dhautyādiṣaṭkarmakaraṇāt// śarīrasya śuddhir bhavati// \N2
%atha   vidhotyādiṣaṭkarmakaraṇāt   śarīrasya śuddhir bhavati/ \U1
%atha ca dhautyādiṣaṭkarmakaraṇāt// śarīrasya śuddhir bhavati// \U2 %%%408.jpg 
%------------------------------
%And then due to the six practices(\textit{ṣaṭkarma}), like \textit{dhauti} etc. the purification of the body arises. 
%------------------------------        
 atha
 \app{\lem[wit={ceteri}]{ca}
   \rdg[wit={U1}]{\om}}
 \app{\lem[wit={ceteri}, alt={dhautyādi}]{dhauty:\\ādi}
   \rdg[wit={B}]{dhotyādi}
   \rdg[wit={U1}]{vidhotyādi}
 }ṣaṭkarmakāraṇāt śarīrasya śuddhir\skp{-}bhavati/
 %------------------------------
%sūryanāḍīmadhye       pavanaḥ pūrṇo yadā tiṣṭati/   \E %!
%sūryanāḍīmadhye       pavanaḥ pūrṇo yadā tiṣṭati    \P
%sūryanāḍīmadhye       pavanapūrṇo   yadāti/         \L
%sarvasūryanāḍīmadhye  pavanapūrṇo   yadāti/         \B
%sūryanāḍīmadhye       pavanaḥ pūrṇo yadā tiṣṭhati/  \N1
%sūryanāḍīmadhye       pavanaḥ pūrṇo yadā tiṣṭhati   \D
%sūryanāḍīmadhye       pvanaḥ  pūrṇo yadā tiṣṭhati/  \N2
%sūryanāḍīmadhye       pavanaḥ pūrṇo yadā tiṣṭhati/  \U1
%sūryanāḍīmadhye       pavanaḥ sūryo yadā tiṣṭhati// \U2
%------------------------------
%When the full breath abides in the middle of the sun-channel, ... 
%------------------------------
 \app{\lem[wit={ceteri}]{sūryanāḍīmadhye}
   \rdg[wit={B}]{sarvasūryanāḍīmadhye}}
 \app{\lem[wit={ceteri}]{pavanaḥ pūrṇo}
   \rdg[wit={B,L}]{pavanapūrṇo}
   \rdg[wit={N2}]{pvanaḥ pūrṇo}}
 \app{\lem[wit={ceteri}]{yadā tiṣṭhati}
   \rdg[wit={B,L}]{yadāti}}
%------------------------------
%tadā mano  niścalaṃ bhavati/  \E
%tadā mano  niścalo  bhavati   \P
%tadā mano  niścalo  bhavati/  \L
%tadā mano  niścalo  bhavatī// \B
%tadā manaḥ niścalaṃ bhavati/  \N1
%tadā manaḥ niścalaṃ bhavati   \D
%tadā manaḥ niścalaṃ bhavati   \N2
%tadā manaḥ niścalaṃ bhavati   \U1
%tadā mano  niścalaṃ bhavati// \U2
%------------------------------
%Then the mind is unmovable. 
%------------------------------
 tadā
 \app{\lem[wit={Y}]{mano}
   \rdg[wit={X}]{manaḥ}}
\app{\lem[wit={ceteri}]{niścalaṃ}
  \rdg[wit={B,L,P}]{niścalo}}
bhavati/
%------------------------------
%manaso  niścalatvena ānandarūpaṃ      pratyakṣaṃ bhāsate/  \E
%manaso  niścalatve   ānandaṃ svarūpa--pratyakṣaṃ bhāsate   \P %%%%7640.jpg
%manaso  niścalatve   ānandaṃ svarūpaṃ pratyakṣaṃ bhāsate/  \L
%manaso  niścalatve   ānaṃdaṃ svarūpaṃ pratyakṣaṃ bhāsate// \B
%manasaḥ niścalatve   ānaṃdasvarūpaṃ   pratyakṣaṃ bhāsate/  \N1
%manasaḥ niścalatve   ānaṃdasvarūpaṃ   pratyakṣaṃ bhāsate/  \D
%manasaḥ niścalatve   ānaṃdasvarūpaṃ   pratyakṣaṃ bhāṣate/  \N2
%manasaḥ niścalatve   ānaṃdasvarūpaṃ   pratyakṣaṃ bhāṣate/  \U1 %%%273.jpg
%manaso  niścalatve   ānaṃdasvarūpaṃ   pratyakṣaṃ bhāsate// \U2
%------------------------------
%The form of bliss immediately shines through the motionless mind.  
%------------------------------
\app{\lem[wit={Y}]{manaso}
  \rdg[wit={X}]{manasaḥ}}
\app{\lem[wit={ceteri}]{niścalatve}
  \rdg[wit={E}]{niścalatvena}}
\app{\lem[wit={ceteri}]{ānandasvarūpaṃ}
  \rdg[wit={B,L}]{ānaṃdaṃ svarūpaṃ}
  \rdg[wit={P}]{ānandaṃ svarūpa°}
  \rdg[wit={E}]{ānandarūpaṃ}}
pratyakṣaṃ
\app{\lem[wit={ceteri}]{bhāsate}
  \rdg[wit={N2,U1}]{bhāṣate}}/
%------------------------------
%haṭhayogakāraṇāt  manaḥ   śūnyamadhye līnaṃ   bhavati/  kālaḥ samīpe   nāgacchati/  \E
%haṭhayogakāraṇāt  manaḥ   śūnyamadhye līnaṃ   bhavati   kālaḥ samīpe   nāgacchati   \P %%%%7640.jpg
%haṭhayogakāraṇāt  manaḥ   śūnyamadhye līnaṃ   bhavati/  kālaḥ samīpe   nāgacchati// \L
%haṭayogākāraṇāt   manaḥ// śūnyamadhye līnaṃ   bhavatī/  kālāsamīpe nāma gacchati//  \B
%haṭhayogakaraṇāt  manaḥ   śūnyamadhye līnaṃ   bhavati/  kālaḥ samīpe   nāgachati//  \N1
%haṭhayogakaraṇāt  manaḥ   śūnyamadhye līnaṃ   bhavati// kālaḥ samīpe   nāgachaṃti// \D
%haṭhayogakaraṇāt  mana----śūnyamadhye līnaṃ   bhavati/  kālasamīpe     nāgachati//  \N2
%haṭhayogakaraṇāt/ manaḥ   śūnyamadhye līnaṃ   bhavati/  kālasamīpe ti  nāgachati    \U1 %%%273.jpg
%haṭhayogakaraṇāt  manaḥ   śūnyamadhye sthānaṃ bhavati// kāsaḥ samīpe   nāgachati//  \U2
%------------------------------
%Due to the execution of haṭhayoga the mind becomes absorbed into emptiness. The time of death does not approach.
%------------------------------
\app{\lem[wit={ceteri}, alt={haṭha°}]{haṭha}
  \rdg[wit={B}]{haṭa°}
}\app{\lem[wit={ceteri},alt={yoga°}]{yoga}
  \rdg[wit={B}]{yogā°}
}\app{\lem[wit={ceteri}]{karaṇāt}
  \rdg[wit={B,E,L,P}]{kāraṇāt}}
\app{\lem[wit={ceteri}]{manaḥ}
  \rdg[wit={N2}]{mana}}
śūnyamadhye
\app{\lem[wit={ceteri}]{līnaṃ}
  \rdg[wit={U2}]{sthānaṃ}}
bhavati/
\app{\lem[wit={ceteri}]{kālaḥ}
  \rdg[wit={B}]{kālā°}
  \rdg[wit={N2,U1}]{kāla°}
  \rdg[wit={U2}]{kāsaḥ}}
samīpe
\app{\lem[wit={ceteri}]{nāgacchati}
  \rdg[wit={B}]{nāma gacchati}
  \rdg[wit={D}]{nāgachaṃti}
  \rdg[wit={U1}]{ti nāgachati}}\linelabel{_131e}\dd{}
\end{prose}
       \ekddiv{
                 head={[\uproman{20}. \textbf{haṭhayogasya dvitīyo bhedaḥ}]},
                 type=section,
                 depth=2, 
                 n=XX
               }
               \xmlhead[h20]{[XX. haṭhayogasya dvitīyo bhedaḥ]}
\label{secondtypehatha}
\begin{prose}[p20_01]
  \noindent
%------------------------------
%idānīṃ haṭhayogasya dvitīyo  bhedaḥ kathyate/   \E
%idānīṃ haṭhayoga----dvitīya--bhedaḥ kathyate    \P
%idānīṃ haṭhayogasya dvitīya--bhedāḥ kathyante/  \L
%idānīṃ haṭayogasya  dvitīyaṃ bhedāḥ kathyaṃte// \B
%idānīṃ haṭhayogasya dvitīyo  bhedaḥ kathyate//  \N1
%idānīṃ haṭhayogasya dvitīya--bhedaḥ kathyate    \D
%idānīṃ haṭayogasya  dvitīyo  bhedaḥ kathyate    \U1
%idānīṃ haṭhayogasya dvitīyo  bhedaḥ kathyate//  \U2 
%------------------------------
%Now, the second division of haṭhayoga is explained.
%------------------------------
idānīṃ
\app{\lem[wit={ceteri}]{haṭhayogasya}
  \rdg[wit={B,U1}]{haṭayogasya}
  \rdg[wit={P}]{haṭhayoga°}}
\app{\lem[wit={ceteri}]{dvitīyo}
  \rdg[wit={D,L,P}]{dvitīya°}
  \rdg[wit={B}]{dvitīyaṃ}}
\app{\lem[wit={ceteri}]{bhedaḥ}
  \rdg[wit={B,L}]{bhedāḥ}}
\app{\lem[wit={ceteri}]{kathyate}
  \rdg[wit={B,L}]{kathyante}}/ \note[type=source, labelb=132, labele=_132e, nosep]{cf. YSv (PT p. 835): idānīṃ haṭhayogasya dvitīyaṃ bhedam acchṛṇu | ākāśe nāsikāgre tu sūryakoṭisamaṃ smaret | śvetaṃ raktaṃ tathā pītaṃ kṛṣṇam ity ādirūpataḥ | evaṃ dhyātvā cirāyus syād aṅgājananavarjitam (\textit{°varjitaḥ} YK 12.25) | śivatulyo mahātmāsau haṭhayogaprasādataḥ (\textit{°prasaṅgataḥ} YK 12.25) | haṭhāj jyotir (\textit{haṭha°} YK 12.26) mayo bhūtvā hyantareṇa śivo bhavet | ato 'yaṃ haṭhayogaḥ syāt siddhidaḥ siddhasevitaḥ |}
%------------------------------
%pādādārabhya śiraḥ paryaṃtaṃ    svaśarīre  koṭisūryatejaḥ   samānaṃ śvetaṃ pītaṃ       raktaṃ kiṃcidvarṇaṃ ciṃtyate/  \E
%pādādārabhya śiraḥ paryaṃtaṃ    svaśarīre  koṭisūryatejaḥ   samānaṃ śvetaṃ pītaṃ nīlaṃ raktaṃ kiṃdrupaṃ    cityate    \P
%pādādārabhya śira--paryaṃtaṃ    svaśarīre  koṭisūryatejaḥ   samānaśvetaṃ nīlaṃ         raktaṃ tiṃdrupaṃ    ciṃtate/   \L
%pādādārabhya śira--paryaṃtaṃ    svaśarīre  koṭisūryatejaḥ// samānaśvetanīlaṃ           raktaṃ kiṃdrupaṃ    ciṃtate//  \B
%pādādārabhyā śiraḥ paryentaṃ    svaśarīre  koṭisūryatejaḥ   samānaṃ śvetaṃ pītaṃ nīlaṃ laktaṃ kiṃcidrūpaṃ  ciṃtyate   \N1 
%pādādārabhyā śiraḥ paryaṃtaṃ    svaśarīre  koṭisūryatejaḥ   samānaṃ śvetaṃ pītaṃ nīlaṃ raktaṃ kiṃcidrūpaṃ  ciṃtyate   \D
%pādādārabhya śiraḥ pariyataṃ    svaśarīraṃ koṭisūryatejaḥ   samānaṃ śvetaṃ pītaṃ nīlaṃ raktaṃ ciṃrūpaṃ     ciṃtyate   \U1
%pādādārabhya śiro  paryaṃtaṃ    svaśarīre  koṭisūryye tejaḥ samānaṃ śvetaṃ pītaṃ nīlaṃ raktaṃ kiṃcidrūpaṃ  ciṃtyate// \U2
%------------------------------
%The shine of ten million suns in one's own body beginning from the feet to the top of head is contemplated in any color equal to white, yellow [or] red.
%------------------------------
\note[type=testium, labelb=132v, labele=_132ex, nosep]{cf. \approx \citetitle{hathasamketacandrikajodhpur} (f.125 ll.4-5): pādādārabhya śiraḥparyaṃtasya śarīre koṭisūryatejaḥsadṛśaṃścetaṃ pītaṃ raktaṃ vā kiṃcidrūpaṃ viciṃtya tasya dhyānakaraṇātsarvāṃge rogajvalanaṃ bhavati ||}\linelabel{132v}
\app{\lem[wit={ceteri}]{pādādārabhya}
  \rdg[wit={N1,D}]{pādādārabhyā}}
\app{\lem[wit={ceteri}]{śiraḥ}
  \rdg[wit={B,L}]{śira°}
  \rdg[wit={U2}]{śiro}}
\app{\lem[wit={ceteri}]{paryantaṃ}
  \rdg[wit={N1}]{paryentaṃ}
  \rdg[wit={U1}]{pariyataṃ}}
\app{\lem[wit={ceteri}]{svaśarīre}
  \rdg[wit={U1}]{svaśarīraṃ}}
\app{\lem[wit={ceteri}]{koṭisūryatejaḥ}
  \rdg[wit={U2}]{koṭisūryye tejaḥ}}
\app{\lem[wit={ceteri}]{samānaṃ}
  \rdg[wit={B,L}]{samāna°}}
  \app{\lem[wit={ceteri}]{śvetaṃ}
  \rdg[wit={B}]{śveta°}}
\app{\lem[wit={ceteri}]{pītaṃ}
  \rdg[wit={B,L}]{\om}}
nīlaṃ
\app{\lem[wit={ceteri}]{raktaṃ}
  \rdg[wit={N1}]{laktaṃ}}
\app{\lem[wit={D,N1,U2}]{kiṃcidrūpaṃ}
  \rdg[wit={B,P}]{kiṃdrupaṃ}
  \rdg[wit={L}]{tiṃdrupaṃ}
  \rdg[wit={U1}]{ciṃrūpaṃ}
  \rdg[wit={E}]{kiṃcidvarṇaṃ}}
\app{\lem[wit={ceteri}]{cintyate}
  \rdg[wit={P}]{cityate}
  \rdg[wit={B,L}]{ciṃtate}}/
\linelabel{_132ex}
%------------------------------
%ttad  dhyānakāraṇāt     sakalaṃ   rogajvalanaṃ     bhavati/                      āyur          vardhate/          \E
%tad   dhyānakāraṇāt     sakalāṃge rogajvalanaṃ  na bhavati                       āyur vṛddhir  bhavati   \P
%tad   dhyānakāraṇāt     sakalaṃge rogajvalanaṃ  na bhavati/                      āyur          vardhate/          \L
%tat   dhyānakāraṇāt     sakalaṃge rogajvalanaṃ  na bhavati/                      āyur vṛddhir  bhavatī/  \B
%na    dhyānaṃ kāraṇāt/  sakalāṃge roga          na bhavati/  jvalanaṃ na bhavati āyur vṛddhir  bhavati/  \N1
%ta    dhyānaṃ karaṇāt// sakalāṃge rogajvalanaṃ  na bhavati//                                             \D
%tad---dhyānaṃ karaṇāt / sakalāṃge roga          na bhavati   jvaranaṃ na bhavati āyu--vṛddhir  bhavati// \N2
%ta    dhyānaṃ karaṇāt   sakalāṃge roga kṣataṃ?  na bhavati                       āyur vṛddhir  bhavati   \U1
%tat   dhyānakāraṇāt     sakalāṃge rogajvalanaṃ     bhavati//                     āyur vṛddhir  bhavati// \U2
%------------------------------
%Due to the execution of meditation in the entire body disease does'nt arise, fever doesn't arise and vitality grows.  
%------------------------------
\app{\lem[wit={E,L,P,N2},alt={tad}]{ta\skp{d-dhyā}}
  \rdg[wit={B,U2}]{tat}
  \rdg[wit={D,U1}]{ta}
  \rdg[wit={N1}]{na}
}\app{\lem[wit={Y},alt={dhyānakāraṇāt}]{\skm{d-dhyā}nakāraṇāt}
  \rdg[wit={X}]{dhyānaṃ karaṇāt}}
\app{\lem[wit={X,P,U2}]{sakalāṅge}
  \rdg[wit={B,L}]{sakalaṃge}
  \rdg[wit={E}]{sakalaṃ}}
\app{\lem[wit={Y,D}]{rogajvalanaṃ}
\rdg[wit={N1,N2}]{roga}
\rdg[wit={U1}]{roga kṣataṃ}}
\app{\lem[wit={E,U2}]{bhavati}
  \rdg[wit={B,L,P,D,U1}]{na bhavati}
  \rdg[wit={N1}]{na bhavati | jvalanaṃ na bhavati}
  \rdg[wit={N2}]{na bhavati | jvaranaṃ na bhavati}}/
\app{\lem[wit={ceteri}, alt={āyur}]{āyu\skp{r-vṛ}}
  \rdg[wit={N2}]{āyu°}
  \rdg[wit={D}]{\om}
}\app{\lem[wit={ceteri},alt={vṛddhir}]{\skm{r-vṛ}ddhi\skp{r-bha}}
  \rdg[wit={D,E,L}]{\om}
}\app{\lem[wit={ceteri},alt={bhavati}]{\skm{r-bha}vati}
  \rdg[wit={B}]{bhavatī}
  \rdg[wit={E,L}]{vardhate}
  \rdg[wit={D}]{\om}}\linelabel{_132e}\dd{}
\end{prose}
  \end{edition}
  \begin{translation}
\ekddiv{
  head={[\uproman{19}. \textbf{Haṭhayoga}]},
  type=section,
  depth=2, 
  n=XIV.1
}
\xmlhead[h19]{[XIX. Haṭhayoga]}
\label{hathayogatrans}
      \begin{tlate}[p19_01]
        \noindent
        \footnote{The YSv's description of the two types of Haṭhayoga is quoted in \citetitle{shabdakalpadruma} p. 501. I would like to thank Franz Veit for providing this reference.} Now, Haṭhayoga is explained. Breath is to be controlled by means of practices such as: "Exalation, inhalation [and] retention etc.\footnote{The term \textit{ādi} should refer to the other common practices of Haṭhayoga such as \textit{mudrā}, \textit{āsana} and \textit{nādānusandhāna}. Cf. \citetitle{hathapradipika2024} 1.56.} And then due to the six actions (\textit{ṣaṭkarma}), like \textit{dhauti} etc. \footnote{See \citetitle{hathapradipika2024} 2.22-37.}, the purification of the body arises. When the full breath abides in the middle of the sun channel\footnote{Usually the \textit{sūryanāḍi} is the \textit{piṅgalā}-channel or right nostril, as previously declared in \uproman{3}. sentence seven (p. \pageref{siddhayoga}, l. 3). Here, it appears more likely that \textit{sūryanaḍī} refers to the central channel, the \textit{suṣūṃṇā}.}, then the mind is unmovable. When the mind is motionless then the nature of bliss immediately appears. Due to Haṭhayoga, the mind becomes absorbed into emptiness. Time [as death] does not approach.
      \end{tlate}
\ekddiv{
  head={[\uproman{20}. \textbf{Second division of Haṭhayoga}]},
  type=section,
  depth=2, 
  n=XX.1
}
\xmlhead[h20]{[XX. Second division of Haṭhayoga]}
\label{secondhathatrans}
      \begin{tlate}[p20_01]
        \indent Now, the second division of Haṭhayoga is explained.\footnote{At this point YSv as quoted with reference in YK 12.23 adds a verse not found in the \citetitle{ramatosana} (\textit{susthāsanaṃ samāsīno nīrajāyatalocanaḥ} | \textit{cintayet paramātmānaṃ yo vadet sa bhaviṣyati} |).} Some kind of form being white, yellow, blue [and] red, equal to the shine of ten million suns shall be contemplated in the own body from the feet to the top of the head. Due to meditation on that, the burning of diseases in the entire body arises. The lifespan increases.\begin{buber}[f20_1]\footnote{Cf. YSv (PT p. 835) as presented in \textbf{sources} for \uproman{20}. p.\pageref{hathayoga}: ''Now, listen to the second variation of Haṭhayoga. Contemplate the space at the tip of the nose as being equal to the radiance of ten million suns in colours such as white, red, yellow, black, and other colours of that nature. By meditating in this way, one can achieve a long life because one is freed from the process of ageing (\textit{aṅgajaraṇavarjitaḥ} ] em. \textit{aṅgājananavarjitaṃ}). Through the devoted practice of Haṭhayoga, one whose self is great becomes like Śiva. Having become like the light, one truly becomes one with Śiva inside. Therefore, the path of Haṭhayoga will bring forth supernatural abilities and is followed by the Siddhas.'' Rāmacandras transfer misses various details, but both description remind of Bāhyalakṣya (see section \uproman{23} on p.\pageref{bahya}). Another light-based technique of Haṭhayoga, which is classified as a technique of \textit{dhyāna} involves visualising equally intense light at the navel, heart and head and results in igniting this light in all six \textit{cakra}s and ultimately leading to liberation from the fetters of birth (\textit{mucyante janmabandhanāt}) can be found in \citetitle{liersch2023} 33-50. Another similarity appears in \ldots}\end{buber}
%        \flushpage 
        \end{tlate}
      \end{translation}
    \end{alignment}
    \pagebreak %after pp. 39-40
%%%%%%%%%%%%%%%%%%%%%%%%%%%%%%%%%%%%%%%%%%
%%%%%%%%%%%%%%%%%%%%%%%%%%%%%%%%%%%%%%%%%% 
%%%%%%%%PAGEBREAK%%%%%%%PAGEBREAK%%%%%%%%%
%%%%%%%%%%%%%%%%%%%%%%%%%%%%%%%%%%%%%%%%%% 
%%%%%%%%%%%%%%%%PAGEBREAK%%%%%%%%%%%%%%%%%
%%%%%%%%%%%%%%%%%%%%%%%%%%%%%%%%%%%%%%%%%% 
%%%%%%%%PAGEBREAK%%%%%%%PAGEBREAK%%%%%%%%%
%%%%%%%%%%%%%%%%%%%%%%%%%%%%%%%%%%%%%%%%%% 
%%%%%%%%%%%%%%%%%%%%%%%%%%%%%%%%%%%%%%%%%% 
%%%%%%%%%%%%%%%%%%%%%%%%%%%%%%%%%%%%%%%%%% 
%%%%%%%%%%%%%%%%%%%%%%%%%%%%%%%%%%%%%%%%%% 
%%%%%%%%PAGEBREAK%%%%%%%PAGEBREAK%%%%%%%%%
%%%%%%%%%%%%%%%%%%%%%%%%%%%%%%%%%%%%%%%%%% 
%%%%%%%%%%%%%%%%PAGEBREAK%%%%%%%%%%%%%%%%%
%%%%%%%%%%%%%%%%%%%%%%%%%%%%%%%%%%%%%%%%%% 
%%%%%%%%PAGEBREAK%%%%%%%PAGEBREAK%%%%%%%%%
%%%%%%%%%%%%%%%%%%%%%%%%%%%%%%%%%%%%%%%%%% 
%%%%%%%%%%%%%%%%%%%%%%%%%%%%%%%%%%%%%%%%%% 
%%%%%%%%%%%%%%%%%%%%%%%%%%%%%%%%%%%%%%%%%% 
%%%%%%%%%%%%%%%%%%%%%%%%%%%%%%%%%%%%%%%%%% 
%%%%%%%%PAGEBREAK%%%%%%%PAGEBREAK%%%%%%%%%
%%%%%%%%%%%%%%%%%%%%%%%%%%%%%%%%%%%%%%%%%% 
%%%%%%%%%%%%%%%%PAGEBREAK%%%%%%%%%%%%%%%%%
%%%%%%%%%%%%%%%%%%%%%%%%%%%%%%%%%%%%%%%%%% 
%%%%%%%%PAGEBREAK%%%%%%%PAGEBREAK%%%%%%%%%
%%%%%%%%%%%%%%%%%%%%%%%%%%%%%%%%%%%%%%%%%% 
%%%%%%%%%%%%%%%%%%%%%%%%%%%%%%%%%%%%%%%%%%
%%%%%%%%%%%%%%%%%%%%%%%%%%%%%%%%%%%%%%%%%%
%%%%%%%%%%%%%%%%%%%%%%%%%%%%%%%%%%%%%%%%%%
%%%%%%%%PAGEBREAK%%%%%%%PAGEBREAK%%%%%%%%%
%%%%%%%%%%%%%%%%%%%%%%%%%%%%%%%%%%%%%%%%%%
%%%%%%%%%%%%%%%%PAGEBREAK%%%%%%%%%%%%%%%%%
%%%%%%%%%%%%%%%%%%%%%%%%%%%%%%%%%%%%%%%%%%
%%%%%%%%PAGEBREAK%%%%%%%PAGEBREAK%%%%%%%%%
%%%%%%%%%%%%%%%%%%%%%%%%%%%%%%%%%%%%%%%%%%
%%%%%%%%%%%%%%%%%%%%%%%%%%%%%%%%%%%%%%%%%%
%%%%%%%%%%%%%%%%%%%%%%%%%%%%%%%%%%%%%%%%%%
%%%%%%%%%%%%%%%%%%%%%%%%%%%%%%%%%%%%%%%%%%
%%%%%%%%PAGEBREAK%%%%%%%PAGEBREAK%%%%%%%%%
%%%%%%%%%%%%%%%%%%%%%%%%%%%%%%%%%%%%%%%%%%
%%%%%%%%%%%%%%%%PAGEBREAK%%%%%%%%%%%%%%%%%
%%%%%%%%%%%%%%%%%%%%%%%%%%%%%%%%%%%%%%%%%%
%%%%%%%%PAGEBREAK%%%%%%%PAGEBREAK%%%%%%%%%
%%%%%%%%%%%%%%%%%%%%%%%%%%%%%%%%%%%%%%%%%%
%%%%%%%%%%%%%%%%%%%%%%%%%%%%%%%%%%%%%%%%%%
%%%%%%%%%%%%%%%%%%%%%%%%%%%%%%%%%%%%%%%%%%
%%%%%%%%%%%%%%%%%%%%%%%%%%%%%%%%%%%%%%%%%%
%%%%%%%%PAGEBREAK%%%%%%%PAGEBREAK%%%%%%%%%
%%%%%%%%%%%%%%%%%%%%%%%%%%%%%%%%%%%%%%%%%%
%%%%%%%%%%%%%%%%PAGEBREAK%%%%%%%%%%%%%%%%%
%%%%%%%%%%%%%%%%%%%%%%%%%%%%%%%%%%%%%%%%%%
%%%%%%%%PAGEBREAK%%%%%%%PAGEBREAK%%%%%%%%%
%%%%%%%%%%%%%%%%%%%%%%%%%%%%%%%%%%%%%%%%%%
%%%%%%%%%%%%%%%%%%%%%%%%%%%%%%%%%%%%%%%%%%
\begin{alignment}[
  texts=edition[class="edition"];
  translation[class="translation"],
  ]
  \begin{edition}
    \ekddiv{
      head={[\uproman{21}. \textbf{jñānayogasya lakṣaṇam}]},
      type=section,
      depth=2, 
      n=XXI
    }
    \xmlhead[h21]{[XXI. jñānayogasya lakṣaṇam]}
    \label{jnanayogastart}
\begin{prose}[p21_01]
%------------------------------
%idānīṃ jñānayogasya lakṣaṇaṃ kathyate/ \E
%idānīṃ jñānayogasya lakṣaṇaṃ kathyate \P
%idānīṃ jñānayogasya lakṣaṇaṃ// \L 5976_0011.jpg 
%idānīṃ jñānayogasya lakṣaṇaṃ// \B
%idānīṃ jñānayogasya lakṣaṇaṃ// \N1 %%%%p.6 verso 
%idānīṃ jñānayogasya lakṣaṇaṃ// \D
%idānīṃ jñānayogasya lakṣaṇaṃ kathyate// \N2
%idānī  jñānayogasya lakṣaṇaṃ kathyate   \U1
%idānīṃ jñānayogasya lakṣaṇaṃ kathyate// \U2
%------------------------------
%Now the characteristic of Jñānayoga is explained. 
%-----------------------------
\note[type=source, labelb=133, nosep]{cf. YSv (PT p. 835): idānīṃ jñānayogasya lakṣaṇaṃ kathyate śive | yaj jñātvā jñānasampūrṇaḥ śivaḥ syān na punarbhavaḥ |}
\app{\lem[wit={ceteri}]{idānīṃ}
  \rdg[wit={U1}]{idānī}}
jñānayogasya lakṣaṇaṃ
\app{\lem[wit={E,P,N2,U1,U2}]{kathyate}
  \rdg[wit={B,D,L,N1}]{\om}}/
\end{prose}
%--------------------------------------
%ekam eva jagat paśyed viśvāva suvibhāsvaram/
%avikalpatayā yuktyā jñānayogaṃ samācaret//1// \E
%
%ekam eva cayat paśyed viśvātmāsuvibhāsvaram       
%avikalpatayā yuktyā jñānayogaṃ samācaret 1 \P
%
%ekam evā jagat paśyed viśvātmāsuvibhāsvaraṃ//
%avikalpatayā yuktā jñānayogaṃ samācaret// \L
%
%ekam evā jagat paśyad visvātmāsuvibhāsvaraṃ//
%avikalpatayā yuktā jñānayogaṃ samācaret// \B
%
%ekam eva jagat paśyed viśvātmā viśvabhāvanaḥ/
%iti kṛtvā tu vai yukto jñānayogaṃ samācaret// SVARODAYA
%
%ekam eva jagat paśyed dviśvātmāsuvibhāsvaraṃ/
%avikalpatayā yuktyā jñānayogaṃ samācaret//1// \N1
%
%ekam eva jagat paśyed dviśvātmāsuvibhāsvaraṃ//
%avikalpatayā yuktyā jñānayogaṃ samācaret//1// \D
%
%ekam eva jagat paśyed dviśvātmāsuvibhāsvaraṃ//
%avikalpatayā yuktyā jñānayogaṃ samācaret//1// \N2
%
%ekam eva jagataḥ paśyed dviśvātmāsuvibhāsvaraṃ
%āvikalpatayā yuktyā jñānayogaṃ samācaret//1// \U1
%
%ekam eva jagataḥ paśyed dviśvātmāsuvibhāsvaraṃ
%āvikalpatayā yuktyā jñānayogaṃ samācaret// \U2
%------------------------------
%He shall see the world als only 
%------------------------------
\begin{tlg}[21_1] 
  \noindent
  \tl{\note[type=source, labelb=134, labele=_134e, nosep]{ \approx  YSv (PT p. 835): ekam eva jagat paśyed viśvātmā viśvabhāvanaḥ | iti kṛtvā tu vai yukto jñānayogaṃ samācaret |}
eka\skp{m-e}\app{\lem[wit={ceteri}, alt={eva}]{\skm{m-e}va}
  \rdg[wit={B,L}]{evā}}
\app{\lem[wit={ceteri},alt={jagat}]{jaga\skp{t-pa}}
  \rdg[wit={P}]{cayat}
}\app{\lem[wit={ceteri},alt={paśyed}]{\skm{t-pa}śye\skp{d-vi}}
  \rdg[wit={B}]{paśyad}
}\app{\lem[wit={ceteri},alt={viśvātmā°}]{\skm{d-vi}śvātmā}
  \rdg[wit={E}]{viśvāva°}
}suvibhāsvaram/}\\
\tl{\app{\lem[wit={ceteri}]{avikalpatayā}
  \rdg[wit={U1,U2}]{āvikalpatayā}}
\app{\lem[wit={ceteri}]{yuktyā}
  \rdg[wit={B,L}]{yuktā}} 
jñānayogaṃ samācaret\dd{} \begin{otherlanguage}{english}\uproman{21}.1\end{otherlanguage} \dd{}}\linelabel{_134e}
\end{tlg}
%------------------------------
%yatra yatra sthito vāpi sarvajñānamayaṃ jagat/ 
%sa evaṃ vetti bodhena so pi jñānādhikāraṇāt//2// \E 
%
%yatra yatra sthito vāpi sarvajñānamayaṃ jagat  
%ya evaṃ vetti bodhena so pi jñānādhikāravān \P
%
%yatra yatra sthito vāpi sarvajñānamayaṃ jagat//  
%ya evaṃ vetti bodhena so pi jñānādhikāravān// \L
%
%yatra yatra sthito vāpi sarvajñānamayaṃ jagat//  
%ya evaṃ ve bodhena so pi jñānādhikāravān// \B
%
%yatra tatra sthito vāpi sarvajñānamayaṃ jagat/
%ya evam asti bodhena so'pi jñānādhikāravān/ \SVARODAYA
%
%yatra yatra sthito vāpi sarvajñānamayaṃ jagat/
%ya evaṃ vetti bodhena so pi jñānādhikāravān//2//\N1
%
%yatra yatra sthito vāpi sarvajñānamayaṃ jagat//
%ya evaṃ vetti bodhena so pi jñānādhikāravān//2//\D
%
%yatra yatra sthito vāpi sarvajñānamayaṃ jagat//
%ya evaṃ vetti bodhena so pi jñānādhikāravān//2//\N2
%
%yatra yatra sthito vāpi sarvajñānamayaṃ jagat  %%%273.jpg
%evaṃ vette na bodhena so pi jñānādhikāravān 2    \U1
%
%yatra yatra sthito hiṃsa sarvajñānamayaṃ jagat//  
%evaṃ vetti bodhena so pi jñānādhikāravān// 2    \U2
%------------------------------
%Wherever one dwells, the world is essentially (\textit{vāpi}) made of all knowledge. He who grasps this in this way, even possesses ultimate knowledge through [this] realisation.
%------------------------------
\begin{tlg}[21_2]
  \noindent
  \tl{\note[type=source, labelb=135, labele=_135e, nosep]{ \approx  YSv (PT p. 835): yatra tatra sthito vāpi sarvajñānamayaṃ jagat | ya evam asti bodhena so'pi jñānādhikāravān |}
    \note[type=source, labelb=135, labele=_135ey, nosep]{ \approx Cf. \textit{Netratantra} 8.55cd: yatra yatra sthito vāpi yena yena vratena vā |}
    yatra tatra sthito \app{\lem[wit={ceteri}]{vāpi}
      \rdg[wit={U2}]{hiṃsa°}} sarvajñānamayaṃ jagat/}\linelabel{_135ey}\\
  \tl{\app{\lem[wit={ceteri}]{ya evaṃ}
      \rdg[wit={U1,U2}]{evaṃ}}
    \app{\lem[wit={ceteri}]{vetti}
      \rdg[wit={U1}]{vette na}
      \rdg[wit={B}]{ve}} bodhena so'pi
    \app{\lem[wit={ceteri}]{jñānādhikāravān}
      %\rdg[wit={E}]{jñānādhikāraṇāt}}\dd{} \begin{otherlanguage}{english}\uproman{21}.2\end{otherlanguage}\hskip-2pt \dd{}}\linelabel{_135e}
      \rdg[wit={E}]{jñānādhikāraṇāt}}\dd{} \begin{otherlanguage}{english}\uproman{21}.2\end{otherlanguage} \dd{}}\linelabel{_135e}
\end{tlg}
%------------------------------
%
%\om!!!!!                                                                                                        \E
%
%prāpnoti śāmbhavīmantrān  sadā nityaparāyaṇaḥ/   yathā nyagrodhavījaṃ hi kṣitau   vaptur drumāyate/               \SVARODAYA  
%prāpnoti śāmbhavīṃ sattāṃ sadāṃdvaitaparāyaṇaḥ   yathā nyagrodhabījaṃ hi kṣitāv   uptaṃ drumāyate likāṃ pa..vāḥ 4 \P  7640.jpg last line check word!!!
%prāpnoti śāmbhavīṃ sattān sadādvaitaparāyaṇaḥ//  yathā nyagrodhavīja  hi kṣitāv   utpadyate yathā//               \L
%prāpnoti śāmbhaviṃ sattāṃ sadādvaitaparāyaṇaḥ//  yathā nyagrodhabījāṃ hi kṣitī    utpadyate//                      \B
%prāpnoti sāṃbhavīṃ satta  sadādvaitaparāyaṇaḥ//  yathā nyagrodhavījaṃ hi kṣitāv   uptaṃ drumāyate 3//              \N1
%prāpnoti sāṃbhavīsattāṃ   sadādvaitaparāyaṇaḥ//  yathā nyagrodhavījaṃ hi kṣitāv   uptaṃ drumāyate//                \D
%prāpnoti sāṃbhavīsattā    sadādvaitaparāyaṇaḥ//  yathā nyagrodhavījaṃ hi kṣitāv   uptaṃ drumāyate//                \N2 %drumaayate=denom. wie ein beim  sein 
%prāpnoti sāṃbhavīsattāṃ   sadādvaitaparāyaṇaḥ    yathā nyagrodhabījaṃ hi kṣitāptā ukta drumāyate 3              \U1
%prāpnoti sāṃbhavīsattāṃ   yadādvaitaparāyaṇaḥ//  yathā nyagrodhabījaṃ hi kṣitāv   uptaṃ drumāyate//               \U2
%------------------------------
%He always attains the reality of śāmbhavī - the supreme goal of non-duality.  
%Just as the seed of the Nyagrodha scattered onto the soil [always] becomes a tree.
%------------------------------
\begin{tlg}[21_3]
  \noindent
  \tl{\note[type=source, labelb=136, labele=_136e, nosep]{ \approx  YSv (PT p. 835): prāpnoti śāmbhavīmantrān sadā nityaparāyaṇaḥ | yathā nyagrodhavījaṃ hi kṣitau vaptur drumāyate |}
    \app{\lem[wit={ceteri}]{prāpnoti}
      \rdg[wit={E}]{\om}}
  \app{\lem[type=emendation, resp=egoscr]{śāṃbhavīsattāṃ}
      \rdg[wit={D,U1,U2}]{sāṃbhavīsattāṃ}
      \rdg[wit={B,P}]{śāmbhavīṃ sattāṃ}
      \rdg[wit={L}]{śāmbhavīṃ sattān}
      \rdg[wit={N1}]{sāṃbhavīṃ satta}
      \rdg[wit={N2}]{sāṃbhavīsattā}
      \rdg[wit={E}]{\om}}
    \app{\lem[wit={ceteri},alt={sadādvaita°}]{sadādvaita}
      \rdg[wit={U1}]{sadāṃdvaita°}
      \rdg[wit={E}]{\om}}parāyaṇaḥ/}\\
  \tl{\app{\lem[wit={ceteri}]{yathā}
      \rdg[wit={E}]{\om}}
    \app{\lem[wit={ceteri}]{nyagrodhabījaṃ}
      \rdg[wit={D,N1,N2}]{nyagrodhavījaṃ}
      \rdg[wit={L}]{nyagrodhavīja}
      \rdg[wit={E}]{\om}}
    \app{\lem[wit={ceteri}]{hi}
      \rdg[wit={E}]{\om}}
    \app{\lem[wit={ceteri},alt={kṣitāv}]{kṣitā\skp{v-u}}
      \rdg[wit={B}]{kṣitī}
      \rdg[wit={U1}]{kṣitāptā}
      \rdg[wit={E}]{\om}
 }\app{\lem[wit={ceteri},alt={uptaṃ drumāyate}]{\skm{v-u}ptaṃ drumāyate}
      \rdg[wit={P}]{uptaṃ drumāyate likāṃ pa..vāḥ}
      \rdg[wit={L}]{utpadyate yathā}
      \rdg[wit={B}]{utpadyate}
      \rdg[wit={U1}]{ukta drumāyate}
      \rdg[wit={E}]{\om}}\dd{} \begin{otherlanguage}{english}\uproman{21}.3\end{otherlanguage} \dd{}}\linelabel{_136e}
\end{tlg}
%------------------------------
%ekāntaṃ  naikadā  svena   dṛśyate  daśadhā  kṛtaḥ/  mūlāṅkurasya  coddaṇḍāḥ śākhākuṇḍalapallavāḥ//3//   \E cod?v%on cud? Wurzel in guṇa + daṇḍa? !!! em. zu śaśvadhā = immer wieder, jederzeit 
%\om                                                                                                    \P
%ekāṃte   nekadhā  svena   dṛśyaṃte daśadhāt kṛp?tā/ mūlāṃkurutva kudaṃḍaḥ  śākhākilekālapallavā        \B
%ekāṃte   nekadhā  svena   dṛśyaṃte daśadhāt kṛtaḥ/  mūlāṃkurutva kudaṃḍa   śākhākalikālapallavā        \L
%ekāṃtaṃ  naikadhā śveta   dṛśyate  daśadhā  kṛtā//  mūlāṃkurutva codaṃḍaḥ  śāvārakumbhalapallavaḥ//4// \N1   
%ekāṃtaṃ  naikadhā śvetana dṛśyate  daśadhā  kṛtā//  mūlāṃkurutva codarāṭaḥ śālavākumapadṛtravā//4//    \D
%ekāṃtaṃ  naikadhā śvetana dṛśyet   śadhā    kṛtā//  mūlāṃkurutva codarāṭaḥ śākhākumbhalapallavā//4//   \N2
%yekāṃtaṃ naikadhā svena   dṛśyate  śadhā    kṛtā    mūlāṃkurutva codaṃḍa   śākhākumbhalapallavaḥ       \U1
%ekāṃtaṃ  naikadhā svetana dṛśyate  daśadhā  kṛtiḥ// mūlāṃkurutva codaṃḍaḥ  śākhākusumapallavāḥ//       \U2
%------------------------------
\begin{tlg}[21_4]
  \noindent
  \tl{\note[type=source, labelb=137, labele=_137e, nosep]{ \approx  YSv (PT p. 835): ādāv ekas tato 'nekaḥ svabhāvāc chādanādibhiḥ | varddhate 'harniśaṃ vṛkṣaḥ patrapallavavistṛtaḥ|}
    \note[type=philcomm, labelb=137a, labele=_137e, lem={ekāntaṃ \ldots pallavāḥ}]{The verse \uproman{23}.4 is omitted in \getsiglum{P}.}
 \app{\lem[wit={ceteri}]{ekāntaṃ}
  \rdg[wit={B,L}]{ekānte}
  \rdg[wit={U1}]{yekāṃtaṃ}}
\app{\lem[wit={ceteri}]{naikadhā}
  \rdg[wit={E}]{naikadā}
  \rdg[wit={B,L}]{nekadhā}}
\app{\lem[wit={ceteri}]{svena}
  \rdg[wit={N1}]{śveta}
  \rdg[wit={D,N2}]{śvetana}}
\app{\lem[wit={ceteri}]{dṛśyate}
  \rdg[wit={B,L}]{dṛśyaṃte}
  \rdg[wit={N2}]{dṛśyet}}
\app{\lem[wit={B,L}]{daśadhāt}
  \rdg[wit={E,N1,N2}]{daśadhā}    %%[type=conjecture, resp=egoscr]{śaśvadhā}????
  \rdg[wit={N2,U1}]{śadhā}}
\app{\lem[type=emendation, resp=egoscr]{kṛtāt}
  \rdg[wit={E,L}]{kṛtaḥ}
  \rdg[wit={X}]{kṛtā}
  \rdg[wit={B}]{kṛptā}
  \rdg[wit={U2}]{kṛtiḥ}}/}\\
 \tl{\app{\lem[wit={E}]{mūlāṅkurasya}
  \rdg[wit={ceteri}]{mūlāṃkurutva}}
\app{\lem[wit={E,N1,U2}]{coddaṇḍāḥ}
  \rdg[wit={D,N2}]{codarāṭaḥ}
  \rdg[wit={B}]{kudaṃjaḥ}
  \rdg[wit={L}]{kudaṃḍa}}
\app{\lem[wit={U2}]{śākhākusumapallavāḥ}
  \rdg[wit={E}]{śākhākuṇḍalapallavāḥ}
  \rdg[wit={B,L}]{śākhākilekālapallavā}
  \rdg[wit={N1,U1}]{śāvārakumbhalapallavaḥ}
  \rdg[wit={N2}]{śākhākumbhalapallavā}
  \rdg[wit={D}]{śālavākumapadṛtravā}}\dd{} \begin{otherlanguage}{english}\uproman{21}.4\end{otherlanguage} \dd{}}\linelabel{_137e}
\end{tlg}
%\flushpage
  \end{edition}
  \begin{translation}
    \ekddiv{
      head={[\uproman{21}. \textbf{The Characteristic of Jñānayoga}]},
      type=section,
      depth=2, 
      n=XXI.1
    }
    \xmlhead[h21]{[XXI. The Characteristic of Jñānayoga]}
    \label{jnanayogatrans}
     \begin{tlate}[p21_01]
       \noindent
       \begin{euber}[f14_1]\blfootnote{\hspace{-2.2em}in \citetitle{birch2013} 2.7-8. (\textit{cittaṃ buddhir ahaṅkāra ṛtvijaḥ somapaṃ manaḥ} | \textit{indriyāṇi daśa prāṇāñ juhoti jyotimaṇḍale} || 7 || \textit{ā mūlād bilaparyantaṃ vibhāti jyotimaṇḍalam} | \textit{yogibhiḥ satataṃ dhyeyam aṇimādyaṣṭasiddhidam} || 8 ||). These verses precede or introduce \textit{śāmbhavī mudrā}. Here, thought, intellect and ego are taught the be the officiants, whereas the mind is the sacrificer who sacrifices the senses and the ten vital breaths into the orb of light (2.7). The orb of light (\textit{jyotimaṇḍala}) shines from the root (possibly the root of the body or spine, but \citeauthor[2013:286]{birch2013} suggests the palate) to the aperture at the top of the head. Yoga practitioners should constantly meditate on it to achieve \textit{siddhi}s (2.8).}\end{euber} Now, the characteristic of Jñānayoga is explained.
     \end{tlate}
     \begin{tlate}[21_1]
       \paragraph{\uproman{21}.1} He shall see the world as only one, illumined by the supreme self. By the method of non-dualistic thinking, he shall accomplish \textit{Jñānayoga}.
     \end{tlate}
     \begin{tlate}[21_2]
       \paragraph{\uproman{21}.2} Wherever one dwells, the world is essentially (\textit{vāpi}) made of all knowledge. He who grasps this in this way, even possesses authority with regard to knowledge through [this] realisation. %%%Cf. NT 8.55cd, vāpi = einfach vā + api = vā (im Prinzip metrischer Füller)
     \end{tlate}
     \begin{tlate}[21_3]
       \paragraph{\uproman{21}.3} The one who is wholly devoted to non-duality always attains the reality of Śāmbhavī\footnote{Rāmacandra uses the term \textit{śāmbhavī} as a designation of the ultimate state to be attained by practising Jñānayoga, which he presents as the realization of absolute unity. The term \textit{śāmbhavī} has its roots in more ancient tantric traditions of Śaivism and refers to an exalted state associated with Śiva a Yogin attains trough various practices. In medieval Yogatexts, particular in the Rājayoga genre, the term \textit{śāmbhavī} most often appears in the context of a non-physical \textit{mudrā}, the so-called \textit{śāṃbhavī mudrā}. The two earliest references for \textit{śāṃbhavī mudrā} are \textit{Candrāvalokana} 1 = \textit{Amanaska} 2.10, who share the same verse. The practice of \textit{śāṃbhavī mudrā} involves focusing the mind at an internal orb of light \citetitle{birch2013} 2.7-8. At the same time, the gaze is directed outwards without closing and opening the eyes \citetitle{birch2013} (2.10). In \citetitle{birch2013} 2.14, the practice is said to bring about \textit{siddhi}s and the no-mind-state (\textit{unmani}) and according to \citetitle{birch2013} 2.14 liberation while alive (\textit{jiivanmukti}). For a detailed discussion of \textit{śāṃbhavī mudrā}, its influence and all references, see \citeauthor[2013:71-79]{birch2013}.}, just as the seed of the banyan tree\footnote{In rituals the banyan tree (\textit{nygarodha}) is associated with the \textit{kṣatriya} class (\citeauthor[1998:27]{smith1998}).} scattered onto the ground [always] becomes a tree.
     \end{tlate}
     \begin{tlate}[21_4]
\paragraph{\uproman{21}.4} The absolute unity (\textit{ekāntaṃ}) is perceived as not uniform by oneself because of being made from ten parts - [like i.e.] the stems, branches, buds and leaves of the original shoot.
\flushpage 
\end{tlate}
\end{translation}
\end{alignment}
\pagebreak %after pp. 41-42
%%%%%%%%%%%%%%%%%%%%%%%%%%%%%%%%%%%%%%%%%% 
%%%%%%%%%%%%%%%%%%%%%%%%%%%%%%%%%%%%%%%%%% 
%%%%%%%%PAGEBREAK%%%%%%%PAGEBREAK%%%%%%%%%
%%%%%%%%%%%%%%%%%%%%%%%%%%%%%%%%%%%%%%%%%% 
%%%%%%%%%%%%%%%%PAGEBREAK%%%%%%%%%%%%%%%%%
%%%%%%%%%%%%%%%%%%%%%%%%%%%%%%%%%%%%%%%%%% 
%%%%%%%%PAGEBREAK%%%%%%%PAGEBREAK%%%%%%%%%
%%%%%%%%%%%%%%%%%%%%%%%%%%%%%%%%%%%%%%%%%% 
%%%%%%%%%%%%%%%%%%%%%%%%%%%%%%%%%%%%%%%%%% 
%%%%%%%%%%%%%%%%%%%%%%%%%%%%%%%%%%%%%%%%%% 
%%%%%%%%%%%%%%%%%%%%%%%%%%%%%%%%%%%%%%%%%% 
%%%%%%%%PAGEBREAK%%%%%%%PAGEBREAK%%%%%%%%%
%%%%%%%%%%%%%%%%%%%%%%%%%%%%%%%%%%%%%%%%%% 
%%%%%%%%%%%%%%%%PAGEBREAK%%%%%%%%%%%%%%%%%
%%%%%%%%%%%%%%%%%%%%%%%%%%%%%%%%%%%%%%%%%% 
%%%%%%%%PAGEBREAK%%%%%%%PAGEBREAK%%%%%%%%%
%%%%%%%%%%%%%%%%%%%%%%%%%%%%%%%%%%%%%%%%%% 
%%%%%%%%%%%%%%%%%%%%%%%%%%%%%%%%%%%%%%%%%% 
%%%%%%%%%%%%%%%%%%%%%%%%%%%%%%%%%%%%%%%%%% 
%%%%%%%%%%%%%%%%%%%%%%%%%%%%%%%%%%%%%%%%%% 
%%%%%%%%PAGEBREAK%%%%%%%PAGEBREAK%%%%%%%%%
%%%%%%%%%%%%%%%%%%%%%%%%%%%%%%%%%%%%%%%%%% 
%%%%%%%%%%%%%%%%PAGEBREAK%%%%%%%%%%%%%%%%%
%%%%%%%%%%%%%%%%%%%%%%%%%%%%%%%%%%%%%%%%%% 
%%%%%%%%PAGEBREAK%%%%%%%PAGEBREAK%%%%%%%%%
%%%%%%%%%%%%%%%%%%%%%%%%%%%%%%%%%%%%%%%%%% 
%%%%%%%%%%%%%%%%%%%%%%%%%%%%%%%%%%%%%%%%%%
\begin{alignment}[
  texts=edition[class="edition"];
  translation[class="translation"],
  ]
  \begin{edition}
%------------------------------
%srehapuṇyaphalaṃ   bīje vistaro yaṃ svabhāvataḥ/  tathāsau   nirmalo  nityo nirvikāro niraṃjanaḥ//4// \E
%snehapuṣpaphalaṃ   bīje vistāro yaṃ svabhāvataḥ   tāthāpasau nirmalau nityo nirvikāro niraṃjanaḥ     \P   %%7641.jpg Z.1
%snehe puṣpaphala---bīja-vistāro ya  svabhāvatāḥ   yāthāsau   nirmalo  nityo nirvikāro niraṃjanaḥ//    \B
%snehe puṣpaphala---bīja-vistāro ya  svabhāvatāḥ// tāthāsau   nirmalo  nityo nirvikāro niraṃjanaḥ//    \L
%snehapuṣpaphalaṃ   bīje vistārā yaṃ svabhāvataḥ/  tathāsau   nirmalo  nityo nirvikāro niraṃjanaḥ//5// \N1
%snehapuṣpaphalaṃ   bīje vistārā yasya  bhāvataḥ// tathāsau   nirmalo  nityo nirvikāro niraṃjanaḥ//5// \D
%snehapuṣpaphalaṃ   vīje vistāro yaṃ svabhāvataḥ// tathāsau   nirmalo  nityo nirvikāro niraṃjanaḥ//5// \N2
%snehapuṣpaṃ phalaṃ bīje vistāro yaḥ svabhāvataḥ   tathāsau   nirmalo  nityo nirvikāro niraṃjanaḥ 5 \U1  %%%%274.jpg
%snehapuṣpaphalaṃ   bīje vistāro yaṃ svabhāvataḥ// tathāsau   nirmalo  nityo nirvikāro niraṃjanaḥ// 5  \U2 %%%first Śloka in this series that is numbered in U2 
%------------------------------
%Aufgrund seines inhärenten Wesens ist dieser Ast mit seinen Zweigen, welcher die Frucht der Blüte der Liebe ist, im Samen.
%Gewiss, ist jenes rein, ewig, unveränderlich und makellos. 
%------------------------------
%By virtue of its inherent nature, this branch with its branches, which is the fruit of the flower of love, is in the seed.
%Certainly, that is pure, eternal, unchanging and immaculate.
%------------------------------
    \begin{tlg}[21_5]
\noindent
      \tl{\note[type=source, labelb=138, labele=_138e, nosep]{ \approx  YSv (PT p. 836): snehapuṣpaphalair vījair vistāro 'yaṃ svabhāvataḥ | tathāsau nirmalo nityo nirvikāro nirañjanaḥ |}
    \app{\lem[wit={D,N1,N2,P,U2}]{snehapuṣpaphalaṃ}
  \rdg[wit={B,L}]{snehe puṣpaphala°}
  \rdg[wit={U1}]{snehapuṣpaṃ phala}
  \rdg[wit={E}]{srehapuṇyaphalaṃ}}
\app{\lem[wit={ceteri}]{bīje}
  \rdg[wit={B,L}]{bīja}}
\app{\lem[wit={ceteri}]{vistāro}
  \rdg[wit={D,N1}]{vistārā}
}\app{\lem[wit={E,P,N1,N2,U2}]{'yaṃ}
  \rdg[wit={B,L}]{ya}
  \rdg[wit={U1}]{yaḥ}
  \rdg[wit={D}]{yasya}}
\app{\lem[wit={ceteri}]{svabhāvataḥ}
  \rdg[wit={B,L}]{svabhāvatāḥ}
  \rdg[wit={D}]{bhāvataḥ}}/}\\
\tl{\app{\lem[wit={ceteri}]{tathāsau}
    \rdg[wit={B}]{yathāsau}
    \rdg[wit={P}]{tathāpasau}}
  \app{\lem[wit={ceteri}]{nirmalo}
    \rdg[wit={P}]{nirmalau}}
nityo nirvikāro nirañjanaḥ\dd{} \begin{otherlanguage}{english}\uproman{21}.5\end{otherlanguage} \dd{}} \linelabel{_138e}
\end{tlg}
%------------------------------
%eko  nekaḥ  svayaṃbhūś ca dhāmnā ca    bahudhā sthitaḥ/   paṃcatattvamanobuddhi-māyāhaṃkāravikriyāḥ //5//   \E
%eko  nekaḥ  svayaṃbhūś ca svadhāmnā    bahudhā sthitāḥ    paṃcatatvamanobuddhir māyāhaṃkāravikriyāḥ   6     \P
%eko  neka   svayaṃbhūś ca dhāmnāya     bahudhā sthitaḥ//  paṃcatatvamanobuddhi--māyāhaṃkāravikriyā  //      \B
%eko  nekaḥ  svayaṃbhūś ca svadhābhāva  bahudhā sthitāḥ//  paṃcatatvamanobuddhi--māyāhaṃkāravikriyā  //      \L
%eko  nekaḥ  svayaṃbhuś ca svayāṃmnā    bahudhā sthitaḥ/   paṃcatatvamanobuddhir māyāhaṃkāravikriyā  //6//   \N1
%eko  nekaḥ  svayaṃbhaś ca svadhā...ṣ   bahudhā sthitāḥ//  paṃcatatvamanobuddhir māyāhaṃkāravikriyā  //6//   \D
%eko  neka   svayaṃbhūś ca svadhāmnāva  bahudhā sthitaḥ//  paṃcatatvamanobuddhir māyāhaṃkāravikriyā  //6//   \N2
%yeko naika/ svayaṃbhūtyā  svabhāvā     bahudhā sthitaḥ    paṃcatatvamanobuddhir māyāhaṃkāravikriyāḥ   6     \U1
%eko  naiko  svayaṃbhūś ca svadhāmnā    bahudhā sthitaḥ//  paṃcatatvamanobuddhir māyāhaṃkāravikriyā  //6//   \U2
%------------------------------
%One, not one, self-existing by it's own power, abiding in multiplicity, as five [gross] elements (\textit{tattva}), thinking mind (\textit{manas}), intellect (\textit{buddhi}), illusion (\textit{māya}), individuation (\textit{ahaṃkāra}), and modifications (\textit{vikriyā}). 
%------------------------------
\begin{tlg}[21_6]
  \noindent
\note[type=source, labelb=139, labele=_139e, nosep]{ \approx  YSv (PT p. 836): eko 'nekaḥ khayaṃ bhūyān sādhanād bahudhā sthitaḥ | pañcatattvamayo buddhimāyāhaṅkāravikriyaḥ |}
  \tl{
\app{\lem[wit={ceteri}]{eko}
  \rdg[wit={U1}]{yeko}}
\app{\lem[type=emendation, resp=egoscr]{naikaḥ}
   \rdg[wit={ceteri}]{nekaḥ}
   \rdg[wit={U1}]{naika}
  \rdg[wit={U2}]{naiko}
  \rdg[wit={B,N2}]{neka}}
\app{\lem[wit={ceteri},alt={svayaṃbhūś ca}]{svayaṃbhūś\skp{-}ca}
  \rdg[wit={U1}]{svayaṃbhūtyā}}
\app{\lem[wit={P,U2}]{svadhāmnā}
  \rdg[wit={E}]{dhāmnā ca}
  \rdg[wit={B}]{dhāmnāya}
  \rdg[wit={L}]{svadhābhāva}
  \rdg[wit={N1}]{svayāṃmnā}
  \rdg[wit={D}]{svadhā..ṣa}
  \rdg[wit={N2}]{svadhāmnāva}
  \rdg[wit={U1}]{svabhāvā}}
bahudhā
\app{\lem[wit={D,L,P}]{sthitāḥ}
  \rdg[wit={ceteri}]{sthitaḥ}}/}\\
\tl{pañcatattvamano\app{\lem[wit={E,P,L},alt={°buddhi°}]{buddhi}
    \rdg[wit={ceteri}]{°buddhir}
  }māyāhaṃkāra\app{\lem[wit={E,P,U1},alt={°vikriyāḥ}]{vikriyāḥ}
    \rdg[wit={ceteri},alt={°vikriyā}]{vikriyā}}\dd{} \begin{otherlanguage}{english}\uproman{21}.6\end{otherlanguage} \dd{}} \linelabel{_139e}
\end{tlg}
%------------------------------ 
%evaṃ daśavidhaṃ viśvaṃ lokālokasavistaram/   eka  eva na cānyo sti yo jānāti sa tattvavit//6// \E
%evaṃ daśavidhaṃ viśvaṃ lokālokasavistaraṃ    eka  eva na cānyo sti yo jānāti sa tatvavit 6 \P
%evaṃ daśavidhā  viśvaṃ lokālokasavistaraṃ//  eka  eva na cānyā sti yo jānāti sa tatvavit// \B
%evaṃ daśavidhā  viśvaṃ lokālokasavistaraṃ//  eka  eva na cānyo sti yo jānāti sa tatvavit// \L
%evaṃ daśavidhaṃ viśvaṃ lokālokasavistarāṃ/   eka  eva na cānyo sti yo nānāti sa tatvavit//7// \N1
%evaṃ daśavidhaṃ viśvaṃ lokālokasavistaraṃ//  eka  eva na cānyo sti yo jānāti sa tatvavit//7// \D
%evaṃ daśavidhā  viśvaṃ lokālokasavistaraṃ//  eka  eva na cānyo sti yo jānāti sa tatvavit//7// \N2
%evaṃ daśavidha--viśvaṃ lokālokasavistaraṃ    eka yeva na cānyo sti yo jānāti sa tatvavit 7 \U1
%evaṃ daśavidhaṃ viśvaṃ lokāloke savistaraṃ// ekam eva na cānyo sti yo jānāti sa tatvavit//7// \U2 %%%409.jpg 
%------------------------------
%Auf diese Weise durchdringen die zehn Variationen die Welt und die Nicht-Welt im vollen Umfang.  
%Nur das Eine ist und nicht etwas anderes: Wer das weiß ist ein Kenner der Realität.  
%------------------------------
%In this way, the ten variations fully permeate the world and the non-world.
%Only one thing is and not something else: Whoever knows this is a connoisseur of reality.
%------------------------------
\begin{tlg}[21_7]
  \noindent
\note[type=source, labelb=140, labele=_140e, nosep]{ \approx  YSv (PT p. 836): evaṃ bahuvidhaṃ viśvaṃ lokālokasuvistaram | ekam eva na cānvo 'sti yo jānāti sa tattvavit |}
   \tl{
     evaṃ
     \app{\lem[wit={D,E,P,N1,U2}]{daśavidhaṃ viśvaṃ}
       \rdg[wit={B,L,N2}]{daśavidhā viśvaṃ}
       \rdg[wit={U1}]{daśavidhaviśvaṃ}}
     \app{\lem[type=emendation, resp=egoscr]{lokālokasuvistaram}
       \rdg[wit={B,D,E,L,P,N2,U1}]{lokālokasavistaram}  
       \rdg[wit={N1}]{lokālokasavistarāṃ}
       \rdg[wit={U2}]{lokāloke savistaraṃ}}/}\\
   \tl{\app{\lem[wit={ceteri}]{eka}
       \rdg[wit={U2}]{ekam}}
       \app{\lem[wit={ceteri}]{eva}
         \rdg[wit={U1}]{yeva}}
       na cānyo'sti yo jānāti sa tattvavit\dd{} \begin{otherlanguage}{english}\uproman{21}.7\end{otherlanguage} \dd{}} \linelabel{_140e}
\vspace{5mm} %5mm vertical space 
   \end{tlg}
    \begin{prose}[p21_02]     
%------------------------------
%pṛthvīvanaspatiparvatādisthārarūpaḥ         saṃsāra---manuṣyahastyaśvapakṣītyādiko    jaṃgamarūpaḥ   saṃsāraḥ// \E
%pṛthvīvanaśpatiparvatādisthāvararūpaḥ       saṃsāraḥ  manuṣyahastyaś ca pakṣītyādiko  jaṃgamarūpaḥ   saṃsāraḥ \P
%pṛthvīvanaspatīparvatādisthāvararūpā        saṃsāraḥ/ manuṣyahasteśvapakṣītyādiko     jaṃgamarūpaḥ   saṃsāraḥ// \B
%pṛthvīvanaspatiparvatādisthāvararūpā        saṃsāraḥ  manuṣyahasteśvapakṣītyādiko     jaṃgamarūpā    saṃsāraḥ// \L
%pṛthvīvanaspatīparvvate tyādisthāvararūpaḥ  saṃsāraḥ  manuṣyahastīaśvapakṣītyādiko    jaṃgamarūpaḥ   saṃsāraḥ// \N1
%pṛthvīvanaspatīparvato tyādisthāṃvararūpaḥ  saṃsāraḥ  manuṣyahastīaśvapakṣītyādiko    jaṃgamaḥ rūpaḥ saṃsāraḥ// \D
%pṛthvīvanaspatiparvate 'thyādisthāvararūpa  saṃsāraḥ  manuṣyahastipakṣītyādiko        jaṃgamarūpaḥ   saṃsāraḥ// \N2
%pṛthivīvanaspatīparvate iyādisthāvararūpaḥ  saṃsāra---manuṣyahastiasvapakṣītyādiko    jagadrūpaḥ     saṃsāro \U1
%pṛthvīvanaspatiparvatādisthāvararūpaḥ       saṃsāraḥ//manuṣyahasttyaś ca pakṣītyādiko jaṃgamarūpaḥ   saṃsāraḥ//8// \U2
%------------------------------
%The mundane existance (\textit{saṃsāra}) exists in the form of stationary [existances] such as earth, tree mountains and so on. The mundane existance (\textit{saṃsāra}) [also] exists in the form of the mobile [existances] such as humans, animals, birds and so on.
%------------------------------
\note[type=source, labelb=141, labele=_141e, nosep]{cf. YSv (PT p. 832): sthāvarāḥ parvatādyā hi jaṅgamāḥ khecarādayaḥ | jaṅgamasthāvarākāraḥ saṃsāraḥ syāt sa īśvaraḥ |}
\app{\lem[wit={ceteri},alt={pṛthvī°}]{pṛthvī}
        \rdg[wit={U1}]{pṛthivī°}
      }\app{\lem[wit={E,N2,U2},alt={°vanaspati°}]{vanaspati}
        \rdg[wit={P}]{vanaśpati}
        \rdg[wit={B,D,L,N1,U1}]{vanaspatī°}
      }\app{\lem[wit={B,L,P,U2}, alt={°parvatādisthāvara°}]{parvatādisthāvara}
        \rdg[wit={E}]{°parvatādisthāra°}
        \rdg[wit={D}]{°parvato tyādisthāṃvara°}
        \rdg[wit={N1}]{°parvvate tyādisthāvara°}
        \rdg[wit={N2}]{°parvate 'thyādisthāvara°}
        \rdg[wit={U1}]{°parvate iyādisthāvara°}
      }\app{\lem[wit={ceteri}]{rūpaḥ}
        \rdg[wit={B,L}]{rūpā}
        \rdg[wit={N2}]{rūpa}}
      \app{\lem[wit={ceteri}]{saṃsāraḥ}
        \rdg[wit={E,U1}]{saṃsāra°}}/
      manuṣya\app{\lem[wit={B,L},alt={°hasteśvapakṣīty ādiko}]{hasteśvapakṣīty\skp{-}ādiko}
          \rdg[wit={E}]{°hasty aśvapakṣīty ādiko}
          \rdg[wit={D,N1}]{°hastīaśvapakṣīty ādiko}
          \rdg[wit={N2}]{°hastipakṣīty ādiko}
          \rdg[wit={U1}]{°hastiasvapakṣīty ādiko}
          \rdg[wit={U2}]{°hasttyaś ca pakṣīty ādiko}}
        \app{\lem[wit={ceteri}]{jaṃgamarūpaḥ}
          \rdg[wit={D}]{jaṃgamaḥ rūpaḥ}
          \rdg[wit={L}]{°rūpā}
          \rdg[wit={U1}]{jagad°}}
        \app{\lem[wit={ceteri}]{saṃsāraḥ}
          \rdg[wit={U1}]{saṃsāro}}/\linelabel{_141e}
%------------
%atha ca   yo  dṛṣṭiviṣayaḥ  sa dṛśya  ucyate/  yo dṛṣṭyā na vīkṣyate sa adṛśya ity  ucyate/ \E
%atha ca   yo  dṛṣṭiviṣayaḥ  sa dṛśya  ucyate   yo dṛṣṭyā na vīkṣyate sa adṛśya ity  ucyate  %%%7641.jog
%atha ca// yo  daṣṭiviṣayaḥ  sa dṛśya  ucyate// yo dṛṣṭyā na vīkṣyate sa adṛśya ty   ucyate// \B
%atha ca   yo ddṛṣṭiviṣayaḥ  sa dṛśya  ucyate// yo dṛṣṭyā na vīkṣyate sa adṛśye ty   ucyate... \L
%atha ca   ya ddṛṣṭiviṣayaḥ  sa dṛśyad ucyate   yo dṛṣṭyā na vīkṣyate sa adṛśya ity  ucyate// \N1
%atha vā   ya dārṣṭiviṣayaḥ  sa dṛśya  ucyate/  yo dṛṣṭyā na vīkṣyate sa adṛśya ity  ucyate// \D
%atha ca   ya  drṣṭiviṣayaḥ  sa dṛśya  ucyate/  yo dyā    na vīkṣyate sa adṛśya śaty ucyate/ \N2
%atha ca   yaḥ drṣṭiviṣayaḥ  sa dṛśy---ucyate   yo dṛṣṭvā na vīkṣyate sa adṛśya ity  ucyate \U1
%atha ca   yo  dṛṣṭiviṣayaḥ  sa dṛśya  ucyate// yo dṛṣṭyā na vīkṣyate sa adṛśya ity  ucyate// \U2
%------------------------------
%Now, that which is the object of sight is called the seen. That which is not seen by sight is called the unseen.
%------------------------------
\note[type=source, labelb=142, labele=_142e, nosep]{cf. YSv (PT p. 836): svabhāvalīlayā bhāti śūnye 'sau śūnyabuddhitaḥ | yad dṛṣṭaṃ viṣayaṃ vastu tad dṛśyam iti kathyate | yo dṛṣṭātītaḥ so 'dṛśyas tadā dṛṣṭaṃ hi manyate | svatanūbhedam evan tu saṃsāraṃ duḥkhasaṅkulam |}
atha
      \app{\lem[wit={ceteri}]{ca}
        \rdg[wit={D}]{vā}}
      \app{\lem[wit={ceteri}]{yo}
        \rdg[wit={U1}]{yaḥ}
        \rdg[wit={D,N1,N2}]{ya}}
      \app{\lem[wit={ceteri}]{dṛṣṭi}
        \rdg[wit={L,N1}]{ddṛṣṭi}
        \rdg[wit={B}]{daṣṭi}
        \rdg[wit={D}]{dārṣṭi}
}viṣayaḥ sa
\app{\lem[wit={ceteri}]{dṛśya}
  \rdg[wit={N1}]{dṛśyad}
  \rdg[wit={U1}]{dṛṣy°}}
ucyate/
yo
\app{\lem[wit={ceteri}]{dṛṣṭyā}
  \rdg[wit={N2}]{dyā}}
na vīkṣyate sa adṛṣya
\app{\lem[wit={ceteri},alt={ity}]{i\skp{ty-u}}
  \rdg[wit={B,L}]{ty}
  \rdg[wit={N2}]{śaty}
}\skm{ty-u}cyate/
%------------------------------
%evaṃ saṃsārasya svātmano  bhedaṃ dūrīkṛty---aikam eva darśanaṃ sa eva jñānayogaḥ/   \E
%evaṃ saṃsāra----svātmano  bhedaṃ dūrīkṛtya  aikyena   darśanaṃ        jñānayogaḥ    \P
%evaṃ saṃsārasya svātmano  bheda--dūrīkṛtya  aikyona   darśanaṃ        jñānayogaḥ/   \B
%evaṃ saṃsāra----svātmano  bhedaṃ dūrīkṛtya  aikyona   darśanaṃ        jñānayogaḥ... \L
%evaṃ saṃsārasya svātmanaḥ bhedāṃ dūrīkṛtya  ekyena    darśanaṃ        jñānayogaḥ//  \N1
%evaṃ saṃsārasya svātmanaḥ bhedāṃ dūrīkṛtya  ekyena    darśanaṃ        jñānayogaḥ/   \D
%evaṃ saṃsārasya svātmanaḥ bhedaṃ dūrīkṛtya  ekena     darśanaṃ        jñānayogaḥ/   \N2
%evaṃ saṃsārasya svātmanaḥ bhedaṃ dūrīkṛtya  ekānta    darśanaṃ        jñānayogaḥ    \U1
%evaṃ saṃsāra....svātmanoḥ bhedaṃ dūrīkṛtyaṃ ekye?     darśanaṃ        jñānayoga     \U2
%------------------------------
%In this way, the realisation of unity (\textit{aikyena darśanam}) by eliminating the differentiation of the self from the mundane existance is truly Jnānayoga.
%------------------------------
evaṃ
       \app{\lem[wit={ceteri}]{saṃsārasya}
         \rdg[wit={P,L,U2}]{saṃsāra°}}
       \app{\lem[wit={B,E,L,P}]{svātmano}
         \rdg[wit={X}]{svātmanaḥ}
         \rdg[wit={U2}]{svātmanoḥ}}
       \app{\lem[wit={ceteri}]{bhedaṃ}
         \rdg[wit={B}]{bheda}
         \rdg[wit={D,N1}]{bhedāṃ}}
dūrī\app{\lem[wit={U2},alt={°kṛtyaṃ}]{kṛtyaṃ}
  \rdg[wit={ceteri}]{°kṛtya}
  \rdg[wit={E}]{°kṛty}}
\app{\lem[wit={P}]{aikyena}
  \rdg[wit={E}]{aikam eva}
  \rdg[wit={B,L,P}]{aikyona}
  \rdg[wit={D,N1}]{ekyena}
  \rdg[wit={N2}]{ekena}
  \rdg[wit={U1}]{ekānta}
  \rdg[wit={U2}]{ekye}}
darśanaṃ
\app{\lem[wit={E}]{sa eva}
  \rdg[wit={ceteri}]{\om}}
\app{\lem[wit={ceteri}]{jñānayogaḥ}
  \rdg[wit={U2}]{jñānayoga}}/ 
%------------------------------
%tasya         kāraṇāt kālaḥ śarīranāśaṃ na karoti/ \E
%tasya         kāraṇāt kālaḥ śarīranāśaṃ na karoti/ \P
%tasya         karaṇāt kālaḥ śarīranāśaṃ na karoti// \B
%tasya         karaṇāt kālaḥ śarīranāśaṃ na karoti... \L
%tasya         karaṇāt kālaḥ śarīranāśaṃ na karoti// \N1
%tasya         karaṇāt kālaḥ śarīranāśaṃ na karoti// \D
%tasya         karaṇāt kālaḥ śarīranāśaṃ    karoti/ \N2
%gatasya dhyānakaraṇāt kālaḥ śarīranāśaṃ na karoti 8 \U1
%tasya         karaṇāt kālaśarīranāśanaṃ    karoti// \U2
%------------------------------
%Due to this, time does not bring about the destruction of the body.
%------------------------------
\app{\lem[wit={ceteri}]{tasya}
  \rdg[wit={U1}]{gatasya}}
\app{\lem[wit={ceteri}, alt={kāraṇāt}]{kāraṇā\skp{t-kā}}
  \rdg[wit={U1}]{dhyānakaraṇāt}
}\app{\lem[wit={ceteri}, alt={kālaḥ}]{\skm{t-kā}laḥ}
  \rdg[wit={U1}]{kāla°}}
śarīranāśaṃ
\app{\lem[wit={ceteri}]{na}
  \rdg[wit={N2,U2}]{\om}}
karoti\dd{}\linelabel{_142e}
\label{jnanayogaend}
\end{prose}
  \end{edition}
  \begin{translation}
    \begin{tlate}[21_5]
      \paragraph{\uproman{21}.5} By virtue of its inherent nature, this branch of a tree with its new shoots (\textit{vistāra}), being the fruit of the flower of love, is in the seed. Certainly, that [the inherent nature?] is pure, eternal, unchanging, and immaculate.
    \end{tlate}
    \begin{tlate}[21_6]
     \paragraph{\uproman{21}.6} One, not one, self-existing by its own power, abiding in multiplicity, as five [gross] elements (\textit{tattva})\footnote{The term \textit{pañcatattva} refers to the five gross elements. The concept of five gross elements emerging from a supreme god is found in tantric works, cf. e.g. \citetitle{sivasvarodaya} 6-8 (\textit{nirañjano nirākara eko devo maheśvaraḥ} | \textit{tasmād ākāśam utpannam ākāśād vāyusambhavaḥ} || 6 || \textit{vayos tejas tataś cāpas tataḥ pṛthvī samudbhavaḥ} | \textit{etāni pañcatattvāni vistīrṇāni ca pañcadhā} || 7 || \textit{tebhyo brahmāṇḍam utpannaṃ tair eva parivartate} | \textit{vilīyate ca tatraiva tatraiva ramate punaḥ} || 8 ||) 
       ``Faultless and without a body is the one god, the great supreme ruler. From him, the ether element (\textit{ākāśa}) originated, and from the ether element, the air element came into existence (6). From the air element, the fire element and from the fire element, the water element and from the water element, the earth element. These five elements unfold in a fivefold manner (7). From these five elements, the universe has arisen, evolves and dissolves. [Then] right here, right there [it] enjoys again (8).'' In \citetitle{kumbhaka} 122, a technique of breath retention is dedicated to the five \textit{tattva}s (\textit{tatvādau pūreyed vāyuṃ tat tatvānte virecayet} | \textit{tatvakumbhaḥ sa gaditaḥ pañcadhā tatvabhedataḥ} || 122 ||) ``One shall inhale before [the rise] of a particular \textit{tattva} and exhale at the end of that \textit{tattva}. This is called \textit{tattvakumbhaka} being five-fold according to the five divisions of \textit{tattva}. The \citetitle{sivasvarodaya} discusses the rise, duration, properties and application of the \textit{tattva}s in greater detail. An overview of this can be seen in \citeauthor[2021: Appendix-\lowroman{3}]{kumbhaka}.} thinking mind (\textit{manas}), intellect (\textit{buddhi}), illusion (\textit{māya}), individuation (\textit{ahaṃkāra}), and modifications (\textit{vikriyā}).\begin{buber}[f21_1]\footnote{In the tantric traditions of Śaivism tenfold \textit{tattva}-systems existed \ldots.}\end{buber}
   \end{tlate}
   \begin{tlate}[21_7]
     \paragraph{\uproman{21}.7} Thus, everything is tenfold encompassing the world and non-world. There is only one. There nothing else. One who knows this is a knower of the truth.
     \\\\
     \end{tlate}
\begin{tlate}[p21_02]
  Transmigration (\textit{saṃsāra}) exists in the form of stationary [existances] such as earth, trees, mountains and so on. Transmigration (\textit{saṃsāra}) [also] exists in the form of the mobile [existances] such as humans, animals, birds and so on. Now, that which is the object of sight is called the seen. That which is not seen by sight is called the unseen. In this way, the removal of the distinction of the own self from transmigration is to be done by means of unity, only this perception is Jñānayoga. From the execution of this, time does not bring about the destruction of the body.
 \flushpage
    \end{tlate}
  \end{translation}
\end{alignment}
\pagebreak %after pp.43-44
%%%%%%%%%%%%%%%%%%%%%%%%%%%%%%%%%%%%%%%%%%
%%%%%%%%%%%%%%%%%%%%%%%%%%%%%%%%%%%%%%%%%%
%%%%%%%%PAGEBREAK%%%%%%%PAGEBREAK%%%%%%%%%
%%%%%%%%%%%%%%%%%%%%%%%%%%%%%%%%%%%%%%%%%%
%%%%%%%%%%%%%%%%PAGEBREAK%%%%%%%%%%%%%%%%%
%%%%%%%%%%%%%%%%%%%%%%%%%%%%%%%%%%%%%%%%%%
%%%%%%%%PAGEBREAK%%%%%%%PAGEBREAK%%%%%%%%%
%%%%%%%%%%%%%%%%%%%%%%%%%%%%%%%%%%%%%%%%%%
%%%%%%%%%%%%%%%%%%%%%%%%%%%%%%%%%%%%%%%%%%
%%%%%%%%%%%%%%%%%%%%%%%%%%%%%%%%%%%%%%%%%%
%%%%%%%%%%%%%%%%%%%%%%%%%%%%%%%%%%%%%%%%%%
%%%%%%%%PAGEBREAK%%%%%%%PAGEBREAK%%%%%%%%%
%%%%%%%%%%%%%%%%%%%%%%%%%%%%%%%%%%%%%%%%%%
%%%%%%%%%%%%%%%%PAGEBREAK%%%%%%%%%%%%%%%%%
%%%%%%%%%%%%%%%%%%%%%%%%%%%%%%%%%%%%%%%%%%
%%%%%%%%PAGEBREAK%%%%%%%PAGEBREAK%%%%%%%%%
%%%%%%%%%%%%%%%%%%%%%%%%%%%%%%%%%%%%%%%%%%
%%%%%%%%%%%%%%%%%%%%%%%%%%%%%%%%%%%%%%%%%%
%%%%%%%%%%%%%%%%%%%%%%%%%%%%%%%%%%%%%%%%%%
%%%%%%%%%%%%%%%%%%%%%%%%%%%%%%%%%%%%%%%%%%
%%%%%%%%PAGEBREAK%%%%%%%PAGEBREAK%%%%%%%%%
%%%%%%%%%%%%%%%%%%%%%%%%%%%%%%%%%%%%%%%%%%
%%%%%%%%%%%%%%%%PAGEBREAK%%%%%%%%%%%%%%%%%
%%%%%%%%%%%%%%%%%%%%%%%%%%%%%%%%%%%%%%%%%%
%%%%%%%%PAGEBREAK%%%%%%%PAGEBREAK%%%%%%%%%
%%%%%%%%%%%%%%%%%%%%%%%%%%%%%%%%%%%%%%%%%%
%%%%%%%%%%%%%%%%%%%%%%%%%%%%%%%%%%%%%%%%%%
\begin{alignment}[
  texts=edition[class="edition"];
  translation[class="translation"],
  ]
  \begin{edition}
    \ekddiv{
      head={[\uproman{22}. \textbf{svabhāvabhedam}]},
      type=section,
      depth=2, 
      n=XXII
    }
    \xmlhead[h22]{[XXII. svabhāvabhedam]}
       \begin{prose}[p22_01]
            \noindent
%------------------------------
%idānīṃ tasya---bhedaḥ    kathyate/   \E
%idānīṃ svabhāvabhedaḥ kathyate    \P
%idānī  svābhāvabhedaḥ kathyate//  \B
%idānīṃ svābhāvabhedaḥ kathyate//  \L
%idānīṃ svabhāvabhedaṃ kathyate//  \N1
%idānīṃ svabhāvabhedaṃ kathyate//  \D
%idānīṃ svabhāvabheda  kathyate//  \N2
%idānīṃ svabhāvabhedāḥ kathyate    \U1
%idānīṃ svabhāvabhedaḥ kathyate//  \U2
%------------------------------
%Now, the division of the inherent being is described. 
%------------------------------  
\note[type=source, labelb=_143i, labele=_145e, nosep]{cf. YSv (PT p. 836): svabhāvabhedam etat śṛṇu devi prayatnataḥ | yac chrutvā sarvabodhaḥ syāt muktidaḥ siddhivāñchitaḥ | ātmano vā pṛthivyādyāḥ svabhāvaḥ kiñcid ucyate | ātmaiva pṛthivī dhātrī komalā ca kvacid dṛḍhā | kvacin manoharā sā ca vimalā ca malāmalā | durgandhā ca sugandhā ca nirgandhā gandhamohinī | svarṇarūpā dhāturūpā citrā ratnamayī parā | kvacit śvetā kvacid raktā kvacit pītā ca kṛṣṇalā | ūrvarā ūrvarā sā tu viṣāmṛtamayī sadā |}
\linelabel{_143i}
\app{\lem[wit={ceteri}]{idānīṃ}
  \rdg[wit={B}]{idānī}}
\app{\lem[wit={ceteri},alt={svabhāva°}]{svabhāva}
  \rdg[wit={B,L}]{svābhāva°}
  \rdg[wit={E}]{tasya}
}\app{\lem[wit={D,N1},alt={°bhedaṃ}]{bhedaṃ}
  \rdg[wit={N2}]{°bheda}
  \rdg[wit={ceteri}]{°bhedaḥ}}
kathyate/
%------------------------------  
%yathā vaṭabījam/ vaṭarūpeṇa pariṇataṃ    sat    daśadhā    bhedaṃ svabhāvata eva prāpnoti/  \E %%%[P.27]
%yathā vaṭabījaṃ  vaṭarūpeṇa pariṇāte     sat    dṛśadhā    bhedaṃ svabhāvata eva prāpnoti   \P
%yathā vaṭabījena rūpeṇa     pariṇamate/  śata   daśadhā    bhedaṃ svābhāva   eva prāpnotī// \B
%yathā vaṭabījena rūpeṇa     pariṇamate   śata   daśadhā    bhedaṃ svābhāva   eva prāpnotī// \L
%yathā vaṭabījaṃ  vaṭarūpeṇa pariṇataṃ//  satṛ   daśadhā    bhedaṃ svabhāvata eva prāpnoti/  \N1
%yathā vaṭabījaṃ  vaṭarūpeṇa pariṇataṃ/   sa     daśadhā    bhedaṃ svabhāvata eva prāpnoti// \D
%yathā vathabījaṃ vaṭarūpeṇa pariṇataṃ/   sa tu  daśadhā    bhedaṃ svabhāvata eva prāpnoti/  \N2
%yathā vaṭabījaṃ  vaṭarūpeṇa pariṇataṃ    sa tat daśadhā    bhedaṃ svabhāvata eva prāpnotī   \U1
%yathā vaṭabīja---vaṭarūpeṇa pariṇamate// sa     dasat                            prāpnoti// \U2
%------------------------------
%Just as the seed of the banyan tree ripens into the shape of the banyan tree, [and] because of its own inherent being develops such a tenfold division. [Namely]:
%------------------------------
yathā
\app{\lem[wit={ceteri},alt={vaṭa°}]{vaṭa}
  \rdg[wit={N2}]{vatha°}
}\app{\lem[wit={D,P,N1,N2,U1},alt={°bījaṃ}]{bījaṃ}
        \rdg[wit={E}]{°bījam}
        \rdg[wit={U2}]{°bīja°}
        \rdg[wit={B,L}]{°bījena}}
      \app{\lem[wit={ceteri}]{vaṭarūpeṇa}
        \rdg[wit={B,L}]{rūpeṇa}}
      \app{\lem[wit={B,L,U2}]{pariṇamate}
        \rdg[wit={P}]{pariṇāte}
        \rdg[wit={X,E}]{pariṇataṃ}}
      \app{\lem[wit={U1},alt={sa tat}]{sa ta\skp{t-da}}
        \rdg[wit={N2}]{sa tu}
        \rdg[wit={N1}]{satṛ}
        \rdg[wit={E,P}]{sat}
        \rdg[wit={B,L}]{śata}
        \rdg[wit={D,U2}]{sa}
      }\app{\lem[wit={ceteri},alt={daśadhā}]{\skm{t-da}śadhā}
        \rdg[wit={P}]{dṛśadhā}
        \rdg[wit={U2}]{dasat}}
      \app{\lem[wit={ceteri}]{bhedaṃ}
        \rdg[wit={U2}]{\om}}
      \app{\lem[wit={ceteri}]{svabhāvata}
        \rdg[wit={B,L}]{svabhāva}
        \rdg[wit={U2}]{\om}}
      \app{\lem[wit={ceteri}]{eva}
        \rdg[wit={U2}]{\om}}
      \app{\lem[wit={ceteri}]{prāpnoti}
        \rdg[wit={B,L,U1}]{prāpnotī}}/
%------------------------------ %%%%STEMMA POINT!!!!
%mūlāṃkura---tvagdaṇḍaśākhā--kalikāpallavapuṣpaphalasnehā                  iti daśabhedān    prāpnoti// \E
%mūla aṃkura-tvakdaṃdaśākhā----kilpikāpallavā puṣpaphalasneha              iti daśabhedān    prāpnotīti \P  %%%7642.jpg
%mūlaṃ aṃkuratvakdaṃdaśākhā----kilakālapallavā// vistāroyaṃ svābhāvataḥ    iti daśabhedān    prāpnoti// \B DSCN7160 Z. 4
%mūlaṃ aṃkuratvakdaṃdaśākhā----kilāpallavā// vistāroyaṃ svābhāvataḥ//      iti daśabhedān    prāpnoti... \L
%mūlāṃ aṃkuratvakdaṃḍaśākhāṃ kalikāpallavapuṣpaphalasneha//                iti bhedo daśadhā prāpnoti// \N1
%mūlāṃkura---tvakdaṇdaśākhāṃ kalikāpallavapuṣpaphalasnehaṃ                 iti bhedo daśadhā prāpnoti// \D
%mūlāṃkura---tvakdaṇdaśākhāṃ kalikāpallavapuṣpaphalasneha/                 iti bhedo daśadhā prāpnoti// \N2
%mūlāṃaṃkura-tvakdaṇdaśākhā--kalikāpallavapuṣpaphalasneha                  iti bhedo daśadhā prāpnoti \U1
%\om                                                                                \U2
%------------------------------
%"Wurzel, Spross, Rinde, Ast, Zweig, Knospe, die sich entfaltende Blüte, Blüte, Frucht und Nektar." Die Auftheilung erreicht [diese] zehn Teile. 
%------------------------------
%"Root, shoot, bark, branch, twig, bud, the unfolding flower, flower, fruit and nectar." The division reaches [those] ten parts.
%------------------------------
\app{\lem[wit={E}]{mūlāṃkuratvagdaṇḍaśākhākalikāpallavapuṣpaphalasnehā}
          \rdg[wit={P}]{mūla aṃkuratvakdaṃdaśākhākilpikāpallavā puṣpaphalasneha}
          \rdg[wit={B}]{mūlaṃ aṃkuratvakdaṃdaśākhākilakālapallavā || vistāroyaṃ svābhāvataḥ}
          \rdg[wit={L}]{mūlaṃ aṃkuratvakdaṃdaśākhākilāpallavā || vistāroyaṃ svābhāvataḥ ||}
          \rdg[wit={N1}]{mūlāṃ aṃkuratvakdaṃḍaśākhāṃ kalikāpallavapuṣpaphalasneha ||}
          \rdg[wit={N2}]{mūlāṃkuratvakdaṇdaśākhāṃ kalikāpallavapuṣpaphalasneha|}
          \rdg[wit={D}]{mūlāṃkuratvakdaṇdaśākhāṃ kalikāpallavapuṣpaphalasnehaṃ}
          \rdg[wit={U1}]{mūlāṃaṃkuratvakdaṇdaśākhākalikāpallavapuṣpaphalasneha}
          \rdg[wit={U2}]{\om}}
        \app{\lem[wit={ceteri}]{iti}
          \rdg[wit={U2}]{\om}}
        \app{\lem[wit={X}]{bhedo daśadhā}
          \rdg[wit={B,E,L,P}]{daśabhedān}
          \rdg[wit={U2}]{\om}}
        \app{\lem[wit={ceteri}]{prāpnoti}
          \rdg[wit={P}]{prāpnotīti}
          \rdg[wit={U2}]{\om}}/
%------------------------------
%yathā nirmalo  nirvikāraḥ niraṃjana   eka  etādṛśa  ātmā svabhāvād eva/ pṛthivyaptejovāyvākāśamanobuddhimāyāvikārarūpabhedān    prāpnoti/ \E
%tathā nirmalaḥ nirvikāraḥ niraṃjanaḥ  eka  etādṛśa  ātmasvabhāvād eva   pṛthvyetetejo vādvyākāśamanobuddhimāyāvikārarūpabhedāt  prāpnoti \P
%tathā nirmalo  nirvikāraḥ niraṃjanaḥ  eka  etādṛśa  ātmasvabhāvād eva   pṛthvyāpatejovādvyākāśamanobuddhimāyāvikārarūpabhedāna  prāpnoti// \B
%tathā nirmalo  nirvikāraḥ niraṃjanaḥ/ eka  etādṛśa  ātmasvabhāvād eva   pṛthvyāpatejovāybākāśamanobuddhimāyāvikārarūpābhedāna   prāpnoti  \L
%tathā nirmalaḥ nirvikāraḥ niraṃjanaḥ  ekaḥ etādṛśaḥ ātmasvabhāvād eva   pṛthvyāpatejovāybākāśamanobuddhimāyāvikārarūpābhedān    prāpnoti/ \N1
%tathā nirmalaḥ nirvikāraḥ niraṃjanaḥ  eka  etādṛśaḥ ātmasvabhāvād eva   pṛthvīpate/ jīvīkāśamanobuddhir māyāvikārarūpabhedāt    prāpnoti \D
%tathā nirmalaḥ nirvikāraḥ niraṃjanaḥ  ekaḥ etādṛśaḥ ātmasvabhāvād eva   pṛthvīpate/ jīvīkāśamanobuddhir māyāvikārarūpabhedāt    prāpnoti/ \N2
%tathā nirmalaḥ nirvikāraḥ niraṃjanaḥ  ekaḥ etādṛśaḥ ātmascabhāvād eva   pṛthakte jīvāyuvākāśamanobuddhir māyāyāvikārarūpabhedāt prāpnoti \U1 %%%275.jpg
%yathā nirmalaḥ nirvikāraḥ niraṃjanaḥ  eka  etādṛśa  ātmasvabhāvād eva// pṛthvyaptejovāyyākāśa// manobuddhimayāvikārarūpabhedān  prāpnoti/ \U2
%------------------------------
%In dieser Weise erreicht auch das reine, unveränderliche, makellose, eine solche [Auftheilung] eben aufgrund der inhärenten Natur des Selbst. [Nämlich] die Aufteilung "Erde, Wasser, Feuer, Wind, Raum, Geist, Intellektekt, Illusion, Umwandlungen und Gestalt".
%------------------------------
%In this way, the pure, unchanging, unblemished, attains such [a division] precisely, because of the inherent being of the self. [Namely] the division: "Earth, water, fire, wind, space, mind, intellect, illusion, transformations and form".
%------------------------------
        \app{\lem[wit={ceteri}]{tathā}
            \rdg[wit={E,U2}]{yathā}}
          \app{\lem[wit={B,E,L}]{nirmalo}
            \rdg[wit={X,P,U2}]{nirmalaḥ}}
          nirvikāraḥ
          \app{\lem[wit={E}, alt={nirañjana}]{nirañjana}
            \rdg[wit={ceteri}]{niraṃjanaḥ}}
          \app{\lem[wit={ceteri}]{eka}
            \rdg[wit={N1,N2,U1}]{ekaḥ}}
          \app{\lem[wit={E}]{etādṛśa}
            \rdg[wit={N1,N2,U1}]{etādṛśaḥ}}
          \app{\lem[wit={ceteri}, alt={ātmasvabhāvād}]{ātmasvabhāvā\skp{d-e}}
            \rdg[wit={E}]{ātmā svabhāvād}
          }\skm{d-e}va
          \app{\lem[wit={B,L,N1}, alt={pṛthvyāpatejovāybākāśamanobuddhimāyāvikārarūpābhedān}]{pṛthvyāpatejovāybākāśamanobuddhimāyāvikārarūpābhedā\skp{n-prā}}
            \rdg[wit={E}]{pṛthivyapāpatejovāybākāśamanobuddhimāyāvikārarūpābhedān}
            \rdg[wit={P}]{pṛthvyetetejovādvyākāśamanobuddhimāyāvikārarūpābhedān}
            \rdg[wit={D,N2}]{pṛthvīpate | jīvīkāśamanobuddhir māyāvikārarūpabhedāt}
            \rdg[wit={U1}]{pṛthakte jīvāyuvākāśamanobuddhir māyāyāvikārarūpabhedāt}
            \rdg[wit={U2}]{pṛthvyaptejovāyyākāśa || manobuddhimayāvikārarūpabhedā}
          }\skm{n-prā}pnoti/
%------------------------------
%jñānayogaprabhāvād     eka eva  ātmā iti niścayo bhavati// \E
%jñānayogaḥ prabhāvād   eka eka  ātmā iti niścayo bhavati \P
%jñānayogaḥ// prabhāvād eka eka  ātmā iti niścayā bhavatī// \B
%jñānayogaḥ// prabhāvād eka eka  ātmā iti niścayo bhavati// \L
%jñānayogaprabhāvāt     eka eva  ātmā iti niścayo bhavati// \N1
%jñānayogaprabhāvāt     eka eva  ātmā iti niścayo bhavati// \D
%jñānayogaprabhāvāt     eka eva  ātmā iti niścayo bhavati// \N2
%jñānayogaprabhāvāt tu  eka yeva ātmā iti niścayo bhavati \U1
%jñānayogaprabhāvād     eka eva  ātmā iti niścayo bhavati// \U2
%------------------------------
%Because of the power of Jñānayoga, there arises the conviction that "the self is truly one".  
%------------------------------
\app{\lem[wit={E,U2}, alt={jñānayogaprabhāvād}]{jñānayogaprabhāvā\skp{d-e}}
  \rdg[wit={X}]{jñānayogabhavāt}
  \rdg[wit={B,L}]{jñānayogaḥ || prabhāvād°}
  \rdg[wit={P}]{jñānayogaḥ prabhāvād}
}\skm{d-e}ka
\app{\lem[wit={ceteri}]{eva}
  \rdg[wit={B,L,P}]{eka}
  \rdg[wit={U1}]{yeva}}
ātmā iti niścayo bhavati/
%------------------------------
%yathaikaiva   pṛthvī  kvacit komalarūpā                                                   kvacit parimalarūparahitā kvacit suvarṇarūpā   kvacid raupyarūpā    \E %%%p.28 
%yathā ekaika  pṛthvī  kvacit komalarūpā                                                                                                                       \P   
%yathā ekaika  pṛthvī  kvacit komalarūpā// kvacit manohararūpā//  kvacit parimalarūpayuktā// kvacit parimalarohitā// kvacit suvarṇarūpa                        \B
%yathā ekaika  pṛthvī  kvacit komalarūpā   kvacit manohararūpāḥ// kvacit parimalarūpayuktā// kvacit parimalarahitā// kvacit suvarṇarūpā                        \L
%yathā ekaiva  pṛthivī kvacit komalarūpa/  kvacit manoharā/       kvacit parimalarūpāyuktā// kvacit parimalarahitā/  kvacit suvarṇarūpā/  kvacit rūpyarūpā/    \N1
%yathā ekaiva  pṛthivī kvacit komalarūpa   kvacit manoharā//      kvacit parimalarūpāyuktā/  kvacit parimalarohitā   kvacit suvarṇarūpa// kvacit rūpyarūpa//   \D
%yathā ekaṃ ca pṛthivī kvacit komalarūpa   kvacit manoha?rā       kvacit parimalarūpāyuktaḥ/ kvacit parimalarohitā   kvacit suvarṇarūpā   kvacit rūpyarūpa     \N2
%yathā ekai ca pṛthivī kvacit                                                                                              khavarṇakupā   kvacit rūpyarūpā     \U1
%yathā ekaika  pṛthvī  kvacit komalarūpā// kvacit manohararūpa//  kvacit parimalarūpāyuktā/  kvacit parimalarohitā// kvacit suvarṇarūpā// kvacit rajatarūpā//  \U2
%------------------------------
%As some particular soil (\textit{ekaika}) sometimes appears soft, sometimes appears beautiful, sometimes fragrant, sometimes unscented, sometimes golden, sometimes silver,... 
%------------------------------
\app{\lem[type=emendation, resp=egoscr]{yathaikaikaḥ}
  \rdg[wit={E}]{yathaikaiva}
  \rdg[wit={B,L,P,U2}]{yathā ekaika}
  \rdg[wit={D,N1}]{yathā ekaiva}
  \rdg[wit={N2}]{yathā ekaṃ ca}
  \rdg[wit={U1}]{yathā ekai ca}}
\app{\lem[wit={Y}]{pṛthvī}
  \rdg[wit={X}]{pṛthivī}}
kvacit-komala\app{\lem[wit={Y},alt={°rūpā}]{rūpā}
    \rdg[wit={X}]{°rūpa}}\dd{}\note[type=philcomm, labelb=145a, labele=_145e, lem={kvacit manohararūpā \ldots kvacit pītā}]{Section is omitted in \getsiglum{P}.}
\app{\lem[wit={ceteri}, alt={kvacit}]{kvaci\skp{t-ma}}
  \rdg[wit={E,P,U1}]{\om}
}\app{\lem[wit={B}, alt={manohararūpā}]{\skm{t-ma}nohara:\\rūpā}
  \rdg[wit={L}]{manohararūpāḥ}
  \rdg[wit={U2}]{manohararūpa}
  \rdg[wit={D,N1,N2}]{manoharā}
  \rdg[wit={E,P,U1}]{\om}}\dd{}
\app{\lem[wit={ceteri}, alt={kvacit}]{kvaci\skp{t-pa}}
  \rdg[wit={E,P,U1}]{\om}
}\app{\lem[wit={ceteri},alt={°parimala}]{\skm{t-pa}rimala}
  \rdg[wit={E,P,U1}]{\om}
}\app{\lem[wit={B,L},alt={°rūpayuktā}]{rūpayuktā}
  \rdg[wit={D,N1}]{°rūpā°}
  \rdg[wit={N2}]{°rūpāyuktaḥ}
  \rdg[wit={E,U1}]{\om}}\dd{}
\app{\lem[wit={ceteri}, alt={kvacit}]{kvaci\skp{t-pa}}
  \rdg[wit={P,U1}]{\om}
}\app{\lem[wit={ceteri},alt={°parimala}]{\skm{t-pa}rimala}
  \rdg[wit={E}]{°parimalarūpa°}
  \rdg[wit={P,U1}]{\om}
}\app{\lem[wit={E,L,N1},alt={°rahitā}]{rahitā}
  \rdg[wit={B,N2,U2}]{°rohitā}
  \rdg[wit={D,P,U1}]{\om}}\dd{}
\app{\lem[wit={ceteri}, alt={kvacit}]{kvaci\skp{t-su}}
  \rdg[wit={P,U1}]{\om}
}\app{\lem[wit={E,L,N2,U2}, alt={suvarṇarūpā}]{\skm{t-su}varṇarūpā}
  \rdg[wit={B,D}]{suvarṇarūpa}
  \rdg[wit={U1}]{khavarṇakupā}
  \rdg[wit={P}]{\om}}\dd{}
\app{\lem[wit={ceteri}, alt={kvacit}]{kvaci\skp{t-rau}}
  \rdg[wit={B,L,P}]{\om}
}\app{\lem[wit={E}, alt={raupyarūpā}]{\skm{t-rau}pyarūpā}
  \rdg[wit={N1,U1}]{rūpyarūpā}
  \rdg[wit={D,N2}]{rūpyarūpa}
  \rdg[wit={U2}]{rajatarūpā}
  \rdg[wit={B,L,P}]{\om}}\dd{}
%------------------------------
%kvacid ratnamayī   kvacic ca śvetā                                kvacidraktā   kvacitpītā    \E %%%p.28 
%                                                                                             \P   
%kvacid ratnamaī//  kvacit śverūpā// kvacitkṛṣṇā//                 kvacidraktā/  kvacitpītā//  \B
%kvacid ratnamaī//  kvacit śvetarūpā kvacitkṛṣṇā//                 kvacidraktā// kvacitpītā//  \L
%kvacid ratnamayī/  kvacit śveta/    kvacitkṛṣṇa??/                kvacidrakta/  kvacitpītā/   \N1
%kvacid ratnamayī// kvacit śvetā//   kvacitkṛṣṇā [S8., Z.7]        kvacidrakta   kvacitpītā//  \D
%kvacid ratnamayī   kvacit śveta     kvacitkṛṣṇā// [S6. verso]     kvacidrakta   kvacitpītā    \N2
%kvacid ratnamayī   kvacit śveta     kvacitkṛṣṇā                   kvacidrakta   kvacitpītā    \U1
%kvacid ratnamayī// kvacit śvetā//   kvacitkṛṣṇā//                 kvacidraktā// kvacitpītā//  \U2
%------------------------------
% ... manchmal aus Edelstein gemacht ist, manchmal weiß erscheint, manchmal schwarz, manchmal kupfern, manchmal gelb,
%
%... is sometimes made of precious stone, sometimes appearing white, sometimes black, sometimes copper, sometimes yellow, 
%------------------------------
kvaci\skp{d-ra}\app{\lem[wit={ceteri},alt={ratnamayī}]{\skm{d-ra}:\\tnamayī}
  \rdg[wit={B,L}]{ratnamaī}}\dd{}
\app{\lem[wit={ceteri}, alt={kvacit}]{kvaci\skp{t-śve}}
  \rdg[wit={E}]{kvacic ca}
}\app{\lem[wit={E,D,U2}, alt={śvetā}]{śvetā}
  \rdg[wit={N1,N2,U1}]{śveta}
  \rdg[wit={L}]{śvetarūpā}
  \rdg[wit={B}]{śverūpā}}\dd{}
\app{\lem[wit={ceteri}, alt={kvacit kṛṣṇā}]{kvacit-kṛṣṇā}
  \rdg[wit={N1}]{kṛṣṇa}
  \rdg[wit={E}]{\om}}\dd{}
kvaci\skp{d-ra}\app{\lem[wit={B,E,L,U2},alt={raktā}]{\skm{d-ra}ktā}
  \rdg[wit={ceteri}]{°rakta}}\dd{}
kvacit-pītā\dd{}\linelabel{_145e}
    \end{prose}
  \end{edition}
  \begin{translation}
   \ekddiv{
   head={[\uproman{22}. \textbf{Division of the Inherent Being}]},
   type=section,
   depth=2, 
   n=XXII.1 
 }
 \xmlhead[h22]{[XXII. Division of the Inherent Being]}
        \begin{tlate}[p22_01]
            \noindent
            \begin{euber}[f21_1]\blfootnote{\hspace{-2.2em}from an very early age, cf. \citeauthor[2016:82-85]{goodall2016}. Rāmacandra, as can be seen in the sources of the edition for this passage, is faithful to his source text. However, the exact origin of this specific arrangement of \textit{tattva}s remains unknown. Usually \textit{vikriyā} is not a separate \textit{tattva}, but \textit{ahaṃkāravikrīya} (“transformations of \textit{ahaṃkāra}”) refers to lesser \textit{tattva}s like the \textit{jñānendrīya}s, \textit{karmendrīya}s and \textit{tanmātra}s. Here, it seems the term \textit{vikriyā} is taken as a \textit{tattva} on its own, functioning as a placeholder for the modifications of \textit{ahaṃkāra}.}\end{euber} Now, the division of the inherent being is described. Just as the seed of the banyan tree ripens into the shape of the banyan tree, [and] because of its own inherent being develops such a tenfold division. [Namely]: root, shoot, bark, branch, twig, bud, the unfolding flower, flower, fruit and nectar. The division develops [those] ten parts. In this way, the pure, unchanging, unblemished [one] attains such [a division] precisely because of the inherent being of the self. [Namely], the division: earth, water, fire, wind, space, mind, intellect, illusion, transformations and form.\footnote{Rāmacandra’s tenfold taxonomy of \textit{tattva}s appears inconsistent. Here, besides the stable list of the five gross elements, \textit{ahaṃkāra} is replaced with \textit{rūpa} and the order of the other elements are changed. None of the tenfold \textit{tattva}-systems known to me equal Rāmacandra’s systems exactly. Taxonomies of \textit{tattva}s like \citetitle{bhagavadgita} 7.4 in which Kṛṣṇa presents a list of eight divisions of \textit{prakṛti} are almost identical (\textit{bhūmir āpo’ nalo vāyuḥ khaṃ mano buddhir eva cha} | \textit{ahankāra itīyaṃ me bhinnā prakṛitir aṣhṭadhā} ||7.4||). “Earth, water, fire, air, space, mind, intellect, and ego - these are the eight divisions of the original nature.” In this list, we find most of the elements of Rāmacandra’s list, except terms like \textit{māyā}, \textit{vikriyā} or \textit{rūpā}. The description of \textit{kuṇḍalinī} in \citetitle{yajnavalkya} 4.21 picks up this system. Another system of ten \textit{tattva}s appears in \textit{Uttarasūtra} 1.9 - 1.13 of the \citetitle{nishvasa2015} in which the tenfold nature of Sadāśiva is homologised with the \textit{mantra}. Here the following list is given: \textit{prakṛti}, \textit{puruṣa}, \textit{niyati}, \textit{kāla}, \textit{māyātattva}, \textit{vidyā}, \textit{Īśvara}, \textit{Sadāśiva}, \textit{dehavyāpin} and \textit{Śakti}, cf. \citeauthor[2016: 83-84]{goodall2016}. There is no similarity between the two tenfold \textit{tattva}-systems. However, since Rājayoga is deeply rooted in ancient Śaivsim (see \citeauthor{birch2019saiva}) 2019, but Rāmacandra’s usually tends to present simplified and transsectarian systems. However, the choice of a tenfold \textit{tattva}-system might be a remnant of those ancient systems.} Because of the power of Jñānayoga, there arises the conviction that “the self is truly one”. Just as some particular soil (\textit{ekaika}) sometimes appears soft, sometimes appears beautiful, sometimes fragrant, sometimes unscented, sometimes golden, sometimes silver, is sometimes made of precious stone, sometimes appearing white, sometimes black, sometimes copper, sometimes yellow, \ldots
%            \flushpage 
            \end{tlate}
  \end{translation}
\end{alignment}
\pagebreak %after pp. 45-46
%%%%%%%%%%%%%%%%%%%%%%%%%%%%%%%%%%%%%%%%%%
%%%%%%%%%%%%%%%%%%%%%%%%%%%%%%%%%%%%%%%%%%
%%%%%%%%PAGEBREAK%%%%%%%PAGEBREAK%%%%%%%%%
%%%%%%%%%%%%%%%%%%%%%%%%%%%%%%%%%%%%%%%%%%
%%%%%%%%%%%%%%%%PAGEBREAK%%%%%%%%%%%%%%%%%
%%%%%%%%%%%%%%%%%%%%%%%%%%%%%%%%%%%%%%%%%%
%%%%%%%%PAGEBREAK%%%%%%%PAGEBREAK%%%%%%%%%
%%%%%%%%%%%%%%%%%%%%%%%%%%%%%%%%%%%%%%%%%%
%%%%%%%%%%%%%%%%%%%%%%%%%%%%%%%%%%%%%%%%%%
%%%%%%%%%%%%%%%%%%%%%%%%%%%%%%%%%%%%%%%%%%
%%%%%%%%%%%%%%%%%%%%%%%%%%%%%%%%%%%%%%%%%%
%%%%%%%%PAGEBREAK%%%%%%%PAGEBREAK%%%%%%%%%
%%%%%%%%%%%%%%%%%%%%%%%%%%%%%%%%%%%%%%%%%%
%%%%%%%%%%%%%%%%PAGEBREAK%%%%%%%%%%%%%%%%%
%%%%%%%%%%%%%%%%%%%%%%%%%%%%%%%%%%%%%%%%%%
%%%%%%%%PAGEBREAK%%%%%%%PAGEBREAK%%%%%%%%%
%%%%%%%%%%%%%%%%%%%%%%%%%%%%%%%%%%%%%%%%%%
%%%%%%%%%%%%%%%%%%%%%%%%%%%%%%%%%%%%%%%%%%
%%%%%%%%%%%%%%%%%%%%%%%%%%%%%%%%%%%%%%%%%%
%%%%%%%%%%%%%%%%%%%%%%%%%%%%%%%%%%%%%%%%%%
%%%%%%%%PAGEBREAK%%%%%%%PAGEBREAK%%%%%%%%%
%%%%%%%%%%%%%%%%%%%%%%%%%%%%%%%%%%%%%%%%%%
%%%%%%%%%%%%%%%%PAGEBREAK%%%%%%%%%%%%%%%%%
%%%%%%%%%%%%%%%%%%%%%%%%%%%%%%%%%%%%%%%%%%
%%%%%%%%PAGEBREAK%%%%%%%PAGEBREAK%%%%%%%%%
%%%%%%%%%%%%%%%%%%%%%%%%%%%%%%%%%%%%%%%%%%
%%%%%%%%%%%%%%%%%%%%%%%%%%%%%%%%%%%%%%%%%%
\begin{alignment}[
  texts=edition[class="edition"];
  translation[class="translation"],
  ]
  \begin{edition}
    \begin{prose}[p22_02]
    \noindent
%------------------------------
%kvacitkarburā   kvacin nānāvidharūpā        kvacid viṣarūpā    kvacid amṛtarūpamayī svabhāvata eva bhavati//  \E  %%%p.28
%                                                               kvacid amṛtamayī     svabhāvata eva bhavati    \P  %%%rest is \om
%kvacitkarburā// kvacin nānāvidhaphalarūpā   kvacit viṣarūpā//  kvacid amṛtamaī/     svabhāvata eva bhavataḥ// \B
%kvacitkarburā// kvacin nānāvidhāphalarūpā   kvacit viṣarūpā//  kvacid amṛtamaī//    svabhāvata eva bhavataḥ// \L
%kvacitkarburā,  kvacin nānāvidhaphalarūpā/  kvacid puṣparūpā,  kvacid amṛtamayī     svabhāvata eva bhavati/   \N1
%kvacitkarburā   kvacin nānāvidhaphalarūpā// kvacid puṣparūpā// kvacid amṛtamayī/    svabhāvata eva bhavati//  \D
%kvacitkarburā   kvacin nānāvidhaphalarūpā                      kvacid amṛtamayī/    svabhāvata eva bhavati//  \N2
%kvacitkarpurā   kvacin nānāvidhophalarūpā   kvacid ....[rest omitted]                                         \U1
%kvacitkarburā// kvacit nānāvidhaphalarūpā// kvacir viśarūpā//  kvacit amṛtamayī//   svabhāvata eva bhavati//  \U2
%------------------------------
%machmal gesprenkelt, machmal wie verschiedenartige Frucht erscheint, manchmal wie Blumen erscheint, machmal wie der Nektar der Unsterblichkeit erscheint, [und das nur] nur aufgrund seiner inhärenten Natur.
%------------------------------
%sometimes mottled, sometimes appearing like various fruit, sometimes appearing like flowers, sometimes appearing like the nectar of immortality, only because of its inherent being. 
%------------------------------
kvavi\skp{t-ka}\app{\lem[wit={ceteri}, alt={karburā}]{\skm{t-ka}rburā}
  \rdg[wit={U1}]{karpurā}}\dd{}
\app{\lem[wit={ceteri}]{kvaci\skp{n-nā}}
  \rdg[wit={U2}]{kvacit}
  \rdg[wit={P}]{\om}
}\app{\lem[wit={ceteri},alt={nānāvidhaphalarūpā}]{\skm{n-nā}nāvidhaphalarūpā}
  \rdg[wit={U1}]{nānāvidhophalarūpā}
  \rdg[wit={E}]{nānāvidharūpā}
  \rdg[wit={P}]{\om}}\dd{}
\app{\lem[wit={B,L},alt={kvacit}]{kvaci\skp{t-pu}}
  \rdg[wit={D,N1,U1}]{kvacid}
  \rdg[wit={U2}]{kvacir}
  \rdg[wit={P,N2}]{\om}
}\app{\lem[wit={D,N1},alt={puṣparūpā}]{\skm{t-pu}ṣparūpā}
\rdg[wit={B,E,L}]{viṣarūpā}
\rdg[wit={U2}]{vśarūpā}
\rdg[wit={U1}]{\om}}\dd{}
\app{\lem[wit={ceteri}, alt={kvacid}]{kvaci\skp{d-a}}
  \rdg[wit={U2}]{kvacit}
  \rdg[wit={U1}]{\om}
}\app{\lem[wit={ceteri},alt={amṛtamayī}]{\skm{d-a}mṛtamayī}
  \rdg[wit={E}]{amṛtarūpamayī}
  \rdg[wit={B,L}]{amṛtamaī}
  \rdg[wit={U1}]{\om}}\dd{}
\app{\lem[wit={ceteri}]{svabhāvata}
  \rdg[wit={U1}]{\om}}
\app{\lem[wit={ceteri}]{eva}
  \rdg[wit={U1}]{\om}}
\app{\lem[wit={ceteri}]{bhavati}
  \rdg[wit={B,L}]{bhavataḥ}
  \rdg[wit={U1}]{\om}}\dd{}\\
%------------------------------
%tathaivātmā   manuṣyapakṣihariṇahastividyādharagandharvakinnaramahāpaṃḍitamahāmūrkha  rogyarogikrodhi---śāṃtarūpaḥ      svabhāvād eva bhavati/ \E
%tathaivātmā   manuṣyapakṣihariṇāhastividyādharagaṃdharvakinnaramahāpiṃḍitamahārmūkha  rogī-----krodhi---śāṃtarūpāḥ      svabhāvād eva bhavati \P
%tathaivātmā// manuṣyapakṣihariṇahastividyādharagaṃdharvakinnaramahāpiṃḍatamahāmūrkha  rogī-----krodhadhiśāṃtarūpaḥ      svabhāvād eva bhavatī/ \B
%tathaivātmā   manuṣyapakṣihariṇahastividyādharagaṃdharvakinnaramahāpaṃḍitamahāmūrkha  rogī-----krodhadhīśāṃtarūpāḥ      svabhāvād eva bhavatī/ \L
%tathātmā//    manuṣyapakṣihariṇahastīvidyādharagandharvakiṃnaramahāpaṃḍitamahāmūrva   rogīarogīkrodhī---śāntarūpa-------svabhāvād eva bhati/ \N1 %%%%%%%CRAZY SWITCH BETWEEN DAṆḌA AND COMMA
%tathātmā//    manuṣyapakṣihariṇahastīvidyādharagandharvakinnaramahāpaṃḍitamahāmūrva   rogīarogīkrodhī---śāṃtarūpa-------svabhāvād eva bhavati/ \D
%tathātmā//    manuṣyapakṣihariṇahastividyādharagandharvakinnaramahāpaṇḍitamahāmūrkha  rogīarogīkrodhī---śāṃtarūpa-------svabhāvād eva bhavati/ \N2
%                                     vidyādharagaṃdharvakinnaramahāpaṇḍitamahāmūrṣa   rogīarogīkrodhī---śāṃtarūpa       evaṃ svabhāvaṃ dharati  \U1
%tathaivātmā   manuṣyapakṣihariṇahastividyādharagaṃdharvakinnaramahāpaṃḍitamahāmūrkha  rogīarogīkrodhi---śāṃtarūpaḥ      svabhāvād eva bhavati// \U2 %%%410.jpg
%------------------------------
%Auf diese Weise nimmt auch das Selbst aufgrund seiner inhärenten Natur die Form eines Menschen, Vogels, einer Gazelle, eines Elefants, eines Vidyādharas, eines Gandharvas, Zentauren, eines großen Gelehrten oder großen Dummkopfes, eines Kranken oder Gesunden, eines Zornigen oder Friedlichen an.
%
%In the same way, the self also takes the form of a human, a bird, a gazelle, an elephant, a vidyādhara, a gandharva, a centaur, great scholar or a great fool, a sick or healthy, an angry or or peaceful person, by virtue of its inherent being.       
%------------------------------      
\note[type=source, labelb=146, labele=_146e, nosep]{cf. YSv (PT p. 836): tathā ca devagandharvakinnarādyāḥ khagādayaḥ | sukhasampiṇḍito rogī tathaiva krodhaśāntadhīḥ |aśeṣarūpabalito nānābuddhirataḥ svayam | devatattvaṃ bhūtaśaktyā jīvasaṃjñā bhramātmikā | jñānayogī nirvikāro nistāpa eka īśvaraḥ | ātmaikamūrttimān bhūtvā nirvikalpo nirañjanaḥ | sukhī duḥkhī mohayukto 'nantacetāḥ svabhāvataḥ |}
\app{\lem[wit={Y}, alt={tathaivātmā}]{tathaivātmā}
  \rdg[wit={X}]{tathātmā}}
\app{\lem[wit={ceteri},alt={manuṣya°}]{manuṣya}
  \rdg[wit={U1}]{\om}
}\app{\lem[wit={ceteri},alt={°pakṣi°}]{pakṣi}
  \rdg[wit={U1}]{\om}
}\app{\lem[wit={ceteri},alt={°hariṇa°}]{hariṇa}
  \rdg[wit={P}]{°hariṇā°}
  \rdg[wit={U1}]{\om}
}\app{\lem[wit={D,N1},alt={°hastī°}]{hastī}
  \rdg[wit={ceteri}]{hasti}
  \rdg[wit={U1}]{\om}
}vidyādharagaṃdharvakinnaramahā\app{\lem[wit={ceteri},alt={°paṇḍita°}]{paṇḍita}
  \rdg[wit={B}]{piṃḍata}
}mahā\app{\lem[wit={ceteri},alt={°mūrkha°}]{mūrkha}
  \rdg[wit={P}]{°rmūkha°}
  \rdg[wit={D,N1}]{°mūrva°}
  \rdg[wit={U1}]{°mūrṣa°}
}\app{\lem[type=emendation, resp=egoscr]{rogyarogī}
  \rdg[wit={E}]{°rogyarogi}
  \rdg[wit={X,U2}]{°rogī arogī}
  \rdg[wit={B,L,P}]{°rogī}
}\app{\lem[wit={ceteri},alt={°krodhī°}]{krodhī}
  \rdg[wit={E,P}]{°krodhi°}
  \rdg[wit={B,L}]{°krodha°}
}\app{\lem[wit={ceteri},alt={°śānta°}]{śānta}
  \rdg[wit={B,L}]{°dhiśānta°}
}\app{\lem[wit={ceteri},alt={°rūpaḥ}]{rūpaḥ}
  \rdg[wit={P,L}]{°rūpāḥ}
  \rdg[wit={X}]{°rūpa}}
\app{\lem[wit={ceteri},alt={svabhāvād eva}]{sva:\\bhāvād-eva}
  \rdg[wit={U1}]{evaṃ svabhāvaṃ}}
\app{\lem[wit={ceteri}]{bhavati}
  \rdg[wit={B,L}]{bhavatī}
  \rdg[wit={N1}]{bhati}
  \rdg[wit={D}]{dharati}}\dd{}
%------------------------------      
%jñānayogādhikārarūparahito  jñāyate/  yathā plakṣasyotpattiḥ/ sthānam eva bhavati// \E
%jñānayogādhikārarūparahito  jñāyate   yathā phalasyotpattisthānam ekam eva bhavati \P  %%%7643.jpg        
%jñānayogādhikārarūparahito  jñāyate// yathā phalasyotpattisthānam ekam eva bhavatī// \B
%jñānayogādhikārarūparahito  jñāyate// yathā phalasyotpattisthānam ekam eva bhavati// \L
%jñānayogād vikārarūparahito jñāyate/  yathā phalasyotpattisthānam ekam eva bhavati/ \N1
%jñānayogādhikārarūparahito  jñāyate// yathā phalasyotpattisthānam ekaseva  bhavati// \D
%jñānayogadhikārarūparahito  jñāyate// yathā phalasyotpattisthānam eva kameva bhavati// \N2
%jñānayogāt vikārarūparahito jñāyate   yathā phalasyotpattisthāna  ekam eva ti \U1
%jñānayogādhikārarūparahito  jāyate//  yathā phalasyotpattisthānam ekam eva bhavati// \U2
%------------------------------
%Through Jñānayoga he realizes the emptiness of the mutability of form. Just as the place of origin of the fruit is%%only one.
%------------------------------
\app{\lem[wit={N1,U1}, alt={jñānayogād vikāra}]{jñānayogād-vikāra}
  \rdg[wit={ceteri}]{jñānayogadhikāra}
}rūparahito
\app{\lem[wit={ceteri}]{jñāyate}
  \rdg[wit={U2}]{jāyate}}/
yathā
\app{\lem[wit={ceteri}]{phalasyotpatti}
  \rdg[wit={E}]{plakṣasyotpattiḥ}
}\app{\lem[wit={ceteri},alt={°sthānam}]{sthāna\skp{m-e}}
  \rdg[wit={E}]{sthānam}
  \rdg[wit={U1}]{°sthāna}
}\app{\lem[wit={ceteri},alt={ekam}]{\skm{m-e}ka\skp{m-e}}
  \rdg[wit={D}]{ekas}
  \rdg[wit={N2}]{eva}
  \rdg[wit={E}]{\om}
}\app{\lem[wit={ceteri},alt={eva}]{\skm{m-e}va}
  \rdg[wit={N2}]{kam eva}}
\app{\lem[wit={ceteri}]{bhavati}
  \rdg[wit={B}]{bhavatī}
  \rdg[wit={U1}]{ti}}/
%------------------------------
%atha ca phalasya gatir bahudhā dṛśyate/ \E
%atha ca phalasya gati  bahudhā dṛśyate    \P
%atha ca phalasya gatir bahudhā dṛśyate// \B
%atha ca phalasya gatir bahudhā dṛśyate// \L
%atha ca phalasya gatir bahudhā dṛśyate/ \N1
%atha ca phalasya gatir bahudhā dṛśyate// \D
%atha ca phalasya gati  bahudhā dṛśyate/ \N2
%atra ca phalasya gati  bahudhā dṛśyate \U1
%atha ca phalasya gatir bahudhā dṛśyate// \U2
%------------------------------
%But the development of the fruit is seen manifold. 
%------------------------------
atha ca phalasya \app{\lem[wit={ceteri},alt={gatir}]{gati\skp{r-ba}}
  \rdg[wit={P,N2,U1}]{gati}
}\skm{r-ba}hudhā dṛśyate\dd{}
%------------------------------ %%%STEMMAPOINT śuklaṃ//śuṣkaṃ
% ekaṃ phalaṃ pṛthvīmadhye  patati/  śuklaṃ bhavati/   \E
% ekaṃ phalaṃ pṛthvīmadhye  patati   śuklaṃ bhavati    \P
% ekaṃ phalaṃ pṛthvīmadhye  patiśuklaṃ      bhavatī//  \B
% ekaṃ phalaṃ pṛthvīmadhye  patati   śuṣkaṃ bhavatī    \L
% ekaṃ phala--pṛthvīmadhye  patati/  śuklaṃ bhavati/   \N1 %%%p.7 recto letzte Zeile 
% ekaṃ phala--pṛthvīmadhye  patati// śuklaṃ bhavati//  \D
% eva  phala--pṛthvīmadhye  patati   śuklaṃ bhavati//  \N2
% ekaṃ phalaṃ pṛthivīmadhye patati   śuṣkaṃ bhavati    \U1
% ekaphalaṃ   pṛthvīmadhye  patati// śuṣkaṃ bhavati//  \U2
%------------------------------
%One fruit falls onto the ground, and becomes dry. 
%------------------------------
\app{\lem[wit={ceteri}]{ekaṃ}
  \rdg[wit={U2}]{eka°}
  \rdg[wit={N2}]{eva}}
\app{\lem[wit={ceteri}]{phalaṃ}
  \rdg[wit={D,N1,N2}]{phala°}}
\app{\lem[wit={ceteri},alt={pṛthvī°}]{pṛthvī}
  \rdg[wit={U1}]{pṛthivī°}
}madhye patati/
\app{\lem[wit={L,U1,U2}]{śuṣkaṃ}
  \rdg[wit={ceteri}]{śuklaṃ}}
\app{\lem[wit={ceteri}]{bhavati}
  \rdg[wit={B}]{bhavatī}}/
%------------------------------
% ekasya phalasya makaraṃdaṃ bhramaraḥ  pibati/  \E
% ekasya phalasya makaraṃdaṃ bhramaraḥ  pibaṃti  \P
% ekasya            karaṃdaṃ bhramaraṃ  pibatī/  \B
% ekasya          makaraṃdaṃ bhramaraṃ  pibati   \L
% ekasya phalasya makaraṃdabhramaraḥ    pibati/  \N1 %%%p.7 recto letzte Zeile 
% ekasya phalasya makaraṃdabhramaraḥ    pibati/  \D
% ekasya phalasya makaraṃdaṃ bhramara   pibati/  \N2
% ekasya phalasya makaraṃdaṃ bhramanaḥ  pibati   \U1
% ekasya phalasya makaraṃdaṃ bhramaraḥ  pibati// \U2
% ------------------------------
% Eine Biene trinkt den Saft der einen Frucht.
% A bee drinks the juice of the one fruit.     
%------------------------------
ekasya
\app{\lem[wit={ceteri}]{phalasya}
  \rdg[wit={P,L}]{\om}} 
\app{\lem[wit={E,L,P,N2,U1,U2}]{makarandaṃ}
  \rdg[wit={L,N1}]{makaraṃda°}
  \rdg[wit={B}]{karaṃdaṃ}}
\app{\lem[wit={ceteri}]{bhramaraḥ}
  \rdg[wit={B,L}]{bhramaraṃ}
  \rdg[wit={N2}]{bhramara}}
\app{\lem[wit={ceteri}]{pibati}
  \rdg[wit={P}]{pibaṃti}
  \rdg[wit={B}]{pibatī}}/
%------------------------------
% ekasya phalasya  mālāṃ kāminī tuṃgakucamaṃḍalopari dadhāti/ \E
% ekasya phalasya  mālāṃ kāminī tuṃgakucamaṃḍalopari dadhāti \P
% ekasya phalasya  mālāṃ kāminī tuṃgakucamaṃḍalopari dadhātī// \B
% ekasya phalasya  mālāṃ kāminī tuṃgakucamaṃḍalopari dadhāti// \L
% ekasya phalasya  mālāṃ kāminī tuṃgakucamaṃḍalopari dadhāvati/ \N1 %%%p.7 recto letzte Zeile 
% ekasya phalasya  mālāṃ kāmibī tuṃgakucamaṇḍalopari dadhāti// \D
% ekasya phalasyaṃ mālākāminī   tuṃgakucamaṇḍalopari dadhovati// \N2
% ekasya phalasya  mālāṃ kāmini tuṃ  kucamaṃḍalopari dadhāti \U1
% ekasya phalasya  mālāṃ kāminī tuṃgakucamaṃḍalopari dadhāti// \U2
%------------------------------
% of the one fruit Blütenkranz/Girlande die Verliebte (biene) führt ein unmittelbar über dem Kreis des Blütenstempels der wie eine Brust ist ein.  %tu.mga = hervorstehend 
% Die [nach Blumensaft] Verlangende [Biene] platziert sich auf dem Blütenkranz über dem emportstehenden kreisförmigen Blütenstempel.
%[Or] a woman places a garland [made of] the one fruit above her voluptuous bosom.   
%------------------------------
ekasya
\app{\lem[wit={ceteri}]{phalasya}
  \rdg[wit={N2}]{phalasyaṃ}}
\app{\lem[wit={ceteri}]{mālāṃ}
  \rdg[wit={N2}]{mālā°}}
\app{\lem[wit={ceteri}]{kāminī}
  \rdg[wit={D}]{kāmibī}}
\app{\lem[wit={ceteri},alt={tuṅga°}]{tuṅga}
  \rdg[wit={U1}]{tuṃ°}
}kucamaṇḍalopari
\app{\lem[wit={ceteri}]{dadhāti}
  \rdg[wit={N1}]{dadhāvati}
  \rdg[wit={N2}]{dadhovati}}/\linelabel{_146e}
%------------------------------ 
%ekaṃ phalaṃ mṛtamanuṣyopari   kṣipyate/  ayaṃ vastunaḥ svabhāvaḥ/  tathā eka  evātmā   svīyabhāvād evāṣṭau    bhogān  bhunakti/ \E
%ekaṃ phalaṃ mṛtamanuṣyopari   kṣipyate   ayaṃ vastunaḥ svabhāvaḥ   tathā eka  evātmā   svīyabhāvād evāṣṭau    bhogān  bhunakti \P
%ekaṃ phalaṃ mṛtamanuṣyopari   kṣapyate// ayaṃ vastunaḥ svabhāvaḥ/  tathā eka  evātmā   svabhāvād   evāṣṭau    bhogān  bhunakte// \B
%ekaṃ phalaṃ mṛtamanuṣyopari  kṣipyate//  ayaṃ vastunaḥ svabhāvaḥ   tathā eka  evātmā   svabhāvād   evāṣṭau    bhogān  bhunakte// \L
%ekaphalaṃ   mṛtamanuṣyopari   kṣipyate// ayaṃ vastunaḥ svabhāvaḥ/  tathā eka  evātmā   svīyabhāvād evāṣṭau    bhogān ābhunakti/ \N1
%ekaphalaṃ   mṛtamanuṣyopari  kṣipyate//  ayaṃ vastunaḥ svabhāvaḥ// tathā eka  evātmā   svīyabhāvād evāṣṭau    bhogān  bhunakti// \D
%ekaphalaṃ   mṛtamanuṣyopari   kṣipyate/  ayaṃ castunaḥ svabhāvaḥ/  tathā eka  evātmā    svīyabhāvād evāstau   bhogāt  bhunakti/ \N2
%ekaphalaṃ   mṛtamanuṣyopari   kṣipyate/  ayaṃ castunaḥ svabhāvaḥ/  tathā eka  evātmā    svīyabhāvād evāstau   bhogāt  bhunakti/ \U1 %%%276.jpg
%ekaṃ phalaṃ mṛtamanuṣyopari   kṣipyate// ayaṃ vastunaḥ svabhāvaḥ// tathā ekam eva ātmā svīyabhāvād evāṣṭabhogān    bhunakti// \U2
%------------------------------
%[Or] the one fruit is thrown onto a dead man. Dies ist das inhärente Wesen der Sache. So [ist es das inhärente Wesen der Sache] das [auch] eine Selbst aufgrund des eigenen Wesens die acht Genüsse genießt.  
%------------------------------
\note[type=testium, labelb=147, nosep]{cf. YSv (PT p. 837): strīpuṃrūpī mahān so hi parasparavimohitaḥ | amanaskaḥ svīyabhāvāt jñānayogī nirākulaḥ | srakcandanādivāmāsu svabhāvād bhogam icchukaḥ |}
\app{\lem[wit={B,E,L,P,U2}]{ekaṃ phalaṃ}
  \rdg[wit={X}]{ekaphalaṃ}
}mṛtamanuṣyopari
\app{\lem[wit={ceteri}]{kṣipyate}
  \rdg[wit={B}]{kṣapyate}}/
ayaṃ vastunaḥ svabhāvaḥ/
tathā \app{\lem[wit={ceteri}]{eka}
  \rdg[wit={U2}]{ekam}}
\app{\lem[wit={ceteri}]{evātmā}
  \rdg[wit={U2}]{eva ātmā}}
\app{\lem[wit={ceteri},alt={svīyabhāvād}]{svīyabhāvā\skp{d-e}}
  \rdg[wit={B,L}]{svabhāvād}
}\app{\lem[wit={ceteri},alt={evāṣṭau}]{\skm{d-e}vāṣṭau}
  \rdg[wit={N2,U1}]{evāstau}
  \rdg[wit={U2}]{evāṣṭa}}
\app{\lem[wit={ceteri}, alt={bhogān}]{bhogā\skp{n-bhu}}
  \rdg[wit={N2,U1}]{bhogāt}
}\app{\lem[wit={ceteri}, alt={bhunakti}]{\skm{n-bhu}nakti}
  \rdg[wit={N1}]{ābhunakti}}/
%------------------------------
%ke te ṣṭau  bhogāḥ – suvāsaś ca   suvastrañ ca  suśayyā    sunitaṃbinī/       susthānañ cānnapānāni    aṣṭau bhogāś ca dhīmatām/      \E
%ke te ṣṭau  bhogauḥ  suvāsaś ca   suvāsaś   ca  suyyā      sunitāṃbinīḥ//     susthānāś cānpanānp------aṣṭau bhogāś ca dhīmatāṃ 1     \P %%%7643.jpg
%      aṣṭau bhogāḥ   suvāsac ca   suvasaś   ca  suśayyāḥ   sūnitaṃbinī/       susthānaś vānnapānāny----aṣṭau bhogāś cā sudhīmatām//1//\B
%      aṣṭau bhogāḥ   suvāsaś ca   suvāsaś   ca  suśayyāḥ   sūnitaṃbinī//      susthānāś cānnapānāny----aṣṭau bhogāś cā sudhīmatāṃ//1// \L
%ke te ṣṭau  bhogāḥ// suvāyaś ca/                suśayyā    sunitaṃbinī/       susthātāś cātmapanasyā----ṣṭau bhogāḥ    sudhipaṇa\N1
%ke te ṣṭau  bhogāḥ// suvāyaś ca//               suśayyā    sunittaṃbinī//     susthātāś cānmanasyā------ṣṭau bhogāḥ    sudhiṣaṇa \D
%ke te ṣṭau  bhogāḥ   suvāyaś ca                 suśayya    sunitaṃbinī/       susthānāś cānmanasyā------ṣṭau bhogāḥ    sudhiyane \N2
%ke te ṣṭe   bhogā –  suvāsaś ca                 suśayyā ca sunītavinīta       susthātāś cānnapānaḥ syādaṣṭau bhogāḥ   sudhiṣaṇāṃ\U1
%ke te aṣṭau bhogā // suvāsaś ca// suvaṃśaś ca// suśayyā//  sunitaṃbinī//      sudehaṃ// sukhasaṃtānaṃ// abhayādicāṣṭakaṃ//  \U2
%------------------------------
%What are the eight enjoyments? A nice perfume, good clothing, a good bed, a beautiful womna, a nice dwelling, food & drink. Those are the eight enjoyments of the wise. Clothes made from silk.
%------------------------------
\app{\lem[wit={ceteri}]{ke te}
  \rdg[wit={B,L}]{\om}}
\app{\lem[wit={ceteri}]{'ṣṭau}
  \rdg[wit={B,L}]{aṣṭau}
  \rdg[wit={U1}]{ṣṭe}}
\app{\lem[wit={ceteri}]{bhogāḥ}
  \rdg[wit={P}]{bhobauḥ}
  \rdg[wit={U1,U2}]{bhogā}}\dd{}
\end{prose}
\begin{tlg}[22_1]
  \noindent
\tl{\app{\lem[wit={ceteri},alt={suvāsaś ca}]{suvāsaś-ca}
  \rdg[wit={B}]{suvāsac ca}}
\app{\lem[wit={E},alt={suvastrañ ca}]{suvastrañ-ca}
  \rdg[wit={U2}]{suvaṃśaś ca}}
\app{\lem[wit={ceteri}]{suśayyā}
  \rdg[wit={U1}]{suśayyā ca}
  \rdg[wit={B,L}]{suśayyāḥ}
  \rdg[wit={P}]{suyyā}}
\app{\lem[wit={ceteri}]{sunitaṃbinī}
  \rdg[wit={P}]{sunitāṃbinīḥ}
  \rdg[wit={U1}]{sunītavinīta}}/}\\
\tl{\app{\lem[wit={E},alt={susthānañ ca}]{susthāna\skp{ñ-cā}}
  \rdg[wit={P,L,N2}]{susthānāś}
  \rdg[wit={D,N1,U1}]{susthātāś}
  \rdg[wit={U2}]{sudehaṃ}
}\app{\lem[wit={L},alt={°ānnapānāny}]{\skm{ñ-cā}nnapānā\skp{ny-a}}
  \rdg[wit={B}]{vānnapānāny}
  \rdg[wit={E}]{cānnapānāni}
  \rdg[wit={P}]{cānpanānp°}
  \rdg[wit={N1}]{cātmapanasyā°}
  \rdg[wit={D,N2}]{cānmanasyā°}
  \rdg[wit={U1}]{cānnapānaḥ syād°}
  \rdg[wit={U2}]{sukhasaṃtānaṃ}
}\app{\lem[wit={E,P},alt={aṣṭau bhogāś ca dhīmatām}]{\skm{ny-a}ṣṭau bhogāś-ca dhīmatām}
  \rdg[wit={B,L}]{aṣṭau bhogāś cā sudhīmatām}
  \rdg[wit={N1}]{ṣṭau bhogāḥ sudhipaṇa°}
  \rdg[wit={D}]{ṣṭau bhogāḥ sudhiṣaṇa°}
  \rdg[wit={U1}]{aṣṭau bhogāḥ sudhiṣaṇāṃ}
  \rdg[wit={N2}]{aṣṭau bhogāḥ}
  \rdg[wit={U2}]{abhayādicāṣṭakaṃ}}\dd{} \begin{otherlanguage}{english}\uproman{22}.1\end{otherlanguage} \dd{}}
\end{tlg}
  \end{edition}
  \begin{translation}
    \begin{tlate}[p22_02]
      \noindent
      \ldots sometimes mottled, sometimes like various fruit, sometimes like flowers, sometimes like the nectar of immortality, [and that only] because of its inherent being. In this way, the self also takes the form of a human, a bird, a gazelle, an elephant, a Vidyādhara, a Gandharva, a centaur, a great scholar or a great fool, a sick or healthy, an angry or peaceful person, by virtue of its inherent nature. Through Jñānayoga he realizes the emptiness of the mutability of form. Just as the place of origin of the fruit is only one, but the fruit's actions and outcomes are seen as manifold.\\

      One fruit falls onto the ground and becomes dry: A bee drinks the fruit's juice; [or] a woman places a garland made of the fruit over her voluptuous bosom; [or] the fruit is thrown onto a dead person. This is the inherent being of the thing. Thus [in the same way], due to its being, the one self enjoys eight enjoyments.\footnote{The passage appears to describe a single object's multifaceted nature, using fruit as an example. The author suggests that even though the origin of the fruit is singular, the actions and outcomes that can arise from it are many and diverse. This can be seen in the various examples, such as the fruit falling onto the ground, a bee drinking the fruit's juice, a woman adorning herself with a garland made of the fruit, or the fruit being thrown onto a dead person. The final statement about the self enjoying eight pleasures suggests that just as the fruit can have different outcomes and experiences, the self can also have different experiences and enjoyments. Overall, the passage emphasizes the multifaceted and diverse nature of things and experiences. Additionally, the passage suggests that the eight pleasures are perfectly natural to Rāmacandra and his readership.}
\\
What are the eight pleasures?\footnote{I was not able to allocate the source of the \textit{aṣṭau bhogāḥ} yet. In the \textit{Mānasollāsa} of King Someśvara, one finds the mention of twenty royal \textit{upabhoga}s, which, however, includes all of the eight pleasures in greater detail \parencite[5]{manasollasa}. This alludes to the possibility of an exceptional wealthy lifestyle of Rāmacandra's audience.}
\end{tlate}
\begin{tlate}[22_1]
  \paragraph{\uproman{22}.1} A good perfume, fine clothing, a good bed, a beautiful women, a good dwelling (\textit{susthāna}) food and drink.\footnote{Suprisingly, the verse only gives seven enjoyments. What is lacking in comparison to the list given a little later is the horse.} Those are the eight enjoyments of the wise.\footnote{Right after the list presented on the next page, Rāmacandra teaches that the eight enjoyments cause suffering and attachment. However, the end of the verse with \textit{aṣṭau bhogāś ca dhīmatām}, ``the eight enjoyments of the wise/clever person'' suggests a rather positive connotation.}
 % \flushpage
    \end{tlate}
  \end{translation}
\end{alignment}
\pagebreak %after pp. 47-48
\chapter{Appendix}
\section{Figures}

% \begin{landscape}
\clearpage
  \begin{figure}[ht]
	\centering
  \includegraphics[width=1\textwidth]{pics/Vishnu_Vishvarupa_cropped.jpg}
	\caption{Viṣṇu Viśvarūpa, India, Rajasthan, Jaipur, ca. 1800–1820, Opaque watercolor and gold on paper, 38.5 × 28 cm, Victoria and Albert Museum, London, Given by Mrs. Gerald Clark.}
	\label{fig1}
      \end{figure}
\clearpage
  \begin{figure}[ht]
	\centering
  \includegraphics[width=0.5\textwidth]{pics/The_Equivalence_of_Self_and_Universe_(detail),_folio_6_from_the_Siddha_Siddhanta_Paddhati,_(Bulaki),_1824_(Samvat_1881);_122_x_46_cm._Mehrangarh_Museum_Trust..jpg}
	\caption{The Equivalence of Self and Universe (detail), folio 6 from the \textit{Siddhasiddhāntapaddhati} (Bulaki), India, Rajasthan, Jodhpur, 1824 (Samvat 1881), 122 x 46 cm, RJS 2378, Mehragarh Museum Trust.}
	\label{fig2}
      \end{figure}
      % \end{landscape}


\chapter{Bibliography}
 \label{sec:bibli}
   \clearpage
\newpage 
\thispagestyle{empty}
\quad  \addtocounter{page}{-1}

\printbibliography[heading=subbibintoc, title=Consulted Manuscripts, keyword=codex]

\printbibliography[heading=subbibintoc, title=Printed Editions, keyword=printsource]

\printbibliography[heading=subbibintoc, title=Secondary Literature, keyword=seclit]

\printbibliography[heading=subbibintoc, title=Online Sources, keyword=onlinesource]

\end{document}
