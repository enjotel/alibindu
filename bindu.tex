\input{preamble.tex}
\FormatDiv{1}{\begin{center}\Large}{\end{center}}
\FormatDiv{2}{\begin{center}\small}{\end{center}}
\FormatDiv{3}{\bfseries}{.}
\title{Yogatattvabindu of Rāmacandra\\ A Critical Edition and Annotated Translation together with a Comparative Analysis of the \\Complex Early Modern Yoga Yaxonomies}
\date{\today}

\parindent=15pt
\begin{document}

%Zitiermöglichkeiten:
%\footcite[See][p.\,1]{goldstein01:_tibet_englis_diction_moder_tibet}
%\footnote{\cite{goldstein01:_tibet_englis_diction_moder_tibet}.}

\frontmatter
\thispagestyle{empty}
\begin{center}
  {\Large \emph{The Yogatattvabindu}}\\[3mm]
\end{center}



\newpage

\

\thispagestyle{empty}



\normalsize


\newpage


\begin{center}
\thispagestyle{empty}

\

\vskip 2mm

\begin{otherlanguage}{iast}
\LARGE \sanskritfont{Yogatattvabindu}
\end{otherlanguage}

\vskip .4cm

\Huge Yogatattvabindu \\[7mm]
\Large Critical Edition\\
and annotated Translation\\
together with a Comparative Analysis of the \\Complex Early Modern Yoga Yaxonomies 


\large

\vspace{3cm}

By

Nils Jacob Liersch
\small
\vfill

\vfill

Indica et Tibetica Verlag \\ % $\cdot$ 
Marburg 2024

\vskip 6mm

\end{center}

\newpage
\newpage \ \thispagestyle{empty}
\small  \

\noindent

\
\vfill


\small
\noindent \textbf{Bibliographische Information Der Deutschen Bibliothek}

\noindent
Die Deutsche Bibliothek verzeichnet diese Publikation in der Deutschen Nationalbibliographie;
detaillierte bibliographische Informationen sind im Internet über http://dnb.ddb.de abrufbar.

\noindent
\textbf{Bibliographic information published by Die Deutschen Bibliothek}

\noindent
Die Deutsche Bibliothek lists this publication in the Deutsche Nationalbibliographie; detailed
bibliographic data is available in the Internet at http://dnb.ddb.de.  


\vskip 1cm

\noindent
\copyright\ Indica et Tibetica Verlag, Marburg 2024

\medskip

\noindent
Alle Rechte vorbehalten / All rights reserved

\medskip

\noindent
Ohne ausdrückliche Genehmigung des Verlages ist es nicht gestattet, das Werk oder einzelne Teile
daraus nachzudrucken, zu vervielfältigen oder auf Datenträger zu speichern.

\smallskip

\noindent
Apart from any fair dealing for the purpose of private study, research, criticism or review, no
part of this book may be reproduced or translated in any form, by print, photo form, microfilm, or
any other means without written permission. Enquiries should be made to the publishers.

\bigskip

\noindent
Satz: \ \ Nils Jacob Liersch \\
Herstellung: \ \ BoD – Books on Demand GmbH, Norderstedt  \\

\bigskip

\noindent
%\ISBN     

\normalsize

\newpage

%\maketitle
\clearpage
\tableofcontents
\addtocounter{page}{-1}
\thispagestyle{empty}
\clearpage


\mainmatter

\chapter{Conventions in the Critical Apparatus}
\section{Sigla in the Critical Apparatus}

\begin{itemize}
\item E : Printed Edition
\item P : Pune BORI 664
\item L : Lalchand Research Library LRL5876
\item B : Bodleian Oxford D 4587
‚\item \None : NGMPP B 38-31
\item \Ntwo : NGMPP B 38-35 / A 1327-14
\item \Done : IGNCA 30019
\item \Uone : SORI 1574
\item \Utwo : SORI 6082
\end{itemize}

\chapter{Critical Edition \& Annotated Translation}
\cleardoublepage
\begin{alignment}[
  texts=edition[class="edition"];
  translation[class="translation"],
  ]
  \begin{edition}
%-----------------------------
%svātmā  sthiratvam  āyāti   tasmai śrīgurave  namaḥ/    nānāviplava--viśrāntiḥ  kathanāt  kurute tataḥ// 4// \E
%svātmā/ sthiratvam  āyāti   tasmai śrīgurave  namaḥ     nānāvikalpaḥ viśrāṃtiḥ  kathanāt  kurute tu yaḥ      \P einziges daṇḍa im Text! 
%svātmā  sthiraṃtvam āyāti   tasmai śrīgurave  namaḥ/    nānāvikalpa--viśrāṃti   kathanāt/ kurute tu yaḥ/   \B
%svātmā  sthiratvam  āyāti   tasmai śrīgurubho namaḥ//   nānāvikalpa--viśrāṃti   kathanāt  kurute tu yaḥ//   \L
%svātmā  sthiratvam  āyāti   tasmai śrīgurave  namaḥ/    nānāvikalpa--viśrāṃtiṃ  kathanāt  kurute tu saḥ/   \N1
%svātmā  sthiratvam  āyāti   tasmai śrīgurave  namaḥ/    nānāvikalpa--viśrāṃti   kathanāt  kurute tu saḥ//5// \N2
%svātmā  sthiratvam  āyāti   tasmai śrīgurave  namaḥ//   nānāvikalpaṃ viśrāṃtiṃ  kathanāt  kurute tu saḥ/      \D
%svātmā  sthiratvam  āyāti   tasmai śrīguru namo namaḥ// nānāvikalpaviśrāntiṃ    kathanāt  kurute tu yaḥ// \U2
%\om \U1
%-----------------------------
%... the own self goes into stability, homage to that teacher, who brings all doubts to stop because of [his] advice.   
%-----------------------------
\begin{tlg}[58_5]
   \noindent
 \note[type=source, labelb=410, labele=_410e, nosep]{ ≈  SSP 5.66cd (Ed. pp. 101): nānāvikalpaviśrāntiṃ kathayā kurute tu yaḥ | sadguruḥ sa tu vijñeyo na tu mithyāviḍambakaḥ|}
 \note[type=source, labelb=411, labele=_410e, nosep]{ ≈  YSv (PT p. 848): nānāvikalpavibhrāntināśañ ca kurute tu yaḥ | sadguruḥ sa tu vijñeyo na tu vairaprakalpakaḥ |}
\tl{
\app{\lem[wit={N1,U2}]{nānāvikalpaviśrāntiṃ}
  \rdg[wit={D}]{nānāvikalpaṃ viśrāṃtiṃ}
  \rdg[wit={E}]{nānāviplavaviśrāntiḥ}
  \rdg[wit={P}]{nānāvikalpaḥ viśrāṃtiḥ}
  \rdg[wit={B,L}]{nānāvikalpaviśrāṃti}
  \rdg[wit={N2}]{nānāvikalpaviśrāṃti}}
\app{\lem[wit={ceteri},alt={kathanāt}]{kathanā\skp{t-ku}}
  \rdg[wit={B}]{kathanāt |}
}\skm{t-ku}rute
\app{\lem[wit={B,L,P,U2}]{tu yaḥ}
  \rdg[wit={E}]{tataḥ}
  \rdg[wit={D,N1,N2}]{tu saḥ}}/}\\
%-----------------------------
%sadguruḥ sa tu vijñeyo na tu vai priyajalpakaḥ// 5// \E
%sadguruḥ sa tu vijñeyo na tu vipriyajalpakaḥ      5 \P
%sadguruḥ sa tu vijño   nnu   viprāyajalākaḥ// 5//        \B
%sadguruḥ sa tu vijño   nnu   viprāyajalākaḥ// 5//        \L
%sadguruḥ sa tu vijñeyo na tu vipriyajalpakaḥ//       \N1
%sadguruḥ sa tu vijñeyo na tu vipriyajalpakaḥ//6//    \N2
%sadguruḥ sa tu vijñeyo na tu vipriyajalpakaḥ/       \D
%sadguruḥ sa tu vijñeyo na tu vipriyajalpakaḥ// \U2
%\om \U1
%-----------------------------
%He is known to be true teacher, not an unpleasant disputant.  
%-----------------------------
\tl{
  sadguruḥ sa tu
  \app{\lem[wit={ceteri}]{vijñeyo}
    \rdg[wit={B,L}]{vijño}}
  \app{\lem[wit={ceteri}]{na tu}
    \rdg[wit={B,L}]{nnu}}
\app{\lem[wit={ceteri}]{vipriyajalpakaḥ}
  \rdg[wit={B,L}]{viprāyajalākaḥ}
  \rdg[wit={E}]{vai priyajalpakaḥ}}\dd{} \begin{otherlanguage}{english}\uproman{58}.5\end{otherlanguage}\hskip-2pt\dd{}}\\
\linelabel{_410e}
\end{tlg}
\begin{prose}[p58_02]
      \noindent
%-----------------------------
%ata eva paramapadasya prāpty arthaṃ sadguruḥ sevyaḥ sarvadā           yaḥ puruṣaḥ satyavādī bhavati/ \E
%ata eva paramapadasya prāpty arthaṃ sadguruḥ sevyaḥ sarvadā           yaḥ puruṣaḥ satyavādī bhavati  \P %%%7677.jpg
%ata eva paramapada----prāpty arthaṃ sadguruḥ sevya--sarvadā/          yaḥ puruṣaḥ satyavādī bhavati/ \B
%ata eva paramapada----prāpty arthaṃ sadguruḥ sevya--sarvadā//         yaḥ puruṣaḥ satyavādī bhavatī// \L
%ata eva paramapada----prāpty arthaṃ sadguruḥ        sarvadā vaṃdyaḥ?/ yaḥ puruṣaḥ satyavādī bhavati/ \N1
%ata eva paramapada----prāpty arthaṃ sadguruḥ        sarvadā vaṃdyaḥ/  yaḥ puruṣaḥ satyavādī bhavati/ \N2 S. 11 verso Zeile 6. 
%ata eva paramapada----prāpty arthaṃ sadguruḥ        sarvadā vaṃdyaḥ/  yaḥ puruṣaḥ satyavādī bhavati \D
%ata eva paramapada----prāpty arthaṃ sadguruḥ sevyaḥ sarvadā           yaḥ puruṣaḥ satyavādī bhavati// \U2
%\om \U1
%-----------------------------
%Hence, the true teacher is always is to be frequented in order to attain the highest place. That person becomes a speaker of truth. 
%-----------------------------
\note[type=source, labelb=411, labele=_412e, nosep]{cf. SSP 5.67 (Ed. p. 101): ata eva paramapadaprāpty arthaṃ sa sadguruḥ sadā vandanīyaḥ |}
\note[type=source, labelb=412, labele=_412e, nosep]{cf. YSv (PT p. 848): ata eva maheśāni sadguruḥ śiva āditaḥ | satyavādī ca sacchīlo gurubhakto dṛḍhavrataḥ |}
ata eva
\app{\lem[wit={ceteri},alt={paramapadaprāpty}]{paramapadaprā\skp{pty-a}}
  \rdg[wit={E,P}]{paramapadasya prāpty}
}\skm{pty-a}rthaṃ sadguruḥ
\app{\lem[wit={D,N1,N2}]{sarvadā vandyaḥ}
  \rdg[wit={E,P,U2}]{sevyaḥ sarvadā}
  \rdg[wit={B,L}]{sevyasarvadā}}/
yaḥ puruṣaḥ satyavādī bhavati/\linelabel{_412e}
%-----------------------------
%niraṃtaraṃ gurusevā tatparo bhavati/ \E [p.79]
%niraṃtara--gurusevā rato bhavati   \P
%niraṃtaraṃ gurusevā taro bhavati/   \B
%niraṃtaraṃ gurusevā rato bhavati...   \L
%niraṃtaraṃ gurusevā rato bhavati   \N1
%niraṃtaraṃ gurusevā rato bhavati   \N2
%niraṃtaraṃ gurusevā rato bhava   \D
%niraṃtaraṃ gusevā   rato bhavati// \U2
%\om \U1
%-----------------------------
%Uninterrupted delight for frequenting the teacher arises, 
%-----------------------------11
\app{\lem[wit={ceteri}]{nirantaraṃ}
  \rdg[wit={P}]{niraṃtara°}}
\app{\lem[wit={ceteri},alt={gurusevā}]{gurusevā}
  \rdg[wit={U2}]{gusevā°}}\app{\lem[wit={ceteri}]{rato}
  \rdg[wit={B}]{taro}
  \rdg[wit={E}]{tatparo}}
\app{\lem[wit={ceteri}]{bhavati}
  \rdg[wit={D}]{bhava}}/
%-----------------------------
%yasya manasi pāpaṃ na bhavati/ \E
%yasya manasi pāpaṃ na bhavati  \P
%yasya manasi pāpa nna bhavati/ \B
%yasya manasi pāpaṃ nna bhavati/ \L
%yasya manasi pāpaṃ nna bhavati/ \N1
%yasya manasi pāpaṃ na bhavati/ \N2
%yasya manasi pāpaṃ nna bhavati/ \D
%\om \U1
%yasya manasi pāpa na bhavati// \U2
%-----------------------------
% [for him] in whose mind evil does not arise. 
%-----------------------------12
yasya manasi
\app{\lem[wit={ceteri}]{pāpaṃ}
  \rdg[wit={B}]{pāpa}}
na bhavati/ 
%-----------------------------
%svācārarataḥ   snānādiśīlo bhavati/ \E
%svācārarataḥ   snānodiśīlo bhavati \P
%svācāraratāḥ   snānādiśīlo bhavatī/ \B
%svācāraratāḥ   snānādiśīlo bhavatī/ \L
%svasyācārarato snānādiśīlo bhavati/ \N1
%svasyācārarato snānādiśīlo bhavati/ \N2
%svasyācārarato snānādiśīlo bhavati/ \D
%\om \U1
%svācārataḥ// snānādiśīlo bhavati// \U2
%-----------------------------
%[He becomes] someone who delights in good habits, habits such as ceremonial bathing etc. arise. 
%-----------------------------13
\note[type=source, labelb=413, labele=_413e, nosep]{cf. YSv (PT p. 848): svalpācāraratātmā yo dānādiśīlasaṃyutaḥ | kāpaṭyalobhavinyāsau mahāvaṃśasamudbhavaḥ |}
\app{\lem[wit={E,P}]{svācārarataḥ}
  \rdg[wit={B,L}]{svācāraratāḥ}
  \rdg[wit={U2}]{svācārataḥ ||}
  \rdg[wit={D,N1,N2}]{svasyācārarato}}
snānādiśīlo bhavati/ 
%-----------------------------
%kāpaṭyaṃ na bhavati   yasya vaṃśaparaṃparā jñāyate/ \E
%kāpaṭyaṃ na bhavati   yasya vaṃśaparaṃparā jñāyate  \P
%kāpaṭyaṃ bhavati/     yasya vaṃśaparaṃparā jñāyate/ \B
%kāpaṭyaṃ na bhavati/  yasya vaṃśaparaṃparā jñāyate... \L
%kāpaṭyaṃ nāsti/       yasya vaṃśaparaṃparā jñāyate/ \N1
%kāpaṭyaṃ nāsti/       yasya vaṃśaparaṃparā jñāyate/ \N2
%kāpaṭyaṃ nāsti/       yasya parāparaṃparā  jñāyate/ \D
%\om \U1
%kāpaṭyaṃ na bhavati// yasya vaṃśaparaṃparā jñāyate// \U2
%-----------------------------
%Deceitdoes not arise [for him] whose noble race is recognized.
%-----------------------------14-15
kāpaṭyaṃ \app{\lem[wit={E,P,L,U2}]{na bhavati}
  \rdg[wit={B}]{bhavati}
  \rdg[wit={D,N1,N2}]{nāsti}}
yasya \app{\lem[wit={ceteri}]{vaṃśaparaṃparā}
  \rdg[wit={D}]{parāparaṃparā}}
jñāyate/\linelabel{_413e}
%-----------------------------
%etādṛśasya sadguroḥ saṃgatiḥ karttavyā  tena puruṣasya manaḥ śāṃtiṃ prāpnoti/ \E
%etādṛśasya sadguroḥ saṃgati  karttavyā       puruṣasya manaḥ śāṃtiṃ prāpnoti  \P
%etādṛśasya sadguroḥ saṃgatī  karttavyā/      puruṣasya manaḥ śāṃti  prāpnoti/ \B
%etādṛśasya sadguroḥ saṃgatī  karttavyā       puruṣasya manaḥ śāṃti  prāpnoti... \L
%etādṛśasya sadguroḥ saṃgatiḥ  kattavyāḥ      puruṣasya manaḥ śāṃtiṃ  prāpnoti/ \N1
%etādṛśasya sadguroḥ saṃgati   karttavyāḥ     puruṣasya manaḥ śāṃtiṃ  prāpnoti// \N2
%etādṛśasya sadguroḥ saṃgatiḥ  kattavyāḥ/     puruṣasya manaḥ śāṃtiṃ  prāpnoti/ \D
%etādṛśasya guroḥ    saṃgatiḥ  karttavyā//    puruṣasya mano śāṃtiṃ  prāpnoti// \U2
%\om \U1
%-----------------------------
%One shall associate with such a true teacher. The mind of such a person attains peace.
%-----------------------------16-17
\note[type=source, labelb=414, labele=_414e, nosep]{cf. YSv (PT p. 848): īdṛśaḥ sadgurus tasya saṅgatau yatnavān bhavet | tad eva manasaḥ śāntiṃ prāpnoti paramaṃ padam |}
etādṛśasya
\app{\lem[wit={ceteri}]{sadguroḥ}
  \rdg[wit={U2}]{guroḥ}}
  \app{\lem[wit={D,E,N1,U2}]{saṃgatiḥ}
    \rdg[wit={P,N2}]{saṃgati}
    \rdg[wit={B,L}]{saṃgatī}}
  \app{\lem[wit={Y}]{karttavyā}
    \rdg[wit={D,N1}]{kattavyāḥ}
    \rdg[wit={N2}]{karttavyāḥ}}
  \app{\lem[wit={E}]{tena}
    \rdg[wit={ceteri}]{\om}}/
  puruṣasya
  \app{\lem[wit={ceteri}]{manaḥ}
    \rdg[wit={U2}]{mano}}
  \app{\lem[wit={ceteri}]{śāntiṃ}
    \rdg[wit={B,L}]{śāṃti}}
  prāpnoti/\linelabel{_414e}\\\\
%-----------------------------
%atha ca yasya manomadhye sthira ānanda utpadyate  so pi sadguruḥ kathyate/ \E
%atha ca yasya manomadhye sira   ānanda utpadyate  so pi sadguruḥ kathyate \P
%atha ca yasya manomadhye sīraḥ  ānaṃda utpadyate  so pi sadguruḥ kathyate... \B
%atha ca yasya manomadhye sīraḥ  ānaṃda utpadyate  so pi sadguruḥ kathyate \L
%atha ca yasya manomadhye sthira ānanda utpadyate  so pi sadguruḥ kathyate \N1
%atha ca yasya manomadhye sthīrānaṃda   utpadyate/ so pi sadguruḥ kathyate// \N2
%atha ca yasya manomadhye sthira ānaṃda utpadyate/ so pi sadguruḥ kathyate/ \D
%atha ca       manomadhye sthira ānanda utpadyate//so pi sadguruḥ kathyate// \U2
%\om \U1
%-----------------------------
%Furthermore, he in whose mind steady bliss arises is surely called a true teacher. 
%-----------------------------18-19
  atha ca \app{\lem[wit={ceteri}]{yasya}
    \rdg[wit={U2}]{\om}}
  manomadhye
  \app{\lem[wit={D,E,N1,N2}]{sthira}
    \rdg[wit={B,L}]{sīraḥ}
    \rdg[wit={P}]{sira}
    \rdg[wit={N2}]{sthīrā°}} ānanda utpadyate/ so-'pi sadguruḥ kathyate/ 
%-----------------------------
% \om                                                                                                           \E
%atha ca                ghaṭikārdhaṃ   ghaṃṭikā   caturthāṃśo       vā  yasya pārśvam upaviṣṭe satyatādṛṣo bhāvo manomadhya utpadyate    \P
%atha ca                ghaṭikārdhaṃ   ghaṃṭikā   caturthāṃśo       vā  yasya pārśvam upaviṣṭe satyatādṛṣo bhāvo manomadhya uppapadyate/ \B
%atha ca                ghaṭikārdhaṃ   ghaṭikā    caturthaṃśo       vā  yasya pārśvam upaviṣṭe satyetādṛṣo bhāvo manomadhya upapadyate/  \L
%atha ca ghaṭī .. .. mo ghaṭikārddhaṃ  ghaṭikāyāḥ caturtho ḍaṃśo    vā  yasya pārśvam upaviṣṭe satyetādṛśo bhavo manomadhye utpadyate/  \N1 
%atha ca ghaṭimātra-----ghaṃṭikārddhaṃ ghaṭikā----caturtho daṃśo    vā  yasya pārśvem upaviṣṭe satyatādṛṣo .. .. manomadhye utpadyate/ \N2
%atha ca ghaṭīṃ mātraṃ  ghaṭikārddhaṃ  ghaṭikāyāḥ caturtho  aṃśo    vā  yasya pārśvam upaviṣṭe satyetādṛṣo bhāvo manomadhye utpadyate/ \D %%%%p.18 verso
%atha                   ghaṭikā        ghaṭikā    caturthāṃśo       vā  yasya pārśvam upaviṣṭe satyatādṛṣo bhāvo manomadhye utpadyate//    \U2
%\om \U1
%-----------------------------
%Furthermore, the state of such a reality is generated in the mind of one who is seated at the side of him [the teacher] for a \textit{ghaṭikā}, half a \textit{ghaṭikā}, of a quarter [of a \textit{ghaṭikā}]. 
%-----------------------------
%\note[type=philcomm, labelb=416, labele=_416e, lem={atha ca \ldots so 'pi sadguruḥ kathyate}]{\getsiglum{E} omits the three sentences.}
atha
\app{\lem[wit={ceteri}]{ca}
  \rdg[wit={E,U2}]{\om}} 
\app{\lem[type=emendation, resp=egoscr]{ghaṭikāmātraṃ}
  \rdg[wit={N2}]{ghaṭimātra°}
  \rdg[wit={D}]{ghaṭīṃ mātraṃ}
  \rdg[wit={N1}]{ghaṭī....mo}
  \rdg[wit={Y}]{\om}}
\app{\lem[wit={B,L,P,N1,D}]{ghaṭikārdhaṃ}
  \rdg[wit={N2}]{°ghaṃṭikārddhaṃ}
  \rdg[wit={U2}]{ghaṭikā}
  \rdg[wit={E}]{\om}}
\app{\lem[wit={L,N2,U2},alt={ghaṭikā°}]{ghaṭikā}
  \rdg[wit={N1,D}]{ghaṭikāyāḥ}
  \rdg[wit={B,P}]{ghaṃṭikā°}}\app{\lem[wit={B,P,L,U2}]{caturthāṃśo}
  \rdg[wit={N1}]{caturtho ḍaṃśo}
  \rdg[wit={N2}]{caturtho daṃśo}
  \rdg[wit={D}]{caturtho aṃśo}
  \rdg[wit={E}]{\om}}
\app{\lem[wit={ceteri}]{vā yasya pārśvaṃ upaviṣṭe}
 \rdg[wit={E}]{\om}}
\app{\lem[wit={ceteri}]{satyatādṛṣo}
  \rdg[wit={D,L,N1}]{satyetādṛśo}
  \rdg[wit={E}]{\om}} 
\app{\lem[wit={ceteri}]{bhāvo}
  \rdg[wit={N2}]{....}
   \rdg[wit={E}]{\om}}
 \app{\lem[wit={ceteri}]{manomadhye}
   \rdg[wit={E}]{\om}} 
 \app{\lem[wit={ceteri}]{utpadyate}
  \rdg[wit={B,L}]{uppapadyate}
   \rdg[wit={E}]{\om}}/
%---------
%\om                           \E
%gatvā vanamadhye  sthiyate  gṛhaṃ tyajyate so pi sadguruḥ kathyate    \P
%gatvā vanamadhye/ sthiyate/ gṛhaṃ tyajyate so pi sadguruḥ kathyate/\B
%gatvā vanamadhye sthīyate... gṛhaṃ tyajyate so pi sadguruḥ kathyate...\L
%gatvā vanamadhye sthīyate/ gṛhaṃ tyajyate \N1
%gatvā vanamadhye sthīyate/ gṛhaṃ tyajyate \N2
%gatvā vanamadhye sthīyate/ gṛhaṃ tyajyate \D
%\om \U1
%gatvā vanamadhye  sthīyate// gṛhaṃ tyajyate// so pi sadguruḥ kathyate//  \U2
%-----------------------------
%[One who] has left the house and has gone into the forst in order to dwell there is said to be a true teacher.
%-----------------------------21-22
\app{\lem[wit={ceteri}]{gatvā vanamadhye sthīyate gṛhaṃ tyajyate}
  \rdg[wit={E}]{\om}} 
\app{\lem[wit={ceteri}, alt={so 'pi sadguruḥ kathyate}]{so-'pi sadguruḥ kathyate}
  \rdg[wit={D,E,N1,N2}]{\om}}/\linelabel{_416e}
%-----------------------------
%kasyāpi duḥkhaṃ na dīyate/ \E
%kasyāpi duḥkhaṃ na dīyate  \P
%kasyāpi duḥkhaṃ na dīyate/ \B
%kasyāpi duḥkhaṃ na dīyate... \L
%kasyāpi duḥkhaṃ na dīyate  \N1
%kasyāpi duḥkhaṃ na dīyate//  \N2
%kasyāpi duḥkhaṃ na dīyate//  \D
%\om \U1
%kasyāpi duḥkhaṃ na dīyate/ \U2
%-----------------------------
% one who does not harm anyone,  
%-----------------------------23
kasyāpi duḥkhaṃ na dīyate
%-----------------------------
%prāṇimātreṇa saha maitrī kriyate  kasyāpi doṣaṃ na kathayati         so pi sadguruḥ kathyate// \E
%prāṇimātreṇa saha maitrī kriyate  kasyāpi doṣo  na prakāśyate        so pi sadguruḥ kathyate  \P %%%7678.jpg
%prāṇimātreṇa saha maitrī krīyate/ kasyāpi doṣau na prākāśate/        so pi sadguruḥ kathyate/  \B
%prāṇimātreṇa saha maitrī krīyate  kasyāpi doṣo  na prākāśate         so pi sadguruḥ kathyate \L
%prāṇimātreṇa saha maitrī krīyate/ kasyāpi doṣo  na prākāśyate/  yena so pi sadguruḥ kathyate// \N1
%prāṇimātreṇa saha maitrī yate/    kasyāpi doṣaṃ na prakāśyate// yena so pi sadguruḥ kathyate// \N2
%prāṇimātreṇa saha maitrī krīyate/ kasyāpi doṣo  na prākāśyate/  yena so pi sadguruḥ kathyate/ \D
%\om \U1
%prāṇimātre   saha maitrī kriyate// kasyāpi doṣo na prakāśyate//      so pi sadguruḥ kathyate// \U2 %%%430.jpg
%-----------------------------
%One who practices loving kindness towards living beings. one who will not expose anyone's badness. He is said to be a true teacher.
%-----------------------------24-26
\app{\lem[wit={ceteri}]{prāṇimātreṇa}
  \rdg[wit={U2}]{prāṇimātre}}
saha maitrī \app{\lem[wit={ceteri}]{krīyate}
  \rdg[wit={N2}]{yate}}
kasyāpi
\app{\lem[wit={E,N2}]{doṣaṃ}
  \rdg[wit={P,L,N1,D,U2}]{doṣo}
  \rdg[wit={B}]{doṣau}}
na
\app{\lem[wit={P,N2}]{prakāśyate}
  \rdg[wit={D,N1}]{prākāśyate}
  \rdg[wit={B,L}]{prākāśate}
  \rdg[wit={E}]{kathayati}}
\app{\lem[wit={Y}]{so}
  \rdg[wit={D,N1,N2}]{yena so}
}'pi sadguruḥ kathyate/
\end{prose}
  \end{edition}
  \begin{translation}
    \begin{tlate}[57_5]
 \paragraph{\uproman{58}.5} One who brings about the end of [my] various discursive thoughts through his teachings should be known as the true teacher and not an unpleasant blowhard.\\\\
  \end{tlate}
    \begin{tlate}[p58_02]
      \noindent
      Hence, the true teacher is always to be praised in order to attain the supreme place. The person who is a speaker of truth, [for him] uninterrupted delight in serving the teacher, arises. In whose mind there is no evil, he is devoted to good customs, [and] practices [such as] ceremonial bathing, etc. He who knows his noble lineage, he is not deceitful.\footnote{In the light of my hypothesis of Rāmacandra addressing an audience of \textit{kṣatriya}s I translate \textit{vaṃśa} here as ``noble lineage''.} He shall associate with a true teacher of such a kind. The mind of the person attains peace.\\\\
      \indent
      Furthermore, he, in whose mind steady bliss arises, he alone is called a true teacher. Thus, a state characterized by seeing the truth arises in the mind of whom is seated at the side [of a true teacher] for merely a \textit{ghaṭikā}\footnote{One \textit{ghaṭikā} equals 1/60 of a day (cf. \citeauthor[1966: 114]{sircar1966}). 1/60 of a day corresponds to 24 minutes. A day has 1440 minutes (24 hours x 60 minutes), so 1/60 of a day corresponds to 24 minutes.}, half a \textit{ghaṭikā}\footnote{12 minutes.}, or a quarter [of a \textit{ghaṭikā}]. Having gone forth, he who dwells in the forest [and] abandons his home is called the true teacher. He is called a true teacher who does not induce suffering to others, [who] bestows friendliness towards living beings, [who] will not expose anyone's badness.
  \flushpage
\end{tlate}
  \end{translation}
\end{alignment}
\pagebreak %after pp. 137-138
%%%%%%%%%%%%%%%%%%%%%%%%%%%%%%%%%%%%%%%%%%
%%%%%%%%%%%%%%%%%%%%%%%%%%%%%%%%%%%%%%%%%% 
%%%%%%%%PAGEBREAK%%%%%%%PAGEBREAK%%%%%%%%%
%%%%%%%%%%%%%%%%%%%%%%%%%%%%%%%%%%%%%%%%%% 
%%%%%%%%%%%%%%%%PAGEBREAK%%%%%%%%%%%%%%%%%
%%%%%%%%%%%%%%%%%%%%%%%%%%%%%%%%%%%%%%%%%% 
%%%%%%%%PAGEBREAK%%%%%%%PAGEBREAK%%%%%%%%%
%%%%%%%%%%%%%%%%%%%%%%%%%%%%%%%%%%%%%%%%%% 
%%%%%%%%%%%%%%%%%%%%%%%%%%%%%%%%%%%%%%%%%% 
%%%%%%%%%%%%%%%%%%%%%%%%%%%%%%%%%%%%%%%%%% 
%%%%%%%%%%%%%%%%%%%%%%%%%%%%%%%%%%%%%%%%%% 
%%%%%%%%PAGEBREAK%%%%%%%PAGEBREAK%%%%%%%%%
%%%%%%%%%%%%%%%%%%%%%%%%%%%%%%%%%%%%%%%%%% 
%%%%%%%%%%%%%%%%PAGEBREAK%%%%%%%%%%%%%%%%%
%%%%%%%%%%%%%%%%%%%%%%%%%%%%%%%%%%%%%%%%%% 
%%%%%%%%PAGEBREAK%%%%%%%PAGEBREAK%%%%%%%%%
%%%%%%%%%%%%%%%%%%%%%%%%%%%%%%%%%%%%%%%%%% 
%%%%%%%%%%%%%%%%%%%%%%%%%%%%%%%%%%%%%%%%%% 
%%%%%%%%%%%%%%%%%%%%%%%%%%%%%%%%%%%%%%%%%% 
%%%%%%%%%%%%%%%%%%%%%%%%%%%%%%%%%%%%%%%%%% 
%%%%%%%%PAGEBREAK%%%%%%%PAGEBREAK%%%%%%%%%
%%%%%%%%%%%%%%%%%%%%%%%%%%%%%%%%%%%%%%%%%% 
%%%%%%%%%%%%%%%%PAGEBREAK%%%%%%%%%%%%%%%%%
%%%%%%%%%%%%%%%%%%%%%%%%%%%%%%%%%%%%%%%%%% 
%%%%%%%%PAGEBREAK%%%%%%%PAGEBREAK%%%%%%%%%
%%%%%%%%%%%%%%%%%%%%%%%%%%%%%%%%%%%%%%%%%% 
%%%%%%%%%%%%%%%%%%%%%%%%%%%%%%%%%%%%%%%%%%
\begin{alignment}[
  texts=edition[class="edition"];
  translation[class="translation"],
  ]
  \begin{edition}
%-----------------------------
%ajñātakulaśīlānāṃ yatīnāṃ brahmacāriṇām/   upadeśaṃ na  gṛhṇīyād anyathā narakaṃ dhruvam// \E %[p.80]
%ajñātakulaśīlānāṃ yatīnāṃ brahmacāriṇām    upadeśo  na  gṛhṇīyād anyathā narakaṃ dhruvam 1 \P
%ajñānakulaśilānāṃ yatīnāṃ brahmacāriṇām    upadeśa  na  gṛhītyāsthā/ yadānyathā na narakaṃ dhruvaṃ// 1// \B
%ajñānakulaśilānāṃ yatīnāṃ brahmacāriṇām//  upadeśaṃ na  gṛhṇīyād anyathā narakaṃ dhruvaṃ// 1// \L  0040.jpg
%ajñānakulaśīlānāṃ, yatīnāṃ brahmacāriṇām/  upadeśaṃ na  gṛhnīyāt anyathā narakaṃ dhruvaṃ// \N1 %%%S.14 
%ajñātakulaśīlānāṃ yatīnāṃ brahmacāriṇām//  upadeśaṃ na  gṛhnīyāt anyathā narakaṃ dhruvaṃ//1// \N2 
%ajñānakulaśīlānāṃ yatīnāṃ brahmacāriṇām/   upadeśaṃ na  gṛhnīyāt anyathā narakaṃ dhruvaṃ/ \D
%\om \U1
%ajñātakulaśīlānāṃ yatīnāṃ brahmacāriṇām    upadeśo na gṛhṇīyāt// anyathā narakaṃ dhruvaṃ// \U2
%-----------------------------
%One should not accept the teaching of celibate ascetics whose nature and character is confused by ignorance; otherwise, hell is certain.
%----------------------27 
\begin{tlg}[58_6]
  \noindent
  \tl{\app{\lem[type=emendation, resp=egoscr, alt={ajñānā°}]{ajñānā}
      \rdg[wit={B,D,L,N1,U2}]{ajñāna°}
      \rdg[wit={E,P,N2}]{ajñāta°}
}kulaśīlānāṃ yatīnāṃ brahmacāriṇām/}\\
  \tl{\app{\lem[wit={ceteri}]{upadeśaṃ}
      \rdg[wit={P,U2}]{upadeśo}}
    na
    \app{\lem[wit={E,P,L},alt={gṛṇhīyād}]{gṛhṇīyā\skp{d-a}}
      \rdg[wit={B}]{gṛhītyāsthā |}
      \rdg[wit={ceteri}]{gṛhnīyāt}
}\app{\lem[wit={ceteri},alt={anyathā}]{\skm{d-a}nyathā}
  \rdg[wit={B}]{yadānyathā}}
\app{\lem[wit={ceteri}]{narakaṃ}
  \rdg[wit={B}]{na narakaṃ}}
\app{\lem[wit={ceteri}]{dhruvaṃ}
  \rdg[wit={E,P}]{dhruvam}}\dd{} \begin{otherlanguage}{english}\uproman{58}.6\end{otherlanguage}\hskip-2pt\dd{}}\\
\end{tlg}
    \begin{prose}[p58_03]
      \noindent
%----------------------
%yasya vacasi manasi dhṛte sati  svātmanaḥ parameśvarasyaikyaṃ bhavati/ \E
%yasya vacasi manasi dhṛte sati  svātmanaḥ parameśvarasyaikyaṃ bhavati  \P
%yasya vacasi manasi dhṛte sati  svātmanaḥ parameśvarasakyaṃ   bhavati/ \B
%yasya vacasi manasi dhṛte sati  svātmanaḥ parameśvarasakyaṃ   bhavati// \L
%yasya vacasi manasi dhṛte sati/ svātmanaḥ parameśvarasyaikyaṃ bhavati/ \N1
%yasya vacasi manasi dhṛte sati/ svātmanaḥ parameśvarasyaikaṃ  bhavati/ \N2
%yasya vacasi manasi dhṛte sati/ svātmanaḥ parameśvarasyaikyaṃ bhavati/ \D
%\om \U1
%yasya cavi          dhṛte sati  svātmanaḥ parameśvarasyaikyaṃ bhavati// \U2
%-----------------------------
%For one who is steadfast in mind and speech there arises unity of the supreme deity and the own self.  
%-----------------------------28
yasya
\app{\lem[wit={ceteri}]{vacasi}
  \rdg[wit={U2}]{cavi}}
\app{\lem[wit={ceteri}]{manasi}
  \rdg[wit={U2}]{\om}}
dhṛte
\app{\lem[wit={ceteri}]{sati}
  \rdg[wit={D,N1,N2}]{sati |}}
svātmanaḥ
\app{\lem[wit={E,P,N1,D,U2}]{parameśvarasyaikyaṃ}
  \rdg[wit={N2}]{parameśvarasyaikaṃ}
  \rdg[wit={B,L}]{parameśvarasakyaṃ}}
bhavati/ 
%-----------------------------
%etādṛśo manomadhye niścayo bhavati/ \E
%etādṛśo manomadhye niścayo bhavati \P
%etādṛśo manomadhye niścayo bhavati/ \B
%etādṛśo manomadhye niścayo bhavati// \L
%etādṛśo manomadhye niścayo bhavati \N1
%etādṛśo manomadhye niścayo bhavati \N2
%etādṛśo manomadhye niścayo bhavati/ \D
%etādṛśo manomadhye niścayo bhavati// \U2
%\om \U1
%-----------------------------
%Such certainty arises within the mind.  
%-----------------------------29
etādṛśo manomadhye niścayo bhava:\\ti/
\end{prose}
    \begin{prose}[p58_04]
      \noindent
%\note[type=philcomm, labelb=400x, labele=_400xe, lem={taṃ sadguruṃ jānīyāt \ldots yogarahasyaṃ karttavyaṃ}]{Passage possibly authorial.}
%-----------------------------
%taṃ sadguruṃ vijānīyāt vikalpa etādaśo  yathā samudramadhye mahattaraṃ kallolāḍambaram/ prapaṃce vāsanā   tādṛśī yathodakamadhye mahattaraṃgāḥ/    \E
%taṃ sadguruṃ jānīyāt   vikalpa etādṛśo  yathā samudramadhye mahattara--kallolāḍambaraṃ  prapaṃca-vāsanā  etādṛśī yathodakamadhye mahattarati        \P
%taṃ sadguruṃ jānīyāt   vikalpa etādṛśo  yathā samudramadhye mahattara--kallolāḍambara   prapaṃca-vāsanā  etādṛśī yathodakamadhye mahattarati ....   \B
%taṃ sadguruṃ jānīyāt   vikalpa etādṛśo  yathā samudramadhye mahattara--kallolāḍambara   prapaṃca-vāsanā  etādṛśī yathodakamadhye mahattarati ....   \L
%taṃ sadguruṃ jānīyāt// vikalpa etādṛśo  yathā samudramadhye mahattara--kallolāḍambaraḥ/ prapaṃca-vāsanā  etādṛśī yathodakamadhye mahattarati        \N1
%taṃ sadguruṃ jānīyāt// vikalpa etādṛśaṃ yathā samudramadhye mahattara--kallolāḍambaraḥ  prapaṃca-vāsanā  etādṛśī yathodakamadhye mahattarati        \N2 %%S.12
%taṃ sadguruṃ jānīyāt/  vikalpa etādṛśo  yathā samudramadhye mihattara--kallolāḍambaraḥ/ prapaṃca-vāsanā  etādṛśī yathodakamadhye mahattarati        \D
%taṃ sadguruṃ jānīyāt// vikalpa etādṛśo  yathā samudramadhye mahattara--kallolāḍambaraṃ  prapaca vāsanā// etādṛśī yathodakamadhye mahattarī          \U2
%\om \U1
%-----------------------------
%E: One should know this true teacher. The changing thought is like the roar of waves within the ocean. The manifold mental imprints are like the ripples in the water...
%----------------------------30-31
taṃ sadguruṃ
\app{\lem[wit={ceteri}]{jānīyāt}
  \rdg[wit={E}]{vijānīyāt}}/
vikalpa
\app{\lem[wit={ceteri}]{etādṛśo}
  \rdg[wit={N2}]{etādṛśaṃ}}
yathā samudramadhye
\app{\lem[wit={ceteri},alt={mahattara°}]{mahattara}
  \rdg[wit={D}]{mihattara}
  \rdg[wit={E}]{mahattaraṃ}
}kallolā\app{\lem[wit={ceteri},alt={°ḍambaraḥ}]{ḍambaraḥ}
  \rdg[wit={B,L}]{°ḍambara°}
  \rdg[wit={E,P,U2}]{°ḍambaraṃ}}
\app{\lem[wit={ceteri},alt={prapañca°}]{prapañca}
  \rdg[wit={U2}]{prapaca}}vāsanā 
\app{\lem[wit={ceteri}]{etādṛśī}
  \rdg[wit={E}]{tādṛśī}}
yathodakamadhye
\app{\lem[wit={E}]{mahattaraṅgāḥ}
  \rdg[wit={U2}]{mahattarī}
  \rdg[wit={ceteri}]{mahattarati}}/ 
%----------------------------
%tādṛśasya saṃsārasāgarasya   yaḥ svavākyanāvā paraṃ pāraṃ  prāpayati/  sa sadguruḥ kathyate// \E
%tādṛśya   saṃsārā---rṇāvād   yo          nāvā paraṃ        prāpayati   sa sadguruḥ kathyate  \P
%tādṛśāt   saṃsārā---rṇavavād yau         nāvā paraṃ        prāpayati/  sa sadguruḥ kathyate// \B
%tādṛśāt   saṃsārā---rṇavād   yau         nāvā paraṃ        prāpayati   sa sadguruḥ kathyate// \L
%tādṛśāt   saṃsārāt arṇavād  yo           nāvaraṃ           prāpayati/  sa sadguruḥ kathyate// \N1 %%%S.14 Z. 3
%tādṛśāt   saṃsārāt arṇavād  yo           nāvaraṃ           prāpayati/  sa sadguruḥ kathyate// \N2
%tādṛśāt   saṃsārāt aṛṇavād   yo          nāvā paraṃ        prāpayati/     sadguruḥ kathyate/ \D
%tādṛśāt   saṃsārā---rṇavād   yo          nāvā pāraṃ pāraṃ  prāpayati// sa sadguruḥ kathyate// \U2
%\om \U1
%-----------------------------
%He who causes to navigate/convey the boat from such an ocean of Saṃsāra to the other shore is called a true teacher. 
%-----------------------------32-33
\app{\lem[wit={ceteri}, alt={tādṛśāt}]{tādṛśā\skp{t-saṃ}}
  \rdg[wit={E}]{tādṛśasya}
}\app{\lem[wit={P,L,U2}, alt={saṃsārārṇavād}]{\skm{t-saṃ}sārārṇavā\skp{d-yo}}
  \rdg[wit={B}]{saṃsārārṇavavād}
  \rdg[wit={D,N1,N2}]{saṃsārāt arṇavād}
}\app{\lem[wit={ceteri},alt={yo}]{\skm{d-yo}}
  \rdg[wit={B,L}]{yau}
  \rdg[wit={E}]{yaḥ}}
\app{\lem[wit={B,L,P,D,U2}]{nāvā}
  \rdg[wit={N1,N2}]{nāvaraṃ}
  \rdg[wit={E}]{svavākyanāvā}}
\app{\lem[wit={E}]{paraṃ pāraṃ}
  \rdg[wit={U2}]{pāraṃ pāraṃ}
  \rdg[wit={B,L,P,D}]{paraṃ}
  \rdg[wit={N1,N2}]{\om}}
prāpayati/
\app{\lem[wit={ceteri}]{sa}
  \rdg[wit={D}]{\om}}
sadguruḥ ka:\\thyate/ 
%-----------------------------
%yasya puruṣasya mano  'khaṇḍe paramapade līnaṃ bhavati/ [p.81]     \E
%yasya puruṣasya mano  'khaṃḍe       pade līnaṃ bhavati/                 \P
%yasya puruṣasya manaḥ akhaṃḍe       pade līnaṃ bhavatī/ DSCN7180.JPG   \B
%yasya puruṣasya manaḥ akhaṃḍe       pade līnaṃ bhavati/                \L
%yasya puruṣasya mano   khaṃḍe   parapada-līna bhavati/                \N1
%yasya puruṣasya mano   khaṃḍe   paramada-lita bhavati/                \N2
%yasya puruṣasya mano   khaṃḍe   parapada-līnaṃ bhavati/                \D
%yasya puruṣasya mano  'khaṇḍe   parapade līnaṃ bhavati     \U2
%\om \U1
%----------------------------
%The mind of the person becomes absorbed into the indivisible supreme place.  
%-----------------------------34
yasya puruṣasya
\app{\lem[wit={ceteri}]{mano}
  \rdg[wit={B,L}]{manaḥ}
}\app{\lem[wit={ceteri}]{'khaṇḍe}
  \rdg[wit={B,L}]{akhaṃḍe}}
\app{\lem[wit={E}]{paramapade}
  \rdg[wit={D,N1}]{parapada°}
  \rdg[wit={N2}]{paramada°}
  \rdg[wit={U2}]{parapade}}
\app{\lem[wit={ceteri}]{līnaṃ}
  \rdg[wit={N1}]{°līna}
  \rdg[wit={N2}]{°lita}}
\app{\lem[wit={ceteri}]{bhavati}
  \rdg[wit={B}]{bhavatī}}/  
%-----------------------------
%yaḥ puruṣaḥ svakulaṃ     trividhāt   tāpān nivartya  parame  muktipade  rakṣati/ \E
%yaḥ puruṣaḥ svīyaṃ kulaṃ trividhāt   tapān nivartta--parama--muktipade  rakṣati \P %%%7579.jpg
%yaḥ puruṣaḥ svikulaṃ     trividhaṃ/  tāpān nivartya  para----muktipade  rakṣati/ \B
%yaḥ puruṣaḥ sviyaṃ kulaṃ trividhat   āpān nivartya   para----muktipade  rakṣati... \L
%yaḥ puruṣaḥ svīyaṃ kūlaṃ trividhāt   tāpān nivarttya parama--muktipade  rakṣati/ \N1
%yaḥ puruṣa  svīyaṃ kūlaṃ trividhāt   tāpān nivartya  paramamamuktipade  rakṣati/ \N2
%yaḥ puruṣaḥ svīyaṃ kūlaṃ trividhāt---tāpān mivarttya parama--muktipade  rakṣati/ \D
%yaḥ puruṣa  svīyaṃ kulaṃ trividhat---āpān nivartya   parama--muktipakṣe rakṣati// \U2
%\om \U1
%-----------------------------
%The person situated in the place of supreme liberation who turned away from the threefold misery (adhyātmika, adhibhautika, adhidaivika) protects the own kula (lineage? noble family? tribe?). 
%-----------------------------35
yaḥ \app{\lem[wit={ceteri}]{puruṣaḥ}
  \rdg[wit={N2,U2}]{puruṣa}}
\app{\lem[wit={ceteri}]{svīyaṃ kūlaṃ}
  \rdg[wit={B}]{svikulaṃ}
  \rdg[wit={E}]{svakulaṃ}}
\app{\lem[wit={E,D,P,N1,N2},alt={trividhā}]{trividhā}
  \rdg[wit={L,U2}]{trividhat}
  \rdg[wit={B}]{trividhaṃ |}
}\app{\lem[wit={ceteri},alt={tāpān}]{tāpā\skp{n-ni}}
  \rdg[wit={L,U2}]{āpān}}\skm{n-ni}vartya
\app{\lem[wit={P,D,N1}]{paramamuktipade}
  \rdg[wit={E}]{parame muktipade}
  \rdg[wit={N2}]{paramamamuktipade}
  \rdg[wit={B,L}]{paramuktipade}
  \rdg[wit={U2}]{paramamuktipakṣe}} rakṣati/
%-----------------------------
%etādṛśasya puruṣasya śravaṇād   darśanāt  samagravighnā naśyanti/    \E
%etādṛśā    puruṣasya śravaṇā    darśanāt  samagravighnā naśyaṃti     \P
%etādṛśā    puruṣasya śravaṇāt   darśanāt/ samagravighnā na naśyaṃtī/ \B
%etādṛśā    puruṣasya śravaṇāt   darśanāt  samagravighnā na naśyaṃti... \L
%etādṛśa/   puruṣaṃ   śravaṇād   darśanāt  samagravighnā naśyaṃti/ \N1
%etādṛśaṃ   puruṣaṃ   śravaṇād   darśanāt  samagravighnā naśyaṃti/ \N2
%etādṛśa----puruṣaṃ   śravaṇād   darśanāt  samagravighnā naśyaṃti/ \D
%etādṛśaṃ   puruṣaṃ   śravaṇād   darśanāt  samagraviśvaś ca vaśāṃ? bhavati \U1
%etādṛśa    puruṣasya śravaṇāt// darśanāt  samagravighna naśyaṃti//    \U2
%-----------------------------
%From hearing [or] from seeing about such a person all obstacles are destroyed. 
%-----------------------------36
\note[type=philcomm, labelb=417, lem={etādṛśaṃ \ldots naśyanti}]{Textual evidence of \getsiglum{U1} resumes from this sentence onwards.}
\app{\lem[wit={ceteri}]{etādṛśaṃ}
  \rdg[wit={D,U2}]{etādṛśa}
  \rdg[wit={N1}]{etādṛśa |}
  \rdg[wit={B,L,P}]{etādṛśā}
  \rdg[wit={E}]{etādṛśasya}}
\app{\lem[wit={X}]{puruṣaṃ}
  \rdg[wit={Y}]{puruṣasya}}
\app{\lem[wit={ceteri},alt={śravaṇād}]{śravaṇā\skp{d-da}}
  \rdg[wit={B,L}]{śravaṇāt}
  \rdg[wit={U2}]{śravaṇāt ||}
  \rdg[wit={P}]{śravaṇā}
}\app{\lem[wit={ceteri},alt={darśanāt}]{\skm{d-da}rśanā\skp{t-sa}}
  \rdg[wit={B}]{darśanāt |}
}\skm{t-sa}magra\app{\lem[wit={ceteri}]{vighnā}
\rdg[wit={U1}]{viśvaś ca vaśāṃ}}
\app{\lem[wit={ceteri}]{naśyanti}
  \rdg[wit={L}]{na naśyaṃti}
  \rdg[wit={B}]{na naśyaṃtī}
  \rdg[wit={U1}]{bhavati}}/ 
%-----------------------------
%dine dine kalyāṇaṃ  bhavati/ niṣkalaṃkā buddhir utpadyate/    \E
%dine dine kalyāṇaṃ  bhavati  niṣkalaṃkā buddhir utpadyate     \P
%dine dine kalyāṇaṃ  bhavati  niṣkalaṃkā buddhir utpadyate/    \B
%dine dine kalyāṇaṃ  bhavati  niṣkalaṃkā buddhir utpadyate...  \L
%dine dine kalyāṇaṃ  bhavati/ niṣkalaṃ   buddhir utpadyate//     \N1
%dine dine kalyāṇaṃ  bhavati/ niṣkalaṃ   buddhir utpadyate//     \N2
%dine dine kalyāṇaṃ  bhavati/ niṣkalaṃkā buddhir utpadyate/    \D
%dine      kalyāṇāṃ  bhavatīr niṣkalaṃkā buddhir utpadyate     \U1
%dine dine kalyāṇaṃ  bhavati/ niṣkalaṃko buddhir utpadyate// \U2
%-----------------------------
%Day by day prosperity arises. A flawless intellect arises. 
%-----------------------------37-38
\app{\lem[wit={ceteri}]{dine dine}
  \rdg[wit={U1}]{dine}}
\app{\lem[wit={ceteri}]{kalyāṇaṃ}
  \rdg[wit={U1}]{kalyāṇāṃ}}
\app{\lem[wit={ceteri}]{bhavati}
  \rdg[wit={U1}]{bhavatīr}}/
\app{\lem[wit={ceteri}]{niṣkalaṅkā}
  \rdg[wit={N1,N2}]{niṣkalaṃ}
  \rdg[wit={U2}]{niṣkalaṃko}}
buddhir-utpadyate/\end{prose}
  \end{edition}
  \begin{translation}
    \begin{tlate}[58_6]
  \paragraph{\uproman{58}.6} One should not accept the teaching of celibate ascetics whose nature is confused by ignorance; otherwise, hell is inevitable.\footnote{I could not allocate a source for this verse. The verse is possibly authorial.}\\\\
\end{tlate}
\begin{tlate}[p58_03]
  \indent
For one who is steadfast in mind and speech, unity arises between the supreme lord and the own self. Conviction of such a kind arises within the mind.  
    \end{tlate}
    \begin{tlate}[p58_04]
      \noindent
      One should know a true teacher. Such discursive thinking is like the roar of mighty waves within the ocean [and] such manifold mental residues are like great waves within the water, one who causes the boat to reach the farther shore of such an ocean of Saṃsāra is called the true teacher. That person’s mind that has dissolved into the undivided supreme state and who has turned his lineage away from the threefold miseries\footnote{The threefold misery consists of: 1. \textit{adhyātmika} (``internal''), which refers to any physical and mental misery caused by diseases; 2. \textit{adhibhautika} (``external''), which refers to any misery caused by external living beings or objects; and 3. \textit{adhidaivika}, which refers to any misery caused by the gods or comes from heaven, such as cold, heat, storm, drought, etc. For a more detailed account, cf. \citetitle{mainkar2004} (2004), especially the respective explanations provided in the \textit{bhāṣya} of Gaudapāda for the first \textit{kārikā}.} and protects [them] in the state of supreme liberation, all obstacles disappear because of listening to and seeing that person. Day by day, prosperity arises. A flawless intellect arises.%\footnote{In Sanskrit literature, it is usually the king who protects the own lineage, cf. \citetitle{ramayana} 3.36.26; \citetitle{mahabharata} 7.11.10.}
      \flushpage 
\end{tlate}
  \end{translation}
\end{alignment}
\end{document}
%%%%%%%%%%%%%%%%%%%%%%%%%%%%%%%%%%%%%%%%%%
%%%%%%%%%%%%%%%%%%%%%%%%%%%%%%%%%%%%%%%%%% 
%%%%%%%%PAGEBREAK%%%%%%%PAGEBREAK%%%%%%%%%
%%%%%%%%%%%%%%%%%%%%%%%%%%%%%%%%%%%%%%%%%% 
%%%%%%%%%%%%%%%%PAGEBREAK%%%%%%%%%%%%%%%%%
%%%%%%%%%%%%%%%%%%%%%%%%%%%%%%%%%%%%%%%%%% 
%%%%%%%%PAGEBREAK%%%%%%%PAGEBREAK%%%%%%%%%
%%%%%%%%%%%%%%%%%%%%%%%%%%%%%%%%%%%%%%%%%% 
%%%%%%%%%%%%%%%%%%%%%%%%%%%%%%%%%%%%%%%%%% 
%%%%%%%%%%%%%%%%%%%%%%%%%%%%%%%%%%%%%%%%%% 
%%%%%%%%%%%%%%%%%%%%%%%%%%%%%%%%%%%%%%%%%% 
%%%%%%%%PAGEBREAK%%%%%%%PAGEBREAK%%%%%%%%%
%%%%%%%%%%%%%%%%%%%%%%%%%%%%%%%%%%%%%%%%%% 
%%%%%%%%%%%%%%%%PAGEBREAK%%%%%%%%%%%%%%%%%
%%%%%%%%%%%%%%%%%%%%%%%%%%%%%%%%%%%%%%%%%% 
%%%%%%%%PAGEBREAK%%%%%%%PAGEBREAK%%%%%%%%%
%%%%%%%%%%%%%%%%%%%%%%%%%%%%%%%%%%%%%%%%%% 
%%%%%%%%%%%%%%%%%%%%%%%%%%%%%%%%%%%%%%%%%% 
%%%%%%%%%%%%%%%%%%%%%%%%%%%%%%%%%%%%%%%%%% 
%%%%%%%%%%%%%%%%%%%%%%%%%%%%%%%%%%%%%%%%%% 
%%%%%%%%PAGEBREAK%%%%%%%PAGEBREAK%%%%%%%%%
%%%%%%%%%%%%%%%%%%%%%%%%%%%%%%%%%%%%%%%%%% 
%%%%%%%%%%%%%%%%PAGEBREAK%%%%%%%%%%%%%%%%%
%%%%%%%%%%%%%%%%%%%%%%%%%%%%%%%%%%%%%%%%%% 
%%%%%%%%PAGEBREAK%%%%%%%PAGEBREAK%%%%%%%%%
%%%%%%%%%%%%%%%%%%%%%%%%%%%%%%%%%%%%%%%%%% 
%%%%%%%%%%%%%%%%%%%%%%%%%%%%%%%%%%%%%%%%%%
\begin{alignment}[
  texts=edition[class="edition"];
  translation[class="translation"],
  ]
  \begin{edition}
               \ekddiv{
                 head={[\uproman{59}. \textbf{yogaśāstrarahasyam}]},
                 type=section,
                 depth=2, 
                 n=LIX
               }\xmlhead[h5]{[LVIX. yogaśāstrarahasyam]}
\label{secretofscriptures}
\begin{prose}[p59_01]
\noindent
%-----------------------------
%idaṃ yogaśāstrasya rahasyaṃ samastaśāstraprame            yasya manaḥ   yathāṃdhakārasya    madhye dīpa----tejaḥ praviśati/        \E
%idaṃ yogaśāstrasya rahasyaṃ samagraśāstramadhye           yasya manaḥ   yathāṃdhakārasya    madhye dīpasya tejaḥ praviśati   \P
%idaṃ yogaśāstrarahasyaṃ/    samastaśāstramadhye     manaḥ yasya mana    yathāṃdhakārasya    madhye dīpasya tejaḥ praviśyati... \B
%idaṃ yogaśāstrarahasyaṃ//   samastaśāstramadhye     mano  yasya mana    yathāṃdhakārasya    madhye dīpasya tejaḥ praviśyati...  \L
%idaṃ yogaśāstrarahasyaṃ/    samagraśāstramadhye           yasya mana/   yathāṃdhakāras      madhye dīpasya tejaḥ praviśati/     \N1
%idaṃ yogaśāstrarahasya      samagraśāstramadhye                                                                               \N2
%idaṃ yogaśāstrarahasya/     samagraśāstramadhye           yasya mana/   yathāṃdhakāramadhye        dīpasya tejaḥ praviśati/     \D
%idaṃ yogaśāstreṣu rahasyaṃ  samagraśāstramadhye           yasya manaḥ   yathāṃdhakārasya    madhye dīpasya tejaḥ praviśyati...  \U1
%idaṃ yogaśāstrarahasyaṃ     samagraśāstramadhye           yasya manaḥ// yathāṃdhakārasya    madhye dīpasya tejaḥ vipraśati//        \U2
%-----------------------------
%This is the secret of the scriptures of Yoga within all scriptures of yoga. Just as he whose mind is like the light of a lamp that enters into the midst of darkness, ...  
%-----------------------------1-2
      idaṃ
      \app{\lem[wit={B,L,N1,U2}]{yogaśāstrarahasyaṃ}
        \rdg[wit={D,N2}]{yogaśāstrarahasya}
        \rdg[wit={U1}]{yogaśāstreṣu rahasyaṃ}
        \rdg[wit={E,P}]{yogaśāstrasya rahasyaṃ}}
      samagraśāstramadhye/\linelabel{_418e}
      \app{\lem[wit={ceteri}]{yasya}
        \rdg[wit={U2}]{\om}} 
\app{\lem[type=emendation, resp=egoscr]{mano}
  \rdg[wit={E,P,U1,U2}]{manaḥ}
  \rdg[wit={ceteri}]{mana}
  \rdg[wit={N2}]{\om}}
\app{\lem[wit={ceteri}]{yathāndhakārasya}
  \rdg[wit={N1}]{yathāṃdhakāras}
  \rdg[wit={D}]{yathāṃdhakāra°}
  \rdg[wit={N2}]{\om}}
\app{\lem[wit={ceteri}]{madhye}
  \rdg[wit={N2}]{\om}} 
\app{\lem[wit={ceteri}]{dīpasya}
  \rdg[wit={E}]{dīpa°}
  \rdg[wit={N2}]{\om}}
\app{\lem[wit={ceteri}]{tejaḥ}
  \rdg[wit={N2}]{\om}}
\app{\lem[wit={D,E,P,N1}]{praviśati}
  \rdg[wit={B,L,U1}]{praviśyati}
  \rdg[wit={U2}]{vipraśati}
  \rdg[wit={N2}]{\om}}/ 
%-----------------------------
%tathā śāstramadhye mano praviśati/     \E
%tathā śāstramadhye manaḥ praviśati     \P
%tathā.... \om                          \B
%tathā.... \om                          \L
%tathā śāstramadhye tasya manaḥ praviśati/     \N1
%                   tasya manaḥ praviśati/     \N2
%tathā śāstramadhye tasya manaḥ praviśati/     \D
%tathā.... \om                          \U1
%yathā śāstramadhye mano praviśati//     \U2
%-----------------------------
%similarly his mind enters into the teaching. %%%%1+2=He whose mind enters into the scriptures as the light of a lamp enters into darkness.
%-----------------------------
\app{\lem[wit={ceteri}]{tathā}
  \rdg[wit={U2}]{yathā}
  \rdg[wit={N2}]{\om}}
\app{\lem[wit={ceteri}]{śāstramadhye}
  \rdg[wit={B,L,N2,U1}]{\om}}
\app{\lem[wit={D,N1,N2}]{tasya manaḥ}
  \rdg[wit={P}]{manaḥ}
  \rdg[wit={E,U2}]{mano}
  \rdg[wit={B,L,U1}]{\om}}
\app{\lem[wit={ceteri}]{praviśati}
  \rdg[wit={B,L,U1}]{\om}}/
%-----------------------------
%yasya rājño madhye        kalaho nāsti/ \E
%yasya rājño manomadhye    kapaṭaṃ nāsti \P
%yasya rājño madhye manasi kapaṭaṃ nāsti/ \B
%yasya rājño madhye manasi kapaṭaṃ nāsti... \L
%yasya rājño manomadhye    kapaṭaṃ nāsti/ \N1
%yasya rājño manomadhye    kapaṭaṃ nāsti  \N2
%yasya rājño manomadhye    kapaṭaṃ nāsti/ \D
%      rājño manomadhye    kapaṭaṃ nāsti \U1
%yasya rājño manomadhye    kapaṭaṃ nāsti// \U2
%-----------------------------
%Deceit does not exists in the mind of such a king.
%Of whose king there is no deceit in the midst of the mind,
%-----------------------------4
\app{\lem[wit={ceteri}]{yasya}
  \rdg[wit={U1}]{\om}}
rājño
\app{\lem[wit={ceteri}]{manomadhye}
  \rdg[wit={B,L}]{madhye manasi}
  \rdg[wit={E}]{madhye}}
\app{\lem[wit={ceteri}]{kapaṭaṃ}
  \rdg[wit={E}]{kalaho}} nāsti/
%-----------------------------
%yasmin dṛṣṭe deśikatrāso na bhavati/ \E
%yasmin dṛṣṭe deśikasya trāso na bhavati  \P
%yasmiṃ dṛṣṭe deśikasya trāso na bhavati/ \B
%yasmiṃ dṛṣṭe deśikasya trāso na bhavati \L
%yasmiṃ dṛṣṭe deśakasya trāso na bhavati/  \N1
%yasmin dṛṣṭe deśakasya trāso na bhavati/  \N2
%yasmiṃ dṛṣṭe deśakasya trāso na bhavati/  \D
%yasmiṃ dṛṣṭe darśakasya trāso na bhavati \U1
%yasmin dṛṣṭe deśikasya trāso na bhavati//  \U2
%-----------------------------
%In whose sight fear of the teacher does not arise, 
%-----------------------------5
\app{\lem[wit={ceteri},alt={yasmin}]{yasmi\skp{n-dṛ}}
  \rdg[wit={B,L,N1,D,U1}]{yasmiṃ}
}\skm{n-dṛ}ṣṭe
\app{\lem[wit={ceteri}]{deśakasya}
  \rdg[wit={U1}]{darśakasya}
  \rdg[wit={E}]{deśika°}}
trāso na bhavati/
%-----------------------------
%tasya manaḥ śuddhaṃ bhavati/  \E
%tasya manaḥ śuddhaṃ bhavati   \P
%tasya manaḥ śuddhaṃ bhavati/  \B
%tasya manaḥ śuddhaṃ bhavati   \L
%tasya manaḥ śuddhaṃ bhavati/  \N1
%tasya manaḥ śuddhaṃ bhavati/  \N2
%tasya manaḥ śuddhaṃ bhavati/  \D
%yasya manaḥ śuddhaṃ bhavati   \U1
%tasya manaḥ śuddhaṃ bhavati//  \U2
%-----------------------------
%his mind becomes pure.   
%-----------------------------6
\app{\lem[wit={U1}]{yasya}
  \rdg[wit={ceteri}]{tasya}} 
manaḥ śuddhaṃ bhavati/
%-----------------------------
%yasya pṛthvyāṃ  vītir   bhavati/ \E
%yasya pṛthivyāṃ kīrttir bhavati \P
%yasya pṛthvyāṃ  kīrtir  bhavati/ \B
%yasya pṛthivyāṃ kīrtir  bhavati \L
%yasya pṛthivī   kīrttir bhavati/ \N1
%yasya pṛthivī   kīrttir bhavati/ \N2
%yasya pṛthivī   kīrttir bhavati/ \D
%      pṛithīvī  kīrti   bhavati \U1
%yasya pṛthvyāṃ  kītīr   bhavati// \U2
%-----------------------------
%Whose fame becomes renowned on the Earth, 
%-----------------------------7
\app{\lem[wit={ceteri}]{yasya}
  \rdg[wit={U1}]{\om}}
\app{\lem[wit={P,L}]{pṛthivyāṃ}
  \rdg[wit={B,E,U2}]{pṛthvyāṃ}
  \rdg[wit={D,N1,N2}]{pṛthivī}
  \rdg[wit={U1}]{pṛithīvī}}
\app{\lem[wit={ceteri}, alt={kīrtir}]{kīrti\skp{r-bha}}
  \rdg[wit={E}]{vītir}
  \rdg[wit={U1}]{kīrti}
  \rdg[wit={U2}]{kītīr}}\skm{r-bha}vati/
%-----------------------------
%yasya manomadhye satpuruṣasya vaco viśvāso bhavati/           yo rājā sadānaṃdarūpo bhavati// \E
%yasya manomadhye satpuruṣavacanaviśvāso    bhavati            yo rājā sadānaṃdapūrṇo bhavati   \P
%yasya manomadhye satpuruṣavacanaviśvāso    bhavati            yo rājā sadānaṃdapūrṇo bhavati/  \B
%yasya manomadhye satpuruṣavacanaviśvāso    bhavati            yo rājā sānaṃdapūrṇo bhavati   \L
%yasya manomadhye satpuruṣavacanaviśvāso    bhavati/           yo rājā sadānaṃdapūrṇo bhavati   \N1
%yasya manomadhye satpuruṣavacanaḥ viśvāso  bhavati/           yo rājā sadānaṃdapūrṇo bhavati   \N2
%yasya manomadhye satpuruṣavacanaviśvāso    bhavati/           yo rājā sadānaṃdapūrṇo bhavati/   \D
%yasya manomadhye satpuruṣasya vacanaviśvabhyāso    bhavati    yo rājā sadānaṃdapūrṇo bhavati   \U1
%yasya manomadhye satpuruṣavacanaviśvāso           bhavati//   yo rājā sadānaṃdapūrṇo bhavati// \U2
%-----------------------------
%In whose mind, there is trust in the words of exquisite individuals. Who is a king always filled with bliss.
%-----------------------------8-9
yasya manomadhye
\app{\lem[wit={ceteri}]{satpuruṣavacanaviśvāso}
  \rdg[wit={N2}]{satpuruṣavacanaḥ viśvāso}
  \rdg[wit={E}]{satpuruṣasya vaco viśvāso}
  \rdg[wit={U1}]{vacanaviśvabhyāso}} bhavati/
yo rājā
\app{\lem[wit={ceteri}]{sadānandapūrṇo}
  \rdg[wit={E}]{sadānaṃdarūpo}
  \rdg[wit={L}]{sānaṃdapūrṇo}} bhavati/ 
%-----------------------------
%yasya pārśve pratyakṣam aneka--manohārivastūni tiṣṭhaṃti/ \E
%yasya pārśve pratyakṣam anekaṃ manohārivastu bhavati  \P
%yasya pārśve pratyakṣam aneka--manohārivastu bhavati/ \B
%yasya pārśve pratyakṣam aneka--manohārivastu bhavati \L
%yasya pārśve pratyakṣam anekaṃ manohārivastu bhavati/ \N1
%yasya pārśve pratyakṣam anekaṃ manohārivastu bhavati// \N2
%yasya pārśve pratyakṣam anekaṃ manohārivastu bhavati/ \D
%yasya pārśve pratyakṣam anekaṃ manohārivastu bhavati \U1
%yasya pārśve pratyakṣam anekaṃ manohārivastu bhavati// \U2 %%%431
%-----------------------------
%By whose side there is an abundance of enchanting objects, 
%-----------------------------10
yasya pārśve pratyakṣa\skp{m-a}\app{\lem[wit={ceteri},alt={anekaṃ}]{\skm{m-a}nekaṃ}
  \rdg[wit={B,L,E}]{aneka°}}
\app{\lem[wit={E}]{manohārivastūni}
  \rdg[wit={ceteri}]{manohārivastu}}
\app{\lem[type=emendation, resp=egoscr]{bhavanti}
  \rdg[wit={E}]{tiṣṭhaṃti}
  \rdg[wit={ceteri}]{bhavati}}/
%------------------------
%etādṛśasya rājña idaṃ yogarahasyaṃ   kathanīyam/ \E
%etādṛśasya rājño ye   yogarahasyaṃ   kathanīyaṃ \P
%etādṛśasya rājño ye   thogarahasyaṃ  kathyaniyaṃ/ \B
%etādṛśasya rājño yad  yogarahasyaṃ   kathyanīyaṃ// \L
%etādṛśasya rājño 'gre yogarahasyaṃ   karttavyaṃ// \N1
%etādṛśasya rājño 'gre yogarahasyaṃ   karttavyaṃ// \N2
%etādṛśasya rājño gre  yogarahasyaṃ   karttavya/ \D
%etādṛśasya rājño gre  yogarahasyaṃ   karttavyaṃ \U1
%etādṛśasya rājño ye   yogarahasyaṃ   kathyate// \U2
%-----------------------------
%in front of such a king the secret teaching of yoga shall be presented. 
%-----------------------------11
etādṛśasya
\app{\lem[wit={ceteri}]{rājño}
  \rdg[wit={E}]{rājña}
}\app{\lem[wit={X}]{'gre}
  \rdg[wit={B,P,U2}]{ye}
  \rdg[wit={L}]{yad}
  \rdg[wit={E}]{idaṃ}}
\app{\lem[wit={ceteri}]{yogarahasyaṃ}
  \rdg[wit={B}]{thogarahasyaṃ}}
\app{\lem[wit={N1,N2,U1}]{karttavyaṃ}
  \rdg[wit={D}]{karttavya}
  \rdg[wit={E,P}]{kathanīyam}
  \rdg[wit={B,L}]{kathyaniyaṃ}
  \rdg[wit={U2}]{kathyate}}/\linelabel{_400xe}
\end{prose}
  \end{edition}
  \begin{translation}
               \ekddiv{
                 head={[\uproman{59}. \textbf{Secret teaching of the scriptures of yoga}]},
                 type=section,
                 depth=2, 
                 n=LIX.1
               }
               \xmlhead[h59]{[LVIX. Secret teaching of the scriptures of yoga]}
\begin{tlate}[p59_01]
  \label{secretofscriptures}
  \noindent
This is the secret teaching of the scriptures of yoga in all of the scriptures.\footnote{The \emph{Yogasvarodaya} introduces a section with ``\textit{idaṃ yogaśāstrarahasyañ} \ldots'' but Rāmacandra's version of the \textit{yogaśāstrarahasya} differs drastically. This is why the beginning of this passage is not presented in the source of the apparatus. The \emph{Yogasvarodaya} (Ed. p. 847) reads: \textit{idaṃ yogarahasyañ ca na vācyaṃ mūrkhasannidhau} || \textit{yogadeśas tu tatraiva} || \textit{utpātarahite deśe kaṇṭakādivivarjite} | \textit{abhyasyate sadā yogaḥ samaḥ syāt sukhaduḥkhayoḥ} | \textit{surājani samāśritya karttavyo nirupadrave} | \textit{deśe tu sarvaśasyāḍhye lobhamohavivarjite} |. The rest of the \emph{Yogasvarodaya}'s section on the \textit{yogaśāstrarahasya} is will play a role from verse LVIII.2 onwards once again.} Just as the light of a lamp enters into the midst of darkness, similarly, his mind enters into the teaching.
The king in whose mind mind there is no deceit, in whose sight fear of the teacher\footnote{The topic of fearing the teacher to my knowledge does not appear in other yoga literature and is unique to the \textit{Yogatattvabindu}.} does not arise, whose mind is pure, whose fame spreads accross the earth, in whose mind, there is trust in the words of exquisite individuals, who is a king always filled with bliss, by whose side there is an abundance of enchanting objects, in front of such a king the secret of yoga shall be revealed.
\flushpage
    \end{tlate}
  \end{translation}
\end{alignment}
\pagebreak %after pp. 139x-140x %%nachträglich eingefügte Seite
%%%%%%%%%%%%%%%%%%%%%%%%%%%%%%%%%%%%%%%%%%
%%%%%%%%%%%%%%%%%%%%%%%%%%%%%%%%%%%%%%%%%% 
%%%%%%%%PAGEBREAK%%%%%%%PAGEBREAK%%%%%%%%%
%%%%%%%%%%%%%%%%%%%%%%%%%%%%%%%%%%%%%%%%%% 
%%%%%%%%%%%%%%%%PAGEBREAK%%%%%%%%%%%%%%%%%
%%%%%%%%%%%%%%%%%%%%%%%%%%%%%%%%%%%%%%%%%% 
%%%%%%%%PAGEBREAK%%%%%%%PAGEBREAK%%%%%%%%%
%%%%%%%%%%%%%%%%%%%%%%%%%%%%%%%%%%%%%%%%%% 
%%%%%%%%%%%%%%%%%%%%%%%%%%%%%%%%%%%%%%%%%% 
%%%%%%%%%%%%%%%%%%%%%%%%%%%%%%%%%%%%%%%%%% 
%%%%%%%%%%%%%%%%%%%%%%%%%%%%%%%%%%%%%%%%%% 
%%%%%%%%PAGEBREAK%%%%%%%PAGEBREAK%%%%%%%%%
%%%%%%%%%%%%%%%%%%%%%%%%%%%%%%%%%%%%%%%%%% 
%%%%%%%%%%%%%%%%PAGEBREAK%%%%%%%%%%%%%%%%%
%%%%%%%%%%%%%%%%%%%%%%%%%%%%%%%%%%%%%%%%%% 
%%%%%%%%PAGEBREAK%%%%%%%PAGEBREAK%%%%%%%%%
%%%%%%%%%%%%%%%%%%%%%%%%%%%%%%%%%%%%%%%%%% 
%%%%%%%%%%%%%%%%%%%%%%%%%%%%%%%%%%%%%%%%%% 
%%%%%%%%%%%%%%%%%%%%%%%%%%%%%%%%%%%%%%%%%% 
%%%%%%%%%%%%%%%%%%%%%%%%%%%%%%%%%%%%%%%%%% 
%%%%%%%%PAGEBREAK%%%%%%%PAGEBREAK%%%%%%%%%
%%%%%%%%%%%%%%%%%%%%%%%%%%%%%%%%%%%%%%%%%% 
%%%%%%%%%%%%%%%%PAGEBREAK%%%%%%%%%%%%%%%%%
%%%%%%%%%%%%%%%%%%%%%%%%%%%%%%%%%%%%%%%%%% 
%%%%%%%%PAGEBREAK%%%%%%%PAGEBREAK%%%%%%%%%
%%%%%%%%%%%%%%%%%%%%%%%%%%%%%%%%%%%%%%%%%% 
%%%%%%%%%%%%%%%%%%%%%%%%%%%%%%%%%%%%%%%%%%
\begin{alignment}[
  texts=edition[class="edition"];
  translation[class="translation"],
  ]
  \begin{edition}
    \begin{tlg}[59_1]
      \noindent
%-----------------------------
%na snehān na bhayān na lobhān na mohān na dhanād   balān    na maitrībhāvān  nau dāryān na sauṃdaryān na sevanāt/ \E
%na snehān no bhayāl lno?                                             bhāvān  no  dānān  na sau daryān na sevanāt  \P %%%7680.jpg
%ni śnehān nā bhayāl lobhān    na mohān na dhanād   balāt/   na maitrībhāvān  no  dānāt  na sauṃdaryān na sevanāt/ \B
%ni śnehān nā bhayāl lobhān    na mohān na nadhanād balāta// na maitrībhāvān  no  dānāt  na sauṃdayan  ni sevanāt// \L
%na śnehān a  bhayāl obhān     na mohān na dhanād   balāt/   na maitrībhāvān  na  dāsān  na sauṃdaryān na sevanāt/ \N1
%na śnehān a  bhayāl obhān     na mohān na dhanād   balāt//  na maitrībhāva   nā  dānān  na saudaryān  na sevanāt// 1 \N2
%na śnehān a  bhayāl lobhān    na mohān na dhanād   balāt/   na maitrī                                                \D
%na śnehān na bhayān lobhān    na mohān na dhanāt   balāt    na maitrībhāvān  na  dāsān  na sauṃdaryān na sevatā \U1
%na snehān na bhayāl lon       na mohān na dhanād   balāt//  na maitrībhāvān  no  dānān  na sauṃdaryān na sevanāt// \U2
%-----------------------------
%Not because of love, not because of fear, not because of greed, not because of ignorance, not because of gift, not because of violence, not because of friendship, not because of hostility, not because of nobility, not because of service, 
%-----------------------------12
  \tl{\app{\lem[wit={ceteri}]{na}
      \rdg[wit={B,L}]{ni}}
    \app{\lem[wit={E,P,U2},alt={snehān}]{snehā\skp{n-na}}
      \rdg[wit={ceteri}]{śnehān}
    }\app{\lem[wit={E,P,U2},alt={na}]{\skm{n-na}}
      \rdg[wit={B,L}]{nā°}
      \rdg[wit={D,N1,N2}]{a°}
    }\app{\lem[wit={ceteri},alt={bhayāl}]{bhayā\skp{l-lo}}
      \rdg[wit={E,U1}]{bhayān}
    }\app{\lem[wit={B,D,L,U1},alt={lobhān}]{\skm{l-lo}bhā\skp{n-na}}
      \rdg[wit={N1,N2}]{obhān}
      \rdg[wit={P}]{lno}
      \rdg[wit={U2}]{lon}
    }\app{\lem[wit={ceteri},alt={na}]{\skm{n-na}}
      \rdg[wit={P}]{\om}
    }\app{\lem[wit={ceteri},alt={mohān}]{mohā\skp{n-na}}
      \rdg[wit={P}]{\om}
    }\app{\lem[wit={ceteri},alt={na}]{\skm{n-na}}
      \rdg[wit={P}]{\om}
    }\app{\lem[wit={ceteri}]{dhānā\skp{d-ba}}
      \rdg[wit={L}]{na dhanād}
      \rdg[wit={P}]{\om}
    }\app{\lem[wit={ceteri},alt={balāt}]{\skm{d-ba}lāt}
      \rdg[wit={B}]{balāta}
      \rdg[wit={P}]{\om}}/}\\
  \tl{\app{\lem[wit={ceteri}]{na}
      \rdg[wit={P}]{\om}}
    \app{\lem[wit={ceteri},alt={maitrībhāvān}]{maitrībhāvā\skp{n-na}}
      \rdg[wit={N2}]{maitrībhāva}
      \rdg[wit={D}]{maitrī}
      \rdg[wit={P}]{bhāvān}
    }\note[type=philcomm, labelb=418, lem={maitrī \ldots}]{A \textit{lacuna} in \getsiglum{D} starts here. The omissions are not recorded. The reader will be informed when \getsiglum{D} resumes.}
    \app{\lem[wit={N1,U1},alt={na}]{\skm{n-na}}
      \rdg[wit={B,L,P,U2}]{no}
      \rdg[wit={E}]{nau}
      \rdg[wit={N2}]{nā}
      \rdg[wit={D}]{\om}}
    \app{\lem[wit={N1,U1},alt={dāsān}]{dāsā\skp{n-na}}
      \rdg[wit={P}]{dānān}
      \rdg[wit={E}]{dāryān}
      \rdg[wit={B,L}]{dānāt}
      \rdg[wit={N2,U2}]{dānān}
      \rdg[wit={D}]{\om}
    }\app{\lem[wit={ceteri},alt={na}]{\skm{n-na}}
      \rdg[wit={D}]{\om}
    }\app{\lem[wit={ceteri},alt={sauṃdaryān}]{sauṃdaryā\skp{n-na}}
      \rdg[wit={P,N2}]{saudaryān}
      \rdg[wit={L}]{sauṃdayan}
      \rdg[wit={D}]{\om}
    }\app{\lem[wit={ceteri},alt={na}]{\skm{n-na}}
      \rdg[wit={L}]{ni}
      \rdg[wit={D}]{\om}}
    \app{\lem[wit={ceteri}]{sevanāt}
      \rdg[wit={U1}]{sevatā}}\dd{} \begin{otherlanguage}{english}\uproman{59}.1\end{otherlanguage}\hskip-2pt\dd{}}\\
\end{tlg}
\begin{prose}[p59_02]
%-----------------------------
%sāmānyāgre     yogo na kathanīyaḥ/ \E
%sāmānyād agre  yogo na kathanīyaḥ \P
%sāmānyāgre     yogo na kathaniyaṃ/ \B
%sāmānyāgre     yogo na kathanīyaṃ// \L
%sāmānyād agre  yogo na kathanīyaḥ/ \N1
%sāmānyād agre  yogo na kanīyaḥ/ \N2
%\om \D
%sāmānyāgre     yogo na kathanīyaḥ \U1
%sāmānyād agre  yogo na kathanīyaḥ \U2
%-----------------------------
%shall yoga be taught in front of everyone. 
%-----------------------------13
\app{\lem[wit={P,N1,N2,U2}]{sāmānyād\skp{-}agre}
  \rdg[wit={B,E,L,U1}]{sāmānyāgre}}
yogo na
\app{\lem[wit={E,P,N1,U1,U2}]{kathanīyaḥ}
  \rdg[wit={B}]{kathaniyaṃ}
  \rdg[wit={L}]{kathanīyaṃ}
  \rdg[wit={N2}]{kanīyaḥ}}/ 
%-----------------------------
%yaḥ paraniṃdā  rato bhavati/ \E
%yaḥ paraniṃdā  rato bhavati \P
%yaḥ paraniṃdāṃ      karoti/ \B
%yaḥ paraniṃdāṃ      karoti \L
%yaḥ paraniṃdā  rato bhavati \N1
%yaḥ paraniṃdā  rato bhavati \N2
%    paraniṃdāṃ rato bhavati \U1
%yaḥ paraniṃdā  rato bhavati// \U2
%\om \D
%-----------------------------
%He, who loves it to criticise others, 
%-----------------------------14
\app{\lem[wit={ceteri}]{yaḥ}
  \rdg[wit={U1}]{\om}}
\app{\lem[wit={ceteri}]{paranindā}
  \rdg[wit={B,L,U1}]{paraniṃdāṃ}}
\app{\lem[wit={ceteri}]{rato}
  \rdg[wit={B,L}]{\om}}
\app{\lem[wit={ceteri}]{bhavati}
  \rdg[wit={B,L}]{karoti}}/  
%-----------------------------
%durācāro bhavati/ \E
%durācāro bhavati  \P
% \om              \B
% \om              \L
%dūrācāro bhavati/ \N1
%dūrācāro bhavati/ \N2
%\om \D
%dūrācāro bhavati  \U1
%dūrācāro bhavati//  \U2
%-----------------------------
%who is behaving badly, 
%-----------------------------15
\app{\lem[wit={ceteri}]{dūrācāro bhavati}
  \rdg[wit={B,L}]{\om}}/
%-----------------------------
%durmaitryānyasya            vastu na dadāti/ \E
%bhrātur mitrasya yogyaṃ     vastu na dadāti \P
%durmaitryānyasya            vastu na dadāti/ \B
%bhrātu mitrasya ca yogyaṃ ca vastu na dadāti/ \N1
%bhrātu mitrasya ca yogyaṃ    vastu na dadāti/ \N2
%bhrātṛr mitraṃ ca yogyaṃ    vastu na dadāti \U1
%bhrātur mitrasya yogyaṃ     vastu na dadāti// \U2
%\om                               \L
%\om \D
%-----------------------------
%who does not gives [single] thing, which benefits friend and brother,  
%-----------------------------16
\app{\lem[wit={P,U2},alt={bhrātur}]{bhrātu\skp{r-mi}}
  \rdg[wit={N1,N2}]{bhrātu°}
  \rdg[wit={U1}]{bhrātṛr}
  \rdg[wit={B,E}]{dur°}
  \rdg[wit={L}]{\om}
}\app{\lem[wit={ceteri}]{mitrasya}
  \rdg[wit={U1}]{mitraṃ}
  \rdg[wit={B,E}]{maitryānyasya}
  \rdg[wit={L}]{\om}}
\app{\lem[wit={N2,U1}]{ca yogyaṃ}
  \rdg[wit={N1}]{ca yogyaṃ ca}
  \rdg[wit={P,U2}]{yogyaṃ}
  \rdg[wit={B,L,E}]{\om}}
\app{\lem[wit={ceteri}]{vastu na dadāti}
 \rdg[wit={L}]{\om}}/ 
%\note[type=philcomm, labelb=420, lem={bhrātur \ldots na dadāti}]{Sentence omitted in \getsiglum{L}.}
%-----------------------------
%[p.83]
%ya asatyaṃ vadati/  yo yoga-----------------nindāṃ karoti/  \E
%yo 'satyaṃ vadati   yo yogināṃ   manomadhye niṃdāṃ karoti   \P
%so 'satyaṃ vadati/     yoginā    manomadhye niṃdāṃ karoti/  \N1
%so 'satyaṃ vadati/     yoginā    manomadhye niṃdāṃ karoti/  \N2
%so satyaṃ  vadati      yogināṃ   manomadhye ni-----karoti   \U1
%yo 'satyaṃ vadati//    yogināṃ   manomadhye niṃdāṃ karoti// \U2
%\om                                                         \B
%\om                                                         \L
%\om                                                         \D
%-----------------------------
%who does not speak the truth and despises yoga in spirit, 
%-----------------------------17-18
\note[type=source, labelb=421, labele=_421e, nosep]{cf. YSv (PT p. 847): stutir nindā na karttavyā sādhunā satyavādinā || yogānadhikāriṇam āha tatraiva ||}
%\note[type=philcomm, labelb=422, labele=_421e, lem={yo 'satyaṃ \ldots nindāṃ karoti}]{Both sentences omitted in \getsiglum{B} and \getsiglum{L}.}
\app{\lem[wit={P,U2}]{yo}
  \rdg[wit={N1,N2,U1}]{so}
  \rdg[wit={E}]{ya}
\rdg[wit={B,L}]{\om}
}
\app{\lem[wit={ceteri}]{satyaṃ}
  \rdg[wit={E}]{asatyaṃ}
\rdg[wit={B,L}]{\om}}
\app{\lem[wit={ceteri}]{vadati}
 \rdg[wit={B,L}]{\om}}/
\app{\lem[wit={E,P}]{yo}
  \rdg[wit={ceteri}]{\om}}
\app{\lem[wit={ceteri}]{yogināṃ}
  \rdg[wit={N1,N2}]{yoginā}
  \rdg[wit={E}]{yoga°}
\rdg[wit={B,L}]{\om}}
\app{\lem[wit={ceteri}]{manomadhye}
  \rdg[wit={B,E,L}]{\om}}
\app{\lem[wit={ceteri}]{nindāṃ}
  \rdg[wit={U1}]{ni°}
\rdg[wit={B,L}]{\om}}
\app{\lem[wit={ceteri}]{karoti}
  \rdg[wit={B,L}]{\om}}/\linelabel{_421e}
%-----------------------------
%yasya manomadhye dayā na bhavati/ \E
%yasya manomadhye dayā na bhavati  \P
%yasya manomadhye dayā na bhavati/ \B
%yasya manomadhye dayā na bhavati/ \L
%yasya manomadhye dayā na bhavati  \N1
%yasya manomadhye dayā na bhavati  \N2
%\om \D
%yasya manomadhye dayā na bhavati  \U1
%yasya manomadhye dayā na bhavati// \U2
%-----------------------------
%In whose mind compassion does not arise.  
%-----------------------------19
\note[type=source, labelb=423, labele=_423e, nosep]{cf. YSv (PT p. 847): manomadhye dayā nāsti sadā yaḥ kalahapriyaḥ |}
yasya manomadhye dayā na bhavati/ 
%-----------------------------
%yaḥ    kalaha--priyo    bhavati/ \E
%yasya  kalaha--priyo    bhavati/ \P
%yasya  kalahaṃ priyo na bhavati/ \B
%yasya  kalahaṃ priyo na bhavati// \L
%yaḥ    kalaha--priyo    bhavati/ \N1 [em. to  yasya kalahapriyo bhavati ||
% \om N2
%\om \D
%yaḥ    kalaha--priyo    bhavati \U1
%yasya  kalahaḥ priyo    bhavati// \U2
%-----------------------------
%who is one that has love for quarrel
%-----------------------------20
%\note[type=philcomm, labelb=424, lem={yasya kalaha \ldots bhavati}]{Sentence omitted in \getsiglum{D} and \getsiglum{N2}.}
\app{\lem[wit={E,N1,U1}]{yaḥ}
  \rdg[wit={B,L,P,U2}]{yasya}
\rdg[wit={D,N2}]{\om}}
\app{\lem[wit={E,P,N1,U1}]{kalahapriyo}
  \rdg[wit={B,L}]{kalahaṃ priyo}
  \rdg[wit={U2}]{kalahaḥ priyo}
\rdg[wit={D,N2}]{\om}}
\app{\lem[wit={ceteri}]{bhavati}
  \rdg[wit={B,L}]{na bhavati}
\rdg[wit={D,N2}]{\om}}/\linelabel{_423e}
%-----------------------------
%svakāryakaraṇe  sāvadhāno bhavati/ \E
%svakāryakaraṇe  sāvadhāno bhavati  \P
%svakāryākaraṇeṃ sāvadhāno bhavati/ \B
%svakāryākaraṇe  sāvadhāno bhavati// \L
%svakāryyākaraṇe sāvadhāno bhavati/ \N1
%svakāryyākaraṇā sāvadhāno bhavati/ \N2
%\om \D
%svakāryakaraṇe  sāvadhāno bhavati \U1
%svakāryakaraṇe  sāvadhāno bhavati// \U2
%-----------------------------
%, attention arises for him regarding his selfish intentions; 
%-----------------------------21
\app{\lem[wit={E,P,U1,U2}]{svakāryakaraṇe}
  \rdg[wit={L,N1}]{svakāryākaraṇe}
  \rdg[wit={B}]{svakāryākaraṇeṃ}
  \rdg[wit={N2}]{svakāryyākaraṇā}}
sāvadhāno bhavati/
%-----------------------------
%guroḥ  kāryakaraṇe na dattacitto bhavati/ \E
%guroḥ  kāryakaraṇe 'nādṛto??? bhavati        \P
%guro   kārye karaṇe anādarano    bhavati/ \B [em. to anādaro = disrespect] 
%guroḥ  kāryakaraṇe  anādare no    bhavati/ \Ĺ
%guroḥ  kāryakaraṇe  ādaro na     bhavati/ \N1  [em. zu kāryakāraṇa]
%guro   kāryakaraṇe  ādaro na     bhavati/ \N2
%\om \D
%guroḥ  kāryakaraṇe  ādaro na    bhavati  \U1
%guro   kāryakaraṇe  nādṛto      bhavati//  \U2
%-----------------------------------
%disrespect arises towards the intentions of the teacher.
%-----------------------------
\note[type=source, labelb=425, labele=_425e, nosep]{cf. YSv (PT p. 847): svakāryalobhane śīlo gurukāryaparāṅmukhaḥ | etasmai ca na dātavyaṃ vaktavyaṃ tasya sannidhau |}
\app{\lem[wit={ceteri}]{guroḥ}
  \rdg[wit={B,N2,U2}]{guro}}
\app{\lem[type=emendation, resp=egoscr]{kāryakāraṇe}
  \rdg[wit={ceteri}]{kāryakaraṇe}
  \rdg[wit={B}]{kārye karaṇe}}
\app{\lem[wit={P,U2}]{'nādṛto}
  \rdg[wit={N1,N2,U1}]{ādaro na}
  \rdg[wit={B}]{anādarano}
  \rdg[wit={L}]{anādare no}
  \rdg[wit={E}]{na dattacitto}}
bhavati/
%-----------------------------
%etādṛśasyāgre   na yogaḥ  kriyate   na paṭhyate// \E
%etādṛśasyāgre   na yogaḥ  kriyate   na paṭhyate    \P
%etādṛśasyāgre   na yogaḥ  kriyate/  na paṭhayate/ \B
%etādṛśasyāgre   na yogaḥ  kriyate   na paṭhayate... \L
%etādṛśasyāgre   na        kriyate/  na padyaṃte//  \N1
%etādṛśasyāgre   na        kriyate/  na padyaṃte//  \N2
%\om \D
%etādṛśasya agre na        kriyate   na paṭhyate \U1
%etādṛśasyāgre   na yogaḥ  kriyate// na paṭhyate// \U2
%-------------------------
%In front of someone like this yoga is neither done nor taught. 
%-----------------------------23
\app{\lem[wit={ceteri}]{etādṛśasyāgre}
  \rdg[wit={U1}]{etādṛśasya agre}}
na \app{\lem[wit={ceteri}]{yogaḥ}
  \rdg[wit={N1,N2,U1}]{\om}}
kriyate na \app{\lem[wit={E,P,U1,U2}]{paṭhyate}
  \rdg[wit={N1,N2}]{padyaṃte}
  \rdg[wit={B,L}]{paṭhayate}}/\linelabel{_425e}\\
%-----------------------------
%śrṛṇvan prītādikān  śabdān  paśyan rūpaṃ manoharam/ \E
%śrṛṇvan gītādikān   śabdān  paśyan rūpaṃ manoharaṃ   \P
%śrṛṇvan gītādikān   śabdān  paśyan rūpaṃ manoharaṃ/ \B
%śṛṇvan  gītādikān   śabdān/ paśyan rūpaṃ manoharaṃ/ \N1
%śuśvana gītādikāna  śabdāt/ paśyan rūpaṃ manoharaṃ// \N2
%śṛṇvan  gītādikān   śabdān  paśyan rūpaṃ manoharaṃ// \L
%śṛṇvan  gītādikān   śabdān  paśyat rūpaṃ manoharaṃ \U1
%śrṛṇvan gītādikān// śabdān  paśyan rūpaṃ manoharam// \U2
%\om \D
%-----------------------------
%While hearing the sound of music etc., while seeing heart pounding forms,  
%-----------------------------24
\app{\lem[wit={N1,L,U1},alt={śṛṇvan}]{śṛṇva\skp{n-gī}}
  \rdg[wit={N2}]{śuśvana}
  \rdg[wit={ceteri}]{śrṛṇvan}
}\app{\lem[wit={ceteri},alt={gītādikān}]{\skm{n-gī}tādikān}
  \rdg[wit={E}]{prītādikān}}\dd{}
 \app{\lem[wit={ceteri},alt={śabdān}]{śabdā\skp{n-pa}}
  \rdg[wit={N2}]{śabdāt |}
}\app{\lem[wit={ceteri},alt={paśyan}]{\skm{n-pa}śya\skp{n-rū}}
    \rdg[wit={U1}]{paśyat}
  }\skm{n-rū}paṃ manoharaṃ\dd{}
%-----------------------------
%jāgrat                  sphuran  spṛśan sparśa---mṛdupriyam     svādān manoramān          bhrāmyan    deśān/ manoramān \E
%jighran gaṃdhāṃś ca     surabhin spṛśan sparśaṃ  mṛḍupriyaṃ     svādān manoramān  khādan  bhrāmyan    deśān manoramān \P
%jighran agachan       sphurabhi  spṛśaṃ          mṛdupriyaṃ     svādān manorathān khādavan bhrāman    deśān manoramān \B
%jighran agachan        surabhin  spṛśaṃ          mṛdupriyaṃ     svādān manorathān khādavan bhrāman    deśān manoramān \L
%jighran gaṃdhāṃ       śusurabhīn spṛśyanasya     mṛdupriyaṃ/    svādān manomān svādan      bhrāmye na deśān manoramān \N1
%jighraṃ gaṃdhāṃ       śusurabhīn spṛśyanasyaṃ śarmṛdupriyaṃ     svādān manomān             bhrāmya na deśān manoramān \N2
%\om \D
%jighraṃ nāṃdhaś ca      surabhīn        sparśaṃ  mṛdupriyaṃ     svādān manoramān khādaṃta--bhrāmyan tveṣāṃn manoramān/ \U1 %%%302.jpg
%jighran spṛśan  gaṃdhan surabhīn spṛśan sparśaṃ  mṛḍu// priyaṃ  svādān manoramān           bhrāmyan    deśān manoramān//  \U2
%-----------------------------
%while smelling fragnent scent, while touching lovely and soft touch, while eating food that pleases the mind, while travelling to beautiful places,    
%-----------------------------25
\app{\lem[wit={ceteri},alt={jighran}]{jighra\skp{n-ga}}
  \rdg[wit={E}]{jāgrat}
  \rdg[wit={U1}]{jighraṃ}
}\app{\lem[wit={N1,N2},alt={gandhān}]{gandhā\skp{n-su}}
  \rdg[wit={P}]{gaṃdhāṃś ca}
  \rdg[wit={U1}]{nāṃdhaś ca}
  \rdg[wit={B,P}]{agachan}
  \rdg[wit={U2}]{spṛśan gaṃdhan}
  \rdg[wit={E}]{\om}
}\app{\lem[wit={U1,U2},alt={surabhīn}]{\skm{n-su}rabhī\skp{n-spṛ}}
  \rdg[wit={E}]{sphuran}
  \rdg[wit={P,L}]{surabhin}
  \rdg[wit={B}]{sphurabhi}
  \rdg[wit={N1,N2}]{śusurabhīn}
}\app{\lem[wit={Y},alt={spṛśan}]{spṛśa\skp{n-spa}}
  \rdg[wit={N1}]{spṛśyanasya}
  \rdg[wit={N2}]{spṛśyanasyaṃ}
  \rdg[wit={U1}]{\om}
}\app{\lem[wit={P,U1,U2},alt={sparśaṃ}]{\skp{n-spa}rśaṃ}
  \rdg[wit={E}]{sparśa°}
  \rdg[wit={ceteri}]{\om}
}\app{\lem[wit={ceteri}]{mṛḍupriyaṃ}
  \rdg[wit={N2}]{śarmṛdupriyaṃ}
  \rdg[wit={U2}]{mṛḍu || priyaṃ}}\dd{}
svādā\skp{n-ma}\app{\lem[wit={ceteri},alt={manoramān}]{\skm{n-ma}noramā\skp{n-khā}}
  \rdg[wit={B,L}]{manorathān}
  \rdg[wit={N1,N2}]{manomān}
}\app{\lem[wit={ceteri},alt={khādan}]{\skm{n-khā}da\skp{n-bhrā}}
  \rdg[wit={B,L}]{khādavan}
  \rdg[wit={U1}]{khādaṃta°}
  \rdg[wit={N1}]{svādan}
  \rdg[wit={E,N2}]{\om}
}\app{\lem[wit={ceteri},alt={bhrāmyan}]{\skm{n-bhrā}mya\skp{n-de}}
  \rdg[wit={B,L}]{bhrāman}
  \rdg[wit={N1}]{bhrāmyena}
  \rdg[wit={N2}]{bhrāmya na}
}\app{\lem[wit={ceteri},alt={deśān}]{\skm{n-de}śā\skp{n-ma}}
  \rdg[wit={U1}]{tveṣāṃn}
}\skm{n-ma}noramān\dd{}
    \end{prose}
  \end{edition}
  \begin{translation}
    \begin{tlate}[59_1]
      \paragraph{\uproman{59}.1 } Not because of love, not because of fear, not because of greed, not because of gift, not because of friendship, not because of hostility, not because of nobility, not because of service, \ldots\footnote{So far, I have not been able to identify the source of this verse. However, the smooth transition to prose suggests that Rāmacandra himself may have written the verse. }\\\\
    \end{tlate}
    \begin{tlate}[p59_02]
      \ldots shall yoga be taught in front of everyone. He, who loves to blame others, who loves to criticise others, who is misbehaving, who does not give [single] thing, which benefits friend and brother, who does not speak the truth and despises yoga in spirit, in whose mind compassion does not arise, who delights in quarrel, attention arises for him regarding his selfish intentions, [and] disrespect arises towards the intentions of the teacher. In front of someone like this, yoga is neither done nor taught.\footnote{In my assessment, the passage reads like an educational measure that teaches proper social behaviour for a precocious youth. All other pre-colonial yoga texts known to me are, without exception, written for an adult audience and would not advise good behaviour like being brave or sharing things with friends. They would not mention the desired injunction from disputes or disrespecting the teacher, particularly not in their main teaching. For an adult yoga student, these statements would appear superfluous and self-evident. These statements, however, start to make much sense if one would assume a young audience (more precisely, given the numerous allusions to the audience's material wealth, possibly young princes). Also cf. the passage on the eight enjoyments (p. \pageref{eightenjoyments}), the prohibition of \textit{prāṇāyāma} for young persons (p.\pageref{ashtanga}), and the promise of becoming attractive for virgins, as well as the promise of control over women (p.\pageref{women}). Additionally, the overall simplicity of Sanskrit is another indicator that this text addresses young princes in their education.}
\flushpage
    \end{tlate}
  \end{translation}
\end{alignment}
\pagebreak %after pp.141-142
%%%%%%%%%%%%%%%%%%%%%%%%%%%%%%%%%%%%%%%%%%
%%%%%%%%%%%%%%%%%%%%%%%%%%%%%%%%%%%%%%%%%% 
%%%%%%%%PAGEBREAK%%%%%%%PAGEBREAK%%%%%%%%%
%%%%%%%%%%%%%%%%%%%%%%%%%%%%%%%%%%%%%%%%%% 
%%%%%%%%%%%%%%%%PAGEBREAK%%%%%%%%%%%%%%%%%
%%%%%%%%%%%%%%%%%%%%%%%%%%%%%%%%%%%%%%%%%% 
%%%%%%%%PAGEBREAK%%%%%%%PAGEBREAK%%%%%%%%%
%%%%%%%%%%%%%%%%%%%%%%%%%%%%%%%%%%%%%%%%%% 
%%%%%%%%%%%%%%%%%%%%%%%%%%%%%%%%%%%%%%%%%% 
%%%%%%%%%%%%%%%%%%%%%%%%%%%%%%%%%%%%%%%%%% 
%%%%%%%%%%%%%%%%%%%%%%%%%%%%%%%%%%%%%%%%%% 
%%%%%%%%PAGEBREAK%%%%%%%PAGEBREAK%%%%%%%%%
%%%%%%%%%%%%%%%%%%%%%%%%%%%%%%%%%%%%%%%%%% 
%%%%%%%%%%%%%%%%PAGEBREAK%%%%%%%%%%%%%%%%%
%%%%%%%%%%%%%%%%%%%%%%%%%%%%%%%%%%%%%%%%%% 
%%%%%%%%PAGEBREAK%%%%%%%PAGEBREAK%%%%%%%%%
%%%%%%%%%%%%%%%%%%%%%%%%%%%%%%%%%%%%%%%%%% 
%%%%%%%%%%%%%%%%%%%%%%%%%%%%%%%%%%%%%%%%%% 
%%%%%%%%%%%%%%%%%%%%%%%%%%%%%%%%%%%%%%%%%% 
%%%%%%%%%%%%%%%%%%%%%%%%%%%%%%%%%%%%%%%%%% 
%%%%%%%%PAGEBREAK%%%%%%%PAGEBREAK%%%%%%%%%
%%%%%%%%%%%%%%%%%%%%%%%%%%%%%%%%%%%%%%%%%% 
%%%%%%%%%%%%%%%%PAGEBREAK%%%%%%%%%%%%%%%%%
%%%%%%%%%%%%%%%%%%%%%%%%%%%%%%%%%%%%%%%%%% 
%%%%%%%%PAGEBREAK%%%%%%%PAGEBREAK%%%%%%%%%
%%%%%%%%%%%%%%%%%%%%%%%%%%%%%%%%%%%%%%%%%% 
%%%%%%%%%%%%%%%%%%%%%%%%%%%%%%%%%%%%%%%%%%
\begin{alignment}[
  texts=edition[class="edition"];
  translation[class="translation"],
  ]
  \begin{edition}
    \begin{tlg}[59_2]
      \noindent
%-----------------------------
%bhāṣamāṇaḥ             ramamāṇaḥ svalīlayā/   bhāvābhāvavinirmukto  sarvagrahavivarjitaḥ// 1// %[p.84] \E
%bhāṣamāṇaḥ  sumadhuraṃ ramamāṇaḥ svalīlayā    bhāvābhāvavinirmuktaḥ sarvagrahavivarjjitaḥ   \P  %%%7681.jpg 
%bhakṣamāṇa  samaghura--ramāṇa    svalilayā    bhāvāvinir muktaḥ/    sarvagrāhavivarjitaḥ/            \B
%bhakṣamāṇaḥ samadhura--ramamāṇaṃ svalīlayā    bhāvāvinir muktaḥ     sarvagrāhavivarjitaḥ...      \L  [bhakṣamāṇaḥ= PPA] [rasamānaḥ =PPA] 
%bhāṣamāṇasya madhuraṃ  rasamānaḥ svalīlayā // bhāvābhāvavinirmuktaḥ sarvāgrahavivarjitaḥ// N1
%bhāṣamāṇasya madhuraṃ  rasamānaḥ svalīlayā // bhāvābhāvavinirmuktaḥ sarvāgrahavivarjitaḥ// N2
%\om \D
%bhāṣamāṇasya madhuraṃ  ramamāṇaḥ svalīlayā//  bhāvābhāvavinirmuktaḥ sarvagrāhavivarjitaḥ  \U1
%bhāṣamāṇaḥ sumadhuraṃ  ramamāṇaḥ svalīlayā//  bhāvābhāvavinirmuktaḥ sarvāgrahavivarjjitaḥ// \U2
%-----------------------------
%While he consumes excessively sweet things and delights in his own play, he is one who is liberated from both being and non-being, and free from all attachments.  
%-----------------------------26
      \note[type=source, labelb=426, labele=_426e, nosep]{cf. YSv (PT p. 847): nañubhayatra sambadhyate na vaktavyamityarthaḥ | yogādhikāriṇo'pi tatraiva || bhāvābhāvavinirmuktaḥ sarvagrahavivarjitaḥ |}
\tl{
  \app{\lem[wit={L}]{bhakṣamāṇaḥ}
    \rdg[wit={B}]{bhakṣamāṇa}
    \rdg[wit={E,P,U2}]{bhāṣamāṇaḥ}
    \rdg[wit={N1,N2,U1}]{bhāṣamāṇasya}}
    \app{\lem[wit={P,U2}]{sumadhuraṃ}
      \rdg[wit={B}]{samaghura°}
      \rdg[wit={L}]{samadhura°}
      \rdg[wit={N1,N2,U1}]{madhuraṃ}
      \rdg[wit={E}]{\om}}
    \app{\lem[wit={E,P,U1,U2}]{ramamāṇaḥ}
      \rdg[wit={N1,N2}]{rasamānaḥ}
      \rdg[wit={L}]{°ramamāṇaṃ}
      \rdg[wit={B}]{°ramāṇa}}
    svalilayā/}\\
  \tl{
    \note[type=source, labelb=_426xx, labele=_426e, nosep]{ \approx \emph{Amanaska} 1.12ab: bhāvābhāvavinirmuktaṃ vināśotpattivarjitam |}
    \app{\lem[wit={ceteri}]{bhāvābhāvavinirmuktaḥ}\linelabel{_426xx}
      \rdg[wit={E}]{bhāvābhāvavinirmukto}
      \rdg[wit={B,L}]{bhāvāvinir muktaḥ}}
    sarvagrāhavivarjitaḥ\dd{} \begin{otherlanguage}{english}\uproman{59}.2\end{otherlanguage}\hskip-2pt\dd{}}
  \linelabel{_426e}
\end{tlg}
  \begin{tlg}[59_3]
\noindent
% -----------------------------
%sadānaṃdamayo yogī sadābhyāsī sadā bhavet/     viruddhaduḥkhade  deśe virūpe   tibhayānake//1// \E
%sadānaṃdamayo yogī sadābhyāsī sadā bhavet      viruddhaduḥkhade  deśe virūpe   tibhayānake      \P
%sadāmayo      yogī sadābhyāsī sadā bhavet/     viruddhe duḥkhe   deśe śovirūpe   bhayānake/ \B
%sadāmayo    yoyogī sadābhyāsī sadā bhavet//    viruddhe duḥkha---deśe śovirūpe   bhayānake... \L
%sadānaṃdamayo yogī sadābhyāsī sadā bhavet/     viruddhe duḥkhade deśe śovirūpe tibhayānake/ \N1
%sadānaṃdamayo yogī sadābhyāsī sadā bhavet//1// viruddhe duḥkhade deśe virūpe   tibhayānake/ \N2 %%%last folio verso!!!!
%\om \D
%sadānaṃdamayo yogī sadābhyāso sadā bhavetd     viruddhe duḥkhade deśe vivarūpe   bhayānake \U1
%\om \U2
%-----------------------------
%The Yogin that is made of permanent bliss, is always engaged in practice, [even] in land which is hostile and uncomfortable, ugly and extremely terrible.  
%-----------------------------27
    \note[type=source, labelb=527, nosep]{cf. YSv (PT p. 847): sadānandamayo yogī sadābhyāsī sadā bhavet | viruddhe duḥkhadeśe ca virūpe 'tibhayānake |}
%  \note[type=philcomm, labelb=528, lem={sadānandamayo yogī \ldots 'tibhayānake}]{The serse is omitted in \getsiglum{U2}.}
\tl{
\app{\lem[wit={ceteri}]{sadānandamayo}
  \rdg[wit={B,L}]{sadāmayo}
\rdg[wit={U2}]{\om}}
\app{\lem[wit={ceteri}]{yogī}
  \rdg[wit={L}]{yoyogī}
\rdg[wit={U2}]{\om}}
\app{\lem[wit={ceteri}]{sadābhyāsī}
  \rdg[wit={U1}]{sadābhyāso}
\rdg[wit={U2}]{\om}}
\app{\lem[wit={ceteri}]{sadā bhavet}
\rdg[wit={U2}]{\om}}/}\\
\tl{
\app{\lem[wit={B,L,N1,N2,U1}]{viruddhe}
  \rdg[wit={E,P}]{viruddha°}
\rdg[wit={U2}]{\om}}
\app{\lem[type=emendation, resp=egoscr]{duḥkhadeśe ca}
  \rdg[wit={E,P,N1,N2,U1}]{duḥkhade deśe}
  \rdg[wit={B}]{duḥkhe deśe}
  \rdg[wit={L}]{duḥkhadeśe}
\rdg[wit={U2}]{\om}}
\app{\lem[wit={E,P,N2}]{virūpe}
  \rdg[wit={B,L,N1}]{śovirūpe}
  \rdg[wit={U1}]{vivarūpe}
  \rdg[wit={U2}]{\om}
}\app{\lem[wit={E,P,N1,N2}]{'tibhayānake}
  \rdg[wit={B,L,U1}]{bhayānake}
\rdg[wit={U2}]{\om}}\dd{} \begin{otherlanguage}{english}\uproman{59}.3\end{otherlanguage}\hskip-2pt\dd{}}
\end{tlg}
\begin{tlg}[59_4]
\noindent
%-----------------------------
%iṣṭādyaniṣṭasaṃsparśe     rase  ca lavaṇādike/       pratyādāv api gaṃdhe  ca kaṃkoṣṇādivivarjayet//2// \E
%iṣṭādhaniṣṭaṃ saṃsparśe   rase  ca lavaṇādike        pratyādāv api gaṃdhe  ca kaṃṭakoṣyādivivarjjite   \P
%iṣṭādyaniṣṭasaṃsparśe     rase  ca lavaṇādike        pratyādāv api gaṃdhe  ca kaṭakoṣmādivarji/         \B
%iṣṭādyaniṣṭasaṃsparśe     rase  ca lavaṇādike        pūtyādāv  api gaṃdhe  ca kaṃṭakoṣmādivarji//        \L
%iṣṭādyaniṣṭasaṃsparśe/    rase  ca lavaṇādike        pūtyādāv  api gaṃdhe  ca kaṃṭakoṣmādivarjjite/     \N1
%iṣṭādyaniṣṭaṃ saṃsparśe/  rasaṃ ca lavaṇādiko //2//  pūtyādāv  api gaṃdhaṃ ca kaṇṭakeṣmādivarjjite//     \N2
%iṣṭādyaniṣṭasaṃsparśe     rase  ca lavaṇādike        pūjādāv   api gaṃdhe  ca kuṃṭakoṣmādivarjite         \U1
% \om \U2
%\om \D
%-----------------------------
%In desireable and undesireable contact in tastes like salty etc., evil smells, thorns, etc. [and in the state of being] abandoned. 
%-----------------------------
%\note[type=philcomm, labelb=529, lem={iṣṭādyaniṣṭasaṃsparśe \ldots kaṇṭakoṣmādivarjane}]{The verse is omitted in \getsiglum{U2}.}
  \tl{
  \app{\lem[wit={ceteri}]{iṣṭādyaniṣṭasaṃsparśe}
    \rdg[wit={P,N2}]{iṣṭādhaniṣṭaṃ saṃsparśe}
  \rdg[wit={U2}]{\om}} 
\app{\lem[wit={ceteri}]{rase}
  \rdg[wit={N2}]{\om}
\rdg[wit={U2}]{\om}}
\app{\lem[wit={ceteri}]{lavaṇādike}
  \rdg[wit={N2}]{lavaṇādiko}
\rdg[wit={U2}]{\om}}/}\\
\tl{
  \app{\lem[wit={L,N1,N2},alt={pūtyādāv api}]{pūtyādāv-api}
    \rdg[wit={B,E,P}]{pratyādāv api}
    \rdg[wit={U1}]{pūjādāv api}
  \rdg[wit={U2}]{\om}}
  \app{\lem[wit={ceteri}]{gandhe}
    \rdg[wit={N2}]{gaṃdhaṃ}}
  \app{\lem[wit={ceteri}]{ca}
    \rdg[wit={U2}]{\om}}
  \app{\lem[wit={N1}]{kaṃṭakoṣmādivarjjite}
    \rdg[wit={N2}]{kaṇṭakeṣmādivarjjite}
    \rdg[wit={E}]{kaṃkoṣṇādivivarjayet}
    \rdg[wit={P}]{kaṃṭakoṣyādivivarjjite}
    \rdg[wit={B}]{kaṭakoṣmādivarji}
    \rdg[wit={L}]{kaṃṭakoṣmādivarji}
    \rdg[wit={U1}]{kuṃṭakoṣmādivarjite}}\dd{} \begin{otherlanguage}{english}\uproman{59}.4\end{otherlanguage}\hskip-2pt\dd{}}
\end{tlg}
\begin{tlg}[59_5]
\noindent
%-----------------------------
%sarvadaiva  sadābhyāsaḥ samaḥ syāt  sukhaduḥkhayoḥ/   evaṃ yogasya  karmmāṇi saṃkalparahitāni ca//3// %[p.85] \E
%sarvadaiva  sadābhyāsaḥ samaḥ syāt  sukhaduḥkhayoḥ    evaṃ yogasya  karmāṇi  saṃkalparahitāni ca              \P
%sarvadeva   sadābhyāsāḥ samaḥ syāt  sukhaduḥkhayoḥ/   evaṃ yogasya  karmmāṇi saṃkalparahitāni ca/ \B
%sarvadeva   sadābhyāsāḥ samaḥ sya/t sukhaduḥkhayoḥ//  evaṃ yogasya  karmmāṇi saṃkalparahitāni ca// \L
%sarvadeva   sadābhyāsāḥ sama  syāt  sukhaduḥkhayoḥ/   evaṃ yogasya  karmmāṇi saṃkalparahitāni ca/ \N1
%sarvadaiva  sadābhyāsāḥ mana  syāt  sukhaduḥkhayoḥ/   evaṃ bhūtasya karmāṇi saṃkalparahitāni ca/ \N2
%\om \D
%sarvadeva   sadābhyāsāḥ sama  syā   sukhaduḥkhayoḥ    evaṃ bhūta----karmāṇī saṃkalparahitāni ca        \U1
%sarvadaivaṃ sadābhyāsaḥ samaḥ syāt  sukhaduḥkhayoḥ//  evaṃ yogasya  karmāṇi saṃkalparahitāni ca//   \U2 
%-----------------------------
%He who is always truly in permanent practice shall be equanimous towards happiness and suffering. In this way the actions of the great yogin (\textit{bhūtasya}) are free from desire. 
%-----------------------------
\tl{
\note[type=source, labelb=530, labele=_530e, nosep]{ ≈  YSv (PT p. 847): etad aniṣṭasaṃsparśe nyūnādhike balādhike | evam bhūtasya karmāṇi saṅkalparahitāni ca |}
\note[type=source, labelb=531, labele=_530e, nosep]{ ≈  YSv (PT p. 847): utpātarahite deśe kaṇṭakādivivarjite | abhyasyate sadā yogaḥ samaḥ syāt sukhaduḥkhayoḥ |}
\app{\lem[wit={E,P,N2}]{sarvadaiva}
  \rdg[wit={B,L,N1}]{sarvadeva}
  \rdg[wit={U2}]{sarvadaivaṃ}}
\app{\lem[wit={E,P,U2}]{sadābhyāsaḥ}
  \rdg[wit={B,L,N1,N2,U1}]{sadābhyāsāḥ}}
\app{\lem[wit={ceteri}]{samaḥ}
  \rdg[wit={N1,U1}]{sama}
  \rdg[wit={N2}]{mana}}
\app{\lem[wit={ceteri},alt={syāt}]{syā\skp{t-su}}
  \rdg[wit={L}]{sya | t}
  \rdg[wit={U1}]{syā}
}\skm{t-su}khaduḥkhayoḥ/}\\
\tl{
evaṃ \app{\lem[wit={N2}]{bhūtasya}
  \rdg[wit={U1}]{bhūta}
  \rdg[wit={ceteri}]{yogasya}}
karmāṇi saṃkalparahitāni ca\dd{} \begin{otherlanguage}{english}\uproman{59}.5\end{otherlanguage}\hskip-2pt\dd{}}
\linelabel{_530e}
\end{tlg}
\begin{tlg}[59_6]
\noindent
%-----------------------------
%gacchan nṝṇāṃ ca saṃsparśāt tapaḥ kurvan na lipyate/  utpannatattvabodhasya hy udāsīnasya sarvadā//4// \E %%%% Amanaska 2.36 
%gacchan nṝṇāṃ ca saṃsparśāt pāpaḥ kurvan na lipyate   utpannatattvabodhasya udāsīnasya sarvadā        \P
%gacchan nṝṇāṃ ca saṃsparśot pāpaṃ kurvaṃ na lipyate/  utpannatattvabodhasya udāsīnasya sarvadā    \B
%gachan  nṝṇāṃ ca saṃsparśāt pāpaṃ kurvan na lipyate/  utpannatattvabodhasya udāsīnasya sarvadā    \L
%gacchan nṝṇāṃ ca saṃsparśot pāpaṃ kurvan na lipyate/  utpannatattvabodhasya udāsīnasya sarvadā//    \N1
%gacchan nṝṇāṃ ca saṃsparśāt pāpaṃ kurvaṃ na lipyate/  utpannatattvabodhasya udāsīnasya sarvadā//    \N2
%\om \D
%gacha   nṛṇāṃ ca saṃsparśāt pāpaṃ kurvaṃ nna lipyate  utpannatatvabodhasyād udāsīnasya sarvadā    \U1
%gacchan nṝṇāṃ ca saṃsparśāt pāpaṃ kurvan na lipyate// utpannatattvabodhasya udāsīnasya sarvadā// \U2
%-----------------------------
%He does not become tainted by sin by touching men while walking for him who has arisen to the awakening of reality, who is in every way equaminous. 
%-----------------------------
\tl{
  \note[type=source, labelb=532, labele=_532e, nosep]{ \approx  YSv (PT p. 847): evaṃ gacchan svapan paśyan pāpapuṇyairna lipyate | utpannatattvabodhaḥ syāt sadā śīlasya sarvadā |}
\note[type=source, labelb=533, labele=_532e, nosep]{ \approx  \citetitle{amanaskaed} 2.36: utpannatattvabodhasya hy udāsīnasya sarvadā | sadābhyāsaratasyaitan naikatrāpy upayujyate ||}
\app{\lem[wit={ceteri},alt={gacchan}]{gaccha\skp{n-nṛ}}
  \rdg[wit={U1}]{gacha}
}\skm{n-nṛ}ṇāṃ ca
\app{\lem[wit={ceteri},alt={saṃsparśāt}]{saṃsparśā\skp{t-pā}}
  \rdg[wit={B,N1}]{saṃsparśot}
}\app{\lem[wit={ceteri},alt={pāpaṃ}]{\skm{t-pā}paṃ}
  \rdg[wit={P}]{pāpaḥ}
  \rdg[wit={E}]{tapaḥ}}
kurvan-na lipyate/}\\
\tl{
utpannatattvabodhasya
\app{\lem[wit={ceteri}]{udāsīnasya}
  \rdg[wit={E}]{hy udāsīnasya}}
sarvadā\dd{} \begin{otherlanguage}{english}\uproman{59}.6\end{otherlanguage}\hskip-2pt\dd{}}\linelabel{_532e}
\end{tlg}
  \end{edition}
  \begin{translation}
    \begin{tlate}[59_2]
\paragraph{\uproman{59}.2} While he consumes very lovely things and delights in his own play, he is one who is liberated from both being and non-being and free from all attachments.
\end{tlate}
\begin{tlate}[59_3]
\paragraph{\uproman{59}.3} The Yogin one who is continously engaged in practice is always made of permanent bliss, [even] in a land which is hostile and uncomfortable, ugly and extremely terrible \ldots
\end{tlate}
\begin{tlate}[59_4]
\paragraph{\uproman{59}.4} \ldots in desireable and undesireable contact, in tastes like salty etc., evil smells, thorns, heat, etc. [or] being abandoned. 
\end{tlate}
\begin{tlate}[59_5]
\paragraph{\uproman{59}.5} He who is always truly in permanent practice shall be equanimous towards happiness and suffering. In this way, the actions of the great yogin (\textit{bhūtasya}) are free from desire.
\end{tlate}
\begin{tlate}[59_6]
  \paragraph{\uproman{59}.6} He who has arisen to the awakening of reality, who is in every way equanimous does not become tainted by sin going amidst people and engaging in contact.
  \end{tlate}
\flushpage 
  \end{translation}
\end{alignment}
\pagebreak %after pp. 143-144
%%%%%%%%%%%%%%%%%%%%%%%%%%%%%%%%%%%%%%%%%%
%%%%%%%%%%%%%%%%%%%%%%%%%%%%%%%%%%%%%%%%%% 
%%%%%%%%PAGEBREAK%%%%%%%PAGEBREAK%%%%%%%%%
%%%%%%%%%%%%%%%%%%%%%%%%%%%%%%%%%%%%%%%%%% 
%%%%%%%%%%%%%%%%PAGEBREAK%%%%%%%%%%%%%%%%%
%%%%%%%%%%%%%%%%%%%%%%%%%%%%%%%%%%%%%%%%%% 
%%%%%%%%PAGEBREAK%%%%%%%PAGEBREAK%%%%%%%%%
%%%%%%%%%%%%%%%%%%%%%%%%%%%%%%%%%%%%%%%%%% 
%%%%%%%%%%%%%%%%%%%%%%%%%%%%%%%%%%%%%%%%%% 
%%%%%%%%%%%%%%%%%%%%%%%%%%%%%%%%%%%%%%%%%% 
%%%%%%%%%%%%%%%%%%%%%%%%%%%%%%%%%%%%%%%%%% 
%%%%%%%%PAGEBREAK%%%%%%%PAGEBREAK%%%%%%%%%
%%%%%%%%%%%%%%%%%%%%%%%%%%%%%%%%%%%%%%%%%% 
%%%%%%%%%%%%%%%%PAGEBREAK%%%%%%%%%%%%%%%%%
%%%%%%%%%%%%%%%%%%%%%%%%%%%%%%%%%%%%%%%%%% 
%%%%%%%%PAGEBREAK%%%%%%%PAGEBREAK%%%%%%%%%
%%%%%%%%%%%%%%%%%%%%%%%%%%%%%%%%%%%%%%%%%% 
%%%%%%%%%%%%%%%%%%%%%%%%%%%%%%%%%%%%%%%%%% 
%%%%%%%%%%%%%%%%%%%%%%%%%%%%%%%%%%%%%%%%%% 
%%%%%%%%%%%%%%%%%%%%%%%%%%%%%%%%%%%%%%%%%% 
%%%%%%%%PAGEBREAK%%%%%%%PAGEBREAK%%%%%%%%%
%%%%%%%%%%%%%%%%%%%%%%%%%%%%%%%%%%%%%%%%%% 
%%%%%%%%%%%%%%%%PAGEBREAK%%%%%%%%%%%%%%%%%
%%%%%%%%%%%%%%%%%%%%%%%%%%%%%%%%%%%%%%%%%% 
%%%%%%%%PAGEBREAK%%%%%%%PAGEBREAK%%%%%%%%%
%%%%%%%%%%%%%%%%%%%%%%%%%%%%%%%%%%%%%%%%%% 
%%%%%%%%%%%%%%%%%%%%%%%%%%%%%%%%%%%%%%%%%%
\begin{alignment}[
  texts=edition[class="edition"];
  translation[class="translation"],
  ]
  \begin{edition}
    \begin{tlg}[59_7]
      \noindent
%-----------------------------
%tadā dṛṣṭiviśeṣaḥ syād vividhāny āsanāni ca/      aṃtaḥkaraṇajā bhāvā yogino nopayoginaḥ//5// \E %%%Amanaska 2.37
%tadā dṛṣṭiviśeṣa--syād vidhāny   āsanāni ca       aṃtaḥkaraṇajā bhāvā yogino nopayoginaḥ \P
%tadā dṛṣṭiviśeṣa--syād vidhāny   āsanāni ca/      aṃtaḥkaraṇajā bhāvā yogino nopayoginaḥ/ \B
%tadā dṛṣṭiviśeṣa--syād vidhāny   āsanāni ca//     aṃtaḥkaraṇajā bhāvā yogino nopi yoginaḥ// \L
%tadā dṛṣṭiviśeṣaḥ syād vidhāny   āsanāni ca/      aṃtaḥkaraṇajā bhavā yogino nopayoginaḥ/ \N1
%tadā dṛṣṭiviśeṣaḥ syād vividhāny āsanāni ca//5//  aṃtaḥkaraṇajā bhavā yogino nopayoginaḥ/ \N2
%tadā dṛṣṭir viśeṣasyād vidhāny   āsanāni ca       aṃtaḥkaraṇayo bhāvā yogino nopayoginaḥ \U1
%tadā dṛṣṭiviśeṣaḥ syād vividhāny āsanāni ca//     aṃtaḥkaraṇajā bhāvā yogino nopi yoginaḥ// \U2
%\om \D
%-----------------------------
%Then the different gazing points, the various postures and the states born from the internal organ are useless to the yogin. 
%-----------------------------
\note[type=source, labelb=534, labele=_534e, nosep]{ \approx  YSv (PT p. 847): pare dṛṣṭivilaṃ na syād vividhāni mṛtāni ca | antaḥkaraṇam etasya yogino niṣkriyaṃ [niṣkalā?] tu sa |}
\note[type=source, labelb=535, labele=_534e, nosep]{ \approx  \citetitle{amanaskaed} 2.37: tadā dṛṣtiviśeṣāś ca vividhāny āsanāni ca | antaḥkaraṇabhāvāś ca yogino nopayoginaḥ ||} %%%Variants see Birch 57
\tl{
tadā
\app{\lem[wit={E,N1,N2,U2}]{dṛṣṭiviśeṣaḥ}
  \rdg[wit={B,L,P}]{dṛṣṭiviśeṣa}
  \rdg[wit={U1}]{dṛṣṭir viśeṣa}}
syā\skp{d-vi}\app{\lem[wit={E,N2,U2},alt={vividhāny}]{\skm{d-vi}vidhā\skp{ny-ā}}
  \rdg[wit={ceteri}]{vidhāny}
}\skm{ny-ā}sanāni ca/}\\
\tl{
  \app{\lem[wit={ceteri}]{antaḥkaraṇajā}
    \rdg[wit={U1}]{aṃtaḥkaraṇayo}}
  bhavā yogino
  \app{\lem[wit={ceteri}]{nopayoginaḥ}
    \rdg[wit={L,U2}]{nopi yoginaḥ}}\dd{} \begin{otherlanguage}{english}\uproman{59}.7\end{otherlanguage}\hskip-2pt\dd{}}\linelabel{_534e}
\end{tlg}
\begin{tlg}[59_8]
  \noindent
%-----------------------------
%sarva rājapadasthasya  niṣkalādhyātmavedinaḥ/   yadyat prayatnaniḥpāyaṃ   tattatsarvam akāraṇam// 6// \E
%sarvadā sahajasthasya  niṣkalādhyātmavedinaḥ    yadyat prayatnaniḥpārdhaṃ tattatsarvam akāraṇam    \P
%sarvadya-sahajasya     niṣkalādhyātmavedinā/    yadyat prayatnaniḥpādya   tat sarvam   akāraṇāt/ \B
%sarvadya-sahajasthasya niṣkalādhyātmavedinā//   yadyat prayatnaniḥpādya   tat sarvem  ikāraṇāt//2// \L
%sarvadā  sahajasthasya niṣkalādhyātmavedina/    yadyat prayatnaniḥpādyaṃ  tattatsarvam akāraṇaṃ/ \N1
%sarvadā  sahajasthasya niṣkalādhyātmavedina/    yadyat prayatnaniḥpādyaṃ  tattatsarvam na kāraṇaṃ//7//  \N2
%sarvadā  mahajarasya   niṣkalādhyātmavedanā     yadyat aprayatra niṣyayiṃ tat sarvam  akāraṇāṃ  \U1
%sarvadā  sahajasthasya niṣkalādhyātmavedinaḥ//  yadyat prayatnaniḥpādyaṃ  tat sarvaṃ  kāraṇa//  \U2
%\om \D
%-----------------------------
%For the knower of the undivided supreme self, who is always in the natural state, whatever is to be generated with effort, all of that becomes causeless. 
%-----------------------------
  \note[type=source, labelb=536, labele=_536e, nosep]{ ≈  YSv (PT p. 847): sarvadā sahajas tasya niṣkalādhyātmavādinaḥ | yadā prayatnaniṣpādyaṃ grāhyaṃ sarvam akāraṇam |}
\tl{
  \app{\lem[wit={ceteri}]{sarvadā}
    \rdg[wit={B,L}]{sarvadya}
    \rdg[wit={E}]{sarva°}}
\app{\lem[type=emendation, resp=egoscr,alt={sahajas tasya}]{sahajas\skp{-}tasya}
  \rdg[wit={L,P,N1,N2,U2}]{sahajasthasya}
  \rdg[wit={B}]{sahajasya}
  \rdg[wit={U1}]{mahajarasya}
  \rdg[wit={E}]{rājapadasthasya}}
\app{\lem[wit={E,P,U2}]{niṣkalādhyātmavedinaḥ}
  \rdg[wit={B,L,U1}]{niṣkalādhyātmavedinā}
  \rdg[wit={N1,N2}]{niṣkalādhyātmavedina}}/}\\
\tl{
  yatya\skp{t-pra}\app{\lem[wit={N1,N2,U2},alt={prayatnaniṣpādyaṃ}]{\skm{t-pra}yatnaniṣpādyaṃ}
    \rdg[wit={U1}]{aprayatra niṣyayiṃ}
    \rdg[wit={B,L}]{prayatnaniḥpādya}
    \rdg[wit={P}]{prayatnaniḥpārdhaṃ}
    \rdg[wit={E}]{prayatnaniḥpāyaṃ}}
\app{\lem[wit={E,P,N1,N2},alt={tat tat sarvam}]{tat-tat-sarva\skp{m-a}}
  \rdg[wit={B,U1,U2}]{tat sarvam}
  \rdg[wit={L}]{tat sarvem}
}\app{\lem[wit={E,P}, alt={akāraṇāṃ}]{\skm{m-a}kāraṇam}
  \rdg[wit={U1}]{akāraṇāṃ}
  \rdg[wit={B}]{akāraṇāt}
  \rdg[wit={L}]{ikāraṇāt}
  \rdg[wit={N2}]{na kāraṇaṃ}
  \rdg[wit={U2}]{kāraṇa}}\dd{}\begin{otherlanguage}{english}\uproman{59}.8\end{otherlanguage}\hskip-2pt\dd{}}\linelabel{_536e}\\
\end{tlg}
\begin{prose}[p59_03]
  \noindent
%-----------------------------
%vilāsinīnāṃ manohārigānaśravaṇāt/ \E
%vilāsinīnāṃ manohārigānaśravaṇāt  \P
%vilāsinīnāṃ manohārigānaśravaṇāt/ \B
%vilāsinīnāṃ manohārigānaśravaṇāt// \L
%vilāsinīnāṃ manohārigītaśravaṇāt/ \N1
%vilāsinīnāṃ manohārigītaśravaṇāt/ \N2
%vilāsinīnāṃ manohārigītaśravaṇāt \U1
%vilāsinīnāṃ manohārigānaśravaṇāt// \U2
%\om \D
%-----------------------------
%Because of listening to mindblowing musical performance of charming women, ... 
%-----------------------------33
vilāsinīnāṃ
\app{\lem[wit={N1,N2,U1}]{manohārigītaśravaṇāt}
  \rdg[wit={Y}]{manohārigānaśravaṇāt}}\dd{} 
%-----------------------------
%atisauṃdaryakāminīnāṃ rūpadarśanāt/ \E
%atisuṃdaraṃ kāmināṃ   rūpadarśanāt    \P
%atisauṃdarakāminināṃ  rūpadarśanāt/ \B
%atisuṃdarakāminināṃ   rūpadarśanāt// \L
%atisuṃdarakāminīnāṃ   rūpadarśanāt// \N1
%atisuṃdarakāminīnāṃ   rūpadarśanāt// \N2
%\om \D
%atisuṃdarakāminīnāṃ   rūpadarśanāt  \U1
%atisuṃdarakāminīnāṃ   rūpadarśanāt//  \U2
%-----------------------------
%because of seeing the form of a extremely beautiful women, ... 
%-----------------------------33
\app{\lem[wit={N1,N2,U1,U2}]{atisundarakāminīnāṃ}
  \rdg[wit={E}]{atisauṃdaryakāminīnāṃ}
  \rdg[wit={P}]{atisuṃdaraṃ kāmināṃ}
  \rdg[wit={B}]{atisauṃdarakāminināṃ}
  \rdg[wit={L}]{atisauṃdarakāminināṃ}}
rūpadarśanāt\dd{}
%-----------------------------
%kastūrī karpūrayor gaṃdhagrahaṇāt/ \E
%kastūrī karpūrayor gaṃdhagrahaṇāt \P
%kastūrī karpūrayor gaṃdhagrahaṇāt/ \B
%kastūrī karpūra----gaṃdhayor grahaṇāt/ \L  %%%S.15 Anfang!!
%\om \D
%kastūrī karpūra----gaṃdhayār gaṃdhagrahaṇāt/ \N1???
%kasturī karpūra----gandhagrahaṇāt/ \N2
%kastūri karpuro    gaṃdhagrahaṇāt \U1
%kastūrī karpūrayo  gaṃdhagrahaṇāt// \U2
%-----------------------------
%from smelling the fragnance of camphor and musk, ... 
%-----------------------------33
\app{\lem[wit={ceteri},alt={kastūrī°}]{kastūrī}
  \rdg[wit={U1}]{kastūri°}
}\app{\lem[wit={L}]{karpūragandhayo\skp{r-gra}}
  \rdg[wit={B,E,P}]{karpūrayor gaṃdha°}
  \rdg[wit={N1}]{karpūragaṃdhayār gaṃdha°}
  \rdg[wit={N2}]{karpūragandha°}
  \rdg[wit={U1}]{karpuro gaṃdha°}
  \rdg[wit={U2}]{karpūrayo gaṃdha°}}
grahaṇāt\dd{}
%-----------------------------
%manaḥ śaityakāri    komala----vastunaḥ  sparśakāraṇāt/ \E
%manaḥ śaityakāri    komala----vastunaḥ  sparśakāraṇāt \P
%manaḥ śaityakāri    komala----vastunaḥ  saṃsparśakāṃ... \B
%manaḥ śaityakāri    komala----vastunaḥ/ saṃsparśakaṃ... \L
%manaḥ śītalakārī atikomalaparavastunaḥ  sparśakaraṇāt// \N1
%manaḥ śītalakārī atikomalaparavastunaḥ  sparśakaraṇāt \N2
%\om \D
%manaḥ sīlakārī   atikomalaparavastunaḥ  sparśakaraṇāt \U1
%manaḥ śaityakāri    komala----vastunaḥ  sparśakāraṇāt// \U2
%-----------------------------
%because of the execution of touching of very soft things  .... the mind that is free from passion 
%-----------------------------33
manaḥ
\app{\lem[type=emendation, resp=egoscr, alt={śītalakārī}]{śītalakā\skp{ry-a}}
  \rdg[wit={N1,N2}]{śītalakārī}
  \rdg[wit={U1}]{sīlakārī}
  \rdg[wit={ceteri}]{śaityakāri}
}\app{\lem[wit={N1,N2,U1}, alt={°atikomalaparavastunaḥ}]{\skm{ry-a}tikomalaparavastunaḥ}
  \rdg[wit={ceteri}]{komalavastunaḥ}}
\app{\lem[wit={ceteri}]{sparśakāraṇāt}
  \rdg[wit={B}]{saṃsparśakāṃ}
  \rdg[wit={L}]{saṃsparśakaṃ}}\dd{}
%-----------------------------
%atimādhuryaṃ citte karoti/ \E
%atimādhuryaṃ citte karoti \P
%atimādhuryaṃ citte karoti/ \B
%atimādhuryaṃ citte karoti// \L
%atimādhuryaṃ citte karoti/ \N1
%atidhūryaṃ  cittaṃ karoti/ \N2
%\om \D
%atimādhuryaṃ citte karoti \U1
%atimādhuryaṃ cikrī karoti/ \U2
%-----------------------------
%assumes a supremely gentle state within the mental faculty.  
%-----------------------------34
atimādhuryaṃ \app{\lem[wit={ceteri}]{citte}
  \rdg[wit={N2}]{cittaṃ}
  \rdg[wit={U2}]{cikrī}}
karoti/\\
%-----------------------------
%tādṛśaḥ svādanāt/ \E
%tādṛśaḥ svādanāt \P
%tādṛśaḥ svādanāt/ \B
%tādṛśaḥ svādanāt// \L
%tādṛśā  svādanāt/ \N1
%tādṛśā  svādanāt/ \N2
%\om \D
%tādṛśā  svādanāt \U1
%tādṛśā  svādanāt// \U2
%-----------------------------
%From such enjoyment 
%-----------------------------35
\app{\lem[wit={B,E,L,P}]{tādṛśaḥ}
  \rdg[wit={N1,N2,U1,U2}]{tādṛśā}}
svādanāt\dd{}
%-----------------------------
%anekadeśānāṃ sādhvasādhusthānadarśanāt/ \E
%anekadeśānāṃ sādhvasādhusthānadarśanāt  \P
%anekadeśānāṃ sādhvasādhusthānadarśanāt/ \B
%anekadeśānāṃ sādhvasādhusthānadarśanāt// \L
%anekadeśānāṃ sādhvasādhusthānadarśanāt/ \N1
%anekadeśānāṃ sādhvasādhusthānadarśanāt/ \N2
%\om \D
%anekadeśānāṃ       sādhūsthānadarśanāt \U1
%anekadeśānāṃ sādhvasādhusthānadarśanāt// \U2
%-----------------------------
%From seeing good and bad places of many countries, 
%-----------------------------35
anekadeśānāṃ \app{\lem[wit={ceteri}]{sādhvasādhusthānadarśanāt}
  \rdg[wit={N1}]{sādhūsthānadarśanāt}}\dd{} 
%-----------------------------
%mitreṇa  saha komalavacanāt/ \E
%maitreṇa saha komalavacanāt \P
%maitreṇa saha komalavacanāt/ \B
%maitreṇa saha komalavacanāt// \L
%maitreṇa saha komalavacanāt/ \N1
%maitreṇa saha komalavacanāt/ \N2
%\om \D
%maitreṇa saha komalavacanāt \U1
%maitreṇa saha komalavacanāt// \U2
%-----------------------------
%from speaking sweet with friends,  
%-----------------------------35
\app{\lem[wit={ceteri}]{maitreṇa}
  \rdg[wit={E}]{mitreṇa}}
saha komalavacanāt\dd{}
%-----------------------------
%śatruṇā  saha kaṭhinavacanāt/ \E
%śatruṇā  saha kaṃvinya vacanāt  \P
%śatruṇā  saha kaṭhinya vacanāt/ \B
%śatruṇā  saha kāṭhinya vacanāt// \L
%śatruṇāṃ saha kaṭhinya vacanān \N1
%śatruṇāṃ saha kavinya vacanād- \N2
%\om \D
%śatṛṇā   saha kāṭhinya vacanāt    \U1
%śatruṇāṃ saha kāṭhinya vacanāt// \U2
%-----------------------------
%from speaking with firmness of character to enemies, 
%-----------------------------35
\app{\lem[wit={B,E,L,P,U1}]{śatruṇā}
  \rdg[wit={N1,N2,U2}]{śatruṇāṃ}}
saha \app{\lem[wit={L,U1,U2},alt={kāṭhinya°}]{kāṭhinya}
  \rdg[wit={E}]{kaṭhina°}
  \rdg[wit={P}]{kaṃvinya°}
  \rdg[wit={B}]{kaṭhinya°}
  \rdg[wit={N2}]{kavinya°}}
\app{\lem[wit={ceteri}, alt={vacanāt}]{vacanāt}
  \rdg[wit={N1}]{vacanān}
  \rdg[wit={N2}]{vacanād}}\dd{}
\note[type=philcomm, labelb=537, lem={vacanāt}]{The evidence of manuscript \getsiglum{B} stops here. The last folio of the manuscript is missing.}
%-----------------------------
%yasya manasi harṣo vā dveṣo na    bhavati   sa puruṣa  īśvaropadeśiko jñeyaḥ/ \E
%yasya manasi harṣo vā dveṣo na    bhavati   sa puruṣa  īśvaropadeśako jñeyaḥ \P
%yasya mana   harṣo vā dveṣo       bhavati   sa puruṣa  īśvaropade ko  jñeyaḥ/ \L
%yasya manasi harṣo vā dveṣo na    bhavati/  sa puruṣa  īśvaropadeśako jñeyaḥ// \N1
%yasya manasi harṣo vā dveṣo na    bhavati/  sa puruṣa  īśvaropadeśako jñeyaḥ  \N2
%yasya manasī harṣo vā dveṣo vā na bhavati   sa puruṣa  īśvaropadeśako jñeyaḥ \U1
%yasya manasī harṣo vā dveṣo na    bhavati//    pururṣo īśvaropadeśako jñeyaḥ// \U2
%                                     vati// sa puruṣa  īśvaropadeśako jñeyaḥ// \D
% missing last folio \B
%-----------------------------
%love and hatred does not arise in his mind. This person is to be known as a teacher of the supreme god. 
%-----------------------------35-36
yasya
\app{\lem[wit={ceteri}]{manasi}
    \rdg[wit={U1,U2}]{manasī}
    \rdg[wit={L}]{mana}}
harṣo vā dveṣo
\app{\lem[wit={ceteri}]{na}
    \rdg[wit={U1}]{vā na}
    \rdg[wit={L}]{\om}}
bhavati/
\note[type=philcomm, labelb=538, lem={dveṣo na bhavati}]{The \textit{lacuna} of witness \getsiglum{D} ends here and its textual evidence resumes.}
\app{\lem[wit={ceteri}]{sa puruṣa}
   \rdg[wit={U2}]{puruṣo}}
\app{\lem[wit={ceteri}]{īśvaropadeśako}
   \rdg[wit={L}]{īśvaropade ko}}
 jñeyaḥ/
%-----------------------------
%svalīlayā    vadati calati      bhāvābhāvayoś cittam udāsīnaṃ  bhavati   kasyāṃcid vārtāyāṃ  harṣaviṣādaṃ na karoti   yasya manaḥ   sahajānaṃde    magnaṃ  bhavati/ \E   [p.87] 
%svalīlayā    vadati calati va   bhāvābhāvayoś cittam udāsīnaṃ  bhavati   kasyāṃcid vārttāyāṃ       haṭhaṃ na karoti   yasya manaḥ   sahajānaṃde    magnaṃ  bhavati \P
%svalīlayā    vadati calati ca   bhāvābhāvayoś cittam udāsīna   bhavati// kasyāṃcid vārttāyāṃ        haṭaṃ na karoti// yasya manaḥ// sahajānaṃdam    añjaṃ  bhavati/ \L
%svalīyayā    vadati calati ca   bhāvābhāvayoś cittam udāsīnaṃ  bhavati/  kasyāṃcid vārttāyāṃ       haṭhaṃ na karoti   yasya manaḥ   sahajānaṃde    magnaṃ  bhavati/ \N1
%svalīyayā    vadati calati ca/  bhāvābhāvayoś cittam udāsīnaṃ  bhavati/  kasyāṃcid vārttāyāṃ       haṭhaṃ na karoti/  yasya mana    sahajānaṃde    magnaṃ  bhavati \N2
%svalīlayā    vadati calati ca   bhāvābhāvayoś cittam udāsīnaṃ  bhavati// kasyāṃcid vārttāyāṃ       haṭhaṃ na karoti   yasya manaḥ   sahajānaṃde    magnaṃ  bhavati/ \D
%svalīlayā    vadati calati ca   bhāvābhāvayoś cittam udāsīnaṃ  bhavati   kasyāṃcid vārttāyāṃ        haṭaṃ na karoti   yasya manaḥ   sahajānaṃda    saṃjñaṃ bhavati \U1 %%%304.jpg
%svalīlayā    vadati calatī ca// bhāvābhāvayoś cittam udāsīnaṃ  bhavati// kasyāṃcid vārttāyāṃ       haṭhaṃ na karoti// yasya manaḥ   sahajānaṃ daṃde magnaṃ bhavati// \U2
%missing last folio \B  -----------------------------------------------------------vārttāyāṃ= Loc. Singular 
%-----------------------------
%Through his own [divine] play, he speaks and moves. The mind is equanimous in both existence and non-existence. It is not even possible to express in words forcefully [how it is] when his mind is immersed in inherent bliss. 
%-----------------------------37-39
 \label{udasin}
\app{\lem[wit={ceteri}]{svalīlayā}
  \rdg[wit={N1,N2}]{svalīyayā}}
vadati calati
\app{\lem[wit={ceteri}]{ca}
  \rdg[wit={P}]{va}
  \rdg[wit={E}]{\om}}
bhāvābhāvayoś-cittam-udāsīnaṃ bhavati/
kasyāṃcid-vārttāyāṃ
\app{\lem[wit={ceteri}]{haṭhaṃ}
  \rdg[wit={E}]{harṣaviṣādaṃ}
  \rdg[wit={L,U1}]{haṭaṃ}} na karoti/
yasya
\app{\lem[wit={ceteri}]{manaḥ}
  \rdg[wit={N2}]{mana°}}
\app{\lem[wit={ceteri}]{sahajānande}
  \rdg[wit={L}]{sahajānaṃdam}
  \rdg[wit={U1}]{sahajānaṃda}
  \rdg[wit={U2}]{sahajānaṃ daṃde}}
\app{\lem[wit={ceteri}]{magnaṃ}
  \rdg[wit={L}]{añjaṃ}
  \rdg[wit={U1}]{saṃjñaṃ}}
bhavati/
    \end{prose}
  \end{edition}
  \begin{translation}
    \begin{tlate}[59_7]
       \paragraph{\uproman{59}.7} Then the different gazing points, the various postures\footnote{Postures (\textit{āsana}s are not discussed in Rāmacandra's text. Together with this passage, however, they are mentioned thrice (cf. p.\pageref{ashtanga} l.6 and p.\pageref{majesty} l.2) and given these three statements, one must assume that the audience of this text practised them, but other works were consulted.} and the states produced from the internal organ\begin{buber}[f59_1]\footnote{\citeauthor{birch2013} (2013: 368, n. 52) wrote a long endnote on the compound \textit{antaḥkaraṇabhāvā} that occurs in \citetitle{amanaskaed} 2.37c which is a possible source of this verse in Rāmacandra's \textit{Yogatattvabindu}. I briefly summarize the endnote and complement my findings: Even though the Nepalese commentator of the \citetitle{amanaskaed} interprets it as the ``intention of the heart'' (\textit{antaḥkaraṇakā manasuvā pani}) \citeauthor{birch2013}, who has not found the compound in other texts back in 2013 proposes the more plausible explanation that \textit{antaḥkaraṇabhāva} refers to the eight mental states characterizing the higher faculty of discernment (\textit{buddhibhāva}). These eight states, as outlined in \citetitle{mainkar2004} 23, include religious merit (\textit{dharma}), gnosis (\textit{jñāna}), dispassion (\textit{virāga}), and power (\textit{aiśvarya}) as positive aspects, while their opposites (\textit{adharma, ajñāna, avirāga}, and \textit{anaiśvarya}) form the negative states (\textit{adhyavasāyo buddhir dharmo jñānaṃ virāga aiśvaryam} | \textit{sāttvikam etad rūpaṃ tāmasam asmād viparyastam}). According to \citeauthor{birch2013}'s research, this division of \textit{buddhi} is adopted by Śaiva traditions, such as \citetitle{malinivijay} 1.30c-d. The eightfold \textit{buddhi} in the \textit{Mālinīvijayottaratantra} has been noted in \citeauthor[2004: 353, n. 8]{vasudeva2004}. The concept of the \textit{buddhibhāva}s is also present in other contexts, like the hierarchical system of religions in the \citetitle{svacchandatantra} (11.179-182). In this scheme, deluded individuals are influenced by dark \textit{buddhibhāva}s. The specific formulation of Rāmacandra with the phrase \textit{antaḥkaraṇajā bhāvā} however, I was able to locate within \citetitle{bodhasara} 17.5: \textit{mano} buddhir \ldots}\end{buber} are useless to the yogin.
    \end{tlate}
    \begin{tlate}[59_8]
\paragraph{\uproman{59}.8} For the knower of the undivided supreme self, who is always in the natural state, whatever is to be generated with effort, all that is entirely unnecessary.\\\\
\end{tlate}
\begin{tlate}[p59_03]
Because of listening to the mindblowing musical performances of charming women, seeing the form of stunning women, smelling the fragrance of camphor and musk, and touching very soft and high-quality objects, the cooled-down mind, free from passion, assumes an extremely pleasant state within the mental faculty.

As a result of such enjoyment, as a result of seeing good and bad places of many countries, speaking sweetly with friends, and speaking with firmness of character to enemies, love and hatred do not arise in his mind. This person is to be known as a teacher of the supreme lord. Through his own [divine] play, he speaks and moves. The mind is equanimous in both existence and non-existence. It is not even possible to express in words forcefully [how it is] when his mind is immersed in inherent bliss.
%\flushpage 
    \end{tlate}
  \end{translation}
\end{alignment}
\pagebreak %after 145-146
%%%%%%%%%%%%%%%%%%%%%%%%%%%%%%%%%%%%%%%%%%
%%%%%%%%%%%%%%%%%%%%%%%%%%%%%%%%%%%%%%%%%% 
%%%%%%%%PAGEBREAK%%%%%%%PAGEBREAK%%%%%%%%%
%%%%%%%%%%%%%%%%%%%%%%%%%%%%%%%%%%%%%%%%%% 
%%%%%%%%%%%%%%%%PAGEBREAK%%%%%%%%%%%%%%%%%
%%%%%%%%%%%%%%%%%%%%%%%%%%%%%%%%%%%%%%%%%% 
%%%%%%%%PAGEBREAK%%%%%%%PAGEBREAK%%%%%%%%%
%%%%%%%%%%%%%%%%%%%%%%%%%%%%%%%%%%%%%%%%%% 
%%%%%%%%%%%%%%%%%%%%%%%%%%%%%%%%%%%%%%%%%% 
%%%%%%%%%%%%%%%%%%%%%%%%%%%%%%%%%%%%%%%%%% 
%%%%%%%%%%%%%%%%%%%%%%%%%%%%%%%%%%%%%%%%%% 
%%%%%%%%PAGEBREAK%%%%%%%PAGEBREAK%%%%%%%%%
%%%%%%%%%%%%%%%%%%%%%%%%%%%%%%%%%%%%%%%%%% 
%%%%%%%%%%%%%%%%PAGEBREAK%%%%%%%%%%%%%%%%%
%%%%%%%%%%%%%%%%%%%%%%%%%%%%%%%%%%%%%%%%%% 
%%%%%%%%PAGEBREAK%%%%%%%PAGEBREAK%%%%%%%%%
%%%%%%%%%%%%%%%%%%%%%%%%%%%%%%%%%%%%%%%%%% 
%%%%%%%%%%%%%%%%%%%%%%%%%%%%%%%%%%%%%%%%%% 
%%%%%%%%%%%%%%%%%%%%%%%%%%%%%%%%%%%%%%%%%% 
%%%%%%%%%%%%%%%%%%%%%%%%%%%%%%%%%%%%%%%%%% 
%%%%%%%%PAGEBREAK%%%%%%%PAGEBREAK%%%%%%%%%
%%%%%%%%%%%%%%%%%%%%%%%%%%%%%%%%%%%%%%%%%% 
%%%%%%%%%%%%%%%%PAGEBREAK%%%%%%%%%%%%%%%%%
%%%%%%%%%%%%%%%%%%%%%%%%%%%%%%%%%%%%%%%%%% 
%%%%%%%%PAGEBREAK%%%%%%%PAGEBREAK%%%%%%%%%
%%%%%%%%%%%%%%%%%%%%%%%%%%%%%%%%%%%%%%%%%% 
%%%%%%%%%%%%%%%%%%%%%%%%%%%%%%%%%%%%%%%%%%
\begin{alignment}[
  texts=edition[class="edition"];
  translation[class="translation"],
  ]
  \begin{edition}
    \begin{prose}[p59_04]
      \noindent
%-----------------------------
%tena      puruṣeṇa dṛṣṭiḥ sthirā karttavyā/ \E
%tena bhya puruṣeṇa dṛṣṭiḥ sthirā karttavyā \P
%tena      puruṣeṇa dṛṣṭiḥ sthirā karttavyā/ \L
%tena      puruṣeṇa dṛṣṭiḥ sthirā karttavyaṃ// \N1
%tena      puruṣeṇa dṛṣṭiḥ sthirā karttavyaṃ \N2
%tena   svapuruṣeṇa dṛṣṭiḥ sthirā karttavyaṃ// \D
%tena      puruṣeṇa dṛṣṭi--sthirā karttavyaḥ \U1
%tena      puruṣeṇa dṛṣṭiḥ sthirā karttavyā// \U2
% missing last folio \B
%-----------------------------
%By this person the stabilization of the gaze shall be performed. 
%-----------------------------40
tena
\app{\lem[wit={ceteri}]{puruṣeṇa}
  \rdg[wit={P}]{bhya puruṣeṇa}
  \rdg[wit={D}]{svapuruṣeṇa}}
\app{\lem[wit={ceteri}]{dṛṣṭiḥ}
  \rdg[wit={U1}]{dṛṣṭi°}}
sthirā
\app{\lem[wit={ceteri}]{karttavyā}
  \rdg[wit={D,N1,N2}]{karttavyaṃ}}/
%-----------------------------
%āsanaṃ dṛḍhaṃ karttavyam/ \E
%āsanaṃ dṛḍhaṃ karttavyaṃ  \P
%āsanaṃ dṛḍhaṃ karttavyaṃ/  \L
%āsanaṃ dṛḍhaṃ karttavyaṃ/  \N1
%āsanaṃ dṛḍhaṃ karttavyaṃ/  \N2
%āsanaṃ dṛḍhaṃ karttavyaṃ//  \D
%āsana--dṛḍhaṃ karttavyaṃ  \U1
%āsanaṃ dṛḍhaṃ karttavyam// \U2
%missing last folio \B
%-----------------------------
%[And] the posisiton shall be made stable. 
%-----------------------------41
\app{\lem[wit={ceteri}]{āsanaṃ}
  \rdg[wit={U1}]{āsana°}}
dṛḍhaṃ karttavyaṃ/ 
%-----------------------------
%pavanaḥ sthiraḥ karttavyaḥ/ \E
%pavanaḥ sthiraḥ karttavyaḥ \P
% pavanaḥ sthiraḥ karttavyaḥ \N1
% pavanaḥ sthiraḥ karttavyaṃ// \N2
%pavanaḥ sthiraḥ karttavyaḥ// \D
%pavanaḥ sthiraḥ karttavyaḥ \U1
%pavanaḥ sthiraḥ karttavyaḥ// \U2
% missing last folio \B
% \om \L
%-----------------------------
%[And] the breath shall be stabilized.  
%-----------------------------42
%\note[type=philcomm, labelb=539, labele= _539e, lem={pavanaḥ sthiraḥ karttavyaḥ}]{Sentence omitted in \getsiglum{L}.}
\app{\lem[wit={ceteri}]{pavanaḥ sthiraḥ}
  \rdg[wit={L}]{\om}} 
\app{\lem[wit={ceteri}]{karttavyaḥ}
  \rdg[wit={N2}]{karttavyaṃ}
  \rdg[wit={L}]{\om}}/  
%-----------------------------
%etādṛśaḥ   kaścin niyamaḥ/  siddhasya noktaḥ \E
%etādṛśaḥ   kaścin niyamaḥ   siddhasya noktaḥ  \P
%etādṛśaḥ// kaścin niyamaḥ// siddhasya noktaḥ//  \L
%etādṛśaḥ   kaścinniyamaḥ/   siddhasya noktaḥ/ \N1
%etādṛśaḥ   kaścinniyamaḥ    siddhasya noktaḥ  \N2
%etādṛśaḥ   kaścin niyamaḥ// siddhasya noktaḥ//   \D
%etādṛśaḥ   kaścinīyamaḥ     siddhasya noktaḥ \U1
%etādṛśaḥ   kaścinnīyamaḥ//  siddhasya noktaḥ \U2
% missing last folio \B
%-----------------------------
%Such a discipline is not taught in front of an accomplished one.
%-----------------------------43
etādṛśaḥ \app{\lem[wit={ceteri},alt={kaścin niyamaḥ}]{kaści:\\n\skp{-}niyamaḥ}
  \rdg[wit={U1}]{kaści nīyamaḥ}
  \rdg[wit={U2}]{kaścin nīyamaḥ}}
siddhasya noktaḥ/
%-----------------------------
%manaḥpavanābhyāṃ  yadā sahajānaṃda----svasvarūpeṇa prakāśyate   sa sahajayogaḥ   kathyate/ \E
%manaḥpavanābhyāṃ  yadā sahajānaṃdaḥ   svasvarūpeṇa prakāśyate   sa sahajayogaḥ   kathyate \P
%mana--pavanābhyāṃ yadā sahajānaṃda----svasvarūpeṇa prakāśyate   sa sahajayogaḥ// kathyate... \L
%manaḥpavanābhyāṃ  yadā sahajānaṃdaḥ/  svasvarūpeṇa prakāśyate   sa sahajayoga    kathyate/ \N1
%manaḥpavanābhyāṃ  yadā sahajānaṃdaḥ   svasvarūpeṇa prakāśyate   sa sahajo yogya  kathyate// \N2
%manaḥpavanābhyāṃ  yadā sahajānaṃdaḥ// svasvarūpeṇa prakāśyate   sa sahajayoga    kathyate// \D
%manaḥpavanābhyāṃ  yadā sahajānaṃdaḥ   svasvarūpeṇa prakāśate    sa sahayogaḥ     kathyate \U1
%manaḥpavanābhyāṃ  yadā sahajānaṃdaḥ   svasvarūpeṇa prakāśyate// sa sahajayogaḥ   kathyate// \U2
% missing last folio \B
%-----------------------------
%When by means of mind and breath the natural bliss appears through ones own true nature, it is called natural yoga (\textit{sahajayoga}.
%-----------------------------44-45
\app{\lem[wit={ceteri}]{manaḥpavanābhyāṃ}
  \rdg[wit={L}]{manapavanābhyāṃ}}
yadā
\app{\lem[wit={ceteri}]{sahajānandaḥ}
  \rdg[wit={E,L}]{sahajānaṃda°}}
svasvarūpeṇa
\app{\lem[wit={ceteri}]{prakāśyate}
  \rdg[wit={U1}]{prakāśate}}/
sa
\app{\lem[wit={ceteri}]{sahajayogaḥ}
  \rdg[wit={D,N1}]{sahajayoga}
  \rdg[wit={N2}]{sahajo yogya}
  \rdg[wit={U1}]{sahayogaḥ}}
kathyate/
%-----------------------------
%te      madhye     iti cakravarttikathanam// \E
%te      madhye     iti cakravarttikathanam// \P
%rājayogamadhye     iti cakravarti         kathyate  \L
%rājayogamadhye     iti cakravartti  nāma kathanaṃ// \N1
%rājayogamadhye     iti cakravarttīnāmakathanam // \N2
%rājayogamadhye     iti cakravarttī  nāma kathanaṃ// \D
%rājayogamadhye     iti cakravaktya  nāma madhye iti cakravartye nāma madhye kathanaṃ  \U1
%rājayogasya madhye iti cakravarti         kathyate// \U2
% missing last folio \B
%-----------------------------
%Within Rājayoga "universal ruler" (\textbf{cakravarttī}) narration of the name.
%"Within the realm of Rajayoga, it is referred to as the narrative 'Universal Ruler.'"
%-----------------------------46
\app{\lem[wit={ceteri}]{rājayogamadhye}
  \rdg[wit={U2}]{rājayogasya madhye}
  \rdg[wit={E,P}]{te madhye}}
iti
\app{\lem[wit={D,N2}]{cakravartī}
  \rdg[wit={E,P,L,N1,U2}]{cakravarti}
  \rdg[wit={U1}]{cakravaktya}}
\app{\lem[wit={X},alt={nāma}]{nāma}
  \rdg[wit={Y}]{\om}}\app{\lem[wit={ceteri}, alt={kathanaṃ}]{kathanaṃ}
  \rdg[wit={L,U2}]{kathyate}
  \rdg[wit={U1}]{madhye iti cakravartye nāma madhye kathanaṃ}}/
\end{prose}
  \end{edition}
  \begin{translation}
    \begin{tlate}[p59_04]
      \noindent
      \begin{euber}[f59_1]\blfootnote{\hspace{-2.2em}\textit{ahaṅkāraś cittaṃ ceti catuṣṭayam} | \textit{antaḥkaraṇajā bhāvā ātmā śuddho nirañjanaḥ} || 17.15 || ``The quadruplet — mind, intellect, ego, and consciousness — are states produced by the inner organ. The self, however, is pure and untainted.'' This fourfold scheme is very close to the fivefold version taught by Rāmacandra in his metaphysics section on p.\pageref{greatelements2} in which \textit{consciousness} (\textit{caitanya}) is added as a fifth item. The verses of chapter 17 provide information about further resulting states and their differentiations. The states born of the inner organ originally derive from the qualities (\textit{guṇa}s) of original matter (\textit{prakṛti}) (17.4). The four states of the inner organs produce more differenciated states (\textit{bhāvāḥ}) - \textit{manas}, viz., ``will and indecision'' (\textit{saṃkalpa} and \textit{vikalpa}) (17.6); \textit{buddhi}, viz., ``ascertainment'' (\textit{niścaya}) (17.7); \textit{ahaṃkāra}, viz., ``knowing, doing, experiencing, the sense of agency, and the concepts of being the doer or the enjoyer, as well as notions of injury and harm'' (\textit{jñatvakartṛtvabhoktṛtvavadhyaghātakatādayaḥ}) (17.8) of the \textit{citta}, viz., ``memory of past experiences and recognition'' (\textit{smṛtiḥ pūrvānubhūtasya pratyabhijñā}) (17.9). For the whole account of further divisions of the \textit{bhāva}s see \citetitle{bodhasara} 17.1-51.}\end{euber}
By a [unaccomplished] person, the gaze shall be stabilized. The position shall be stabilized. The breath shall be stabilized. Such a rule [however] is not prescribed for the accomplished [person].\footnote{With this statement, Rāmacandra repeats the message of verse \uproman{59}.7. Once the highest state of yoga is reached, yoga practice becomes unnecessary.} When by means of mind and breath, the natural bliss appears through one's own true nature, it is called Sahajayoga (``natural yoga''). Among [the methods] of Rajayoga, it is referred to by the name of ``Universal Ruler''\begin{buber}[f59_2]\footnote{Beyond this very passage, the term \textit{cakravartin} is absent in the Haṭha- and Rājayoga literature known to me. However, there are a few noteworthy occurrences of the term in yogic literature, i.e. Hemacandra's \citetitle{hemacandras} 4.19-21: \textit{dhanahīnaḥ śatam ekaṃ sahasraṃ śatavān api} | \textit{sahasrādhipatir lakṣaṃ koṭiṃ lakṣeśvaro ’pi ca} ||4.19|| \textit{koṭīśvaro narendratvaṃ narendraś cakravartitām} | \textit{cakravartī ca devatvaṃ devo ’pīndratvam icchati} ||4.20|| \textit{indratve ’pi hi saṃprāpte yadīcchā na nivartate} | \textit{mūle laghīyāṃs tallobhaḥ śarāva iva vardhate} ||4.21|| \citeauthor{hemacandras} (Ed. p. 80) translates: ``One who is poor [wants] a hundred [rupees]; one who has a hundred [wants] a thousand; one who has a thousand [wants] one hundred thousand, and one who has one hundred thousand [wants] ten million [rupees]. One who rules ten million [people] wishes to be a king of men, a king of men [wants to be a] world-emperor, a world-emperor a god, [and] a god wishes even to be a king of gods. [But] even when one has become the king of gods, [one's] desire does not end [with that]. In the beginning, such greed is rather small, [but in the course of time], it increases [and increases] like a [begging-]bowl, [which at the root is shallow, but which expands more and more from the centre to the edge].'' A \textit{cakravartin} is, therefore, a ruler who rules over the kings of the earth. The next higher rank would be the rank of a god. Thus, the \textit{cakravartin} is the highest possible status of a human being in terms of human possibilities to attain power and prosperity. Here, within Rājayoga, the term \textit{cakravartin} is used to underline the supreme position of \textit{sahajayoga} among all the other methods of Rājayoga, which are listed at the beginning of Rāmacandra's treatise. Several tantras promise the status of a \ldots}.\end{buber}
\flushpage 
    \end{tlate}
  \end{translation}
\end{alignment}
%%%%%%%%%%%%%%%%%%%%%%%%%%%%%%%%%%%%%%%%%%
%%%%%%%%%%%%%%%%%%%%%%%%%%%%%%%%%%%%%%%%%% 
%%%%%%%%PAGEBREAK%%%%%%%PAGEBREAK%%%%%%%%%
%%%%%%%%%%%%%%%%%%%%%%%%%%%%%%%%%%%%%%%%%% 
%%%%%%%%%%%%%%%%PAGEBREAK%%%%%%%%%%%%%%%%%
%%%%%%%%%%%%%%%%%%%%%%%%%%%%%%%%%%%%%%%%%% 
%%%%%%%%PAGEBREAK%%%%%%%PAGEBREAK%%%%%%%%%
%%%%%%%%%%%%%%%%%%%%%%%%%%%%%%%%%%%%%%%%%% 
%%%%%%%%%%%%%%%%%%%%%%%%%%%%%%%%%%%%%%%%%% 
%%%%%%%%%%%%%%%%%%%%%%%%%%%%%%%%%%%%%%%%%% 
%%%%%%%%%%%%%%%%%%%%%%%%%%%%%%%%%%%%%%%%%% 
%%%%%%%%PAGEBREAK%%%%%%%PAGEBREAK%%%%%%%%%
%%%%%%%%%%%%%%%%%%%%%%%%%%%%%%%%%%%%%%%%%% 
%%%%%%%%%%%%%%%%PAGEBREAK%%%%%%%%%%%%%%%%%
%%%%%%%%%%%%%%%%%%%%%%%%%%%%%%%%%%%%%%%%%% 
%%%%%%%%PAGEBREAK%%%%%%%PAGEBREAK%%%%%%%%%
%%%%%%%%%%%%%%%%%%%%%%%%%%%%%%%%%%%%%%%%%% 
%%%%%%%%%%%%%%%%%%%%%%%%%%%%%%%%%%%%%%%%%% 
%%%%%%%%%%%%%%%%%%%%%%%%%%%%%%%%%%%%%%%%%% 
%%%%%%%%%%%%%%%%%%%%%%%%%%%%%%%%%%%%%%%%%% 
%%%%%%%%PAGEBREAK%%%%%%%PAGEBREAK%%%%%%%%%
%%%%%%%%%%%%%%%%%%%%%%%%%%%%%%%%%%%%%%%%%% 
%%%%%%%%%%%%%%%%PAGEBREAK%%%%%%%%%%%%%%%%%
%%%%%%%%%%%%%%%%%%%%%%%%%%%%%%%%%%%%%%%%%% 
%%%%%%%%PAGEBREAK%%%%%%%PAGEBREAK%%%%%%%%%
%%%%%%%%%%%%%%%%%%%%%%%%%%%%%%%%%%%%%%%%%% 
%%%%%%%%%%%%%%%%%%%%%%%%%%%%%%%%%%%%%%%%%% 
\begin{alignment}[
  texts=edition[class="edition"];
  translation[class="translation"],
  ]
  \begin{edition}
    \begin{prose}[p59_06]
%-----------------------------
%iti śrīsarvaguṇasampannapaṃḍita-sukhānandamiśrasūrisūnupaṇḍita-jvālāprasādamiśrakṛtabhāṣāṭīkāsahito rājayoge binduyogaḥ samāptaḥ// śubhamastu//śrīrastu// \E
%iti śrīrāmacaṃdraparamahaṃsa viracitas tatvabinduyogasamāptaḥ saṃvat 1867 pauṣakṛṣṇaḥ 12 ravau śubham bhuyāt //??//\P
% missing last folio \B
% iti rājamacaṃdraparahaṃsa viracites tatvabiṃduyogasamāptaṃ// śrī kṛṣṇārpaṇamastu// cha// \L
% iti śrī paramarahasyāṃ śrīrāmacaṃdraviracitāyāṃ tatvayogabiṃdu samāptaḥ// //śrī svasti// //saṃvat 837   \N1  %%%% 1716 n. Chr.
% iti śrī paramarahasye  śrīrāmacaṃdraviracitāyāṃ tatvayogabindu samāptam// //śubham// yadakṣarapadabhraṣṭaṃ mātrāhīnaṃcayaḍ? bhavet// tat sarvaṃ kṣamya tā?m eva prasīdaparameśvara //1 // sūrye turaṅge navacandraghasre jyeṣṭhākhyakṛṣṇe bhṛguvārayuktam || tattvaprayogaḥ ṣaḍaharṣasaṇjñaṃ likhitaṃ suhetoḥ bhavatīha dehi || bhūyāt \N2
% iti paramarahasyāṃ śrīrāmacaṃdraviracitāyāṃ tatvayogabiṃdu samāptaḥ// śubhamastu/ saṃvat 1841// bhādau śudha 15tnīo vesarva śake rārāma rāma cha    \D
% iti śrī pāramahaṃsyāṃ śrī rāmacaṃdraviracitāyāṃ tatvayogaviduḥ samāptaḥ śubhaṃ bhūyāt // // atarlakṣyaṃ bahi dṛḍhir nirmeṣomeṣa varjitaḥ saiṣāśāṃbhavīmudrā sarvata,n treṣugopitā 1 aṃtark ..... %see last 2 folios verses beyond text quote from fourth chapter of HP \U1
% iti śrī rāmacaṃdraparamahaṃsaviracitas tatvabiṃduyogasamāptaḥ// śrī śubhaṃ bhavatu// śrīsītārāmārpaṇamastuḥ// idaṃ pustakaṃ// śake 1805// vikramārka saṃmat// 1140// jayanām asaṃvatsare// udagayaṇe// griṣmaṛtau?// vaiśālhemāse// kṛṣṇapakṣe// tithau 23// bhānuvāsare// prathamayāmye// śrī kṣetra avaṃtikāyāṃ// śrī mahārudramahākālasaṃnidhāne na saṃpūrṇaṃ// lekhanaṃ ānaṃt? suta bābājoo rājadherakareṇa likhyate// yādṛśaṃ pustakaṃ dṛṣtvā tādṛsaṃ likhitaṃ mayā// yadi śuddhaṃ aśuddho cā mama doṣo na dīyate//1// śrīrāma// cha//    \U2
%-----------------------------47
% \app{\lem[type=emendation, resp=egoscrconj, alt={iti śrīrāmacandraparamahaṃsaviracitas-yogatattvabinduḥ samāptaḥ}]{iti śrīrāmacandraparamahaṃsaviracitas-yogatattvabinduḥ samāptaḥ}
%\rdg[wit={E}]{iti śrīsarvaguṇasampannapaṃḍitasukhānandamiśrasūrisūnupaṇḍitajvālāprasādamiśrakṛtabhāṣāṭīkāsahito rājayoge binduyogaḥ samāptaḥ || śubhamastu || śrīrastu ||}
%\rdg[wit={P}]{iti śrīrāmacaṃdraparamahaṃsa viracitas tatvabinduyogasamāptaḥ saṃvat 1867 pauṣakṛṣṇaḥ 12 ravau śubham bhuyāt ||cha||}%check dots!!
%\rdg[wit={L}]{iti rājamacaṃdraparahaṃsa viracites tatvabiṃduyogasamāptaṃ || śrī kṛṣṇārpaṇam astu || cha ||}
%\rdg[wit={N1}]{iti śrī paramarahasyāṃ śrīrāmacaṃdraviracitāyāṃ tatvayogabiṃdu samāptaḥ|| ||śrī svasti|| ||saṃvat 837}
%\rdg[wit={N2}]{iti śrī paramarahasye  śrīrāmacaṃdraviracitāyāṃ tatvayogabindu samāptam|| ||śubham|| yadakṣarapadabhraṣṭaṃ mātrāhīnaṃcayaḍ? bhavet|| tat sarvaṃ kṣamya tā?m eva prasīdaparameśvara ||1|| sūrye turaṅge navacandraghasre jyeṣṭhākhyakṛṣṇe bhṛguvārayuktam || tattvaprayogaḥ ṣaḍaharṣasaṇjñaṃ likhitaṃ suhetoḥ bhavatīha dehi || bhūyāt}
%\rdg[wit={D}]{iti paramahaṃsyāṃ śrī rāmacaṃdraviracitāyāṃ tatvayogabiṃdu samāptaḥ || śubham astu | saṃvat 1841 || bhādau śudha 15 lī 0 ve sarva śake rā rāma rāma cha} %check
%\rdg[wit={U1}]{iti śrī pāramahaṃsyāṃ śrī rāmacaṃdraviracitāyāṃ tatvayogaviduḥ samāptaḥ śubhaṃ bhūyāt || || atarlakṣyaṃ bahi dṛḍhir nirmeṣomeṣa varjitaḥ saiṣāśāṃbhavīmudrā sarvata,n treṣugopitā 1 aṃtark ..... } %check
%\rdg[wit={U2}]{iti śrī rāmacaṃdraparamahaṃsaviracitas tatvabiṃduyogasamāptaḥ || śrī śubhaṃ bhavatu || śrīsītārāmārpaṇamastuḥ || idaṃ pustakaṃ || śake 1805 || vikramārka saṃmat || 1140 || jayanām asaṃvatsare || udagayaṇe || griṣmaṛtau? || vaiśālhemāse || kṛṣṇapakṣe || tithau 23 || bhānuvāsare || prathamayāmye || śrī kṣetra avaṃtikāyāṃ || śrī mahārudramahākālasaṃnidhāne na saṃpūrṇaṃ || lekhanaṃ ānaṃt? suta bābājoo rājadherakareṇa likhyate || yādṛśaṃ pustakaṃ dṛṣtvā tādṛsaṃ likhitaṃ mayā || yadi śuddhaṃ aeśuddho cā mama doṣo na dīyate ||1|| śrīrāma || cha ||}}\dd{}
      \linelabel{_abazaba1}
\noindent iti śrīrāmacandraparamahaṃsaviracitas-yogatattvabinduḥ samāptaḥ |\linelabel{_abazaba2}
\note[type=philcomm, labelb=_abazaba1, labele=_abazaba2, lem={iti śrīrāmacandraparamahaṃsaviracitas yogatattvabinduḥ samāptaḥ}]{The manuscripts and printed edition offer the following colophons with diverging variants of the title: \\

\textit{iti śrī paramarahasyāṃ śrīrāmacaṃdraviracitāyāṃ tatvayogabiṃdu samāptaḥ} || || \textit{śrī svasti} || || \textit{saṃvat 837} || \textit{vinā guru na siddhati} || \textit{eka vacana sosyā sālikaseṃ caudha bhuvana kā mola} || \textit{kahane soka haḍiyā avakyā vajāye ṃe ḍhola} || 1 || \textit{popoṣṭakaṃmā 10} | \textit{11} | \textit{12} | \textit{13} \textit{ja(m)mā 4 patra aghaḍiṣī ṭaṭāye .. ho} - \getsiglum{N1}\\

\textit{iti śrī paramarahasye śrīrāmacaṃdraviracitāyāṃ tatvayogabindu samāptam} || \textit{śubham} || \textit{yad akṣarapadabhraṣṭaṃ mātrāhīnaṃ cayad bhavet} || \textit{tat sarvaṃ kṣamya tām eva prasīdaparameśvara} ||1|| \textit{sūrye turaṅge navacandraghasre jyeṣṭhākhyakṛṣṇe bhṛguvārayuktam} || \textit{tattvaprayogaḥ ṣaḍaharṣasaṃjñaṃ likhitaṃ suhetoḥ bhavatīha dehi} || \textit{bhūyāt} || - \getsiglum{N2}\\

\textit{iti paramahaṃsyāṃ śrī rāmacaṃdraviracitāyāṃ tatvayogabiṃdu samāptaḥ} || \textit{śubham astu} | \textit{saṃvat 1841} || \textit{bhādau śudha 15 lī 0 ve sarva śake rā rāma rāma cha} - \getsiglum{D}\\

\textit{iti śrī pāramahaṃsyāṃ śrī rāmacaṃdraviracitāyāṃ tatvayogaviduḥ samāptaḥ} \textit{śubhaṃ bhūyāt} || \getsiglum{U1}\\

\textit{iti śrī rāmacaṃdraparamahaṃsaviracitas tatvabiṃduyogasamāptaḥ} || \textit{śrī śubhaṃ bhavatu} ||  \textit{śrīsītārāmārpaṇam astuḥ} ||  \textit{idaṃ pustakaṃ} ||  \textit{śake 1805} ||  \textit{vikramārka saṃmat} ||  \textit{1940} || \textit{jayanām asaṃvatsare} ||  \textit{udagayaṇe} || \textit{griṣmaṛtau}? || \textit{vaiśākhe māse} ||  \textit{kṛṣṇapakṣe} ||  \textit{tithau 23} ||  \textit{bhānuvāsare} ||  \textit{prathamayāmye} ||  \textit{śrīkṣetra avaṃtikāyāṃ} ||  \textit{śrī mahārudramahākālasaṃnidhāne na saṃpūrṇaṃ} ||  \textit{lekhanaṃ ānaṃt? suta? bābājī rājadherakareṇa likhyate} ||  \textit{yādṛśaṃ pustakaṃ dṛṣtvā tādṛsaṃ likhitaṃ mayā} ||  \textit{yadi śuddhaṃ aśuddho vā mama doṣo na dīyate} ||1|| \textit{śrīrāma} || cha ||- \getsiglum{U2}\\

\textit{iti śrīrāmacaṃdraparamahaṃsa viracitas tatvabinduyogasamāptaḥ} \textit{saṃvat 1867 pauṣakṛṣṇaḥ 12 ravau śubham bhuyāt} || \textit{cha} ||  - \getsiglum{P}\\

 \textit{iti rājayoge candraparamahaṃsaparipūrṇapīṭhamāhātmyaprakāśakaḥ binduyogaḥ samāptaḥ} || \textit{śubham astu} || \textit{iti śrīsarvaguṇasampannapaṃḍitasukhānandamiśrasūrisūnupaṇḍitajvālāprasādamiśrakṛtabhāṣāṭīkāsahito rājayoge binduyogaḥ samāptaḥ} || \textit{śubham astu} || \textit{śrīr astu} || - \getsiglum{E}\\

\textit{iti rājamacaṃdraparahaṃsa viracites tatvabiṃduyogasamāptaṃ} || \textit{śrī kṛṣṇārpaṇam astu} || \textit{cha} || - \getsiglum{L}\\}
    \end{prose}
  \end{edition}
  \begin{translation}
    \begin{tlate}[p59_06]
\begin{euber}[f59_2]\blfootnote{\hspace{-2.2em}\textit{cakravartin} through the performance of specific rituals. For example, \citetitle{guhyanisvasa} 10.110-112: \textit{saṃyatā kāpilena tu} | \textit{ghṛtasahasrahome na 'stu kṛṣṇake tu tṛrātrikam} || 10.110 || \textit{candrasūryoparāge vā ekadravyaṃ tu sādhayet} | \textit{trividhā siddhi jāyate cakravartipadottamam} || 10.111 || \textit{dhūme vidyādharottamā bhavateti na saṃśayaḥ} | \textit{sarvasiddhottamā jvālā sarvalokāṃ parikramet} || 10:112 || ``Controlled by Kāpila, by means of the ritual of a thousand ghee oblations, one should praise Kṛṣṇa for three nights. A single substance should be accomplished during the eclipse of the sun and the moon. In this way, threefold accomplishments arise, attaining the supreme status of the wheel-turning monarch. In the smoke, the best of Vidyādharas should be invoked. Without a doubt, the best of all accomplishments, the flame, surpasses all worlds.'' Additionally, cf. \citetitle{merutantra} 151-152: \textit{evaṃ svarṇādike patre likhed vā yantram uttamam} | \textit{tatra vācaṃ samāvāhya pūjayitvā vidhānataḥ} || 151 || \textit{ṛṣyādikaṃ tu saṃyojya dhārayec ca śubhe dine} | \textit{kavīnāṃ tārkikāṇāṃ ca cakravartī bhaved asau} || 152 || ``In this way, one shall write the supreme yantra on the best material like gold, etc. After invoking [the deity] with appropriate words, worship according to the prescribed [method]. On an auspicious day, having joined [oneself] with a sage, etc., he shall hold it. One becomes a universal ruler among poets and logicians.'' The association of the supreme yogic state with the term \textit{cakravartin} is already found in the commentary to \citetitle{pratyabhijnahrdaya} \textit{sūtra} 20 by Kṣemarāja. His commentary on \textit{sūtra} 20 (\textit{tadā prakāśānandasāramahāmantravīryātmakapūrṇāhantāveśāt sadā sarvasargasaṃhārakārinijasaṃviddevatācakreśvaratāprāptir bhavatīti śivam |}), which concludes the \emph{Pratyabhijñāhṛdaya} and glosses the state of the person who has entered the self as someone with lordship (\textit{īśvaratā}) over the own group of consciousness deities or consciousness forces, which cause every emanation or reabsorption of the universe. In this context Kṣemarāja writes: \textit{stutaṃ ca mayā svatantraś citicakrāṇāṃ cakravartī maheśvaraḥ} | \textit{saṃvittidevatācakrajuṣṭaḥ ko 'pi jayaty asau} || \textit{iti} | ``This was praised by me in the following verse: 'Whoever becomes the universal ruler over the group of \textit{śakti}s, i.e. who is no longer under the control of the senses and has become the great lord, to whom the group of sense-deities is a rare being who stands above all.' ''}\end{euber}
\noindent Thus concludes the \textit{Yogatattvabindu},\footnote{I consider it necessary to emend the title of the work to \textit{Yogatattvabindu} since there seems to be great confusion among the manuscripts. The printed edition calls the text \textit{Binduyoga}, the \alpha-group \textit{Tattvayogabindu} and the \beta-group \textit{Tattvabinduyoga}, the library card of manuscript \getsiglum{N2}, however, calls the text \textit{Rājayogatattvabindu}. All titles except the ones in the library card appear doubtful. This confusion of the compound members of the title is likely the result of metathesis by an early scribe in the transmission of the text since there is no other indication or usage of the compound \textit{tattvayoga} or \textit{tattvabindu} in the text itself. Additionally, those two options for the title are unlikely since a convincing meaning would be difficult to construe. However, the term \textit{yogatattva} can be attested in the text and is used to highlight the penultimate teaching of the text, which is \textit{gurusevā}. Other texts like, i.e. \citetitle{stb1} use \textit{°tattvabindu} as the final members of the title compound. Thus, I propose that the original title must have been \textit{Yogatattvabindu}. I present a more detailed discussion of the title on p. \pageref{titlediscussion}.} composed by Srī Rāmacandra Paramahaṃsa.
\flushpage 
    \end{tlate}
  \end{translation}
\end{alignment}
\cleardoublepage
\selectlanguage{english}
\chapter{Appendix}
\section{Figures}

% \begin{landscape}
\clearpage
  \begin{figure}[ht]
	\centering
  \includegraphics[width=1\textwidth]{pics/Vishnu_Vishvarupa_cropped.jpg}
	\caption{Viṣṇu Viśvarūpa, India, Rajasthan, Jaipur, ca. 1800–1820, Opaque watercolor and gold on paper, 38.5 × 28 cm, Victoria and Albert Museum, London, Given by Mrs. Gerald Clark.}
	\label{fig1}
      \end{figure}
\clearpage
  \begin{figure}[ht]
	\centering
  \includegraphics[width=0.5\textwidth]{pics/The_Equivalence_of_Self_and_Universe_(detail),_folio_6_from_the_Siddha_Siddhanta_Paddhati,_(Bulaki),_1824_(Samvat_1881);_122_x_46_cm._Mehrangarh_Museum_Trust..jpg}
	\caption{The Equivalence of Self and Universe (detail), folio 6 from the \textit{Siddhasiddhāntapaddhati} (Bulaki), India, Rajasthan, Jodhpur, 1824 (Samvat 1881), 122 x 46 cm, RJS 2378, Mehragarh Museum Trust.}
	\label{fig2}
      \end{figure}
      % \end{landscape}


\chapter{Bibliography}
 \label{sec:bibli}
   \clearpage
\newpage 
\thispagestyle{empty}
\quad  \addtocounter{page}{-1}

\printbibliography[heading=subbibintoc, title=Consulted Manuscripts, keyword=codex]

\printbibliography[heading=subbibintoc, title=Printed Editions, keyword=printsource]

\printbibliography[heading=subbibintoc, title=Secondary Literature, keyword=seclit]

\printbibliography[heading=subbibintoc, title=Online Sources, keyword=onlinesource]

\end{document}
