%Ultimatives Tool zur Datierung:
%https://www.cc.kyoto-su.ac.jp/~yanom/pancanga/
%skp = ignored in edition
%skm = ignored in xml
%%%---2-DO---%%%:
% - add xml ids for cladistics
% - produce diplomatic transcripts for saktumiva
% - make extra layer in Apparatus for parallels in SVARODAYA, Siddhasiddhantapaddhati and Amanaska
% - check all daṇḍas!!! now I think that it's more likely that many of them were lost in copies. Lectio difficilior! Very unconventional style of the autor! 
% - read Sarvangayogapradipika, Maya Burger! 
% - maybe add second ciritical edition of yogasvarodaya?!
% - Korrekturlesen von \E!! 
% - Verspattern einbauen!
% - add all Testtimonia of SSP & Ysv
% - Sigla alphabetisch ordnen und! daṇḍas mit einkollationieren
% - präambel auslagern wie Jürgen 
% - grep-search alle Verse!!!!
% - Mss spreadsheet
% - sort N1,D1,B2 zu N1,N2,D1
% - sort all sigla alphabetically 
% - additions to U2: make footnotes for the bahir mātrā-s: explaining the inventions of female deities and tell that this is "schwer interpretierbar"
% - Belege für source und testimonia einfügen!!!
% - GIVE UNIQUE LABELS for TESTIMONIO AND SOurces
% - Edition mit Sätzen übereinander nennt sich: Synoptische Edition
% - Consider changing Lakṣya to Lakṣa
% - vEREINHEITLICHUNG von source und testium! 
%%%%%%%%%%%%%%%%%%%%%%%%%%%%%%%%%%%%%%%%%
% Don't forget
% Siddhasiddhantapaddhati Yogic Body descriptions are followed by Rāmacandra
% Quotes of the Yogasvarodaya in the Yoga Karṇikā
% Rāmacandra more a compiler than an author!!!
% Identify quotes of YTB in Haṭhasanketacandrikā 
%%%%%%%%%%%%%%%%%%%%%%%%%%%%%%%%%%%%%%%%%%%
%MSS notes
%
%--B: i and ī are not differenciated
%--P: no punctuation no daṇdas nothing
%--U1: dot . serves as daṇḍa 
%--\L and \U2 very similar
%--figure out for U2: // ajapājapaḥ sahasra // 6000 //gha 0 16 pa 0 40// \U2?!?!?!?!?!?
%%%%%%%%%%%%%%%%%%%%%%%%%%%%%%%%%%%%%%%%%%
%
% Einleitung Ideen 
% - sprachliche Simplizität
% - Potenzial als Anfängertext
% - Großartige Einführung in die Textkritik -> Synoptische Edition 
% - Gelegenheit Yogasvarodaya und Yogatattvabindu zu edieren 
% - Historische Evidenz entweder für das königliche Leben in einer Maṭha in der Nähe von Benares während der Muslimischen Herrschaft, oder sogar Lehrtext für die Bildung junger Prinzen  
% - eines der raren Beispiele der engen Verknüfung mehrerer Texte 
% - eines der raren Beispiele der Prosaisierung eines metrischen Textes 
% - Anwendung rezenter Technologie! 
% - How the text was construed -> intermingling of Ysv and SSP
% - Martin Straube: "jeder kleine Dorfhäuptling kann Rāja genannt werden". 
%%%%%%%%%%%%%%%%%%%%%%%%%%%%%%%%%%%%%%%%%%%
%Ich habe dieses Zitat gefunden
%Franz
%Franz Veit
%हठयोगः [Printed book page 5-501-c]
%हठयोगः , पुं, (हठेन योगः ।) योगविशेषः ।
%यथा, —
%“इदानीं हठयोगस्तु कथ्यते हठसिद्धिदः ।
%कृत्वासनं पवनाशं शरीरे रोगहारकम् ॥
%पूरकं कुम्भकञ्चैव रेचकं वायुना भजेत् ।
%इत्थं क्रमोत्क्रमं ज्ञात्वा पवनं सग्धयेत् सदा ॥
%धौत्यादिकर्म्मषट्कञ्च संस्कुर्य्याद्धठसाधकः ।
%एतन्नाड्यान्तु देवेशि ! वायुपूर्णं प्रतिष्ठितम् ॥
%ततो मनो निश्चलं स्यात्तत आनन्द एव हि ।
%हठयोगान्न कालः स्यान्मनः शून्ये भवेद्यदि ॥
%इदानीं हठयोगस्य द्वितीयं भेदवत् शृणु ।
%आकाशे नासिकाग्रे तु सूर्य्यकोटिसमं स्मरेत् ॥
%श्वेतं रक्तं तथा पीतं कृष्णमित्यादिरूपतः ।
%एवं ध्यात्वा चिरायुः स्यादङ्गाजननवर्ज्जितः ॥
%शिवतुल्यो महात्मासौ हठयोगप्रसादतः ।
%हठाज्ज्योतिर्म्मयो भूत्वा ह्यन्तरेण शिव भवेत् ।
%अतोऽयं हठयोगः स्यात् सिद्धिदः सिद्धसेवितः ॥”
%इति योगस्वरोदयः ॥ [ID=41348]

%Now, Haṭhayoga indeed is explained as that which gives the siddhi (accomplishment) of haṭha (persistence).
%One performs the wind-eating/serpent āsanam, which removes illness in the body
%and filling – kumbhaka – emptying may distribute the vāyu/wind.
%In this way, while being aware of the progress and regress of the breath one may feed on the wind continually.
%And with the six karmmas, dhauti etc., the Sādhaka of Haṭha may prepare/embellish himself.
%Thus/thereby, in the channel (nāḍī), Oh supreme Goddess, all of the winds (vāyu) are consecrated/placed.
%Then the mind may be unmoved and then bliss it really is.
%Through Haṭhayoga time will be no more, when the mind in emptiness abides.
%
%Now listen to the second disclosure of haṭhayoga:
%In space, on the tip of the nose indeed, one may remember equal to ten million suns,
%the primal forms: white, red, likewise yellow, dark blue.
%Thus meditating/visualizing, one may have a long life, free of the birth of the body,
%Equal to Śiva, this great soul, due to the blessing of Haṭhayoga,
%shall become through persistence (haṭha) a being of light and internally śiva.
%Therefore this Haṭhayoga grants accomplishment – it’s practiced by the Siddhas (accomplished ones).
%Franz
%Im śabdakalpadruma
%Franz
%Franz Veit
%fj.veit@gmail.com
\input{preamble.tex}
\author{Nils Jacob Liersch}
\title{Yogatattvabindu of Rāmacandra\\ A Critical and Synoptic Edition and Annotated Translation}
\date{\today}

\parindent=15pt
\begin{document}

% Zitiermöglichkeiten:
%\footcite[See][p.\,1]{goldstein01:_tibet_englis_diction_moder_tibet}
%\footnote{\cite{goldstein01:_tibet_englis_diction_moder_tibet}.}

\frontmatter
\thispagestyle{empty}
\begin{center}
  {\Large \emph{The Yogatattvabindu}}\\[3mm]
\end{center}



\newpage

\

\thispagestyle{empty}



\normalsize


\newpage


\begin{center}
\thispagestyle{empty}

\

\vskip 2mm

\begin{otherlanguage}{iast}
\LARGE \sanskritfont{Yogatattvabindu}
\end{otherlanguage}

\vskip .4cm

\Huge Yogatattvabindu \\[7mm]
\Large Critical and Synoptic \\
Edition with annotated Translation


\large

\vspace{3cm}

Von

Nils Jacob Liersch
\small
\vfill

\vfill

Indica et Tibetica Verlag \\ % $\cdot$ 
Marburg 2024

\vskip 6mm

\end{center}

\newpage
\newpage \ \thispagestyle{empty}
\small  \

\noindent

\
\vfill


\small
\noindent \textbf{Bibliographische Information Der Deutschen Bibliothek}

\noindent
Die Deutsche Bibliothek verzeichnet diese Publikation in der Deutschen Nationalbibliographie;
detaillierte bibliographische Informationen sind im Internet über http://dnb.ddb.de abrufbar.

\noindent
\textbf{Bibliographic information published by Die Deutschen Bibliothek}

\noindent
Die Deutsche Bibliothek lists this publication in the Deutsche Nationalbibliographie; detailed
bibliographic data is available in the Internet at http://dnb.ddb.de.  


\vskip 1cm

\noindent
\copyright\ Indica et Tibetica Verlag, Marburg 2024

\medskip

\noindent
Alle Rechte vorbehalten / All rights reserved

\medskip

\noindent
Ohne ausdrückliche Genehmigung des Verlages ist es nicht gestattet, das Werk oder einzelne Teile
daraus nachzudrucken, zu vervielfältigen oder auf Datenträger zu speichern.

\smallskip

\noindent
Apart from any fair dealing for the purpose of private study, research, criticism or review, no
part of this book may be reproduced or translated in any form, by print, photo form, microfilm, or
any other means without written permission. Enquiries should be made to the publishers.

\bigskip

\noindent
Satz: \ \ Nils Jacob Liersch \\
Herstellung: \ \ BoD – Books on Demand GmbH, Norderstedt  \\

\bigskip

\noindent
%\ISBN     

\normalsize

\newpage

%\maketitle
\clearpage
\tableofcontents
\addtocounter{page}{-1}
\thispagestyle{empty}
\clearpage

\chapter{Introduction}
\mainmatter

\chapter{The List of the 15 Yogas}
\label{yogas_list}
The authenticity of the list specifying the fifteen Yogas at the beginning of the text is ambiguous. This is due to the discrepancy between the structure of the Yogas presented in the text and the order presented in the list. For example, the text commences with a description of \textit{kriyāyoga} and goes on to describe \textit{siddhakuṇḍaliniyoga} and then mentions \textit{mantrayoga} without adhering to the order presented in the list. This incongruity raises questions as to why the text structure deviates from the list. However, the reference to \textit{jñānotpattav upāyaḥ} may provide some insight into why \textit{jñānayoga} is included as the second \textit{yoga} in the list. To reconcile these apparent inconsistencies, there are several possible explanations: 1) The text is severely corrupted. 2) The list was added by a different hand at a later time. 3) The term \textit{jñānayoga} is included as a result of the practice of \textit{siddhakuṇḍalinīyoga}, which is said to generate knowledge through the central channel, as stated in the text. These explanations may be combined to provide a comprehensive understanding of the situation.

\section{Lakṣyayoga}

\begin{itemize}
\item origin tantric Traditions -> e.g. Netratantra
\item also check Mālinivijayottara 2004 Vasudeva pp. 256-257
\item also \citetitle{birch2013} 2.10 Śāmbhavī Mudrā
  \end{itemize} 

\chapter{Sources}
\section{The Additions of  SORI 6082 - U\textsubscript{2}}
\label{discussionu2}
Analyse the additions of U\textsubscript{2} and present the \textit{cakra}s and their attriubutes in a table .
\begin{itemize}
\item  Muktabodha-Texte sehe ich 3 Belege für bahiśśakti Muktabodha/krīyakramādyotikā.html 2938 suṣirānte bahiśśaktiṃ vinyasedvyomarūpiṇīm | tasyā madhye tu Muktabodha/sakalāgamasārasaṅgraha.html 2186 suṣirāntabahiśśaktiṃ vyāpinīṃ cintayet tataḥ || Muktabodha/kriyakramadyotikavyākhyā.html 1846 tanmadhye ca bahiśśaktiṃ sudhābindu parisrutim
  \item  Parā\footnote{Im Kaśm. Śiv. °das ewige Wort, in welchem potentiell alle Begriffe und Worte ruhen; vgl. das śabdabrahma des Vyākaraṇa. [B.]― Schmidt S. 246}.
  \end{itemize}

\chapter{Conventions in the Critical Apparatus}
\section{Sigla in the Critical Apparatus}

\begin{itemize}
\item E : Printed Edition
\item P : Pune BORI 664
\item L : Lalchand Research Library LRL5876
\item B : Bodleian Oxford D 4587
\item \None : NGMPP B 38-31
\item \Ntwo : NGMPP B 38-35 / A 1327-14
\item \Done : IGNCA 30019
\item \Uone : SORI 1574
\item \Utwo: SORI 6082
\end{itemize}

The order of the readings in the critical apparatus is arranged according to the quality of readings in decending order. The critical apparatus is positive. Gemitation is not recorded. 

\section{Abbreviations}
\begin{itemize}
  \item qcr: quote cum notatio (quoted with reference)
  \end{itemize}

\section{Marking the Reliability of Sources and Testimonia in the Critical Apparatus}
\label{kennz}

To accurately depict information about the textual relationship and estimated degree of relatedness of a passage from the \textit{Yogatattvabindu} in the layers for sources and testimonia of the critical apparatus, a system of sigla was introduced.\footnote{This type of identification system is based on the use of the critical apparatus in \parencite[lii-liii]{steinkellner2005}. It was modified for the text-critical work on the \textit{Yogatattvabindu}.} The sigla are meaningful when a passage is corrupted in all witnesses and can only be reconstructed by means of other texts. The layers of the critical apparatus for sources and testimonia use the following sigla:

\begin{enumerate}
\item[\textbf{Ce}] \textit{citatum ex alio} / quotation from another (text).\footnote{The sigla \textbf{Ce} indicates an identical or largely identical content in the lesser witness and only allows for minor deviations in the wording of the passage.}
\item[\textbf{Cee}] \textit{citatum ex alio modo edendi} / quotation from another (text) with editorial changes.\footnote{The sigla \textbf{Cee} identifies passages with noticeable deviations in the lesser witness.}
\item[\textbf{Ci}] \textit{citatum in alio} / quotation in another (text).\footnote{The sigla \textbf{Ci} indicates an identical or largely identical content in the lesser witness and only allows for minor deviations in the wording of the passage.}
\item[\textbf{Cie}] \textit{citatum in alio modo edendi} / quotation in another (text) with editorial changes.\footnote{The sigla \textbf{Cie} identifies passages in the lesser witness with noticeable deviations that have the intended character of the composer.}
\item[\textbf{Re}] \textit{relatum ex alio} / (content), attested from another text.\footnote{The sigla \textbf{Re} identifies content parallels in the lesser witness that are relevant to the constitution of the critical text. It further indicates in certain cases that the composer might have used this source when composing his text.}
\item[\textbf{Ri}] \textit{relatum in alio} / (content), attested in another text.\footnote{The sigla \textbf{Ri} identifies content parallels in the lesser witness that are relevant to the constitution of the critical text.}
\end{enumerate}

The following acronyms refer to passages that originated from texts that the author of the \textit{Yogatattvabindu} utilized in compiling his work: \textbf{Ce}, \textbf{Cee}, \textbf{Re}. These texts must predate the \textit{Yogatattvabindu}. The other acronyms, such as \textbf{Ci}, \textbf{Cie}, and \textbf{Ri}, are texts that have adopted passages from the \textit{Yogatattvabindu}, or verses or passages that share similar content with the \textit{Yogatattvabindu}, but their relation is given literally, making it impossible to determine who adopted from whom. \textbf{Re} and \textbf{Ri} each refer to passages that are so closely related in content to those of the \textit{Yogatattvabindu} that they are significant in reconstructing a passage.\footnote{\textbf{Ce} and \textbf{Cee} have the highest degree of reliability, \textbf{Ci} and \textbf{Cie} have a moderate degree, and \textbf{Re} and \textbf{Ri} have the lowest.}

\section{Punctuation}

The inconsistent use of punctuation marks in the available witnesses necessitates standardization. Upon close examination, it appears that punctuation has frequently been dropped or added during the transmission of the texts. The neglect or improper handling of punctuation by the copists has resulted in different versions of lists with and without punctuation. In many instances, missing punctuation has led to the addition of case endings, alteration of the text, and the combination of list items into compound formations that were not present in the original text. Although punctuation plays an important role, deviations in punctuation at the end of sentences, lists, and verse-numbering will only be extensively documented in the critical apparatus of the printed edition. This means that emendations of obvious punctuation mistakes will not be recorded in the critical apparatus. However, the digital edition of this work provides a more detailed documentation of deviations in punctuation through diplomatic transcripts of each witness, and even has a function to display sentences cumulatively.

In the printed edition of the \textit{Yogatattvabindu}, standard conventions of punctuation are followed. In verse poetry, a \textit{daṇḍa} (|) marks the end of a half-verse or half of the \textit{śloka}, and a double \textit{daṇḍa} (||) marks the end of a verse. In prose, a single \textit{daṇḍa} indicates the end of a sentence, and a double \textit{daṇḍa} marks the end of a paragraph. Variations in the use of \textit{avagraha} will be recorded, and items in lists will be separated by a double-\textit{daṇḍa}.

\section{Sandhi}

Among the witnesses we see deviating and inconsistent application of \textit{sandhi}. There is no clear evidence that originally \textit{sandhi} was intentionally not applied. This edition will therefore apply \textit{sandhi} consistently throughout the constituted text to provide a readable text sticking to contemporary conventions in Sanskrit. The variant readings concerning \textit{sandhi} are recorded consistently in the apparatus criticus. This is due to various textcritical problems arising from the inconsistent usage of punctuation which results in application or non-application of \textit{sandhi} wheter the respective witness applied a \textit{daṇḍa} or not. This is particularly the case within lists, which frequently occur in our compilation. Items were most likely originally separated by \textit{daṇḍa}. 


\section{Class Nasals}

Due to inconsistent use of class nasals among the witnesses \textit{anusvāra}s have been substituted with the respective class nasals throughout the edition.

\section{Lists}

Lists are a frequent feature in the \textit{Yogatattvabindu}. The text opens with a list of 15 Yogas and there are many more lists utilized throughout its content. To produce a consistent and easily readable edition, all lists have been identified, normalized to the Nominative Singular or Nominative Plural form of the respective item, or in the case of explanatory lists, to the Ablative Singular or Plural. The items are separated by a double \textit{daṇḍa}. Differences in punctuation and simple punctuation emendations, unless they are text-critically or systematically significant, will not be recorded in the apparatus criticus.
\clearpage

\section{Structural Analysis of the Yogatattvabindu}

\chapter{Critical Edition \& Annotated Translation}
\cleardoublepage
\begin{alignment}[
  texts=edition[class="edition"];
  translation[class="translation"],
  ]
  \begin{edition}
    \ekddiv{type=ed}
    \begin{prose}
      \noindent
\note[type=source, labelb=178a, lem={\textbf{Re}}]{PT\textsuperscript{qcr \cdot YSV} (Ed. pp. 838-839): netramadhye kūrmanāmā nimeṣonmeṣakṛdayam | udgāre nāga ākhyātaḥ ūrddhavāyuḥ pracālane | kṛkaraḥ kṣutkaro jñeyo devadatto vijṛmbhaṇe | dhanañcayaḥ saccidākāro mṛtadehaṃ na muñcati | yady api sargakāṇḍe sarvametaduktaṃ tathāpi kāryakāraṇabhāvajñāpanāya punarnirdiṣṭamiti na punaruktam |}
\note[type=source, labelb=179a, lem={\textbf{Ri}}]{SSP 1.67 (Ed. pp. 23-24): kūrmavāyuḥ cakṣuṣor unmeṣakārakaś ca| kṛkalaḥ udgārakaḥ kṣutkārakaś ca | devadatto mukhavijṛmbhakaḥ | dhanañjayo nādaghoṣakah ||1.67|| iti daśavāyvavalokanena piṇḍotpattiḥ naranārīrūpam |}
%-----------------------------
%kūrmavāyur netramadhye tiṣṭhati/ nimeṣonmeṣaṃ karoti/ \E
%kūrmavāyur netramadhye           nimeṣonmeṣaṃ karoti \P
%kūrmavāyoḥ netramadhye           nimeṣonmeṣaṃ karotī/ \B
%kūrmavāyoḥ netramadhye           nimiṣonmeṣaṃ karotī... \L
%kūrmo vāyunetramadhye tiṣṭhati/  unmeṣaṃ nimeṣaṃ karoti/ \N1
%kūrmo vāyunetramadhye tiṣṭhati/  unmeṣaṃ nimeṣaṃ ca karoti// \D
%kūrmo vāyunetramadhye tiṣṭhati/  unmeṣaṃ nimeṣaṃ karoti/ \N2
%\om                                                     \U1
%kūrmavāyur netramadhye           nimiṣonmeṣaṃ karoti//            \U2
%-----------------------------
%The Kūrma vitalwind exists within the eyes. It causes [the] opening and closing [of the eyes]. 
%-----------------------------
\app{\lem[wit={E,P,U2}, alt={kūrmavāyur}]{kūrmavāyu\skp{r-ne}}
  \rdg[wit={B,L}]{kūrmavāyoḥ}
  \rdg[wit={D,N1,N2}]{kūrmo vāyu}
}\skm{r-ne}tramadhye
\app{\lem[wit={D,E,N1,N2}]{tiṣṭhati}
  \rdg[wit={ceteri}]{\om}}/  
\app{\lem[wit={E,P,B,U2}]{nimeṣonmeṣaṃ}
  \rdg[wit={N1,N2}]{unmeṣaṃ nimeṣaṃ}
  \rdg[wit={D}]{unmeṣaṃ nimeṣaṃ ca}}
\app{\lem[wit={ceteri}]{karoti}
  \rdg[wit={B,L}]{karotī}}/
\note[type=philcomm, labelb=179b, lem={\uproman{27}.\textsuperscript{\lowroman{17}-\lowroman{18}}}]{Sentences \om in \getsiglum{U1}.}
%-----------------------------
%kṛkalakartāvāyur  udgāraṃ karoti      \E
%kṛkalavāyur       udhāraṃ karoti      \P
%kṛkalavāyur       udhāraṃ karotī      \B
%kṛkalavāyur       uhāraṃ karotī        \L
%kṛkalavāyor       ūdgāro bhavati//    \N1
%kṛkalavāyor-------ūdgāto bhavati/      \D
%kṛkaravāyor-------ūdgāro bhavati/      \N2
%                                       \U1
%puṣkaravāyur      udgāraṃ karoti//    \U2
%-----------------------------
%From the Kṛkala vitalwind gagging arises. 
%-----------------------------
\app{\lem[wit={D,N1,N2},alt={kṛkalavāyor}]{kṛkalavāyo\skp{r-u}}
  \rdg[wit={B,L,P}]{kṛkalavāyur}
  \rdg[wit={E}]{kṛkalakartāvāyur}
  \rdg[wit={U2}]{puṣkaravāyur}
}\app{\lem[type=emendation, resp=egoscr, alt={udgāro}]{\skm{r-u}dgāro}
  \rdg[wit={E,U2}]{udgāraṃ}
  \rdg[wit={B,P}]{udhāraṃ}
  \rdg[wit={L}]{uhāraṃ}
  \rdg[wit={N1,N2}]{ūdgāro}
  \rdg[wit={D}]{ūdgāto}}
\app{\lem[wit={D,N1,N2}]{bhavati}
  \rdg[wit={E,P,U2}]{karoti}
  \rdg[wit={B,L}]{karotī}}/
%-----------------------------
% devadattavāyoḥ  jṛmbhaṇaṃ bhavati/ dhanaṃjayavāyoḥ śabda utpadyate// \E
% devadattavāyor  jumbhā bhavati     dhanaṃjayavāyo  śabdāḥ utpadyete  \P
% devadattavāyor  jumbhā bhavaṃtī    dhanaṃjayavāyoḥ śabda utpadyate// \B
% devadattavāyor  jṛṃbhā bhavatī     dhanaṃjayavāyoḥ śabdaḥ utpadyate// \L
% devadattavāyor  jṛṃbha utpadyate// dhanaṃjayavāyo  śabda utpadyate// \N1
% devadattavāyor  jṛṃbha utpadyate// dhanaṃjayavāyo  śabda utpadyate// \D
% devadattavāyo   jṛṃbhotpadyate/    dhanaṃjayavāyo  śabdotpadyate// \N2
% devadattavāyor  jaṃbhā utpadyate   dhanaṃjayavāyoḥ sabta utpadyate \U1
% devadattavāyo   jṛṃbhā bhavati//   dhanaṃjayavāyoḥ śabda utpadyate// \U2
%-----------------------------
%From the Devadatta vitalwind jawning arises. From the Dhanaṃjaya vitalwind speech arises. 
%-----------------------------
\app{\lem[wit={ceteri}, alt={devadattavāyor}]{devadattavāyo\skp{r-jṛ}}
  \rdg[wit={E}]{devadattavāyoḥ}
  \rdg[wit={N2,U2}]{devadattavāyo}
}\app{\lem[wit={D,N1,U2},alt={jṛmbha}]{\skm{r-jṛ}mbha}
  \rdg[wit={E}]{jṛmbhaṇaṃ}
  \rdg[wit={B,P}]{jumbhā}
  \rdg[wit={L}]{jṛṃbhā}
  \rdg[wit={N2}]{jṛṃbho°}
  \rdg[wit={U1}]{jaṃbhā}}
\app{\lem[wit={X}]{utpadyate}
  \rdg[wit={E,P,U2}]{bhavati}
  \rdg[wit={B}]{bhavaṃtī}
  \rdg[wit={L}]{bhavatī}}/
\app{\lem[wit={Y}]{dhanaṃjayavāyoḥ}
  \rdg[wit={X}]{dhanaṃjayavāyo}}
\app{\lem[wit={ceteri}]{śabda}
  \rdg[wit={P}]{śabdāḥ}
  \rdg[wit={L}]{śabdaḥ}
  \rdg[wit={N2}]{śabdo°}
  \rdg[wit={U1}]{sabta}}
utpadyate\dd{}\textsuperscript{\begin{otherlanguage}{english}\coro{[\lowroman{20}]}\end{otherlanguage}}\vfill
\end{prose}
\nolinenumbers
\centerline{\textrm{\small{[\uproman{27}.\textsuperscript{\coro{\lowroman{1}-\lowroman{5}}}Madhyalakṣya]}}}
\label{madhyalaksya}
    \bigskip
    \linenumbers
    \begin{prose}
      \noindent
%----------------------------
%\om                               \E
%idānī  madhyalakṣaṃ   kathyate      \P
%idānīṃ madhyalakṣaṇaṃ kathyate//  \B DSCN7164 Z.1
%idānīṃ madhye lakṣaṃ  kathyate//   \L
%idānīṃ madhyalakṣyaṃ  kathyate//   \N1
%idānīṃ madhyalakṣyaṃ  kathyate//   \D
%idānīṃ madhyalakṣaṇaṃ kathyate//  \N2
%idānīṃ madhyalakṣyaṃ  kathyate     \U1
%idānīṃ madhye lakṣyaṃ kathyate//  \U2
%-----------------------------
%Now the central fixation is taught. 
%-----------------------------
\note[type=source, labelb=180, lem={\textbf{Re}}]{PT\textsuperscript{qcr \cdot YSV} (Ed. p. 839): idānīṃ madhyalakṣan tu kathyate siddhikārakam | śvetaṃ raktaṃ tathā pītaṃ dhūmrākāran tu nīlabham |}
\app{\lem[wit={ceteri}]{idānīṃ}
  \rdg[wit={P}]{idānī}}
\app{\lem[wit={D,N1,U1}]{madhyalakṣyaṃ}
  \rdg[wit={B,N2}]{madhyalakṣaṇaṃ}
  \rdg[wit={P}]{madhyalakṣaṃ}
  \rdg[wit={L}]{madhye lakṣaṃ}
  \rdg[wit={U2}]{madhye lakṣyaṃ}}
kathyate/\note[type=philcomm, labelb=180a, lem={\uproman{27}\textsuperscript{\lowroman{1}}}]{Introductory sentence is missing in \getsiglum{E}.}
%-----------------------------SSP. S41!!! almost identical! 
%              aṃtha ca pītavarṇaṃ   raktavarṇaṃ vā dhūmrākāraṃ yan  nīlavarṇaṃ vā   agniśikhāsadṛśaṃ vidyutsamānaṃ   sūryamaṇḍalasadṛśaṃ     arddhacandrasadṛśaṃ jvalad  ākāśasamākāraṃ  \E
%śvetavarṇaṃ   atha     pītavarṇaṃ   raktaṃ vā      dhūmrākāraṃ yan  nīlavarṇaṃ vā   'gniśikhāsadṛśaṃ vidyutsamānaṃ   sūryamaṇdalasadṛśaṃ     arddhacaṃdrasadṛśaṃ jvalad  ākāśasamākāraṃ  \P
%śvetavaraṃ    atha     pītavarṇaṃ// rakta  vā      dhūmrākāraṃ yan  nīlavarṇaṃ vā// agniśikhāsadṛśaṃ vidyutsamānaṃ   sūryamaṇdalasadṛśaṃ/    ūrdhvacaṃdrasadṛśaṃ jvalad  ākāśasamākāraṃ// \B
%śvetavarṇaṃ   atha     pītavarṇaṃ   raktaṃ vā      dhūmrākāraṃ yan  nīlavarṇaṃ vā// agniśikhāsadṛśaṃ vidyutsamāne    sūryamaṇdalasadṛśaṃ//   ardhacaṃdrasadṛśaṃ  jvalad  ākāśasamākāra   \L
%śvetavarṇā/   atha vā  pītavarṇaṃ   raktaṃ vā      dhūmāra     va   nīlavarṇaṃ vā   agniśikhāsadṛśaṃ vidyutsamānaṃ   sūryamaṇdalaṃ sadṛśaṃ/  ūrdhvacaṃdrasadṛśaṃ jvalad  ākāśasamānakāraṃ//  \N1
%śvetavarṇaṃ// atha vā  pītavarṇaṃ   raktaṃ vā      dhūmākāro   vā   nīlavarṇaṃ vā   agniśikhāsadṛśaṃ vidyutsamānaṃ// sūryamaṇdalaṃ sadṛśaṃ// ūrdhvacaṃdrasadṛśaṃ jvalad  ākāśasamānakāraṃ//  \D
%śvetavarṇā    atha vā  pītavarṇa    raktavarṇa     dhūmravarṇa      nīlavarṇaṃ vā   agniśikhāsadṛśaṃ vidyutsamānaṃ   sūryamaṇdalasadṛśaṃ     ūrdhvacaṃdrasadṛśaṃ jvalad  ākāśasamānakāraṃ//  \N2
%svetavarṇaṃ   atha vā  pītavarṇaṃ   raktaṃ vā      dhūmrākāra  van  nīlavarṇaṃ vā   agniśikhāsadṛśaṃ vidyutsamānaṃ   sūryamaṇdalasadṛśaṃ     ārdhacaṃdrasadṛśaṃ  jalad---ā----samānākāraṃ \U1
%svatavarṇaṃ   atha vā  pītavarṇaṃ// raktaṃ vā      dhūmrākāraṃ yan  nīlavarṇaṃ vā   agniśikhāsadṛśaṃ vidyutsamānaṃ   sūryamaṇdalasadṛśaṃ     arddhacaṃdrasadṛśaṃ jvalad--ākāraṃ samākāraṃ \U2
%-----------------------------
%White-colored, or also yellow-colored or red-coloured or smoke-coloured or blue-coloured, like the flame of fire, equal to a lightning, like the orb of the sun, like a half-moon, appearing like flaming space, ...  
%-----------------------------
\note[type=source, labelb=181, lem={\textbf{Re}}]{PT\textsuperscript{qcr \cdot YSV} (Ed. p. 839): agnijvālāsamānābhā vidyutpuñjasamaprabhā | ādityamaṇḍalākāramathavā candramaṇḍalam |}
\note[type=source, labelb=181a, lem={\textbf{Ri}}]{SSP 2.29 (Ed. p. 41): śvetavarṇaṃ vā raktavarṇaṃ vā kṛṣṇavarṇaṃ vā agniśikhākāraṃ vā jyotirūpaṃ vā vidyudākāraṃ sūryamaṇḍalākāraṃ vā arddhacandrākāraṃ vā yatheṣṭasvapiṇḍamātraṃ sthānavarjitaṃ manasā lakṣayet ity anekaviddhaṃ madhyamaṃ lakṣyaṃ |}
\app{\lem[wit={ceteri}, alt={°śveta}]{śveta}
  \rdg[wit={U1}]{sveta°}
  \rdg[wit={U2}]{svata°}
  \rdg[wit={E}]{\om}
}\app{\lem[wit={P,L,U1,U2}, alt={°varṇaṃ}]{varṇaṃ}
  \rdg[wit={D}]{°varṇaṃ ||}
  \rdg[wit={P}]{°varaṃ}
  \rdg[wit={N1}]{°varṇā |}
  \rdg[wit={E}]{\om}}
\app{\lem[wit={ceteri}]{atha}
  \rdg[wit={E}]{aṃtha}}
\app{\lem[wit={ceteri}]{vā}
  \rdg[wit={E}]{ca}
  \rdg[wit={B,L,P}]{\om}}
pīta\app{\lem[wit={ceteri}, alt={°varṇaṃ}]{varṇaṃ}
  \rdg[wit={B,U2}]{°varṇaṃ ||}
  \rdg[wit={N2}]{°varṇa}}
\app{\lem[wit={E}, alt={raktavarṇaṃ}]{raktavarṇaṃ}
  \rdg[wit={N2}]{raktavarṇa}
  \rdg[wit={D,L,N1,U1,U2}]{raktaṃ}
  \rdg[wit={B}]{\om}}
\app{\lem[wit={ceteri}]{vā}
  \rdg[wit={N2}]{\om}}
\app{\lem[type=emendation, resp=egoscr]{dhūmravarṇaṃ}
  \rdg[wit={D}]{dhūmākāro}
  \rdg[wit={N1}]{dhūmāra}
  \rdg[wit={N2}]{dhūmravarṇa}
  \rdg[wit={U1}]{dhūmrākāra}
  \rdg[wit={Y}]{dhūmrākāraṃ}}
%\note[type=philcomm, labelb=182, lem={dhūmra°}]{Due to the repetitive use of the term \textit{varṇaṃ}, both preceding and succeeding the mention of \textit{dhūmra}, as well as its previous occurrence within the same compound, it is highly probable that the original reading was \textit{dhūmravarṇaṃ}.}
\app{\lem[wit={D}]{vā}
  \rdg[wit={N1}]{va}
  \rdg[wit={U1}]{van}
  \rdg[wit={Y}]{yan}
  \rdg[wit={N2}]{\om}}
nīlavarṇaṃ
\app{\lem[wit={ceteri}]{vā}
  \rdg[wit={B,L}]{vā ||}}
\app{\lem[wit={P}, alt={'gni°}]{'gni}
  \rdg[wit={ceteri}]{agni°}
}śikhāsadṛśaṃ vidyut\app{\lem[wit={ceteri},alt={°samānaṃ}]{samānaṃ}
  \rdg[wit={D}]{°samānaṃ ||}
  \rdg[wit={L}]{°samāne}}
sūryamaṇdala\app{\lem[wit={ceteri}, alt={°sadṛśaṃ}]{sadṛśaṃ}
  \rdg[wit={D,N1}]{°ṃ sadṛśaṃ}}
\app{\lem[wit={ceteri},alt={ardha°}]{ardha}
  \rdg[wit={B,D,N1,N2}]{ūrdhva°}
  \rdg[wit={U1}]{ārdha°}
}candrasadṛśaṃ
\app{\lem[wit={ceteri}, alt={jvalad°}]{jvala\skp{d-ā}}
  \rdg[wit={U1}]{jalad}
}\app{\lem[wit={ceteri}, alt={°ākāśa°}]{\skm{d-ā}kāśa}
  \rdg[wit={U1}]{°ā°}
  \rdg[wit={U2}]{°ākāraṃ}
}\app{\lem[wit={ceteri}, alt={°samākāraṃ}]{samākāraṃ}
  \rdg[wit={D,N1,N2,U1}]{°samānakāraṃ}
  \rdg[wit={U2}]{samakāraṃ}
  \rdg[wit={L}]{°samākāra}}/
%-----------------------------
%svaśarīraparimitaṃ      tejomanomadhye tathyaṃ kartavyam// \E
%svaśarīraparimitaṃ      tejomanomadhye lakṣyaṃ karttavyaṃ\P
%svaśarīraparimitaṃ      tejomanomadhye lakṣaṃ kartavyaṃ//  \B
%svaśarīraparimitaṃ      tejomanomadhye lakṣaṃ kartavyaṃ//  \L
%svaśarīraparimitaṃ      tejomanomadhye lakṣyaṃ karttavyaṃ//  \N1
%svaśarīraparimitaṃ      tejomanomadhye lakṣyaṃ karttavyaṃ// \D
%svaśarīraparimitaṃ      tejomanomadhye lakṣaṇaṃ karttavyaṃ//  \N2
%svaśarīraparimanomittaṃ tejomadhye     lakṣyaṃ karttavyaṃ  \U1
%svaśarīraparimitaṃ      tejomanomadhye lakṣaṃ kartavyaṃ//  \U2
%-----------------------------
%measured according to ones own body, the fixation shall be directed onto the center of the glowing mind.  
%-----------------------------
\note[type=source, labelb=183, lem={\textbf{Re}}]{PT\textsuperscript{qcr \cdot YSV} (Ed. p. 839): jvaladākāśatulyaṃvā bhāvayed rūpamātmanaḥ | etaj jyotirmayaṃ dehaṃ manomadhye tu lakṣayet |}
svaśarīrapari\app{\lem[wit={ceteri},alt={°mitaṃ}]{mitaṃ}
  \rdg[wit={U1}]{°manomittaṃ}}
tejo\app{\lem[wit={ceteri},alt={°mano}]{mano}
  \rdg[wit={U1}]{\om}
}madhye
\app{\lem[wit={D,P,N1,U1}]{lakṣyaṃ}
  \rdg[wit={E}]{tathyaṃ}
  \rdg[wit={B,L,U2}]{lakṣaṃ}
  \rdg[wit={N2}]{lakṣaṇaṃ}}
kartavyaṃ/
%-----------------------------
%ekasmin lakṣye    kṛte sati manomadhye sthitasya malasya dāho bhavati/ \E [p.38]%
%etasmil lakṣye    kṛte sati manomadhye sthitasya         dāho bhavati \P
%etasmin lakṣe     kṛte satī manomadhye sthitasya malasya dāho bhavati \B
%etasmil lakṣe     kṛte satī manomadhye sthitasya malasya dāho bhavati... \L
%ekasmin lakṣye    kṛte sati manomadhye sthitasya malasya dāho bhavati/ \N1
%ekasmin lakṣye    kṛte sati manomadhye sthitasya malasya dāho bhavati// \D
%ekasmin lakṣaṇo   kṛte sati manomadhye sthitasya malasya dāho bhavati/ \N2
%etasmin na lakṣye kṛte satī manomadhye sthitasya malasya dāho bhavati \U1 %%%281.jpg
%etasmil lakṣe     kṛte satī manomadhye sthitasya malasya dāho bhavati// \U2
%-----------------------------
%While abiding in this fixation the burning of the impurity in the center of the mind arises. 
%-----------------------------
\note[type=source, labelb=184, lem={\textbf{Re}}]{PT\textsuperscript{qcr \cdot YSV} (Ed. p. 839): eteṣāñ ca kṛte lakṣe nānāduḥkhaṃ praṇaśyati | manas astu malo yāti mahānando bhavet tataḥ |}
\app{\lem[wit={P,L,U2},alt={etasmil}]{etasmi\skp{ll-a}}
  \rdg[wit={U1}]{etasmin}
  \rdg[wit={ceteri}]{ekasmin}
}\app{\lem[wit={ceteri},alt={lakṣye}]{\skm{ll-a}kṣye}
  \rdg[wit={B,L,U2}]{lakṣe}
  \rdg[wit={U1}]{na lakṣye}
  \rdg[wit={N2}]{lakṣaṇo}}
kṛte
\app{\lem[wit={ceteri}]{sati}
  \rdg[wit={B,L,U1,U2}]{satī}}
manomadhye sthitasya
\app{\lem[wit={ceteri}]{malasya}
  \rdg[wit={P}]{\om}}
dāho bhavati/ 
%-----------------------------
%manasaḥ   sattvaguṇaprakāśo   bhavati/     puruṣa ānandamayo bhūtvā tiṣṭhati//   \E 
%manasaḥ   sattvaguṇaḥ prakaṭo bhavati      puruṣa ānandamayo bhūtvā tiṣṭhati    \P   %%%7650.jpg
%manasaḥ// sattvaguṇo  prakaṭo  bhavati//   puruṣa ānandamayo bhūtvā tiṣṭhati// \B
%manasaḥ// sattvaguṇaḥ prakaṭo bhavati      puruṣa ānandamayo bhūtvā tiṣṭhati//  \L
%manasaḥ   sattvaguṇe  prakaṭo  bhavati/    puruṣa ānandamayo bhūtvā tiṣṭhati//    \N1
%manaḥ saḥ sattvaguṇo  prakaṭo  bhavati//   puruṣa ānandamayo bhūtvā tiṣṭhati// \D
%manasaḥ   sattvaguṇo  prakaṭo  bhavati/    puruṣa ānandamayo bhūtvā tiṣṭhati//    \N2
%manasaḥ   sattvaguṇo  prakaṭo  bhavati     puruṣa ānandamayo bhūtvā tiṣṭhati       \U1
%manasaḥ   satvaguṇaprakaśo    bhavati//    puruṣa ānandamayo bhūtvā tiṣṭhati//    \U2 %%414.jpg
%-----------------------------
%The Sattva-quality of the mind becomes revealed. After this has happend the person abides supreme bliss. 
%-----------------------------
mana\app{\lem[wit={ceteri},alt={°saḥ}]{saḥ}
  \rdg[wit={B,L}]{°saḥ ||}
  \rdg[wit={D}]{manaḥ saḥ}}
sattva\app{\lem[wit={B,D,N2,U1},alt={°guṇo}]{guṇo}
  \rdg[wit={N1}]{°guṇe}
  \rdg[wit={E,U2}]{°guṇa°}
  \rdg[wit={P,L}]{°guṇaḥ}}
\app{\lem[wit={ceteri}]{prakaṭo}
  \rdg[wit={E,U2}]{°prakāśo}}
bhavati/ puruṣa ānandamayo bhūtvā tiṣṭhati\dd{}\textsuperscript{\begin{otherlanguage}{english}\coro{[\lowroman{5}]}\end{otherlanguage}}\vfill
\end{prose}
  \end{edition}
  \begin{translation}
    \ekddiv{type=trans}
    \begin{tlate}
\noindent
The Kūrma vital wind exists within the eyes. It causes [the] opening and closing [of the eyes]. From the Kṛkala vital wind gagging arises. From the Devadatta vital wind jawning arises. From the Dhanaṃjaya vital wind speech arises.\textsuperscript{\coro{[\lowroman{20}]}}
\end{tlate}
  \ekddiv{type=trans}
  \bigskip
\centerline{\textrm{\small{[\uproman{26}.\textsuperscript{\coro{\lowroman{1}-\lowroman{5}}}Madhyalakṣya]}}}
\bigskip
\begin{tlate}
Now the central fixation is taught. White-coloured or also yellow-coloured or red-coloured or smoke-coloured or blue-coloured, like the flame of fire, equal to a lightning, like the orb of the sun, like a half-moon, appearing like flaming space, measured according to one's own body, the fixation shall be directed onto the centre of the glowing mind.\footnote{\textit{Śivayogapradīpikā} 4.47cd-48: \begin{quote} śṛṇuṣva madhyalakṣyaṃ ca kathitaṃ pūrvasūribhiḥ || 4.47 \\
 śvetādivarṇanavakhaṇḍasucandrasūryasaudāminīvahniśikhena bimbāt | \\
 jvalan nabho vā sthalahīnam ekaṃ vilakṣayet tat khalu madhyalakṣyam 4.48 ||
 \end{quote}
``(47cd) Hear now the central target taught by the ancient sages. (48) Through the image consisting of rays of fire, lightning, sun, moon, and nine [different] colors such as white, etc., one should fixate on the luminous ether or on that which is locationless. Verily, this is the central fixation.''}
While abiding in the fixation, the burning of the impurity in the centre of the mind arises. The Sattva quality of the mind becomes revealed.\footnote{The generation of the sattvic quality through the practice of \textit{madhyalakṣ(y)a} also appears in \citetitle{sarvangayoga} 3.28: \begin{quote}madhya lakṣa mana madhya bicārai | vapu pramāna koi rūpa nihārai |\\
yāte sātvik upajai āī | madhya lakṣa jo sādhai bhāī || 28 ||\end{quote} ``The central Lakṣa directs the mind to reside at its center, revealing the true form of the body. It produces a sattvic quality in those who practice it.'' (28)} After this has happened, the person abides in supreme bliss.   
\end{tlate}
  \end{translation}
\end{alignment}
%\ekdpb*{}
%%%%%%%%%%%%%%%%%%%%%%%%%%%%%%%%%%%%%%%%%%
%%%%%%%%%%%%%%%%%%%%%%%%%%%%%%%%%%%%%%%%%% 
%%%%%%%%PAGEBREAK%%%%%%%PAGEBREAK%%%%%%%%%
%%%%%%%%%%%%%%%%%%%%%%%%%%%%%%%%%%%%%%%%%% 
%%%%%%%%%%%%%%%%PAGEBREAK%%%%%%%%%%%%%%%%%
%%%%%%%%%%%%%%%%%%%%%%%%%%%%%%%%%%%%%%%%%% 
%%%%%%%%PAGEBREAK%%%%%%%PAGEBREAK%%%%%%%%%
%%%%%%%%%%%%%%%%%%%%%%%%%%%%%%%%%%%%%%%%%% 
%%%%%%%%%%%%%%%%%%%%%%%%%%%%%%%%%%%%%%%%%% 
%%%%%%%%%%%%%%%%%%%%%%%%%%%%%%%%%%%%%%%%%% 
%%%%%%%%%%%%%%%%%%%%%%%%%%%%%%%%%%%%%%%%%% 
%%%%%%%%PAGEBREAK%%%%%%%PAGEBREAK%%%%%%%%%
%%%%%%%%%%%%%%%%%%%%%%%%%%%%%%%%%%%%%%%%%% 
%%%%%%%%%%%%%%%%PAGEBREAK%%%%%%%%%%%%%%%%%
%%%%%%%%%%%%%%%%%%%%%%%%%%%%%%%%%%%%%%%%%% 
%%%%%%%%PAGEBREAK%%%%%%%PAGEBREAK%%%%%%%%%
%%%%%%%%%%%%%%%%%%%%%%%%%%%%%%%%%%%%%%%%%% 
%%%%%%%%%%%%%%%%%%%%%%%%%%%%%%%%%%%%%%%%%% 
%%%%%%%%%%%%%%%%%%%%%%%%%%%%%%%%%%%%%%%%%% 
%%%%%%%%%%%%%%%%%%%%%%%%%%%%%%%%%%%%%%%%%% 
%%%%%%%%PAGEBREAK%%%%%%%PAGEBREAK%%%%%%%%%
%%%%%%%%%%%%%%%%%%%%%%%%%%%%%%%%%%%%%%%%%% 
%%%%%%%%%%%%%%%%PAGEBREAK%%%%%%%%%%%%%%%%%
%%%%%%%%%%%%%%%%%%%%%%%%%%%%%%%%%%%%%%%%%% 
%%%%%%%%PAGEBREAK%%%%%%%PAGEBREAK%%%%%%%%%
%%%%%%%%%%%%%%%%%%%%%%%%%%%%%%%%%%%%%%%%%% 
%%%%%%%%%%%%%%%%%%%%%%%%%%%%%%%%%%%%%%%%%% 
\begin{alignment}[
  texts=edition[class="edition"];
  translation[class="translation"],
  ]
  \begin{edition}
    \ekddiv{type=ed}
    \centerline{\textrm{\small{[\uproman{27}.\textsuperscript{\coro{\lowroman{1}-\lowroman{12}}}The Divisions of Space]}}}
    \label{madhyalakṣya}
\bigskip
\begin{prose}
 \noindent
%-----------------------------
%idānīm-ākāśabhedāḥ kathyante/ \E
%idānīm ākaśabhedāḥ kathyaṃte   \P
%idānīṃ ākaśabhedāḥ kathyaṃte/ \B
%idānīṃ ākaśabhedāḥ kathyate/  \L
%idānīṃ ākaśabhedāḥ kathyaṃte/ \N1
%idānīṃ ākaśabhedāḥ kathyaṃte// \D
%idānīṃ ākāśabhedāḥ kathyate/  \N2
%idānīṃ ākāśabhedāḥ kathyaṃte  \U1
%idānīm ākāśabhedāḥ kathyate// \U2
%-----------------------------
%Now the divisions of space are taught. 
%-----------------------------
   \note[type=source, labelb=185, lem={\textbf{Re}}]{PT\textsuperscript{qcr \cdot YSV} (Ed. p. 839): kathyate tu devyadhunākāśaṃ pañcabhirlakṣaṇaiḥ | ākāśan tu mahākāśaṃ parākāśaṃ parātparam | tattvākāśaṃ sūryakāśamākāśaṃ pañcalakṣaṇam |}
\note[type=source, labelb=185, lem={\textbf{Re}}]{PT\textsuperscript{qcr \cdot YSV} (Ed. p. 839) = YK\textsuperscript{ccn \cdot YSV} 1.37 (Ed. p. 26): ākāśan tu mahākāśaṃ parākāśaṃ parātparam | tattvākāśaṃ sūryakāśamākāśaṃ pañcalakṣaṇam |}
\note[type=testium, labelb=186, lem={\textbf{Ri}}]{SSP 2.30 (Ed. p. 42): ākāśaṃ parākāśaṃ mahākāśaṃ tatvākaśaṃ sūryākāśamiti vyomapañcakam | bāhyābhyantare 'tyantaṃ nirmalaṃ nirākāraṃ ākāśaṃ lakṣayet |}
\app{\lem[wit={E,P,U2},alt={idānīm}]{idānī\skp{m-ā}}
  \rdg[wit={ceteri}]{idānīṃ}
}\skm{m-ā}kāśabhedāḥ
\app{\lem[wit={ceteri}]{kathyante}
  \rdg[wit={L,N2,U2}]{kathyate}}/
%-----------------------------SSP!
%te                          ākāśaḥ paramākāśaḥ mahākāśaḥ tattvākāśaḥ sūryākāśaḥ/    bāhyābhyantare nirmalaṃ nirākāram ākāśa---lakṣyaṃ  karttavyam/ \E
%teṣāṃ lakṣyāni ca kathyaṃte ākāśaḥ parākāśaḥ mahākāśaḥ tatvākāśaḥ sūryakāśaḥ        bāhyābhyaṃtare nirmalaṃ nirākāram ākāśaṃ  lakṣyaṃ  karttavyaṃ  \P
%                            ākāśaḥ paramākāśaḥ// mahākāśa// tattvākāśaḥ sūryākāśa// bāhyābhyaṃtaro nirmalaṃ nirākāram ākāśaṃ  lakṣaṃ   kartavyaṃ// \B
%                            ākāśaḥ paramākāśaḥ// mahākāśaḥ tattvākāśaḥ sūryākāśaḥ   bāhyābhyaṃtare nirmalaṃ nirākāram ākāśaṃ  lakṣaṃ   kartavyaṃ// \L
%teṣāṃ lakṣyāni  kathyate//  ākāśa, parākāśa,mahākāśa,tatvākāśa,sūryakāśa//          bāhyābhyaṃtare nirmalaṃ nirākāraṃ ākāśa---lakṣyaṃ  kartavyaṃ// \N1
%teṣāṃ lakṣyāṇi  kathyaṃte// ākāśa--parākāśamahākāśatatvākāśasūryakāśa               bāhyābhyaṃtare nirmalaṃ nirākāraṃ ākāśa---lakṣyaṃ  karttavyaṃ// \D   %%%p.11 verso
%teṣāṃ lakṣaṇāni kathyate//  ākāśa--parākāśamahākāśatatvākāśasūryakāśaḥ              bāhyābhyaṃtare nirmalaṃ nirākāraṃ ākāśa---lakṣaṇaṃ kartavyaṃ// \N2
%ṣāṃ   lakṣyāṇi  kathyaṃte   ākāśa--parākāśamahākāśatatvākāśasūryakāśa---------------bāhyābhyaṃtare nirmalaṃ nirākāraṃ ākāśa---lakṣyaṃ  karttavyaṃ  \U1
%teṣāṃ lakṣyāni  kathyaṃte// ākāśaḥ parākāśa// mahākāśaḥ// tatvākāśaḥ// sūryakāśaḥ// bāhyābhyaṃtare nirmalaṃ nirākāraṃ mākāśaṃ lakṣyaṃ  karttavyaṃ// \U2
%-----------------------------
%The fixations of them are taught: Space, beyond space, great space, space of reality, the space of the sun. The fixation onto the pure and formless space \textit{akāśa} shall be done internally as well as externally.  
%-----------------------------SSP!
\app{\lem[wit={ceteri}]{teṣāṃ}
  \rdg[wit={E}]{te}
  \rdg[wit={U1}]{ṣaṃ}}
\app{\lem[wit={ceteri}]{lakṣyāni}
  \rdg[wit={N2}]{lakṣaṇāni}}
\app{\lem[wit={D,U1,U2}]{kathyante}
  \rdg[wit={P}]{ca kathyante}
  \rdg[wit={N1,N2}]{kathyate}}/
\note[type=philcomm, labelb=186a, lem={\uproman{27}\textsuperscript{\lowroman{2}}}]{Sentence \om in \getsiglum{B} and \getsiglum{L}. \getsiglum{E} preserves only the first \textit{akṣara} ``te'' and omits the rest.}
\app{\lem[wit={B,E,L,P}]{ākāśaḥ}
  \rdg[wit={D,N1,N2,U1}]{ākāśa°}}\dd{}
\app{\lem[wit={B,E,L}]{paramākāśaḥ}
  \rdg[wit={P,U2}]{parākāśaḥ}
  \rdg[wit={N1}]{parākāśa}
  \rdg[wit={D,N2,U1}]{parākāśa°}}\dd{}
\app{\lem[wit={E,L,P,U2}]{mahākāśaḥ}
  \rdg[wit={B,N1}]{mahākāśa}
  \rdg[wit={D,N2,U1}]{mahākāśa°}}\dd{}
\app{\lem[wit={B,E,L,U2}]{tattvakāśaḥ}
  \rdg[wit={N1}]{tatvakāśa}
  \rdg[wit={D,N2,U1}]{tatvakāśa°}}\dd{}
\app{\lem[wit={B,E,L}]{sūryākāśaḥ}
  \rdg[wit={N2,P,U2}]{sūryakāśaḥ}
  \rdg[wit={N1}]{sūryakāśa}
  \rdg[wit={D,U1}]{sūryakāśa°}}\dd{}
bāhyābhyantare nirmalaṃ nirākāram
\app{\lem[wit={ceteri},alt={ākāśa°}]{ākāśa}
  \rdg[wit={U2}]{mākāśaṃ}
  \rdg[wit={B,L,P}]{ākāśaṃ}
}\app{\lem[wit={ceteri}, alt={°lakṣyaṃ}]{lakṣyaṃ}
  \rdg[wit={B,L}]{lakṣaṃ}
  \rdg[wit={N2}]{°lakṣaṇaṃ}}
kartavya\app{\lem[wit={E}]{kartavyam}
  \rdg[wit={ceteri}]{kartavyaṃ}}\dd{}
%-----------------------------
%tataḥ paraṃ bāhyābhyantare  ṣvanandhakārasadṛśaṃ   parākāśaikyaṃ lakṣyaṃ  karttavyam// \E
%tataḥ paraṃ bāhyābhyantarai ghanāṃdhakāraṃ sadṛśa--parākāśasya   lakṣyaṃ  karttavyam \P
%tataḥ paraṃ bāhyābhyaṃtare  ghanāṃghakārasadṛśaḥ   parākāśa------lakṣaṃ   kartavyaṃ// \B
%tataḥ paraṃ bāhyābhyaṃtare        dhakārasadṛśaḥ   parākāśa------lakṣaṃ   kartavyaṃ... \L %%%%%%%%%%%%%%%%%ghana hier = dunkel, schwarz%%%% andhakāra=  finster, dunkel. Finsterniss
%tataḥ paraṃ bāhyābhyantare  ghanāṃdhakārasadṛśa----parākāśasya   lakṣyaṃ  kattavyam// \N1
%tataḥ paraṃ bāhyābhyantare  ghanāṃdhakārasadṛśa----parākāśasya   lakṣyaṃ  kattavyaṃ// \D
%tataḥ paraṃ bāhyābhyantare  ghanāṃdhakārasadṛśa----parākāśasya   lakṣaṇaṃ karttavyam// \N2
%tataḥ paraṃ bāhyābhyantare  ghanāṃdhakārasadṛśa----parākāśasya   lakṣyaṃ  karttavyaṃ \U1
%tataḥ       bāhyābhyantare  ghanāṃdhakārasadṛśaṃ   parākāśasya   lakṣaṃ   karttavyaṃ// \U2
%-----------------------------
%Moreover, the fixation of the beyond-space \textit{parākāśa} which is equal to dense darkness shall be done internally and externally.
%-----------------------------SSP!
\note[type=source, labelb=187, lem={\textbf{Re}}]{PT\textsuperscript{qcr \cdot YSV} (Ed. p. 839): sabāhyābhyantare nityaṃ nirākāśantu (\textit{nirākāśas tu} YK\textsuperscript{ccn \cdot YSV} 2.38 Ed. p. 26) nirmalam | karttavyaṃ lakṣam ākāśaṃ sādhayet sādhanaṃ vinā | ghanāntarālasadṛśaṃ parākāśaṃ tathaiva ca |}
tataḥ
\app{\lem[wit={ceteri}]{paraṃ}
  \rdg[wit={U2}]{\om}}
\note[type=testium, labelb=188, lem={\textbf{Ri}}]{SSP 2.30 (Ed. p. 42): atha vā bāhyābhyantare 'tyantāndhakāranibhaṃ parākāśam avalokayet |}
\app{\lem[wit={ceteri}]{bāhyābhyantare}
  \rdg[wit={P}]{bāhyābhyantarai}}
\app{\lem[wit={ceteri},alt={ghanāndha°}]{ghanāndha}
  \rdg[wit={B}]{ghanāṃgha°}
  \rdg[wit={E}]{ṣvanandha°}
  \rdg[wit={L}]{dha°}
}\app{\lem[wit={ceteri},alt={°kāra°}]{kāra}
  \rdg[wit={P}]{°kāraṃ}
}\app{\lem[wit={ceteri},alt={°sadṛśa°}]{sadṛśa}
  \rdg[wit={E,U2}]{sadṛśaṃ}
  \rdg[wit={B,L}]{sadṛśaḥ}
}\app{\lem[wit={ceteri}]{parākāśasya}
  \rdg[wit={E}]{parākāśaikyaṃ}
  \rdg[wit={B,L}]{parākāśa°}}
\app{\lem[wit={ceteri}]{lakṣyaṃ}
  \rdg[wit={B,L,U2}]{lakṣaṃ}
  \rdg[wit={N2}]{lakṣaṇaṃ}}
kartavyaṃ/\textsuperscript{\begin{otherlanguage}{english}\coro{[\lowroman{5}]}\end{otherlanguage}}
%-----------------------------
%tataḥ paraṃ pralayakālīna--jvalad-dāvā---nala-pūrṇaṃ  bāhyābhyantare, mahākāśalakṣyaṃ karttavyam/ \E
%tataḥ paraṃ pralayakālīna--jalad--vaḍavā-nala-pūrṇaṃ  bāhyābhyaṃtare  mahākāśaṃ lakṣyaṃ karttavyaṃ \P
%tataḥ paraṃ pralayakālīnaḥ jalad--vaḍavā-nala-pūrṇaṃ  bāhyābhyaṃtare  mahākāśalakṣaṃ kartavyaṃ// \B
%tataḥ paraṃ pralayakālīnaḥ jvalad-vaḍavā-nala-pūrṇaṃ  bāhyābhyaṃtare  mahākāśalakṣaṃ kartavyaṃ// \L
%tataḥ paraṃ pralayakālīna--jvalad-vṛddha-nala-pūrṇa---bāhyābhyaṃtare  mahākāśalakṣyaṃ karttavyaṃ// \N1 ?[S.9 verso letzte Zeile] 
%tataḥ paraṃ pralayakālīna--jvalad-dāvā---nala-pūrṇaṃ  bāhyābhyaṃtare  mahākāśaṃ lakṣaṃ karttavyaṃ// \D
%tataḥ paraṃ pralayakālīna--jvalad-vṛ-----nala-pūrṇa---bāhyābhyaṃtare  mahākāśalakṣaṃ karttavyaṃ// \N2
%tataḥ paraṃ pralayakālīta--jjala--vaḍavā-nala-pūrṇaṃ  bāhyābhyaṃtare  mahākāśaṃ lakṣyaṃ kartavyaṃ \U1
%tataḥ       pralayakālīna--jvalad-vaḍavā-nala-pūrṇa---bāhyābhyaṃtare  ghanāṃ dhakārasadṛśaṃ mahākāśasya lakṣaṃ karttavyaṃ \U2
%-----------------------------
%Moreover, the fixation of the great space (\textit{mahākāśa}) which is the plethora of the burning fire of the time of dissolution shall be done internally and externally. 
%-----------------------------SSP!
\note[type=source, labelb=190, lem={\textbf{Re}}]{PT\textsuperscript{qcr \cdot YSV} (Ed. p. 839): kalpāntāgnisamaṃ (\textit{kālāntāgnisamaṃ} YK\textsuperscript{ccn \cdot YSV} 2.39cd Ed. p. 26) jyotir mahākāśaṃ smaret tathā |}
\note[type=testium, labelb=191, lem={\textbf{Ri}}]{SSP 2.30 (Ed. p. 42): bāhyābhyantare kālānalasaṃkāśaṃ mahākāśam avalokayet |}
tataḥ
\app{\lem[wit={ceteri}]{paraṃ}
  \rdg[wit={ceteri}]{U2}}
pralayakālī\app{\lem[wit={ceteri},alt={°na}]{na}
  \rdg[wit={B,L}]{°naḥ}
}\app{\lem[wit={ceteri},alt={°jvalad°}]{jvalad}
  \rdg[wit={P,B}]{°jalad°}
  \rdg[wit={U1}]{°jjala°}
}\app{\lem[wit={E,D},alt={°dāvā°}]{dāvā}
    \rdg[wit={B,L,P,U1,U2}]{°vaḍavā°}
    \rdg[wit={N1}]{°vṛddha°}
    \rdg[wit={N2}]{°vṛ°}
}nalapū\app{\lem[wit={ceteri},alt={°rṇaṃ}]{rṇaṃ}
  \rdg[wit={N1,N2,U2}]{°rṇa}}
bāhyābhyantare
\app{\lem[wit={ceteri},alt={mahākāśa°}]{mahākaśa}
  \rdg[wit={D,P,U1}]{mahākāśaṃ}
  \rdg[wit={U2}]{ghanāṃ dhakārasadṛśaṃ mahākāśasya}
}\app{\lem[wit={ceteri}, alt={°lakṣyaṃ}]{lakṣyaṃ}
  \rdg[wit={B,D,L,N2,U2}]{°lakṣaṃ}}
kartavvyaṃ/
%-----------------------------
%\om                                                                                                                         \E
%tataḥ paraṃ bāhyābhyaṃtare koṭidīpānāṃ prakāśaprāptau  yādṛśam aujvalyaṃ bhavati   tādṛśaṃ   tatvākāśaṃ lakṣyaṃ karttavyaṃ  \P
%tataḥ paraṃ bāhyābhyaṃtare koṭidīpānāṃ prakāśaprāpto   yādṛśam aujvalaṃ  bhavatī/  tādṛśaṃ   tatvāśa----lakṣaṃ kartavyaṃ//  \B
%tataḥ paraṃ bāhyābhyaṃtare koṭidīpānāṃ prakāśaprāpto   yādṛśam  ujvalaṃ  bhavatī/  tādṛśaṃ   tatvāśa----lakṣaṃ kartavyaṃ    \L  
%tataḥ paraṃ bāhyābhyaṃtare koṭidīpānāṃ prakāśaprāptau  yādṛśam aujvalyaṃ bhavati/  tādṛśaṃ   tatvākāśaṃ lakṣyaṃ kartavyaṃ// \N1
%tataḥ paraṃ bāhyābhyaṃtare koṭidīpānāṃ prakāśaprāptau  yādṛśam aujvalyaṃ bhavati// tādṛśaṃ   tatvākāśaṃ lakṣaṃ kartavyaṃ//  \D
%tataḥ paraṃ bāhyābhyaṃtare koṭidīpānāṃ prakāśaprāptau  yādṛśam aujvala   bhavati/  tādṛśaṃ   tatvākāśaṃ lakṣaṃ kartavyaṃ//  \N2
%tataḥ paraṃ bāhyābhyaṃtare koṭidīpānāṃ prakāśaprāptau  yādṛśam aujvalaṃ  bhavati   tādṛśaṃ   tatvākāśaṃ lakṣyaṃ kartavyaṃ   \U1
%tataḥ paraṃ bāhyābhyaṃtare koṭidīpānāṃ prakāśaprāptau  yādṛśem aujvalyaṃ bhavati   tādṛśaṃ// tatvākāśaṃ lakṣaṃ karttavyaṃ// \U2
%-----------------------------
%Then, internally and externally the brightness of millions of blazing lights arises, he shall execute the fixation [directed onto] the reality-space (\textit{tattvakāśa}) which is as such.   
%-----------------------------SSP!
tataḥ paraṃ bāhyābhyaṃtare koṭidīpānāṃ
\note[type=philcomm, labelb=192, lem={\uproman{27}\textsuperscript{lowroman{7}-\lowroman{8}}}]{Sentences are omitted in \getsiglum{E}.}
\note[type=testium, labelb=193, lem={\textbf{Ri}}]{SSP 2.30 (Ed. p. 42): bāhyābhyantare nijatatvakharūpaṃ tatvākāśam avalokayet |}
\note[type=source, labelb=194, lem={\textbf{Re}}]{PT\textsuperscript{qcr \cdot YSV} (Ed. p. 839) = YK\textsuperscript{ccn \cdot YSV} 2.40ab (Ed. p. 26): koṭikoṭipradīpābhaṃ tattvākāśaṃ smaret tathā |}
\app{\lem[wit={ceteri}]{prakāśaprāptau}
  \rdg[wit={B,L}]{prakāśaprāpto}}
yādṛśaṃ
\app{\lem[wit={ceteri}]{aujvalyaṃ}
  \rdg[wit={L}]{ujvalaṃ}}
\app{\lem[wit={ceteri}]{bhavati}
  \rdg[wit={B,L}]{bhavatī}}/
tādṛśaṃ
\app{\lem[wit={ceteri}]{tattvākāśaṃ}
  \rdg[wit={B,L}]{tattvāśa°}}
\app{\lem[wit={P,N1,U1}]{lakṣyaṃ}
  \rdg[wit={B,D,L,N2,U2}]{lakṣaṃ}}
kartavyaṃ/
\end{prose}
  \end{edition}
  \begin{translation}
    \ekddiv{type=trans}
    \centerline{\textrm{\small{[\uproman{27}.\textsuperscript{\coro{\lowroman{1}-\lowroman{12}}}The Divisions of Space]}}}
    \bigskip
    \begin{tlate}
      \noindent
 Now the divisions of space are taught.\footnote{In contrast to the \textit{Yogatattvabindu}, \citetitle{ssplonavla} and the quotes of \textit{Yogasvarodaya} in the \citetitle{ramatosana} and \citetitle{yogakarnika}, the \citetitle{advaya} 7 (Ed. pp. 4-5) does not spaerate the practice of Madhyalakṣya and the five spaces. Here, both practices form a unified whole and follow a specific progression:
\begin{quote}
  atha madhyalakṣyalakṣaṇaṃ | prātaścitrādivarṇākhaṇḍasūryacakravat vahnijvālāvalīvat tadvihīnāntarikṣavat paśyati | tadākārākāritayā avatiṣṭhati | tadbhūyodarśanena guṇarahitākāśaṃ bhavati | visphurattārakākāradīpyamānagāḍhatamopamaṃ paramākāśaṃ bhavati | kālānalasamadyotamānaṃ mahākāśaṃ bhavati | sarvotkṛṣṭaparamadyutipradyotamānaṃ tattvākāśaṃ bhavati | koṭisūryaprakāśavaibhavasaṃkāśaṃ sūryākāśaṃ bhavati | evaṃ bāhyābhyantarasthavyomapañcakaṃ tārakalakṣyam | taddarśī vimuktaphalas tādṛgvyomasamāno bhavati | tasmāt tāraka eva lakṣyaṃ amanaskaphalapradaṃ bhavati || 7 ||
\end{quote}
\begin{quote}
%Then, the characteristic of the central fixation: At daybreak he sees [that which is], like the bright etc. colourful indivisible and disc of the sun, like a row of flames of dire, like that [which is] free from the intermediate space between heaven and earth. He dwells in a state [in which he is mentally] assuming the form of the apparition of that. By [mentally] contemplating about that a space without qualities arises. Supreme space resembling absolute darkness  
 Then, the description of Intermediate Introspection: He sees, at daybreak, like the vast disc of the Sun resplendent with variegated and other colours, like a huge conflagration of Fire and like the mid-etherial regions devoid of these. He stands with a form identical with theirs. By seeing them over and over again, there ensues Ether devoid of qualities; there ensues transcendent Ether resembling palpable darkness brought into relief by the splendour of the radiant Tāraka form; there ensues the great Ether shining like the fire of deluge; there ensues the Tattvākāśa (Ether of Verity) effulgent with transcendent lustre excelling; and there ensues the Ether of Sun dazzling with the splendour of a hundred thousand Suns. Thus the five Ethers, external and internal, become visible to the Introspection of a Tāraka-yogin. He who sees it, released from fruits, becomes like such Ether. Hence the Introspectuion of Tāraka alone becomes the bestower of the fruit of non-mindedness.  
(\citeauthor[1938:4]{yogaupaniṣads})
\end{quote}} The fixations of them are taught: Space, beyond space, great space, space of reality, the space of the sun. The fixation onto the pure and formless space \textit{akāśa} shall be done internally as well as externally. Therafter, the fixation of the beyond-space \textit{parākāśa} which is equal to dense darkness\footnote{Instead of extreme brightness as in the \citetitle{ssplonavla} (Ed. p. 29) and \citetitle{advaya} (Ed. p. 5), Rāmacandra has choosen dense darkness to characterize his \textit{parākāśa}-visualization.} shall be done internally and externally. Then, the fixation of the great space (\textit{mahākāśa}) which is the plethora of the burning fire of the time of dissolution shall be done internally and externally. Then, when internally and externally the brightness of millions of blazing lights arises, he shall execute the fixation [directed onto] the reality-space (\textit{tattvakāśa}) which is as such.\end{tlate}
  \end{translation}
\end{alignment}
%%%%%%%%%%%%%%%%%%%%%%%%%%%%%%%%%%%%%%%%%%
%%%%%%%%%%%%%%%%%%%%%%%%%%%%%%%%%%%%%%%%%% 
%%%%%%%%PAGEBREAK%%%%%%%PAGEBREAK%%%%%%%%%
%%%%%%%%%%%%%%%%%%%%%%%%%%%%%%%%%%%%%%%%%% 
%%%%%%%%%%%%%%%%PAGEBREAK%%%%%%%%%%%%%%%%%
%%%%%%%%%%%%%%%%%%%%%%%%%%%%%%%%%%%%%%%%%% 
%%%%%%%%PAGEBREAK%%%%%%%PAGEBREAK%%%%%%%%%
%%%%%%%%%%%%%%%%%%%%%%%%%%%%%%%%%%%%%%%%%% 
%%%%%%%%%%%%%%%%%%%%%%%%%%%%%%%%%%%%%%%%%% 
%%%%%%%%%%%%%%%%%%%%%%%%%%%%%%%%%%%%%%%%%% 
%%%%%%%%%%%%%%%%%%%%%%%%%%%%%%%%%%%%%%%%%% 
%%%%%%%%PAGEBREAK%%%%%%%PAGEBREAK%%%%%%%%%
%%%%%%%%%%%%%%%%%%%%%%%%%%%%%%%%%%%%%%%%%% 
%%%%%%%%%%%%%%%%PAGEBREAK%%%%%%%%%%%%%%%%%
%%%%%%%%%%%%%%%%%%%%%%%%%%%%%%%%%%%%%%%%%% 
%%%%%%%%PAGEBREAK%%%%%%%PAGEBREAK%%%%%%%%%
%%%%%%%%%%%%%%%%%%%%%%%%%%%%%%%%%%%%%%%%%% 
%%%%%%%%%%%%%%%%%%%%%%%%%%%%%%%%%%%%%%%%%% 
%%%%%%%%%%%%%%%%%%%%%%%%%%%%%%%%%%%%%%%%%% 
%%%%%%%%%%%%%%%%%%%%%%%%%%%%%%%%%%%%%%%%%% 
%%%%%%%%PAGEBREAK%%%%%%%PAGEBREAK%%%%%%%%%
%%%%%%%%%%%%%%%%%%%%%%%%%%%%%%%%%%%%%%%%%% 
%%%%%%%%%%%%%%%%PAGEBREAK%%%%%%%%%%%%%%%%%
%%%%%%%%%%%%%%%%%%%%%%%%%%%%%%%%%%%%%%%%%% 
%%%%%%%%PAGEBREAK%%%%%%%PAGEBREAK%%%%%%%%%
%%%%%%%%%%%%%%%%%%%%%%%%%%%%%%%%%%%%%%%%%% 
%%%%%%%%%%%%%%%%%%%%%%%%%%%%%%%%%%%%%%%%%% 
\begin{alignment}[
  texts=edition[class="edition"];
  translation[class="translation"],
  ]
  \begin{edition}
    \ekddiv{type=ed}
    \begin{prose}
      \noindent
%-----------------------------
%tataḥ        bāhyābhyantare  prakāśa-mānayarsūsahitaṃ        sūryākāśaṃ lakṣyaṃ karttavyam/ \E [p.39]
%tataḥ paścād bāhyābhyaṃtare  prakāśa-māgasūryaṃ biṃbasahitaṃ sūryākāśalakṣyaṃ   karttavyaṃ ... \P
%      paccā  bāhyābhyaṃtare  prakāśa-mān sūryabiṃbasahita----sūryakāśalakṣaṃ    kartavyaṃ mataḥ ... \B
%      paccā  bāhyābhyaṃtare  prakāśa-mān sūryabiṃbasahita----sūryakāśalakṣaṃ    kartavyaṃ mataḥ ... \L 
%tataḥ paścāt bāhyābhyaṃtare  prakāśa-mānasūryabiṃbasahitaṃ   sūryakāśaṃ lakṣyaṃ karttavyaṃ// \N1
%tataḥ paścāt bāhyābhyaṃtare  prakāśa-mānasūryabiṃbasahitaṃ   sūryakāśaṃ lakṣyaṃ karttavyaṃ// \D
%tataḥ paścād     ābhyaṃtare  prakāśa-mānasūryabiṃbasahitaṃ   sūryakāśaṃ lakṣaṃ  karttavyaṃ// \N2
%tataḥ paścāt bāhyabhyaṃttare prakāśa-mānasūryabiṃbasāhitaṃ   sūryakāśaṃ lakṣyaṃ karttavyaṃ \U1
%tataḥ paścād bāhyābhyaṃtare  prakāśa-mānasūryabiṃbasāhitaṃ   sūryākāśaṃ lakṣyaṃ karttavyaṃ// \U2
%-----------------------------
%After that the fixation of the sun-space (\textit{sūryakāśa}) which is associated with sundisk's appearance of light shall be done internally and externally.   
%-----------------------------SSP!
\note[type=testium, labelb=195, lem={\textbf{Ri}}]{SSP 2.30 (Ed. p. 42): atha vā bāhyābhyantare sūryakoṭisadṛśaṃ sūryākāśam avalokayet |}
\note[type=source, labelb=196, lem={\textbf{Re}}]{PT\textsuperscript{qcr \cdot YSV} (Ed. p. 839): sūryākāśaṃ tathā koṭisūryavindusamaṃ (\textit{°bimbasamaṃ} YK\textsuperscript{ccn \cdot YSV} 2.40d Ed. p. 26) smaret | sabāhyābhyantare caivam ākāśaṃ (\textit{caiva sākāśaṃ} YK\textsuperscript{ccn \cdot YSV} 2.41b Ed. p. 26) lakṣayet tu yaḥ |}
\app{\lem[wit={ceteri}]{tataḥ}
  \rdg[wit={B,L}]{\om}}
\app{\lem[wit={ceteri}, alt={paścād}]{paścā\skp{d-bā}}
  \rdg[wit={N1,N2,U1}]{paścāt}
  \rdg[wit={B,L}]{paccā}
  \rdg[wit={E}]{\om}}
\app{\lem[wit={ceteri},alt={bāhyābhyaṃtare}]{\skm{d-bā}hyābhyaṃtare}
  \rdg[wit={N2}]{ābhyaṃtare}}
\app{\lem[wit={ceteri},alt={prakāśamāna°}]{prakāśamāna}
  \rdg[wit={P}]{prakāśamāga°}
  \rdg[wit={B,L}]{prakāśamān}
}\app{\lem[wit={ceteri},alt={°sūrya°}]{sūrya}
  \rdg[wit={E}]{°yarsū°}
  \rdg[wit={P}]{°sūryaṃ}
}\app{\lem[wit={ceteri},alt={°bimba°}]{bimba}
  \rdg[wit={E}]{\om}
}\app{\lem[wit={ceteri},alt={°sahitaṃ}]{sahitaṃ}
  \rdg[wit={B,L}]{°sahita°}}
\app{\lem[wit={ceteri}]{sūryakāśaṃ}
  \rdg[wit={B,L,P}]{sūryakāśa°}}
\app{\lem[wit={ceteri}]{lakṣyaṃ}
  \rdg[wit={B,L,N2}]{lakṣaṃ}}
\app{\lem[wit={ceteri}]{kartavyaṃ}
  \rdg[wit={B,L}]{kartavyaṃ mataḥ}}/      
%-----------------------------
%eteṣāṃ lakṣyāṇāṃ kāraṇāt   śarīraṃ rogāsaṃsargi    bhavati// \E
%eteṣāṃ lakṣāṇāṃ  karaṇāt   śarīre  rogasaṃsargo na bhavati \P %%%7651.jpg
%eteṣāṃ lakṣaṇaṃ  karaṇāt// śarīre  rogasaṃsargo na bhavatī/ \B
%eteṣāṃ lakṣaṃ    karaṇāt   śarīre  rogasaṃsargo na bhavati... \L
%eteṣāṃ lakṣyaṇāṃ karaṇāt   śarīra--rohasaṃsarge na bhavati/ \N1
%eteṣāṃ lakṣyāṇāṃ karaṇāt   śarīra--rohasaṃsargo na bhavati// \D
%eteṣāṃ lakṣāṇā---kāraṇāc---charīra-rogāsaṃsargo na bhavati// \N2
%eteṣāṃ lakṣyāṇāṃ karaṇāt   śarīra--rogāsaṃsargo na bhavati \U1
%eteṣāṃ lakṣyāṇāṃ karaṇāt// śarīre  rogāsaṃsargo na bhavati \U2
%-----------------------------
%From the execution of these fixations contact of diseases does not arise within the body. 
%-----------------------------
\note[type=source, labelb=196, lem={\textbf{Re}}]{PT\textsuperscript{qcr \cdot YSV} (Ed. p. 839): śivavad vihare dviśve pāpapuṇyavivarjitaḥ | eteṣāñ caiva lakṣeṇa karmadvārā 'ghamāharet (\textit{karmmadvārānapāharet} YK\textsuperscript{ccn \cdot YSV} 2.41d Ed. p. 26) |}
eteṣāṃ \app{\lem[wit={ceteri}]{lakṣyāṇāṃ}
  \rdg[wit={P}]{lakṣāṇāṃ}
  \rdg[wit={B}]{lakṣaṇaṃ}
  \rdg[wit={L}]{lakṣaṃ}
  \rdg[wit={N2}]{lakṣāṇā}}
\app{\lem[wit={N2}, alt={kāraṇāc}]{kāraṇā\skp{c-cha}}
  \rdg[wit={E}]{kāraṇāt}
  \rdg[wit={ceteri}]{karaṇāt}
}\app{\lem[wit={N2}, alt={charīre}]{\skm{c-cha}rīre}
  \rdg[wit={D,N1}]{śarīra°}
  \rdg[wit={B,P,L,U2}]{śarīre}
  \rdg[wit={E}]{°śarīraṃ}}
\app{\lem[wit={ceteri}, alt={rogāsaṃsargo}]{rogāsaṃsargo}
  \rdg[wit={E}]{rogāsaṃsargi}}
\app{\lem[wit={ceteri}]{na}
  \rdg[wit={E}]{\om}}
\app{\lem[wit={ceteri}]{bhavati}
  \rdg[wit={B}]{bhavatī}}/\textsuperscript{\begin{otherlanguage}{english}\coro{[\lowroman{10}]}\end{otherlanguage}}
% -----------------------------
%tathā valitapalitaṃ   puṇyaṃ pāpaṃ    na bhavati//    \E
%tathā valitapalitaṃ   puṇyāṃ pāpaṃ ca na bhavati   \P
%tathā// valitapalitaṃ puṇyāṃ pāpaṃ ca na bhavatī// \B
%tathā valitaṃ palitaṃ puṇyāṃ pāpaṃ ca na bhavatī// \L
%tathā valitaṃ palitaṃ puṇyaṃ pāpaṃ ca na bhavati// \N1
%tathā valitaṃ palitaṃ puṇyaṃ pāpaṃ ca na bhavati// \D
%tathā valitaṃ palitaṃ puṇyaṃ pāpaṃ ca na bhavati// \N2
%tathā valitaṃ palitaṃ puṇyaṃ pāpaṃ ca na bhati \U1
%tathā valīpalitaṃ     puṇyaṃ pāpaṃ ca na bhavati \U2
%-----------------------------
%Thus wrinkles and grey hair, sin or merit does not arise. 
%-----------------------------
tathā
\app{\lem[wit={D,L,N1,N2}, alt={valitaṃ palitaṃ}]{valitaṃ palitaṃ}
  \rdg[wit={N2}]{valīpalitaṃ}
  \rdg[wit={B,E,P}]{valitapalitaṃ}}
\app{\lem[wit={ceteri}]{puṇyaṃ}
  \rdg[wit={B,L}]{puṇyāṃ}}
pāpaṃ
\app{\lem[wit={ceteri}]{ca}
  \rdg[wit={E}]{\om}}
na
\app{\lem[wit={ceteri}]{bhavati}
  \rdg[wit={B,L}]{bhavatī}
  \rdg[wit={U1}]{bhati}}/
\end{prose}
\begin{tlg}
% -----------------------------
%          navacakraṃ kalādhāraṃ trilakṣyaṃ vyomapaṃcakam/ \E
%          navacakraṃ kalādhāraṃ trilakṣyaṃ vyomapaṃcakaṃ  \P
%śloka     navacakraṃ kalādhāraṃ trilakṣaṃ  vyomapaṃcakam/ \B
%//śloka// navacakraṃ kalādhāraṃ trilakṣaṃ  vyomapaṃcakam... \L %%%%%%%%%%%%GREP THIS%%%%%%%%%%%%% SSP 2.31!!!
%          navacakra--kalādhāraṃ trilakṣyaṃ vyomapaṃcakaṃ/ \N1
%          navacakra--kalādhāraṃ trilakṣyaṃ vyomapaṃcakaṃ// \D
%          navacakra--kalādhāraṃ trilakṣaṃ  vyomapaṃcakaṃ/ \N2
%          navacakraṃ kalādhāraṃ trilakṣyaṃ vyomapaṃcakaṃ \U1 %%%282.jpg
%          navacakraṃ kalādhāraṃ trilakṣyaṃ vyomapaṃcakaṃ// \U2
%-----------------------------
%The nine cakras, the sixteen Adhāras, the three lakṣyas and die five spaces. 
%-----------------------------
\tl{
\app{\lem[wit={ceteri}]{navacakraṃ}
  \rdg[wit={B,L}]{śloka navacakraṃ}
  \rdg[wit={D,N1,N2}]{navacakra°}}
kalādhāraṃ
\note[type=source, labelb=197, lem={\textbf{Ri}}]{\citetitle{netratantra} 7.1: ataḥ paraṃ pravakṣyāmi dhyānaṃ sūkṣmam anuttamam | ṛtucakraṃ svarādhāraṃ trilakṣyaṃ vyomapañcakam |}
\note[type=testium, labelb=197a, lem={\textbf{Ri}}]{SSP 2.31 (Ed. p. 43): navacakraṃ kalādhāraṃ trilakṣyaṃ vyomapañcakam | samyag etan na jānāti sa yogī nāmadhārakaḥ |}
\note[type=testium, labelb=198, lem={\textbf{Re}}]{PT\textsuperscript{qcr \cdot YSV} (Ed. p. 832): navacakraṃ kalādhāraṃ trilakṣaṃ vyomapañcakam | svadehe yo na jānāti sa yogī nāmadhārakaḥ |}
\note[type=testium, labelb=198a, lem={\textbf{Re}}]{PT\textsuperscript{qcr \cdot YSV} (Ed. p. 839): navacakraṃ kalādhāraṃ dvilakṣaṃ vyomapañcakam | samagraṃ yo na jānāti sa yogī nāmadhārakaḥ |}
\app{\lem[wit={ceteri}, alt={°kṣyaṃ}]{trilakṣyaṃ}
    \rdg[wit={B,L,N2}]{trilakṣaṃ}}
  vyomapaṃcakaṃ/}\\
%-----------------------------
%svadehe yo na jānāti sa yogī nāmadhārakaḥ//       \E
%svadehe yo na jānāti sa yogī nāmadhārakaḥ 1       \P
%svadehe yo na jānāti sa yogī nāmadhārakaḥ//1//    \B
%svadehe yo na jānāti sa yogī nāmadhārakaḥ//1//   \L
%samakriyā  na jānāti sa yogī nāmadhāraka//           \N1
%samakriyā  na jānāti sa yogī nāmadhārakaḥ//           \D
%samakriyā  na jānāti sa yogī nāmadhāraka//           \N2
%samakriyā  na jānāti sa yogī nāmadhārakaḥ            \U1
%svadehe yo na jānāti sa yogī nāmadhārakaḥ        \U2
%-----------------------------
%Who does not know [them?] within ones own body, he is only a Yogin by name. 
%-----------------------------
\tl{\app{\lem[wit={ceteri}]{svadehe yo}
  \rdg[wit={D,N1,N2,U1}]{samakriyā}} 
na jānāti sa yogī nāmadhārakaḥ\dd{}}\textsuperscript{\begin{otherlanguage}{english}\coro{[\lowroman{12}]}\end{otherlanguage}}
\end{tlg}
  \end{edition}
  \begin{translation}
    \ekddiv{type=trans}
    \begin{tlate}
\noindent
After that the fixation of the sun-space (\textit{sūryakāśa}) which is associated with sundisk's appearance of light shall be done internally and externally. From the execution of these fixations contact of diseases does not arise within the body. Thus wrinkles and grey hair, sin or merit do not arise.
 
\paragraph{\uproman{27}.\textsuperscript{\lowroman{12}}.} The nine Cakras, the sixteen Adhāras, the three Lakṣyas and die five spaces. Who does not know [them?] within ones own body, he is only a Yogin by name.
    \end{tlate}
  \end{translation}
\end{alignment}


\chapter{Bibliography}
 \label{sec:bibli}
   \clearpage
\newpage 
\thispagestyle{empty}
\quad  \addtocounter{page}{-1}

\printbibliography[heading=subbibintoc, title=Consulted Manuskripts, keyword=codex]

\printbibliography[heading=subbibintoc, title=Printed Editions, keyword=printsource]

\printbibliography[heading=subbibintoc, title=Secondary Literature, keyword=seclit]

\printbibliography[heading=subbibintoc, title=Online Sources, keyword=onlinesource]

\end{document}




