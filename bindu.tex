%Ultimatives Tool zur Datierung:
%https://www.cc.kyoto-su.ac.jp/~yanom/pancanga/
%skp = ignored in edition
%skm = ignored in xml
\input{preamble.tex}
\FormatDiv{1}{\begin{center}\Large}{\end{center}}
\FormatDiv{2}{\begin{center}\small}{\end{center}}
\FormatDiv{3}{\bfseries}{.}
\title{Tattvayogabindu of Rāmacandra\\ A Critical Edition and Annotated Translation\\ and a Comparative Analysis of the \\Complex Early Modern Yoga Yaxonomies }
\date{\today}
\parindent=15pt

\begin{document}

\frontmatter
\thispagestyle{empty} % Verhindert Seitenzahl auf der Seite
\begin{center}

%\vspace{0.5in}

%\begin{otherlanguage}{iast}
%   \large\sanskritfont{Tattvayogabindu}\\
%\end{otherlanguage}

\vspace{0.25in}


\huge\textbf{\MakeUppercase{The Tattvayogabindu \\of Rāmacandra}}\\

\vspace{0.2in}

\Large  Critical Edition and Annotated Translation of an Early Modern Text on Rājayoga, with a Comparative Analysis of the Complex Yoga Taxonomies from the Same Period\\ 

\vspace{0.45in}

\thispagestyle{empty}
\end{center}
%\newpage
%\thispagestyle{empty}
%\mbox{}
%\newpage

\newpage

  \thispagestyle{empty}
  \begin{figure}[p]
    \centering
    \includegraphics[width=0.25\textwidth]{pics/purna.jpg}
  \end{figure}
  
\newpage

\begin{landscape}
\thispagestyle{empty}
  \begin{figure}[p]
	\centering
  \includegraphics[width=1.5\textwidth]{pics/folio1.jpg}
	\caption{Folio 1v of Ms. \getsiglum{N1}.}
	 \phantomsection\label{fig_folio1}
\end{figure}
\end{landscape}

\cleardoublepage
\tableofcontents
\thispagestyle{empty}
\newpage 
\listoffigures
\thispagestyle{empty}
\newpage
\listoftables
\thispagestyle{empty}
\newpage

\mainmatter
\pagestyle{defaultstyle}
\counterwithout{footnote}{chapter}
\counterwithout{figure}{chapter}
\counterwithout{table}{chapter}
\renewcommand{\thetable}{\arabic{table}}
%%%tables 
\setsecnumdepth{section}
\maxsecnumdepth{subsubsection}
\newpage
\chapter{Introduction}
\cleardoublepage

\section{General remarks}
 \phantomsection\label{generalremarks}
 \lettrine{T}{he} \textit{Tattvayogabindu} of Rāmacandra\footnote{A discussion about the author Rāmacandra is found on p. \pageref{ramarama}.} is an early modern Sanskrit text on Rājayoga that was written in the first half of the seventeenth century\footnote{The dating of the text is discussed on p. \pageref{dating}.} in northern India.\footnote{The detailed discussion of the place of origin is found on p. \pageref{riversrivers}, n. \ref{riversrivers}.} The most salient feature of the work that makes it historically significant is its highly differentiated taxonomy of types of yoga.\footnote{This is a remarkable increase in the number of declared yogas compared to the standard medieval tetrad of Mantra, Laya, Haṭha and Rājayoga.} In the \textit{Tattvayogabindu}'s introduction, most manuscripts name fifteen types of yoga, presented as methods of Rājayoga. These are 1. Kriyāyoga, 2. Jñānayoga, 3. Caryāyoga, 4. Haṭhayoga, 5. Karmayoga, 6. Layayoga, 7. Dhyānayoga, 8. Mantrayoga, 9. Lakṣyayoga, 10. Vāsanāyoga, 11. Śivayoga, 12. Brahmayoga, 13. Advaitayoga, 14. Siddhayoga, and 15. Rājayoga itself. The text is a yogic compendium written in a mix of mainly prose and 47 verses in textbook-style, where its 59 topics are introduced in sections most of the time launched by recognizable phrases. The sections deal with the methods of Rājayoga and their effects, but others also cover topics like yogic physiology, the Avadhūta, the importance of the guru, cosmogony, and a \textit{yogaśāstrarahasya}.  

The \textit{Tattvayogabindu} has not been discussed comprehensively or considered in the secondary literature on yoga. The only exception is \citeauthor{birch2014} (2014: 415–416) who briefly described its list of fifteen yogas in the context of the ``fifteen medieval yogas'' and noted that a similar taxonomy occurs in Nārāyaṇatīrtha’s \textit{Yogasiddhāntacandrikā} (17th century), a commentary on the \textit{Pātañjalayogaśāstra} that integrates fifteen medieval yogas within its \textit{aṣṭāṅga} format. An incomplete account of the fifteen yogas is found within the Sanskrit yoga text \textit{Yogasvarodaya}, which is known only through quotations in the \textit{Prāṇatoṣinī}, the \textit{Yogakarṇikā} and the \emph{Śabdakalpadruma}.\footnote{Manuscripts under the name of \textit{Yogasvarodaya} seem to be lost. I was not able to locate the manuscripts of the text in any manuscript catalogue at hand.} The \textit{Yogasvarodaya} announces a total of fifteen yogas but names only eight of them in its introductory \textit{śloka}s. It is the primary source and template for the compilation of the \textit{Tattvayogabindu}. Besides several passages, Rāmacandra, in many instances, follows its content and structure by rewriting the \textit{Yogasvarodaya}’s \textit{śloka}s into prose or quoting them directly without attribution. Due to the incomplete transmission of the \textit{Yogasvarodaya}, Rāmacandra’s \textit{Tattvayogabindu} is a natural and valuable starting point for an unprecedented in-depth study of the complex early modern yoga taxonomies, a phenomenon that can be narrowed down precisely in terms of time and as I will show regarding its localisation. The other source text that Rāmacandra used is the \textit{Siddhasiddhāntapaddhati} whose content he draws on, particularly in the second half of his composition. Another text that includes an almost similar taxonomy of twelve yogas divided into three tetrads\footnote{See p.\pageref{sarvasarva} for a detailed discussion of the \textit{Sarvāṅgayogapradīpikā}.} is Sundardās’s \textit{Brajbhāṣā} yoga text named \textit{Sarvāṅgayogapradīpikā} which not just shares most of the types of yogas but also provides a different and valuable perspective on the addressed yoga categories.\footnote{For a comparative table of the complex early modern yoga taxonomies see table \ref{tab:complextaxonomies} on p. \pageref{tab:complextaxonomies}.}

These complex taxonomies that emerged during the 17th century crossed sectarian divides and were adapted to the specific needs of different authors and traditions. The \textit{Tattvayogabindu} thus encapsulates a large proportion of the diversity of yoga types and teachings after the \textit{Haṭhapradīpikā} (15th century) that were adopted and practised by a broad spectrum of religious traditions and strata of Indian society. In the particular case of the \textit{Tattvayogabindu}, there are various statements throughout the text that reveal a strategy to detach yoga from its ascetic and renunciate connotations and to stylise Rājayoga as a practice that can bring the desired soteriological benefits even to practitioners who enjoy worldly pleasures and expensive lifestyles. Textual evidence suggests that the \textit{Tattvayogabindu} is an important example of a text that provides an early modern adaptation of Rājayoga for \textit{kṣatriya}s in a courtly environment.

One printed edition of the \textit{Tattvayogabindu} was published in 1905 with a Hindi translation and based on (an) unknown manuscript(s).\footnote{\emph{Binduyoga}. \textit{Binduyogaḥ with Bhāṣaṭīkā}. Ed. by Jvālāprasāda Miśra. Mumbai, 1905.} This publication has the title ``\textit{Binduyoga}'' confirmed by the printed text’s colophon. However, as I will discuss in the introduction, the text was originally known as \textit{Tattvayogabindu}. The consulted manuscripts contain significant discrepancies, structural differences and variant readings between them and the printed edition.\footnote{For example, the printed edition does not contain the complex yoga taxonomy presented in the manuscripts of the \emph{Tattvayogabindu}.} Furthermore, the manuscripts are scattered over the northern half of the Indian subcontinent and Nepal, which suggests that the text was widely transmitted at some point. Lengthy passages of the \textit{Tattvayogabindu} are quoted without attribution in a text called \textit{Yogasaṃgraha} and Sundaradeva’s \textit{Haṭhasaṅketacandrikā}.

The first chapter of this dissertation contains a general introduction to Rāmacandra's \textit{Tattvayogabindu}. The chapter gives a brief overview of the content of the text and discusses its origin, the author and the author's intended audience. Subsequently, the textual witnesses, source texts and testimonies of the \textit{Tattvayogabindu} are described. A stemmatic analysis of the text is then presented, based on manual philological observation and computer-assisted stemmatics to present a \textit{stemma codicum}. The chapter concludes with a presentation of the editorial policies, which form the basis for the second chapter of this thesis.
The second chapter, the core of this dissertation, is a critical edition and annotated translation of the \textit{Tattvayogabindu}. The critical edition significantly improves the text and sheds new light on its historical significance.
The third chapter contains a comparative analysis of the complex early modern yoga taxonomies based on hermeneutics of difference.\footnote{The conceptof hermeneutics of difference is discussed on p. \pageref{hermeneutics}, n. \ref{hemerneutics}.}  Using the new critical edition of the \textit{Tattvayogabindu} and the texts mentioned above, \emph{Yogasvarodaya}, \emph{Yogasiddhāntacandrikā} and \emph{Sarvāṅgayogapradīpikā}, the complex yogic taxonomies of the four texts are compared in detail. Based on this comparative analysis, a differentiated hypothesis on the emergence of the complex yoga taxonomies was developed, and the complex yoga taxonomies were located und explained in the broader context of the historical development of the yoga traditions. The comparison includes a nuanced description of each yoga category used by the authors of the texts with complex yoga taxonomies. While the authors of the four texts often operate with identical terms for the individual yoga categories, they interpret these categories according to their religious backgrounds and agendas, with intriguing and exciting differences. Contrasting the comparanda, i.e. the authors, the texts, the yoga taxonomies and the yoga categories, therefore provides a deep insight into the discursive negotiation processes of the Indian yoga traditions of the 17th century.


\chapter{Conventions in the Critical Apparatus}
\section{Sigla in the Critical Apparatus}

\begin{itemize}
\item \beta : \getsiglum{D}, \getsiglum{J}, \getsiglum{K1}, \getsiglum{N1}, \getsiglum{N2}, \getsiglum{U1}
\item \gamma : \getsiglum{B}, \getsiglum{E}, \getsiglum{L}, \getsiglum{P}, \getsiglum{U2}
\item B : Bodleian Oxford D 4587
\item C : \emph{Haṭhasaṅketacandrikā} GOML Ms. No. R 3239
\item C\textsubscript{pc} : \emph{Haṭhasaṅketacandrikā} GOML Ms. No. R 3239
\item cett.: ceteri (all manuscripts except the ones mentioned in the lemma)
\item \Done : IGNCA 30019
\item E : Printed Edition
\item J : JNUL Ms. No. 55769
\item Jo : \emph{Haṭhasaṅketacandrikā} MMPP MS. No. 2244
\item \Kone : AS G 11019
\item L : Lalchand Research Library LRL5876
\item M : \emph{Haṭhasaṅketacandrikā} ORI Ms. No. B 220
\item \Ntwo : NGMPP B 38-35 / A 1327-14
\item \None : NGMPP B 38-31
\item P : Pune BORI 664
\item PT : \emph{Prāṇatoṣiṇī}
\item \Uone : SORI 1574
\item \Utwo : SORI 6082
\item V : OI MSU 10558
\item YK : \emph{Yogakarṇikā}% 
\item YSv : \emph{Yogasvarodaya}
\end{itemize}
\newpage

\chapter[Critical Edition \& Annotated Translation of the \emph{Tattvayogabindu}]{The \emph{Tattvayogabindu} of Rāmacandra \\ \huge  
  Critical Edition \& Annotated Translation}
\pagestyle{chapter2style}
\cleardoublepage
 
\newpage
\cleardoublepage
\selectlanguage{english}
\chapter{Appendix}
\section{Figures}

% \begin{landscape}
\clearpage

  \begin{figure}[ht]
	\centering
  \includegraphics[width=1\textwidth]{pics/Wolpertinger.png}
\caption[The \textit{dehasvarūpa} of \textit{ajapāgāyatrī}]{The \textit{dehasvarūpa} of \textit{ajapāgāyatrī}. The image, reminiscent of a hippogriff, is part of an illustrated Sanskrit manuscript written in the Śāradā script. Preserved as a single large scroll under Acc. No. 1334 at the Oriental Institute in Srinagar (Kashmir), it is entitled \textit{Nāḍīcakra}. The manuscript contains a depiction of the yogic body’s \textit{cakra}s and \textit{nāḍī}s. The text surrounding the figure closely corresponds to the additional material found in manuscript \getsiglum{U2} of the \textit{Tattvayogabindu}. The manuscript reads (diplomatic transcription): \textit{oṃ daśame pūrṇagiripīṭhe lalāṭamaṇḍale candro devatā amṛtāśaktiḥ paramātmā ṛṣiḥ dvāviṃśaddalāni amṛtavāsinikalā 4: ambikā 1 lambikā 2 gha(ṃ)ṭkā 3 tālikā 4 dehasvarūpaṃ kākamukhaṃ 1 naranetraṃ 2 gośṛṅgaṃ 3 lalāṭabrahmapara 4 hayagrīvā 5 mayūramuśchaṃ 6 haṃsacārītani 7 sthāna.}}
	\phantomsection\label{fig_wolpertinger}
      \end{figure}

      \clearpage

  \begin{figure}[ht]
	\centering
  \includegraphics[width=1\textwidth]{pics/Vishnu_Vishvarupa_cropped.jpg}
	\caption{Viṣṇu Viśvarūpa, India, Rajasthan, Jaipur, ca. 1800–1820, Opaque watercolor and gold on paper, 38.5 × 28 cm, Victoria and Albert Museum, London, Given by Mrs. Gerald Clark.}
	\label{fig1}
      \end{figure}
\clearpage
  \begin{figure}[ht]
	\centering
  \includegraphics[width=0.5\textwidth]{pics/The_Equivalence_of_Self_and_Universe_(detail),_folio_6_from_the_Siddha_Siddhanta_Paddhati,_(Bulaki),_1824_(Samvat_1881);_122_x_46_cm._Mehrangarh_Museum_Trust..jpg}
	\caption{The Equivalence of Self and Universe (detail), folio 6 from the \textit{Siddhasiddhāntapaddhati} (Bulaki), India, Rajasthan, Jodhpur, 1824 (Samvat 1881), 122 x 46 cm, RJS 2378, Mehragarh Museum Trust.}
	\label{fig2}
      \end{figure}
      % \end{landscape}

      \newpage
      \cleardoublepage
\chapter{Bibliography}
 \label{sec:bibli}
\clearpage
\newpage 
\thispagestyle{empty}
\quad  \addtocounter{page}{-1}

\newrefcontext[sorting=tixel]
\printbibliography[heading=subbibintoc, title=Primary Sources, keyword=primary]

\newrefcontext[sorting=nyt]
\printbibliography[heading=subbibintoc, title=Secondary Literature, keyword=seclit]

\printbibliography[heading=subbibintoc, title=Catalogues, keyword=catalogues]

\printbibliography[heading=subbibintoc, title=Online Sources, keyword=onlinesource]

\end{document}
