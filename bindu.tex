%Ultimatives Tool zur Datierung:
%https://www.cc.kyoto-su.ac.jp/~yanom/pancanga/
%skp = ignored in edition
%skm = ignored in xml
\input{preamble.tex}
\FormatDiv{1}{\begin{center}\Large}{\end{center}}
\FormatDiv{2}{\begin{center}\small}{\end{center}}
\FormatDiv{3}{\bfseries}{.}
\title{Tattvayogabindu of Rāmacandra\\ A Critical Edition and Annotated Translation\\ and a Comparative Analysis of the \\Complex Early Modern Yoga Yaxonomies }
\date{\today}
\parindent=15pt

\begin{document}

\frontmatter
\thispagestyle{empty} % Verhindert Seitenzahl auf der Seite
\begin{center}

%\vspace{0.5in}

%\begin{otherlanguage}{iast}
%   \large\sanskritfont{Tattvayogabindu}\\
%\end{otherlanguage}

\vspace{0.25in}


\huge\textbf{\MakeUppercase{The Tattvayogabindu \\of Rāmacandra}}\\

\vspace{0.2in}

\Large  Critical Edition and Annotated Translation of an Early Modern Text on Rājayoga, with a Comparative Analysis of the Complex Yoga Taxonomies from the Same Period\\ 

\vspace{0.45in}

\thispagestyle{empty}
\end{center}
%\newpage
%\thispagestyle{empty}
%\mbox{}
%\newpage

\newpage

  \thispagestyle{empty}
  \begin{figure}[p]
    \centering
    \includegraphics[width=0.25\textwidth]{pics/purna.jpg}
  \end{figure}
  
\newpage

\begin{landscape}
\thispagestyle{empty}
  \begin{figure}[p]
	\centering
  \includegraphics[width=1.5\textwidth]{pics/folio1.jpg}
	\caption{Folio 1v of Ms. \getsiglum{N1}.}
	 \phantomsection\label{fig_folio1}
\end{figure}
\end{landscape}

\cleardoublepage
\tableofcontents
\thispagestyle{empty}
\newpage 
\listoffigures
\thispagestyle{empty}
\newpage
\listoftables
\thispagestyle{empty}
\newpage

\mainmatter
\pagestyle{defaultstyle}
\counterwithout{footnote}{chapter}
\counterwithout{figure}{chapter}
\counterwithout{table}{chapter}
\renewcommand{\thetable}{\arabic{table}}
%%%tables 
\setsecnumdepth{section}
\maxsecnumdepth{subsubsection}
\newpage
\chapter{Introduction}
\cleardoublepage




\chapter[Critical Edition \& Annotated Translation of the \emph{Tattvayogabindu}]{The \emph{Tattvayogabindu} of Rāmacandra \\ \huge  
  Critical Edition \& Annotated Translation}
\pagestyle{chapter2style}
\newpage

\newpage
\selectlanguage{english}



\chapter{Appendix}
\section{Figures}
 
% \begin{landscape}
\clearpage

  \begin{figure}[ht]
	\centering
  \includegraphics[width=1\textwidth]{pics/Wolpertinger.png}
\caption[The \textit{dehasvarūpa} of \textit{ajapāgāyatrī}]{The \textit{dehasvarūpa} of \textit{ajapāgāyatrī}. The image, reminiscent of a hippogriff, is part of an illustrated Sanskrit manuscript written in the Śāradā script. Preserved as a single large scroll under Acc. No. 1334 at the Oriental Institute in Srinagar (Kashmir), it is entitled \textit{Nāḍīcakra}. The manuscript contains a depiction of the yogic body’s \textit{cakra}s and \textit{nāḍī}s. The text surrounding the figure closely corresponds to the additional material found in manuscript \getsiglum{U2} of the \textit{Tattvayogabindu}. The manuscript reads (diplomatic transcription): \textit{oṃ daśame pūrṇagiripīṭhe lalāṭamaṇḍale candro devatā amṛtāśaktiḥ paramātmā ṛṣiḥ dvāviṃśaddalāni amṛtavāsinikalā 4: ambikā 1 lambikā 2 gha(ṃ)ṭkā 3 tālikā 4 dehasvarūpaṃ kākamukhaṃ 1 naranetraṃ 2 gośṛṅgaṃ 3 lalāṭabrahmapara 4 hayagrīvā 5 mayūramuśchaṃ 6 haṃsacārītani 7 sthāna.}}
	\phantomsection\label{fig_wolpertinger}
      \end{figure}

      \clearpage

  \begin{figure}[ht]
	\centering
  \includegraphics[width=1\textwidth]{pics/Vishnu_Vishvarupa_cropped.jpg}
	\caption{Viṣṇu Viśvarūpa, India, Rajasthan, Jaipur, ca. 1800–1820, Opaque watercolor and gold on paper, 38.5 × 28 cm, Victoria and Albert Museum, London, Given by Mrs. Gerald Clark.}
	\label{fig1}
      \end{figure}
\clearpage
  \begin{figure}[ht]
	\centering
  \includegraphics[width=0.5\textwidth]{pics/The_Equivalence_of_Self_and_Universe_(detail),_folio_6_from_the_Siddha_Siddhanta_Paddhati,_(Bulaki),_1824_(Samvat_1881);_122_x_46_cm._Mehrangarh_Museum_Trust..jpg}
	\caption{The Equivalence of Self and Universe (detail), folio 6 from the \textit{Siddhasiddhāntapaddhati} (Bulaki), India, Rajasthan, Jodhpur, 1824 (Samvat 1881), 122 x 46 cm, RJS 2378, Mehragarh Museum Trust.}
	\label{fig2}
      \end{figure}
      % \end{landscape}

      \newpage
      \cleardoublepage
\chapter{Bibliography}
 \label{sec:bibli}
\clearpage
\newpage 
\thispagestyle{empty}
\quad  \addtocounter{page}{-1}

\newrefcontext[sorting=tixel]
\printbibliography[heading=subbibintoc, title=Primary Sources, keyword=primary]

\newrefcontext[sorting=nyt]
\printbibliography[heading=subbibintoc, title=Secondary Literature, keyword=seclit]

\printbibliography[heading=subbibintoc, title=Catalogues, keyword=catalogues]

\printbibliography[heading=subbibintoc, title=Online Sources, keyword=onlinesource]

\end{document}


%%% Local Variables:
%%% mode: latex
%%% TeX-master: t
%%% End:
