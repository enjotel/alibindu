%Ultimatives Tool zur Datierung:
%https://www.cc.kyoto-su.ac.jp/~yanom/pancanga/
%skp = ignored in edition
%skm = ignored in xml
\input{preamble.tex}
\FormatDiv{1}{\begin{center}\Large}{\end{center}}
\FormatDiv{2}{\begin{center}\small}{\end{center}}
\FormatDiv{3}{\bfseries}{.}
\title{Tattvayogabindu of Rāmacandra\\ A Critical Edition and Annotated Translation\\ and a Comparative Analysis of the \\Complex Early Modern Yoga Yaxonomies }
\date{\today}
\parindent=15pt

\begin{document}

\frontmatter
\thispagestyle{empty} % Verhindert Seitenzahl auf der Seite
\begin{center}

%\vspace{0.5in}

%\begin{otherlanguage}{iast}
%   \large\sanskritfont{Tattvayogabindu}\\
%\end{otherlanguage}

\vspace{0.25in}


\huge\textbf{\MakeUppercase{The Tattvayogabindu \\of Rāmacandra}}\\

\vspace{0.2in}

\Large  Critical Edition and Annotated Translation of an Early Modern Text on Rājayoga, with a Comparative Analysis of the Complex Yoga Taxonomies from the Same Period\\ 

\vspace{0.45in}

\thispagestyle{empty}
\end{center}
%\newpage
%\thispagestyle{empty}
%\mbox{}
%\newpage

\newpage

  \thispagestyle{empty}
  \begin{figure}[p]
    \centering
    \includegraphics[width=0.25\textwidth]{pics/purna.jpg}
  \end{figure}
  
\newpage

\begin{landscape}
\thispagestyle{empty}
  \begin{figure}[p]
	\centering
  \includegraphics[width=1.5\textwidth]{pics/folio1.jpg}
	\caption{Folio 1v of Ms. \getsiglum{N1}.}
	 \phantomsection\label{fig_folio1}
\end{figure}
\end{landscape}

\cleardoublepage
\tableofcontents
\thispagestyle{empty}
\newpage 
\listoffigures
\thispagestyle{empty}
\newpage
\listoftables
\thispagestyle{empty}
\newpage

\mainmatter
\pagestyle{defaultstyle}
\counterwithout{footnote}{chapter}
\counterwithout{figure}{chapter}
\counterwithout{table}{chapter}
\renewcommand{\thetable}{\arabic{table}}
%%%tables 
\setsecnumdepth{section}
\maxsecnumdepth{subsubsection}
\newpage
\chapter{Introduction}
\cleardoublepage

\section{General remarks}
 \phantomsection\label{generalremarks}
 \lettrine{T}{he} \textit{Tattvayogabindu} of Rāmacandra\footnote{A discussion about the author Rāmacandra is found on p. \pageref{ramarama}.} is an early modern Sanskrit text on Rājayoga that was written in the first half of the seventeenth century\footnote{The dating of the text is discussed on p. \pageref{dating}.} in northern India.\footnote{The detailed discussion of the place of origin is found on p. \pageref{riversrivers}, n. \ref{riversrivers}.} The most salient feature of the work that makes it historically significant is its highly differentiated taxonomy of types of yoga.\footnote{This is a remarkable increase in the number of declared yogas compared to the standard medieval tetrad of Mantra, Laya, Haṭha and Rājayoga.} In the \textit{Tattvayogabindu}'s introduction, most manuscripts name fifteen types of yoga, presented as methods of Rājayoga. These are 1. Kriyāyoga, 2. Jñānayoga, 3. Caryāyoga, 4. Haṭhayoga, 5. Karmayoga, 6. Layayoga, 7. Dhyānayoga, 8. Mantrayoga, 9. Lakṣyayoga, 10. Vāsanāyoga, 11. Śivayoga, 12. Brahmayoga, 13. Advaitayoga, 14. Siddhayoga, and 15. Rājayoga itself. The text is a yogic compendium written in a mix of mainly prose and 47 verses in textbook-style, where its 59 topics are introduced in sections most of the time launched by recognizable phrases. The sections deal with the methods of Rājayoga and their effects, but others also cover topics like yogic physiology, the Avadhūta, the importance of the guru, cosmogony, and a \textit{yogaśāstrarahasya}.  

The \textit{Tattvayogabindu} has not been discussed comprehensively or considered in the secondary literature on yoga. The only exception is \citeauthor{birch2014} (2014: 415–416) who briefly described its list of fifteen yogas in the context of the ``fifteen medieval yogas'' and noted that a similar taxonomy occurs in Nārāyaṇatīrtha’s \textit{Yogasiddhāntacandrikā} (17th century), a commentary on the \textit{Pātañjalayogaśāstra} that integrates fifteen medieval yogas within its \textit{aṣṭāṅga} format. An incomplete account of the fifteen yogas is found within the Sanskrit yoga text \textit{Yogasvarodaya}, which is known only through quotations in the \textit{Prāṇatoṣinī}, the \textit{Yogakarṇikā} and the \emph{Śabdakalpadruma}.\footnote{Manuscripts under the name of \textit{Yogasvarodaya} seem to be lost. I was not able to locate the manuscripts of the text in any manuscript catalogue at hand.} The \textit{Yogasvarodaya} announces a total of fifteen yogas but names only eight of them in its introductory \textit{śloka}s. It is the primary source and template for the compilation of the \textit{Tattvayogabindu}. Besides several passages, Rāmacandra, in many instances, follows its content and structure by rewriting the \textit{Yogasvarodaya}’s \textit{śloka}s into prose or quoting them directly without attribution. Due to the incomplete transmission of the \textit{Yogasvarodaya}, Rāmacandra’s \textit{Tattvayogabindu} is a natural and valuable starting point for an unprecedented in-depth study of the complex early modern yoga taxonomies, a phenomenon that can be narrowed down precisely in terms of time and as I will show regarding its localisation. The other source text that Rāmacandra used is the \textit{Siddhasiddhāntapaddhati} whose content he draws on, particularly in the second half of his composition. Another text that includes an almost similar taxonomy of twelve yogas divided into three tetrads\footnote{See p.\pageref{sarvasarva} for a detailed discussion of the \textit{Sarvāṅgayogapradīpikā}.} is Sundardās’s \textit{Brajbhāṣā} yoga text named \textit{Sarvāṅgayogapradīpikā} which not just shares most of the types of yogas but also provides a different and valuable perspective on the addressed yoga categories.\footnote{For a comparative table of the complex early modern yoga taxonomies see table \ref{tab:complextaxonomies} on p. \pageref{tab:complextaxonomies}.}

These complex taxonomies that emerged during the 17th century crossed sectarian divides and were adapted to the specific needs of different authors and traditions. The \textit{Tattvayogabindu} thus encapsulates a large proportion of the diversity of yoga types and teachings after the \textit{Haṭhapradīpikā} (15th century) that were adopted and practised by a broad spectrum of religious traditions and strata of Indian society. In the particular case of the \textit{Tattvayogabindu}, there are various statements throughout the text that reveal a strategy to detach yoga from its ascetic and renunciate connotations and to stylise Rājayoga as a practice that can bring the desired soteriological benefits even to practitioners who enjoy worldly pleasures and expensive lifestyles. Textual evidence suggests that the \textit{Tattvayogabindu} is an important example of a text that provides an early modern adaptation of Rājayoga for \textit{kṣatriya}s in a courtly environment.

One printed edition of the \textit{Tattvayogabindu} was published in 1905 with a Hindi translation and based on (an) unknown manuscript(s).\footnote{\emph{Binduyoga}. \textit{Binduyogaḥ with Bhāṣaṭīkā}. Ed. by Jvālāprasāda Miśra. Mumbai, 1905.} This publication has the title ``\textit{Binduyoga}'' confirmed by the printed text’s colophon. However, as I will discuss in the introduction, the text was originally known as \textit{Tattvayogabindu}. The consulted manuscripts contain significant discrepancies, structural differences and variant readings between them and the printed edition.\footnote{For example, the printed edition does not contain the complex yoga taxonomy presented in the manuscripts of the \emph{Tattvayogabindu}.} Furthermore, the manuscripts are scattered over the northern half of the Indian subcontinent and Nepal, which suggests that the text was widely transmitted at some point. Lengthy passages of the \textit{Tattvayogabindu} are quoted without attribution in a text called \textit{Yogasaṃgraha} and Sundaradeva’s \textit{Haṭhasaṅketacandrikā}.

The first chapter of this dissertation contains a general introduction to Rāmacandra's \textit{Tattvayogabindu}. The chapter gives a brief overview of the content of the text and discusses its origin, the author and the author's intended audience. Subsequently, the textual witnesses, source texts and testimonies of the \textit{Tattvayogabindu} are described. A stemmatic analysis of the text is then presented, based on manual philological observation and computer-assisted stemmatics to present a \textit{stemma codicum}. The chapter concludes with a presentation of the editorial policies, which form the basis for the second chapter of this thesis.
The second chapter, the core of this dissertation, is a critical edition and annotated translation of the \textit{Tattvayogabindu}. The critical edition significantly improves the text and sheds new light on its historical significance.
The third chapter contains a comparative analysis of the complex early modern yoga taxonomies based on hermeneutics of difference.\footnote{The conceptof hermeneutics of difference is discussed on p. \pageref{hermeneutics}, n. \ref{hemerneutics}.}  Using the new critical edition of the \textit{Tattvayogabindu} and the texts mentioned above, \emph{Yogasvarodaya}, \emph{Yogasiddhāntacandrikā} and \emph{Sarvāṅgayogapradīpikā}, the complex yogic taxonomies of the four texts are compared in detail. Based on this comparative analysis, a differentiated hypothesis on the emergence of the complex yoga taxonomies was developed, and the complex yoga taxonomies were located und explained in the broader context of the historical development of the yoga traditions. The comparison includes a nuanced description of each yoga category used by the authors of the texts with complex yoga taxonomies. While the authors of the four texts often operate with identical terms for the individual yoga categories, they interpret these categories according to their religious backgrounds and agendas, with intriguing and exciting differences. Contrasting the comparanda, i.e. the authors, the texts, the yoga taxonomies and the yoga categories, therefore provides a deep insight into the discursive negotiation processes of the Indian yoga traditions of the 17th century.


\chapter{Conventions in the Critical Apparatus}
\section{Sigla in the Critical Apparatus}

\begin{itemize}
\item \beta : \getsiglum{D}, \getsiglum{J}, \getsiglum{K1}, \getsiglum{N1}, \getsiglum{N2}, \getsiglum{U1}
\item \gamma : \getsiglum{B}, \getsiglum{E}, \getsiglum{L}, \getsiglum{P}, \getsiglum{U2}
\item B : Bodleian Oxford D 4587
\item C : \emph{Haṭhasaṅketacandrikā} GOML Ms. No. R 3239
\item C\textsubscript{pc} : \emph{Haṭhasaṅketacandrikā} GOML Ms. No. R 3239
\item cett.: ceteri (all manuscripts except the ones mentioned in the lemma)
\item \Done : IGNCA 30019
\item E : Printed Edition
\item J : JNUL Ms. No. 55769
\item Jo : \emph{Haṭhasaṅketacandrikā} MMPP MS. No. 2244
\item \Kone : AS G 11019
\item L : Lalchand Research Library LRL5876
\item M : \emph{Haṭhasaṅketacandrikā} ORI Ms. No. B 220
\item \Ntwo : NGMPP B 38-35 / A 1327-14
\item \None : NGMPP B 38-31
\item P : Pune BORI 664
\item PT : \emph{Prāṇatoṣiṇī}
\item \Uone : SORI 1574
\item \Utwo : SORI 6082
\item V : OI MSU 10558
\item YK : \emph{Yogakarṇikā}% 
\item YSv : \emph{Yogasvarodaya}
\end{itemize}
\newpage

\chapter[Critical Edition \& Annotated Translation of the \emph{Tattvayogabindu}]{The \emph{Tattvayogabindu} of Rāmacandra \\ \huge  
  Critical Edition \& Annotated Translation}
\pagestyle{chapter2style}
\newpage
\begin{alignment}[
  texts=edition[class="edition"];
  translation[class="translation"],
]
\begin{edition}
  \ekddiv{type=ed}
    \begin{prose}[p22_04]
      %------------------------------
      %paṭṭasūtramayāni vasrāṇi//  \E
      %padasūtramayāni vastrāṇi?? \P %%%7643.jpg
      %paṭasūtrāmayāni vasrāṇi//  \B
      %paṭasūtrāmayāni vastrāṇi// \L
      %paṭṭasūtrayāni   vasrāṇi    \N1
      %paṭṭasūtrayāni   vasrāṇi    \D
      %paṭṭasūtrayāni   vasrāṇi    \N2
      %padasūtramayāni vasrāṇi    \U1
      %paṭasūtramayāni vasrāṇi    \U2
      %------------------------------
      %Clothes made from silk,...
      %------------------------------
      %paṃcasaptā dṛālikā         yuktāni harmyāṇi teṣu vāsaḥ    ativipulā  mṛdutarasukhāsuśayyā/     \E
      %paṃcasaptā dadhikā         yuktāni harmyāṇi teṣu cāsaḥ 2  ativipulā  mṛduttarachadavatīśayyā 2  \P
      %paṃcasatyā dātikā          yuktāni harmyāṇi teṣu vāstu    ativipulā  mṛdutaralāśayyā//2//        \B
      %paṃcasatyā dātikā          yuktāni harmyāṇi teṣu vāstu    ativipulā  mṛdutaralāśayyā//3//        \L
      %paṃca vā sapta vā dṛālikā  yuktāni harmyāṇi/              ativapulā  mṛdu/uttaracchaṃdavatīśayyā/  \N1
      %paṃca vā sapta vā dṛāṃlikā yuktāni harmyāṇi               ativapulā  mṛduuttarachaṃdavatīśayyā/     \D
      %paṃca vā sapta vā tālikā---yuktāni harmyāni               ativipulā  mṛduuttarachaṃdavatīśayyā    \N2
      %paṃca vā sapta vā dālikā---yuktāni harmyāṇi               ativipulāṃ mṛduuttarachadavatiśaiyyā     \U1
      %--------------------------saudhāni harmyāṇi vāsāya kecit// aṣṭau bhogān āha// sugrahaṃ// suvastraṃ// suśayā sustrī//  \U2
      %--------------------------------------------
      %,a site of the palace in which there are mainsions endowned with five or seven rooms. A very large and soft bed with an excellent blanket. 
      %-------------------------------------------
            \noindent
            \note[type=source, labelb=147a, labele=_147e, nosep]{cf. YSv (PT, p. 837): ātmāvivekam āgamya calac cittaṃ mahākulam | viṣayāndhatamo dṛṣṭvā no vetti paramātmanaḥ | amāyātmā tattvātītaḥ satsandhānavivarjitaḥ | sukhī duḥkhī janmamṛtyuṃ yāti satyaṃ punaḥ punaḥ | vairāgyādidhanaṃ tyaktvā viṣavad duḥkhakṛddhiyaḥ | koṭisūryasamātmeti jñānayogād vimucyate |}
      \app{\lem[wit={D,E,N1,N2}, alt={paṭṭa°}]{paṭṭa}
        \rdg[wit={B,L,U2}]{paṭa°}
        \rdg[wit={P,U1}]{pada°}
      }\app{\lem[wit={ceteri},alt={°sūtra°}]{sūtra}
        \rdg[wit={B,L}]{°sūtrā°}
      }\app{\lem[wit={ceteri}, alt={°mayāni}]{mayāni}
        \rdg[wit={D,N1,N2}]{°yāni}}
      \app{\lem[wit={E,P,L}]{vastrāṇi}
        \rdg[wit={ceteri}]{vasrāṇi}} 1\dd{}
      \app{\lem[wit={X}]{pañca vā sapta vā}
        \rdg[wit={E,P}]{paṃcasaptā}
        \rdg[wit={L,B}]{paṃcasatyā}}
      \app{\lem[type=emendation, resp=egoscr]{śālikā}
        \rdg[wit={E,N1}]{dṛālikā}
        \rdg[wit={D}]{dṛāṃlikā}
        \rdg[wit={P}]{dadhikā}
        \rdg[wit={B,L}]{dātikā}
        \rdg[wit={N2}]{tālikā}
        \rdg[wit={U1}]{dālikā}
      }\app{\lem[wit={ceteri}]{yuktāni}
        \rdg[wit={U2}]{saudhāni}}
\app{\lem[wit={X}]{harmyāṇi}
        \rdg[wit={L,B}]{harmyāṇi teṣu vāstu}
        \rdg[wit={E}]{harmyāṇi teṣu vāsaḥ}
        \rdg[wit={P}]{harmyāṇi teṣu cāsaḥ}
        \rdg[wit={U2}]{harmyāṇi vāsāya kecit}} 2\dd{}
      \app{\lem[wit={ceteri},alt={ativipulā°}]{ativipulā}
        \rdg[wit={D,N1}]{ativapulā°}
        \rdg[wit={U1}]{ativipulāṃ}
        \rdg[wit={U2}]{aṣṭau bhogān āha ||}}
      \app{\lem[type=emendation, resp=egoscr,alt={mṛdūttara°}]{mṛdūttara}
        \rdg[wit={B,E,L,P}]{mṛdutara°}
        \rdg[wit={X}]{mṛdu | uttara°}
        \rdg[wit={U2}]{sugrahaṃ ||}
      }\app{\lem[wit={P},alt={°chadavatī°}]{chadavatī}
        \rdg[wit={D,N1,N2}]{°chandavatī°}
        \rdg[wit={U1}]{°chadavati°}
        \rdg[wit={U2}]{suvastraṃ ||}}
      \app{\lem[wit={ceteri}, alt={°śayyā}]{śayyā}
        \rdg[wit={U2}]{suśayā sustrī}} 3\dd{}
      %------------------------------
      %padminī tārūṇyavatī  manoharā guṇavatī  tatropaviṣṭā kāṃtā/      \E
      %padminī tārūṇyavatī  manoharā guṇavatī  tatopaviṣṭā  kāṃtā 4     \P
      %padminī tārūnyavatī  manoharā guṇavatī//tatrāpavistā kāṃtā 4     \B
      %padminī tārūnyavatī  manoharā guṇavatī//tatropavistā kāṃtā// 4// \L
      %padmanī tārūṇyavatī  manoharā guṇavatī  tatropavistā//           \N1
      %padminī tārūrāyavatī manoharā guṇavatī  tatropavistā//           \D
      %padminī tārūnyavatī  manoharā guṇavatī  tatropavistā             \N2
      %padminī tārūnyavati  manoharā guṇavati  tatropavistā             \U1
      %                                                                 \U2
      %--------------------------------------------
      %[On which] there is situated [tatropaviṣṭā] a excellent [em. zu tāruṇyavatī] youthful, charming and virtuous wife.
      %-------------------------------------------
      \app{\lem[wit={ceteri}]{padminī}
        \rdg[wit={N1}]{padmanī}
        \rdg[wit={U2}]{\om}}
      \app{\lem[type=emendation, resp=egoscr]{tāruṇyavatī}
        \rdg[wit={ceteri}]{tārūṇyavatī}
        \rdg[wit={N2}]{tārūrāyavatī}
        \rdg[wit={U2}]{\om}}
   manoharā guṇavatī
\app{\lem[wit={E}]{tatropaviṣṭā}
        \rdg[wit={P}]{tatopaviṣṭā} 
        \rdg[wit={X}]{tatropavistā}
        \rdg[wit={B}]{tatrāpavistā}
        \rdg[wit={U2}]{\om}}
      \app{\lem[wit={B,E,L,P}]{kāntā}
        \rdg[wit={ceteri}]{\om}} 4\dd{}
      %------------------------------
      %sādhu āśanam/      atimūlyañ ca/         manoramam annaṃ/       tathā vidhaṃ pānam/   \E
      %sādhu āsanaṃ 5     atimūlo 'śvaḥ 6       manoramam annaṃ    7   tathā vidhaṃ pānaṃ 8  \P
      %sādhu āsanaṃ 5     atimūlyo asvaṃ//6     manoramyam annaṃ //7   tathā vidhapānaṃ//8   \B
      %sādhu āsanaṃ// 5// atimūlyo aśvaṃ//6//   manoramyam annaṃ //7// tathā vidhapānaṃ//8// \L
      %sādhyāsanaṃ//      amūlyo svaś ca//      manoramam annaṃ        tathā vidhaṃ pānaṃ/   \N1
      %sādhyāsanaṃ//      amūlyo svaś ca//      manoramam annaṃ        tathā vidhaṃ pānaṃ//  \D
      %sādhyāsanaṃ        amūlyo svaś ca        manotamam annaṃ        tathā vidhapānaṃ//    \N2
      %sādhyāsanaṃ        amolyo svaś ca        manoramam annaṃ        tathā vidhaṃ pānaṃ    \U1
      %sādhu āsanaṃ//           suśvaḥ//        suṣṭu annaṃ//          tathā vidhayānaṃ//    \U2
      %--------------------------------------------
      %good throne/seat; atimūlyo (überaus wertvolles) 'śvaṃ (Pferd), manorama ( die Sinne erfreuendes) Essen, verschiedenes Trinken. 
      %-------------------------------------------
      \app{\lem[type=emendation, resp=egoscr]{sādhvāsanam}
        \rdg[wit={E}]{sādhu āśanam}
        \rdg[wit={B,L,P,U2}]{sādhu āsanaṃ}
        \rdg[wit={D,N1,N2}]{sādhyāsanaṃ}} 5\dd{}
      \app{\lem[type=emendation, resp=egoscr, alt={atimūlyo 'śvaś ca}]{atimūlyo'śvaś-ca}
        \rdg[wit={X}]{amūlyo svaś ca}   
        \rdg[wit={E}]{atimūlyañ ca}
        \rdg[wit={P}]{atimūlo 'śvaḥ}
        \rdg[wit={B,L}]{atimūlyo asvaṃ}
        \rdg[wit={U2}]{suśvaḥ}} 6\dd{}
      \app{\lem[wit={ceteri},alt={manoramam}]{manorama\skp{m-a}}
        \rdg[wit={B}]{manoramyam}
        \rdg[wit={L}]{manoramyam}
          \rdg[wit={U2}]{suṣṭu}}\skm{m-a}nnam 7\dd{}
      tathā\app{\lem[wit={ceteri}]{vidhaṃ pānam}
        \rdg[wit={B,L,N2}]{vidhapānaṃ}
        \rdg[wit={U2}]{vidhayānaṃ}} 8\dd{}\phantomsection\label{eightenjoyments}
      %------------------------------
      %ete   ṣṭau bhogāḥ   kathitāḥ/   eke  duḥkhaṃ   bhajante/  bhikṣāṃ  yācante// kiñca \E
      %ete   ṣṭau bhogāḥ   kathitā 9   eke  duḥkha    bhajaṃte   bhikṣāṃ  yāṃcaṃte ca  \P
      %ete   ṣṭau bhogāḥ//             eka  duḥkhā    bhajaṃte/  bhikṣā   yāṃcate ca//  \B
      %ete   ṣṭau bhogāḥ//             eka  duḥkhā    bhajaṃte// bhikṣā   yāṃcate ca//  \L
      %ete  aṣṭau bhogā    kathyate/   eke  duḥkhaṃ   bhajaṃte/  bhikṣyāṃ yācate ca/   \N1
      %ete  aṣṭau bhogāḥ   kathyaṃte// ete  duḥkhaṃ   bhajaṃte/  bhikṣyāṃ yācaṃte ca// \D
      %ete  aṣṭau ghogā    kathyate//  ete  duḥkhataṃ bhajate    bhikṣāṃ  yācate ca//  \N2
      %rāte aṣṭau bhogāḥ   kathyate    ete  duḥkhaṃ   bhajate    bhikṣāṃ  pācate ca    \U1
      %ete  ṣṭau  bhogāḥ// kathitāḥ//  ekaṃ duḥkhaṃ   bhajaṃte// bhikṣā   yācaṃte ca// \U2
      %------------------------------
      %The eight enjoyments are being described. They impart suffering, and [make one] begging for their sustenance.
      %------------------------------
      %\note[type=philcomm, labelb=148, lem={'ṣṭau bhogāḥ}]{The eight enjoyments are not attested in any of the sources.}
      \app{\lem[wit={ceteri}]{ete}
        \rdg[wit={U1}]{rāte}
      }\app{\lem[wit={Y}]{'ṣṭau}
        \rdg[wit={X}]{aṣṭau}}
      \app{\lem[wit={ceteri}]{bhogāḥ}
        \rdg[wit={N1,N2}]{bhogā}
        \rdg[wit={U1}]{ghogā}}
      \app{\lem[wit={D}]{kathyante}
        \rdg[wit={N1,N2,U1}]{kathyate}
        \rdg[wit={E,U2}]{kathitāḥ}
        \rdg[wit={P}]{kathitā}
        \rdg[wit={B,L}]{\om}}/
      \app{\lem[wit={D,N2,U1}]{ete}
        \rdg[wit={E,P,N1}]{eke}
        \rdg[wit={B,L}]{eka}
        \rdg[wit={U2}]{ekaṃ}}
      \app{\lem[wit={D,E,N1,U1,U2}]{duḥkhaṃ}
        \rdg[wit={P}]{duḥkha}
        \rdg[wit={B,L}]{duḥkhā}
        \rdg[wit={N2}]{duḥkhataṃ}}
      \app{\lem[wit={ceteri}]{bhajante}
        \rdg[wit={N2,U1}]{bhajate}}/
      \app{\lem[wit={E,P,N2,U1}]{bhikṣāṃ}
        \rdg[wit={D,N1}]{bhikṣyāṃ}
        \rdg[wit={B,L,U2}]{bhikṣā}}
      \app{\lem[wit={ceteri}]{yācante}
        \rdg[wit={P}]{yāṃcaṃte}
        \rdg[wit={B,L}]{yāṃcate}
        \rdg[wit={N2}]{yācate}
        \rdg[wit={U1}]{pācate}}
      \app{\lem[wit={ceteri}]{ca}
        \rdg[wit={E}]{kiñca}}/\linelabel{_147e}
    \end{prose}
    \begin{prose}[p22_05]
      %------------------------------
      %      yathā sūryasya tejaḥ   dugdhasya    ghṛtam   agner jvalanaṃ viṣān mūrchā   tilāttailam/    vṛkṣāc-chāyā/  phalāt parimalaḥ       kāṣṭhād agniḥ    arkarādibhyo   madhuro rasaḥ/ \E
      %      yathā sūryasys tejaḥ   dugdhasya    ghṛtaḥ   agne dvāhaḥ    viṣān mūrchāti tilāttailaṃ     vṛkṣāt-chāyā   phalāsarimalaḥ         kāṣṭād  agniḥ    śarkvarādibhyo madhuro rasaḥ  \P
      %      yathā sūryasye tejāḥ   dugdha-------ghṛtaḥ   agne dvāhaḥ//  viṣān mūrchā   tilāttailaṃ//   vṛkṣā--chāyā   phalāsarimalaḥ         kaṣṭād  agniḥ    śarkadībhyo    madhuro  \B
      %      yathā sūryasya tejāḥ   dugdha-------ghṛtaḥ   agne dvāhaḥ//  viṣān mūrchā   tilātailaṃ//    vṛkṣā--chāyā   phalāt parimalaḥ       kaṣṭād  agniḥ    śarkadībhyo    madhuro  \L
      %      yathā sūryasya tejaḥ/  dugdhasya    ghṛtaṃ/  agne dahiḥ??   viṣān mūrchā   tilāttailaṃ,    vṛkṣāc-chāyā/  phalāt parimalaḥ/      kāṣṭhād āgniḥ/   śarkkarādibhyo madhuro rasaḥ/ \N1
      %      yathā sūryasya tejaḥ// dugdhasya    ghṛtaṃ// agne dadhiḥ    viṣān mūrchā   tilāttailaṃ//   vṛkṣā--chāyā// phalāt palātparimalaḥ//kāṣṭhād āgniḥ//  śarkarādibhyo  madhuro rasaḥ/ \D
      %      yathā sūryasya tejaḥ   dusya        ghṛtaṃ   agne dadhi     viṣān mūrchā   tilatailaṃ      vṛkṣā--chāyā   phalāt parimalaḥ       kāṣṭhād āgniḥ    śarkarādibhyo  madhuro rasaḥ/ \N2
      %      yathā sūryaśca tejaḥ   dugdhasy     ghṛttaṃ  agne dārhaṃ    viṣāt mūrchā   tilātailaṃ      vrakṣā-chāyā   phalāt parimalaḥ       kāṣṭhād āgniḥ    śarkarādibhyo  madhuro rasaḥ \U1
      %      yathā sūryasya tejaḥ// dugdhasya    ghṛtaṃ// agne dāhiḥ//   viṣān mūrchā   tilātailaṃ//    vṛkṣā--chāyā// phalāt parimalaḥ//     kāṣṭād  agniḥ    śarkarādibhyo  madhuro rasaḥ// \U2
      %------------------------------
      %Gleichwie die Strahlen der Sonne, die Butter der Milch, das Brennen des Feuers, die Betäubung aufgrund von Gift, das Sesamöl aus dem Sesamkorn, der Schatten vom Baum, der Wohlgeruch von einer Frucht, das Feuer von einem Holzscheid, der Süße Saft [em. zu śārkara] a liquor prepared from Dhātakī with sugar] und so weiter,   
      %------------------------------
      %Like the rays of the sun, the butter of milk, the burning of fire, the stupor of poison, the sesame oil from the sesame seed, the shade from the tree, the sweet odor from a fruit, the fire from a scabbard, the sweet sap [em . to śārkara] a liquor prepared from Dhātakī with sugar] and so on,
      %------------------------------
      \note[type=source, labelb=149, labele=_149e, nosep]{cf. YSv (PT, p. 837): ravī tejo ghṛtaṃ dugdhe tile tailaṃ svabhāvataḥ | śaśam indau kule śākhaṃ kṣāre ca lavaṇaṃ yathā | tathā brahmaṇi saṃsāro hyakhaṇḍaparipūrvake |}
      yathā
      \app{\lem[wit={ceteri}]{sūryasya}
        \rdg[wit={U1}]{sūryaś ca}}
      \app{\lem[wit={ceteri}]{tejaḥ}
        \rdg[wit={B,L}]{tejāḥ}}\dd{}
      \app{\lem[wit={D,E,P,N1,U2}]{dugdhasya}
        \rdg[wit={B,L}]{dugdha°}
        \rdg[wit={N2}]{dusya}
        \rdg[wit={U1}]{dugdhasy}}
      \app{\lem[wit={ceteri}]{ghṛtam} %!
        \rdg[wit={B,L,P}]{ghṛtaḥ}}\dd{}
      \app{\lem[wit={E}, alt={agner}]{agne\skp{r-dā}}
        \rdg[wit={ceteri}]{agne}
      }\app{\lem[type=emendation, resp=egoscr, alt={dāhaḥ}]{\skm{r-dā}haḥ}
        \rdg[wit={B,L,P}]{dvāhaḥ}
        \rdg[wit={N1}]{dahiḥ}
        \rdg[wit={N2}]{dadhi}
        \rdg[wit={D}]{dadhiḥ}
        \rdg[wit={U1}]{dārhaṃ}
        \rdg[wit={U2}]{dāhiḥ}
        \rdg[wit={E}]{jvalanaṃ}}\dd{}
      \app{\lem[wit={ceteri},alt={viṣān}]{viṣā\skp{n-mū}}
        \rdg[wit={U1}]{viṣāt}
      }\skm{n-mū}rchā\dd{}
      \app{\lem[wit={ceteri},alt={tilāt}]{tilā\skp{t-tai}}
        \rdg[wit={P}]{titilāt}
        \rdg[wit={N2}]{tila}
        \rdg[wit={U1}]{tilā}
      }\skm{t-tai}lam\dd{} %!
      \app{\lem[wit={E,N1}, alt={vṛkṣāt}]{vṛkṣā\skp{c-chā}}
        \rdg[wit={P}]{vṛkṣāt}
        \rdg[wit={B,D,L,N2,U2}]{vṛkṣā}
        \rdg[wit={U1}]{vrakṣā}
      }\skm{c-chā}yā\dd{}
      \app{\lem[wit={ceteri},alt={phalāt}]{phalā\skp{t-pa}}
        \rdg[wit={B,L}]{phalā}
      }\app{\lem[wit={ceteri},alt={parimalaḥ}]{\skm{t-pa}rimalaḥ}
        \rdg[wit={B,L}]{sarimalaḥ}
        \rdg[wit={D}]{palāt parimalaḥ}}\dd{}%\note[type=philcomm, labelb=150, lem={parimalaḥ}]{Clarification: Witness \getsiglum{D} reads \textit{phalāt palāt parimala}.}
      \app{\lem[wit={ceteri}, alt={kāṣṭhād}]{kāṣṭhā\skp{d-a}}
        \rdg[wit={P,U2}]{kāṣṭād}
        \rdg[wit={B,L}]{kaṣṭād}
      }\app{\lem[wit={Y}, alt={agniḥ}]{\skm{d-a}gniḥ}
        \rdg[wit={X}]{āgniḥ}}\dd{}
      \app{\lem[type=emendation, resp=egoscr]{śārkarādibhyo}
        \rdg[wit={E}]{arkarādibhyo}
        \rdg[wit={P}]{śarkvarādibhyo}
        \rdg[wit={L,B}]{śarkadībhyo}}
      madhuro
      \app{\lem[wit={ceteri}]{rasaḥ}
        \rdg[wit={B,L}]{\om}}\dd{}
      %------------------------------
      %himānībhyaḥ   śītam      ityādipadārthānāṃ svabhāvaḥ         tathā    saṃsāro'pi parameśvarasvarūpamadhye      tiṣṭhati/ \E
      %himānībhyaḥ   śītaṃ      ityādipadārthasvabhāva        eva   tathā    saṃsāro'pi parameśvarasvarūpamadhye      tiṣṭhati \P
      %sahīmānībhyaḥ śītaḥ/     ityādipadārthāsvabhāvataḥ// eva     tathā    saṃsāro pi paremesvara svarūpasya madhye tiṣṭhatī/ \B
      %sahimānibhyaḥ śītaḥ//    ityādiphadārthāḥ svabhāvataḥ// eva  tathā    saṃsāro pi paremesvara svarūpasya madhye tiṣṭhati// \L
      %himānibhyaḥ   śaityāṃ    ityādipadārthasvabhāva evā/         tathā    saṃsāro pi parameśvarasvarūpamadhye      tiṣṭhati// \N1
      %himānibhyaḥ   śaityaṃ // ityādipadārthasvabhāva eva//        tathā    saṃsāro pi parameśvarasvarūpamadhye      tiṣṭhati// \D 
      %himānitpa     śaityāś    atyādipadārtharthasvabhāva eva//    tathā    saṃsāro pi parameśvarasvarūpamadhye      tiṣṭhati \N2
      %himānībhyaḥ   śaityaṃ    ityādipadārthasvabhāvaḥ ravaḥ?      tathā vā saṃsāro pi parameśvararūpamadhye         tiṣṭhati/ \U1
      %himānībhyaḥ   śītyaṃ//   ityādipadārthāsvabhāva eva//        tathā    saṃsāro pi parameśvarasvarūpamadhye      tiṣṭhaṃti// \U2
      %------------------------------
      %die Kälte von Schneehaufen, und so weiter ist das inhärente Wesen der Dinge. IN gleicher Weise befindet sich auch der Weltengang im Zentrum der eigenen Gestalt von höchsten Gott.
      %the cold of piles of snow, and so on is the inherent essence of things. In the same way, the course of the world is also in the center of the highest God's own form. 
      %------------------------------
      \app{\lem[wit={ceteri}]{himānībhyaḥ}
        \rdg[wit={B,L}]{sahimānibhyaḥ}
        \rdg[wit={N2}]{himānitpa}}
      \app{\lem[wit={D,U1}]{śaityam} %!
        \rdg[wit={N1}]{śaityāṃ}
        \rdg[wit={U2}]{śītyaṃ}
        \rdg[wit={N2}]{śaityāś}
        \rdg[wit={E,P}]{śītaṃ}
        \rdg[wit={B,L}]{śītaḥ}}\dd{}
      \app{\lem[wit={D,N1,P}, alt={ityādipadārthasvabhāva}]{ityādipadārthasvabhāva}
        \rdg[wit={U2}]{ityādipadārthā°}
        \rdg[wit={B}]{ityādipadārthāsvabhāvataḥ}
        \rdg[wit={N2}]{atyādipadārtharthasvabhāva}
        \rdg[wit={U1}]{ityādisvabhāvaḥ}
        \rdg[wit={L}]{ityādiphadārthāḥ svabhāvataḥ}
        \rdg[wit={E}]{ityādipadārthānāṃ svabhāvaḥ}}
      \app{\lem[wit={ceteri}]{eva}
        \rdg[wit={N1}]{evā}
        \rdg[wit={U1}]{ravaḥ}
        \rdg[wit={E}]{\om}}\dd{}
      \app{\lem[wit={ceteri}]{tathā}
        \rdg[wit={U1}]{tathā vā}}
      saṃsāro'pi
      \app{\lem[wit={ceteri}]{parameśvarasvarūpamadhye}
        \rdg[wit={B,L}]{paremesvara svarūpasya madhye}
        \rdg[wit={U1}]{parameśvararūpamadhye}}
      \app{\lem[wit={ceteri}]{tiṣṭhati}
        \rdg[wit={B}]{tiṣṭhatī}
        \rdg[wit={U2}]{tiṣṭhaṃti}}/
      %------------------------------
      %parameśvaro 'khaṇḍa--paripūrṇaḥ/  \E
      %parameśvaro khaṃḍa---paripūrṇaś ca    \P
      %parameśvaro khaṃḍa---paripūrṇaś ca// \B
      %parameśvaro khaṃḍa---paripūrṇaś ca//  \L
      %parameśvaro 'ṣaṃḍa---paripūrṇaś ca//  \N1
      %parameśvaro  ṣaṃḍa---paripūrṇaś ca//  \D %%%S.9 verso
      %parameśvaro yarāṇḍa--paripūrṇaś ca//  \N2
      %parameśvaro khaṃḍaḥ  paripūrṇaś ca   \U1 %%%277.jpg
      %parameśvaro 'khaṃḍa--paripūrṇaś ca//   \U2
      %------------------------------
      %Und der höchste Gott ist unteilbar und das All erfüllend.
      %And the Most High God is indivisible and all-filling.
      %------------------------------
      parameśvaro\app{\lem[wit={ceteri}, alt={'khaṇḍa°}]{'khaṇḍa}
        \rdg[wit={D,N1}]{'ṣaṃḍa°}
        \rdg[wit={N2}]{yarānda°}
        \rdg[wit={U1}]{khaṃḍaḥ}
      }\app{\lem[wit={ceteri},alt={°paripūrṇaś ca}]{paripūrṇaś\skp{-}ca}
        \rdg[wit={E}]{paripūrṇaḥ}}\dd{}\linelabel{_149e}
   \end{prose}
\end{edition}
\begin{translation}
  \ekddiv{type=trans}
    \begin{tlate}[p22_04]
      \noindent
  1. Clothes made from silk thread;\footnote{Within the twenty \textit{upabhoga}s of the \textit{Mānasollāsa} there is the topic of \textit{Vastropabhoga} (``enjoyment of garments''). Particularly in summer, the king is asked to wear silk or cotton clothes which are thin and charming, cf. \citeauthor[1939: 14]{manasollasa}.} 2. Mansions endowed with five or seven rooms.\footnote{The first \textit{adhyāya} of the third \textit{viṃśati} of the \textit{Mānasollāsa} discusses astrology for finding out auspicious moments while building new houses for princes. The section describes houses with one to four \textit{śālā}s, cf. \citeauthor[1939: 6-7]{manasollasa}.} 3. A very large bed with a soft and lovely blanket;\footnote{This is found as \textit{Śayyābhoga} within the \textit{Mānasollāsa}. The section describes seven kinds of beds and eight kinds of bed-steads, cf. \citeauthor[1939: 21]{manasollasa}.} 4. [on which] there is seated a wife belonging to the Padminī-class\footnote{Cf. \citetitle{ratirahasya}, Ed. p. 6.} of women - youthful, beautiful and virtuous;\footnote{This is resembled as \textit{yoṣidupabhoga} (``enjoyment of young women'') within the \textit{Mānasollāsa}. In this chapter, King Someśvara describes the qualifications of women a king should marry. The two most important qualities he gives are beauty and full youth. Out of the four kinds of women: (a) Padminī, (b) Citriṇī, (c) Śaṅkhinī, and (d) Hastinī, he suggests that the latter two kinds are not worth enjoying, cf. \citeauthor[1935: 21]{manasollasa}.\phantomsection\label{padmini}} 5. An excellent seat;\footnote{The \textit{āsanopabhoga} (``the enjoyment of seats'') within the \textit{Mānasollāsa} describes various kinds of royal seats, cf. \citeauthor[1939: 15]{manasollasa}.} 6. An exceptionally valuable horse;\footnote{This is resembled as \textit{yānopabhoga} (``enjoyment of vehicles'') within the \textit{Mānasollāsa}. In this section, King Someśvara lists nine kinds of vehicles, including horses, cf. \citeauthor[1939: 24]{manasollasa}.} 7. Appetising food;\footnote{This is resembled as \textit{annabhoga} (``enjoyment of food'') within the \textit{Mānasollāsa}. In this section, King Someśvara describes the names of various kinds of delicious food and the directions as to the preparations of various dishes, cf. \citeauthor[1939: 21]{manasollasa}.} 8. Similarly [tasty] drinks.\footnote{Drinks are the subject of the \textit{pānīyabhoga} (``enjoyment of drinks'')  section within the \textit{Mānasollāsa}. This section describes everything related to drinking and drinks, cf. \citeauthor[1939: 23]{manasollasa}.} 
  The eight enjoyments are described. They impart suffering. And [they] require begging.\begin{buber}[f22_1]\footnote{To the genre connoisseur, the sentence ``\textit{bhīkṣāṃ yācante ca} |'' initially seems suspiciously strange and suggests a corruption of the text. However, the passage is well preserved in the \beta\hspace{0.15em} and \gamma-group. The subject of the sentence is undoubtedly the \textit{aṣṭau bhogāḥ}. Nevertheless, \ldots}\end{buber}\\
\end{tlate}
    \begin{tlate}[p22_05]
  \indent Just like the rays of the sun, the ghee of milk, the burning of fire, the stupor from poison, the sesame oil from the sesame seed, 
  the shade from the tree, the sweet odour from a fruit, the fire from a wood log, the sweet taste of sugary things, the cold from piles of snow, etc., is the nature of the thing. In the same way, the circuit of mundane existence is within the highest God's nature. Moreover, the highest God is indivisible and complete.
  \phantomsection\label{endsvabhava}
   \flushpage 
  \end{tlate}
\end{translation}
\end{alignment}
\pagebreak %after pp. 69-70
\cleardoublepage
\selectlanguage{english}
\chapter{Appendix}
\section{Figures}

% \begin{landscape}
\clearpage

  \begin{figure}[ht]
	\centering
  \includegraphics[width=1\textwidth]{pics/Wolpertinger.png}
\caption[The \textit{dehasvarūpa} of \textit{ajapāgāyatrī}]{The \textit{dehasvarūpa} of \textit{ajapāgāyatrī}. The image, reminiscent of a hippogriff, is part of an illustrated Sanskrit manuscript written in the Śāradā script. Preserved as a single large scroll under Acc. No. 1334 at the Oriental Institute in Srinagar (Kashmir), it is entitled \textit{Nāḍīcakra}. The manuscript contains a depiction of the yogic body’s \textit{cakra}s and \textit{nāḍī}s. The text surrounding the figure closely corresponds to the additional material found in manuscript \getsiglum{U2} of the \textit{Tattvayogabindu}. The manuscript reads (diplomatic transcription): \textit{oṃ daśame pūrṇagiripīṭhe lalāṭamaṇḍale candro devatā amṛtāśaktiḥ paramātmā ṛṣiḥ dvāviṃśaddalāni amṛtavāsinikalā 4: ambikā 1 lambikā 2 gha(ṃ)ṭkā 3 tālikā 4 dehasvarūpaṃ kākamukhaṃ 1 naranetraṃ 2 gośṛṅgaṃ 3 lalāṭabrahmapara 4 hayagrīvā 5 mayūramuśchaṃ 6 haṃsacārītani 7 sthāna.}}
	\phantomsection\label{fig_wolpertinger}
      \end{figure}

      \clearpage

  \begin{figure}[ht]
	\centering
  \includegraphics[width=1\textwidth]{pics/Vishnu_Vishvarupa_cropped.jpg}
	\caption{Viṣṇu Viśvarūpa, India, Rajasthan, Jaipur, ca. 1800–1820, Opaque watercolor and gold on paper, 38.5 × 28 cm, Victoria and Albert Museum, London, Given by Mrs. Gerald Clark.}
	\label{fig1}
      \end{figure}
\clearpage
  \begin{figure}[ht]
	\centering
  \includegraphics[width=0.5\textwidth]{pics/The_Equivalence_of_Self_and_Universe_(detail),_folio_6_from_the_Siddha_Siddhanta_Paddhati,_(Bulaki),_1824_(Samvat_1881);_122_x_46_cm._Mehrangarh_Museum_Trust..jpg}
	\caption{The Equivalence of Self and Universe (detail), folio 6 from the \textit{Siddhasiddhāntapaddhati} (Bulaki), India, Rajasthan, Jodhpur, 1824 (Samvat 1881), 122 x 46 cm, RJS 2378, Mehragarh Museum Trust.}
	\label{fig2}
      \end{figure}
      % \end{landscape}

      \newpage
      \cleardoublepage
\chapter{Bibliography}
 \label{sec:bibli}
\clearpage
\newpage 
\thispagestyle{empty}
\quad  \addtocounter{page}{-1}

\newrefcontext[sorting=tixel]
\printbibliography[heading=subbibintoc, title=Primary Sources, keyword=primary]

\newrefcontext[sorting=nyt]
\printbibliography[heading=subbibintoc, title=Secondary Literature, keyword=seclit]

\printbibliography[heading=subbibintoc, title=Catalogues, keyword=catalogues]

\printbibliography[heading=subbibintoc, title=Online Sources, keyword=onlinesource]

\end{document}


%%% Local Variables:
%%% mode: latex
%%% TeX-master: t
%%% End:
