%Ultimatives Tool zur Datierung:
%https://www.cc.kyoto-su.ac.jp/~yanom/pancanga/
%skp = ignored in edition
%skm = ignored in xml
%%%---2-DO---%%%:
% - add xml ids for cladistics
% - produce diplomatic transcripts for saktumiva
% - make extra layer in Apparatus for parallels in SVARODAYA, Siddhasiddhantapaddhati and Amanaska
% - check all daṇḍas!!! now I think that it's more likely that many of them were lost in copies. Lectio difficilior! Very unconventional style of the autor! 
% - read Sarvangayogapradipika, Maya Burger! 
% - maybe add second ciritical edition of yogasvarodaya?!
% - Korrekturlesen von \E!! 
% - Verspattern einbauen!
% - add all Testtimonia of SSP & Ysv
% - Sigla alphabetisch ordnen und! daṇḍas mit einkollationieren
% - präambel auslagern wie Jürgen 
% - grep-search alle Verse!!!!
% - Mss spreadsheet
% - sort N1,D1,B2 zu N1,N2,D1
% - sort all sigla alphabetically 
% - additions to U2: make footnotes for the bahir mātrā-s: explaining the inventions of female deities and tell that this is "schwer interpretierbar"
% - Belege für source und testimonia einfügen!!!
% - GIVE UNIQUE LABELS for TESTIMONIO AND SOurces
% - Edition mit Sätzen übereinander nennt sich: Synoptische Edition
% - Consider changing Lakṣya to Lakṣa
% - vEREINHEITLICHUNG von source und testium! 
%%%%%%%%%%%%%%%%%%%%%%%%%%%%%%%%%%%%%%%%%
% Don't forget
% Siddhasiddhantapaddhati Yogic Body descriptions are followed by Rāmacandra
% Quotes of the Yogasvarodaya in the Yoga Karṇikā
% Rāmacandra more a compiler than an author!!!
% Identify quotes of YTB in Haṭhasanketacandrikā 
%%%%%%%%%%%%%%%%%%%%%%%%%%%%%%%%%%%%%%%%%%%
%MSS notes
%
%--B: i and ī are not differenciated
%--P: no punctuation no daṇdas nothing
%--U1: dot . serves as daṇḍa 
%--\L and \U2 very similar
%--figure out for U2: // ajapājapaḥ sahasra // 6000 //gha 0 16 pa 0 40// \U2?!?!?!?!?!?
%%%%%%%%%%%%%%%%%%%%%%%%%%%%%%%%%%%%%%%%%%
%
% Einleitung Ideen 
% - sprachliche Simplizität
% - Potenzial als Anfängertext
% - Großartige Einführung in die Textkritik -> Synoptische Edition 
% - Gelegenheit Yogasvarodaya und Yogatattvabindu zu edieren 
% - Historische Evidenz entweder für das königliche Leben in einer Maṭha in der Nähe von Benares während der Muslimischen Herrschaft, oder sogar Lehrtext für die Bildung junger Prinzen  
% - eines der raren Beispiele der engen Verknüfung mehrerer Texte 
% - eines der raren Beispiele der Prosaisierung eines metrischen Textes 
% - Anwendung rezenter Technologie! 
% - How the text was construed -> intermingling of Ysv and SSP
% - Martin Straube: "jeder kleine Dorfhäuptling kann Rāja genannt werden". 
%%%%%%%%%%%%%%%%%%%%%%%%%%%%%%%%%%%%%%%%%%%
%Ich habe dieses Zitat gefunden
%Franz
%Franz Veit
%हठयोगः [Printed book page 5-501-c]
%हठयोगः , पुं, (हठेन योगः ।) योगविशेषः ।
%यथा, —
%“इदानीं हठयोगस्तु कथ्यते हठसिद्धिदः ।
%कृत्वासनं पवनाशं शरीरे रोगहारकम् ॥
%पूरकं कुम्भकञ्चैव रेचकं वायुना भजेत् ।
%इत्थं क्रमोत्क्रमं ज्ञात्वा पवनं सग्धयेत् सदा ॥
%धौत्यादिकर्म्मषट्कञ्च संस्कुर्य्याद्धठसाधकः ।
%एतन्नाड्यान्तु देवेशि ! वायुपूर्णं प्रतिष्ठितम् ॥
%ततो मनो निश्चलं स्यात्तत आनन्द एव हि ।
%हठयोगान्न कालः स्यान्मनः शून्ये भवेद्यदि ॥
%इदानीं हठयोगस्य द्वितीयं भेदवत् शृणु ।
%आकाशे नासिकाग्रे तु सूर्य्यकोटिसमं स्मरेत् ॥
%श्वेतं रक्तं तथा पीतं कृष्णमित्यादिरूपतः ।
%एवं ध्यात्वा चिरायुः स्यादङ्गाजननवर्ज्जितः ॥
%शिवतुल्यो महात्मासौ हठयोगप्रसादतः ।
%हठाज्ज्योतिर्म्मयो भूत्वा ह्यन्तरेण शिव भवेत् ।
%अतोऽयं हठयोगः स्यात् सिद्धिदः सिद्धसेवितः ॥”
%इति योगस्वरोदयः ॥ [ID=41348]

%Now, Haṭhayoga indeed is explained as that which gives the siddhi (accomplishment) of haṭha (persistence).
%One performs the wind-eating/serpent āsanam, which removes illness in the body
%and filling – kumbhaka – emptying may distribute the vāyu/wind.
%In this way, while being aware of the progress and regress of the breath one may feed on the wind continually.
%And with the six karmmas, dhauti etc., the Sādhaka of Haṭha may prepare/embellish himself.
%Thus/thereby, in the channel (nāḍī), Oh supreme Goddess, all of the winds (vāyu) are consecrated/placed.
%Then the mind may be unmoved and then bliss it really is.
%Through Haṭhayoga time will be no more, when the mind in emptiness abides.
%
%Now listen to the second disclosure of haṭhayoga:
%In space, on the tip of the nose indeed, one may remember equal to ten million suns,
%the primal forms: white, red, likewise yellow, dark blue.
%Thus meditating/visualizing, one may have a long life, free of the birth of the body,
%Equal to Śiva, this great soul, due to the blessing of Haṭhayoga,
%shall become through persistence (haṭha) a being of light and internally śiva.
%Therefore this Haṭhayoga grants accomplishment – it’s practiced by the Siddhas (accomplished ones).
%Franz
%Im śabdakalpadruma
%Franz
%Franz Veit
%fj.veit@gmail.com
\documentclass[10pt]{memoir}
\setstocksize{220mm}{155mm} 	        
\settrimmedsize{220mm}{155mm}{*}	
\settypeblocksize{170mm}{116mm}{*}	
\setlrmargins{18mm}{*}{*}
\setulmargins{*}{*}{1.2}
%\setlength{\headheight}{5pt}%
\checkandfixthelayout[lines]
\linespread{1.16}
\flushbottom

%%% Hyphenation settings
\usepackage[htt]{hyphenat}
\hyphenation{he-lio-trope opos-sum}
\tracingparagraphs=1
%Hyphenation in Devanāgarī of the edition still missing? Probably this needs to be modified in babel-iast package? 

%%% babel
\usepackage[english]{babel}
\usepackage{babel-iast/babel-iast}

\babelfont[iast]{rm}[Renderer=Harfbuzz, Scale=1.3]{AdishilaSan}%AdishilaSan}
\babelfont[english]{rm}{Adobe Text Pro}

%%% more functionality
\PassOptionsToPackage{hyphens}{url}
\usepackage{hyperref}
\usepackage{pdflscape}
\usepackage{cleveref}
\usepackage{url}
\usepackage{cleveref}
\usepackage{microtype}
\usepackage{lineno}

%\usepackage{bigfoot}
%%% more functions
\usepackage[dvipsnames]{xcolor}
%\usepackage[para,perpage]{footmisc}

%%%für den Counter von Kapiteln und Sätzen! 
\newcommand{\uproman}[1]{\uppercase\expandafter{\romannumeral#1}}
\newcommand{\lowroman}[1]{\romannumeral#1\relax}

\makeindex
\newfontfamily\sanskritfont[Script=Devanagari,Mapping=RomDev,Scale=1.1]{Sanskrit2003}
\usepackage{pifont,fourier-orns,lettrine,psvectorian,paralist,enumitem,pdfpages,wrapfig,tabulary,lettrine,longtable}
\setlist[enumerate]{itemsep=0mm}
\usepackage[autostyle]{csquotes}
\usepackage[defaultlines=2,all]{nowidow}
\usepackage{ellipsis,adforn,booktabs,longtable,url,tikz}
\lineskiplimit=-3pt          

\makechapterstyle{IeT}{%
  \chapterstyle{default}
  \renewcommand*{\printchapternonum}{\centering}
  \renewcommand*{\clearforchapter}{\cleartorecto} 
  \aliaspagestyle{chapter}{empty}}
\chapterstyle{IeT}
\setsecnumdepth{none}  \openright  \nouppercaseheads
\settocdepth{subsubsection}

%%%% test better pagebreaks
%\def\fussy{%
%  \emergencystretch\z@
%  \tolerance 200%
%  \hfuzz .1\p@
%  \vfuzz\hfuzz}

%\interfootnotelinepenalty=10000\relax

%\usepackage[maxfloats=256]{morefloats}

%\maxdeadcycles=500

%raggedbottomsectiontrue
%%\checkandfixthelayout


%%%%%%%  biblatex
%\newcommand{\noun}[1]{\textsc{#1}}    %  philosophy-verbose
\usepackage[backend=biber, sorting=nyt, style=verbose]{biblatex} %%%%ORIGINAL TiE
\renewcommand*{\mkbibnamefamily}[1]{\textsc{#1}}


\DeclareFieldFormat{url}{%
  \mkbibacro{URL}\addcolon\space
  \href{#1}{\nolinkurl{\thefield{urlraw}}}}

\DeclareFieldFormat{citeurl}{%
  \href{#1}{\nolinkurl{\thefield{urlraw}}}} 


\DeclareFieldFormat{postnote}{#1}
\renewcommand{\postnotedelim}{, }
\addbibresource{bindu.bib}

%%% ekdosis
\usepackage[teiexport=tidy,parnotes=true]{ekdosis}% =tidy cleans up HTML and XML documents by fixing markup errors and upgrading legacy code to modern standards. parnotes=footnotes below or above critical apparatus

\SetLineation{lineation=page, modulo} %lineation=page sets thenumbering to start afresh at the top of each page. =modulo makes every fifth line numbered. {lineation=page} makes every line numbered! 

\renewcommand{\linenumberfont}{\selectlanguage{english}\footnotesize} %sets language of lines to English

\SetTEIxmlExport{autopar=false} %autopar=falseinstructs ekdosis to ignore blank lines in the.tex sourcefile as markers for paragraph boundaries. As a result, each paragraph of the edition must be found within an environment associated with the xml <p> element

\SetHooks{
  lemmastyle=\bfseries,
  %refnumstyle=\selectlanguage{english}\bfseries,
  refnumstyle=\selectlanguage{english}\color{blue}\bfseries,
  appheight=0.8\textheight,
}

\newif\ifinapparatus
\DeclareApparatus{source}[
%bhook=\inapparatustrue,
lang=english,
notelang=english,
% bhook=\selectlanguage{english},
bhook=\selectlanguage{english}\textbf{Sources:},%
%maxentries=4, 
%ehook=.]
%sep={] },
%nosep,
]

\newif\ifinapparatus
\DeclareApparatus{testium}[
%bhook=\inapparatustrue,
lang=english,
notelang=english,
% bhook=\selectlanguage{english},
bhook=\selectlanguage{english}\textbf{Testimonia:},
%maxentries=4, 
%ehook=.]
%nosep, 
]

% Declare \ifinapparatus and set \inapparatustrue at the beginning of
% the apparatus criticus block. Also set the language.  
\newif\ifinapparatus
  \DeclareApparatus{default}[
  %bhook=\inapparatustrue, 
  lang=english,
  %maxentries=33,
  %bhook=\selectlanguage{english},
  sep = {] },
  delim=\hskip 0.75em,
  rule=\rule{0.7in}{0.4pt},
]

\newif\ifinapparatus
\DeclareApparatus{philcomm}[
%bhook=\inapparatustrue,
lang=english,
notelang=english,
bhook=\selectlanguage{english}\textbf{Philological Commentary:},
%bhook=\selectlanguage{english},
sep={: },
]

\ekdsetup{
showpagebreaks,
spbmk = \textcolor{blue}{spb},
hpbmk = \textcolor{red}{hpb}
}

%\usepackage{fnpos}
%\makeFNmid
%\makeFNbottom
\usepackage[bottom]{footmisc}
%%%%%%%%%%%%%%%%%%%%%%%%%%%
\makeatletter
\def\blfootnote{\gdef\@thefnmark{}\@footnotetext}
\makeatother
%%%%%%%%%%%%%%%%%%%%%%%%%


% Macros and Definitions for the Print of Sigla
\def\acpc#1#2#3{{#1}\rlap{\textrm{\textsuperscript{#3}}}\textsubscript{\textrm{#2}}\space}
\def\sigl#1#2{{{#1}}\textsubscript{\textrm{#2}}}
\def\None{{\sigl{N}{1}}} \def\Noneac{\acpc{N}{1}{ac}\,} \def\Nonepc{\acpc{N}{1}{pc}\,}
\def\Ntwo{{\sigl{N}{2}}} \def\Noneac{\acpc{N}{2}{ac}\,} \def\Nonepc{\acpc{N}{2}{pc}\,}
\def\Done{{\sigl{D}{1}}} \def\Doneac{\acpc{D}{1}{ac}\,} \def\Donepc{\acpc{D}{1}{pc}\,}
\def\Dtwo{{\sigl{D}{2}}} \def\Dtwoac{\acpc{D}{2}{ac}\,} \def\Dtwopc{\acpc{D}{2}{pc}\,}
\def\Uone{{\sigl{U}{1}}} \def\Uoneac{\acpc{U}{1}{ac}\,} \def\Uonepc{\acpc{U}{1}{pc}\,}                 
\def\Utwo{{\sigl{U}{2}}} \def\Utwoac{\acpc{U}{2}{ac}\,} \def\Utwopc{\acpc{U}{2}{pc}\,}

%%%%%%%%%%%%%% Tattvabinduyoga - List of Witnesses   %%%%%%%%%%%%%%%%%%%
\DeclareWitness{ceteri}{\selectlanguage{english}cett.}{ceteri}[]   
\DeclareWitness{E}{\selectlanguage{english}E}{Printed Edition}[]    
\DeclareWitness{P}{\selectlanguage{english}P}{Pune BORI 664}[]  
\DeclareWitness{B}{\selectlanguage{english}B}{Bodleian 485}[]       
\DeclareWitness{N1}{\selectlanguage{english}N\textsubscript{1}}{NGMPP 38/31}[]
\DeclareWitness{N2}{\selectlanguage{english}N\textsubscript{2}}{NGMPP B 38/35}[]
\DeclareWitness{L}{\selectlanguage{english}L}{LALCHAND 5876}[]  
\DeclareWitness{D}{\selectlanguage{english}D}{IGNCA 30019}[] 
%\DeclareWitness{D2}{\selectlanguage{english}D\textsubscript{2}}{IGNCA 30020}[]  
\DeclareWitness{U1}{\selectlanguage{english}U\textsubscript{1}}{SORI 1574}[] 
\DeclareWitness{U2}{\selectlanguage{english}U\textsubscript{2}}{SORI 6082}[]
%%%%%%%%%%%%%% Tattvabinduyoga - Groups of Witnesses   %%%%%%%%%%%%%%%%%%%
\DeclareWitness{X}{\selectlanguage{english}\alpha}{Alpha Group: D,N1,N2,U1}[]
\DeclareWitness{Y}{\selectlanguage{english}\beta}{Beta Group: B,E,L,P,U2}[]
%%%%%%%%%%%%% Testimonia
\DeclareWitness{Ysv}{\selectlanguage{english}Ysv}{Yogasvarodaya}[] %%%add infos!  

%%%%%%%%%%%%%%%%%%%%%%%%%%%%%%%%%%%%%%%%%%%
% Macro for Editing Abbrevs.
\def\om{\textrm{\footnotesize \textit{om.}\ }} %prints om. for omitted in apparatus
\def\korr{\textrm{\footnotesize \textit{em.}\ }} %prints em. for emended in apparatus
\def\conj{\textrm{\footnotesize \textit{conj.}\ }} %prints conj. for conjectured in apparatus

% \supplied{text} EDITORIAL ADDITION -> Within \lem oder \rdg
% \surplus{text} EDITORIAL DELETION -> Within \lem oder \rdg
% \sic{text} CRUX
% \gap{text} LACUNAE -> [reason=??, unit=??, quantity=??, extent=??]


%%%%%%%%%%%%%%%%%%%%%%%%%%%%%%%%%%%%%%%%%%% All macros of this list can be used in 
% Macro for Editing Abbrevs.
\def\eyeskip{\textrm{{ab.\,oc. }}}
\def\aberratio{\textrm{{ab.\,oc. }}}
\def\ad{\textrm{{ad}}}
\def\add{\textrm{{add.\ }}}
\def\ann{\textrm{{ann.\ }}}
\def\ante{\textrm{{ante }}} 
\def\post{\textrm{{post }}}
%\def\ceteri{cett.\,}                   
\def\codd{\textrm{{codd.\ }}}

\def\coni{\textrm{{coni.\ }}}
\def\contin{\textrm{{contin.\ }}}
\def\corr{\textrm{{corr.\ }}}
\def\del{\textrm{{del.\ }}}
\def\dub{\textrm{{ dub.\ }}}

\def\expl{\textrm{{explic.\ }}} 
\def\explica t{\textrm{{explic.\ }}}
\def\fol{\textrm{{fol.\ }}}
\def\foll{\textrm{{foll.\ }}}
\def\gloss{\textrm{{glossa ad }}}
\def\ins{\textrm{{ins.\ }}}      
\def\inseruit{\textrm{{ins.\ }}} 
\def\im{{\kern-.7pt\lower-1ex\hbox{\textrm{\tiny{\emph{i.m.}}}\kern0pt}}} %\textrm{\scriptsize{i.m.\ }}}      
\def\inmargine{{\kern-.7pt\lower-.7ex\hbox{\textrm{\tiny{\emph{i.m.}}}\kern0pt}}}%\textrm{\scriptsize{i.m.\ }}}      
\def\intextu{{\kern-.7pt\lower-.95ex\hbox{\textrm{\tiny{\emph{i.t.}}}\kern0pt}}}%\textrm{\scriptsize{i.t.\ }}}           
\def\indist{\textrm{{indis.\ }}}  
\def\indis{\textrm{{indis.\ }}}
\def\iteravit{\textrm{{iter.\ }}} 
\def\iter{\textrm{{iter.\ }}}
\def\lectio{\textrm{{lect.\ }}}   
\def\lec{\textrm{{lect.\ }}}
\def\leginequit{\textrm{{l.n. }}} 
\def\legn{\textrm{{l.n. }}}
\def\illeg{\textrm{{l.n. }}}

\def\primman{\textrm{{pr.m.}}}
\def\prob{\textrm{{prob.}}}
\def\rep{\textrm{{repetitio }}}
\def\secundamanu{\textrm{\scriptsize{s.m.}}}            \def\secm{{\kern-.6pt\lower-.91ex\hbox{\textrm{\tiny{\emph{s.m.}}}\kern0pt}}}%   \textrm{\scriptsize{s.m.}}}
\def\sequentia{\textrm{{seq.\,inv.\ }}}  
\def\seqinv{\textrm{{seq.\,inv.\ }}}
\def\order{\textrm{{seq.\,inv.\ }}}
\def\supralineam{{\kern-.7pt\lower-.91ex\hbox{\textrm{\tiny{\emph{s.l.}}}\kern0pt}}} %\textrm{\scriptsize{s.l.}}}
\def\interlineam{{\kern-.7pt\lower-.91ex\hbox{\textrm{\tiny{\emph{s.l.}}}\kern0pt}}}   %\textrm{\scriptsize{s.l.}}}
\def\vl{\textrm{v.l.}}   \def\varlec{\textrm{v.l.}} \def\varialectio{\textrm{v.l.}}
\def\vide{\textrm{{cf.\ }}}
\def\cf{\textrm{{cf.\ }}} 
\def\videtur{\textrm{{vid.\,ut}}}
\def\crux{\textup{[\ldots]} }
\def\cruxx{\textup{[\ldots]}}
\def\unm{\textit{unm.}}
%%%%%%%%%%%%%%%%%%%%%%%%%%%%%%%%%%%%

% List of Scholars
\DeclareScholar{ego}{ego}[
forename=Nils Jacob,
surname=Liersch]

% Persons:14\DeclareScholar{ego}{ego}[15forename=Robert,16surname=Alessi]17% Useful shorthands:18\DeclareShorthand{codd}{codd.}{V,I,R,H}19\DeclareShorthand{edd}{edd.}{Lit,Erm,Sm}20\DeclareShorthand{egoscr}{\emph{scripsi}}{ego}

%Useful shorthands:
%\DeclareShorthand{codd}{codd.}{V,I,R,H}
%\DeclareShorthand{edd}{edd.}{Lit,Erm,Sm}
\DeclareShorthand{egoscr}{em.}{ego}
\DeclareShorthand{egoscrconj}{conj.}{ego}
\DeclareShorthand{egomute}{\unskip}{ego}

\usepackage{xparse}

\NewDocumentEnvironment{tlg}{O{}O{}}{\setlength{\leftskip}{0pt}\vspace{-1ex}\begin{quotation}}{\hfill #1\ \vspace{-1ex}\end{quotation}\vspace{-1ex}} %verse environment
%\NewDocumentEnvironment{tlg}{O{}O{}}{\begin{verse}}{॥#1\hskip-4pt ॥\\ \end{verse}}
\NewDocumentCommand{\tl}{m}{{\selectlanguage{iast} #1}}

\NewDocumentCommand{\extra}{m}{{\textcolor{gray}{#1}}} %command for additions to U2
\NewDocumentCommand{\crazy}{m}{{\textcolor{red}{#1}}} %totally corrupted passage
\NewDocumentCommand{\coro}{m}{{\textcolor{violet}{#1}}} %colour for sentence counter! 

\NewDocumentEnvironment{prose}{O{}}{\begin{otherlanguage}{iast}}{\end{otherlanguage}}
% \NewDocumentEnvironment{padd}{O{}}{\begin{otherlanguage}{iast}}{\end{otherlanguage}}
\NewDocumentEnvironment{tlate}{O{}}
%\NewDocumentEnvironment{tadd}{O{}}

%Define two commands: \skp ("sanskrit plus"), to be ignored by TeX in
%the edition text, but processed in the TEI output. Conversely, \skm
%("sanskrit minus") is to be processed in the edition text, but
%ignored if found in the apparatus criticus and in the TEI output:

\NewDocumentCommand{\skp}{m}{}
\TeXtoTEIPat{\skp {#1}}{#1}

%\NewDocumentCommand{\skpp}{m}{}
%\TeXtoTEIPat{\skpp {#1}}{#1}

\NewDocumentCommand{\skm}{m}{\unless\ifinapparatus#1-\fi}
\TeXtoTEIPat{\skm {#1}}{}

% \NewDocumentCommand{\dd}{}{/\hskip-4pt/}
\NewDocumentCommand{\dd}{}{\mbox{/\hskip-4pt/}}
\TeXtoTEIPat{\dd {}}{//}


%%% modify environments and commands
%%% TEI mapping
\TeXtoTEIPat{\begin {tlg}}{<lg>} %lg=(Group of verse (s)) contains one or more verses or lines of verse that together form a formal unit (e.g. stanza, chorus).
\TeXtoTEIPat{\end {tlg}}{</lg>}

\TeXtoTEIPat{\begin {prose}}{<p>}
\TeXtoTEIPat{\end {prose}}{</p>}

\TeXtoTEIPat{\begin {tlate}}{<p>}
\TeXtoTEIPat{\end {tlate}}{</p>}

\TeXtoTEIPat{\\}{}
\TeXtoTEIPat{\linebreak}{<br/>}
\TeXtoTEIPat{\noindent}{}
%\TeXtoTEI{tl}{l}
\TeXtoTEI{emph}{hi}
\TeXtoTEI{bigskip}{}
\TeXtoTEI{None}{N1}
\TeXtoTEI{Ntwo}{N2}
\TeXtoTEI{Done}{D1}
\TeXtoTEI{Dtwo}{D2}
\TeXtoTEI{Uone}{U1}
\TeXtoTEI{Utwo}{U2}
%\TeXtoTEIPat{/}{ |}
%\TeXtoTEI{//}{ ||}
\TeXtoTEIPat{\korr}{em. }
\TeXtoTEIPat{\conj}{conj.}
\TeXtoTEIPat{\om}{om.}
\TeXtoTEIPat{english}{}
\TeXtoTEIPat{\hskip}{}
\TeXtoTEIPat{\hskip-4pt}{}
\TeXtoTEIPat{\hskip-2pt}{}
\TeXtoTEIPat{-}{ }
\TeXtoTEIPat{4pt}{}
\TeXtoTEIPat{2pt}{}
\TeXtoTEIPat{\textcolor {#1}{#2}}{<hi rend="#1">#2</hi>} 

% Nullify \selectlanguage in TEI as it has been used in
% \DeclareWitness but should be ignored in TEI.
\TeXtoTEI{selectlanguage}{}



\author{Nils Jacob Liersch}
\title{Yogatattvabindu of Rāmacandra\\ A Critical and Synoptic Edition and Annotated Translation}
\date{\today}

\parindent=15pt
\begin{document}

% Zitiermöglichkeiten:
%\footcite[See][p.\,1]{goldstein01:_tibet_englis_diction_moder_tibet}
%\footnote{\cite{goldstein01:_tibet_englis_diction_moder_tibet}.}

\frontmatter
\thispagestyle{empty}
\begin{center}
  {\Large \emph{The Yogatattvabindu}}\\[3mm]
\end{center}



\newpage

\

\thispagestyle{empty}



\normalsize


\newpage


\begin{center}
\thispagestyle{empty}

\

\vskip 2mm

\begin{otherlanguage}{iast}
\LARGE \sanskritfont{Yogatattvabindu}
\end{otherlanguage}

\vskip .4cm

\Huge Yogatattvabindu \\[7mm]
\Large Critical and Synoptic \\
Edition with annotated Translation


\large

\vspace{3cm}

Von

Nils Jacob Liersch
\small
\vfill

\vfill

Indica et Tibetica Verlag \\ % $\cdot$ 
Marburg 2024

\vskip 6mm

\end{center}

\newpage
\newpage \ \thispagestyle{empty}
\small  \

\noindent

\
\vfill


\small
\noindent \textbf{Bibliographische Information Der Deutschen Bibliothek}

\noindent
Die Deutsche Bibliothek verzeichnet diese Publikation in der Deutschen Nationalbibliographie;
detaillierte bibliographische Informationen sind im Internet über http://dnb.ddb.de abrufbar.

\noindent
\textbf{Bibliographic information published by Die Deutschen Bibliothek}

\noindent
Die Deutsche Bibliothek lists this publication in the Deutsche Nationalbibliographie; detailed
bibliographic data is available in the Internet at http://dnb.ddb.de.  


\vskip 1cm

\noindent
\copyright\ Indica et Tibetica Verlag, Marburg 2024

\medskip

\noindent
Alle Rechte vorbehalten / All rights reserved

\medskip

\noindent
Ohne ausdrückliche Genehmigung des Verlages ist es nicht gestattet, das Werk oder einzelne Teile
daraus nachzudrucken, zu vervielfältigen oder auf Datenträger zu speichern.

\smallskip

\noindent
Apart from any fair dealing for the purpose of private study, research, criticism or review, no
part of this book may be reproduced or translated in any form, by print, photo form, microfilm, or
any other means without written permission. Enquiries should be made to the publishers.

\bigskip

\noindent
Satz: \ \ Nils Jacob Liersch \\
Herstellung: \ \ BoD – Books on Demand GmbH, Norderstedt  \\

\bigskip

\noindent
%\ISBN     

\normalsize

\newpage

%\maketitle
\clearpage
\tableofcontents
\addtocounter{page}{-1}
\thispagestyle{empty}
\clearpage

\chapter{Introduction}
\mainmatter

\begin{quote}
nādakoṭisahasrāṇi bindukoṭiśatāni ca/
sarve tatra layaṃ yānti yatra devo nirañjanaḥ//
\end{quote}

Thousands of crores of resonances and hundreds of crores of visual focal points, all dissolve into the place where the unadorned god is.

\citetitle{hathapradipika2024}

\chapter{The List of the 15 Yogas}
\label{yogas_list}
The authenticity of the list specifying the fifteen Yogas at the beginning of the text is ambiguous. This is due to the discrepancy between the structure of the Yogas presented in the text and the order presented in the list. For example, the text commences with a description of \textit{kriyāyoga} and goes on to describe \textit{siddhakuṇḍaliniyoga} and then mentions \textit{mantrayoga} without adhering to the order presented in the list. This incongruity raises questions as to why the text structure deviates from the list. However, the reference to \textit{jñānotpattav upāyaḥ} may provide some insight into why \textit{jñānayoga} is included as the second \textit{yoga} in the list. To reconcile these apparent inconsistencies, there are several possible explanations: 1) The text is severely corrupted. 2) The list was added by a different hand at a later time. 3) The term \textit{jñānayoga} is included as a result of the practice of \textit{siddhakuṇḍalinīyoga}, which is said to generate knowledge through the central channel, as stated in the text. These explanations may be combined to provide a comprehensive understanding of the situation.

\section{Lakṣyayoga}

\begin{itemize}
\item origin tantric Traditions -> e.g. Netratantra
\item also check Mālinivijayottara 2004 Vasudeva pp. 256-257
\item also \citetitle{birch2013} 2.10 Śāmbhavī Mudrā
  \end{itemize} 

\chapter{Sources}
\section{The Additions of  SORI 6082 - U\textsubscript{2}}
\label{discussionu2}
Analyse the additions of U\textsubscript{2} and present the \textit{cakra}s and their attriubutes in a table .
\begin{itemize}
\item  Muktabodha-Texte sehe ich 3 Belege für bahiśśakti Muktabodha/krīyakramādyotikā.html 2938 suṣirānte bahiśśaktiṃ vinyasedvyomarūpiṇīm | tasyā madhye tu Muktabodha/sakalāgamasārasaṅgraha.html 2186 suṣirāntabahiśśaktiṃ vyāpinīṃ cintayet tataḥ || Muktabodha/kriyakramadyotikavyākhyā.html 1846 tanmadhye ca bahiśśaktiṃ sudhābindu parisrutim
  \item  Parā\footnote{Im Kaśm. Śiv. °das ewige Wort, in welchem potentiell alle Begriffe und Worte ruhen; vgl. das śabdabrahma des Vyākaraṇa. [B.]― Schmidt S. 246}.
  \end{itemize}

\chapter{Conventions in the Critical Apparatus}
\section{Sigla in the Critical Apparatus}

\begin{itemize}
\item E : Printed Edition
\item P : Pune BORI 664
\item L : Lalchand Research Library LRL5876
\item B : Bodleian Oxford D 4587
\item \None : NGMPP B 38-31
\item \Ntwo : NGMPP B 38-35 / A 1327-14
\item \Done : IGNCA 30019
\item \Uone : SORI 1574
\item \Utwo: SORI 6082
\end{itemize}

The order of the readings in the critical apparatus is arranged according to the quality of readings in decending order. The critical apparatus is positive. Gemitation is not recorded. 

\section{Abbreviations}
\begin{itemize}
  \item qcr: quote cum notatio (quoted with reference)
  \end{itemize}

\section{Marking the Reliability of Sources and Testimonia in the Critical Apparatus}
\label{kennz}

To accurately depict information about the textual relationship and estimated degree of relatedness of a passage from the \textit{Yogatattvabindu} in the layers for sources and testimonia of the critical apparatus, a system of sigla was introduced.\footnote{This type of identification system is based on the use of the critical apparatus in \parencite[lii-liii]{steinkellner2005}. It was modified for the text-critical work on the \textit{Yogatattvabindu}.} The sigla are meaningful when a passage is corrupted in all witnesses and can only be reconstructed by means of other texts. The layers of the critical apparatus for sources and testimonia use the following sigla:

\begin{enumerate}
\item[\textbf{Ce}] \textit{citatum ex alio} / quotation from another (text).\footnote{The sigla \textbf{Ce} indicates an identical or largely identical content in the lesser witness and only allows for minor deviations in the wording of the passage.}
\item[\textbf{Cee}] \textit{citatum ex alio modo edendi} / quotation from another (text) with editorial changes.\footnote{The sigla \textbf{Cee} identifies passages with noticeable deviations in the lesser witness.}
\item[\textbf{Ci}] \textit{citatum in alio} / quotation in another (text).\footnote{The sigla \textbf{Ci} indicates an identical or largely identical content in the lesser witness and only allows for minor deviations in the wording of the passage.}
\item[\textbf{Cie}] \textit{citatum in alio modo edendi} / quotation in another (text) with editorial changes.\footnote{The sigla \textbf{Cie} identifies passages in the lesser witness with noticeable deviations that have the intended character of the composer.}
\item[\textbf{Re}] \textit{relatum ex alio} / (content), attested from another text.\footnote{The sigla \textbf{Re} identifies content parallels in the lesser witness that are relevant to the constitution of the critical text. It further indicates in certain cases that the composer might have used this source when composing his text.}
\item[\textbf{Ri}] \textit{relatum in alio} / (content), attested in another text.\footnote{The sigla \textbf{Ri} identifies content parallels in the lesser witness that are relevant to the constitution of the critical text.}
\end{enumerate}

The following acronyms refer to passages that originated from texts that the author of the \textit{Yogatattvabindu} utilized in compiling his work: \textbf{Ce}, \textbf{Cee}, \textbf{Re}. These texts must predate the \textit{Yogatattvabindu}. The other acronyms, such as \textbf{Ci}, \textbf{Cie}, and \textbf{Ri}, are texts that have adopted passages from the \textit{Yogatattvabindu}, or verses or passages that share similar content with the \textit{Yogatattvabindu}, but their relation is given literally, making it impossible to determine who adopted from whom. \textbf{Re} and \textbf{Ri} each refer to passages that are so closely related in content to those of the \textit{Yogatattvabindu} that they are significant in reconstructing a passage.\footnote{\textbf{Ce} and \textbf{Cee} have the highest degree of reliability, \textbf{Ci} and \textbf{Cie} have a moderate degree, and \textbf{Re} and \textbf{Ri} have the lowest.}

\section{Punctuation}

The inconsistent use of punctuation marks in the available witnesses necessitates standardization. Upon close examination, it appears that punctuation has frequently been dropped or added during the transmission of the texts. The neglect or improper handling of punctuation by the copists has resulted in different versions of lists with and without punctuation. In many instances, missing punctuation has led to the addition of case endings, alteration of the text, and the combination of list items into compound formations that were not present in the original text. Although punctuation plays an important role, deviations in punctuation at the end of sentences, lists, and verse-numbering will only be extensively documented in the critical apparatus of the printed edition. This means that emendations of obvious punctuation mistakes will not be recorded in the critical apparatus. However, the digital edition of this work provides a more detailed documentation of deviations in punctuation through diplomatic transcripts of each witness, and even has a function to display sentences cumulatively.

In the printed edition of the \textit{Yogatattvabindu}, standard conventions of punctuation are followed. In verse poetry, a \textit{daṇḍa} (|) marks the end of a half-verse or half of the \textit{śloka}, and a double \textit{daṇḍa} (||) marks the end of a verse. In prose, a single \textit{daṇḍa} indicates the end of a sentence, and a double \textit{daṇḍa} marks the end of a paragraph. Variations in the use of \textit{avagraha} will be recorded, and items in lists will be separated by a double-\textit{daṇḍa}.

\section{Sandhi}

Among the witnesses we see deviating and inconsistent application of \textit{sandhi}. There is no clear evidence that originally \textit{sandhi} was intentionally not applied. This edition will therefore apply \textit{sandhi} consistently throughout the constituted text to provide a readable text sticking to contemporary conventions in Sanskrit. The variant readings concerning \textit{sandhi} are recorded consistently in the apparatus criticus. This is due to various textcritical problems arising from the inconsistent usage of punctuation which results in application or non-application of \textit{sandhi} wheter the respective witness applied a \textit{daṇḍa} or not. This is particularly the case within lists, which frequently occur in our compilation. Items were most likely originally separated by \textit{daṇḍa}. 


\section{Class Nasals}

Due to inconsistent use of class nasals among the witnesses \textit{anusvāra}s have been substituted with the respective class nasals throughout the edition.

\section{Lists}

Lists are a frequent feature in the \textit{Yogatattvabindu}. The text opens with a list of 15 Yogas and there are many more lists utilized throughout its content. To produce a consistent and easily readable edition, all lists have been identified, normalized to the Nominative Singular or Nominative Plural form of the respective item, or in the case of explanatory lists, to the Ablative Singular or Plural. The items are separated by a double \textit{daṇḍa}. Differences in punctuation and simple punctuation emendations, unless they are text-critically or systematically significant, will not be recorded in the apparatus criticus.
\clearpage

\section{Structural Issues of the Yogatattvabindu}
\label{structure}
\chapter{Related Texts}
\section{Śivayogapradīpikā}
In the \citetitle{shivayogapradipika} 4.41cd-47ab we find descriptions closely resembling those of \citetitle{advaya}:
\begin{quote}
antarlakṣyam iti jñeyaṃ bahirlakṣyam atha śṛṇu ||41||\\
nāsāgradeśāc caturaḥ ṣaḍ aṣṭau tathā daśa dvādaśa saṃkhyayāṅguliḥ |\\
bahiḥ smaren nīlasudhūmraraktataraṅgapītābhasutattvapañcakam ||42||\\
athavā sanmukhākāśaṃ sthiradṛṣṭyā vilakṣayet |\\
jyotirmayūkhā dṛśyante yogibhir dhīramānasaiḥ ||43||\\
dṛṣṭyagre vāpy apāṅge vā taptakāñcanasaṃnibham | \\
bhūmiṃ saṃlakṣayed dṛṣṭiḥ sthirā bhavati yoginaḥ ||44||\\
athavā śirasaś cordhve dvādaśāṅgulasaṃmite |\\
jyotiḥpuñjaṃ nirākāraṃ lakṣayen muktidaṃ bhavet ||45||\\
yatra yatrārthavān yogī tatra tatra vilakṣayet |\\
ākāśam eva yas tasya cittaṃ bhavati tādṛśam ||46||\\
ity anekavidhākāraṃ bahirlakṣyam udīritam |\\
\end{quote}

Revise translation! see Powell 2023! 

``(41cd) That was the inner fixation. Now hear the external fixation that needs to be understood.(42) From the tip of the nose, counting with four, six, ten, and twelve, using the numerical system of the fingers. The five elements in [the colours of] outdoor blue, intense grey, wave of red and yellow mystery. (43) Alternatively, one may gaze steadily towards the space [directly] in front of [the face]. Luminous rays are perceived by steadfast-minded yogins. (44) In front of the gaze or at the outer corner of the eye space, resembling the shine of molten gold, the gaze should be fixed on the ground - [thus] stability arises for the yogin. (45) Alternatively, above the head, with a [distance of] twelve finger-breadths, one should fixate the formless cluster of light, which bestows liberation. (46) Wherever the yogin is suitable to the object, there he should fixate only space, in order for his mind to becomes as such. (47ab) Thus, various external fixations have been mentioned.''
\section{Netratantra}

Netratantra

Gavin Flood, Bjarne Wernicke-Olesen and Rajan Khatiwoda
Consultants: Alexis Sanderson, Diwakar Acharya

The Netratantra (NT), the ‘Tantra of the Eye’, is an important text in Kashmir and Nepal, dating from around the early ninth century, and widely disseminated during the eleventh and probably tenth centuries. The text takes its name from Śiva as Netranātha or ‘Lord of the Eye’. It was commented on by the Pratyabhijñā philosopher Kṣemarāja (c. 1000-1050) in his extant Netratantroddyota, that itself bears witness to its importance in his desire to bring the text into the orbit of his non-dualist metaphysics. The project will edit, translate, and describe its traditions as borne witness to in the Nepalese recension of the text. Alexis Sanderson has shown how the Netratantra was connected with royalty and used in the courts by Śaiva officiants in the role of royal priest or rājapurohita. That Śaiva and Mahāyāna gurus performed ‘apatropaic, restorative and aggressive Mantra rituals’ for the protection of king and kingdom is well attested in the kingdoms of south and south-east Asia from the ninth to eleventh century and the Netratantra is a text that bears witness to Śaiva gurus in the service of kings.[1] The principle use of the text would have been the protection of the king and his family through the propagation of its ritual procedures and particularly the recitation of the netra mantra (OṂ JUṂ SAḤ in the short version). Thus, the text is a ‘universal’ (sarvasāmānya-) tantra, which ‘overrides the distinctions between the various branches of the Mantramārga […] and that between the Mantramārga and the Kulamārga by propagating a form of worship for use by royal officiants that can be inflected as required to take on the character of any of these divisions and indeed of others outside Śaivism.’[2]

The text was first brought to our attention by Hélène Brunner who describes each chapter in some detail in her 1974 paper;[3] an extremely useful source for not only the contents of the text, but for her comments on its structure and relation to other texts, and has been researched by André Padoux in his studies of the correspondences between cosmos, sound, and body[3] and of the way the netramantra is formed. Somadeva Vasudeva has done research on yoga in the text, particularly the subtle visualization and subtle body of chapter seven,[5] as has James Mallinson.[6]

It is probable that the Netratantra was composed over a long period of time and the redactor is bringing together diverse elements into a whole. There are parallels between the Netra and the Svacchandatantra although more work on the parallels and influence of the Svacchanda needs to be done.[7] David White argues that the oldest or original section of the work is the material concerned with possession and exorcism[8] and this systematic treatment of possession is indeed a notable feature of it, akin to similar treatment in the Īśānaśivagurudevapaddhati Mantrapāda chapter 42.

The central deity of the Netratantra is Amṛteśvara, called Amṛtīśa in the Nepalese recension, also known as Amṛteśabhairava, Mṛtyunjit, and Mṛtyuñjaya, whose consort is Lakṣmī/Śrī called Amṛtalakṣmī in ritual manuals based on the text.[9] After an initial chapter in which Amṛteśvara, referred to as Bhairava, responds to the questions of the Goddess by extolling the virtues and powers of Śiva’s eye, the text presents a number of visualisations of a number of deities, catholic in its range, not only from the systems of the Mantramārga but from Vaiṣṇava traditions as well.[10] Furthermore, a strong Śākta influence is evident in the text with its many references to deities and practices characteristic of the Kulamārga (e.g. chapter 7 on the subtle visualising meditation and chapter 20 on yoginīs).

The project to study the text will especially focus on the theme of models of the person or self that the text entails. Based on close philological reading, we hope to account for different understandings of the person implicit in the text. Chapters on ritual and meditation reflect the understandings of the person in the wider community of which the text is an index. In particular, three chapters, six, seven, and eight, that the text calls the mundane or gross meditation (sthūladhyānam), the subtle meditation (sūkṣmadhyānam), and the supreme meditation (para­dhyānam), correspond to three types or levels of the body, gross, subtle and supreme.[11] It seems that this threefold hierarchical structure is an attempt to order a range of practices that the Netra is incorporating and it does so with some coherence. The lowest level of meditation practice is concerned with magical protection (primarily of the king [6.35] and his family) from demonic beings. This involves the practitioner, the Sādhaka or Mantrin, constructing diagrams within which the name of the person to be protected is written along with other rites of appeasement (śāntiḥ) and prosperity (puṣṭiḥ). The subtle level concerns the visualisation of the body and the powers moving within it. The subtle meditation is especially interesting because it presents two different systems of visualisation, one in which subtle energy rises up through the body, piercing the levels to the location of Śiva at the crown of the head and a second in which that same power rising through the body releases nectar at the crown of the head that then floods the body.[12] In his commentary Kṣemarāja calls these the tantra-prakrīyā and the kula-prakrīyā respectively, the latter being an index of the Śākta kulamārga. Finally, the supreme meditation is principally a reinterpretation of the ‘limbs’ of classical yoga from the perspective of supreme reality, the level of Śiva.[13] All of these entail distinct understandings of what a person is (e.g. a permeable self in ch. 6 and 19, a processual self in ch. 7 and a gnostic self in ch. 8).

There are two major recensions of the text, one in Kashmir (where four manuscripts exist to our knowledge) and one in Nepal where again there are four manuscripts (to be described presently). These have been preserved by the Nepal-German Manuscript Preservation Project (NGMCP). The Nepalese manuscripts probably represent an older recension of the text, a judgement based on its slightly less polished language, which the Kashmiris have amended at times in the interests of producing a better text although Sanderson argues for the Kashmir origin of the text between 700 and 850 AD.[14] Of the four Nepalese witnesses, the oldest is a palm leaf manuscript (N1) of which there is a much more recent (19th century?) devanāgarī apograph (N2). N1 is dated to February or March 1200, the copying being done by Pandit Kīrttidhara, commissioned by the author of a ritual manual Viśveśvara, and completed during Caitra in saṃvat 320 (= 1200 AD).[15] Often the Kashmir reading is better semantically and grammatically, but we intend to preserve the text as it stands while noting the Kashmir variants.

Project output:
A full annotated translation of the Netratantra with an introduction in two volumes in the Routledge Studies in Tantric Traditions series.

[1] Alexis Sanderson, ‘Religion and the State: Śaiva Officiants in the Territory of the King’s brahmanical Chaplain,’ p. 238, Indo-Iranian Journal vol. 47, 2004, pp. 229-300. This is corroborated by texts such as the Amṛteśadīkṣāvidhi that prescribe initiation and ritual for the royal family (p. 241).
[2] Alexis Sanderson, ‘The Śaiva Literature,’ p. 30, Journal of Indological Studies, Nos. 24 \& 25 (2012–2013), pp. 1-113.
[3] Hélène Brunner, ‘Un Tantra du Nord: le Netra Tantra’, Bulletin l’École Français d’Extreme Orient, vol. 61, 1974, pp. 125-97.
[4] André Padoux, Vāc: A Study of the Word in Selected Hindu Tantras, trans. J. Gontier (Albany: SUNY Press, 1991). Also, his useful and lucid paper ‘Corps et cosmos: l’image du corps du yogin tantrique,’ in V. Boullier and Gilles Tarabout (eds.), Images du corps dans le monde hindou (Paris: CNRS, 2002), pp. 163-87. See also Gavin Flood, ‘Body, Breath, and Representation in Śaiva Tantrism,’ in Axel Michaels and Christoph Wulf (eds.), Images of the Body in India (London: Routledge, 2011), pp. 70-83.
[5] Somadeva Vasudeva, ‘The Śaiva Yogas and their Relation to Other Systems of Yoga,’ pp. 7-8, RINDAS Series of Working Papers, Traditional Indian Thought 26, 2017, pp. 1-16.
[6] James Mallinson and Mark Singleton, The Roots of Yoga (London: Penguin, 2017), ch 5.
[7] André Padoux, Tantric Mantras (London: Routledge, 2011), pp. 90. 95.
[8] David White, ‘Netra Tantra at the Crossroads of the Demonological Cosmopolis,’ Journal of Hindu Studies, vol. 5, 2012, pp. 145-71.
[9] Sanderson, ‘Religion and the State,’ p. 239, n. 18.
[10] For example, it describes Viṣṇu as a sixteen-year old, ityphallic youth seated on a ram (13.10-13b), as well as visualisations of Tumburu and his sisters (chapter 11).
[11] Padoux (2002, p. 172) cites Kṣemarāja’s commentary on the Śivasūtra 3.4 where a triple body is related to the cosmic hierarchy.
[12] Bjarne Wenicke-Olesen has referred to the latter as being a ‘Śākta anthropology’ that can be contrasted with the earlier idea of the retention of semen (bindu) in the head. In an article with Silje Lyngar Einarsen he writes: ‘Es zeigt sich, daß eine ursprüngliche oder frühe Binduyoga-Anthropologie, die auf das Zurückhalten des Samens (bindhudhāraṇa) ausgerichtet war, von einem mit dem Kuṇḍalinī-System verknüpften Śākta-Anthropologie ersetzt wird, die auf die Überströmung des Körpers mit Unsterblichkeitselexir (amṛtaplavana) ausgerichtet ist’ (Wernicke-Olesen, B. and S. L. Einarsen. 2018. ’Übungswissen in Yoga, Tantra und Asketismus des frühen indischen Mittelalters’, in A.-B. Renger and A. Stellmacher (eds), Übungswissen in Religion und Philosophie: Produktion, Weitergabe, Wandel, pp. 241-257. Berlin: LIT Verlag). Also see James Mallinson, ‘Śāktism and Haṭha Yoga’ in B. Wernicke-Olesen (ed.), Goddess Traditions in Tantric Hinduism: History, Practice and Doctrine (London: Routledge, 2015), pp. 109-40.
[13] Vasudeva has written on the six ancillaries of yoga. Concerning those in the Netratantra he observes that ‘it may actually be more appropriate to compare the eight ancillaries of the Netratantra with the formulaic dhāraṇās taught in the Vijñānabhairava, which show an even greater tendency towards the transcendence of the inherited complex of ritual and yogic procedures’ (Vasudeva 2004, p. 382).
[14] Sanderson, ‘Religion and the State,’ p. 242.
[15] N1 folio 49. Amṛteśatantra, NAK MS 1-285, NGMPP Reel No. B 25/5. Palm Leaf; Nepalese variant of proto-Bengali script, 1200 AD (= Saṃvat 320). NAK 5-4866, NGMPP Reel No. A 171/12.

Link to chapter 7: Netratantra VII: Subtle Visualisation (sample chapter)
The Lord of Immortality: An Introduction, Critical Edition, and Translation of the Netra Tantra, chapter 7. Critically edited, translated and introduced by Gavin Flood, Bjarne Wernicke-Olesen, Rajan Khatiwoda (Oxford: OCHS 2019).
https://saktatraditions.org/netratantra/


\chapter{notes}
4.9.6 The Śivatattvaratnākara
The Śivatattvaratnākara is a large compendium attributed to a king named Keḷadi Basavabhūpāla (also
known as Basavarāja, Basavāppa Nāyaka I) who reigned from 1696–1714 in Ikkeri, Karnataka. In the
seventh chapter of the Śivatattvaratnākara, in a section providing instructions on yoga for the king, a
large portion of the Śivayogapradīpikā is quoted. 338 The Śivatattvaratnākara also at times provides
further details or interpretations of the verses, for example, supplying the mantras referred to in
Śivayogapradīpikā 1.5. 339 \textbf{The text thus provides an intriguing early modern example of the adapation of
yoga in a non-ascetic and courtly environment.}

Powell 2024:146

\chapter{Critical Edition \& Annotated Translation}
\cleardoublepage
\begin{alignment}[
  texts=edition[class="edition"];
  translation[class="translation"],
  ]
  \begin{edition}
  \ekddiv{type=ed}
      \centerline{\textrm{\small{[\uproman{30}.\textsuperscript{\coro{\lowroman{1}-\lowroman{11}}}The Order of Cakras]}}}
  \bigskip
\begin{prose}
\noindent
%-----------------------------
%idānīṃ cakrāṇām anukramaḥ  kathyate/    \E
%idānīṃ cakrāṇām anukramaḥ  kathyate     \P
%idānīṃ cakrāṇām anukramaḥ//             \B
%idānīṃ cakrāṇām anukramaḥ//             \L 19.jpg 
%idānīṃ cakrāṇām anukrama   kathyaṃte/   \N1
%idānīṃ cakrāṇām anukramā   kathyaṃte//  \D
%idānīṃ cakrānām-anukramā   kathyaṃte/   \N2
%idānīṃ cakrānām anukramaḥ  kathyate     \U1
%idānīṃ cakrānām anukramaḥ  kathyate//   \U2
%-----------------------------
%Now the sequence of the cakras is explained. 
%-----------------------------
idānīṃ cakrānā\skp{m-a}\app{\lem[wit={ceteri}, alt={anukramaḥ}]{\skm{m-a}nukramaḥ}
  \rdg[wit={N1}]{anukrama}
  \rdg[wit={D,N2}]{anukramā}}
\app{\lem[wit={ceteri}]{kathyate}
  \rdg[wit={D,N1,N2}]{kathyaṃte}}/ 
%\note[type=testium, labelb=199, lem={\textbf{Re}}]{SSP 2.1 (Ed. p. 29): atha piṇḍavicāraḥ kathyate piṇḍe navacakrāṇi |}
\note[type=philcomm, labelb=200, lem={\uproman{30}.\textsuperscript{\lowroman{1}-\lowroman{11}}}]{Once again Rāmacandra presents descriptions of the \textit{cakra}s, but this time very brief covering merely the names and locations of each \textit{cakra}. Given their detailled description in chapters \uproman{4}-\uproman{12} it seems redundant to mention them once again at this place. Either we could assume that Rāmacandra was a very thoughtless and unstructured author/compiler, or this phenomenon must be taken as a further indication that Rāmacandra had an educational intention for his audience and that this text was used as a textbook so that he immediately used the preceding verse to repeat the sequence of the \textit{cakra} system that he already discussed in detail at the beginning of the text.}
%-----------------------------
%ādhāre brahmacakram/    ādhāropari liṃgamūle sbādhiṣṭhānacakram/     nābhau maṇipūrakacakram/     hṛdaye anāhatacakram/     kaṇṭhasthāne viśuddhicakram/     \E
%ādhāre brahmacakraṃ 1   ādhāropari liṃgamūle svādhiṣṭhānacakram 2    nābhau maṇipūrakacakraṃ      hṛdaye 'nāhatacakraṃ 4    kaṃṭhasthāne viśuddhicakraṃ 5    \P
%ādhāro brahmacakram/    ādhāropari liṃgamūle svādhiṣṭhānacakraṃ//2// nābhau maṇipūrakacakram//3   hṛdaye anāhatacakram// 4  kaṇṭhasthāne viśuddhicakraṃ//    \B
%ādhāro brahmacakram//   ādhāropari liṃgamūle svādhiṣṭhānacakraṃ//2// nābhau maṇipūrakacakram//3// hṛdaye anāhatacakram//4// kaṇṭhasthāne viśuddhacakraṃ//    \L
%ādhāre brahmacakraṃ                liṃge     svādhiṣṭhānacakram/     nābhau maṇipūrakacakram/     hṛdaye viśuddhacakraṃ/    kaṇṭhasthāne anāhatacakraṃ/      \N1
%ādhāre brahmacakraṃ                liṃge     svādhiṣṭhānacakram//    nābhau maṇipūrakacakraṃ//    hṛdaye viśuddhacakraṃ//   kaṃṭhasthāne anāhatacakraṃ//     \D
%ādhāre brahmacakraṃ                liṃge     svādhiṣṭhānacakram//    nābhau maṇipūrakacakram/     hṛdaye viśuddhacakraṃ/    kaṇṭhasthāne anāhatacakraṃ       \N2
%ādhāre brahmacakraṃ                liṃge     svādhiṣṭhānacakraṃ      nābhau maṇipūrakacakraṃ      hṛdaye viśuddhacakraṃ     kaṇṭhasthāne anāhatacakraṃ       \U1
%ādhāre brahmacakraṃ//1// ādhāropariliṃgamūle svādhiṣṭhānacakraṃ//2// nābhau maṇipūrakacakraṃ//3// hṛdaye anāhatacakraṃ//4// kaṇṭhasthāne viśuddhacakraṃ//5// \U2
%-----------------------------
%At the support there is the Brahmacakra. Above the support at the root of the gender is the Svadiṣṭhānacakra. At the navel there is the Maṇipūrakacakra. At the heart the Anāhatacakra. Situated within the throat is the Viśuddhicakra. 
%-----------------------------
%\note[type=source, labelb=201, lem={brahmacakram}]{SSP: ādhāre brahmacakraṃ tridhāvartaṃ bhagamaṇḍalākāram | tatra mūlakandaḥ | tatra śaktiṃ pāvakākārāṃ dhyāyet | tatraiva kāmarūpapīṭhaṃ sarvakāmaphalapradaṃ bhavati ||2.1||}
\app{\lem[wit={ceteri}]{ādhāre}
  \rdg[wit={B,L}]{ādhāro}}
\app{\lem[wit={B,E,L}]{brahmacakram}
  \rdg[wit={ceteri}]{brahmacakraṃ}} 1 \dd{} 
\app{\lem[wit={ceteri}]{ādhāropari}
  \rdg[wit={D,N1,N2,U1}]{\om}}
\app{\lem[wit={ceteri}]{liṅgamūle}
  \rdg[wit={D,N1,N2,U1}]{liṅge}}
%\note[type=source, labelb=202, lem={svādhiṣṭhāna°}]{SSP: dvitīyaṃ svādhiṣṭhānacakram | tanmadhye paścimābhimukhaṃ liṅgaṃ pravālāṅkurasadṛśaṃ dhyāyet | tatraivoḍyānapīṭhaṃ jagadākarṣaṇaṃ bhavati ||2.2||}
\app{\lem[wit={E,D,P,N1,N2}]{svādhiṣṭhānacakram}
  \rdg[wit={ceteri}]{svādhiṣṭhānacakraṃ}} 2 \dd{}
%\note[type=source, labelb=203, lem={maṇipūraka°}]{SSP:tṛtīyaṃ nābhicakraṃ pañcāvartaṃ sarpavat kuṇḍalākāram | tanmadhye kuṇḍalinīṃ śaktiṃ bālārkakoṭisannibhāṃ dhyāyet | sā madhyā śaktiḥ sarvasiddhidā bhavati ||2.3||}
nābhau \app{\lem[wit={E,L,P,N1,N2}]{maṇipūrakacakram}
  \rdg[wit={ceteri}]{maṇipūrakacakraṃ}} 3 \dd{} 
%\note[type=source, labelb=204, lem={anāhata°}]{SSP: caturthaṃ hṛdayacakram aṣṭadalakamalam adhomukhaṃ tanmadhye karṇikāyāṃ liṅgākārāṃ jyotīrūpām dhyāyet | saiva haṃsakalā sarvendriyavaśyā bhavati ||2.4||}
hṛdaye
\app{\lem[wit={P}, alt={'nāhata°}]{'nāhata}
  \rdg[wit={B,E,L,U2}]{anāhata°}
  \rdg[wit={ceteri}]{viśuddha°}
}\app{\lem[wit={B,E,L}]{cakram}
  \rdg[wit={ceteri}]{cakraṃ}} 4 \dd{}\textsuperscript{\begin{otherlanguage}{english}\coro{[\lowroman{5}]}\end{otherlanguage}}
kaṇṭhasthāne
\app{\lem[wit={Y}]{viśuddhicakram}
  \rdg[wit={X}]{anāhatacakraṃ}} 5 \dd{} 
%\note[type=source, labelb=205, lem={viśuddhi°}]{SSP: pañcamaṃ kaṇṭhacakraṃ caturaṅgulam | tatra vāma iḍā candranāḍī | dakṣiṇe piṅgalā sūryanāḍī | tanmadhye suṣumnāṃ dhyāyet | saiva anāhatakalā anāhatasiddhidā bhavati ||2.5||}
%-----------------------------
%ṣaṣṭhaṃ tālucakram/     bhruvor madhye ājñācakram/       brahmasthāne        kālacakram/     navamam         ākāśacakram/       etat--paraṃ śūnyam/              \E
%ṣaṣṭhaṃ tālucakraṃ 6    bhruvor madhye agnejacakraṃ 7    brahmasthāne        kālacakraṃ 8    navamaṃ         ākāśacakraṃ 8      tataḥ paraṃ śūnyaṃ               \P
%ṣaṣṭhaṃ tālucakre/6     bhruvor madhye ājñāyacakraṃ/     brahmasthāne        kālacakraṃ// 8  navamaṃ         ākāśacakraṃ/9      tat---paraṃ śūnyam/              \B
%ṣaṣṭha  tālucakre//6//  bhruvor madhye āgneyacakraṃ//7// brahmasthāne        kālacakraṃ//8// navamaṃ         ākāśacakraṃ//9//   tat---paraṃ śūnyam//             \L
%ṣaṣṭhaṃ tālucakram/     bhruvor madhye ājñācakram        brahmaraṃdhrasthāne kālacakraṃ/     navamaṃ         ākāśacakram/       tat---paramaśūnyaṃ/              \N1
%ṣaṣṭhaṃ tālucakraṃ//    bhruvor madhye ājñācakraṃ//      brahmaraṃdhrasthāne kālacakraṃ//    navamaṃ         ākāśacakram/       tat---paraṃ// tatparamaśūnyaṃ// \D
%ṣaṣṭhaṃ tālucakram/     bhruvor madhye ājñācakram        brahmaraṃdhrasthāne kālacakraṃ/     navama          ākāśacakram       tata---paraśūnyaṃ/               \N2
%ṣaṣṭhaṃ tālucakraṃ      bhruvor madhye ājñācakram        brahmaraṃdhrasthāne brahmacakraṃ    navamaṃ rattu?! ākāśacakram         tat--paraśūnyaṃ                \U1
%        tālucakra //6// bhruvor madhye ājñācakram//7//   brahmaraṃdhrasthāne kalācakraṃ//8//                 ākāśacakram ūrdhvaṃ tat--paraṃ śūnyaṃ//9//         \U2
%-----------------------------
%The sixth is the cakra of the palate. In the center of the eyebrows is the Ājñācakra. At the opening of Brahma is the Kālacakra. The ninth is the Ākāśacakra. It is supreme emptiness. 
%-----------------------------
%\note[type=source, labelb=206, lem={tālu°}]{SSP: ṣaṣṭhaṃ tālucakram | tatrāmṛtadhārāpravāhaḥ | ghaṃṭikāliṅgaṃ mūlarandhraṃ rājadantaṃ śaṃkhinīvivaraṃ daśamadvāram | tatra śūnyaṃ dhyāyet | cittalayo bhavati ||2.6||}
\app{\lem[wit={ceteri}]{ṣaṣṭhaṃ}
  \rdg[wit={L}]{ṣaṣṭha°}}
\app{\lem[wit={E,N1,N2}]{tālucakram}
  \rdg[wit={D,P,U1}]{tālucakraṃ}
  \rdg[wit={B,L}]{tālucakre}
  \rdg[wit={U2}]{tālucakra}} 6 \dd{} 
bhruvor-madhye
%\note[type=source, labelb=207, lem={ājñā°}]{SSP: aptamaṃ bhrūcakraṃ madhyamāṅguṣṭhamatram | tatra jñānanetraṃ dīpaśikhākāraṃ dhyāyet | tatra vāksiddhir bhavati ||2.7||}
\app{\lem[wit={ceteri}, alt={°ājñā}]{ājñā}
  \rdg[wit={P}]{agneja}
  \rdg[wit={L}]{āgneya}
  \rdg[wit={B}]{ājñāya}
}\app{\lem[wit={D,E,N1,N2,U1,U2}]{cakraṃ}
  \rdg[wit={B,D,P,L}]{cakram}}  7 \dd{}
%\note[type=source, labelb=208, lem={brahmarandhra°}]{SSP: aṣṭamaṃ brahmarandhraṃ nirvāṇacakraṃ sūcikāgrabhedyam | tatra dhūmaśikhākāraṃ dhyāyet | tatra jālandharapīṭhaṃ mokṣapradaṃ bhavati ||2.8||}
brahma\app{\lem[wit={ceteri}, alt={°randhra°}]{randhra}
  \rdg[wit={B,E,L,P}]{\om}}sthāne
\app{\lem[wit={ceteri}, alt={°kāla}]{kāla}
  \rdg[wit={U1}]{brahma°}
}\app{\lem[wit={E}]{cakram}
  \rdg[wit={ceteri}]{cakraṃ}} 8 \dd{}
%\note[type=source, labelb=209, lem={ākāśa°}]{SSP: navamam ākāśacakraṃ soḍaśadalakamalam ūrdhvamukham | tanmadhye karṇikāyāṃ trikūṭākārāṃ tadūrdhvaśaktiṃ tāṃ paramaśunyāṃ dhyāyet | tatraiva pūrṇagiripīṭhaṃ sarveṣṭasiddhir bhavati ||2.9|| iti navacakravicāraḥ ||}
\app{\lem[wit={E}, alt={navamam}]{navama\skp{m-ā}}
  \rdg[wit={N2}]{navama}
  \rdg[wit={U1}]{navamaṃ rattu}
  \rdg[wit={ceteri}]{navamaṃ}
}\skm{m-ā}kāśa\app{\lem[wit={D,E,N1,N2,U1,U2}]{cakram}
  \rdg[wit={B,L,P}]{cakraṃ}} 9 \dd{}\textsuperscript{\begin{otherlanguage}{english}\coro{[\lowroman{10}]}\end{otherlanguage}}\\
\app{\lem[wit={B,D,L,N1,U1,U2}, alt={tat°}]{ta\skp{t-pa}}
  \rdg[wit={E}]{etat}
  \rdg[wit={P}]{tataḥ}
  \rdg[wit={N2}]{tata}
}\app{\lem[wit={N1},alt={°parama°}]{\skm{t-pa}rama}
  \rdg[wit={Y,D}]{°paraṃ}
  \rdg[wit={N2,U1}]{para°}
}\app{\lem[wit={B,E,L}, alt={°śūnyam}]{śūnyam}
  \rdg[wit={P,N1,N2,U1,U2}]{°śūnyaṃ}
  \rdg[wit={D}]{tatparamaśūnyaṃ}}\dd{}
    \end{prose}
  \end{edition}
  \begin{translation}
    \ekddiv{type=trans}
      \centerline{\textrm{\small{[\uproman{30}.\textsuperscript{\coro{\lowroman{1}-\lowroman{11}}}The Order of Cakras]}}}
 \bigskip
    \begin{tlate}
\noindent
Now the sequence of the \textit{cakra}s is taught. At the support\footnote{In the previous section on \textit{cakra}s Rāmacandra situates the first \textit{cakra}, which he there calls \textit{mūlacakra}, at the beginning (\textit{ādau}) [of supposedly the central channel], which should be at the anus if we assume an ascending order, cf. p.\pageref{cakra1}. This assumption is further supported by the additional descriptions of \getsiglum{U2} folio 2v. l.2 in which it is explicitly located at the anus (\textit{gudasthānam}) and called ``\textit{ādhāracakra'' (``\textit{cakra} of the support''}). Thus, the location presented here as \textit{ādhāre} (``at the support'') should be understood as \textit{mūlādhāre} (``at the root-support'').}, there is the Brahmacakra. Above the support at the root of the gender is the Svadiṣṭhānacakra. At the navel, there is the Maṇipūrakacakra. In the heart [there is] the Anāhatacakra.\textsuperscript{\coro{[\lowroman{5}]}} Situated within the throat is the Viśuddhicakra. The sixth is the Tālucakra. In the centre of the eyebrows is the Ājñācakra. At the opening of Brahma\footnote{The \textit{brahmarandhrasthāne} (``at the place of the aperture of Brahmā/Brahman'') is the fontanelle, through which the vital principle of the yogi exists at death, cf. \citeauthor[2017:438]{rootsofyoga}.} is the Kālacakra\footnote{I have not been able to identify the term ``\textit{kālacakra}'' (``\textit{cakra} of time'') for the eighth \textit{cakra} in other texts. All other texts that I am aware of designate the eighth \textit{cakra} in a ninefold \textit{cakra} system with different terms: \citetitle{peterson1888} 4359/256.13 calls it \textit{nirvāṇa°} (``absolute extinction''); \citetitle{shivayogapradipika} 3.15 also designates it with \textit{nirvāṇa°}; the \citetitle{ssplonavla} 2.8 again calls it the \textit{nirvāṇacakraṃ}; \textit{Yogasvarodaya} as quoted in \citetitle{ramatosana} (Ed. p. 833) only calls it \textit{aṣṭamaṃ cakraṃ} (``the eighth \textit{cakra}'') and \textit{siddhapuṃsaḥ sthalaṃ} (``place of the accomplished human''); Rāmacandra himself in the previous chapter on \textit{cakra}s in \uproman{11}.\textsuperscript{\lowroman{1}} picks this up and calls it \textit{aṣṭamacakraṃ} (``eighth \textit{cakra}'') and \textit{siddhapuruṣasya sthānaṃ} (``place of the accomplished person''); \textit{Yogasaṃgraha} (IGNCA 30020 folio 2r. ll. 2-3) calls it \textit{siddhapuruṣasyānacakraṃ} (``the \textit{cakra} of the accomplished person''); \citetitle{saubhagya} again calls it \textit{nirvāṇacakram}, but provides us with yet another unique designation - \textit{parabrahmacakram} (``\textit{cakra} of the supreme Brahmā/Brahman'').} The ninth is the Ākāśacakra\footnote{The term \textit{ākāśacakra} for the ninth \textit{cakra} in the known ninefold \textit{cakra} systems only occurs in \citetitle{shivayogapradipika} 3.16 and \citetitle{ssplonavla} 2.9}.\textsuperscript{\coro{[\lowroman{10}]}} It is supreme emptiness.
    \end{tlate}
  \end{translation}
\end{alignment}
\ekdpb*{}
%%%%%%%%%%%%%%%%%%%%%%%%%%%%%%%%%%%%%%%%%%
%%%%%%%%%%%%%%%%%%%%%%%%%%%%%%%%%%%%%%%%%% 
%%%%%%%% PAGEBREAK%%%%%%%PAGEBREAK%%%%%%%%%
%%%%%%%%%%%%%%%%%%%%%%%%%%%%%%%%%%%%%%%%%% 
%%%%%%%%%%%%%%%% PAGEBREAK%%%%%%%%%%%%%%%%%
%%%%%%%%%%%%%%%%%%%%%%%%%%%%%%%%%%%%%%%%%% 
%%%%%%%% PAGEBREAK%%%%%%%PAGEBREAK%%%%%%%%%
%%%%%%%%%%%%%%%%%%%%%%%%%%%%%%%%%%%%%%%%%% 
%%%%%%%%%%%%%%%%%%%%%%%%%%%%%%%%%%%%%%%%%% 
%%%%%%%%%%%%%%%%%%%%%%%%%%%%%%%%%%%%%%%%%% 
%%%%%%%%%%%%%%%%%%%%%%%%%%%%%%%%%%%%%%%%%% 
%%%%%%%% PAGEBREAK%%%%%%%PAGEBREAK%%%%%%%%%
%%%%%%%%%%%%%%%%%%%%%%%%%%%%%%%%%%%%%%%%%% 
%%%%%%%%%%%%%%%% PAGEBREAK%%%%%%%%%%%%%%%%%
%%%%%%%%%%%%%%%%%%%%%%%%%%%%%%%%%%%%%%%%%% 
%%%%%%%% PAGEBREAK%%%%%%%PAGEBREAK%%%%%%%%%
%%%%%%%%%%%%%%%%%%%%%%%%%%%%%%%%%%%%%%%%%% 
%%%%%%%%%%%%%%%%%%%%%%%%%%%%%%%%%%%%%%%%%% 
%%%%%%%%%%%%%%%%%%%%%%%%%%%%%%%%%%%%%%%%%% 
%%%%%%%%%%%%%%%%%%%%%%%%%%%%%%%%%%%%%%%%%% 
%%%%%%%% PAGEBREAK%%%%%%%PAGEBREAK%%%%%%%%%
%%%%%%%%%%%%%%%%%%%%%%%%%%%%%%%%%%%%%%%%%% 
%%%%%%%%%%%%%%%% PAGEBREAK%%%%%%%%%%%%%%%%%
%%%%%%%%%%%%%%%%%%%%%%%%%%%%%%%%%%%%%%%%%% 
%%%%%%%% PAGEBREAK%%%%%%%PAGEBREAK%%%%%%%%%
%%%%%%%%%%%%%%%%%%%%%%%%%%%%%%%%%%%%%%%%%% 
%%%%%%%%%%%%%%%%%%%%%%%%%%%%%%%%%%%%%%%%%% 
\begin{alignment}[
  texts=edition[class="edition"];
  translation[class="translation"],
  ]
  \begin{edition}
\ekddiv{type=ed}
\centerline{\textrm{\small{[\uproman{31}.\textsuperscript{\coro{\lowroman{1}-\lowroman{50}}}The Sixteen Container]}}}
 \bigskip
 \begin{prose}
   \noindent
%-----------------------------
%idānīm ādhāracakrasya bhedāḥ kathyanta/   \E
%idānīm ādhāracakrasya bhedaḥ kathyate     \P
%idānīm ādhāracakrasya bhedā  kathyaṃte/    \B DSCN7165.jpg Z.3
%idānīm ādhāracakrasya bhedā  kathyaṃte//   \L
%idānīm ādhāracakrasya bhedaḥ kathyate/    \N1
%idānīṃ ādhāracakrasya bhedaḥ kathyate//   \D
%idānī  ādhāracakrasya bhedaḥ kathyaṃte/   \N2
%idānīṃ ādhāracakrasya bhedāḥ kathyaṃte    \U1
%idānīṃ ādhāracakrasya bhedāḥ kathyaṃte // \U2
%-----------------------------
%Now the divisions of the totality of container [for concentration] are taught.
%-----------------------------
\note[type=source, labelb=210, lem={\textbf{Ri}}]{SSP 2.1 (Ed. p. 29): atha ṣoḍaśādhārāḥ kathyante |}
\note[type=source, labelb=211, lem={\textbf{Re}}]{PT\textsuperscript{qcr \cdot YSV} (Ed. p. 832) = YK\textsuperscript{ccn \cdot YSV} 2.15 Ed. p. 24: ṣoḍaśādhārabhedan tu śṛṇu devi viśeṣataḥ |}
\app{\lem[wit={ceteri}, alt={idānīm}]{idānī\skp{m-ā}}
  \rdg[wit={N2}]{idānī}
}\skm{m-ā}dhāracakrasya
\app{\lem[wit={ceteri}]{bhedāḥ}
  \rdg[wit={B,L}]{bhedā}}
\app{\lem[wit={ceteri}]{kathyante}
  \rdg[wit={E}]{kathyanta}
  \rdg[wit={D,N1}]{kathyate}}/ 
%-----------------------------
%pādayor aṃguṣṭhe  tejaso  lakṣyakāraṇāt              dṛṣṭiḥ sthirā bhavati/ \E
%pādayor aṃguṣṭhe  tejaso  lakṣyakaraṇāt              dṛṣṭiḥ sthirā bhavati  \P
%pādayor aṃguṣṭhai tejasaṃ lakṣaṃ kartavyaṃ kāraṇāt// dṛṣṭiḥ sthirā bhavati/ \B
%pādayor aṃguṣṭhe  tejasaṃ lakṣaṃ karttavyaṃ kāraṇāt  dṛṣṭiḥ sthirā bhavatī/ \L
%pādayor aṃguṣṭhe  tejaso  lakṣyakāraṇāt              dṛṣṭisthirā   bhavati/ \N1
%pādayor aṃguṣṭhe  tejaso  lakṣyakāraṇāt              dṛṣṭiḥ sthirā bhavati \D
%pādayor aṃguṣṭhe  tejaso  lakṣakāraṇāt               dṛṣṭisthirā   bhavati/ \N2
%pādayor aṃguṣṭhe  tejaso  lakṣyakāraṇāt              dṛṣṭisthirā   bhavati \U1
%pādayor aṃguṣṭhe  tejaso  lakṣyakāraṇāt              dṛṣṭisthirā   bhavati// \U2 %%%415.jpg
%-----------------------------
%The first is the container of the big toe. From the execution of the fixation onto the light at the big toe of the feet stability of the gaze arises.
%-----------------------------
\note[type=source, labelb=212, lem={\textbf{Ri}}]{SSP 2.10 (Ed. p. 32): tatra prathamaḥ pādāṅguṣṭhādhāraḥ | tatrāgratas tejomayaṃ dhyāyet | dṛṣṭiḥ sthirā bhavati |}
\note[type=source, labelb=213, lem={\textbf{Re}}]{PT\textsuperscript{qcr \cdot YSV} (Ed. p. 839): aṅguṣṭhapādayos tejaḥ salakṣasthiradṛṣṭimān | pādāṅguṣṭhe ya ādhāraḥ prathamo (\textit{prathamaṃ} YK\textsuperscript{ccn \cdot YSV} 2.16 Ed. p. 24) yogatattvataḥ |}
\app{\lem[type=conjecture, resp=egoscrconj]{prathamaḥ pādāṅguṣṭhādhāraḥ}
  \rdg[wit={ceteri}]{\om}}\dd{}
\note[type=philcomm, labelb=214, lem={\uproman{31}.\textsuperscript{\lowroman{2}}}]{Judging by the sources and parallels for this passage as well as the introductory statements Rāmacandra presents for the following \textit{ādhāra}s, it appears likely that originally the first \textit{ādhāra} was introduced, too.}
pādayo\skp{r-aṃ}\app{\lem[wit={ceteri}, alt={aṃguṣṭhe}]{\skm{r-aṃ}guṣṭhe}
  \rdg[wit={B}]{aṃguṣṭhai}}
\app{\lem[wit={ceteri}]{tejaso}
  \rdg[wit={B,L}]{tejasaṃ}}
\app{\lem[wit={ceteri}, alt={lakṣya°}]{lakṣya}
  \rdg[wit={N2}]{lakṣa°}
  \rdg[wit={B,L}]{lakṣaṃ kartavyaṃ}
}\app{\lem[wit={ceteri}, alt={°kāraṇāt}]{kāraṇāt}
  \rdg[wit={P}]{°karaṇāt}} 
 \app{\lem[wit={ceteri}]{dṛṣṭiḥ}
   \rdg[wit={N1,N2,U1,U2}]{dṛṣṭi°}}sthirā
 \app{\lem[wit={ceteri}]{bhavati}
   \rdg[wit={L}]{bhavatī}}/
%-----------------------------
%dvitīyo mūlādhāraḥ/  pādāṃguṣṭhasya mūle parapādasya  pārṣṇiḥ                                         sthāpyate tadāgniḥ prabalo bhavati/ \E
%dvitīyo mūlādhāraḥ   pādāṃguṣṭhasya mūle 'parapādasya dhāraḥ pādāṃduṣṭhasya mūleḥ parapādasya pārṣṇiḥ sthāpyate tadāgniḥ prabalo bhavati \P
%dvitīyo mūlādhāraḥ/  pādāṃguṣṭhasya mūle aparasya pādapārṣṇiḥ                                         syāpyate tadāgniḥ  prabalo bhavatī/ \B
%dvitīyo mūlādhāraḥ   pādāṃguṣṭhasya mūle aparasya pādapārṣṇīḥ                                         syāpyate tadāgniḥ  prabalo bhavatī/ \L
%dvitīyo mūlādhāraḥ/  pādāṃguṣṭhasya mūle aparapādasya pārṣṇiḥ                                         sthāpyate agniḥ    prabalo bhavati/   \N1
%dvitīyo mūlādhāraḥ// pādāṃguṣṭhasya mūle aparapādasya pārṣṇiḥ                                         sthāpyate agni-----prabalo bhavati//   \D  %%%p.12 recto
%dvitīyo mūlādhāraḥ   pādāṃguṣṭhasya mūle aparapādasya pārṣṇiḥ                                         sthāpyate/ \om                     \N2
%dvitīyo mūlādharaḥ   pādāṃguṣṭhasya mūle aparapādasya pārṣṇiḥ                                         sthāpyate agniṃ ---prabalo bhavati    \U1
%dvitīyo mūlādhare    pādāṃguṣṭhasya mūle 'parapādasya pārṣṇiḥ                                         sthāyyaṃte//                       \U2
%-----------------------------
%The second root-container is the second [one]. The heel of the other foot is caused to be placed at the root of the big toe. As a result the fire is strengthened. 
%-----------------------------
\note[type=source, labelb=215, lem={\textbf{Re}}]{SSP 2.11 (Ed. p. 33): dvitīyo mūlādhāras taṃ vāmapādapārṣṇinā niṣpīḍya sthātavyam | tatrāgnidīpanaṃ bhavati |}
 %The second is the Mūlādhara which is to be pressend with the left heel. This enhances the bodily fire. 
 \note[type=source, labelb=216, lem={\textbf{Re}}]{PT\textsuperscript{qcr \cdot YSV} (Ed. p. 839): dvitīyaṃ pādamūlan tu pādamūlaparaṃ (\textit{pādamūlaṃ paraṃ} YK\textsuperscript{ccn \cdot YSV} 2.16 Ed. p. 24) sa vai | pādasya pārṣṇī (\textit{pārṣṇi} YK\textsuperscript{ccn \cdot YSV} 2.17a, Ed. p. 24) saṃsthāpya balavān prabhaven muniḥ | pādamūle 'thavā pādāṅguṣṭhamūlaṃ (\textit{pṛṣṭhe pādāṅguṣṭhe} YK\textsuperscript{ccn \cdot YSV} 2.17 Ed. p. 24) vidhārayet ||}
%The second is the root of the foot. That root of the foot is truly superior. Having placed himself on the heel of the foot the Muni becomes powerful. He shall hold [the gaze?] at the root of the foot or at the big toe.
dvitīyo
\app{\lem[wit={ceteri}]{mūlādhāraḥ}
  \rdg[wit={U1}]{mūlādharaḥ}
  \rdg[wit={U2}]{mūlādhare}}\dd{}
\crazy{pādāṃguṣṭhasya mūle
\app{\lem[wit={ceteri},alt={'para°}]{'para}
  \rdg[wit={D,N1,N2,U1}]{apara°}
  \rdg[wit={B,L}]{aparasya}
}\app{\lem[wit={ceteri}]{pādasya}
  \rdg[wit={B,L}]{pāda°}}
\app{\lem[wit={ceteri}]{pārṣṇiḥ}
  \rdg[wit={L}]{°pārṣṇīḥ}
  \rdg[wit={P}]{dhāraḥ pādāṃduṣṭhasya mūleḥ parapādasya pārṣṇiḥ}}
\app{\lem[wit={ceteri}]{sthāpyate}
  \rdg[wit={B,L}]{syāpyate}
  \rdg[wit={U2}]{sthāyyaṃte}}
\app{\lem[wit={ceteri}]{tadāgniḥ}
  \rdg[wit={D}]{agni°}
  \rdg[wit={N1}]{agniḥ}
  \rdg[wit={U2}]{\om}}
\app{\lem[wit={ceteri}]{prabalo}
  \rdg[wit={N2,U2}]{\om}}
\app{\lem[wit={ceteri}]{bhavati}
  \rdg[wit={B,L}]{bhavatī}
  \rdg[wit={N2,U2}]{\om}}/}\textsuperscript{\begin{otherlanguage}{english}\coro{[\lowroman{5}]}\end{otherlanguage}}
%-----------------------------
%ekaḥ  pārṣṇir ādau  mūlādhāre  sthāpyate/      \E [P.41]
%ekā   pārṣṇir ādau  mūlādhāre  sthāpyate      \P
%ekā   pārṣṇir ādau  mūlādhāra  sthāpyate      \B
%ekā   pārṣṇir ādau  mūlādhārā  sthāpyate      \L
%ekā   pārṣṇiḥ       mūladdhāre sthāpyate/      \N1
%ekā   pārṣṇiḥ       mūlādhārai sthāpyate//     \D
% \om -------------------------------------     \N2
%ekāṃ pārṣṇir        mūlādhāra sthāpyate        \U1
% \om                                          \U2
%-----------------------------
%One heel is caused to be placed at the Root-container. 
%-----------------------------
\app{\lem[wit={ceteri}]{ekā}
  \rdg[wit={E}]{ekaḥ}
  \rdg[wit={U1}]{ekāṃ}}
\app{\lem[wit={U1},alt={pārṣṇiḥ}]{pārṣṇi\skp{r-mū}}
  \rdg[wit={D,N1}]{pārṣṇiḥ}
  \rdg[wit={B,E,L,P}]{pārṣṇir ādau}
}\app{\lem[wit={ceteri},alt={mūlādhāre}]{\skm{r-mū}lādhāre}
  \rdg[wit={B,U1}]{mūlādhāra}
  \rdg[wit={L}]{mūlādhārā}
  \rdg[wit={D}]{mūlādhārai}}
sthāpyate/
\note[type=philcomm, labelb=216a, lem={\uproman{31}.\textsuperscript{\lowroman{6}}}]{Sentence \om in \getsiglum{N2} and \getsiglum{U2}.}
%-----------------------------
% tasya pādasyāṃguṣṭhamūle      parasya  pādasya pārṣṇiḥ sthāpyate// tadagniḥ pradīpyate//  \E [P.41]
% tasya pādasyāṃguṣṭhamūle     'parasya  pādasya pārṣṇiḥ sthāpyate   tadagnīḥ pradipyate    \P
% tasya pādasyāṃguṣṭhamūle     aparasya  pādasya pārṣṇiḥ sthāpyate// tadagnīḥ pradipyate//  \B
% tasya pādasyāṃguṣṭhamūle     aparasya  pādasya pārṣṇiḥ sthāpyate// tadāgnīḥ pradivyate//  \L
% tasya pādasya aṃguṣṭhamūlaṃ/ aparasya  pādasya pārṣṇiḥ sthāpyaṃ       agnir    dāpyate?!/ \N1
% tasya pādasyāṃguṣṭhamūle//   aparasya  pādasya pārṣṇiḥ sthāpyaṃ//     agnir    dīpyate//  \D
% tasya pādasyāṃguṣṭhamūle//   aparasya  pādasya pārṇi---sthāpyaṃ       agni     dīpate//   \N2
% tasya pādasya aṃguṣṭhamūlaṃ  aparasya          pārṣṇo  sthāpyate      agni     dīpyate    \U1
% \om                                                                tadagnīḥ pradipyate//  \U2
%-----------------------------
%The heel of the other foot is caused to be placed at the root of the big toe of this foot. The fire of it is caused to be kindled. 
%-----------------------------
\note[type=philcomm, labelb=216a, lem={\uproman{31}.\textsuperscript{\lowroman{7}}}]{Sentence \om in \getsiglum{U2}.}
tasya
\app{\lem[wit={ceteri}]{pādasyāṅguṣṭhamūle}
   \rdg[wit={N1,U1}]{pādasya aṃguṣṭhamūlaṃ}}
\app{\lem[wit={E,P}]{parasya}
  \rdg[wit={ceteri}]{aparasya}}
\app{\lem[wit={ceteri}]{pādasya}
  \rdg[wit={U1}]{\om}}
\app{\lem[wit={ceteri}]{pārṣṇiḥ}
  \rdg[wit={N2}]{pārṇi}
  \rdg[wit={U1}]{pārṣṇo}}
\app{\lem[wit={B,E,L,P,U1}]{sthāpyate}
  \rdg[wit={D,N1,N2}]{sthāpyaṃ}}/
\app{\lem[wit={E}]{tadagniḥ}
  \rdg[wit={B,L,P,U2}]{tadagnīḥ}
  \rdg[wit={D,N1}]{agnir}
  \rdg[wit={N2,U1}]{agni}}
\app{\lem[wit={E}]{pradīpyate}
  \rdg[wit={B,L,P,U2}]{pradipyate}
  \rdg[wit={D,U1}]{dīpyate}
  \rdg[wit={N1}]{dāpyate}
  \rdg[wit={N2}]{dīpate}}/ 
%-----------------------------
%tṛtīyaṃ gudādhārasthānaṃ   tanmadhye saṃkocavikāsākuṃcana--kāraṇāt pavanaḥ sthiro bhavati// \E
%tṛtīyaṃ gudādhārasthānaṃ   tanmadhye saṃkocavikāśākuṃcana--kāraṇāt pavanaḥ sthiro bhavati   \P
%tṛtīyaṃ gudādhārasthāne    tanmadhye saṃkocavikāśākuṃcana--kāraṇāt pavanaḥ sthiro bhavati// \B
%tṛtīyaṃ gudādhārasthānaṃ   tanmadhye saṃkocavikāśa ākuṃcanakāraṇāt pavanasthiro   bhavatī// \L
%tṛtīyaṃ gudādhārasthānaṃ   tanmadhye saṃkocavikāśākuṃcana--kāraṇāt pavanaḥ sthiro bhavati// \N1
%tṛtīyaṃ gudādhārasthānaṃ   tanmadhye saṃkocavikāśākuṃcanaṃ kāraṇāt pavanasthiro   bhavati// \D
%tṛtīyaṃ gudādhārasthānaṃ   taṃmadhye saṃkocavikāśākuṃcanaṃ kāraṇāt pavanasthiro   bhavati// \N2
%tṛtīyaṃ gudādhārasthānaṃ   taṃmadhye saṃkocavikāśā akuṃcanakāraṇāt pavanasthiro   bhavati \U1
%tṛtīya  gudādhārasthānaṃ// tanmadhye saṃkocavikāśākuṃcana--kāraṇāt pavanasthiro   bhavati// \U2
%-----------------------------
%The third is the place of the anus-container. From the execution of expansion and contraction a stable vitalwind arises.   
%-----------------------------
\note[type=source, labelb=217, lem={\textbf{Re}}]{SSP 2.12 (Ed. p. 33): tṛtīyo gudādhāra taṃ vikāsasaṃkocanena nirākuñcayet | apānavāyuḥ sthiro bhavati}
\note[type=source, labelb=218, lem={\textbf{Re}}]{PT\textsuperscript{qcr \cdot YSV} (Ed. p. 839): tṛtīyan tu gudādhāro (\textit{gudādhāre} YK\textsuperscript{ccn \cdot YSV} 2.18 Ed. p. 24) gudasaṅkocanakriyā | vikāśākuñcanaṃ tasya sthiravāyau ca mṛtyujit |}
\app{\lem[wit={ceteri}]{tṛṭīyaṃ}
  \rdg[wit={U2}]{tṛtīya}}
gudādhāra\app{\lem[wit={ceteri},alt={°sthānaṃ}]{sthānaṃ}
  \rdg[wit={B}]{°sthāne}}\dd{}
tanmadhye
saṃkoca\app{\lem[wit={ceteri},alt={°vikāśā}]{vikāśā}
  \rdg[wit={L}]{°vikāśa°}
}\app{\lem[wit={ceteri},alt={°kuṃcana}]{kuṃcana}
  \rdg[wit={L}]{ākuṃcana}
  \rdg[wit={U1}]{akuṃcana}
  \rdg[wit={D,N2}]{kuṃcanaṃ}
}kāraṇā\skp{t-pa}\app{\lem[wit={ceteri},alt={pavanaḥ}]{\skm{t-pa}pavanaḥ}
  \rdg[wit={D,U1,U2,N2}]{pavana°}}
sthiro
\app{\lem[wit={ceteri}]{bhavati}
  \rdg[wit={B}]{bhavatī}}/\textsuperscript{\begin{otherlanguage}{english}\coro{[\lowroman{10}]}\end{otherlanguage}}
%-----------------------------
%anyac ca/ puruṣasya maraṇaṃ na bhavati/ \E
%anu ca puruṣasya maraṇaṃ bhavati  \P
%anucarapuruṣasya maraṇaṃ bhavatī/ \B
%anucakrapuruṣasya maraṇaṃ bhavatī/ \L
%anū ca puruṣasya maraṇaṃ na bhavati ve?/ \N1
%anu ca puruṣasya maraṇaṃ na bhavati// \D
%anū ca puruṣasya maraṇaṃ na bhavati// \N2
%anu ca puruṣasya maraṇaṃ na bhavati  \U1
%anu ca puruṣasya maraṇaṃ na bhavati//  \U2
%-----------------------------
%And therefore death of the person does not arise.
%-----------------------------
\app{\lem[wit={D,P,U1,U2}]{anu ca}
  \rdg[wit={E}]{anyac ca}
  \rdg[wit={N1,N2}]{anūca}
  \rdg[wit={B}]{anucara°}
  \rdg[wit={L}]{anucakra°}}
puruṣasya maraṇaṃ
\app{\lem[wit={ceteri}]{na}
  \rdg[wit={B,P,L}]{\om}}
\app{\lem[wit={ceteri}]{bhavati}
  \rdg[wit={B,L}]{bhavatī}}/
\end{prose}
  \end{edition}
  \begin{translation}
    \ekddiv{type=trans}
    \centerline{\textrm{\small{[\uproman{31}.\textsuperscript{\coro{\lowroman{1}-\lowroman{50}}}The Sixteen Container]}}}
    \bigskip
    \begin{tlate}
      \noindent
      Now the divisions of the totality\footnote{I understand \textit{cakra} here in the less common meaning of "troup, crowd, totality", cf. \citeauthor[vol. 2, 1958:209]{petersburger}.} of supports [for concentration] are taught.
The first is the support at the big toe. From the execution of the fixation onto the light at the big toe of the feet stability of the gaze arises.
The root support is the second [one]. \crazy{The heel of the other foot is caused to be placed at the root of the big toe. As a result the fire is strengthened.}\textsuperscript{\coro{[\lowroman{5}]}} One heel is caused to be placed at the Root-container. The heel of the other foot is caused to be placed at the root of the big toe of this foot. The fire of it is caused to be kindled. The third is the location of the anus spuport. From the execution of expansion and contraction a stable vitalwind arises.\textsuperscript{\coro{[\lowroman{10}]}} And therefore death of the person does not arise.
    \end{tlate}
  \end{translation}
\end{alignment}
\ekdpb*{}
%%%%%%%%%%%%%%%%%%%%%%%%%%%%%%%%%%%%%%%%%%
%%%%%%%%%%%%%%%%%%%%%%%%%%%%%%%%%%%%%%%%%% 
%%%%%%%% PAGEBREAK%%%%%%%PAGEBREAK%%%%%%%%%
%%%%%%%%%%%%%%%%%%%%%%%%%%%%%%%%%%%%%%%%%% 
%%%%%%%%%%%%%%%% PAGEBREAK%%%%%%%%%%%%%%%%%
%%%%%%%%%%%%%%%%%%%%%%%%%%%%%%%%%%%%%%%%%% 
%%%%%%%% PAGEBREAK%%%%%%%PAGEBREAK%%%%%%%%%
%%%%%%%%%%%%%%%%%%%%%%%%%%%%%%%%%%%%%%%%%% 
%%%%%%%%%%%%%%%%%%%%%%%%%%%%%%%%%%%%%%%%%% 
%%%%%%%%%%%%%%%%%%%%%%%%%%%%%%%%%%%%%%%%%% 
%%%%%%%%%%%%%%%%%%%%%%%%%%%%%%%%%%%%%%%%%% 
%%%%%%%% PAGEBREAK%%%%%%%PAGEBREAK%%%%%%%%%
%%%%%%%%%%%%%%%%%%%%%%%%%%%%%%%%%%%%%%%%%% 
%%%%%%%%%%%%%%%% PAGEBREAK%%%%%%%%%%%%%%%%%
%%%%%%%%%%%%%%%%%%%%%%%%%%%%%%%%%%%%%%%%%% 
%%%%%%%% PAGEBREAK%%%%%%%PAGEBREAK%%%%%%%%%
%%%%%%%%%%%%%%%%%%%%%%%%%%%%%%%%%%%%%%%%%% 
%%%%%%%%%%%%%%%%%%%%%%%%%%%%%%%%%%%%%%%%%% 
%%%%%%%%%%%%%%%%%%%%%%%%%%%%%%%%%%%%%%%%%% 
%%%%%%%%%%%%%%%%%%%%%%%%%%%%%%%%%%%%%%%%%% 
%%%%%%%% PAGEBREAK%%%%%%%PAGEBREAK%%%%%%%%%
%%%%%%%%%%%%%%%%%%%%%%%%%%%%%%%%%%%%%%%%%% 
%%%%%%%%%%%%%%%% PAGEBREAK%%%%%%%%%%%%%%%%%
%%%%%%%%%%%%%%%%%%%%%%%%%%%%%%%%%%%%%%%%%% 
%%%%%%%% PAGEBREAK%%%%%%%PAGEBREAK%%%%%%%%%
%%%%%%%%%%%%%%%%%%%%%%%%%%%%%%%%%%%%%%%%%% 
%%%%%%%%%%%%%%%%%%%%%%%%%%%%%%%%%%%%%%%%%% 
\begin{alignment}[
  texts=edition[class="edition"];
  translation[class="translation"],
  ]
  \begin{edition}
    \ekddiv{type=ed}
    \begin{prose}
      \noindent
%-----------------------------
%caturthaṃ liṃgādhāraṃ   tanmadhye/ liṃgasaṃkocanābhyāsāt  paścimadaṇḍamadhye prajñā nāḍī bhavati/  tanmadhye punar abhyāsakaraṇān manaḥ pavanayoḥ saṃcāro bhavati/ \E
%caturthaṃ liṃgādhāraṃ   tanmadhye  liṃgasaṃkocanābhyāsāt  paścīmadaṇḍamadhye vajñā nāḍī  bhavati   tanmadhye punar abhyāsakaraṇān manaḥ pavanayoḥ saṃcāro bhavati \P
%caturtha--liṃgādhāraṃ   tanmadhye  liṃgasaṃkocanābhyāsāt  paścīmadaṇḍamadhye vajñā nāḍī  bhavatī/  tanmadhye punar abhyāsakaraṇāt punaḥ pavanayo  saṃcāro bhavatī/     \B
%caturtha--liṃgādhāraṃ// tanmadhye  liṃgasaṃkocanābhyāsāt  paścamadaṇḍamadhye vajñā nāḍī  bhavatī// tanmadhye punar abhyāsakaraṇāt punaḥ pavanayo  saṃcāro bhavatī//     \L %%%%%%%%%%%20.jpg
%caturthaṃ liṃgādhāraṃ   tanmadhye/ liṃgasaṃkocanābhyāsāt/ paścimadaṇḍamadhye vajranāḍī   bhavati/  tanmadhye punaḥ abhyāsakaraṇāt manaḥpavanayoḥ saṃcāro bhavati/ \N1
%caturtha--liṃgādhāraṃ// tanmadhye/ liṃgasaṃkocanābhyāsāt//paścimadaṇḍamadhye vajrānāḍī   bhavati// tanmadhye punaḥ abhyāsakaraṇāt manaḥpavanayoḥ saṃcoro bhavati// \D
%caturthaṃ liṃgādhāraṃ   tanmadhye  liṃgasakoṇābhyāsāt//   paścimadaṇḍamadhye vajranāḍī   bhavati/  tanmadhye punar ābhyāsakaraṇāt manaḥpavanayoḥ saṃcāro bhavati// \N2
%caturthaṃ liṃgādhāraṃ   tanmadhye  liṃgasaṃkocanābhyāsāt  paścimadaṇḍamadhye vajranāḍī   bhavati   tanmadhye punar ābhyāsakaraṇāt manaḥpavanayoḥ saṃcāro bhavati    \U1    %%%283.jpg
%caturthaṃ liṃgādhāraṃ   tanmadhye  liṃgasaṃkocanābhyāsāt  paścimadaṇḍamadhye vajranāḍī   bhavati   tanmadhye punar ābhyāsakaraṇān manaḥpavanayoḥ saṃcāro bhavati//   \U2
%-----------------------------
%The fourth is the penis support. Due to the execution of repeated practice of contracting the penis in the midst of therof, the adamantine channel appears in the middle of the staff of the back. From the repeated practice again [and again] the transition of both breath and mind into its center arises.  
%-----------------------------
\note[type=source, labelb=219, lem={\textbf{Re}}]{SSP 2.13 (Ed. pp. 33-34): caturtho meḍhrādhāraḥ | liṅgasaṃkocanena brahmagranthitrayaṃ bhitvā bhramaraguhāyāṃ viśramya tata ūrdhvamukhe bindustambhanaṃ bhavati| eṣā vajrolī prasiddhā}
\note[type=source, labelb=220, lem={\textbf{Re}}]{PT\textsuperscript{qcr \cdot YSV} (Ed. pp. 839-840): liṅgādhāraṃ caturthan tu liṅgasaṅkocanan tu ca | liṅgasaṅkocanābhyāsāt paścimādaṇḍamadhyagaḥ | vajranāḍīti (\textit{vajrānāḍī tu} YK\textsuperscript{ccn \cdot YSV} 2.20 Ed. p. 24) tanmadhye punar abhyasayaṃs (\textit{abhyasanan} YK\textsuperscript{ccn \cdot YSV} 2.20 Ed. p. 24) tathā | sañcāro vāyumanasor atisañcāra iti (\textit{ratiṃ sañcarati} YK\textsuperscript{ccn \cdot YSV} 2.20 Ed. p. 24) tridhā | granthitrayavibhedas (\textit{°bhedan} YK\textsuperscript{ccn \cdot YSV} 2.21 Ed. p. 24) tu tadbhedo brahmamārgataḥ | brahmapadmo (\textit{°padme} YK\textsuperscript{ccn \cdot YSV} 2.21 Ed. p. 24) vāyupūrṇo (\textit{°pūrṇe} YK\textsuperscript{ccn \cdot YSV} 2.21, Ed. p. 24) bhūtvā tiṣṭhati yogirāṭ | vīryastambho bhavet tena sādhayet tu sadā yuvā | mūlādhāre brahmapadme ṣaṭpadme ca tathā tathā |}
\app{\lem[wit={ceteri}]{caturthaṃ}
  \rdg[wit={B,D,L}]{caturtha°}}
liṅgādhāraṃ \dd{}
tanmadhye
liṅga\app{\lem[wit={ceteri},alt={saṃkocanā°}]{saṃkocanā}
  \rdg[wit={N2}]{sakoṇā°}
}bhyāsāt
\app{\lem[wit={ceteri}, alt={paścima°}]{paścima}
  \rdg[wit={B,P}]{paścīma°}
  \rdg[wit={L}]{paścama°}
}daṇḍamadhye
\app{\lem[wit={ceteri}, alt={vajra°}]{vajra}
  \rdg[wit={B,P,L}]{vajñā°}
  \rdg[wit={E}]{prajñā°}
}nāḍī
\app{\lem[wit={ceteri}]{bhavati}
  \rdg[wit={B,L}]{bhavatī}}/\textsuperscript{\begin{otherlanguage}{english}\coro{[\lowroman{14}]}\end{otherlanguage}}
tanmadhye punar-ābhyāsa\app{\lem[wit={E,P,U2}, alt={°karaṇān}]{karaṇā\skp{n-ma}}
  \rdg[wit={ceteri}]{karaṇāt}
}\app{\lem[wit={ceteri}, alt={manaḥ}]{\skm{n-ma}naḥ}
  \rdg[wit={B,L}]{punaḥ}}
\app{\lem[wit={ceteri}]{pavanayoḥ}
  \rdg[wit={B,L}]{pavanayo}}
\app{\lem[wit={ceteri}]{saṃcāro}
  \rdg[wit={D}]{saṃcoro}}
\app{\lem[wit={ceteri}]{bhavati}
  \rdg[wit={B,L}]{bhavatī}}/
%-----------------------------
%tayoḥ saṃcārān  madhye granthitrayaṃ truṭyati/   \E
%tayoḥ saṃcārān  madhye graṃthitrayaṃ truṭyati    \P
%tayo  saṃcārān  madhye granthitrayaṃ truṭyatī/   \B
%tayoḥ saṃcārān  madhye graṃthitrayaṃ truṭayatī   \L
%tayoḥ saṃcārān  madhye granthitrayaṃ truṭyati/   \N1 %truṭyati="zerbrechen"
%tayoḥ saṃcārāt  madhye graṃthitrayaṃ truṭyati//  \D 
%tayoḥ saṃcārān  madhye granthitrayaṃ ... ..ti/   \N2
%tayoḥ saṃccārāt madhye graṃthitrayaṃ trudyati    \U1
%tayoḥ saṃccārān madhye graṃthitrayaṃ truṭyati//  \U2
%-----------------------------
%Caused by the transition of them both into the center the trinity of knots breaks.
%-----------------------------
\app{\lem[wit={ceteri}]{tayoḥ}
  \rdg[wit={B}]{tayo}}
\app{\lem[wit={ceteri},alt={saṃcārān}]{saṃcārā\skp{n-ma}}
  \rdg[wit={D,U1}]{saṃcārāt}
}\skm{n-ma}dhye
granthitrayaṃ
\app{\lem[wit={ceteri}]{truṭyati}
  \rdg[wit={B}]{truṭyatī}
  \rdg[wit={L}]{truṭayatī}
  \rdg[wit={U1}]{trudyati}
  \rdg[wit={N2}]{ti}}/\textsuperscript{\begin{otherlanguage}{english}\coro{[\lowroman{15}]}\end{otherlanguage}}
%-----------------------------
% tatroṭanāt        pavano  brahmakamalamadhye pūrṇo bhūtvā tiṣṭhati/  \E
%                                                                      \P
% tatroṭanāt        pavano  brahmakamadhye     pūrṇā bhūtvā tiṣṭhati// \B
% tatroṭanāt        pavano  brahmakamadhye     pūrṇā bhūtvā tiṣṭhati// \L
% tattroṭanāt       pavanaḥ brahmakamalamadhye pūrṇo bhūtvā tiṣṭhati/  \N1 
% tata troṭanāt     pavanaḥ brahmakamalamadhye pūrṇo bhūtvā tiṣṭhati// \D 
% tata troṭanāt     pavanaḥ brahmakamalamadhye pūrṇo bhūtvā tiṣṭhati/  \N2
% tatroṭaṇāt        pavanaḥ brahmakamalamadhye pūrṇo bhūtvā tiṣṭhati   \U1
% tattroṭaṇāt       pavanaḥ brahmakamalamadhye pūrṇo bhūtvā tiṣṭhati// \U2
%-----------------------------
% There, from the breaking of that, the vitalwind, after having filled up (the central channel?) resides within the Brahma-lotus. 
%-----------------------------
\app{\lem[wit={N1,U2},alt={°tattroṭanāt}]{tattroṭanā\skp{t-pa}}
  \rdg[wit={B,E,L,U1}]{tatroṭanāt}
  \rdg[wit={D,N2}]{tata troṭanāt}}
\app{\lem[wit={B,E,L},alt={pavano}]{\skm{t-pa}vano}
  \rdg[wit={ceteri}]{pavanaḥ}}
brahma\app{\lem[wit={ceteri}, alt={°kamala°}]{kamala}
  \rdg[wit={B,L}]{°ka°}
}madhye
\app{\lem[wit={ceteri}]{pūrṇo}
  \rdg[wit={B,L}]{pūrṇā}}
bhūtvā tiṣṭhati/
%-----------------------------
% tato vīryastambho bhavati/  puruṣaḥ sadaiva   yuvā      bhavati/ \E
% tato vīryastaṃbho bhavati   puruṣaḥ saṃdaivaṃ yuve   prabhavati  \P
% tato vīryastambho bhavatī// puruṣaḥ sadaiva   yuvai     bhavatī/ \B
% tato vīryastaṃbho bhavati   puruṣaḥ sadaiva   yuvaiva   bhavati// \L
% tato vīryastambho bhavati/  puruṣaḥ sadaiva   yuvā/e va bhavati// \N1 %truṭyati="zerbrechen"
% tato vīryastambho bhavati// puruṣaḥ sadaiva   yuvaiva   bhavati// \D 
% tato vīryastambho bhavati/  puruṣa  sadaiva   yurvaiva  bhavati// \N2
% tato vīryastaṃbho bhavati/  puruṣaḥ sadaiva   yuvaivaṃ  bhavati \U1
% tato vīryastaṃbho bhavati   puruṣaḥ sadaiva   vaibhavo  bhavati// \U2
%-----------------------------
% From that virility and strength arise. The person becomes youthful forever.
%-----------------------------
tato vīryastambho bhavati/
\app{\lem[wit={ceteri}]{puruṣaḥ}
  \rdg[wit={N2}]{puruṣa}}
\app{\lem[wit={ceteri}]{sadaiva}
  \rdg[wit={P}]{saṃdaivaṃ}}
\app{\lem[wit={D,L}]{yuvaiva}
  \rdg[wit={E}]{yuvā}
  \rdg[wit={P}]{yuve}
  \rdg[wit={B}]{yuvai}
  \rdg[wit={N1}]{yuve va}
  \rdg[wit={N2}]{yurvaiva}
  \rdg[wit={U1}]{yuvaivaṃ}
  \rdg[wit={U2}]{yuvaivaṃ}}
\app{\lem[wit={ceteri}]{bhavati}
  \rdg[wit={B}]{bhavatī}
  \rdg[wit={P}]{prabhavati}}/
%-----------------------------
%paṃcama  udgīryāṇāṃ svādhiṣṭhānaṃ tatra bandhanān      malamūtrayor nāśo   bhavati/  \E
%paṃcamaṃ uḍḍīyāṇāṃ  svādhiṣṭhānaṃ tatra baṃdhadānān    malamūtrayor nāśo   bhavati   \P
%paṃcama  uḍḍiyānāṃ  svādhiṣṭhānaṃ tatra baṃdha dīyate/ malamūtrayor nāśo   bhavatī// \B
%paṃcamaṃ uḍḍiyānāṃ  svādhiṣṭhānaṃ tatra baṃdha dīyate/ mūlamūcayor  nāśo   bhavati// \L 
%paṃcamaṃ udyānaṃ                  tatra baṃdhanāt      malamūtrayor nāśe/o bhavati// \N1 [s.10, verso, z4]
%paṃcamaṃ udyāṇāṃ                  tatra vaṃdhanāt      malamūtrayor nāśo   bhavati// \D
%paṃcam   odyānaṃ                  tatra baṃdhanāt      malamūtrayor nāśo   bhavati/  \N2
%paṃcamaṃ uddyānaṃ                 tatra baṃdhadānāt    malamūtrayor nāśo   bhavati   \U1
%paṃcamaṃ uḍḍīyāṇaṃ  svādhiṣṭhānaṃ tatra badhadānān     malamūtrayor nāśo   bhavati// \U2
%-----------------------------
%The fifth is Udyāna. From performing \textit{bandha} there, urine and faeces disappear.  
%-----------------------------
\note[type=source, labelb=221, lem={\textbf{Re}}]{SSP 2.14 (Ed. p. 34): pañcame oḍīyāṇādhārayor bandhanān malamūtrasaṃkocanaṃ bhavati | *uḍyānā° etc. in various mss.}
\note[type=source, labelb=222, lem={\textbf{Re}}]{PT\textsuperscript{qcr \cdot YSV} (Ed. p. 840): pañcamaṃ jaṭharādhāraṃ tadā bandhayati kramāt | mṛtyunā bhaṅgasiddho 'yaṃ (\textit{mṛtyunā māṅga°} YK\textsuperscript{ccn \cdot YSV} 2.23 Ed. p. 25) mṛtyor (\textit{mṛtyur} YK\textsuperscript{ccn \cdot YSV} 2.23 Ed. p. 25) eva kṣayaṅkaraḥ | anena paścimād ūrddhaṃ (\textit{ūrdhvaṃ} YK\textsuperscript{ccn \cdot YSV} 2.24 Ed. p. 25) vāyuḥ kuryād viśāladhīḥ | bandho 'yaṃ buddhimanasoḥ pañcamādhārakālajit |}
\app{\lem[wit={ceteri}]{paṃcamaṃ}
  \rdg[wit={B}]{paṃcama}
  \rdg[wit={N2}]{paṃcam}}
\app{\lem[wit={D,N1}]{udyānaṃ}
  \rdg[wit={N2}]{odyānaṃ}
  \rdg[wit={U1}]{uddyānaṃ}
  \rdg[wit={P,U2}]{uḍḍīyāṇāṃ svādhiṣṭhānaṃ}
  \rdg[wit={B,L}]{uḍḍiyānāṃ svādhiṣṭhānaṃ}
  \rdg[wit={E}]{udgīryāṇāṃ svādhiṣṭhānaṃ}} \dd{}
\note[type=philcomm, labelb=223, lem={\uproman{31}\textsuperscript{\lowroman{19}}}]{Spellings for this component of the yogic body vary dramatically across yogic literature. Since this sentence very likely based on the SSP and the prevelant variant of the component is *\textit{uḍyānā}° etc., the reading of \getsiglum{N1} and \getsiglum{D} is most convincing. B,E,L,P,U\textsubscript{2} add the expression \textit{svādhiṣṭhānaṃ}. Since this reading is absent in the source and parallels it seems to be a later addition.}
tatra
\app{\lem[wit={E}]{bandhanā\skp{n-ma}}
  \rdg[wit={U2}]{badhadānān}
  \rdg[wit={N1,N2}]{baṃdhanāt}
  \rdg[wit={D}]{vaṃdhanāt}
  \rdg[wit={U1}]{baṃdhadānāt}
  \rdg[wit={P}]{baṃdhadānān}
  \rdg[wit={B,L}]{baṃdha dīyate}
}\app{\lem[wit={ceteri},alt={malamūtrayor}]{\skm{n-ma}lamūtrayo\skp{r-nā}}
  \rdg[wit={L}]{mūlamūcayor}}
\skm{r-nā}śo
\app{\lem[wit={ceteri}]{bhavati}
  \rdg[wit={B}]{bhavatī}}/
%-----------------------------
%ṣaṣṭho nābhyādhāraḥ/    \E
%ṣaṣṭho nābhyādhāraḥ   tatra         prāṇavābhyāsād  nāhato   nāraḥ   svayam utpadyate / \P
%ṣaṣṭho nābhyādhāraḥ   tatra         prāṇavābhyāsād  anāhato  nādaḥ// svayam utpadyate// \B
%ṣaṣṭho nābhyādhāraḥ   tatra         prāṇavābhyāsād  anāhato  nādaḥ// svayam utpadyate... \L 
%ṣaṣṭho nābhyādhāraḥ/  tatra         praṇavābhyāsāt  anāhato  nādaḥ   svayam ūtpadyate/  \N1
%ṣaṣṭho nābhyādhāraḥ// tatra         prāṇavābhyāsāt  anāhato  nādaḥ// svayam utpadyate// \D
%ṣaṣṭho nābhyādhāraḥ   tatra         praṇavābhyāsāt  anāhato  tādaḥ   svayaṃ utpadyate/ \N2
%ṣaṣṭho nābhyādhāras   tatra         praṇavābhyāṃsad ānāhato  nadaḥ   svayam utpadyate   \U1
%ṣaṣṭho nābhyādhāre//  tatra         prāṇavābhyāsād  anohato  nādaḥ   svayam utpadyate// \U2
%-----------------------------
%The sixth is the support of the navel. From repeated practice of \textit{praṇava}, the unstruck sound arises by itself. 
%-----------------------------
\note[type=source, labelb=224, lem={\textbf{Re}}]{SSP 2.15 (Ed. p. 34): ṣaṣṭhe nābhyādhāra oṃkāram ekacittenoccārayet | nādalayo bhavati |}
\note[type=source, labelb=225, lem={\textbf{Re}}]{PT\textsuperscript{qcr \cdot YSV} (Ed. p. 840): nābhyādhāro bhavet ṣaṣṭhas (\textit{ṣaṣṭhaṃ} YK\textsuperscript{ccn \cdot YSV} 2.25 Ed. p. 25) tatra prāṇaṃ samabhyaset | svayam utpadyate nādo nādato muktidantataḥ (\textit{muktidaṇḍataḥ} YK\textsuperscript{ccn \cdot YSV} 1.25 Ed. p. 25) |}
ṣaṣṭho
\app{\lem[wit={ceteri}]{nābhyādhāraḥ}
  \rdg[wit={U1}]{nābhyādhāras}
  \rdg[wit={U2}]{nābhyādhāre}}\dd{}\textsuperscript{\begin{otherlanguage}{english}\coro{[\lowroman{20}]}\end{otherlanguage}}
\app{\lem[wit={ceteri}]{tatra}
  \rdg[wit={E}]{\om}}
\app{\lem[wit={B,L,P,U2}]{prāṇavābhyāsā\skp{d-a}}
  \rdg[wit={D,N1,N2}]{prāṇavābhyāsā1}
  \rdg[wit={U1}]{prāṇavābhyāṃsad}
}\app{\lem[wit={ceteri},alt={°anāhato}]{\skm{d-a}nāhato}
    \rdg[wit={P}]{nāhato}
    \rdg[wit={U1}]{ānāhato}
    \rdg[wit={U2}]{anohato}}
  \app{\lem[wit={ceteri}]{nādaḥ}
    \rdg[wit={P}]{nāraḥ}
    \rdg[wit={N2}]{tādaḥ}}
  \app{\lem[wit={ceteri}]{svaya\skp{m-u}}
    \rdg[wit={N2}]{svayaṃ}
}\app{\lem[wit={ceteri},alt={utpadyate}]{\skm{m-u}tpadyate}
  \rdg[wit={N1}]{ūtpadyate}}/
\note[type=philcomm, labelb=225a, lem={\uproman{31}.\textsuperscript{\lowroman{21}}}]{Sentence \om in \getsiglum{E}.}
    \end{prose}
  \end{edition}
  \begin{translation}
    \ekddiv{type=trans}
    \begin{tlate}
      \noindent
The fourth is the penis support. Due to the execution of repeated practice of contracting the penis in the midst of therof, the adamantine channel appears in the middle of the staff of the back\footnote{The staff of the back (\textit{paścimadaṇḍa}) is another name for the central channel.}. From the repeated practice again [and again] the transition of both breath and mind into its center arises. Caused by the transition of both [breath and mind] into the center the trinity of knots\footnote{Explain the three knots.} breaks. There, from the breaking of that, the vitalwind, after having filled up (the central channel?) resides within the Brahma-lotus\footnote{Explain the Brahma-lotus. Is it the heart? or is it the final cakra?}. From that virility and strength arise. The person becomes youthful forever. The fifth is Udyāna. From performing \textit{bandha} there, urine and faeces disappear. The sixth is the support of the navel. From repeated practice of \textit{praṇava}, the unstruck sound arises by itself. 
    \end{tlate}
  \end{translation}
\end{alignment}
%%%%%%%%%%%%%%%%%%%%%%%%%%%%%%%%%%%%%%%%%%
%%%%%%%%%%%%%%%%%%%%%%%%%%%%%%%%%%%%%%%%%% 
%%%%%%%% PAGEBREAK%%%%%%%PAGEBREAK%%%%%%%%%
%%%%%%%%%%%%%%%%%%%%%%%%%%%%%%%%%%%%%%%%%% 
%%%%%%%%%%%%%%%% PAGEBREAK%%%%%%%%%%%%%%%%%
%%%%%%%%%%%%%%%%%%%%%%%%%%%%%%%%%%%%%%%%%% 
%%%%%%%% PAGEBREAK%%%%%%%PAGEBREAK%%%%%%%%%
%%%%%%%%%%%%%%%%%%%%%%%%%%%%%%%%%%%%%%%%%% 
%%%%%%%%%%%%%%%%%%%%%%%%%%%%%%%%%%%%%%%%%% 
%%%%%%%%%%%%%%%%%%%%%%%%%%%%%%%%%%%%%%%%%% 
%%%%%%%%%%%%%%%%%%%%%%%%%%%%%%%%%%%%%%%%%% 
%%%%%%%% PAGEBREAK%%%%%%%PAGEBREAK%%%%%%%%%
%%%%%%%%%%%%%%%%%%%%%%%%%%%%%%%%%%%%%%%%%% 
%%%%%%%%%%%%%%%% PAGEBREAK%%%%%%%%%%%%%%%%%
%%%%%%%%%%%%%%%%%%%%%%%%%%%%%%%%%%%%%%%%%% 
%%%%%%%% PAGEBREAK%%%%%%%PAGEBREAK%%%%%%%%%
%%%%%%%%%%%%%%%%%%%%%%%%%%%%%%%%%%%%%%%%%% 
%%%%%%%%%%%%%%%%%%%%%%%%%%%%%%%%%%%%%%%%%% 
%%%%%%%%%%%%%%%%%%%%%%%%%%%%%%%%%%%%%%%%%% 
%%%%%%%%%%%%%%%%%%%%%%%%%%%%%%%%%%%%%%%%%% 
%%%%%%%% PAGEBREAK%%%%%%%PAGEBREAK%%%%%%%%%
%%%%%%%%%%%%%%%%%%%%%%%%%%%%%%%%%%%%%%%%%% 
%%%%%%%%%%%%%%%% PAGEBREAK%%%%%%%%%%%%%%%%%
%%%%%%%%%%%%%%%%%%%%%%%%%%%%%%%%%%%%%%%%%% 
%%%%%%%% PAGEBREAK%%%%%%%PAGEBREAK%%%%%%%%%
%%%%%%%%%%%%%%%%%%%%%%%%%%%%%%%%%%%%%%%%%% 
%%%%%%%%%%%%%%%%%%%%%%%%%%%%%%%%%%%%%%%%%% 
\begin{alignment}[
  texts=edition[class="edition"];
  translation[class="translation"],
  ]
  \begin{edition}
    \ekddiv{type=ed}
    \begin{prose}
      \noindent
%-----------------------------
%                             tasmin sthāne prāṇavāyor  nirodhāt            ṣaḍapi kamalāny ūrdhvamukhāni             vikasaṃti// \E                                                     
%saptamo hṛdayarūpadhāraḥ     tasmin sthāne prāṇavāyor  nirodhāt            ṣadapi kamalāny ūrdhvamukhāni             vikasaṃti  \P  %%%7653.jpg 
%                             tasmin sthāne prāṇavāyo   nirodhāt/           ṣaḍapi kamalāny ūrdhvamukhāni             vikasaṃti// \B
%saptamo hṛdayarūpadhāraḥ//   tasmin sthāne prāṇavāyor  nirodhāt            ṣadapi kamalāny ūrdhvamukhāni             vikasaṃti// \L
%saptamo hṛdayarūpa ādhāraḥ   tasmin sthāne prāṇavāyor  nirūṃdhanāt/        ṣadapi kamalāny ūrdhvamukhaṃ              vikasaṃti// \N1
%saptamo hṛdayarūpa ādhāraḥ// tasmin sthāne prāṇavāyor  nir???ūṃ???dhanāt// ṣadapi kamalāny ūrdhvamukhaṃ              vikasaṃti// \D
%saptamo hṛdayarūpādhāraḥ     tasmin sthāne prāṇavāyor  nirūṃdhanāt/        ṣadapi kamalāny ūrdhvemukhaṃ              vikasaṃti// \N2 %%%%%%%%%[S.9, recto, z.4]
%saptamo hṛdayarūpādhāraḥ     tasmin sthāne prāṇavāyor  nirūṃdhanāt         ṣadapi kamalāny ūrusyordha mukhaṃ bhavati vikasaṃti  \U1
%saptamo hṛdayādhāraḥ         tasmin sthāne prāṇavāyor  nirodhāt//          ṣadapi kamalāny ūrddhvamukhāni            vikasaṃti//  \U2
%-----------------------------
%The seventh is the support of the heart-form. From the restraint of the Prāṇa vitalwind in this location also the six upward-facing lotusses open up.   
%-----------------------------
\note[type=source, labelb=226, lem={\textbf{Re}}]{SSP 2.16 (Ed. p. 34): saptame hṛdayādhāre prāṇaṃ nirodhayet | kamalavikāso bhavati |}
\note[type=source, labelb=227, lem={\textbf{Re}}]{PT\textsuperscript{qcr \cdot YSV} (Ed. p. 840): saptamo hṛdayādhāras tasmin vāyunibandhanāt | ūrddhaktrāṇi (\textit{ūrdhvavaktrāṇi} YK\textsuperscript{ccn \cdot YSV} 2.26 Ed. p. 25) padmāni vikasanti mahān bhavet |}
\app{\lem[wit={ceteri}]{saptamo}
  \rdg[wit={E,B}]{\om}}
\app{\lem[wit={ceteri}]{hṛdaya}
  \rdg[wit={U2}]{hṛdayā°}
  \rdg[wit={E,B}]{\om}
}\app{\lem[wit={N2,U1},alt={°rūpādhāraḥ}]{rūpādhāraḥ}
  \rdg[wit={L}]{°rūpadhāraḥ}
  \rdg[wit={D,N1}]{rūpa ādhāraḥ}
  \rdg[wit={U2}]{°dhāraḥ}
  \rdg[wit={E,B}]{\om}}\dd{}
tasmin-sthāne
\app{\lem[wit={ceteri}]{prāṇavāyo\skp{r-ni}}
  \rdg[wit={B}]{prāṇavāyo}
}\app{\lem[wit={ceteri},alt={nirodhāt}]{skm{r-ni}rodhā\skp{t-ṣa}}
  \rdg[wit={D,N1,N2,U1}]{nirūṃdhanāt}
}\app{\lem[wit={B},alt={ṣaḍ api}]{\skm{t-ṣa}ḍapi}
  \rdg[wit={ceteri}]{ṣadapi}}   
kamalā\skp{ny-ū}\app{\lem[wit={ceteri},alt={ūrdhvamukhāni}]{\skm{ny-ū}rdhvamukhāni}
  \rdg[wit={D,N1,N2}]{ūrdhvamukhaṃ}
  \rdg[wit={U1}]{ūrusyordha mukhaṃ bhavati}}
vikasaṃti/
%-----------------------------
%aṣṭamaṃ kaṇṭhādhāraḥ/  tatra  jālaṃdharo bandho dīyate/  tasmin satīḍāyāṃ   piṃgalāyāṃ pavanaḥ sthiro bhavati/  \E %%[p.43]
%aṣṭamaḥ kaṃṭhādhāraḥ   tatra  jālaṃdharo baṃdho dīyate   tasmin satīḍāyāṃ   piṃgalāyāṃ pavanaḥ sthiro bhavataḥ  \P
%aṣṭame  kaṇṭhādhāraḥ/  tatra  jalaṃ baṃdho      dīyate   tasmin satīyāṃ     piṃgalāyāṃ pavanaḥ sthiro bhavatī/ \B  %%%%DSCN7166.jpg Z.3
%aṣṭame  kaṇṭhādhāraḥ/  tatra  jalaṃ baṃdho      dīyate   tasmin satīyāṃ     piṃgalāyāṃ pavanaḥ sthiro bhavatī// \L
%aṣṭamaḥ kaṇṭhādhāraḥ/  tatra  jālaṃdharo baṃdho dīyate/  tasmin sati iḍāyāṃ piṃgalāyāṃ pavanaḥ sthiro bhavati/ \N1
%aṣṭamaḥ kaṃṭhādhāraḥ// tatraḥ jālaṃdharo baṃdho dīyate// tasmin sati iḍāyāṃ piṃgalāyāṃ pavanasthiro bhavati// \D  %%%p.12 verso
%aṣṭama--kaṇṭhādhāraḥ/  tatra  jālaṃdharabandho  dīyate// tasmin satiśadāyāṃ piṃgalāyāṃ pavanaḥ sthiro bhavati/ \N2
%aṣṭamaḥ kaṇṭhādhāraḥ   tatra  jālaṃdharo bandho dīpyate  tasmin sati iḍāyāṃ piṃgalāyāṃ pavanaḥ sthiro bhavati \U1
%aṣṭamaḥ kaṇṭhādhāraḥ   tatra  jālaṃdharo bandho dīyate   tasmin sati piḍāyā piṃgalāyāṃ pavanaḥ sthiro bhavati// \U2
%-----------------------------
%The support of the throat is the eighth. There the binding of Jālaṃdhara is produced. While abiding therein the vitalwind in the Iḍā and Piṅgalā channel becomes stable.   
%-----------------------------
\note[type=source, labelb=228, lem={\textbf{Re}}]{SSP 2.17 (Ed. p. 34): aṣṭame kaṇṭhādhāre kaṇṭhamūlaṃ cibukena nirodhayet | iḍāpiṅgalayor vāyuḥ sthiro bhavati |}
\note[type=source, labelb=229, lem={\textbf{Re}}]{PT\textsuperscript{qcr \cdot YSV} (Ed. p. 840) =  YK\textsuperscript{ccn \cdot YSV} 2.27 Ed. p. 25: kaṇṭhādhāro 'ṣṭamas tatra kaṇṭhasaṅkocalakṣaṇaḥ | jālandharākhyo bandhaḥ syāt tasmin sati marud dṛḍhaḥ |}
\app{\lem[wit={D,P,N1,U1,U2}]{aṣṭamaḥ}
  \rdg[wit={E}]{aṣṭamaṃ}
  \rdg[wit={B,L}]{aṣṭame}
  \rdg[wit={N2}]{aṣṭama°}}
kaṇṭhādhāraḥ/
\app{\lem[wit={ceteri}]{tatra}
  \rdg[wit={D}]{tatraḥ}}
\app{\lem[wit={ceteri}]{jālaṃdharo}
  \rdg[wit={N2}]{jālaṃdhara°}
  \rdg[wit={B,L}]{jalaṃ}}
bandho
\app{\lem[wit={ceteri}]{dīyate}
  \rdg[wit={U1}]{dīpyate}}/\textsuperscript{\begin{otherlanguage}{english}\coro{[\lowroman{25}]}\end{otherlanguage}}
tasmin \app{\lem[wit={E,P}]{satīḍāyāṃ}
  \rdg[wit={B,L}]{satīyāṃ}
  \rdg[wit={D,N1,U1,U2}]{sati iḍāyāṃ}
  \rdg[wit={N2}]{satiśadāyāṃ}}
piṅgalāyāṃ
\app{\lem[wit={ceteri}]{pavanaḥ}
  \rdg[wit={D}]{pavana°}}
sthiro
\app{\lem[wit={ceteri}]{bhavati}
  \rdg[wit={B,L}]{bhavatī}}/
%-----------------------------
%navamo ghaṃṭikādhāraḥ/   tatra jihvāgraṃ   lagnaṃ bhavati/    tato mṛtakalāyā     amṛtaṃ sravati/  tadamṛtapānāt             śarīramadhye rogasaṃcāro na bhavati/ \E
%navamo ghaṭikādhāraḥ     tatra jihvāgraṃ   lagnaṃ bhavati     tato mṛtakakalāyā   amṛta  sravati   tadamṛtapānāc            charīramadhye rogasaṃcāro na bhavati  \P
%navo   ghaṃṭikādhāraḥ//  tatra jihvāgraṃ   lagnaṃ bhavatī/    tato mṛtakalāyā     amṛtaṃ sravati/  tadamṛtakalāyāṃ amṛtapānīcharīramadhye rogasaṃcāro bhavatī/ \B
%navamo ghaṃṭādhāraḥ//    tatra jihvāgraṃ   lagnaṃ bhavati//   tato mṛtakalāyāṃ                        amṛtapānā-------------charīramadhye rogasaṃcāro bhavati// \L %eyeskip in line.. :(
%navamo ghaṃṭikādhāraḥ/   tatra jihvāgraṃ   lagnaṃ bhavati/    tato mṛtakalāyā     amṛtaṃ sravati/  tadamṛtapānāt             śarīramadhye rogasaṃcāro na bhavati/ \N1
%navamo ghaṃṭikādhāraḥ//  tatra jihvāyāgraṃ lagnaṃ bhavati//   tataḥ amṛtakalāyāḥ  amṛtaṃ sravati// tadamṛtapānāc           charīramadhye  rogasaṃcāro na bhavati// \D
%navamo ghaṃṭikādhāraḥ/   tatra jihvāgraṃ   lagnaṃ bhavati/    tato mṛtakalāyā     amṛtaṃ sravati/  tadamṛtapānāt             śarīramadhye rogasaṃcāro na bhavati/ \N2
%navamo ghaṃṭikādhāras    tatra juhvāyāṃ    lagnaṃ bhavati vā  tataḥ amṛtakalāyāḥ  amṛtaṃ sravati   tadamṛtapānāt            charīramadhye rogasaṃcāro na bhavati \U1
%navamo ghaṃṭikādhāraḥ    tatra jihvāgraṃ   lagnaṃ bhavati//   tato mṛtakalāyāḥ    amṛtaṃ sravati// tadamṛtapānā             charīramadhye rogasaṃcāro na bhavati// \U2
%-----------------------------
%The ninth is the support of the uvula. There the tip of the tongue becomes attached [to the uvula]. Because of that the nectar of immortality flows from the immortality digit. From drinking the nectar of immortality diseases do not spread in the body. 
%-----------------------------
\note[type=source, labelb=230, lem={\textbf{Re}}]{SSP 2.18 (Ed. p. 35): navame ghaṇṭikādhāre jihvāgraṃ dhārayet | amṛtakalā sravati |}
\note[type=source, labelb=231, lem={\textbf{Re}}]{PT\textsuperscript{qcr \cdot YSV} (Ed. p. 840): navamo ghaṇṭikādhāras tatra jihvāgramagrataḥ (\textit{jihvāgrataḥ kṛte} YK\textsuperscript{ccn \cdot YSV} 2.28 Ed. p. 25) | sampivatyamṛtaṃ tasmād yogajinmṛtyujitparaḥ |}
\app{\lem[wit={ceteri}]{navamo}
  \rdg[wit={B}]{navo}}
\app{\lem[wit={ceteri},alt={ghaṃṭikā°}]{ghaṃṭikā}
  \rdg[wit={P}]{ghaṭikā°}
  \rdg[wit={L}]{ghaṃṭā°}
}\app{\lem[wit={ceteri},alt={°dhāraḥ}]{dhāraḥ}
  \rdg[wit={U1}]{dhāras}}/
tatra
\app{\lem[wit={ceteri}]{jihvāgraṃ}
  \rdg[wit={D}]{jihvāyāgraṃ}
  \rdg[wit={U1}]{juhvāyāṃ}}
lagnaṃ
\app{\lem[wit={ceteri}]{bhavati}
  \rdg[wit={B}]{bhavatī}
  \rdg[wit={U1}]{bhavati vā}}/
\app{\lem[wit={ceteri}]{tato}
  \rdg[wit={N1,U1}]{tataḥ}}
\app{\lem[wit={B,E,P,N1,N2}]{'mṛtakalāyā}
  \rdg[wit={L}]{mṛtakalāyāṃ}
  \rdg[wit={D,U1}]{amṛtakalāyāḥ}}
\app{\lem[wit={ceteri}]{amṛtaṃ}
  \rdg[wit={P}]{amṛta}
  \rdg[wit={L}]{\om}}
\app{\lem[wit={ceteri}]{sravati}
  \rdg[wit={L}]{\om}}/
\app{\lem[wit={P,D},alt={tadamṛtapānāc}]{tadamṛtapānā\skp{c-cha}}
  \rdg[wit={E,N1,N2,U1}]{tadamṛtapānāt}
  \rdg[wit={B}]{tadamṛtakalāyāṃ amṛtapānī°}
  \rdg[wit={L}]{amṛtapānā}
  \rdg[wit={U2}]{tadamṛtapānā}
}\app{\lem[wit={ceteri},alt={charīra°}]{\skm{c-cha}rīra}
  \rdg[wit={E,N1,N2}]{śarīra°}
}madhye 
rogasaṃcāro
\app{\lem[wit={ceteri}]{na}
  \rdg[wit={B,L}]{\om}}
\app{\lem[wit={ceteri}]{bhavati}
  \rdg[wit={B}]{bhavatī}}/\textsuperscript{\begin{otherlanguage}{english}\coro{[\lowroman{30}]}\end{otherlanguage}}
%-----------------------------
%daśamaṃ tālvādhāraḥ/  tanmadhye    vānaṃ dollahanaṃ      ca kṛtvā              laṃbikāpraveśe sati    tāluni magnā jihvā tiṣṭhati/ \E
%daśamas tālvādhāraḥ   tanmadhye  cālanaṃ dohanaṃ         ca kratvā             laṃbikāpraveśe śe sati tālumagnā    jihvā tiṣṭhati  \P %%%7654.jpg
%daśamaṃ stālvādhāraḥ/ tanmadhye  cālanaṃ dohanaṃ         ca kratvā             laṃbikāpraveśe sati    tālumagnā    jihvā tiṣṭhati/ \B
%daśamas tālvādhāraḥ// tanmadhye  cālanaṃ dohanaṃ         ca kṛtvā              laṃbikāpraveśe sati    tālumagnā    jihvā tiṣṭhati ... \L
%daśama  tālvādhāraḥ// tanmadhye  cānanaṃ dohanaṃ         ca kṛtvā              laṃbikāpraveśe grati   tāluni magnā jihvā tiṣṭhati/ \N1
%daśamas tālvādhāraḥ   tanmadhye  cānanaṃ dohanaṃ         ca kṛtvā              laṃbikāpravese grati   tāluni magnā jihvā tiṣṭhati// \D
%daśama  tālvādhāraḥ   tanmadhye  cālanaṃ dohanaṃ         ca kṛtvā              laṃbikāpraveśe grati   tālūni magnā                    \N2
%daśamas tālvādhāraḥ  staṃnmadhye cālanaṃ dohanaṃ         ca sva/sca? kṛtvā cālaṃ vikā praveśe sati    tālūni lagnā juhvā tiṣṭhati \U1 %%%284.jpg
%daśamas tālvādhāraḥ   tanmadhye  cālanaṃ dohanaṃ chedanaṃ ca kṛtvā             laṃbikāpraveśe sati    tāluni magnā jihvā tiṣṭhati// \U2 %%416.jpg
%-----------------------------
%The tenth is the support of the palate. After the moving and milking has been done therein, [and] while abiding at entrance of the uvula, the tongue resides inserted within the palate.
%-----------------------------
\note[type=source, labelb=232, lem={\textbf{Re}}]{SSP 2.19 (Ed. p. 35): daśame tālvādhāre tālvantar garbhe lambikāṃ cālanadohanābhyāṃ dīrghīkṛtvā viparītena praveśayet | kāṣṭhībhavati |}
\note[type=source, labelb=233, lem={\textbf{Re}}]{PT\textsuperscript{qcr \cdot YSV} (Ed. p. 840): daśamas tālukādhāras tatra jihvāgrataḥ kṛte (hemistich \textit{\om} YK\textsuperscript{ccn \cdot YSV} | calane dohane caiva jihvā jaḍati lambitā (\textit{jāyeta lambitam} YK\textsuperscript{ccn \cdot YSV} 2.28cd Ed. p. 25) | nāsikāprāptajihveyaṃ tālulagnā bhavet tataḥ |}
\app{\lem[wit={ceteri},alt={daśamas}]{daśama\skp{s-tā}}
  \rdg[wit={B}]{daśamaṃs}
  \rdg[wit={E}]{daśamaṃ}
  \rdg[wit={N1,N2}]{daśama}
}\skm{s-tā}lvādhāraḥ/
\app{\lem[wit={ceteri}]{tanmadhye}
  \rdg[wit={U1}]{staṃnmadhye}}
\app{\lem[wit={ceteri}]{cālanaṃ}
  \rdg[wit={D}]{cānanaṃ}
  \rdg[wit={E}]{vānaṃ}}
\app{\lem[wit={ceteri}]{dohanaṃ}
  \rdg[wit={E}]{dollahanaṃ}
  \rdg[wit={U2}]{dohanaṃ chedanaṃ}}
ca \app{\lem[wit={ceteri}]{kṛtvā}
  \rdg[wit={B,L}]{kratvā}
  \rdg[wit={U1}]{sva kṛtvā}}
\app{\lem[wit={ceteri}]{laṃbikā}
  \rdg[wit={U1}]{cālaṃ vikā}
}praveśe
\app{\lem[wit={ceteri}]{sati}
  \rdg[wit={P}]{śe sati}
  \rdg[wit={D,N1,N2}]{grati}}
\app{\lem[wit={ceteri}]{tālunimagnā}
  \rdg[wit={N2,U1,U2}]{tālūnimagnā}
  \rdg[wit={B,P,L}]{tālumagnā}}
\app{\lem[wit={ceteri}]{jihvā}
  \rdg[wit={U1}]{juhvā}
  \rdg[wit={N2}]{\om}}
\app{\lem[wit={ceteri}]{tiṣṭhati}
  \rdg[wit={N2}]{\om}}/
%-----------------------------
%ekādaśo           jihvādhāraḥ/  tasmin   jihvāgreṇa manthanaṃ kriyate   tasmin  kṛte   timadhuraṃ  pānīyaṃ sravati/  tadā                            ca kavitva------cchandonāṭakādiviṣayajñānam utpadyate/ \E
%ekādaśo jihvātale jihvādhāraḥ   tasmin   jihvāgreṇa mathanaṃ  kriyate   tasmin  kṛte   timadhuraṃ  pānīyaṃ sravati   tathā                           ca kavitva------chaṃdonāṭakādiviṣayajñānam  utpadyate  \P
%ekādaśo jihvātale jihvādhāraḥ// tasmin   jihvāgreṇa manthanaṃ kṛtvā//   tasmiṃ  kṛte satimadhuraṃ  pānīyaṃ sravatī// tathā                              kvacitva-----cchaṃdonāṭakādiviṣayapānam  utpadyaṃte/ \B
%ekādaśo jihvātale jihvādhāraḥ// tasmin   jihvāgreṇa mathanaṃ  kṛtvā//   tasmiṃ  kṛte satimadhuraṃ  pānīyaṃ sravati// tathā                              kvacitva-----chaṃdonāṭakādiviṣayajñānam  utpadyate// \L
%ekādaśo           jihvādhāraḥ/  tasmin   jihvāgreṇa manthanaṃ kriyate/  tasmin  kṛte atimadhuraṃ   pānīyaṃ sravati/  tathā                           ca kavitva--gītacchaṃdanāṭakādiviṣaye jñānam utpadyate/ \N1
%ekādaśo jihvātale jihvādhāraḥ// tasmin   jihvāgreṇa mathanaṃ  kriyate// tasmiṃ  kṛte satimadhuraṃ  pānīyaṃ sravati// tathā                           ca kvacitta-----chaṃdanāṭakādiviṣayajñānam   utpadyate \D 
%                                         jihvāgreṇa manthanaṃ kriyate// tasmin  kṛte atimadhuraṃ   pānīyaṃ sravati// kaminnāsikā phatkāravat// tathā ca kavitvagīta--chaṃdanāṭakādiviṣaye jñānam  utpadyate/ \N2
%ekādaśā jihvātale jihvādhāraḥ  tasmin na jihvāgreṇa manthanaṃ kriyate   tasminn kṛte  timadhuraṃ   pānīyaṃ sravati   tathā                           ca kavitvagīta--chaṃdavacchaṃdanāḍīviṣayaṃ jñānānam utpadyate \U1
%ekādaśo jihvātale jihvādhāraḥ   tasmin   jihvāgreṇa manthanaṃ kriyate// tasminn kṛte 'timadhuraṃ   pānīyaṃ sravati// tathā                           ca kavitvaṃ     chaṃdonāṭakādiviṣayajñānam  utpadyate// \U2
%-----------------------------
%The eleventh is the tongue support at the base of the tongue. Therein the tip of the tongue has to be churned. While doing that, an very sweet drink flows out. And in that manner the knowledge of areas like poetry, singing, metric and dance is generated. 
%----------------------------
\note[type=source, labelb=234, lem={\textbf{Re}}]{SSP 2.20 (Ed. p. 35): ekādaśe atha jihvādhāre tatra jihvāgraṃ dhārayet | sarvaroganāśo bhavati |}
\note[type=source, labelb=235, lem={\textbf{Re}}]{PT\textsuperscript{qcr \cdot YSV} (Ed. p. 840): ekādaśī (\textit{ekādaśo} YK\textsuperscript{ccn \cdot YSV} 2.29 Ed. p. 25) bhavej jihvā talajādhāra īśvari | jihvāgramathane tasmin pānīyaṃ madhuraṃ bhavet | tatpīteṣu kavir gītijyotiś (\textit{gītir} YK\textsuperscript{ccn \cdot YSV} 2.29 Ed. p. 25) chandovidāṃ (\textit{chandovidur} YK\textsuperscript{ccn \cdot YSV} 1.259 Ed. p. 21) varaḥ |}
\app{\lem[wit={ceteri}]{ekādaśo}
  \rdg[wit={N2}]{\om}}
\app{\lem[wit={ceteri}]{jihvātale}
  \rdg[wit={E,N1,N2}]{\om}}
\app{\lem[wit={ceteri}]{jihvādhāraḥ}
  \rdg[wit={N2}]{\om}}/
\app{\lem[wit={ceteri}]{tasmin}
  \rdg[wit={U1}]{tasmin na}
  \rdg[wit={N2}]{\om}}
jihvāgreṇa
\app{\lem[wit={ceteri}]{manthanaṃ}
  \rdg[wit={D,L,P}]{mathanaṃ}}
\app{\lem[wit={ceteri}]{kriyate}
  \rdg[wit={B,L}]{kṛtvā}}/
tasmin-kṛte
\app{\lem[wit={ceteri}]{'timadhuraṃ}
  \rdg[wit={N1,N2}]{atimadhuraṃ}
  \rdg[wit={B,D,L}]{satimadhuraṃ}}
pānīyaṃ
\app{\lem[wit={ceteri}]{sravati}
  \rdg[wit={B}]{sravatī}}/ 
\app{\lem[wit={ceteri}]{tathā}
  \rdg[wit={E}]{tadā}
  \rdg[wit={N2}]{kamin nāsikā phatkāravat || tathā}}
\app{\lem[wit={ceteri}]{ca}
  \rdg[wit={B,L}]{\om}}
\app{\lem[wit={ceteri},alt={kavitva°}]{kavitva}
  \rdg[wit={B,L}]{kvacitva°}
  \rdg[wit={D}]{kvacitta°}
  \rdg[wit={U2}]{kavitvaṃ}
}\app{\lem[wit={N1,N2,U1},alt={°gīta°}]{gīta}
  \rdg[wit={ceteri}]{\om}
}\app{\lem[wit={Y},alt={°chando°}]{chando}
  \rdg[wit={U1}]{°chaṃdavacchaṃda°}
  \rdg[wit={ceteri}]{°chaṃda°}
}\app{\lem[wit={ceteri},alt={°nāṭakādi°}]{nāṭakādi}
  \rdg[wit={U1}]{°nāḍī°}} 
\app{\lem[wit={Y,D},alt={°viṣaya°}]{viṣaya}
  \rdg[wit={N1,N2}]{°viṣaye}
  \rdg[wit={U1}]{viṣayaṃ}}
\app{\lem[wit={ceteri},alt={jñānam}]{jñāna\skp{m-u}}
  \rdg[wit={U1}]{jñānānam}
}\app{\lem[wit={ceteri},alt={utpadyate}]{\skm{m-u}tpadyate}
  \rdg[wit={B}]{utpadyaṃte}}/\textsuperscript{\begin{otherlanguage}{english}\coro{[\lowroman{35}]}\end{otherlanguage}}
    \end{prose}
  \end{edition}
  \begin{translation}
    \ekddiv{type=trans}
    \begin{tlate}
      \noindent
The seventh is the support of the heart-form. From the restraint of the Prāṇa vitalwind in this location also the six upward-facing lotusses open up. The throat support is the eighth. There the binding of Jālaṅdhara\footnote{Write about the pīṭha and the bandha!} is produced.\textsuperscript{\coro{[\lowroman{25}]}} While abiding therein the vitalwind in the Iḍā and Piṅgalā channel becomes stable. The ninth is the support of the uvula. There the tip of the tongue becomes attached [to the uvula]. Because of that the nectar of immortality flows from the immortality digit. From drinking the nectar of immortality diseases do not spread in the body.\textsuperscript{\coro{[\lowroman{30}]}}
The tenth is the support of the palate. After the moving and milking has been done therein, [and] while abiding at entrance of the uvula, the tongue resides inserted within the palate. The eleventh is the tongue support at the base of the tongue. Therein the tip of the tongue has to be churned. While doing that, an very sweet drink flows out. And in that manner the knowledge of areas like poetry, singing, metric and dance is generated.\textsuperscript{\coro{[\lowroman{35}]}}
    \end{tlate}
  \end{translation}
\end{alignment}
%\ekdpb*{}
%%%%%%%%%%%%%%%%%%%%%%%%%%%%%%%%%%%%%%%%%%
%%%%%%%%%%%%%%%%%%%%%%%%%%%%%%%%%%%%%%%%%% 
%%%%%%%% PAGEBREAK%%%%%%%PAGEBREAK%%%%%%%%%
%%%%%%%%%%%%%%%%%%%%%%%%%%%%%%%%%%%%%%%%%% 
%%%%%%%%%%%%%%%% PAGEBREAK%%%%%%%%%%%%%%%%%
%%%%%%%%%%%%%%%%%%%%%%%%%%%%%%%%%%%%%%%%%% 
%%%%%%%% PAGEBREAK%%%%%%%PAGEBREAK%%%%%%%%%
%%%%%%%%%%%%%%%%%%%%%%%%%%%%%%%%%%%%%%%%%% 
%%%%%%%%%%%%%%%%%%%%%%%%%%%%%%%%%%%%%%%%%% 
%%%%%%%%%%%%%%%%%%%%%%%%%%%%%%%%%%%%%%%%%% 
%%%%%%%%%%%%%%%%%%%%%%%%%%%%%%%%%%%%%%%%%% 
%%%%%%%% PAGEBREAK%%%%%%%PAGEBREAK%%%%%%%%%
%%%%%%%%%%%%%%%%%%%%%%%%%%%%%%%%%%%%%%%%%% 
%%%%%%%%%%%%%%%% PAGEBREAK%%%%%%%%%%%%%%%%%
%%%%%%%%%%%%%%%%%%%%%%%%%%%%%%%%%%%%%%%%%% 
%%%%%%%% PAGEBREAK%%%%%%%PAGEBREAK%%%%%%%%%
%%%%%%%%%%%%%%%%%%%%%%%%%%%%%%%%%%%%%%%%%% 
%%%%%%%%%%%%%%%%%%%%%%%%%%%%%%%%%%%%%%%%%% 
%%%%%%%%%%%%%%%%%%%%%%%%%%%%%%%%%%%%%%%%%% 
%%%%%%%%%%%%%%%%%%%%%%%%%%%%%%%%%%%%%%%%%% 
%%%%%%%% PAGEBREAK%%%%%%%PAGEBREAK%%%%%%%%%
%%%%%%%%%%%%%%%%%%%%%%%%%%%%%%%%%%%%%%%%%% 
%%%%%%%%%%%%%%%% PAGEBREAK%%%%%%%%%%%%%%%%%
%%%%%%%%%%%%%%%%%%%%%%%%%%%%%%%%%%%%%%%%%% 
%%%%%%%% PAGEBREAK%%%%%%%PAGEBREAK%%%%%%%%%
%%%%%%%%%%%%%%%%%%%%%%%%%%%%%%%%%%%%%%%%%% 
%%%%%%%%%%%%%%%%%%%%%%%%%%%%%%%%%%%%%%%%%% 
\begin{alignment}[
  texts=edition[class="edition"];
  translation[class="translation"],
  ]
  \begin{edition}
    \ekddiv{type=ed}
    \begin{prose}
\noindent
%----------------------------
%tadupari dvādaśadantayo   madhye   dantādhāraḥ/  tasmin sthāne jihvāyā  agraṃ  ghaṭīmātraṃ                    balātkāreṇa  sthāpyate/  tasmin  sati sādhakasya samagrā rogā naśyanti// \E %%%[p.44]
%tadupari dvādaśo daṃtayor madhye   daṃtādhāraḥ   tasmin sthāne jihvāyā  agraṃ  ghaṭīmātram ārghaghaṭīmātraṃ   bālātkāreṇa  sthāpyate   tasmin  sati sādhakasya samagrā rogā naśyaṃti \P
%tadupari dvādaśo daṃtayor madhye// daṃtādhāraḥ// tasmin sthāne jihvāyā  'agnaṃ ghaṭīmātram ārghaghaṭimātraṃ   bālākāreṇa   sthāpyate// tasmiṃ       sādhakasya samagrā rogā naśyaṃtī// \B
%tadupari dvādaśo daṃtayor madhye// daṃtādhāraḥ   tasmin sthāne jihvāyā  agnaṃ  ghaṭīmātram ārddhaghaṭimātraṃ  bālākāreṇa   sthāpyate// tasmiṃ       sādhakasya samagrā rogā naśyaṃti... \L
%tadupari dvādaśayor       madhye   daṃtādhāraḥ/  tasmin sthāne jihvāyā  agraṃ  ghaṭīmātraṃ arddhaghaṭimātraṃ  balātkāreṇa  sthāpyate// tasmin  sati sādhakasya samagrā rogā naśyaṃti// \N1
%tadupari dvādaśayor       madhye   daṃtādhāraḥ// tasmin sthāne jihvāyā  agraṃ  ghaṭīmātraṃ arddhaghaṭimātraṃ  balātkāreṇa  sthāpyate// tasmin  sati sādhakasya samagrā rogā naśyaṃti// \D
%tadupari dvādaśayor       madhye   daṃtādhāraḥ// tasmin sthāne jihvāyā   graṃ  ghaṭīmātraṃ arddhaghaṭimātraṃ  balātkāreṇa  sthāpyate// tasmin  sati sādhakasya samagrā rogā naśyanti \N2
%tadupari dvādaśo daṃtayor madhye   daṃtādhāraḥ   tasmin sthāne jihvāyāṃ agraṃ  ghaṭīmātram ārdhaghaṭikāmātraṃ bālātkāreṇa  sthāpyate   tasminn sati sādhakasya samagra rogā naśyaṃti \U1
%tadupari dvādaśor daṃtayo madhye   daṃtādhāraḥ   tasmin sthāne jihvāyā  agraṃ  ghaṭīmātram ārghaghaṭīmātraṃ   bālātkāreṇa  sthāpyate// tasmin  sati sādhakasya samagrā rogā naśyaṃti// \U2
%-----------------------------
%Above that is the twelfth - within the teeth is the tooth support. At this place the tip of the tongue is to be positioned with force for the duration of one and a half \textit{ghāṭī}s (24+12 = 36 minutes). Abiding therein the diseases of the practitioner will entirely disappear!
%----------------------------
\note[type=source, labelb=236, lem={\textbf{Re}}]{SSP 2.21 (Ed. p. 36): dvādaśe bhrūmadhyādhāre tatra candramaṇḍalaṃ dhyāyet śītalatāṃ yāti |}
\note[type=source, labelb=237, lem={\textbf{Re}}]{PT\textsuperscript{qcr \cdot YSV} (Ed. p. 840): dantādhāro (\textit{dvandvādhāro} YK\textsuperscript{ccn \cdot YSV} 1.259 Ed. p. 21) dvādaśeti sarvarogakṣayaṅkaraḥ (\textit{sarvarogaḥ} YK\textsuperscript{ccn \cdot YSV} 1.259 Ed. p. 21) | dhārayed dantayor madhye jihvāgrañ ca balād api | dhṛtvārddhaghaṭikāmātraṃ sarvarogan (\textit{sarvarogāṃs} YK\textsuperscript{ccn \cdot YSV} 1.259 Ed. p. 21) tu nāśayet |}
tadupari
\app{\lem[wit={P,B,L,U1},alt={dvādaśo daṃtayor}]{dvādaśo dantayo\skp{r-ma}}
  \rdg[wit={E}]{dvādaśadantayo}
  \rdg[wit={U2}]{dvādaśor daṃtayo}
  \rdg[wit={D,N1,N2}]{dvādaśayor}
}\skm{r-ma}dhye
dantādhāraḥ/
tasmin sthāne
\app{\lem[wit={ceteri}]{jihvāyā}
  \rdg[wit={U1}]{jihvāyāṃ}}
\app{\lem[wit={ceteri}]{agraṃ}
  \rdg[wit={B,L}]{agnaṃ}
  \rdg[wit={N2}]{graṃ}}
\app{\lem[wit={ceteri},alt={ghaṭīmātraṃ}]{ghaṭīmātra\skp{m-a}}
  \rdg[wit={D,N1,N2}]{ghaṭīmātraṃ}
}\app{\lem[type=emendation, resp=egoscr,alt={ardhagaṭīmātraṃ}]{\skm{m-a}rdhagaṭīmātraṃ}
  \rdg[wit={D,N1,N2}]{\korr arddhaghaṭimātraṃ}
  \rdg[wit={U1}]{ārdhaghaṭikāmātraṃ}
  \rdg[wit={P,U2}]{ārghaghaṭīmātraṃ}
  \rdg[wit={B}]{ārghaghaṭimātraṃ}
  \rdg[wit={L}]{ārddhaghaṭimātraṃ}
  \rdg[wit={E}]{\om}}
\app{\lem[wit={E,D,N1,N2}]{balātkāreṇa}
  \rdg[wit={P,U1,U2}]{bālātkāreṇa}
  \rdg[wit={B,L}]{bālākāreṇa}}
sthāpyate/
\app{\lem[wit={ceteri}]{tasmin}
  \rdg[wit={B,L}]{tasmiṃ}}
\app{\lem[wit={ceteri}]{sati}
  \rdg[wit={B,L}]{\om}}
sādhakasya samagrā rogā
\app{\lem[wit={ceteri}]{naśyanti}
  \rdg[wit={B}]{naśyaṃtī}}/
%----------------------------
%trayodaśo nāsikāgrādhāraḥ/ tasmin lakṣye kṛte sati manaḥ sthiraṃ bhavati/ \E
%trayodaśo nāsikāgrādhāraḥ  tasmiṃ lakṣye kṛte sati manaḥ sthiraṃ bhavati \P
%trayodaso nāsikādhāraḥ/    tasmin ḍraṣṭe kṛte      minasthire    bhavati/ \B
%trayodaso nāśikādhāraḥ     tasmin ḍraṣṭe kṛte      manaḥ sthiro  bhavati/ \L
%trayodaśo nāsikādhāraḥ/    tasmin lakṣe  kṛte sati manasthiraṃ   bhavati/ \N1
%trayodaśo nāsikādhāraḥ//   tasmin lakṣe  kṛte sati manasthiraṃ   bhavati \D
%trayodaśo nāsikādhāraḥ/    tasmin lakṣe  kṛte sati manasthiraṃ   bhavati/ \N2
%trayodaśo nāsikādhāraḥ     tasmiṃ lakṣye kṛte sati manasthiraṃ   bhavati \U1
%trayodaśo nāsikādhāraḥ     tasmil lakṣe  kṛte sati manasthiraṃ   bhavati// \U2
%-----------------------------
%The thirteenth is the support of the nose. While turning it into the object of fixation the mind becomes stable. 
%----------------------------
\note[type=source, labelb=238, lem={\textbf{Re}}]{SSP 2.22 (Ed. p. 36): trayodaśe nāsādhāre tasyāgraṃ lakṣayet manaḥ sthiraṃ bhavati |}
\note[type=source, labelb=239, lem={\textbf{Re}}]{PT\textsuperscript{qcr \cdot YSV} (Ed. p. 832): nāsādhāras tato (\textit{tataḥ} YK\textsuperscript{ccn \cdot YSV} 2.259 Ed. p. 21) jñeyo nāsālakṣas trayodaśaḥ (\textit{trayodaśa} YK\textsuperscript{ccn \cdot YSV} 1.259 Ed. p. 21) | manaḥsthirakaro yas tu (\textit{sthiraṃ karoty eva} YK\textsuperscript{ccn \cdot YSV} 1.259 Ed. p. 21) vāyusthirakaro (\textit{vāyuḥ} YK\textsuperscript{ccn \cdot YSV} 1.259 Ed. p. 21) mahān |}
\app{\lem[wit={ceteri}]{ nāśikādhāraḥ}
  \rdg[wit={E,P}]{nāsikāgrādhāraḥ}}/
\app{\lem[type=emendation, resp=egoscr]{tasmil-lakṣye}
  \rdg[wit={U2}]{tasmil lakṣe}
  \rdg[wit={E,P,U1}]{tasmiṃ lakṣye}
  \rdg[wit={D,N1,N2}]{tasmin lakṣe}
  \rdg[wit={B,L}]{tasmin ḍraṣṭe}}
kṛte
\app{\lem[wit={ceteri}]{sati}
  \rdg[wit={B,L}]{\om}}
\app{\lem[wit={E,P}]{manaḥ sthiraṃ}
  \rdg[wit={B}]{minasthire}
  \rdg[wit={L}]{manaḥ sthiro}
  \rdg[wit={ceteri}]{manasthiraṃ}}
bhavati/\textsuperscript{\begin{otherlanguage}{english}\coro{[\lowroman{40}]}\end{otherlanguage}}
%----------------------------
%caturdaśo nāsāmūlādhāraḥ/         tasmin dṛṣṭeḥ            sthairyakāraṇāt   ṣaṣṭhe māsi svīyan tejaḥ pratyakṣaṃ bhavati/  tejasaḥ pratyakṣatve pārthivaṃ sakalaṃ bandhanaṃ tuṭyati/   \E
%caturdaśo nāsāmūlādhāro           tasmin dṛṣṭeḥ            sthairyakāraṇāt   ṣaṣṭhe māsi svīyaṃ tejaḥ pratyakṣaṃ bhavati   tejasaḥ pratyakṣatve pārthivaṃ sakalaṃ baṃdhanaṃ truṭyati/ \P %%%7654.jpg vorletzte Zeile
%caturdaśo nāso mūlādhāraḥ//       tasmin llakṣe krute satī sthairyakāraṇāt// ṣaṣṭhe māse svayaṃ tejaḥ pratyakṣaṃ bhavati// tejasaḥ pratyakṣatve pārthivaṃ sakalaṃ baṃdhanaṃ truṭayati/ \B
%caturdaśo nāso mūlādhāraḥ         tasmin lakṣe kṛte satī   sthairyakāraṇāt   ṣaṣṭhe māse svayaṃ tejaḥ pratyakṣaṃ bhavati// tejasaḥ pratyakṣatve pārthivaṃ sakalaṃ baṃdhanaṃ truṭayati/ \L
%caturdaśo nāsāmūle vāyvādhāraḥ/   tasmin dṛṣṭeḥ            sthairyakāraṇāt   ṣaṣṭhe māsi svīyaṃ tejaḥ pratyakṣaṃ bhavati/  tejasaḥ pratyakṣatve pārthivaṃ sakalaṃ baṃdhanaṃ trudyati/  \N1
%caturdaśo nāsāmūle vāyvādhāraḥ//  tasmin dṛṣṭeḥ            sthairyakāraṇāt   ṣaṣṭhe māsi svīyaṃ tejaḥ pratyakṣaṃ bhavati// tejasaḥ pratyakṣatve pārthivaṃ sakalaṃ baṃdhanaṃ trudyati// \D  %%%p.13 recto 
%caturdaśo nāsāmūle vāyvādhāraḥ??/ tasmin dṛṣṭeḥ            sthairyakāraṇāt   ṣaṣṭhe māsi svayaṃ tejaḥ pratyakṣaṃ bhavati   tejasaḥ pratyakṣatve pārthiva  sakalaṃ bandhanaṃ trudyati// \N2
%caturdaśo nāsāmūle vādhāraḥ       tasmiṃ na dṛṣṭeḥ         sthairyakāraṇāt   ṣaṣṭhe māse svīyaṃ tejaḥ pratyakṣaṃ bhavati   tejasaḥ pratyakṣatve pārthivaṃ sakalaṃ baṃdhanaṃ truṭyati   \U1
%caturdaśo nāsāmūlādhāraḥ          tasmin laṣṭhe?           sthairyakāraṇāt   ṣaṣṭhe māsi svayaṃ tejaḥ pratyakṣaṃ bhavati// tejasaḥ pratyakṣatve pārthivaṃ sakalaṃ baṃdhanaṃ truṭyati// \U2
%-----------------------------
%The fourteenth is the support of the vitalwind at the root of the nose. From the execution of stabilizing the gaze therein, direct perception of one's own light arises within sixty months. He breaks all bonds of the mundane world in the direct perception of the light. 
%-----------------------------
\note[type=source, labelb=240, lem={\textbf{Re}}]{SSP 2.23 (Ed. p. 36): caturdaśe nāsāmūle kapāṭādhāre dṛṣṭiṃ dhārayet | ṣaṇmāsāj jyotiḥpuñjaṃ paśyati |}
\note[type=source, labelb=241, lem={\textbf{Re}}]{PT\textsuperscript{qcr \cdot YSV} (Ed. p. 839) =  YK\textsuperscript{ccn \cdot YSV} 2.33ab-34cd Ed. p. 25): nāsāpuṭe sthirā dṛṣṭir ādhāro 'yaṃ caturdaśaḥ | kṛte 'smin svīyatejaḥ syāt pratyakṣaṃ ṣaṭtrimāsataḥ | pārthivaṃ truṭati kṣipraṃ pratyakṣaṃ svīyatejasā |}
caturdaśo
\app{\lem[wit={D,N1,N2}]{nāsāmūle vāyvādhāraḥ}
  \rdg[wit={U1}]{nāsāmūle vādhāraḥ}
  \rdg[wit={P}]{nāsāmūlādhāro}
  \rdg[wit={B,L}]{nāso mūlādhāraḥ}
  \rdg[wit={E,U2}]{nāsāmūlādhāraḥ}}
\app{\lem[wit={ceteri}]{tasmin}
  \rdg[wit={ceteri}]{tasmiṃ na}}
\app{\lem[wit={ceteri}]{dṛṣṭeḥ}
  \rdg[wit={B}]{llakṣe krute satī}
  \rdg[wit={L}]{lakṣe kṛte satī}
  \rdg[wit={U1}]{na dṛṣṭeḥ}
  \rdg[wit={U2}]{laṣṭhe}}
sthairyakāraṇāt
ṣaṣṭhe
\app{\lem[wit={B,L,U1}]{māse}
  \rdg[wit={ceteri}]{māsi}} 
\app{\lem[wit={ceteri}]{svīyaṃ}
  \rdg[wit={B,L,N2,U2}]{svayaṃ}}
tejaḥ pratyakṣaṃ bhavati/
tejasaḥ pratyakṣatve
\app{\lem[wit={ceteri}]{pārthivaṃ}
  \rdg[wit={N2}]{pārthiva}}
bandhanaṃ 
\app{\lem[wit={P,U2,U1}]{truṭyati}
  \rdg[wit={E}]{tuṭyati}
  \rdg[wit={B,L}]{truṭayati}
  \rdg[wit={N1,N2,D}]{trudyati}}/
%----------------------------
%pañcadaśo bhruvormadhyādhāras        tasmin dṛṣṭeḥ sthirīkaraṇāt    koṭikiraṇāḥ  sphuraṃti/ \E
%paṃcadaśo bhruvormadhyādhāraḥ        tasmin ḍṛṣṭeḥ sthirīkaraṇāt    koṭikiraṇāḥ  sphuraṃti  \P  %%%7655.jpg
%paṃcadaśo bhruvormadhye dhāraḥ//     tasmin ḍṛṣṭeḥ sthirikaraṇāt//  koṭikiriṇā   sphuraṃti// \B
%paṃcadaśo bhruvormadhye dhāraḥ//     tasmin ḍṛṣṭe  sthirīkaraṇāt//  koṭikiriṇā   sphuraṃti// \L
%pañcadaśo bhruvormadhye ādhāraḥ/      asmin dṛṣṭeḥ sthirīkaraṇāt    koṭikiraṇāni sphuraṃti/ \N1
%pañcadaśo bhruvormadhye ājñādhāraḥ// ..smin dṛṣṭeḥ sthirīkaraṇāt    koṭikiraṇāni sphuraṃti// \D
%pañcadaśo bhruvormadhye ādhāraḥ      tasmin dṛṣṭeḥ sthirīkaraṇāt    koṭikiraṇāni sphuraṃti/ \N2 [S.9]
%pañcadaśo bhruvormadhye ādhāra         asin na dṛṣṭeḥ sthirīkaraṇāt koṭikiraṇāni sphuraṃti \U1
%pañcadaśo bhruvormadhyādhāra         tasmin dṛṣṭisthirīkaraṇāt      koṭikiraṇaḥ  sphuraṃti// \U2
%-----------------------------
%The fifteenth container is situated in the middle of the eyebrows. Due to stabilized the gaze therein, ten million rays of light sparkle. 
%----------------------------
\note[type=source, labelb=242, lem={\textbf{Re}}]{SSP 2.24 (Ed. pp. 36-37): pañcadaśe lalāṭādhāre tatra jyotiḥpuñjaṃ lakṣayet | tejasvī bhavati |}
\note[type=source, labelb=243, lem={\textbf{Re}}]{PT\textsuperscript{qcr \cdot YSV} (Ed. p. 839): pañcadaśo bhruvormadhye sthira (\textit{sthirā} YK\textsuperscript{ccn \cdot YSV} 2.35 Ed. p. 25) dṛṣṭis tathā dhruvam | asmin dṛṣṭiḥ sthirā koṭiḥ (\textit{koṭi°} YK\textsuperscript{ccn \cdot YSV} 2.35 Ed. p. 25) kiraṇāni sphuranti hi |}
pañcadaśo
\app{\lem[wit={P}]{bhruvormadhyādhāraḥ}
  \rdg[wit={E}]{bhruvormadhyādhāras}
  \rdg[wit={B,L}]{bhruvormadhye dhāraḥ}
  \rdg[wit={D}]{bhruvormadhye ājñādhāraḥ}
  \rdg[wit={N1,N2}]{bhruvormadhye ādhāraḥ}
  \rdg[wit={U1}]{bhruvormadhye ādhāra}
  \rdg[wit={U2}]{bhruvormadhyādhāra}}/
\app{\lem[wit={ceteri}]{tasmin}
  \rdg[wit={N1}]{asmin}
  \rdg[wit={D}]{smin}
  \rdg[wit={U1}]{asin}
}\app{\lem[wit={ceteri}]{ḍṛṣṭeḥ}
  \rdg[wit={L}]{ḍṛṣṭe}
  \rdg[wit={U1}]{na dṛṣṭeḥ}
  \rdg[wit={U2}]{dṛṣṭi°}}
sthirīkaraṇāt
koṭi\app{\lem[wit={D,N1,N2,U1}]{kiraṇāni}
  \rdg[wit={E,P}]{koṭikiraṇāḥ}
  \rdg[wit={U2}]{koṭikiraṇaḥ}
  \rdg[wit={B,L}]{koṭikiriṇā}}
sphuranti/\textsuperscript{\begin{otherlanguage}{english}\coro{[\lowroman{45}]}\end{otherlanguage}}
%----------------------------
%ṣoḍaśo  netrādhāraḥ/  ayam aṃgulyagreṇa cālyate/  tadabhyāsāt/ pṛthvīmadhye  yatkiṃcin  tejo  varttate/  \E   %%%p.45
%ṣoḍaśo  netrādhāraḥ   ayam aṃgulyagreṇa cālyate   tadabhyāsāt  pṛthvīmadhye  yatkiṃcit  tejo  vartate... \P
%ṣoḍaśo  netrā//       ayam aṃgulyagreṇa cālyate// tadabhyāsāt  pṛthivīmadhye yatkiṃcit  tejo  vartate//  \B %%%%%%%%%%%%%%%%DSCN7167.jpg Z. 1
%ṣoḍaśo  netrā//       ayam aṃgulyagreṇa cālyate// tadabhyāsāt  pṛthivīmadhye yatkiṃcit  tejo  vartate... \L
%ṣoḍaśaḥ netrādhāraḥ/  ayaṃ agulyagreṇa  cālyate/  tadabhyāsāt  pṛthvīmadhye  yatkiṃcit  tejaḥ varttate/  \N1
%ṣoḍaśaḥ netrādhāraḥ// ayaṃ agulyagreṇa  cālyate// tadabhyāsāt  pṛthvīmadhye  yatkiṃcit  tejaḥ varttate \D
%ṣoḍaśaḥ netrādhāraḥ/  ayaṃ aṃgugreṇa    cālyate/  tadabhyāsāt  pṛthvīmadhye  yatkiṃcit  tejaḥ varttate/  \N2
%ṣoḍaśo  netrādhāraḥ   ayaṃ aṃgulyagreṇa cālyate   tadābhyāsāt  pṛthvīmadhye  yatkiṃcit        vatate     \U1 %%%%%%%%%%%%%%%%%%285.jpg
%ṣoḍaśo  netrādhāraḥ   ayam aṃgulyagreṇa cālyate// tadabhyāsāt  pṛthivīmadhye yatkiṃcit// tejo vartate//  \U2
%-----------------------------
%[If the gaze] is held at the tip of the finger without wavering, this is the eye support, the sixteenth. Through that practice, some energy arises in the middle of the earth.   
%-----------------------------
\note[type=source, labelb=245, lem={\textbf{Re}}]{SSP 2.25 (Ed. p. 37): avaśiṣṭe ṣoḍaśe brahmarandhram ākāśacakram | tatra śrīgurucaraṇāmbujayugmaṃ sadāvalokayet | ākāśavat pūrṇo bhavati |}
\note[type=source, labelb=246, lem={\textbf{Re}}]{PT\textsuperscript{qcr \cdot YSV} (Ed. pp. 840-41): netrādhāraḥ ṣoḍaśo 'yam (\textit{aṅgulyagre na} YK\textsuperscript{ccn \cdot YSV} 2.36 Ed. p. 25) aṅgulyagreṇa cālayet | pṛthvīmadhye tu yatkiñcid varttate (\textit{sarvajñaḥ prabhavas tena varddhate} YK\textsuperscript{ccn \cdot YSV} 2.36 Ed. p. 25) jaṭharānalaḥ | pratyakṣaṃ tad bhavet sarvaṃ tad ābhyāsān na saṃśayaḥ |}
\app{\lem[wit={ceteri}]{ṣoḍaśo}
  \rdg[wit={D,N1,N2}]{ṣoḍaśaḥ}}
\app{\lem[wit={ceteri}]{netrādhāraḥ}
  \rdg[wit={L,B}]{netrā}}/
\app{\lem[wit={ceteri},alt={ayam}]{aya\skp{m-a}}
  \rdg[wit={D,N1,N2,U1}]{ayaṃ}
}\app{\lem[type=emendation, resp=egoscr, alt={aṅgulyagre na}]{\skm{m-a}ṅgulyagre na}
  \rdg[wit={ceteri}]{aṅgulyagreṇa}
  \rdg[wit={N1,D}]{agulyagreṇa}
  \rdg[wit={N2}]{aṃgugreṇa}}
cālyate/
tadabhyāsāt
\app{\lem[wit={ceteri},alt={pṛthvī°}]{pṛthvī}
  \rdg[wit={L,B,U2}]{pṛthivī°}}madhye
yatkiṃcit
\app{\lem[wit={ceteri}]{tejo}
  \rdg[wit={D,N1,N2}]{tejaḥ}
  \rdg[wit={U1}]{\om}}
\app{\lem[wit={ceteri}]{vartate}
  \rdg[wit={U1}]{vatate}}/
%----------------------------
%tatsarvaṃ tejo   dṛṣṭiviṣayaṃ bhavati/  taddarśanāt  puruṣaḥ sarvajño  bhavati// \E
%tatsarvaṃ tejo   dṛṣṭiviṣayaṃ bhavati   tadarśanāt   puruṣaḥ sarvajño  bhavati     \P
%tatsarvaṃ tejo   dṛṣṭiviṣayaṃ bhavatī// taddarśanāt  puruṣaḥ sarvajño  bhavatī// \B
%tatsarvaṃ tejo   dṛṣṭiviṣayaṃ bhavati// taddarśanāt  puruṣaḥ sarvajño  bhavati// \L
%tatsarvvatejo    dṛṣṭiviṣayaṃ bhavati   taddarśanāt  puruṣaḥ sarvvajño bhavati// \N1
%tatsarvatejo     dṛṣṭiviṣayaṃ bhavati   taddarśanāt  puruṣaḥ sarvvajño bhavati// \D
%tatsarvatejo     dṛṣṭiviṣayaṃ bhavati   taddarśanāt  puruṣaḥ sarvajño  bhavati// \N2
%tatsarvaṃ tejo   dṛṣṭīviṣayaṃ bhavati   tatdarśaḥ    puruṣaḥ sarvajño  bhavati \U1
%tatsarvaṃ tajaso dṛṣṭiviṣayaṃ bhavati// taddarśanāt  puruṣaḥ sarvajño  bhavati// \U2
%-----------------------------
%The entire light of it becomes the object of vision. Through its perception, a person becomes all-knowing.
%-----------------------------
\app{\lem[wit={D,N1,N2}]{tatsarvatejo}
  \rdg[wit={ceteri}]{tatsarvaṃ}}
dṛṣṭiviṣayaṃ
\app{\lem[wit={ceteri}]{bhavati}
  \rdg[wit={B}]{bhavatī}}
\app{\lem[wit={ceteri}]{taddarśanāt}
  \rdg[wit={P}]{tadarśanāt}
  \rdg[wit={U1}]{tatdarśaḥ}}
puruṣaḥ
sarvajño 
\app{\lem[wit={ceteri}]{bhavati}
  \rdg[wit={B}]{bhavatī}}/\textsuperscript{\begin{otherlanguage}{english}\coro{[\lowroman{50}]}\end{otherlanguage}}
    \end{prose}
  \end{edition}
  \begin{translation}
    \ekddiv{type=trans}
    \begin{tlate}
      \noindent
      Above that is the twelfth - within the teeth is the tooth support. At this place the tip of the tongue is to be positioned with force for the duration of one and a half \textit{ghāṭī}s\footnote{(24+12 = 36 minutes)}. Abiding therein the diseases of the practitioner will entirely disappear!
The thirteenth is the support of the nose. While turning it into the object of fixation the mind becomes stable.\textsuperscript{\coro{[\lowroman{40}]}}
The fourteenth is the support of the vitalwind at the root of the nose. From the execution of stabilizing the gaze therein, direct perception of one's own light arises within sixty months. He breaks all bonds of the mundane with by direct perception of the light.
The fifteenth container is situated in the middle of the eyebrows. Due to stabilized the gaze therein, ten million rays of light sparkle.\textsuperscript{\coro{[\lowroman{45}]}}
[If the gaze] is held at the tip of the finger without wavering, this is the eye support, the sixteenth. Through that practice, some energy arises in the middle of the earth. The entire light of it becomes the object of vision. Through its perception, a person becomes all-knowing.\textsuperscript{\coro{[\lowroman{50}]}}
      \end{tlate}
  \end{translation}
\end{alignment}


\chapter{Bibliography}
 \label{sec:bibli}
   \clearpage
\newpage 
\thispagestyle{empty}
\quad  \addtocounter{page}{-1}

\printbibliography[heading=subbibintoc, title=Consulted Manuskripts, keyword=codex]

\printbibliography[heading=subbibintoc, title=Printed Editions, keyword=printsource]

\printbibliography[heading=subbibintoc, title=Secondary Literature, keyword=seclit]

\printbibliography[heading=subbibintoc, title=Online Sources, keyword=onlinesource]

\end{document}



\begin{ekdosis}
  \ekddiv{type=ed}
   \centerline{\textrm{\small{[Aṣṭāṅgayoga]}}}
      \bigskip
      \begin{prose}
        \noindent
%----------------------------
%Note: Rāmacandra does not adopt the yāmas and niyāmas from the Yogasvarodaya! 
%----------------------------
%idānīm aṣṭāṃgayoga----vicāraḥ kathyate/  yamaniyamāsanaprāṇāyāmapratyāhāradhyānadhāraṇāsamādhir iti/  eteṣāṃ lakṣaṇāni kathyante/     \E
%idānīm aṣṭāṃgayogasya vicāraḥ kathyate   yamaniyamāsanaprāṇāyāmapratyāhāradhyānadhāraṇāsamādhir iti   eteṣāṃ lakṣaṇāni kathyaṃte  \P
%idānīm aṣṭāṃgayogasya vicāraḥ kathyate/  yamaniyamāsanaprāṇāyāmapratyāhāradhāraṇādhyānasamādhir iti/  eteṣāṃ lakṣaṇāni kathyaṃte/ \B
%idānīm aṣṭāṃgayogasya vicāraḥ kathyate/  yamaniyamāsanaprāṇāyāmapratyāhāradhāraṇādhyānasamādhir iti/  eteṣāṃ lakṣaṇāni kathyaṃte/ \L
%idānīm aṣṭāṃgayogasya vicāraḥ kathyate// yamaniyamāsanaprāṇāyāmapratyāhāradhyānadhāraṇāsamādhiyaḥ     eteṣāṃ lakṣaṇāni kathyaṃte/   \N1
%idānīm aṣṭāṃgayogasya vicāraḥ kathyate// yamaniyamāsanaprāṇāyāmapratyāhāradhyānadhāraṇāsamādhi//      eteṣāṃ lakṣaṇāni kathyaṃte//   \D
%idānīṃ aṣṭāṃgayogasya vicāraḥ kathyate// yamaniyamāsanaprāṇāyāmapratyāhāradhyānadhāraṇāsamādhiyaḥ     eteṣāṃ lakṣaṇāni kathyaṃte/   \N2
%idānīṃ aṣṭāṅgayogasya vicāraḥ kathyate// yamaniyamāsanaprāṇāyāmapratyāhāradhyānadhāraṇāsamādhi        eteṣāṃ lakṣaṇāni kathyate   \U1
%idānīṃ aṣṭāṅgayogasya vicāra  kathyate// yamaniyamāsanaprāṇāyāmapratyāhāradhyānadhāraṇāsamādhir iti// eteṣāṃ lakṣaṇāni kathyaṃte//   \U2
%-----------------------------
%Now the procedure of the eightfold yoga (\textit{aṣṭāṅgayoga})is explained: "Yama, niyama, āsana, prāṇāyāma, pratyāhāra, dhyāna, dhāraṇā and samādhi." Their characteristics will be explained.   
%----------------------------
\note[type=source, labelb=247, lem={aṣṭāṃga°}]{SSP:yamaniyamāsanaprāṇāyāmapratyāhāradhāraṇādhyānasamādhayoḥ 'ṣṭāṅgāni|}
\note[type=source, labelb=248, lem={aṣṭāṃga°}]{Ysv (PT=YK): idānīṃ yogamaṣṭāṅgaṃ śṛṇu lakṣaṇasaṃyutam | yamaś ca niyamaś caiva cāsanaṃ prāṇasaṃyamaḥ | pratyāhāro dhāraṇā ca samādhiś ca viśeṣataḥ | aṣṭāṅgayoga ebhis tu caiteṣāṃ lakṣaṇaṃ śṛṇu |}
\app{\lem[wit={ceteri},alt={idānīm}]{idānī\skp{m-a}}
  \rdg[wit={N2,U1,U2}]{idānīṃ}
}\app{\lem[wit={ceteri},alt={aṣṭāṅgayogasya}]{\skm{m-a}ṣṭāṅgayogasya}
  \rdg[wit={E}]{aṣṭāṃgayoga°}}
\app{\lem[wit={ceteri}]{vicāraḥ}
  \rdg[wit={U2}]{vicāra}}
kathyate/
yamaniyamāsanaprāṇāyāmapratyāhāra\app{\lem[wit={ceteri},alt={°dhyānadhāraṇāsamādhir iti}]{dhyānadhāraṇāsamādhir\skp{-}iti}
  \rdg[wit={B,L}]{dhāraṇādhyānasamādhir iti}
  \rdg[wit={N1,N2}]{dhyānadhāraṇāsamādhiyaḥ}
  \rdg[wit={D,U1}]{dhyānadhāraṇāsamādhi}}
eteṣāṃ lakṣaṇāni
\app{\lem[wit={ceteri}]{kathyante}
  \rdg[wit={U1}]{kathyate}}/
%----------------------------
%śāntiḥ/ ṣaṇṇām  indriyāṇāṃ jayaḥ/ svalpāhāraḥ/            nidrājayaḥ/      śītoṣṇajayaḥ/               ete yamāḥ/ \E
%śāṃtiḥ  ṣaṇāṃ   iṃdriyāṇāṃ jayaḥ       ahāraḥ svalpaḥ     nidrājayaḥ       śaityajayaḥ   uṣṇa?jayaḥ    ete yamāniyamāḥ ...\P
%śāntiḥ  ṣaṇāṃ   iṃdriṇāṃ   jayaḥ//     ahāraḥ svalpaḥ     nidrāyā jayaḥ//  śaityajayaḥ/  uṣṇājayaḥ// ya te yamaḥ// \B
%śāntiḥ  ṣaṇṇāṃ  iṃdriyāṇāṃ jayaḥ//     ahāraḥ// svalpaḥ// nidrāyāḥ jayaḥ/  śaityajayaḥ   uṣṇajayaḥ   ya te yamaḥ... \L
%śānti---ṣaṇṇāṃ  indriyāṇāṃ jayaḥ/      svalpāḥ            nidrājayaḥ/      śītyajayaḥ/   uṣṇajayaḥ/    ete yamāḥ/ \N1
%śāṃti---ṣaṇṇāṃ  indriyāṇāṃ jayaḥ//     āhāraḥ svalpāḥ     nidrājayaḥ//     śaityajayaḥ// uṣṇajayaḥ/    ete yamāḥ \D
%śānti---ṣaṇṇāṃ  indriyāṇāṃ jayaḥ/      ahāraḥ svalpāḥ     nidrājayaḥ/      śaityajayaḥ   uṣṇajayaḥ/    ete yamāḥ/ \N2
%śāntiḥ  ṣaṇṇām  iṃdriyāṇāṃ jayaḥ       āhāraḥ sajayaḥ     nidrājayaḥ       śaityajayaḥ   auṣṇājayaḥ    ete yamāḥ \U1
%śānti---śaṇa    iṃdriyāṇāṃ jayaḥ//     āhāraḥ svalpaḥ//   nidrāyāḥ jayaḥ// śaityajayaḥ// uṣṇājayaḥ//   ete yamāḥ// \U2 %%%417.jpg 
%----------------------------
%These are the Yāmas: Peace, conquer of the six senses, little food, conquer of sleep, conquer of cold and heat.
%----------------------------
\note[type=source, labelb=249, lem={ete yamāḥ}]{SSP:yama iti upaśamaḥ sarvendriyajayaḥ āhāranidrāśītavātātapajayaś caivaṃ śanaiḥ śanaisādhayet ||2.32||}
\note[type=source, labelb=250, lem={ete yamāḥ}]{Ysv (PT): śāntiḥ santoṣa āhāro nidrālpā [nidrālpaṃ (YK)] manaso damaḥ | śūnyāntaḥ karaṇañceti [karaṇaś ceti (YK)] yamā iti prakīrttitāḥ |}
\app{\lem[wit={ceteri}]{śāntiḥ}
  \rdg[wit={D,N1,N2,U2}]{śānti°}}\dd{}
\app{\lem[wit={E,U1},alt={ṣaṇṇām}]{ṣaṇṇā\skp{m-i}}
  \rdg[wit={D,L,N1,N2}]{ṣaṇṇāṃ}
  \rdg[wit={B,P}]{ṣaṇāṃ}
  \rdg[wit={U2}]{śaṇa}
}\app{\lem[wit={ceteri},alt={indriyāṇāṃ}]{\skm{m-i}ndriyāṇāṃ}
  \rdg[wit={B}]{iṃdriṇāṃ}}
jayaḥ\dd{}
\app{\lem[wit={U2}]{āhāraḥ svalpaḥ}
  \rdg[wit={E}]{svalpāhāraḥ}
  \rdg[wit={B,P}]{ahāraḥ svalpaḥ}
  \rdg[wit={L}]{ahāraḥ|| svalpaḥ ||}
  \rdg[wit={N1}]{svalpāḥ}
  \rdg[wit={N2}]{ahāraḥ svalpāḥ}
  \rdg[wit={D}]{āhāraḥ svalpāḥ}
  \rdg[wit={U1}]{āhāraḥ sajayaḥ}}\dd{}
\app{\lem[wit={ceteri}]{nidrājayaḥ}
  \rdg[wit={B}]{nidrāyā jayaḥ}
  \rdg[wit={L,U2}]{nidrāyāḥ jayaḥ}}\dd{}
\app{\lem[wit={ceteri}]{śaityajayaḥ}
  \rdg[wit={N1}]{śītyajayaḥ}
  \rdg[wit={E}]{śītoṣṇajayaḥ}}\dd{}
\app{\lem[wit={ceteri}]{uṣṇajayaḥ}
  \rdg[wit={B,U2}]{uṣṇājayaḥ}
  \rdg[wit={U1}]{auṣṇājayaḥ}
  \rdg[wit={E}]{\om}}\dd{}
\app{\lem[wit={ceteri}]{ete}
  \rdg[wit={B,L}]{ya te}}
\app{\lem[wit={ceteri}]{yamāḥ}
  \rdg[wit={P}]{yamāniyamāḥ}
  \rdg[wit={B,L}]{yamaḥ}}\dd{}
%----------------------------
%niyamāḥ   khalu       cāpalabhāvān nivārya  sthairye  sthāpyate/  ekāṃte sevanam/ prāṇimātre samābuddhiḥ/ audāsīnyaṃ   kasyāpi vastuna    icchā na karttavyā    yathā lābhasaṃtoṣaḥ/   \E
%          khalu       cāpalābhāvān nirvārya sthairye  sthāpyate   ekāṃta sevānaṃ  prāṇimātre samābuddhiḥ   udāsīnyaṃ   kasyāpi vastuna    icchā na kartavyā     yathā lābhasaṃtoṣaḥ    \P %%%7656.jpg
%          khalu       cāpalabhāvān nirvārya           sthāpyate//           ekāṃta sevānāṃ  prāṇimātre samābuddhiḥ   udāsīnyaṃ   kasyāpi vastunaḥ// icchā na kartavyā     yathā lābhasaṃtoṣaḥ/   \B
%          ḱhalu       cāpalabhāvān nirvārya           sthāpyate//           ekāṃtasevānāṃ   prāṇimātre samābuddhiḥ/  udāsīnyaṃ   kasyāpi vastunaḥ/  icchā na kartavyā     yathā lābhasaṃtoṣaḥ    \L
%niyamaḥ   khalu       capalabhāvān nivārya  sthairye  sthāpyate/  ekāṃte sevanam/ prāṇimātre samābuddhiḥ/  udāsīnya/   kasyāpi vastunaḥ   icchā na karttavyā//  yathā lābhasaṃtoṣaḥ/   \N1
%niyamaḥ   khalu manaḥ capalabhāvān nivārye            sthāpyate//           ekāṃtasevanaṃ// prāṇimātre samābuddhiḥ// udāsīnya//  kasyāpi vastunaḥ   icchā na karttavyā//  yathā lābhasaṃtoṣaḥ//  \D
%niyamaḥ   khalū manaḥ capalabhāvān nivārya  sthairye  sthāpyate   ekāṃtasevanam/  prāṇimātre samābuddhiḥ   udāsīnya    kasyāpi vastunaḥ   icchā na karttavyā/   yathā lābhasaṃtoṣaḥ    \N2
%niyamaḥ   khalū manaḥ capalabhāvān nivāraya sthairye  sthāpyate   ekāṃtasevanaṃ   prāṇimātre samābuddhi    udāsīnyāṃ   kasyāpi vastunaḥ   icchā na karttavyaṃ   yathā lābhasaṃtoṣaḥ    \U1
%niyamaḥ// khalū       cāpalābhāvān nivārya            sthāpyate// ekāṃtasevanaṃ// prāṇimātre samābuddhi//  udāsīnyaṃ// kasyāpi vastuna    icchā na karttavyaṃ// yathā lābhasaṃtoṣaḥ//  \U2
%----------------------------
%parameśvaranāma na vismaraṇīyam/  manomadhye      dainyaṃ    karttavyam/ iti niyamāḥ// \E
%parameśvaranāma na vismaraṇīyaṃ   manomadhye      dainyaṃ    kartavyaṃ   iti niyamāḥ\P %%%7656.jpg
%parameśvaranāma na vismaraṇīyaṃ   manomadhye      dainyaṃ    kartavyaṃ// iti niyamaḥ// \B
%parameśvaranāma na vismaraṇīyaṃ   manomadhye      dainyaṃ    karttavyaṃ/ iti niyamaḥ// \L
%parameśvaranāma----vismaraṇīyam/  manomadhye      dainyaṃ na karttavyam/ //[S.11] \N1
%parameśvaranāma----vismaraṇīyaṃ// manomadhye      dainyaṃ na karttavyaṃ// \D
%parameśvaranāma----vismanīyam/    manomadhye      dainyaṃ na karttavyam// // \N2 \em zu vismāra
%parameśvaraḥ nāma na vismaraṇīyaṃ mano            dainyaṃ na karttavyaṃ  \U1
%parameśvaraḥ nāma na vismaraṇaṃ// yaṃ mano madhye dainyaṃ na karttavyaṃ iti niyamaḥ//  \U2
%----------------------------
%Niyamās are truly: Keeping the mind from the state of unsteadiness [and] ground it in calmness, retreating to a lonely place, refraining from contact to animals, unchanging intellect, keeping equanimous one shall not crave for things, as well as being contend with what is given, never forgetting the name of the highest lord, one shall not bring the mind into depression. 
%----------------------------
\note[type=source, labelb=251, lem={niyamāḥ}]{SSP:niyama iti manovṛttīnāṃ niyamanam iti ekāntavāso niḥsaṃgataudāsīnyaṃ yathāprāptisaṃtuṣṭir vairasyaṃ gurucaraṇāvarūḍhatvam iti niyamalakṣaṇam ||2.33||}
\note[type=source, labelb=252, lem={niyamāḥ}]{Ysv (YK): tyaktvā dūre tu cāpalyaṃ [cāpalyantu dūre tyaktvā (Ysv)] manaḥ sthairyyaṃ vidhāya ca ||31|| ekatra melanaṃ nityaṃ prāṇāmātre na sāmabhiḥ [sā matiḥ (PT)] | sadodāsīnabhāvas tu sarvatrecchāvivarjitaḥ [°vivarjanam (PT)] ||32|| yathālābhena santuṣṭaḥ parameśvaramānasaḥ | mānadānaparityāga ete tu niyamā iti || 33||}
\app{\lem[wit={E}]{niyamāḥ}
  \rdg[wit={D,N1,N2,U1,U2}]{niyamaḥ}
  \rdg[wit={B,P,L}]{\om}}\dd{}
\app{\lem[wit={ceteri}]{khalu}
  \rdg[wit={N1,N2,U2}]{khalū}} 
\app{\lem[wit={D,N2,U1}]{manaḥ}
  \rdg[wit={ceteri}]{\om}}
\app{\lem[wit={B,E,L},alt={cāpala°}]{cāpala}
  \rdg[wit={P,U2}]{cāpalā°}
  \rdg[wit={D,N1,N2,U1}]{capala°}
}bhāvā\skp{n-ni}
\app{\lem[wit={ceteri},alt={nivārya}]{\skm{n-ni}vārya}
  \rdg[wit={D}]{nivārye}
  \rdg[wit={B,L,P}]{nirvārya}
  \rdg[wit={U1}]{nivāraya}}
\app{\lem[wit={ceteri}]{sthairye}
  \rdg[wit={B,L,D,U2}]{\om}}
sthāpyate\dd{}
%----------------------------
%āsanalakṣaṇaṃ     bahuṣu grantheṣu nirūpitam     asti    tenātra na nirūpyate/ \E
%āsanalakṣaṇaṃ     bahuṣu graṃtheṣu nirūpitam     asti    tenātra na nirūpyate \P
%āsanaṃ lakṣaṇāṃ   bahūgraṃtheṣu    nirūpyam      asti    tenātra    nirūpyate/       \B
%āsanalakṣaṇāṃ     bahūgraṃtheṣu    nirūpyam      asti    tenātra    nirūpyate//     \L
%āsanasya lakṣaṇaṃ bahūgraṃthe      nirūpitam/    ataḥ    atrāyaṃ    nirūpyate/   \N1
%āsanasya lakṣaṇaṃ bahūgraṃthe      nirūpitaṃ//   ataḥ    atratyaṃ   nirūpyate// \D %%%p. 13 verso
%āsanasya lakṣaṇaṃ bahugraṃthe      nirūpitam//   ataḥ    atrāyaṃ    nirūpyate/  \N2
%āsanasya lakṣaṇaṃ bahugraṃthe      nirūpitam tan attaḥ   atra    na nirūpyate  \U1
%āsanalakṣaṇaṃ tu  bahugraṃtheṣu    nirūpitam     asti//  tenātra    nirūpyate// \U2
%----------------------------
%The characteristic of posture has been discussed in many works. Because of that it will not be discussed here.  
%----------------------------
\note[type=source, labelb=253, lem={āsanasya}]{SSP: āsanam iti svasvarūpe samāsannatā | svastikāsanaṃ padmāsanaṃ siddhāsanam eteṣāṃ madhye yatheṣṭam ekaṃ vidhāya sāvadhānena sthātavyam ity āsanalakṣaṇam ||2.34||}
\note[type=source, labelb=254, lem={āsanasya}]{Ysv (YK): āsanāni ca tāvanti yāvanto jīvajantavaḥ |[om. YK]}
\app{\lem[wit={D,N1,N2,U1}]{āsanasya lakṣaṇaṃ}
  \rdg[wit={E,P,L}]{āsanalakṣaṇaṃ}
  \rdg[wit={U2}]{āsanalakṣaṇaṃ tu}
  \rdg[wit={B}]{āsanaṃ lakṣaṇāṃ}}
\app{\lem[wit={B,L,U2}]{bahūgrantheṣu}
  \rdg[wit={E,P}]{bahuṣu graṃtheṣu}
  \rdg[wit={D,N1,N2,U1}]{bahūgraṃthe}}
\app{\lem[wit={E,P,U2},alt={nirūpitam}]{nirūpita\skp{m-a}}
  \rdg[wit={D}]{nirūpitaṃ ||}
  \rdg[wit={N1,N2}]{nirūpitam |}
  \rdg[wit={B,L}]{nirūpyam}
  \rdg[wit={U1}]{nirūpitam tan}}
\app{\lem[wit={B,E,L,P,U2},alt={asti}]{\skm{m-a}sti}
  \rdg[wit={D,N1,N2,U1}]{ataḥ}}
\app{\lem[wit={U2}]{/}
  \rdg[wit={ceteri}]{\om}} 
\app{\lem[wit={B,E,L,P,U2}]{tenātra}
  \rdg[wit={N1,N2}]{atrāyaṃ}
  \rdg[wit={D}]{atratyaṃ}
  \rdg[wit={U1}]{atra}}
\app{\lem[wit={E,P,U1}]{na}
  \rdg[wit={ceteri}]{\om}}
nirūpyate/
%---------------------------
%prāṇāyāmas tu sukumāreṇa        sādhituṃ na śakyate   atas tasya nāmamātraṃ kathyate/ \E
%prāṇāyāmas tu sukumāreṇa        sādhituṃ na śakyate   atas tasya nāmamātraṃ kathyate  \P
%prāṇāyāmas tu kumāreṇa          sādhituṃ na śakyate// ataḥ       nāma       kathyate/ \B
%prāṇāyāmas tu kumāreṇa          sādhituṃ na śakyate// ataḥ       nāma       kathyate// \L
%prāṇāyāmas tu kūmāreṇa puruṣeṇa sādhituṃ na śakyate/  ataḥ tasya nāmamātraṃ kathitaṃ/ \N1
%prāṇāyāmas tu kūmāreṇa puruṣeṇa sādhituṃ na śakyate// ataḥ tasya nāmamātre  kathitaṃ// \D
%prāṇāyāmas tu kūmāreṇa puruṣeṇa sādhituṃ na śakyate// ata  tasya nāmamātre  kathitaṃ/ \N2
%prāṇāyāmas tu kūmāreṇa puruṣeṇa sādhituṃ na śakyate   atas tasya nāmamātre  kathitaṃ \U1
%prāṇāyāmas tu kūmāreṇa          sādhituṃ na śakyate// atā  tasya nāmamātraṃ  kathyate// \U2
%----------------------------
%Breath-control can't be practiced by young persons. That's why it is just mentioned by name.
%Practicing breath-control can't be done by a young person. 
%----------------------------
\note[type=source, labelb=255, lem={prāṇāyāmas}]{SSP: prāṇāyāma iti prāṇasya sthiratā recakapūrakakumbhakasaṃghaṭṭakaraṇāni catvāri prāṇāyāmalakaṇam ||2.35||}
\note[type=source, labelb=256, lem={prāṇāyāmas}]{Ysv (YK): prāṇāyāmas tridhā ceti bahudhā prathamaṃ śṛṇu | āsane prāṇasaṃyāme na śaktāḥ sukumārakāḥ | mahāpuṇyaprabhāveṇa śakyate tu mahātmanā | iḍāṃ śaśiprabhāṃ dhyātvā mandendunā [yathāśakti (YK)] tu pūrayet [tu kumbhayet (YK)] | pūrayitvā yathāśakti dhyānayogī tu kumbhayet [sentence om. (YK)] | mahājyotir mano [mayo (YK)] bhūtvā vāyuḥ [vāyu° (YK)] pūrṇakalevaraḥ | śaktitrāsantu santrāsya recayed vāyum arhitaḥ | piṅgalām arkavarṇān [°varṇaṃ (YK)] tu tyajed dhyātvā śanaiḥ śanaiḥ | ayaṃ pataṅgaḥ kāṣṭhāgnipratyāsena punaḥ punaḥ | kṛtvā kalevaraṃ śuddhaṃ kuryād yatnair mahātmanā | mano nivārya saṃsāre viṣayakārye [viṣayeṣu (YK)] tathaiva ca | manovikārabhavañ caiva [manovikārān sarvāś ca (YK)] tyaktvā śūnyamayo bhavet |}
%----------------------------
prāṇāyāmas-tu
\app{\lem[wit={E,P}]{sukumāreṇa}
  \rdg[wit={B,L,U2}]{kumāreṇa}
  \rdg[wit={D,N1,N2,U1}]{kūmāreṇa puruṣeṇa}}
sādhituṃ na śakyate/
\end{prose}
\end{ekdosis}
\ekdpb*{}
%%%%%%%%%%%%%%%%%%%%%%%%%%%%%%%%%%%%%%%%%%
%%%%%%%%%%%%%%%%%%%%%%%%%%%%%%%%%%%%%%%%%%
%%%%%%%%PAGEBREAK%%%%%%%PAGEBREAK%%%%%%%%%
%%%%%%%%%%%%%%%%%%%%%%%%%%%%%%%%%%%%%%%%%%
%%%%%%%%%%%%%%%%PAGEBREAK%%%%%%%%%%%%%%%%%
%%%%%%%%%%%%%%%%%%%%%%%%%%%%%%%%%%%%%%%%%%
%%%%%%%%PAGEBREAK%%%%%%%PAGEBREAK%%%%%%%%%
%%%%%%%%%%%%%%%%%%%%%%%%%%%%%%%%%%%%%%%%%%
%%%%%%%%%%%%%%%%%%%%%%%%%%%%%%%%%%%%%%%%%%
%%%%%%%%%%%%%%%%%%%%%%%%%%%%%%%%%%%%%%%%%%
%%%%%%%%%%%%%%%%%%%%%%%%%%%%%%%%%%%%%%%%%%
%%%%%%%%PAGEBREAK%%%%%%%PAGEBREAK%%%%%%%%%
%%%%%%%%%%%%%%%%%%%%%%%%%%%%%%%%%%%%%%%%%%
%%%%%%%%%%%%%%%%PAGEBREAK%%%%%%%%%%%%%%%%%
%%%%%%%%%%%%%%%%%%%%%%%%%%%%%%%%%%%%%%%%%%
%%%%%%%%PAGEBREAK%%%%%%%PAGEBREAK%%%%%%%%%
%%%%%%%%%%%%%%%%%%%%%%%%%%%%%%%%%%%%%%%%%%
%%%%%%%%%%%%%%%%%%%%%%%%%%%%%%%%%%%%%%%%%%
%%%%%%%%%%%%%%%%%%%%%%%%%%%%%%%%%%%%%%%%%%
%%%%%%%%%%%%%%%%%%%%%%%%%%%%%%%%%%%%%%%%%%
%%%%%%%%PAGEBREAK%%%%%%%PAGEBREAK%%%%%%%%%
%%%%%%%%%%%%%%%%%%%%%%%%%%%%%%%%%%%%%%%%%%
%%%%%%%%%%%%%%%%PAGEBREAK%%%%%%%%%%%%%%%%%
%%%%%%%%%%%%%%%%%%%%%%%%%%%%%%%%%%%%%%%%%%
%%%%%%%%PAGEBREAK%%%%%%%PAGEBREAK%%%%%%%%%
%%%%%%%%%%%%%%%%%%%%%%%%%%%%%%%%%%%%%%%%%%
\begin{ekdosis}
  \begin{prose}
    \noindent
\app{\lem[wit={E,P,U1},alt={atas tasya}]{atas\skp{-}tasya}
  \rdg[wit={D,N1}]{ataḥ tasya}
  \rdg[wit={N2}]{ata tasya}
  \rdg[wit={U2}]{atā tasya}
  \rdg[wit={B,L}]{ataḥ}}
\app{\lem[wit={E,P,N1,U2}]{nāmamātraṃ}
  \rdg[wit={D,N2,U1}]{nāmamātre}
  \rdg[wit={B,L}]{nāma}}
\app{\lem[wit={ceteri}]{kathyate}
  \rdg[wit={D,N1,N2,U1}]{kathitaṃ}}/
%----------------------------
%prāṇāyāmastridhā ceti bahudhā prathamaṃ śrṛṇu।
%āsane prāṇasaṃyāme na śaktāḥ sukumārakāḥ।। 2।।
%mahāpuṇyaprabhāveṇa śakyate tu mahātmamā।
%iḍāṃ śaśiprabhāṃ dhyātvā yathāśakti tu kumbhayet।। 3।।
%mahājyotirmayo bhūtvā vāyupūrṇakalevaraḥ।
%śaktitrāsantu saṃtrāsya recayedvāyumarhitaḥ।। 4।।
%piṅgalāmarkavarṇaṃ tu tyajed dhyātvā śanaiḥ śanaiḥ।
%ayaṃ pataṅgakāṣṭhāgnipratyāsena punaḥ punaḥ।। 5।।
%kṛtvā kalevaraṃ śuddhaṃ kuryād yatnairmahātmanā।
%mano nivārya saṃsāre viṣayeṣu tathaiva ca।। 6।।
%manovikārān sarvāśca tyaktvā śūnyamayo bhavet।
%pratyāhāro bhavatyeṣu sarvanindācamatkṛtaḥ।। 7।।
%dhyānaṃ ca dvividhaṃ proktaṃ sthūlasūkṣmavibhedataḥ।
%sthūlaṃ mantramayaṃ viddhi sūkṣmantu mantravarjitam।। 8।।
%ityetatkathitaṃ sarvaṃ yogasaṅketamuttamam।
%adhunā cāṣṭakumbhasya lakṣaṇaṃ śrṛṇu kathyate।। 9।।
%śītkāraṃ sūryabhedaṃ ca uhyāyī śītalī tathā।
%bhastrikā bhrāmarī mūrcchā kevalī cāṣṭa kumbhakāḥ।। 10।।
%----------------------------
%pratyāhāraḥ pratyato   manaḥ saṃsārān nivartyātmani   sthāpyate// manomadhye ye vikārā  utpadyante/  tepi nivāraṇīyāḥ/  anekacamatkāriṇī         buddhir utpadyate/  sāṃgopāṃgaṃ  \E XX! this one?[P.47]
%pratyāhāraḥ kathyate   manaḥ saṃsārān nivṛtyātmanī    sthāpyate   manomadhye ye vikāraḥ utpadyaṃte   tepi nivāraṇīyāḥ   anekacamatkāriṇi         buddhir utpadyataraṃ  \P
%pratyāhāraḥ kathyate// manaḥ saṃsārān nivṛtyātmanī    sthāpyate// manomadhye ye vikārā  utpadyaṃte   tepi nivāraṇīyā    anekacamatkāriṇī         buddhir utpadyate/  sāgopyā// \B 
%pratyāhāraḥ kathyate   manaḥ saṃsārān nivṛttyātmanī   sthāpyate// manomadhye ye vikārā  utpadyaṃte   tepi nivāraṇīyā    anekacamatkāriṇi         buddhir utpadyate   sāgopyā//  \L %%%%0023.jpg
%pratyāhāraḥ kathyate// manaḥ saṃsārān nivṛtya ātmani  sthāpyate/  manomadhye ye vikārā  utpadyante/  tepi nivāraṇīyāḥ/  anekacamatkārakarakāraṇī buddhi  utpadyate   sāṃgopyāḥ/ \N1
%pratyāhāraḥ kathyate// manaḥ saṃsārān nivṛtya ātmani  sthāpyate// manomadhye ye vikārāḥ utpadyaṃte// tepi nivāraṇīyāḥ// anekacamatkārakāraṇī     buddhi  utpadyate// sāṃgopyāḥ// \D
%pratyāhāraḥ kathyate// manaḥ saṃsārān nivṛtya ātmani                                                        vāraṇīyāḥ// anekacamatkārakarakāraṇī buddhi  utpadyate   sāgopyāḥ/  \N2
%pratyāhāraḥ kathyate   manaḥ saṃsārān nivṛtyātmanī    sthāpyate   manomadhye ye vikārā  utpadyaṃte   tepi nivāraṇīyaḥ   anekacamatkāriṇī         buddhir utpadyate   sāgaupyā \U1 %%%286.jpg
%pratyāhāraḥ kathyate// manaḥ saṃsārān nivṛtyātmanī    sthāpyate// manomadhye ye vikārā  utpadyaṃte   tepi nivāraṇīyaḥ// anekacamatkāriṇī         buddhir utpadyate// sāgopyā// \U2
%-----------------------------
%Pratyāhāra [however] is taught. The mind is supposed to be turn away from the cyclic existence and caused to abide in the self. Changes within the mind arise, but they are kept off. The manyfold admirations the intellect generates are well hidden.    
%----------------------------
\note[type=source, labelb=257, lem={pratyāhāraḥ}]{SSP: pratyāhāram iti caitanyataraṅgānāṃ pratyāharaṇaṃ yathā nānāvikāragrasanotpannavikārasyāpi nivṛttiḥ nirbhātīti pratyāhāralakṣaṇam ||2.36||}
\note[type=source, labelb=258, lem={pratyāhāraḥ}]{Ysv (YK): ayaṃ pataṅgakāṣṭhāgnipratyāsena punaḥ punaḥ ||5|| kṛtvā kalevaraṃ śuddhaṃ kuryād yatnair mahātmanā | mano nivārya saṃsāre viṣayeṣu tathaiva ca ||6|| manovikārān sarvāś ca tyaktvā śūnyamayo bhavet | pratyāhāro bhavaty eṣu sarvanindācamatkṛtaḥ ||7||}
pratyāhāraḥ
\app{\lem[wit={ceteri}]{kathyate}
  \rdg[wit={E}]{pratyato}}/
manaḥ saṃsārā\skp{n-ni}\app{\lem[type=emendation, resp=egoscr, alt={nivṛtyātmani}]{\skm{n-ni}vṛtyātmani}
  \rdg[wit={B,L,P,U1,U2}]{\korr nivṛtyātmanī}
  \rdg[wit={E}]{nivartyātmani}
  \rdg[wit={D,N1,N2}]{nivṛtya ātmani}}
\app{\lem[wit={ceteri}]{sthāpyate}
  \rdg[wit={N2}]{\om}}/
manomadhye ye
\app{\lem[wit={ceteri}]{vikārā}
  \rdg[wit={P}]{vikāraḥ}
  \rdg[wit={D}]{vikārāḥ}
  \rdg[wit={N2}]{\om}}
\app{\lem[wit={ceteri}]{utpadyante}
  \rdg[wit={N2}]{\om}}/
 anekacama\skp{t-kā}\app{\lem[type=emendation, resp=egoscr, alt={°kārīṇi}]{\skm{t-kā}rīṇi}
   \rdg[wit={B,E,L,P,U1,U2}]{\korr kāriṇī}
   \rdg[wit={N1,N2}]{kārakarakāraṇī}
   \rdg[wit={D}]{kārakāraṇī}}
 \app{\lem[wit={ceteri},alt={buddhir}]{buddhi\skp{r-ut}}
   \rdg[wit={D,N1,N2}]{buddhi}
 }\app{\lem[wit={ceteri},alt={utpadyate}]{\skm{r-ut}padyate}
   \rdg[wit={E,B,D,U2}]{utpadyate |}
   \rdg[wit={P}]{utpadyataraṃ}}
 \app{\lem[type=emendation, resp=egoscr]{saṃgopyāḥ}
   \rdg[wit={D,N1}]{\korr sāṃgopyāḥ}
   \rdg[wit={N2}]{sāgopyāḥ}
   \rdg[wit={B,L,U2}]{sāgopyā}
   \rdg[wit={U1}]{sāgaupyā}
   \rdg[wit={E}]{sāṃgopāṃgaṃ}}/ 
%----------------------------
% dhyānaṃ ca bahutaraṃ prāg uktam/ tenātra       nocyate// \E XX! this one?
%                      prāg uktam  tenātra       nocyate  \P
% dhyānaṃ ca bahutaraṃ prāg uktam  tenātra       nocyate// \B 
% dhyānaṃ ca bahutaraṃ prāg uktam  tenātra       nocyate// \L %%%%0023.jpg
% dhyānaṃ ca bahutaraṃ      uktam  tena atra     nocyate/ \N1
% dhyānaṃ ca bahutaraṃ      uktaṃ  tena atra     nocyate// \D
% dhyānaṃ ca bahuttaraṃ     uktam  tenātra       nocyate// \N2
% dhyānaṃ    bahutaraṃ      uktaṃ  tena atra  na ucyate \U1 %%%286.jpg
% dhyānaṃ    bahutaraṃ prāg uktaṃ  tenātra       nocyate// \U2
%-----------------------------
%Dhyāna has been taught many times before. Because of that is not discussed here.
%-----------------------------
\note[type=source, labelb=259, lem={dhyānaṃ}]{SSP: atha dhyānam iti || asti kaś cana paramādvaitasya bhāvaḥ sa eva ātmeti yathā yadyat sphurati tattat svarūpam eveti bhāvayet sarvabhūteṣu samadṛṣṭiś ceti dhyānalakṣaṇam ||2.38||}
\note[type=source, labelb=260, lem={dhyānaṃ}]{Ysv (YK): dhyānan tu dvividhaṃ proktaṃ sthūlasūkṣmavibhedataḥ | sthūlaṃ mantramayaṃ viddhi sūkṣmantu mantravarjjitam | samādhir niścalā buddhiḥ śvāsocchvāsādivarjitaḥ |}
\app{\lem[wit={ceteri}]{dhyānaṃ}
  \rdg[wit={P}]{\om}}
\app{\lem[wit={ceteri}]{ca}
  \rdg[wit={P,U1,U2}]{\om}}
\app{\lem[wit={ceteri}]{bahutaraṃ}
  \rdg[wit={P}]{\om}}
\app{\lem[wit={B,E,L,P,U2},alt={prāg}]{prā\skp{g-u}}
  \rdg[wit={D,N1,N2,U1}]{\om}
}\app{\lem[wit={D,U1,U2},alt={uktaṃ}]{\skm{g-u}ktaṃ}
  \rdg[wit={E}]{uktam |}
  \rdg[wit={ceteri}]{uktam}}
\app{\lem[wit={ceteri}]{tenātra}
  \rdg[wit={D,N1,U1}]{tena atra}}
\app{\lem[wit={ceteri}]{nocyate}
  \rdg[wit={U1}]{na ucyate}}\dd{}
\end{prose}
\end{ekdosis}
%%%%%%%%%%%%%%
%%%%%%%%%%%%%%
%%%%%%%%%%%%%%
%%%%%%%%%%%%%
%%%%%%%%%%%%%%% 
