%Ultimatives Tool zur Datierung:
%https://www.cc.kyoto-su.ac.jp/~yanom/pancanga/
%skp = ignored in edition
%skm = ignored in xml
\input{preamble.tex}
\FormatDiv{1}{\begin{center}\Large}{\end{center}}
\FormatDiv{2}{\begin{center}\small}{\end{center}}
\FormatDiv{3}{\bfseries}{.}
\title{Tattvayogabindu of Rāmacandra\\ A Critical Edition and Annotated Translation\\ and a Comparative Analysis of the \\Complex Early Modern Yoga Yaxonomies }
\date{\today}
\parindent=15pt

\begin{document}

\frontmatter
\thispagestyle{empty} % Verhindert Seitenzahl auf der Seite
\begin{center}

%\vspace{0.5in}

%\begin{otherlanguage}{iast}
%   \large\sanskritfont{Tattvayogabindu}\\
%\end{otherlanguage}

\vspace{0.25in}


\huge\textbf{\MakeUppercase{The Tattvayogabindu \\of Rāmacandra}}\\

\vspace{0.2in}

\Large  Critical Edition and Annotated Translation of an Early Modern Text on Rājayoga, with a Comparative Analysis of the Complex Yoga Taxonomies from the Same Period\\ 

\vspace{0.45in}

\thispagestyle{empty}
\end{center}
%\newpage
%\thispagestyle{empty}
%\mbox{}
%\newpage

\newpage

  \thispagestyle{empty}
  \begin{figure}[p]
    \centering
    \includegraphics[width=0.25\textwidth]{pics/purna.jpg}
  \end{figure}
  
\newpage

\begin{landscape}
\thispagestyle{empty}
  \begin{figure}[p]
	\centering
  \includegraphics[width=1.5\textwidth]{pics/folio1.jpg}
	\caption{Folio 1v of Ms. \getsiglum{N1}.}
	 \phantomsection\label{fig_folio1}
\end{figure}
\end{landscape}

\cleardoublepage
\tableofcontents
\thispagestyle{empty}
\newpage 
\listoffigures
\thispagestyle{empty}
\newpage
\listoftables
\thispagestyle{empty}
\newpage

\mainmatter
\pagestyle{defaultstyle}
\counterwithout{footnote}{chapter}
\counterwithout{figure}{chapter}
\counterwithout{table}{chapter}
\renewcommand{\thetable}{\arabic{table}}
%%%tables 
\setsecnumdepth{section}
\maxsecnumdepth{subsubsection}
\newpage
\chapter{Introduction}
\cleardoublepage




\chapter[Critical Edition \& Annotated Translation of the \emph{Tattvayogabindu}]{The \emph{Tattvayogabindu} of Rāmacandra \\ \huge  
  Critical Edition \& Annotated Translation}
\pagestyle{chapter2style}
\newpage
\begin{alignment}[
  texts=edition[class="edition"];
  translation[class="translation"],
  ]
  \begin{edition}
    \ekddiv{
      head={[\uproman{30}. \textbf{ādhāracakrasya bhedāḥ}]},
      type=section,
      depth=2, 
      n=XXX
    }
    \xmlhead[h30]{[XXX. ādhāracakrasya bhedāḥ]}
    \phantomsection
    \addcontentsline{toc}{section}{XXX. ādhāracakrasya bhedāḥ}
 \begin{prose}[p30_01]
   \noindent
%-----------------------------
%idānīm ādhāracakrasya bhedāḥ kathyanta/   \E
%idānīm ādhāracakrasya bhedaḥ kathyate     \P
%idānīm ādhāracakrasya bhedā  kathyaṃte/    \B DSCN7165.jpg Z.3
%idānīm ādhāracakrasya bhedā  kathyaṃte//   \L
%idānīm ādhāracakrasya bhedaḥ kathyate/    \N1
%idānīṃ ādhāracakrasya bhedaḥ kathyate//   \D
%idānīṃ ādhāracakrasya bhedaḥ kathyate//   \K1
%idānīṃ ādhāracakrasya bhedāḥ kathyate//   \J   
%idānī  ādhāracakrasya bhedaḥ kathyaṃte/   \N2
%idānīṃ ādhāracakrasya bhedāḥ kathyaṃte    \U1
%idānīṃ ādhāracakrasya bhedāḥ kathyaṃte // \U2
%-----------------------------
%Now the divisions of the totality of container [for concentration] are taught.
%-----------------------------
   \note[type=source, labelb=_75b, labele=_75e, nosep]{cf. YSv (PT, p. 839) = YK 2.15: ṣoḍaśādhārabhedan tu śṛṇu devi viśeṣataḥ |}
   \note[type=source, labelb=_75b, labele=_75e, nosep]{cf. SSP 2.10 (Ed. p. 32): atha ṣoḍaśādhārāḥ kathyante |}
   \note[type=testium, labelb=_75b, labele=_75e, nosep]{cf. \citetitle{hathasamketacandrikajodhpur} (MMPP 2244 f. 98r ll. 3-4): ity ādhārāḥ ṣodaśayaṃ athoktānāṃ ṣoḍaśādhārāṇāṃ kartavyatām āha |}
\app{\lem[wit={ceteri}, alt={idānīm}]{idānī\skp{m-ā}}\linelabel{_75b}
  \rdg[wit={N2}]{idānī}
}\skm{m-ā}dhāracakrasya
\app{\lem[wit={ceteri}]{bhedāḥ}
  \rdg[wit={B,L}]{bhedā}}
\app{\lem[wit={ceteri}]{kathyante}
  \rdg[wit={E}]{kathyanta}
  \rdg[wit={D,N1}]{kathyate}}/\linelabel{_75e}
%-----------------------------
%pādayor aṃguṣṭhe  tejaso  lakṣyakāraṇāt              dṛṣṭiḥ sthirā bhavati/ \E
%pādayor aṃguṣṭhe  tejaso  lakṣyakaraṇāt              dṛṣṭiḥ sthirā bhavati  \P
%pādayor aṃguṣṭhai tejasaṃ lakṣaṃ kartavyaṃ kāraṇāt// dṛṣṭiḥ sthirā bhavati/ \B
%pādayor aṃguṣṭhe  tejasaṃ lakṣaṃ karttavyaṃ kāraṇāt  dṛṣṭiḥ sthirā bhavatī/ \L
%pādayor aṃguṣṭhe  tejaso  lakṣyakāraṇāt              dṛṣṭisthirā   bhavati/ \N1
%pādayor aṃguṣṭhe  tejaso  lakṣyakāraṇāt              dṛṣṭiḥ sthirā bhavati \D
%pādayor aṃguṣṭhe  tejaso  lakṣyakāraṇāt//            dṛṣṭiḥ sthirā bhavati \K1
%pādayor aṃguṣṭhe  tejaso  lakṣyakāraṇāt              dṛṣṭiḥ sthirā bhavati// \J
%pādayor aṃguṣṭhe  tejaso  lakṣakāraṇāt               dṛṣṭisthirā   bhavati/ \N2
%pādayor aṃguṣṭhe  tejaso  lakṣyakāraṇāt              dṛṣṭisthirā   bhavati \U1
%pādayor aṃguṣṭhe  tejaso  lakṣyakāraṇāt              dṛṣṭisthirā   bhavati// \U2 %%%415.jpg
%-----------------------------
%From the execution of the fixation onto the light at the big toe of the feet stability of the gaze arises.
%-----------------------------
\note[type=source, labelb=_76b, labele=_76e, nosep]{cf. YSv (PT, p. 839): aṅguṣṭhapādayos tejaḥ salakṣasthiradṛṣṭimān | pādāṅguṣṭhe ya ādhāraḥ prathamo (\textit{prathamaṃ} YK 2.16) yogatattvataḥ |}
\note[type=source, labelb=_76b, labele=_76e, nosep]{cf. SSP 2.10 (Ed. p. 32): tatra prathamaḥ pādāṅguṣṭhādhāraḥ | tatrāgratas tejomayaṃ dhyāyet | dṛṣṭiḥ sthirā bhavati |}
\note[type=testium, labelb=_76b, labele=_76e, nosep]{ \approx  \citetitle{hathasamketacandrikajodhpur} (MMPP 2244 f. 98r l. 4): tatra mūlādhāraḥ 1 pādayor aṃguṣṭhe tejaso lakṣyakaraṇād dṛṣṭiḥ sthirā bhavati 2 ity ādhāracakraṃ |}
pādayo\skp{r-aṃ}\app{\lem[wit={ceteri}, alt={aṅguṣṭhe}]{\skm{r-aṅ}guṣṭhe}\linelabel{_76b}
  \rdg[wit={B}]{aṃguṣṭhai}}
\app{\lem[wit={ceteri}]{tejaso}
  \rdg[wit={B,L}]{tejasaṃ}}
\app{\lem[wit={ceteri}, alt={lakṣya°}]{lakṣya}
  \rdg[wit={N2}]{lakṣa°}
  \rdg[wit={B,L}]{lakṣaṃ kartavyaṃ}
}\app{\lem[wit={ceteri}, alt={°kāraṇād}]{kāraṇā\skp{d-dṛ}}
    \rdg[wit={P}]{°karaṇāt}
  }\app{\lem[wit={ceteri}, alt={dṛṣṭiḥ}]{\skm{d-dṛ}ṣṭiḥ}
   \rdg[wit={N1,N2,U1,U2}]{dṛṣṭi°}} sthirā
 \app{\lem[wit={ceteri}]{bhavati}
   \rdg[wit={L}]{bhavatī}}/\linelabel{_76e}
%-----------------------------
%dvitīyo mūlādhāraḥ/  pādāṃguṣṭhasya mūle parapādasya  pārṣṇiḥ                                         sthāpyate tadāgniḥ prabalo bhavati/ \E
%dvitīyo mūlādhāraḥ   pādāṃguṣṭhasya mūle 'parapādasya dhāraḥ pādāṃduṣṭhasya mūleḥ parapādasya pārṣṇiḥ sthāpyate tadāgniḥ prabalo bhavati \P
%dvitīyo mūlādhāraḥ/  pādāṃguṣṭhasya mūle aparasya pādapārṣṇiḥ                                         syāpyate tadāgniḥ  prabalo bhavatī/ \B
%dvitīyo mūlādhāraḥ   pādāṃguṣṭhasya mūle aparasya pādapārṣṇīḥ                                         syāpyate tadāgniḥ  prabalo bhavatī/ \L
%dvitīyo mūlādhāraḥ/  pādāṃguṣṭhasya mūle aparapādasya pārṣṇiḥ                                         sthāpyate agniḥ    prabalo bhavati/   \N1
%dvitīyo mūlādhāraḥ// pādāṃguṣṭhasya mūle aparapādasya pārṣṇiḥ                                         sthāpyate agni-----prabalo bhavati//   \D  %%%p.12 recto
%dvitīyo mūlādhāraḥ// pādāṃguṣṭhasya mūle aparapādasya pārṣṇi                                         sthāpyate agni-----prabalo bhavati//   \K1
%dvitīyo mūlādhāraḥ pādāṃguṣṭhasya mūle adharapādasya pārṣṇiḥ                                         sthāpyate// agniprabalo bhavati//   \J 
%dvitīyo mūlādhāraḥ   pādāṃguṣṭhasya mūle aparapādasya pārṣṇiḥ                                         sthāpyate/ \om                     \N2
%dvitīyo mūlādharaḥ   pādāṃguṣṭhasya mūle aparapādasya pārṣṇiḥ                                         sthāpyate agniṃ ---prabalo bhavati    \U1
%dvitīyo mūlādhare    pādāṃguṣṭhasya mūle 'parapādasya pārṣṇiḥ                                         sthāyyaṃte//                       \U2
%-----------------------------
%The second root-container is the second [one]. The heel of the other foot is caused to be placed at the root of the big toe. As a result the fire is strengthened. 
%-----------------------------
 \note[type=source, labelb=_77b, labele=_77e, nosep]{cf. YSv (PT, p. 839): dvitīyaṃ pādamūlan tu pādamūlaparaṃ (\textit{pādamūlaṃ paraṃ} YK 2.16) sa vai | pādasya pārṣṇī (\textit{pārṣṇi} YK 2.17a) saṃsthāpya balavān prabhaven muniḥ | pādamūle 'thavā pādāṅguṣṭhamūlaṃ (\textit{pṛṣṭhe pādāṅguṣṭhe} YK 2.17) vidhārayet ||}%The second is the root of the foot. That root of the foot is truly superior. Having placed himself on the heel of the foot the Muni becomes powerful. He shall hold [the gaze?] at the root of the foot or at the big toe.
 \note[type=source, labelb=_77b, labele=_77e, nosep]{cf. SSP 2.11 (Ed. p. 33): dvitīyo mūlādhāras taṃ vāmapādapārṣṇinā niṣpīḍya sthātavyam | tatrāgnidīpanaṃ bhavati |}
 \note[type=testium, labelb=_77b, labele=_77e, nosep]{ \approx  \citetitle{hathasamketacandrikajodhpur} (MMPP 2244 f. 98 ll. 5-7): atha dvitīyādhāraḥ | tatra tatra vāmapādāṃguṣṭasya mūlam aparapādasya pārṣṇis tasmin sthāpyate | tad āgneḥ pradīpanaṃ bhavati | ekaḥ pārṣṇi mūlādhāre dṛḍhaṃ sthāypyate | tasya pādasya mūla aṃguṣṭamūlam aparasya pādasya pārṣṇinā saṃpīḍya ciraṃ sthiraṃ sthīyate tadāgnīm agni dīpyate | iti dvitīyadhāraḥ |}
 %The second is the Mūlādhara which is to be pressend with the left heel. This enhances the bodily fire.
dvitīyo \linelabel{_77b}
\app{\lem[wit={ceteri}]{mūlādhāraḥ}
  \rdg[wit={U1}]{mūlādharaḥ}
  \rdg[wit={U2}]{mūlādhare}}/ 
pādāṅguṣṭhasya mūle\app{\lem[wit={ceteri},alt={'para°}]{'para}
  \rdg[wit={D,K1,N1,N2,U1}]{apara°}
  \rdg[wit={B,L}]{aparasya}
  \rdg[wit={J}]{adhara°}
}\app{\lem[wit={ceteri}]{pādasya}
  \rdg[wit={B,L}]{pāda°}}
\app{\lem[wit={ceteri}]{pārṣṇiḥ}
  \rdg[wit={L}]{°pārṣṇīḥ}
  \rdg[wit={K1}]{pārṣṇi}
  \rdg[wit={P}]{dhāraḥ pādāṃduṣṭhasya mūleḥ parapādasya pārṣṇiḥ}}
\app{\lem[wit={ceteri}]{sthāpyate}
  \rdg[wit={B,L}]{syāpyate}
  \rdg[wit={U2}]{sthāyyaṃte}}/
\app{\lem[wit={N1}]{agniḥ}
  \rdg[wit={U1}]{agniṃ}
  \rdg[wit={D,J,K1,}]{agni°}
  \rdg[wit={B,E,L,P}]{tadāgniḥ}
  \rdg[wit={N2,U2}]{\om}}
\app{\lem[wit={ceteri}]{prabalo}
  \rdg[wit={N2,U2}]{\om}}
\app{\lem[wit={ceteri}]{bhavati}
  \rdg[wit={B,L}]{bhavatī}
  \rdg[wit={N2,U2}]{\om}}/
%-----------------------------
%ekaḥ  pārṣṇir ādau  mūlādhāre  sthāpyate/      \E [P.41]
%ekā   pārṣṇir ādau  mūlādhāre  sthāpyate      \P
%ekā   pārṣṇir ādau  mūlādhāra  sthāpyate      \B
%ekā   pārṣṇir ādau  mūlādhārā  sthāpyate      \L
%ekā   pārṣṇiḥ       mūladdhāre sthāpyate/      \N1
%ekā   pārṣṇiḥ       mūlādhārai sthāpyate//     \D
%ekā   pārṣṇiḥ       mūlādhārai sthāpyate//     \K1
%ekaḥ  pārṣṇiḥ       mūlādhāraṃ sthāpyate// \J 
% \om -------------------------------------     \N2
%ekāṃ pārṣṇir        mūlādhāra sthāpyate        \U1
% \om                                          \U2
%-----------------------------
%One heel is caused to be placed at the Root-container. 
%-----------------------------
\app{\lem[wit={ceteri}]{ekā}
  \rdg[wit={E}]{ekaḥ}
  \rdg[wit={U1}]{ekāṃ}
  \rdg[wit={N2,U2}]{\om}}
\app{\lem[wit={U1},alt={pārṣṇiḥ}]{pārṣṇi\skp{r-mū}}
  \rdg[wit={D,K,K1,N1}]{pārṣṇiḥ}
  \rdg[wit={B,E,L,P}]{pārṣṇir ādau}
    \rdg[wit={N2,U2}]{\om}
}\app{\lem[wit={ceteri},alt={mūlādhāre}]{\skm{r-mū}lādhāre}
a  \rdg[wit={B,U1}]{mūlādhāra}
  \rdg[wit={L}]{mūlādhārā}
  \rdg[wit={D,K1}]{mūlādhārai}
  \rdg[wit={J}]{mūlādhāraṃ}
    \rdg[wit={N2,U2}]{\om}}
\app{\lem[wit={ceteri}]{sthāpyate}
    \rdg[wit={N2,U2}]{\om}}/
%-----------------------------
% tasya pādasyāṃguṣṭhamūle      parasya  pādasya pārṣṇiḥ sthāpyate// tadagniḥ pradīpyate//  \E [P.41]
% tasya pādasyāṃguṣṭhamūle     'parasya  pādasya pārṣṇiḥ sthāpyate   tadagnīḥ pradipyate    \P
% tasya pādasyāṃguṣṭhamūle     aparasya  pādasya pārṣṇiḥ sthāpyate// tadagnīḥ pradipyate//  \B
% tasya pādasyāṃguṣṭhamūle     aparasya  pādasya pārṣṇiḥ sthāpyate// tadāgnīḥ pradivyate//  \L
% tasya pādasya aṃguṣṭhamūlaṃ/ aparasya  pādasya pārṣṇiḥ sthāpyaṃ       agnir    dāpyate?!/ \N1
% tasya pādasyāṃguṣṭhamūle//   aparasya  pādasya pārṣṇiḥ sthāpyaṃ//     agnir    dīpyate//  \D
% tasya pādasyāṃguṣṭhamūlaṃ//  aparasya  pādasya pārṣṇiḥ sthāpyaṃ     agnir    dīpyate//  \K1
% tasya pādasya aṃguṣṭhamūlaṃ   aparasya          pārṣṇi  sthāpyate//          akṣipyate//    \J  
% tasya pādasyāṃguṣṭhamūle//   aparasya  pādasya pārṇi---sthāpyaṃ       agni     dīpate//   \N2
% tasya pādasya aṃguṣṭhamūlaṃ  aparasya          pārṣṇo  sthāpyate      agni     dīpyate    \U1
% \om                                                                tadagnīḥ pradipyate//  \U2
%-----------------------------
%The heel of the other foot is caused to be placed at the root of the big toe of this foot. The fire is kindled. 
%-----------------------------
\app{\lem[wit={ceteri}]{tasya}
  \rdg[wit={U2}]{\om}} 
\app{\lem[wit={ceteri}]{pādasyāṅguṣṭhamūle}
  \rdg[wit={N1,J,U1}]{pādasya aṃguṣṭhamūlaṃ}
  \rdg[wit={U2}]{\om}
}-\app{\lem[wit={E,P}]{'parasya}
  \rdg[wit={ceteri}]{aparasya}
  \rdg[wit={U2}]{\om}}
\app{\lem[wit={ceteri}]{pādasya}
  \rdg[wit={J,U1,U2}]{\om}}
\app{\lem[wit={ceteri}]{pārṣṇiḥ}
  \rdg[wit={J}]{pārṣṇi}
  \rdg[wit={N2}]{pārṇi}
  \rdg[wit={U1}]{pārṣṇo}
  \rdg[wit={U2}]{\om}}
\app{\lem[wit={ceteri}]{sthāpyate}
  \rdg[wit={D,K1,N1,N2}]{sthāpyaṃ}
  \rdg[wit={U2}]{\om}}/
\app{\lem[wit={D,K1,N1}, alt={agnir}]{agni\skp{r-pra}}
   \rdg[wit={N2,U1}]{agni}
  \rdg[wit={E}]{tadagniḥ}
  \rdg[wit={B,P,U2}]{tadagnīḥ}
  \rdg[wit={L}]{tadāgniḥ}
  \rdg[wit={J}]{\om}
}\app{\lem[wit={E}, alt={pradīpyate}]{\skm{r-pra}dīpyate}
  \rdg[wit={B,L,P,U2}]{pradipyate}
  \rdg[wit={D,U1}]{dīpyate}
  \rdg[wit={N1}]{dāpyate}
  \rdg[wit={N2}]{dīpate}
  \rdg[wit={J}]{akṣipyate}}/\linelabel{_77e}
\end{prose}
  \end{edition}
  \begin{translation}
    \ekddiv{
      head={[\uproman{30}. \textbf{Divisions of the wheels of support}]},
      type=section,
      depth=2, 
      n=XXX.1
    }
    \xmlhead[h30]{[XXX. Divisions of the wheels of support]}
    \begin{tlate}[p30_01]
      \noindent
      Now, the divisions of the group\footnote{I took \textit{cakra} in the sense of ``group, crowd, totality'', cf. \citeauthor[1958 (Vol. 2): 209]{petersburger}.} of supports\footnote{The practice of sixteen \textit{ādhāra}s goes back to the yoga traditions of Śaivism and is mentioned in texts such as \citetitle{tantraloka}, \citetitle{manthana} and \citetitle{netratantra}. The techniques were passed on, copied and recycled across the centuries among the yoga traditions of Haṭha- and Rājayoga. Besides Rāmacandra's text, the other texts which present full lists of the sixteen \textit{ādhāra}s are \textit{Netroddyota}-commentary of Kṣemarāja on \textit{Netratantra} 7.5; \citetitle{saradaavalon} 25.24-25; \citetitle{shivayogapradipika} 3.17-33; \citetitle{ssplonavla} 2.10-25; \citetitle{yogatarangini} 1.13 (Ed. p. 72-73) quotation with reference ``\textit{nityanāthapaddhatau}'' (maybe another recension of the \citetitle{ssplonavla}, see \citeauthor[2023: 149]{powell2023}); \citetitle{hathatattvakaumudi} 24.10-23 and 40.19; and \citetitle{jyotsna} on \citetitle{hathapradipika2024}, as well as \citetitle{ramatosana} (Ed. p. 839-841) quotation with reference ``\textit{yogasvarodaye}'' and \citetitle{yogakarnika} quotation with reference ``\textit{yogasvarodaye}'' 14-36. \textit{Haṭhasaṃketacandrikā} (cf. i.e. GOML R3239 f. 201 l. 20 - f. 204 ll. 5-6) directly quotes the \textit{Tattvayogabindu} without reference. Comparing the various lists of \textit{ādhāra}s reveals great variability. Rāmacandra's system draws from the \textit{Yogasvarodaya} and the \citetitle{ssplonavla}. When there are differences in the descriptions of the respective \textit{ādhāra}s among the texts I note them in the annotations without providing a reference again; for the Sanskrit, see the above-provided references.} are taught.%\footnote{Most of the previously mentioned \textit{cakra}s overlap with the \textit{ādhāra}s, except for the \textit{ākāśacakra}.}
      
     \hspace{1em}As a result of focusing on a light at the big toes of both feet, the gaze becomes steady.\footnote{In all previously mentioned systems, the big toe is the first \textit{ādhāra}. In most texts, the practitioner is instructed to fixate the mind onto the big toe - either one shall visualize a light there (as in \citetitle{shivayogapradipika}) or the light is already present. The \citetitle{saradaavalon}, however, instructs to fix \textit{prāṇa} in each \textit{ādhāra} listed. Here, the practice of the \textit{ādhāra}s is subsumed under the \textit{dhāraṇā}-limb of an eight-fold (\textit{aṣṭāṅga}) yoga system.}
      
     \hspace{1em}The root support is the second [one]. The heel of the back-foot is caused to be placed at the base of the big toe of the foot.\footnote{The base of the big toe of the foot (\textit{pādasyāṅguṣṭhamūla}) is probably the big toe joint of the foot or \textit{articulatio metatarsophalangealis hallucis}.} The fire is strengthened. [In other words,] one heel is placed at the root support. The heel of the other foot is placed at the base of the big toe of this foot. The fire is kindled.\footnote{Rāmacandra combines the techniques presented in YSv and SSP for this \textit{ādhāra}, resulting in a \textit{siddhāsana}-like bodily position.}\fnsep\footnote{\textit{Netroddyota}, \citetitle{saradaavalon} and \citetitle{jyotsna} give the ankle (\textit{gulpha}) as the second \textit{ādhāra}.}
      \flushpage 
    \end{tlate}
  \end{translation}
\end{alignment}
\hardbreak %after pp. 93-94
%%%%%%%%%%%%%%%%%%%%%%%%%%%%%%%%%%%%%%%%%%
%%%%%%%%%%%%%%%%%%%%%%%%%%%%%%%%%%%%%%%%%% 
%%%%%%%%PAGEBREAK%%%%%%%PAGEBREAK%%%%%%%%%
%%%%%%%%%%%%%%%%%%%%%%%%%%%%%%%%%%%%%%%%%% 
%%%%%%%%%%%%%%%%PAGEBREAK%%%%%%%%%%%%%%%%%
%%%%%%%%%%%%%%%%%%%%%%%%%%%%%%%%%%%%%%%%%% 
%%%%%%%%PAGEBREAK%%%%%%%PAGEBREAK%%%%%%%%%
%%%%%%%%%%%%%%%%%%%%%%%%%%%%%%%%%%%%%%%%%% 
%%%%%%%%%%%%%%%%%%%%%%%%%%%%%%%%%%%%%%%%%% 
%%%%%%%%%%%%%%%%%%%%%%%%%%%%%%%%%%%%%%%%%% 
%%%%%%%%%%%%%%%%%%%%%%%%%%%%%%%%%%%%%%%%%% 
%%%%%%%%PAGEBREAK%%%%%%%PAGEBREAK%%%%%%%%%
%%%%%%%%%%%%%%%%%%%%%%%%%%%%%%%%%%%%%%%%%% 
%%%%%%%%%%%%%%%%PAGEBREAK%%%%%%%%%%%%%%%%%
%%%%%%%%%%%%%%%%%%%%%%%%%%%%%%%%%%%%%%%%%% 
%%%%%%%%PAGEBREAK%%%%%%%PAGEBREAK%%%%%%%%%
%%%%%%%%%%%%%%%%%%%%%%%%%%%%%%%%%%%%%%%%%% 
%%%%%%%%%%%%%%%%%%%%%%%%%%%%%%%%%%%%%%%%%% 
%%%%%%%%%%%%%%%%%%%%%%%%%%%%%%%%%%%%%%%%%% 
%%%%%%%%%%%%%%%%%%%%%%%%%%%%%%%%%%%%%%%%%% 
%%%%%%%%PAGEBREAK%%%%%%%PAGEBREAK%%%%%%%%%
%%%%%%%%%%%%%%%%%%%%%%%%%%%%%%%%%%%%%%%%%% 
%%%%%%%%%%%%%%%%PAGEBREAK%%%%%%%%%%%%%%%%%
%%%%%%%%%%%%%%%%%%%%%%%%%%%%%%%%%%%%%%%%%% 
%%%%%%%%PAGEBREAK%%%%%%%PAGEBREAK%%%%%%%%%
%%%%%%%%%%%%%%%%%%%%%%%%%%%%%%%%%%%%%%%%%% 
%%%%%%%%%%%%%%%%%%%%%%%%%%%%%%%%%%%%%%%%%%
\begin{alignment}[
  texts=edition[class="edition"];
  translation[class="translation"],
  ]\textappfiddle[.2] 
  \begin{edition}
      \begin{prose}[p30_02] 
      \noindent
%-----------------------------
%tṛtīyaṃ gudādhārasthānaṃ   tanmadhye saṃkocavikāsākuṃcana--kāraṇāt pavanaḥ sthiro bhavati// \E
%tṛtīyaṃ gudādhārasthānaṃ   tanmadhye saṃkocavikāśākuṃcana--kāraṇāt pavanaḥ sthiro bhavati   \P
%tṛtīyaṃ gudādhārasthāne    tanmadhye saṃkocavikāśākuṃcana--kāraṇāt pavanaḥ sthiro bhavati// \B
%tṛtīyaṃ gudādhārasthānaṃ   tanmadhye saṃkocavikāśa ākuṃcanakāraṇāt pavanasthiro   bhavatī// \L
%tṛtīyaṃ gudādhārasthānaṃ   tanmadhye saṃkocavikāśākuṃcana--kāraṇāt pavanaḥ sthiro bhavati// \N1
%tṛtīyaṃ gudādhārasthānaṃ   tanmadhye saṃkocavikāśākuṃcanaṃ kāraṇāt pavanasthiro   bhavati// \D
%tṛtīyaṃ gudādhārasthānaṃ   tanmadhye saṃkocavikāśākuṃcanaṃ karaṇāt! pavanasthiro   bhavati// \K1
%tṛtīyaṃ gudādhārasthānaṃ// tanmadhye saṃkocavikāśākuṃcanakāraṇāt// pavanasthiro bhavati// \J (P17/34)      
%tṛtīyaṃ gudādhārasthānaṃ   taṃmadhye saṃkocavikāśākuṃcanaṃ kāraṇāt pavanasthiro   bhavati// \N2
%tṛtīyaṃ gudādhārasthānaṃ   taṃmadhye saṃkocavikāśā akuṃcanakāraṇāt pavanasthiro   bhavati \U1
%tṛtīya  gudādhārasthānaṃ// tanmadhye saṃkocavikāśākuṃcana--kāraṇāt pavanasthiro   bhavati// \U2
%-----------------------------
%The third is the place of the anus-container. From the execution of expansion and contraction a stable vital wind arises.   
%-----------------------------
\note[type=source, labelb=_78b, labele=_78e, nosep]{cf. YSv (PT, p. 839): tṛtīyan tu gudādhāro (\textit{gudādhāre} YK 2.18) gudasaṅkocanakriyā | vikāsākuñcanaṃ (em. \textit{vikāśā}° PT) tasya sthiravāyau ca mṛtyujit |}
\note[type=source, labelb=_78b, labele=78e, nosep]{cf. SSP 2.12 (Ed. p. 33): tṛtīyo gudādhāra taṃ vikāsasaṃkocanena nirākuñcayet | apānavāyuḥ sthiro bhavati |}
\note[type=testium, labelb=_78b, labele=_78e, nosep]{ \approx  \citetitle{hathasamketacandrikajodhpur} (MMPP 2244 f. 98r ll. 7-9): atha tṛṭīyādhāraḥ tṛtīyaṃ gudādhārasthānaṃ tanmadhye dṛḍhaṃ muhuś ciraṃ saṃkocanavikāsanarūpākuṃcanakaraṇād pānavāyuḥ dvā vāmapādād pārṣṇimūlena gudāsya nipīḍanād apānavāyuḥ sthiro bhavati | cāmaraṃ karoti sādhakaṃ | iti tṛtīyādhāraḥ |}
\app{\lem[wit={ceteri}, alt={tṛtīyaṃ}]{tṛtīyaṃ}\linelabel{_78b}
  \rdg[wit={U2}]{tṛtīya}}
gudādhāra \app{\lem[type=emendation, resp=egoscr, alt={°sthānam}]{sthānam}
  \rdg[wit={B}]{°sthāne}
  \rdg[wit={ceteri}]{°sthānaṃ}} 
\app{\lem[wit={ceteri},alt={°sthānam}]{sthānaṃ}
  \rdg[wit={B}]{°sthāne}}
tanmadhye
saṃkoca\app{\lem[wit={E}, alt={°vikāsā°}]{vikāsā}  
  \rdg[wit={B,D,P,N1,N2}]{°vikāśā°}
  \rdg[wit={L}]{°vikāśa}
  \rdg[wit={U1}]{°vikāśā}
}\app{\lem[wit={ceteri},alt={°kuñcana}]{kuñcana}
  \rdg[wit={L}]{ākuṃcana}
  \rdg[wit={U1}]{akuṃcana}
  \rdg[wit={D,K1,N2}]{kuṃcanaṃ}
}kāraṇā\skp{t-pa}\app{\lem[wit={ceteri},alt={pavanaḥ}]{\skm{t-pa}vanaḥ}
  \rdg[wit={D,J,K1,U1,U2,N2}]{pavana°}}
sthiro bhavati/
%\app{\lem[wit={ceteri}]{bhavati}
%  \rdg[wit={B}]{bhavatī}}/    
%-----------------------------
%anyac ca/ puruṣasya maraṇaṃ na bhavati/ \E
%anu ca puruṣasya maraṇaṃ bhavati  \P
%anucarapuruṣasya maraṇaṃ bhavatī/ \B
%anucakrapuruṣasya maraṇaṃ bhavatī/ \L
%anū ca puruṣasya maraṇaṃ na bhavati ve?/ \N1
%anu ca puruṣasya maraṇaṃ na bhavati// \D
%anu ca puruṣasya maraṇaṃ na bhavati// \K1
%anu ca puruṣasya maraṇaṃ na bhavati// \J
%anū ca puruṣasya maraṇaṃ na bhavati// \N2
%anu ca puruṣasya maraṇaṃ na bhavati  \U1
%anu ca puruṣasya maraṇaṃ na bhavati//  \U2
%-----------------------------
%And therefore death of the person does not arise.
%-----------------------------
\app{\lem[wit={ceteri}]{anu ca}
  \rdg[wit={E}]{anyac ca}
  \rdg[wit={N1,N2}]{anūca}
  \rdg[wit={B}]{anucara°}
  \rdg[wit={L}]{anucakra°}}
puruṣasya maraṇaṃ
\app{\lem[wit={ceteri}]{na}
  \rdg[wit={B,P,L}]{\om}}
bhavati/\linelabel{_78e} 
%-----------------------------
%caturthaṃ liṃgādhāraṃ   tanmadhye/ liṃgasaṃkocanābhyāsāt  paścimadaṇḍamadhye prajñā nāḍī bhavati/  tanmadhye punar abhyāsakaraṇān manaḥpavanayoḥ saṃcāro bhavati/ \E
%caturthaṃ liṃgādhāraṃ   tanmadhye  liṃgasaṃkocanābhyāsāt  paścīmadaṇḍamadhye vajñā nāḍī  bhavati   tanmadhye punar abhyāsakaraṇān manaḥpavanayoḥ saṃcāro bhavati \P
%caturtha--liṃgādhāraṃ   tanmadhye  liṃgasaṃkocanābhyāsāt  paścīmadaṇḍamadhye vajñā nāḍī  bhavatī/  tanmadhye punar abhyāsakaraṇāt punaḥ pavanayo  saṃcāro bhavatī/     \B
%caturtha--liṃgādhāraṃ// tanmadhye  liṃgasaṃkocanābhyāsāt  paścamadaṇḍamadhye vajñā nāḍī  bhavatī// tanmadhye punar abhyāsakaraṇāt punaḥ pavanayo  saṃcāro bhavatī//     \L %%%%%%%%%%%20.jpg
%caturthaṃ liṃgādhāraṃ   tanmadhye/ liṃgasaṃkocanābhyāsāt/ paścimadaṇḍamadhye vajranāḍī   bhavati/  tanmadhye punaḥ abhyāsakaraṇāt manaḥpavanayoḥ saṃcāro bhavati/ \N1
%caturtha--liṃgādhāraṃ// tanmadhye/ liṃgasaṃkocanābhyāsāt//paścimadaṇḍamadhye vajrānāḍī   bhavati// tanmadhye punaḥ abhyāsakaraṇāt manaḥpavanayoḥ saṃcoro bhavati// \D
%caturtha--liṃgādhāraṃ// tanmadhye/ liṃgasaṃkocanābhyāsāt//paścimadaṇḍamadhye vajrānāḍī   bhavati// tanmadhye punaḥ abhyāsakaraṇāt manaḥpavanayoḥ saṃcoro bhavati// \K1
%caturtha--liṃgādhāraṃ tanmadhye liṃgasaṃkocanābhyāsāt//   paścimadaṇḍamadhye vajrānāḍī   bhavati// tanmadhye punaḥ abhyāsakaraṇāt// manaḥpavanayoḥ saṃcāro bhavati// \J
%caturthaṃ liṃgādhāraṃ   tanmadhye  liṃgasakoṇābhyāsāt//   paścimadaṇḍamadhye vajranāḍī   bhavati/  tanmadhye punar ābhyāsakaraṇāt manaḥpavanayoḥ saṃcāro bhavati// \N2
%caturthaṃ liṃgādhāraṃ   tanmadhye  liṃgasaṃkocanābhyāsāt  paścimadaṇḍamadhye vajranāḍī   bhavati   tanmadhye punar ābhyāsakaraṇāt manaḥpavanayoḥ saṃcāro bhavati    \U1    %%%283.jpg
%caturthaṃ liṃgādhāraṃ   tanmadhye  liṃgasaṃkocanābhyāsāt  paścimadaṇḍamadhye vajranāḍī   bhavati   tanmadhye punar ābhyāsakaraṇān manaḥpavanayoḥ saṃcāro bhavati//   \U2
%-----------------------------
%The fourth is the penis support. Due to the execution of repeated practice of contracting the penis in the centre of it, the adamantine channel appears in the middle of the staff of the back. From the repeated practice again [and again] the transition of both breath and mind into its center arises.  
%-----------------------------
      \note[type=source, labelb=_79b, labele=_79e, nosep]{cf. Ysv (PT, pp. 839-840): liṅgādhāraṃ caturthan tu liṅgasaṅkocanan tu ca | liṅgasaṅkocanābhyāsāt paścimādaṇḍamadhyagaḥ | vajranāḍīti (\textit{vajrānāḍī tu} YK 2.20) tanmadhye punar abhyasyaṃs (\textit{abhyasanan} YK 2.20) tathā | sañcāro vāyumanasor atisañcāra iti (\textit{ratiṃ sañcarati} YK 2.20) tridhā | granthitrayavibhedas (\textit{°bhedan} YK 2.21) tu tadbhedo brahmamārgataḥ | brahmapadmo (\textit{°padme} YK 2.21) vāyupūrṇo (\textit{°pūrṇe} YK 2.21) bhūtvā tiṣṭhati yogirāṭ | vīryastambho bhavet tena sādhayet tu sadā yuvā | mūlādhāre brahmapadme ṣaṭpadme ca tathā tathā |}
      \note[type=testium, labelb=_79b, labele=_79e, nosep]{ \approx  \citetitle{hathasamketacandrikajodhpur} (MMPP 2244 f. 98r l. 9 - f. 95v l. 3): atha caturthaṃ liṃgādhāraḥ | tanmadhye liṃgasaṃkocanābhysāt mūlabaṃdhena gudāyā muhuḥ saṃkocane kṛte liṃgasaṃkocanaṃ svayame liṃgasaṃkocanābhyāsat mūlabaṃdhena gudāyā muhuḥ saṃkocane kṛte liṃgasaṃkocanaṃ svayaṃ eva bhavati | paścimadaṃḍamadhye vajranāḍī bhavati tanmadhye punarabhyāsakaraṇān manaḥpavanayoḥ saṃcāro bhavati | tayoḥ saṃcārān madhye graṃthitrayaṃ trudyati | tat troṭaṇāt pavano brahmakamalamadhye pūrṇo bhūtvā tiṣṭhati | tadā vīryastaṃbho bhavati | puruṣa sadaiva yuvā tiṣṭhati | iti caturthādhāraḥ 4\vspace{-2mm}}   
      \note[type=source, labelb=_79b, labele=_79e, nosep]{cf. SSP 2.13 (Ed. pp. 33-34): caturtho meḍhrādhāraḥ | liṅgasaṃkocanena brahmagranthitrayaṃ bhitvā bhramaraguhāyāṃ viśramya tata ūrdhvamukhe bindustambhanaṃ bhavati| eṣā vajrolī prasiddhā\vspace{-2mm}}
\app{\lem[wit={ceteri}]{caturthaṃ}\linelabel{_79b}
  \rdg[wit={B,D,J,K1,L}]{caturtha°}}
liṅgādhāram/
tanmadhye
liṅga\app{\lem[wit={ceteri},alt={saṃkocanā°}]{saṃko:\\canā}
  \rdg[wit={N2}]{sakoṇā°}
}bhyāsā\skp{t-pa}\app{\lem[wit={ceteri}, alt={paścima°}]{\skm{t-pa}ścima}
  \rdg[wit={B,P}]{paścīma°}
  \rdg[wit={L}]{paścama°}
}daṇḍamadhye\hfill
\app{\lem[wit={ceteri}, alt={vajra°}]{vajra}
  \rdg[wit={D,J,K1}]{vajrā°}
  \rdg[wit={B,P,L}]{vajñā°}
  \rdg[wit={E}]{prajñā°}
}nāḍī\hfill bhavati/\hfill 
%\app{\lem[wit={ceteri}]{bhavati}
%  \rdg[wit={B,L}]{bhavatī}}/
tanmadhye\hfill punar-abhyāsa\app{\lem[wit={E,P,U2}, alt={°karaṇān}]{karaṇā\skp{n-ma}}
  \rdg[wit={ceteri}]{karaṇāt}
}\app{\lem[wit={ceteri}, alt={manaḥ°}]{\skm{n-ma}naḥ}
  \rdg[wit={B,L}]{punaḥ°}
}:\\\app{\lem[wit={ceteri}]{pavanayoḥ}
  \rdg[wit={B,L}]{pavanayo}}
\app{\lem[wit={ceteri}]{saṃcāro}
  \rdg[wit={D,K1}]{saṃcoro}}
%\app{\lem[wit={ceteri}]{bhavati}
% \rdg[wit={B,L}]{bhavatī}}/
bhavati/ 
%-----------------------------
%tayoḥ saṃcārān  madhye granthitrayaṃ truṭyati/   \E
%tayoḥ saṃcārān  madhye graṃthitrayaṃ truṭyati    \P
%tayo  saṃcārān  madhye granthitrayaṃ truṭyatī/   \B
%tayoḥ saṃcārān  madhye graṃthitrayaṃ truṭayatī   \L
%tayoḥ saṃcārān  madhye granthitrayaṃ truṭyati/   \N1 %truṭyati="zerbrechen"
%tayoḥ saṃcārāt  madhye graṃthitrayaṃ truṭyati//  \D
%tayoḥ saṃcārāt  madhye graṃthitrayaṃ truṭyati//  \K1
%tayoḥ saṃcārāt  madhye graṃthitrayaṃ truṭyati//  \J
%tayoḥ saṃcārān  madhye granthitrayaṃ ... ..ti/   \N2
%tayoḥ saṃccārāt madhye graṃthitrayaṃ trudyati    \U1
%tayoḥ saṃccārān madhye graṃthitrayaṃ truṭyati//  \U2
%-----------------------------
%Caused by the transition of them both into the center the trinity of knots breaks.
%-----------------------------
\app{\lem[wit={ceteri}]{tayoḥ}
  \rdg[wit={B}]{tayo}}
\app{\lem[wit={ceteri},alt={saṃcārān}]{saṃcārā\skp{n-ma}}
  \rdg[wit={D,J,K1,U1}]{saṃcārāt}
}\skm{n-ma}dhye
granthitrayaṃ
\app{\lem[wit={ceteri}]{truṭyati}
  \rdg[wit={B}]{truṭyatī}
  \rdg[wit={L}]{truṭayatī}
  \rdg[wit={U1}]{trudyati}
  \rdg[wit={N2}]{ti}}/
%-----------------------------
% tatroṭanāt        pavano  brahmakamalamadhye pūrṇo bhūtvā tiṣṭhati/  \E
%                                                                      \P
% tatroṭanāt        pavano  brahmakamadhye     pūrṇā bhūtvā tiṣṭhati// \B
% tatroṭanāt        pavano  brahmakamadhye     pūrṇā bhūtvā tiṣṭhati// \L
% tattroṭanāt       pavanaḥ brahmakamalamadhye pūrṇo bhūtvā tiṣṭhati/  \N1 
% tata troṭanāt     pavanaḥ brahmakamalamadhye pūrṇo bhūtvā tiṣṭhati// \D
% tat troṭanāt      pavanaḥ brahmakamalamadhye pūrṇo bhūtvā tiṣṭhati// \K1
% tatroṭanāt        pavanaḥ brahmakamalamadhye pūrṇo bhūtvā tiṣṭhati// \J 
% tata troṭanāt     pavanaḥ brahmakamalamadhye pūrṇo bhūtvā tiṣṭhati/  \N2
% tatroṭaṇāt        pavanaḥ brahmakamalamadhye pūrṇo bhūtvā tiṣṭhati   \U1
% tattroṭaṇāt       pavanaḥ brahmakamalamadhye pūrṇo bhūtvā tiṣṭhati// \U2
%-----------------------------
% There, from the breaking of that, the vital wind, after having filled up (the central channel?) resides within the Brahma-lotus. 
%-----------------------------
\app{\lem[wit={K1,N1,U2},alt={°tattroṭanāt}]{tattroṭanā\skp{t-pa}}
  \rdg[wit={B,E,J,L,U1}]{tatroṭanāt}
  \rdg[wit={D,N2}]{tata troṭanāt}
}\app{\lem[wit={B,E,L},alt={pavano}]{\skm{t-pa}vano}
  \rdg[wit={ceteri}]{pavanaḥ}}
brahma\app{\lem[wit={ceteri}, alt={°kamala°}]{kamala}
  \rdg[wit={B,L}]{°ka°}
}madhye
\app{\lem[wit={ceteri}]{pūrṇo}
  \rdg[wit={B,L}]{pūrṇā}}
bhūtvā tiṣṭhati/
%-----------------------------
% tato vīryastambho bhavati/  puruṣaḥ sadaiva   yuvā      bhavati/ \E
% tato vīryastaṃbho bhavati   puruṣaḥ saṃdaivaṃ yuve   prabhavati  \P
% tato vīryastambho bhavatī// puruṣaḥ sadaiva   yuvai     bhavatī/ \B
% tato vīryastaṃbho bhavati   puruṣaḥ sadaiva   yuvaiva   bhavati// \L
% tato vīryastambho bhavati/  puruṣaḥ sadaiva   yuvā/e va bhavati// \N1 %truṭyati="zerbrechen"
% tato vīryastambho bhavati// puruṣaḥ sadaiva   yuvaiva   bhavati// \D
% tato vīryastambho bhavati// puruṣaḥ sadaiva   yuvaiva   bhavati// \K1
% tato vīryastambho bhavati// puruṣaḥ sadaiva   yuvaiva   bhavati// \J
% tato vīryastambho bhavati/  puruṣa  sadaiva   yurvaiva  bhavati// \N2
% tato vīryastaṃbho bhavati/  puruṣaḥ sadaiva   yuvaivaṃ  bhavati \U1
% tato vīryastaṃbho bhavati   puruṣaḥ sadaiva   vaibhavo  bhavati// \U2
%-----------------------------
% From that virility and strength arise. The person becomes youthful forever.
%-----------------------------
tato vīryastambho bhavati/
\app{\lem[wit={ceteri}]{puruṣaḥ}
  \rdg[wit={N2}]{puruṣa}}
\app{\lem[wit={ceteri}]{sadaiva}
  \rdg[wit={P}]{saṃdaivaṃ}}
\app{\lem[wit={D,J,K1,L}]{yuvaiva}
  \rdg[wit={E}]{yuvā}
  \rdg[wit={P}]{yuve}
  \rdg[wit={B}]{yuvai}
  \rdg[wit={N1}]{yuve va}
  \rdg[wit={N2}]{yurvaiva}
  \rdg[wit={U1}]{yuvaivaṃ}
  \rdg[wit={U2}]{vai bhavo}}
\vspace{-1mm}\app{\lem[wit={ceteri}]{bhavati}
  %\rdg[wit={B}]{bhavatī}
  \rdg[wit={P}]{prabhavati}}/\linelabel{_79e}
    \end{prose}
  \end{edition}
  \begin{translation}
    \begin{tlate}[p30_02]
\hspace{1em} The third is the place of the anus support.\footnote{\textit{Netroddyota}, \citetitle{saradaavalon} and \citetitle{jyotsna} provide the knee (\textit{jānu}) as the third \textit{ādhāra}.} As a result of expansion, contraction and compression, the vital wind becomes stable [on it]. And then, the person does not die.
      
\hspace{1em} The fourth is the penis support. As a result of the practice of contracting the penis in the middle of the [support], the adamantine channel (i.e., central channel)\footnote{The adamantine channel (\textit{vajranāḍī}) is another synonym for the central channel. Rāmacandra adapted the term from the \textit{Yogasvarodaya}. \citetitle{yogatarangini} in the commentary on 1.13 uses the term \textit{vajragarbha} (``adamantine womb'').} arises in the posterior staff (i.e., spine).\footnote{The posterior staff (\textit{paścimadaṇḍa}) is the spine. Cf. \citetitle{peterson1888} 4365.} From the repeated practice, both breath and mind move into that. Caused by the transition of both [breath and mind] the trinity of knots\footnote{The trinity of knots are: 1. the knot of Brahmā (\textit{brahmagranthi}) is situated in the lower regions of the body (cf. \citetitle{liersch2023} 23-24); 2. the knot of Viṣṇu (\textit{viṣṇugranthi}) at the level of the heart (cf. \citetitle{liersch2023} 25 and \citetitle{gsatacod} 80); and 3. the knot of Rudra (\textit{rudragranthi}) at the level of the head or between the eyebrows (cf. \citetitle{liersch2023} 25 and \citetitle{gsatacod} 81). Depending on text and tradition, it is either the breath (cf. \citetitle{asiddhi} 13.9-11) or the \textit{kuṇḍalinī} (cf. \citetitle{yogabija} 96-97 and \citetitle{gsatacod} 74-86) that enters the central channel and pierces the knots. \citetitle{gsatacod} 48 states that the entrance to the central channel is blocked by phlegm and that the three knots have arisen from the three \textit{guṇa}s. They obstruct the central passage.} within [the central channel] are pierced. Because of the piercing of those [knots], the breath becomes full in Brahmā's lotus and remains there.\footnote{Brahman's lotus refers to the eighth \textit{cakra} in Rāmacandra's system, cf. chapter \uproman{8}, p. \pageref{cakra8}. The same location is expressed in the \citetitle{ssplonavla} 2.13 and \citetitle{yogatarangini} commentary on 1.13 with the term \textit{brahmaraguhā} (``buzzing hive'') situated on top of the head (\citetitle{jogpradipyaka} 932; also cf. \citetitle{peterson1888} 4366 and \citetitle{gorakbhani} 28.2 and 30.4).} As a result of that, the stopping of semen arises.\footnote{Breath, mind and semen are interconnected. If one of them stops its movement, all stop their movement. Cf. \citetitle{asiddhi} 7.19-20 and 23.} The person becomes youthful forever.\footnote{Most of the consulted texts situate the fourth \textit{ādhāra} at the penis (\textit{meḍhra}). \citetitle{saradaavalon} and \citetitle{jyotsna} place the fourth support at the thighs (\textit{ūru}). \citetitle{shivayogapradipika} 3.20 and \citetitle{ssplonavla} 2.13 additionally associate the practice with the arrest of semen (\textit{bindustambha}). However, \citetitle{ssplonavla} calls this \textit{vajrolī}.}
\flushpage
    \end{tlate}
  \end{translation}
\end{alignment}
\hardbreak %after pp. 95-96
%%%%%%%%%%%%%%%%%%%%%%%%%%%%%%%%%%%%%%%%%%
%%%%%%%%%%%%%%%%%%%%%%%%%%%%%%%%%%%%%%%%%% 
%%%%%%%%PAGEBREAK%%%%%%%PAGEBREAK%%%%%%%%%
%%%%%%%%%%%%%%%%%%%%%%%%%%%%%%%%%%%%%%%%%% 
%%%%%%%%%%%%%%%%PAGEBREAK%%%%%%%%%%%%%%%%%
%%%%%%%%%%%%%%%%%%%%%%%%%%%%%%%%%%%%%%%%%% 
%%%%%%%%PAGEBREAK%%%%%%%PAGEBREAK%%%%%%%%%
%%%%%%%%%%%%%%%%%%%%%%%%%%%%%%%%%%%%%%%%%% 
%%%%%%%%%%%%%%%%%%%%%%%%%%%%%%%%%%%%%%%%%% 
%%%%%%%%%%%%%%%%%%%%%%%%%%%%%%%%%%%%%%%%%% 
%%%%%%%%%%%%%%%%%%%%%%%%%%%%%%%%%%%%%%%%%% 
%%%%%%%%PAGEBREAK%%%%%%%PAGEBREAK%%%%%%%%%
%%%%%%%%%%%%%%%%%%%%%%%%%%%%%%%%%%%%%%%%%% 
%%%%%%%%%%%%%%%%PAGEBREAK%%%%%%%%%%%%%%%%%
%%%%%%%%%%%%%%%%%%%%%%%%%%%%%%%%%%%%%%%%%% 
%%%%%%%%PAGEBREAK%%%%%%%PAGEBREAK%%%%%%%%%
%%%%%%%%%%%%%%%%%%%%%%%%%%%%%%%%%%%%%%%%%% 
%%%%%%%%%%%%%%%%%%%%%%%%%%%%%%%%%%%%%%%%%% 
%%%%%%%%%%%%%%%%%%%%%%%%%%%%%%%%%%%%%%%%%% 
%%%%%%%%%%%%%%%%%%%%%%%%%%%%%%%%%%%%%%%%%% 
%%%%%%%%PAGEBREAK%%%%%%%PAGEBREAK%%%%%%%%%
%%%%%%%%%%%%%%%%%%%%%%%%%%%%%%%%%%%%%%%%%% 
%%%%%%%%%%%%%%%%PAGEBREAK%%%%%%%%%%%%%%%%%
%%%%%%%%%%%%%%%%%%%%%%%%%%%%%%%%%%%%%%%%%% 
%%%%%%%%PAGEBREAK%%%%%%%PAGEBREAK%%%%%%%%%
%%%%%%%%%%%%%%%%%%%%%%%%%%%%%%%%%%%%%%%%%% 
%%%%%%%%%%%%%%%%%%%%%%%%%%%%%%%%%%%%%%%%%%
\begin{alignment}[
  texts=edition[class="edition"];
  translation[class="translation"],
  ]\textappfiddle[.2]
  \begin{edition}
    \begin{prose}[p30_03]
      \noindent
%-----------------------------
%paṃcama  udgīryāṇāṃ svādhiṣṭhānaṃ tatra bandhanān      malamūtrayor nāśo   bhavati/  \E
%paṃcamaṃ uḍḍīyāṇāṃ  svādhiṣṭhānaṃ tatra baṃdhadānān    malamūtrayor nāśo   bhavati   \P
%paṃcama  uḍḍiyānāṃ  svādhiṣṭhānaṃ tatra baṃdha dīyate/ malamūtrayor nāśo   bhavatī// \B
%paṃcamaṃ uḍḍiyānāṃ  svādhiṣṭhānaṃ tatra baṃdha dīyate/ mūlamūcayor  nāśo   bhavati// \L 
%paṃcamaṃ udyānaṃ                  tatra baṃdhanāt      malamūtrayor nāśe/o bhavati// \N1 [s.10, verso, z4]
%paṃcamaṃ udyāṇāṃ                  tatra vaṃdhanāt      malamūtrayor nāśo   bhavati// \D
%paṃcamaṃ udyāṇaṃ                  tatra baṃdhanāt      malamūtrayor nāśo   bhavati// \K1
%paṃcamaṃ udyāṇāṃ//                tatra baṃdhadānāt//  malamūtrayor nāśo   bhavati// \J    
%paṃcam   odyānaṃ                  tatra baṃdhanāt      malamūtrayor nāśo   bhavati/  \N2
%paṃcamaṃ uddyānaṃ                 tatra baṃdhadānāt    malamūtrayor nāśo   bhavati   \U1
%paṃcamaṃ uḍḍīyāṇaṃ  svādhiṣṭhānaṃ tatra badhadānān     malamūtrayor nāśo   bhavati// \U2
%-----------------------------
%The fifth is Udyāna. From performing \textit{bandha} there, urine and faeces disappear.  
%-----------------------------
\note[type=source, labelb=_80b, labele=_80e, nosep]{cf. YSv (PT, p. 840): pañcamaṃ jaṭharādhāraṃ tadā bandhayati kramāt | mṛtyunā bhaṅgasiddho 'yaṃ (\textit{mṛtyunā māṅga°} YK 2.23) mṛtyor (\textit{mṛtyur} YK 2.23) eva kṣayaṅkaraḥ | anena paścimād ūrddhaṃ (\textit{ūrdhvaṃ} YK 2.24) vāyuḥ kuryād viśāladhīḥ | bandho 'yaṃ buddhimanasoḥ pañcamādhārakālajit |}
\note[type=source, labelb=_80b, labele=_80e, nosep]{cf. SSP 2.14 (Ed. p. 34): pañcame oḍīyāṇādhārayor bandhanān malamūtrasaṃkocanaṃ bhavati | *uḍyānā° etc. in various mss.}
      \note[type=testium, labelb=_80b, labele=_80e, nosep]{ \approx  \citetitle{hathasamketacandrikajodhpur} (MMPP 2244 f. 98v ll. 3-4: athāmamudrāṇāṃ | tatra baṃdhanān malamūtranāśo bhavati |}
\app{\lem[wit={ceteri}]{pañcamaṃ}\linelabel{_80b}
  \rdg[wit={B}]{paṃcama}
  \rdg[wit={N2}]{paṃcam}}
\app{\lem[wit={P,U2}]{uḍḍīyāṇaṃ svādhiṣṭhānam}
  \rdg[wit={B,L}]{uḍḍiyānāṃ svādhiṣṭhānaṃ}
  \rdg[wit={P}]{uḍḍīyāṇāṃ svādhiṣṭhānaṃ}
  \rdg[wit={E}]{udgīryāṇāṃ svādhiṣṭhānaṃ}
  \rdg[wit={D,K1,N1}]{udyānaṃ}
  \rdg[wit={J}]{udyāṇāṃ}
  \rdg[wit={N2}]{odyānaṃ}
  \rdg[wit={U1}]{uddyānaṃ}}/
tatra
\app{\lem[wit={E}]{bandhanā\skp{n-ma}}
  \rdg[wit={U2}]{badhadānān}
  \rdg[wit={K1,N1,N2}]{baṃdhanāt}
  \rdg[wit={D}]{vaṃdhanāt}
  \rdg[wit={J,U1}]{baṃdhadānāt}
  \rdg[wit={P}]{baṃdhadānān}
  \rdg[wit={B,L}]{baṃdha dīyate}
}\app{\lem[wit={ceteri},alt={malamūtrayor}]{\skm{n-ma}lamūtrayo\skp{r-nā}}
  \rdg[wit={L}]{mūlamūcayor}
}\skm{r-nā}śo
\app{\lem[wit={ceteri}]{bhavati}
  \rdg[wit={B}]{bhavatī}}/\linelabel{_80e}      
%-----------------------------
%ṣaṣṭho nābhyādhāraḥ/    \E
%ṣaṣṭho nābhyādhāraḥ   tatra         praṇavābhyāsād  anāhato  nāraḥ   svayam utpadyate / \P
%ṣaṣṭho nābhyādhāraḥ   tatra         praṇavābhyāsād  anāhato  nādaḥ// svayam utpadyate// \B
%ṣaṣṭho nābhyādhāraḥ   tatra         praṇavābhyāsād  anāhato  nādaḥ// svayam utpadyate... \L 
%ṣaṣṭho nābhyādhāraḥ/  tatra         praṇavābhyāsāt  anāhato  nādaḥ   svayam ūtpadyate/  \N1
%ṣaṣṭho nābhyādhāraḥ// tatra         praṇavābhyāsāt  anāhato  nādaḥ// svayam utpadyate// \D
%ṣaṣṭho nābhyādhāraḥ// tatra         praṇavābhyāsāt  anāhato  nādaḥ svayyam utpadyate// \K1
%ṣaṣṭho nābhyādhāras   tatra         praṇavābhyāsād  anāhato  nādaḥ   svayam utpadyate// \J
%ṣaṣṭho nābhyādhāraḥ   tatra         praṇavābhyāsāt  anāhato  tādaḥ   svayaṃ utpadyate/ \N2
%ṣaṣṭho nābhyādhāras   tatra         praṇavābhyāṃsad anāhato  nadaḥ   svayam utpadyate   \U1
%ṣaṣṭho nābhyādhāre//  tatra         praṇavābhyāsād  anohato  nādaḥ   svayam utpadyate// \U2
%-----------------------------
%The sixth is the support of the navel. From repeated practice of \textit{praṇava}, the unstruck sound arises by itself. 
%-----------------------------
\note[type=source, labelb=_81b, labele=_81e, nosep]{cf. YSv (PT, p. 840): nābhyādhāro bhavet ṣaṣṭhas (\textit{ṣaṣṭhaṃ} YK 2.25) tatra prāṇaṃ samabhyaset | svayam utpadyate nādo nādato muktidantataḥ (\textit{muktidaṇḍataḥ} YK 1.25) |}
\note[type=source, labelb=_81b, labele=_81e, nosep]{cf. SSP 2.15 (Ed. p. 34): ṣaṣṭhe nābhyādhāra oṃkāram ekacittenoccārayet | nādalayo bhavati |}
\note[type=testium, labelb=_81b, labele=_81e, nosep]{ \approx  \citetitle{hathasamketacandrikajodhpur} (MMPP 2244 f. 98v ll. 4-5): atha ṣaṣṭhonābhyādhāraḥ 6 tatra praṇavābhyāse harau samāhitamanaḥ puruṣasya anāhatanādo manaḥ | sthairyaṃ svayam utpadyate |}
ṣaṣṭho\linelabel{_81b}
\app{\lem[wit={ceteri}, alt={nābhyādhāraḥ}]{nābhyā:\\dhāraḥ}
  \rdg[wit={J,U1}]{nābhyādhāras}
  \rdg[wit={U2}]{nābhyādhāre}}/\hfill
\app{\lem[wit={ceteri}]{tatra}
  \rdg[wit={E}]{\om}}\hfill
\app{\lem[wit={J,Y}, alt={praṇavābhyāsād}]{praṇavābhyāsā\skp{d-a}}
  \rdg[wit={D,K1,N1,N2}]{praṇavābhyāsāt}
  \rdg[wit={U1}]{praṇavābhyāṃsad}
}\app{\lem[wit={ceteri},alt={anāhato}]{\skm{d-a}nāhato}
    \rdg[wit={U2}]{anohato}
    \rdg[wit={E}]{\om}}\hfill
  \app{\lem[wit={ceteri}]{nādaḥ}
    \rdg[wit={P}]{nāraḥ}
    \rdg[wit={E}]{\om}}\hfill
  \app{\lem[wit={ceteri}]{svaya\skp{m-u}}
    \rdg[wit={N2}]{svayaṃ}
    \rdg[wit={E}]{\om}\hfill
}\app{\lem[wit={ceteri},alt={utpadyate}]{\skm{m-u}tpadyate}
  \rdg[wit={N1}]{ūtpadyate}
  \rdg[wit={E}]{\om}}/\hfill\linelabel{_81e}
%\note[type=philcomm, labelb=225a, lem={tatra \ldots svayam utpadyate}]{Sentence omitted in \getsiglum{E}.}\linelabel{_225e}
%-----------------------------
%                             tasmin sthāne prāṇavāyor  nirodhāt            ṣaḍapi kamalāny ūrdhvamukhāni             vikasaṃti// \E                                                     
%saptamo hṛdayarūpadhāraḥ     tasmin sthāne prāṇavāyor  nirodhāt            ṣadapi kamalāny ūrdhvamukhāni             vikasaṃti  \P  %%%7653.jpg 
%                             tasmin sthāne prāṇavāyo   nirodhāt/           ṣaḍapi kamalāny ūrdhvamukhāni             vikasaṃti// \B
%saptamo hṛdayarūpadhāraḥ//   tasmin sthāne prāṇavāyor  nirodhāt            ṣadapi kamalāny ūrdhvamukhāni             vikasaṃti// \L
%saptamo hṛdayarūpa ādhāraḥ   tasmin sthāne prāṇavāyor  nirūṃdhanāt/        ṣadapi kamalāny ūrdhvamukhaṃ              vikasaṃti// \N1
%saptamo hṛdayarūpa ādhāraḥ// tasmin sthāne prāṇavāyor  nir???ūṃ???dhanāt// ṣadapi kamalāny ūrdhvamukhaṃ              vikasaṃti// \D
%saptamo hṛdayarūpa ādhāraḥ// tasmin sthāne prāṇavāyor  nirudhanāt//        ṣadapi kamalāny ūrdhvamukhaṃ              vikasaṃti// \K1
%saptamo hṛdayarūpādhāraḥ//   tasmin sthāne prāṇavāyor  nirrudhanāt//       ṣadaṃ pi kamalān puttaṣasyordhvamukhaṃ bhavati//             vikasaṃti// \J
%saptamo hṛdayarūpādhāraḥ     tasmin sthāne prāṇavāyor  nirūṃdhanāt/        ṣadapi kamalāny ūrdhve mukhaṃ              vikasaṃti// \N2 %%%%%%%%%[S.9, recto, z.4]
%saptamo hṛdayarūpādhāraḥ     tasmin sthāne prāṇavāyor  nirūṃdhanāt         ṣadapi kamalāny ūrusyordha mukhaṃ bhavati vikasaṃti  \U1
%saptamo hṛdayādhāraḥ         tasmin sthāne prāṇavāyor  nirodhāt//          ṣadapi kamalāny ūrddhvamukhāni            vikasaṃti//  \U2
%-----------------------------
%The seventh is the support of the heart-form. From the restraint of the Prāṇa vital wind in this location the six upward facing lotusses blossom.   
%-----------------------------
\note[type=source, labelb=_82b, labele=_82e, nosep]{cf. SSP 2.16 (Ed. p. 34): saptame hṛdayādhāre prāṇaṃ nirodhayet | kamalavikāso bhavati |}
\note[type=source, labelb=_82b, labele=_82e, nosep]{cf. YSv (PT, p. 840): saptamo hṛdayādhāras tasmin vāyunibandhanāt | ūrddhaktrāṇi (\textit{ūrdhvavaktrāṇi} YK 2.26) padmāni vikasanti mahān bhavet |}
\note[type=testium, labelb=_82b, labele=_82e, nosep]{ \approx  \citetitle{hathasamketacandrikajodhpur} (MMPP 2244 f. 98v ll. 5-6): atha saptamaṃ hṛdayarūpa ādhāraḥ 7 tasmin yāṇavāyor nirodhā chaṭakamalāni svayam ūrdhamukhaṃ vikasaṃti |}
\app{\lem[wit={ceteri}]{saptamo}\linelabel{_82b}
  \rdg[wit={B,E}]{\om}}\hfill
\app{\lem[wit={ceteri}, alt={hṛdaya°}]{hṛdaya}
  \rdg[wit={U2}]{hṛdayā°}
  \rdg[wit={B,E}]{\om}
}\app{\lem[wit={J,N2,U1},alt={°rūpādhāraḥ}]{rūpādhāraḥ}
  \rdg[wit={L}]{°rūpadhāraḥ}
  \rdg[wit={D,K1,N1}]{rūpa ādhāraḥ}
  \rdg[wit={U2}]{°dhāraḥ}
  \rdg[wit={B,E}]{\om}}/\\
tasmin-sthāne\hfill
\app{\lem[wit={ceteri},alt={prāṇavāyor}]{prāṇavāyo\skp{r-ni}}
  \rdg[wit={B}]{prāṇavāyo}
}\app{\lem[wit={Y},alt={nirodhāt}]{\skm{r-ni}rodhā\skp{t-ṣa}}
  \rdg[wit={J,K1}]{nirudhanāt}
  \rdg[wit={D,N1,N2,U1}]{nirūṃdhanāt}
}\app{\lem[wit={B,E},alt={ṣaḍ api}]{\skm{t-ṣa}ḍ-api}
  \rdg[wit={J}]{ṣaḍaṃ pi}
  \rdg[wit={ceteri}]{ṣadapi}}\hfill
kamalā\skp{ny-ū}\app{\lem[wit={ceteri},alt={ūrdhvamukhāni}]{\skm{ny-ū}rdhvamukhāni}
  \rdg[wit={D,N1,N2}]{ūrdhvamukhaṃ}
  \rdg[wit={J}]{puttaṣasyordhvamukhaṃ}
  \rdg[wit={U1}]{ūrusyordha mukhaṃ bhavati}}\hfill
vikasanti/\hfill\linelabel{_82e}
%-----------------------------
%aṣṭamaṃ kaṇṭhādhāraḥ/  tatra  jālaṃdharo bandho dīyate/  tasmin satīḍāyāṃ   piṃgalāyāṃ pavanaḥ sthiro bhavati/  \E %%[p.43]
%aṣṭamaḥ kaṃṭhādhāraḥ   tatra  jālaṃdharo baṃdho dīyate   tasmin satīḍāyāṃ   piṃgalāyāṃ pavanaḥ sthiro bhavataḥ  \P
%aṣṭame  kaṇṭhādhāraḥ/  tatra  jalaṃ baṃdho      dīyate   tasmin satīyāṃ     piṃgalāyāṃ pavanaḥ sthiro bhavatī/ \B  %%%%DSCN7166.jpg Z.3
%aṣṭame  kaṇṭhādhāraḥ/  tatra  jalaṃ baṃdho      dīyate   tasmin satīyāṃ     piṃgalāyāṃ pavanaḥ sthiro bhavatī// \L
%aṣṭamaḥ kaṇṭhādhāraḥ/  tatra  jālaṃdharo baṃdho dīyate/  tasmin sati iḍāyāṃ piṃgalāyāṃ pavanaḥ sthiro bhavati/ \N1
%aṣṭamaḥ kaṃṭhādhāraḥ// tatraḥ jālaṃdharo baṃdho dīyate// tasmin sati iḍāyāṃ piṃgalāyāṃ pavanasthiro bhavati// \D  %%%p.12 verso
%aṣṭamaḥ kaṃṭhādhāraḥ// tatra jālaṃdharo baṃdho dīyate// tasmin sati iḍāyāṃ piṃgalāyāṃ pavanasthiro bhavati// \K1
%aṣṭamaḥ kaṃṭhādhāraḥ// tatraḥ jālaṃdharo baṃdho dīyate// tasmin sati iḍāyāṃ piṃgalāyāṃ pavanasthiro bhavati// \J 
%aṣṭama--kaṇṭhādhāraḥ/  tatra  jālaṃdharabandho  dīyate// tasmin satiśadāyāṃ piṃgalāyāṃ pavanaḥ sthiro bhavati/ \N2
%aṣṭamaḥ kaṇṭhādhāraḥ   tatra  jālaṃdharo bandho dīpyate  tasmin sati iḍāyāṃ piṃgalāyāṃ pavanaḥ sthiro bhavati \U1
%aṣṭamaḥ kaṇṭhādhāraḥ   tatra  jālaṃdharo bandho dīyate   tasmin sati piḍāyā piṃgalāyāṃ pavanaḥ sthiro bhavati// \U2
%-----------------------------
%The support of the throat is the eighth. There the binding of Jālaṃdhara is produced. While abiding therein the vital wind in the Iḍā and Piṅgalā channel becomes stable.   
%-----------------------------
\note[type=source, labelb=_83b, labele=_83e, nosep]{cf. YSv (PT, p. 840) =  YK 2.27: kaṇṭhādhāro 'ṣṭamas tatra kaṇṭhasaṅkocalakṣaṇaḥ | jālandharākhyo bandhaḥ syāt tasmin sati marud dṛḍhaḥ |}
\note[type=source, labelb=_83b, labele=_83e, nosep]{cf. SSP 2.17 (Ed. p. 34): aṣṭame kaṇṭhādhāre kaṇṭhamūlaṃ cibukena nirodhayet | iḍāpiṅgalayor vāyuḥ sthiro bhavati |\vspace{-2mm}}
\note[type=testium, labelb=_83b, labele=_83e, nosep]{ \approx  \citetitle{hathasamketacandrikajodhpur} (MMPP 2244 f. 203 ll. 5-6): athāṣṭamakaṃṭhādhāraḥ 8 tatra jālaṃdharabaṃdho dīyate tasmin satīḍāpiṃgalāyāṃ pavanaḥ sthiro bhavati |\vspace{-2mm}}
\app{\lem[wit={ceteri}]{aṣṭamaḥ}\linelabel{_83b}
  \rdg[wit={E}]{aṣṭamaṃ}
  \rdg[wit={B,L}]{aṣṭame}
  \rdg[wit={N2}]{aṣṭama°}}\hfill\\
kaṇṭhādhāraḥ/
\app{\lem[wit={ceteri}]{tatra}
  \rdg[wit={D,J}]{tatraḥ}}
\app{\lem[wit={ceteri}, alt={jālandharo}]{jālandharo}
  \rdg[wit={N2}]{jālaṃdhara°}
  \rdg[wit={B,L}]{jalaṃ}}
bandho
\app{\lem[wit={ceteri}]{dīyate}
  \rdg[wit={U1}]{dīpyate}}/
tasmi\skp{n-sa}\app{\lem[wit={E,P}, alt={satīḍāyāṃ}]{\skm{n-sa}tīḍāyāṃ}
  \rdg[wit={B,L}]{satīyāṃ}
  \rdg[wit={N2}]{satiśadāyāṃ}
  \rdg[wit={ceteri}]{sati iḍāyāṃ}}
piṅgalāyāṃ
\app{\lem[wit={ceteri}]{pavanaḥ}
  \rdg[wit={D,J,K1}]{pavana°}}
sthiro
\vspace{-1mm}\app{\lem[wit={ceteri}]{bhavati}
  \rdg[wit={B,L}]{bhavatī}}/\linelabel{_83e}
    \end{prose}
  \end{edition}
  \begin{translation}
    \begin{tlate}[p30_03]
      \hspace{1em} The fifth is Uḍḍīyāṇa,\footnote{For a discussion of the term \textit{uḍḍīyāṇa}, see p. \pageref{cakra2} n. \ref{udiyana}.} that is, Svādhiṣṭhāna. As a result of performing a lock at that place, faeces and urine disappear.\footnote{\citetitle{shivayogapradipika}, \citetitle{ssplonavla} and \citetitle{yogatarangini} share the concept of binding at Uḍḍīyāna. \citetitle{hathatattvakaumudi} instructs to do a pressing (\textit{moṭana}) at the waist (\textit{kaṭau}). \textit{Netroddyota}, along with \citetitle{saradaavalon} and \citetitle{jyotsna} situate the fifth \textit{ādhāra} at the anus (\textit{pāyu} or \textit{sīvanī}). The \textit{Yogasvarodaya} situates the fifth \textit{ādhāra} at the belly (\textit{jaṭharādhāra}).}
      
      \hspace{1em} The sixth is the support of the navel. There, from the repeated practice of \textit{praṇava},\footnote{The syllable \textit{oṃ}. See \citeauthor{bryant2009} 2009, pp. 105-109 and \citeauthor{harimoto2014} 2014, pp. 151-163 for a more detailed discussion of the term \textit{praṇava} in the context of the \textit{Pātañjalayogaśāstra}.} the unstruck sound\footnote{Cf. \citetitle{hathapradipika2024} 4.17 et seqq.} arises by itself.\footnote{\citetitle{ssplonavla} prescribes reciting \textit{oṃ} at the navel; \citetitle{yogatarangini} adds meditation on the form of consciousness (\textit{cidrūpa}); and \citetitle{hathatattvakaumudi} instructs retaining the breath at the navel, causing the sound of \textit{oṃ} to rise into emptiness. In the \textit{Yogasvarodaya}, breath retention at the navel likewise produces spontaneous \textit{nāda}, while the \citetitle{shivayogapradipika} directs contemplation of Kuṇḍalinī there. The \textit{Netroddyota} identifies the \textit{kanda} as the sixth support.}
      
    %\footnote{\citetitle{ssplonavla} instructs recitation of \textit{oṃ} at the navel, \citetitle{yogatarangini} adds meditation on the form of consciousness (\textit{cidrūpa}) to the same recipe, whereas in \citetitle{hathatattvakaumudi} the breath should be restrained at the navel, which causes the rising of the sound of \textit{oṃ} into emptiness. In the \textit{Yogasvarodaya}, the restraint of breath in the navel causes the \textit{nāda} to arise on its own. However, \citetitle{shivayogapradipika} instructs to contemplate Kuṇḍalinī at the navel. \textit{Netroddyota} lists the \textit{kanda} as the sixth support.}\newline      
      \hspace{1em} The seventh is the support in the form of the heart. The six lotuses [become] upward facing [and] open up from the restraint of the breath in this location.\footnote{Rāmacandra's mention of \textit{ṣaḍ api kamalāny} (``six lotuses'') seems inconsistent, as he previously (section \uproman{29}) taught a ninefold \textit{cakra} system. In the \citetitle{ssplonavla}, \textit{Yogasvarodaya}, and \citetitle{yogatarangini}, practice at the seventh \textit{ādhāra} culminates in the blossoming of the heart lotus alone; in the \citetitle{hathatattvakaumudi}, it is consciousness that blossoms in the heart. The \citetitle{shivayogapradipika} describes the heart centre as a downward-facing eight-petaled lotus bestowing desires, where one should place the mind in the pericarp (\textit{karṇikā}) as a \textit{liṅga} of light. \textit{Netroddyota} identifies the seventh \textit{ādhāra} as the \textit{nāḍi}, the middle path between navel and heart (\textit{nābhihṛnmadhyamārge tu sarvakāmābhidho mataḥ} |), while \citetitle{saradaavalon} and \citetitle{jyotsna} list the navel as the seventh.}
      
% \footnote{Rāmacandra's mention of \textit{ṣaḍ api kamalāny} (``six lotusses'') seems inappropriate, since he previously (section \uproman{29}) taught a ninefold \textit{cakra} system. The result of the practice associated with the seventh \textit{ādhāra} in \citetitle{ssplonavla}, \textit{Yogasvarodaya} and \citetitle{yogatarangini} is confined to the blossoming of the heart lotus alone. In the \citetitle{hathatattvakaumudi}, it is not the heart itself, but consciousness blossoming in the heart. In \citetitle{shivayogapradipika}, the heart centre consists of a downward-facing eight-petaled lotus and is declared to be the bestower of one's desires. Here, one should bring the mind into the pericarp (\textit{karṇikā}) in the form of a \textit{liṅga} of light. \textit{Netroddyota} lists the term \textit{nāḍi} as the seventh \textit{ādhāra}. It is described as the middle path between navel and heart and considered the abode of all desires (\textit{nābhihṛnmadhyamārge tu sarvakāmābhidho mataḥ} |), whereas \citetitle{saradaavalon} and \citetitle{jyotsna} list the navel as the seventh \textit{ādhāra}.}
\hspace{1em} The eighth is the throat support. There, the Jālandhara lock\footnote{This passage shows how Rāmacandra shifts between his two sources. In chapter~\uproman{11}, he places Jālaṅdhara at the \textit{brahmarandhra}; see p.~\pageref{cakra8trans} for a discussion.} is performed. When [the lock] is engaged, the breath in the Iḍā [and] Piṅgalā channels becomes firm.\footnote{The \textit{Netroddyota} locates the support in the belly (\textit{jaṭhara}), while the \citetitle{saradaavalon} and \citetitle{jyotsna} place the eighth support in the heart. The remaining texts agree in this regard.}
     \end{tlate}
  \end{translation}
\end{alignment}
\hardbreak %after pp. 97-98
%%%%%%%%%%%%%%%%%%%%%%%%%%%%%%%%%%%%%%%%%%
%%%%%%%%%%%%%%%%%%%%%%%%%%%%%%%%%%%%%%%%%% 
%%%%%%%%PAGEBREAK%%%%%%%PAGEBREAK%%%%%%%%%
%%%%%%%%%%%%%%%%%%%%%%%%%%%%%%%%%%%%%%%%%% 
%%%%%%%%%%%%%%%%PAGEBREAK%%%%%%%%%%%%%%%%%
%%%%%%%%%%%%%%%%%%%%%%%%%%%%%%%%%%%%%%%%%% 
%%%%%%%%PAGEBREAK%%%%%%%PAGEBREAK%%%%%%%%%
%%%%%%%%%%%%%%%%%%%%%%%%%%%%%%%%%%%%%%%%%% 
%%%%%%%%%%%%%%%%%%%%%%%%%%%%%%%%%%%%%%%%%% 
%%%%%%%%%%%%%%%%%%%%%%%%%%%%%%%%%%%%%%%%%% 
%%%%%%%%%%%%%%%%%%%%%%%%%%%%%%%%%%%%%%%%%% 
%%%%%%%%PAGEBREAK%%%%%%%PAGEBREAK%%%%%%%%%
%%%%%%%%%%%%%%%%%%%%%%%%%%%%%%%%%%%%%%%%%% 
%%%%%%%%%%%%%%%%PAGEBREAK%%%%%%%%%%%%%%%%%
%%%%%%%%%%%%%%%%%%%%%%%%%%%%%%%%%%%%%%%%%% 
%%%%%%%%PAGEBREAK%%%%%%%PAGEBREAK%%%%%%%%%
%%%%%%%%%%%%%%%%%%%%%%%%%%%%%%%%%%%%%%%%%% 
%%%%%%%%%%%%%%%%%%%%%%%%%%%%%%%%%%%%%%%%%% 
%%%%%%%%%%%%%%%%%%%%%%%%%%%%%%%%%%%%%%%%%% 
%%%%%%%%%%%%%%%%%%%%%%%%%%%%%%%%%%%%%%%%%% 
%%%%%%%%PAGEBREAK%%%%%%%PAGEBREAK%%%%%%%%%
%%%%%%%%%%%%%%%%%%%%%%%%%%%%%%%%%%%%%%%%%% 
%%%%%%%%%%%%%%%%PAGEBREAK%%%%%%%%%%%%%%%%%
%%%%%%%%%%%%%%%%%%%%%%%%%%%%%%%%%%%%%%%%%% 
%%%%%%%%PAGEBREAK%%%%%%%PAGEBREAK%%%%%%%%%
%%%%%%%%%%%%%%%%%%%%%%%%%%%%%%%%%%%%%%%%%% 
%%%%%%%%%%%%%%%%%%%%%%%%%%%%%%%%%%%%%%%%%%
\begin{alignment}[
  texts=edition[class="edition"];
  translation[class="translation"],
  ]\textappfiddle[.2]
  \begin{edition}
    \begin{prose}[p30_04]
      \noindent
%-----------------------------
%navamo ghaṃṭikādhāraḥ/   tatra jihvāgraṃ   lagnaṃ bhavati/    tato mṛtakalāyā     amṛtaṃ sravati/  tadamṛtapānāt             śarīramadhye rogasaṃcāro na bhavati/ \E
%navamo ghaṭikādhāraḥ     tatra jihvāgraṃ   lagnaṃ bhavati     tato mṛtakakalāyā   amṛta  sravati   tadamṛtapānāc            charīramadhye rogasaṃcāro na bhavati  \P
%navo   ghaṃṭikādhāraḥ//  tatra jihvāgraṃ   lagnaṃ bhavatī/    tato mṛtakalāyā     amṛtaṃ sravati/  tadamṛtakalāyāṃ amṛtapānīcharīramadhye rogasaṃcāro bhavatī/ \B
%navamo ghaṃṭādhāraḥ//    tatra jihvāgraṃ   lagnaṃ bhavati//   tato mṛtakalāyāṃ                        amṛtapānā-------------charīramadhye rogasaṃcāro bhavati// \L %eyeskip in line.. :(
%navamo ghaṃṭikādhāraḥ/   tatra jihvāgraṃ   lagnaṃ bhavati/    tato mṛtakalāyā     amṛtaṃ sravati/  tadamṛtapānāt             śarīramadhye rogasaṃcāro na bhavati/ \N1
%navamo ghaṃṭikādhāraḥ//  tatra jihvāyāgraṃ lagnaṃ bhavati//   tataḥ amṛtakalāyāḥ  amṛtaṃ sravati// tadamṛtapānāc           charīramadhye  rogasaṃcāro na bhavati// \D
%navamo ghaṃṭikādhāraḥ//  tatra jihvāyāgraṃ lagnaṃ bhavati//   tataḥ amṛtakalāyāḥ  amṛtaṃ sravati// tadamṛtapānāc           charīramadhye  rogasaṃcāro na bhavati// \K1
%navamo ghaṃṭikādhāraḥ  tatra jihvāyāgraṃ lagnaṃ bhavati//cha//   tataḥ amṛtakalāyāḥ  amṛtaṃ sravati// tadamṛtapānāc           charīramadhye  rogasaṃcāro na bhavati// \J      
%navamo ghaṃṭikādhāraḥ/   tatra jihvāgraṃ   lagnaṃ bhavati/    tato mṛtakalāyā     amṛtaṃ sravati/  tadamṛtapānāt             śarīramadhye rogasaṃcāro na bhavati/ \N2
%navamo ghaṃṭikādhāras    tatra juhvāyāṃ    lagnaṃ bhavati vā  tataḥ amṛtakalāyāḥ  amṛtaṃ sravati   tadamṛtapānāt            charīramadhye rogasaṃcāro na bhavati \U1
%navamo ghaṃṭikādhāraḥ    tatra jihvāgraṃ   lagnaṃ bhavati//   tato mṛtakalāyāḥ    amṛtaṃ sravati// tadamṛtapānā             charīramadhye rogasaṃcāro na bhavati// \U2
%-----------------------------
%The ninth is the support of the uvula. There the tip of the tongue becomes attached [to the uvula]. Because of that the nectar of immortality flows from the immortality digit. From drinking the nectar of immortality diseases do not spread in the body. 
%-----------------------------
\note[type=source, labelb=_84b, labele=_84e, nosep]{cf. YSv (PT, p. 840): navamo ghaṇṭikādhāras tatra jihvāgram agrataḥ (\textit{jihvāgrataḥ kṛte} YK 2.28) | sampivaty amṛtaṃ tasmād yogajinmṛtyujitparaḥ |}
\note[type=source, labelb=_84b, labele=_84e, nosep]{cf. SSP 2.18 (Ed. p. 35): navame ghaṇṭikādhāre jihvāgraṃ dhārayet | amṛtakalā sravati |}
\note[type=testium, labelb=_84b, labele=_84e, nosep]{ \approx  \citetitle{hathasamketacandrikajodhpur} (MMPP 2244 f. 98v ll. 6-8): atha navamaṃ ghaṃṭikādhāraḥ 9 tatra jihvāyā agraṃ dataṃ cet tata uparītaḥ amṛtaṃ yat sravati | taj jihvāgreṇa yogī pibati | tadāmṛtapānāc charīramadhye rogāṇāṃ saṃcāro na bhavati |}
\app{\lem[wit={ceteri}]{navamo}\linelabel{_84b}
  \rdg[wit={B}]{navo}}\hfill
\app{\lem[wit={ceteri},alt={ghaṇṭikā°}]{ghaṇṭikā}
  \rdg[wit={P}]{ghaṭikā°}
  \rdg[wit={L}]{ghaṃṭā°}
}\app{\lem[wit={ceteri},alt={°dhāraḥ}]{dhāraḥ}
  \rdg[wit={U1}]{dhāras}}\hfill
tatra\hfill
\app{\lem[wit={ceteri}]{jihvāgraṃ}
  \rdg[wit={D}]{jihvāyāgraṃ}
  \rdg[wit={U1}]{juhvāyāṃ}}
lagnaṃ\hfill
\app{\lem[wit={ceteri}]{bhavati}
  \rdg[wit={B}]{bhavatī}
  \rdg[wit={U1}]{bhavati vā}
  \rdg[wit={J}]{bhavati || cha ||}}/\hfill
\app{\lem[wit={ceteri}]{tato}
  \rdg[wit={D,J,K1,N1,U1}]{tataḥ}
}\app{\lem[wit={ceteri}]{'mṛtakalāyā}
  \rdg[wit={L}]{mṛtakalāyāṃ}
  \rdg[wit={D,J,K1,U1}]{amṛtakalāyāḥ}}\hfill
\app{\lem[wit={ceteri}]{amṛtaṃ}
  \rdg[wit={P}]{amṛta}
  \rdg[wit={L}]{\om}}\hfill
\app{\lem[wit={ceteri}, alt={sravati}]{srava:\\ti}
  \rdg[wit={L}]{\om}}/\hfill%\vspace{-0.5mm}
\app{\lem[wit={D,J,K1,P},alt={tadamṛtapānāc}]{tadamṛtapānā\skp{c-cha}}
  \rdg[wit={E,N1,N2,U1}]{tadamṛtapānāt}
  \rdg[wit={B}]{tadamṛtakalāyāṃ amṛtapānī°}
  \rdg[wit={L}]{amṛtapānā}
  \rdg[wit={U2}]{tadamṛtapānā}
}\app{\lem[wit={ceteri},alt={charīra°}]{\skm{c-cha}rīra}
  \rdg[wit={E,N1,N2}]{śarīra°}
}madhye\hfill 
rogasaṃcāro\hfill
\app{\lem[wit={ceteri}]{na}
  \rdg[wit={B,L}]{\om}}\hfill
\app{\lem[wit={ceteri}]{bhavati}
  \rdg[wit={B}]{bhavatī}}/\hfill\linelabel{_84e}      
%-----------------------------
%daśamaṃ tālvādhāraḥ/  tanmadhye    vānaṃ dollahanaṃ      ca kṛtvā              laṃbikāpraveśe sati    tāluni magnā jihvā tiṣṭhati/ \E
%daśamas tālvādhāraḥ   tanmadhye  cālanaṃ dohanaṃ         ca kratvā             laṃbikāpraveśe śe sati tālumagnā    jihvā tiṣṭhati  \P %%%7654.jpg
%daśamaṃ stālvādhāraḥ/ tanmadhye  cālanaṃ dohanaṃ         ca kratvā             laṃbikāpraveśe sati    tālumagnā    jihvā tiṣṭhati/ \B
%daśamas tālvādhāraḥ// tanmadhye  cālanaṃ dohanaṃ         ca kṛtvā              laṃbikāpraveśe sati    tālumagnā    jihvā tiṣṭhati ... \L
%daśama  tālvādhāraḥ// tanmadhye  cānanaṃ dohanaṃ         ca kṛtvā              laṃbikāpraveśe grati   tāluni magnā jihvā tiṣṭhati/ \N1
%daśamas tālvādhāraḥ   tanmadhye  cānanaṃ dohanaṃ         ca kṛtvā              laṃbikāpravese grati   tāluni magnā jihvā tiṣṭhati// \D
%daśamas tālvādhāraḥ   tanmadhye  cālanaṃ dohanaṃ         ca kṛtvā              laṃbikāpravese grati   tāluni magnā jihvā tiṣṭhati// \K1
%daśamas tālvādhāras   tanmadhye  cālanaṃ dohanaṃ         ca kṛtvā              cālaṃbikāpravese sati   tāluni lagnā jihvā tiṣṭhati// \J
%daśama  tālvādhāraḥ   tanmadhye  cālanaṃ dohanaṃ         ca kṛtvā              laṃbikāpraveśe grati   tālūni magnā                    \N2
%daśamas tālvādhāraḥ  staṃnmadhye cālanaṃ dohanaṃ         ca sva/sca? kṛtvā cālaṃ vikā praveśe sati    tāluni lagnā juhvā tiṣṭhati \U1 %%%284.jpg
%daśamas tālvādhāraḥ   tanmadhye  cālanaṃ dohanaṃ chedanaṃ ca kṛtvā             laṃbikāpraveśe sati    tāluni magnā jihvā tiṣṭhati// \U2 %%416.jpg
%-----------------------------
%The tenth is the support of the palate. After the moving and milking has been done therein, [and] while abiding at entrance of the uvula, the tongue resides inserted within the palate.
%-----------------------------
\note[type=source, labelb=_85b, labele=_85e, nosep]{cf. YSv (PT, p. 840): daśamas tālukādhāras tatra jihvāgrataḥ kṛte (hemistich omitted in YK) | calane dohane caiva jihvā jaḍati lambitā (\textit{jāyeta lambitam} YK 2.28cd) | nāsikāprāptajihveyaṃ tālulagnā bhavet tataḥ |}
\note[type=source, labelb=_85b, labele=_85e, nosep]{cf. SSP 2.19 (Ed. p. 35): daśame tālvādhāre tālvantar garbhe lambikāṃ cālanadohanābhyāṃ dīrghīkṛtvā viparītena praveśayet | kāṣṭhībhavati |}
\note[type=testium, labelb=_85b, labele=_85e, nosep]{ \approx  \citetitle{hathasamketacandrikajodhpur} (MMPP 2244 f. 98v l. 8): atha daśamaṃ (\textit{daśamaṃ} GOML R3239 ] \textit{damaṃ} MMPP 2244) tālvādhāraḥ 10 spaṣṭaṃ |}
\app{\lem[wit={ceteri},alt={daśamas}]{daśama\skp{s-tā}}
  \rdg[wit={B}]{daśamaṃs}
  \rdg[wit={E}]{daśamaṃ}
  \rdg[wit={N1,N2}]{daśama}
}\skm{s-tā}lvādhāraḥ/\linelabel{_85b}
\app{\lem[wit={ceteri}]{tanmadhye}
  \rdg[wit={U1}]{staṃnmadhye}}\\
\app{\lem[wit={ceteri}, alt={cālanaṃ}]{cālanaṃ}
  \rdg[wit={D}]{cānanaṃ}
  \rdg[wit={E}]{vānaṃ}}\hfill
\app{\lem[wit={ceteri}]{dohanaṃ}
  \rdg[wit={E}]{dollahanaṃ}
  \rdg[wit={U2}]{dohanaṃ chedanaṃ}}\hfill
ca \app{\lem[wit={ceteri}]{kṛtvā}
  \rdg[wit={B,L}]{kratvā}
  \rdg[wit={U1}]{sva kṛtvā}}\hfill
\app{\lem[wit={ceteri}, alt={lambikā°}]{lambikā}
  \rdg[wit={J}]{cālaṃbikā°}
  \rdg[wit={U1}]{cālaṃ vikā°}
}praveśe\hfill
\app{\lem[wit={ceteri}]{sati}
  \rdg[wit={P}]{śe sati}
  \rdg[wit={D,K1,N1,N2}]{grati}}\hfill
\app{\lem[wit={ceteri}]{tāluni magnā}
  \rdg[wit={N2}]{tālūni lagnā}
  \rdg[wit={J,U1}]{tāluni lagnā}
  \rdg[wit={B,P,L}]{tālumagnā}}\hfill
\app{\lem[wit={ceteri}]{jihvā}
  \rdg[wit={N2}]{\om}}\hfill
\app{\lem[wit={ceteri},alt={tiṣṭhati}]{tiṣṭhati}
  \rdg[wit={N2}]{\om}}/\linelabel{_85e}\hfill      
%-----------------------------
%ekādaśo           jihvādhāraḥ/  tasmin   jihvāgreṇa manthanaṃ kriyate   tasmin  kṛte   timadhuraṃ  pānīyaṃ sravati/  tadā                            ca kavitva------cchandonāṭakādiviṣayajñānam utpadyate/ \E
%ekādaśo jihvātale jihvādhāraḥ   tasmin   jihvāgreṇa mathanaṃ  kriyate   tasmin  kṛte   timadhuraṃ  pānīyaṃ sravati   tathā                           ca kavitva------chaṃdonāṭakādiviṣayajñānam  utpadyate  \P
%ekādaśo jihvātale jihvādhāraḥ// tasmin   jihvāgreṇa manthanaṃ kṛtvā//   tasmiṃ  kṛte satimadhuraṃ  pānīyaṃ sravatī// tathā                              kvacitva-----cchaṃdonāṭakādiviṣayapānam  utpadyaṃte/ \B
%ekādaśo jihvātale jihvādhāraḥ// tasmin   jihvāgreṇa mathanaṃ  kṛtvā//   tasmiṃ  kṛte satimadhuraṃ  pānīyaṃ sravati// tathā                              kvacitva-----chaṃdonāṭakādiviṣayajñānam  utpadyate// \L
%ekādaśo           jihvādhāraḥ/  tasmin   jihvāgreṇa manthanaṃ kriyate/  tasmin  kṛte atimadhuraṃ   pānīyaṃ sravati/  tathā                           ca kavitva--gītacchaṃdanāṭakādiviṣaye jñānam utpadyate/ \N1
%ekādaśo jihvātale jihvādhāraḥ// tasmin   jihvāgreṇa mathanaṃ  kriyate// tasmiṃ  kṛte satimadhuraṃ  pānīyaṃ sravati// tathā                           ca kvacitta-----chaṃdanāṭakādiviṣayajñānam   utpadyate \D
%ekādaśoḥ jihvātale jihvādhāraḥ// tasmin   jihvāgreṇa mathanaṃ  kriyate// tasmiṃ  kṛte atimadhuraṃ  pānīyaṃ sravati// tathā                           ca kavitvagīta-----chaṃdanāṭakādiviṣaye jñānam   utpadyate \K1
%ekādaśa jihvātale jihvādhāraḥ// tasmin   jihvāgreṇa mathanaṃ  kriyate// tasmiṃ  kṛte satimadhuraṃ  pānīyaṃ sravati// tathā                           ca kavitvagītachaṃdanātikādiviṣayaṃ jñānam   utpadyate \J (P18/34) 
%                                         jihvāgreṇa manthanaṃ kriyate// tasmin  kṛte atimadhuraṃ   pānīyaṃ sravati// kaminnāsikā phatkāravat// tathā ca kavitvagīta--chaṃdanāṭakādiviṣaye jñānam  utpadyate/ \N2
%ekādaśā jihvātale jihvādhāraḥ  tasmin na jihvāgreṇa manthanaṃ kriyate   tasminn kṛte  timadhuraṃ   pānīyaṃ sravati   tathā                           ca kavitvagīta--chaṃdavacchaṃdanāḍīviṣayaṃ jñānānam utpadyate \U1
%ekādaśo jihvātale jihvādhāraḥ   tasmin   jihvāgreṇa manthanaṃ kriyate// tasminn kṛte 'timadhuraṃ   pānīyaṃ sravati// tathā                           ca kavitvaṃ     chaṃdonāṭakādiviṣayajñānam  utpadyate// \U2
%-----------------------------
%The eleventh is the tongue support at the base of the tongue. Therein the tip of the tongue has to be churned. While doing that, an very sweet drink flows out. And in that manner the knowledge of areas like poetry, singing, metric and dance is generated. 
%----------------------------
\note[type=source, labelb=_86b, labele=_86e, nosep]{cf. YSv (PT, p. 840): ekādaśī (\textit{ekādaśo} YK 2.29) bhavej jihvā talajādhāra īśvari | jihvāgramathane tasmin pānīyaṃ madhuraṃ bhavet | tatpīteṣu kavir gītijyotiś (\textit{gītir} YK 2.29) chandovidāṃ (\textit{chandovidur} YK 2.30) varaḥ |}
\note[type=source, labelb=_86b, labele=_86e, nosep]{cf. SSP 2.20 (Ed. p. 35): ekādaśe atha jihvādhāre tatra jihvāgraṃ dhārayet | sarvaroganāśo bhavati |\vspace{-2mm}}
\note[type=testium, labelb=_86b, labele=_86e, nosep]{ \approx  \citetitle{hathasamketacandrikajodhpur} (MMPP 2244 f. 98 ll. 8-9): ekādaśo jihvātale jihvādhāraḥ 11 tasmin jihvāgreṇa mathanaṃ kriyate | tasmin kṛte atimadhuraṃ pānīyaṃ sudhāvat sravati | kavitvagītachaṃdanāṭakādijñānaṃ svayam utpadyate |\vspace{-2mm}}
\app{\lem[wit={ceteri}]{ekādaśo}\linelabel{_86b}
  \rdg[wit={J}]{ekādaśa}
  \rdg[wit={N2}]{\om}}\\
\app{\lem[wit={ceteri}, alt={jihvātale}]{jihvātale}
  \rdg[wit={E,N1,N2}]{\om}}
\app{\lem[wit={ceteri}]{jihvādhāraḥ}
  \rdg[wit={N2}]{\om}}/
\app{\lem[wit={ceteri}, alt={tasmin}]{tasmin}
  \rdg[wit={U1}]{tasmin na}
  \rdg[wit={N2}]{\om}}
jihvāgreṇa%\vspace{-0.5mm}
\app{\lem[wit={ceteri}]{manthanaṃ}
  \rdg[wit={D,J,K1,L,P}]{mathanaṃ}}
\app{\lem[wit={ceteri}]{kriyate}
  \rdg[wit={B,L}]{kṛtvā}}/
tasmin-kṛte\app{\lem[wit={ceteri}, alt={'ti°}]{'ti}
  \rdg[wit={N1,N2}]{ati°}
  \rdg[wit={B,D,J,L}]{sati°}
}madhuraṃ pā:\\nīyaṃ
\app{\lem[wit={ceteri}]{sravati}
  \rdg[wit={B}]{sravatī}}/ 
\app{\lem[wit={ceteri}]{tathā}
  \rdg[wit={E}]{tadā}
  \rdg[wit={N2}]{kamin nāsikā phatkāravat || tathā}}
\app{\lem[wit={ceteri}]{ca}
  \rdg[wit={B,L}]{\om}}
\app{\lem[wit={ceteri},alt={kavitva°}]{kavitva}
  \rdg[wit={B,L}]{kvacitva°}
  \rdg[wit={D}]{kvacitta°}
  \rdg[wit={U2}]{kavitvaṃ}
}\app{\lem[wit={N1,N2,U1},alt={°gīta°}]{gīta}
  \rdg[wit={ceteri}]{\om}
}\app{\lem[wit={Y},alt={°chando°}]{chando}
  \rdg[wit={U1}]{°chaṃdavacchaṃda°}
  \rdg[wit={ceteri}]{°chaṃda°}
}\app{\lem[wit={ceteri},alt={°nāṭakādi°}]{nāṭakādi}
  \rdg[wit={J}]{nātikādi}
  \rdg[wit={U1}]{°nāḍī°}
}\app{\lem[wit={Y,D},alt={°viṣaya°}]{viṣaya}
a  \rdg[wit={K1,N1,N2}]{°viṣaye}
  \rdg[wit={J,U1}]{viṣayaṃ}
}\app{\lem[wit={ceteri},alt={jñānam}]{jñāna\skp{m-u}}
  \rdg[wit={U1}]{jñānānam}
}\app{\lem[wit={ceteri},alt={utpadyate}]{\skm{m-u}tpadyate}
  \rdg[wit={B}]{utpadyaṃte}}/\linelabel{_86e}
    \end{prose}
  \end{edition}
  \begin{translation}
    \ekddiv{type=trans}
    \begin{tlate}[p30_04]
      \hspace{1em} The ninth is the support of the uvula. The tip of the tongue becomes attached to it. As a result of that, the nectar of immortality flows from the immortality digit. From drinking the nectar of immortality, diseases do not spread in the body.\footnote{Most texts with the sixteen \textit{ādhāra} system share this concept. Only \citetitle{saradaavalon} and \citetitle{jyotsna} situate the ninth support at the neck (\textit{grīva}), and \textit{Netroddyota} at the heart.}     
      
      \hspace{1em} The tenth is the support of the palate. After the moving and milking have been done, [and] after abiding at the entrance with the tongue in the middle of it, the tongue resides inserted within the [cavity above the] palate.\footnote{The ninth, tenth, eleventh and twelfth supports are all associated with the tongue-related hatḥayogic \textit{khecarīmudrā} and its forerunners. For a detailed account of this \textit{khecarīmudrā}, see \citeauthor[2010]{mallinson2010}. \textit{Netroddyota} places the tenth support at the tortoise channel (\textit{kūrmanāḍī}), whereas \citetitle{saradaavalon} and \citetitle{jyotsna} situates it at the throat (\textit{kaṇṭha}).}
      
      \hspace{1em} The eleventh is the tongue support at the surface of the tongue. In the middle of that [support], the tip of the tongue is churned.\footnote{For a discussion of the term \textit{manthana} in the context of \textit{khecarīmudrā} see \citeauthor[2010: 207-208, n. 250]{mallinson2010}.} When that has been done, a very sweet liquid oozes out. Moreover, after that, the knowledge of areas like poetry, singing, metrics and dance is generated.\footnote{Almost all texts teaching the sixteen \textit{ādhāra}s share the concept of the churning of the tongue with just minor differences: \citetitle{ssplonavla} teaches the destruction of all diseases (\textit{sarvaroganāśa}) as the result of this practice, \citetitle{yogatarangini} calls the practice \textit{jihvādhobhāgādhāra}. The \textit{Netroddyota} alone teaches the throat (\textit{kaṇṭha}) as the eleventh \textit{ādhāra}. Here, it states: \textit{lambhikasya sthitaś cordhve sudhādhāraḥ sudhātmakaḥ} || ``Above the place of the uvula is a stream of nectar resembling nectar itself.''.}%\flushpage 
      \end{tlate}
  \end{translation}
\end{alignment}
\hardbreak % after pp. 99-100
%%%%%%%%%%%%%%%%%%%%%%%%%%%%%%%%%%%%%%%%%%
%%%%%%%%%%%%%%%%%%%%%%%%%%%%%%%%%%%%%%%%%% 
%%%%%%%%PAGEBREAK%%%%%%%PAGEBREAK%%%%%%%%%
%%%%%%%%%%%%%%%%%%%%%%%%%%%%%%%%%%%%%%%%%% 
%%%%%%%%%%%%%%%%PAGEBREAK%%%%%%%%%%%%%%%%%
%%%%%%%%%%%%%%%%%%%%%%%%%%%%%%%%%%%%%%%%%% 
%%%%%%%%PAGEBREAK%%%%%%%PAGEBREAK%%%%%%%%%
%%%%%%%%%%%%%%%%%%%%%%%%%%%%%%%%%%%%%%%%%% 
%%%%%%%%%%%%%%%%%%%%%%%%%%%%%%%%%%%%%%%%%% 
%%%%%%%%%%%%%%%%%%%%%%%%%%%%%%%%%%%%%%%%%% 
%%%%%%%%%%%%%%%%%%%%%%%%%%%%%%%%%%%%%%%%%% 
%%%%%%%%PAGEBREAK%%%%%%%PAGEBREAK%%%%%%%%%
%%%%%%%%%%%%%%%%%%%%%%%%%%%%%%%%%%%%%%%%%% 
%%%%%%%%%%%%%%%%PAGEBREAK%%%%%%%%%%%%%%%%%
%%%%%%%%%%%%%%%%%%%%%%%%%%%%%%%%%%%%%%%%%% 
%%%%%%%%PAGEBREAK%%%%%%%PAGEBREAK%%%%%%%%%
%%%%%%%%%%%%%%%%%%%%%%%%%%%%%%%%%%%%%%%%%% 
%%%%%%%%%%%%%%%%%%%%%%%%%%%%%%%%%%%%%%%%%% 
%%%%%%%%%%%%%%%%%%%%%%%%%%%%%%%%%%%%%%%%%% 
%%%%%%%%%%%%%%%%%%%%%%%%%%%%%%%%%%%%%%%%%% 
%%%%%%%%PAGEBREAK%%%%%%%PAGEBREAK%%%%%%%%%
%%%%%%%%%%%%%%%%%%%%%%%%%%%%%%%%%%%%%%%%%% 
%%%%%%%%%%%%%%%%PAGEBREAK%%%%%%%%%%%%%%%%%
%%%%%%%%%%%%%%%%%%%%%%%%%%%%%%%%%%%%%%%%%% 
%%%%%%%%PAGEBREAK%%%%%%%PAGEBREAK%%%%%%%%%
%%%%%%%%%%%%%%%%%%%%%%%%%%%%%%%%%%%%%%%%%% 
%%%%%%%%%%%%%%%%%%%%%%%%%%%%%%%%%%%%%%%%%%
\begin{alignment}[
  texts=edition[class="edition"];
  translation[class="translation"],
  ]
  \begin{edition}
    \begin{prose}[p30_05]\noindent
%----------------------------
%tadupari dvādaśadantayo   madhye   dantādhāraḥ/  tasmin sthāne jihvāyā  agraṃ  ghaṭīmātraṃ                    balātkāreṇa  sthāpyate/  tasmin  sati sādhakasya samagrā rogā naśyanti// \E %%%[p.44]
%tadupari dvādaśo daṃtayor madhye   daṃtādhāraḥ   tasmin sthāne jihvāyā  agraṃ  ghaṭīmātram ārghaghaṭīmātraṃ   bālātkāreṇa  sthāpyate   tasmin  sati sādhakasya samagrā rogā naśyaṃti \P
%tadupari dvādaśo daṃtayor madhye// daṃtādhāraḥ// tasmin sthāne jihvāyā  'agnaṃ ghaṭīmātram ārghaghaṭimātraṃ   bālākāreṇa   sthāpyate// tasmiṃ       sādhakasya samagrā rogā naśyaṃtī// \B
%tadupari dvādaśo daṃtayor madhye// daṃtādhāraḥ   tasmin sthāne jihvāyā  agnaṃ  ghaṭīmātram ārddhaghaṭimātraṃ  bālākāreṇa   sthāpyate// tasmiṃ       sādhakasya samagrā rogā naśyaṃti... \L
%tadupari dvādaśayor       madhye   daṃtādhāraḥ/  tasmin sthāne jihvāyā  agraṃ  ghaṭīmātraṃ arddhaghaṭimātraṃ  balātkāreṇa  sthāpyate// tasmin  sati sādhakasya samagrā rogā naśyaṃti// \N1
%tadupari dvādaśayor       madhye   daṃtādhāraḥ// tasmin sthāne jihvāyā  agraṃ  ghaṭīmātraṃ arddhaghaṭimātraṃ  balātkāreṇa  sthāpyate// tasmin  sati sādhakasya samagrā rogā naśyaṃti// \D
%tadupari dvādaśayo daṃtayor madhye dantādhāraḥ// tasmin sthāne jihvāyā  agraṃ  ghaṭimātraṃ arddhaghaṭimātraṃ va  balātkāreṇa  sthāpyate// tasmin  sati sādhakasya samagrā rogā naśyaṃti// \K1
%tadupari dvādaśavaṃ tayor madhye   daṃtādhāras tasmin sthāne jihvāyā  agraṃ  ghaṭimātraṃ arddhaghaṭimātraṃ  balātkāreṇa  sthāpyate// tasmin  sati sādhakasya samagrā rogā naśyaṃti// \J      
%tadupari dvādaśayor       madhye   daṃtādhāraḥ// tasmin sthāne jihvāyā   graṃ  ghaṭīmātraṃ arddhaghaṭimātraṃ  balātkāreṇa  sthāpyate// tasmin  sati sādhakasya samagrā rogā naśyanti \N2
%tadupari dvādaśo daṃtayor madhye   daṃtādhāraḥ   tasmin sthāne jihvāyāṃ agraṃ  ghaṭīmātram ārdhaghaṭikāmātraṃ bālātkāreṇa  sthāpyate   tasminn sati sādhakasya samagra rogā naśyaṃti \U1
%tadupari dvādaśor daṃtayo madhye   daṃtādhāraḥ   tasmin sthāne jihvāyā  agraṃ  ghaṭīmātram ārghaghaṭīmātraṃ   bālātkāreṇa  sthāpyate// tasmin  sati sādhakasya samagrā rogā naśyaṃti// \U2
%-----------------------------
%Above that is the twelfth - within the teeth is the tooth support. At this place the tip of the tongue is to be positioned with force for the duration of one and a half \textit{ghāṭī}s (24+12 = 36 minutes). Abiding therein the diseases of the practitioner will entirely disappear!
%----------------------------
\note[type=source, labelb=_87b, labele=_87e, nosep]{cf. YSv (PT, p. 840): dantādhāro (\textit{dvandvādhāro} YK 2.31a) dvādaśeti sarvarogakṣayaṅkaraḥ (\textit{sarvarogaḥ} YK 2.31b) | dhārayed dantayor madhye jihvāgrañ ca balād api | dhṛtvārddhaghaṭikāmātraṃ sarvarogan (\textit{sarvarogāṃs} YK 2.32b) tu nāśayet |}
\note[type=source, labelb=_87b, labele=_87e, nosep]{cf. SSP 2.21 (Ed. p. 36): dvādaśe bhrūmadhyādhāre tatra candramaṇḍalaṃ dhyāyet śītalatāṃ yāti |}
\note[type=testium, labelb=_87b, labele=_87e, nosep]{ \approx  \citetitle{hathasamketacandrikajodhpur} (MMPP 2244 f. 98r l. 9 - 99v l. 1): atha tadupari dvādaśo daṃtayor madhye daṃtādhāraḥ 12 tasmin sthāne jihvāyā agraṃ ghaṭīmātraṃ ardhaghaṭīmātraṃ balāt sthāpyate | tasmin sati samagraroganāśo bhavati |}\linelabel{_87b}tadupari
\app{\lem[wit={B,K1,L,P,U1}, alt={dvādaśo dantayor}]{dvādaśo dantayor madhye}
  \rdg[wit={E}]{dvādaśadantayo madhye}
  \rdg[wit={U2}]{dvādaśor daṃtayo madhye}
  \rdg[wit={D,N1,N2}]{dvādaśayor madhye}
  \rdg[wit={J}]{dvādaśavaṃ tayor madhye}}
\app{\lem[wit={ceteri}]{dantādhāraḥ}
  \rdg[wit={J}]{dantādhāras}}/
tasmin-sthāne
\app{\lem[wit={ceteri}]{jihvāyā}
  \rdg[wit={U1}]{jihvāyāṃ}}
\app{\lem[wit={ceteri}]{agraṃ}
  \rdg[wit={B,L}]{agnaṃ}
  \rdg[wit={N2}]{graṃ}}
\app{\lem[wit={ceteri},alt={ghaṭīmātram}]{ghaṭīmātra:\\\skp{m-a}}
  \rdg[wit={D,N1,N2}]{ghaṭīmātraṃ}
}\app{\lem[type=emendation, resp=egoscr, alt={ardhagaṭīmātraṃ}]{\skm{m-a}rdhagaṭīmātraṃ}
  \rdg[wit={D,J,K1,N1,N2}]{arddhaghaṭimātraṃ}
  \rdg[wit={U1}]{ārdhaghaṭikāmātraṃ}
  \rdg[wit={P,U2}]{ārghaghaṭīmātraṃ}
  \rdg[wit={B}]{ārghaghaṭimātraṃ}
  \rdg[wit={L}]{ārddhaghaṭimātraṃ}
  \rdg[wit={E}]{\om}}\hfill
\app{\lem[type=emendation, resp=egoscr]{vā}
  \rdg[wit={K1}]{va}
  \rdg[wit={ceteri}]{\om}}\hfill
\app{\lem[wit={ceteri}]{balātkāreṇa}
  \rdg[wit={P,U1,U2}]{bālātkāreṇa}
  \rdg[wit={B,L}]{bālākāreṇa}}\hfill
sthāpyate/\hfill
\app{\lem[wit={ceteri}, alt={tasmin}]{tasmi\skp{n-sa}}
  \rdg[wit={B,L}]{tasmiṃ}
}\app{\lem[wit={ceteri}, alt={sati}]{\skm{n-sa}ti}
  \rdg[wit={B,L}]{\om}}
sādhakasya\hfill samagrā\hfill rogā\hfill
\app{\lem[wit={ceteri}, alt={naśyanti}]{na:\\śyanti}
  \rdg[wit={B}]{naśyaṃtī}}/\linelabel{_87e}
%----------------------------
%trayodaśo nāsikāgrādhāraḥ/ tasmin lakṣye kṛte sati manaḥ sthiraṃ bhavati/ \E
%trayodaśo nāsikāgrādhāraḥ  tasmiṃ lakṣye kṛte sati manaḥ sthiraṃ bhavati \P
%trayodaso nāsikādhāraḥ/    tasmin ḍraṣṭe kṛte      minasthire    bhavati/ \B
%trayodaso nāśikādhāraḥ     tasmin ḍraṣṭe kṛte      manaḥ sthiro  bhavati/ \L
%trayodaśo nāsikādhāraḥ/    tasmin lakṣe  kṛte sati manasthiraṃ   bhavati/ \N1
%trayodaśo nāsikādhāraḥ//   tasmin lakṣe  kṛte sati manasthiraṃ   bhavati \D
%trayodaśo nāsikādhāraḥ//   tasmin lakṣe  kṛte sati manasthiraṃ   bhavati \K1
%trayodaśo nāsikādhāraḥ//   tasmin lakṣye sati kṛte manaḥ sthiraṃ   bhavati// \J
%trayodaśo nāsikādhāraḥ/    tasmin lakṣe  kṛte sati manasthiraṃ   bhavati/ \N2
%trayodaśo nāsikādhāraḥ     tasmiṃ lakṣye kṛte sati manasthiraṃ   bhavati \U1
%trayodaśo nāsikādhāraḥ     tasmil lakṣe  kṛte sati manasthiraṃ   bhavati// \U2
%-----------------------------
%The thirteenth is the support of the nose. While turning it into the object of fixation the mind becomes stable. 
%----------------------------
\note[type=source, labelb=_88b, labele=_88e, nosep]{cf. YSv (PT, p. 832): nāsādhāras tato (\textit{tataḥ} YK 2.32b) jñeyo nāsālakṣas trayodaśaḥ (\textit{trayodaśa} YK 2.32d) | manaḥsthirakaro yas tu (\textit{sthiraṃ karoty eva} YK 2.33a) vāyusthirakaro (\textit{vāyuḥ} YK 2.32b) mahān |}
\note[type=source, labelb=_88b, labele=_88e, nosep]{cf. SSP 2.22 (Ed. p. 36): trayodaśe nāsādhāre tasyāgraṃ lakṣayet manaḥ sthiraṃ bhavati |}
\note[type=testium, labelb=_88b, labele=_88e, nosep]{ \approx  \citetitle{hathasamketacandrikajodhpur} (MMPP 2244 f. 99r l. 1-2): atha trayodaśo nāsikādhāraḥ 13 tasmin lakṣye kṛte sati manaḥ sthiraṃ bhavati |}
trayodaśo\linelabel{_88b}
\app{\lem[wit={ceteri}]{nāsikādhāraḥ}
  \rdg[wit={E,P}]{nāsikāgrādhāraḥ}}/
\app{\lem[type=emendation, resp=egoscr, alt={tasmil lakṣye}]{tasmil\skp{-}lakṣye}
  \rdg[wit={U2}]{tasmil lakṣe}
  \rdg[wit={E,P,U1}]{tasmiṃ lakṣye}
  \rdg[wit={J}]{tasmin lakṣye}
  \rdg[wit={D,K1,N1,N2}]{tasmin lakṣe}
  \rdg[wit={B,L}]{tasmin ḍraṣṭe}}
\app{\lem[wit={ceteri}]{kṛte sati}
  \rdg[wit={J}]{sati kṛte}
  \rdg[wit={B,L}]{\om}}
\app{\lem[wit={E,J,P}]{manaḥ sthiraṃ}
  \rdg[wit={B}]{minasthire}
  \rdg[wit={L}]{manaḥ sthiro}
  \rdg[wit={ceteri}]{manasthiraṃ}}
bhavati/\linelabel{_88e}
%----------------------------
%caturdaśo nāsāmūlādhāraḥ/         tasmin dṛṣṭeḥ            sthairyakāraṇāt   ṣaṣṭhe māsi svīyan tejaḥ pratyakṣaṃ bhavati/  tejasaḥ pratyakṣatve pārthivaṃ sakalaṃ bandhanaṃ tuṭyati/   \E
%caturdaśo nāsāmūlādhāro           tasmin dṛṣṭeḥ            sthairyakāraṇāt   ṣaṣṭhe māsi svīyaṃ tejaḥ pratyakṣaṃ bhavati   tejasaḥ pratyakṣatve pārthivaṃ sakalaṃ baṃdhanaṃ truṭyati/ \P %%%7654.jpg vorletzte Zeile
%caturdaśo nāso mūlādhāraḥ//       tasmin llakṣe krute satī sthairyakāraṇāt// ṣaṣṭhe māse svayaṃ tejaḥ pratyakṣaṃ bhavati// tejasaḥ pratyakṣatve pārthivaṃ sakalaṃ baṃdhanaṃ truṭayati/ \B
%caturdaśo nāso mūlādhāraḥ         tasmin lakṣe kṛte satī   sthairyakāraṇāt   ṣaṣṭhe māse svayaṃ tejaḥ pratyakṣaṃ bhavati// tejasaḥ pratyakṣatve pārthivaṃ sakalaṃ baṃdhanaṃ truṭayati/ \L
%caturdaśo nāsāmūle vāyvādhāraḥ/   tasmin dṛṣṭeḥ            sthairyakāraṇāt   ṣaṣṭhe māsi svīyaṃ tejaḥ pratyakṣaṃ bhavati/  tejasaḥ pratyakṣatve pārthivaṃ sakalaṃ baṃdhanaṃ trudyati/  \N1
%caturdaśo nāsāmūle vāyvādhāraḥ//  tasmin dṛṣṭeḥ            sthairyakāraṇāt   ṣaṣṭhe māsi svīyaṃ tejaḥ pratyakṣaṃ bhavati// tejasaḥ pratyakṣatve pārthivaṃ sakalaṃ baṃdhanaṃ trudyati// \D  %%%p.13 recto
%caturdaśo nāsāmūle vāyādhāraḥ//  tasmin dṛṣṭeḥ            sthairyakāraṇāt   ṣaṣṭhe māsi / svīyaṃ tejaḥ pratyakṣaṃ bhavati//    tejasaḥ pratyakṣatve pārthivaṃ sakalaṃ baṃdhanaṃ trudyati// \K1
%caturdaśo nāsāmūle vā adhāraḥ//  tasmin dṛṣṭeḥ            sthairyakāraṇāt//   ṣaṣṭhe māsaḥ svīyaṃ tejaḥ// pratyakṣaṃ bhavati// tejasaḥ pratyakṣatve pārthivaṃ sakalaṃ baṃdhanaṃ trudyati// \J
%caturdaśo nāsāmūle vāyvādhāraḥ??/ tasmin dṛṣṭeḥ            sthairyakāraṇāt   ṣaṣṭhe māsi svayaṃ tejaḥ pratyakṣaṃ bhavati   tejasaḥ pratyakṣatve pārthiva  sakalaṃ bandhanaṃ trudyati// \N2
%caturdaśo nāsāmūle vādhāraḥ       tasmiṃ na dṛṣṭeḥ         sthairyakāraṇāt   ṣaṣṭhe māse svīyaṃ tejaḥ pratyakṣaṃ bhavati   tejasaḥ pratyakṣatve pārthivaṃ sakalaṃ baṃdhanaṃ truṭyati   \U1
%caturdaśo nāsāmūlādhāraḥ          tasmin laṣṭhe?           sthairyakāraṇāt   ṣaṣṭhe māsi svayaṃ tejaḥ pratyakṣaṃ bhavati// tejasaḥ pratyakṣatve pārthivaṃ sakalaṃ baṃdhanaṃ truṭyati// \U2
%-----------------------------
%The fourteenth is the support of the vital wind at the root of the nose. From the execution of stabilizing the gaze therein, direct perception of one's own light arises within sixty months. He breaks all bonds of the mundane world in the direct perception of the light. 
%-----------------------------
\note[type=source, labelb=_89b, labele=_89e, nosep]{cf. YSv (PT, p. 839) =  YK 2.33ab-34cd): nāsāpuṭe sthirā dṛṣṭir ādhāro 'yaṃ caturdaśaḥ | kṛte 'smin svīyatejaḥ syāt pratyakṣaṃ ṣaṭtrimāsataḥ | pārthivaṃ truṭati kṣipraṃ pratyakṣaṃ svīyatejasā |}
\note[type=source, labelb=_89b, labele=_89e, nosep]{cf. SSP 2.23 (Ed. p. 36): caturdaśe nāsāmūle kapāṭādhāre dṛṣṭiṃ dhārayet | ṣaṇmāsāj jyotiḥpuñjaṃ paśyati |}
\note[type=testium, labelb=_89b, labele=_89e, nosep]{ \approx  \citetitle{hathasamketacandrikajodhpur} (MMPP 2244 f. 99r ll. 2-3): atha caturdaśo nāsāmūle lalāṭe 'py ādhāraḥ 14 tasmin dṛṣṭeḥ sthairyakaraṇāt ṣaṣṭhe māsi svīyaṃ tejaḥ pratyakṣaṃ bhavati | tejasaḥ prasakṣatve pārthivasaṃkalaṃ baṃdhanaṃ trudyati |}
caturdaśo \app{\lem[wit={D,N1,N2}, alt={nāsāmūle vāyvādhāraḥ}]{nāsāmūle vāyvādhāraḥ}\linelabel{_89b}
  \rdg[wit={K1}]{nāsāmūle vāyādhāraḥ}
  \rdg[wit={J}]{nāsāmūle vā adhāraḥ}
  \rdg[wit={U1}]{nāsāmūle vādhāraḥ}
  \rdg[wit={P}]{nāsāmūlādhāro}
  \rdg[wit={B,L}]{nāso mūlādhāraḥ}
  \rdg[wit={E,U2}]{nāsāmūlādhāraḥ}}
\app{\lem[wit={ceteri}, alt={tasmin}]{tasmi\skp{n-dṛ}}
  \rdg[wit={ceteri}]{tasmiṃ na}
}\app{\lem[wit={ceteri}, alt={dṛṣṭeḥ}]{\skm{n-dṛ}ṣṭeḥ}
  \rdg[wit={B}]{llakṣe krute satī}
  \rdg[wit={L}]{lakṣe kṛte satī}
  \rdg[wit={U1}]{na dṛṣṭeḥ}
  \rdg[wit={U2}]{laṣṭhe}}
sthairyakāraṇāt-ṣaṣṭhe
\app{\lem[wit={B,L,U1}]{māse}
  \rdg[wit={J}]{māsaḥ}
  \rdg[wit={ceteri}]{māsi}}
\app{\lem[wit={ceteri}]{svīyaṃ}
  \rdg[wit={B,L,N2,U2}]{svayaṃ}}
tejaḥ pra:\\tyakṣaṃ bhavati/
tejasaḥ pratyakṣatve
\app{\lem[wit={ceteri}]{pārthivaṃ}
  \rdg[wit={N2}]{pārthiva}}
bandhanaṃ 
\app{\lem[wit={P,U2,U1}, alt={truṭyati}]{truṭyati}
  \rdg[wit={E}]{tuṭyati}
  \rdg[wit={B,L}]{truṭayati}
  \rdg[wit={D,J,K1,N1,N2}]{trudyati}}/\linelabel{_89e}
    \end{prose}
  \end{edition}
  \begin{translation}
    \begin{tlate}[p30_05]
      \hspace{1em} Above that is the twelfth, the tooth support within the two [top front] teeth.\footnote{SSP 2.21 (Ed. p. 36) teaches the brows as the twelfth \textit{ādhāra}. Rāmacandra decided to stick to the YSv. Given the other descriptions, it is apparent that Rāmacandra switched between both sources when compiling the section on the sixteen \textit{ādhāra}s.} At this place, the tip of the tongue is to be positioned with force for the duration of one or half a \textit{ghaṭi}s\footnote{One \textit{ghaṭi} equals 1/60 of a day (cf. \citeauthor[1966: 114]{sircar1966}), which is 24 minutes. Half a \textit{ghaṭi} would thus equal 12 minutes}. Abiding therein, the diseases of the practitioner will entirely disappear.\footnote{Most of the texts teach a practice that involves contact between the tongue and the teeth. Rāmacandra and \textit{Yogasvarodaya} teach to push the tongue forcefully against the [upper] teeth. \citetitle{shivayogapradipika} instructs to rub the tip of the tongue at the upper teeth for half a year, which would cause the practitioner to see an inner light. \citetitle{hathatattvakaumudi} mixes the two previous ideas. The name of the twelfth \textit{ādhāra} here is \textit{dvijādhāra}, and Sundaradeva also calls it \textit{rājadanta}. The yogin presses the tip of the tongue against this point, and hence, he perceives an inner light within six months. \citetitle{yogatarangini} surprisingly teaches the same technique as Rāmacandra and not the \textit{bhrūmadhyādhāra} of \citetitle{ssplonavla}. \textit{Netroddyota} names the palate as the twelfth support and lets us know that at the root of it is that which is blissful, which is enveloped by the \textit{somakalā}. \citetitle{saradaavalon} and \citetitle{jyotsna} list the nose as the twelth support.}
      
      \hspace{1em} The thirteenth is the support of the nose. When that is set as the focus, the mind becomes stable.\footnote{The majority of texts teach either the nose, the base of the nose as in \citetitle{shivayogapradipika} (\textit{ghrāṇamūla}) and \citetitle{hathatattvakaumudi} (\textit{ghrāṇapada}), or the tip of the nose (\textit{nāsāgra}) as the \citetitle{ssplonavla} and \citetitle{yogatarangini}. Whereas \textit{Netroddyota}, \citetitle{saradaavalon} and \citetitle{jyotsna} teach the place in between the brows as the thirteenth \textit{ādhāra}.}
      
      \hspace{1em} The fourteenth is the support of the vital wind at the bridge of the nose. As a result of stabilizing the gaze therein, one's own brilliance becomes apparent within six months. When the brilliance has manifested the mundane bond breaks.\footnote{\textit{Yogasvarodaya} and \citetitle{ssplonavla} provide the term \textit{kapāṭādhāra}. \citetitle{yogatarangini} teaches the base of the nose as the fourteenth \textit{ādhāra}. All other texts teach fixing the mind and the breath at the forehead. \textit{Netroddyota} calls this place at the forehead ``a wish-fulfilling jewel with its abode at the crossroads of the four channels'' (\textit{cintāmaṇyabhidhānākhyaś catuṣpathanivāsi yat}).}               
      \flushpage 
    \end{tlate}
  \end{translation}
\end{alignment}
\hardbreak %after pp. 101-102
%%%%%%%%%%%%%%%%%%%%%%%%%%%%%%%%%%%%%%%%%%
%%%%%%%%%%%%%%%%%%%%%%%%%%%%%%%%%%%%%%%%%% 
%%%%%%%%PAGEBREAK%%%%%%%PAGEBREAK%%%%%%%%%
%%%%%%%%%%%%%%%%%%%%%%%%%%%%%%%%%%%%%%%%%% 
%%%%%%%%%%%%%%%%PAGEBREAK%%%%%%%%%%%%%%%%%
%%%%%%%%%%%%%%%%%%%%%%%%%%%%%%%%%%%%%%%%%% 
%%%%%%%%PAGEBREAK%%%%%%%PAGEBREAK%%%%%%%%%
%%%%%%%%%%%%%%%%%%%%%%%%%%%%%%%%%%%%%%%%%% 
%%%%%%%%%%%%%%%%%%%%%%%%%%%%%%%%%%%%%%%%%% 
%%%%%%%%%%%%%%%%%%%%%%%%%%%%%%%%%%%%%%%%%% 
%%%%%%%%%%%%%%%%%%%%%%%%%%%%%%%%%%%%%%%%%% 
%%%%%%%%PAGEBREAK%%%%%%%PAGEBREAK%%%%%%%%%
%%%%%%%%%%%%%%%%%%%%%%%%%%%%%%%%%%%%%%%%%% 
%%%%%%%%%%%%%%%%PAGEBREAK%%%%%%%%%%%%%%%%%
%%%%%%%%%%%%%%%%%%%%%%%%%%%%%%%%%%%%%%%%%% 
%%%%%%%%PAGEBREAK%%%%%%%PAGEBREAK%%%%%%%%%
%%%%%%%%%%%%%%%%%%%%%%%%%%%%%%%%%%%%%%%%%% 
%%%%%%%%%%%%%%%%%%%%%%%%%%%%%%%%%%%%%%%%%% 
%%%%%%%%%%%%%%%%%%%%%%%%%%%%%%%%%%%%%%%%%% 
%%%%%%%%%%%%%%%%%%%%%%%%%%%%%%%%%%%%%%%%%% 
%%%%%%%%PAGEBREAK%%%%%%%PAGEBREAK%%%%%%%%%
%%%%%%%%%%%%%%%%%%%%%%%%%%%%%%%%%%%%%%%%%% 
%%%%%%%%%%%%%%%%PAGEBREAK%%%%%%%%%%%%%%%%%
%%%%%%%%%%%%%%%%%%%%%%%%%%%%%%%%%%%%%%%%%% 
%%%%%%%%PAGEBREAK%%%%%%%PAGEBREAK%%%%%%%%%
%%%%%%%%%%%%%%%%%%%%%%%%%%%%%%%%%%%%%%%%%% 
%%%%%%%%%%%%%%%%%%%%%%%%%%%%%%%%%%%%%%%%%%
\begin{alignment}[
  texts=edition[class="edition"];
  translation[class="translation"],
  ]
  \begin{edition}
    \begin{prose}[p30_06]\noindent
%----------------------------
%pañcadaśo bhruvormadhyādhāras        tasmin dṛṣṭeḥ sthirīkaraṇāt    koṭikiraṇāḥ  sphuraṃti/ \E
%paṃcadaśo bhruvormadhyādhāraḥ        tasmin dṛṣṭeḥ sthirīkaraṇāt    koṭikiraṇāḥ  sphuraṃti  \P  %%%7655.jpg
%paṃcadaśo bhruvormadhye dhāraḥ//     tasmin dṛṣṭeḥ sthirikaraṇāt//  koṭikiriṇā   sphuraṃti// \B
%paṃcadaśo bhruvormadhye dhāraḥ//     tasmin dṛṣṭe  sthirīkaraṇāt//  koṭikiriṇā   sphuraṃti// \L
%pañcadaśo bhruvormadhye ādhāraḥ/      asmin dṛṣṭeḥ sthirīkaraṇāt    koṭikiraṇāni sphuraṃti/ \N1
%pañcadaśo bhruvormadhye ājñādhāraḥ// ..smin dṛṣṭeḥ sthirīkaraṇāt    koṭikiraṇāni sphuraṃti// \D
%pañcadaśo bhruvormadhye ādhāraḥ!// asmin dṛṣṭeḥ sthirīkaraṇāt    koṭikiraṇāni sphuraṃti// /K1
%pañcadaśa bhruvormadhye adhāraḥ// assmin dṛṣṭeḥ sthirīkaraṇāt//    koṭikiraṇāni sphuraṃti// \J      
%pañcadaśo bhruvormadhye ādhāraḥ      tasmin dṛṣṭeḥ sthirīkaraṇāt    koṭikiraṇāni sphuraṃti/ \N2 [S.9]
%pañcadaśo bhruvormadhye ādhāra         asin na dṛṣṭeḥ sthirīkaraṇāt koṭikiraṇāni sphuraṃti \U1
%pañcadaśo bhruvormadhyādhāra         tasmin dṛṣṭisthirīkaraṇāt      koṭikiraṇaḥ  sphuraṃti// \U2
%-----------------------------
%The fifteenth container is situated in the middle of the eyebrows. Due to stabilized the gaze therein, ten million rays of light sparkle. 
%----------------------------
\note[type=source, labelb=_90b, labele=_90e, nosep]{cf. YSv (PT, p. 839): pañcadaśo bhruvormadhye sthira (\textit{sthirā} YK 2.35) dṛṣṭis tathā dhruvam | asmin dṛṣṭiḥ sthirā koṭiḥ (\textit{koṭi°} YK 2.35) kiraṇāni sphuranti hi |}
\note[type=source, labelb=_90b, labele=_90e, nosep]{cf. SSP 2.24 (Ed. pp. 36-37): pañcadaśe lalāṭādhāre tatra jyotiḥpuñjaṃ lakṣayet | tejasvī bhavati |}
\note[type=testium, labelb=_90b, labele=_90e, nosep]{ \approx  \citetitle{hathasamketacandrikajodhpur} (MMPP 2244 f. 99r l. 3-4): atha paṃcadaśo bhrūmadhye ajñādhāraḥ 15 asmiṃ dṛṣṭeḥ sthirīkaraṇāt koṭikiraṇāḥ puraḥ sphuraṃti |}
pañcadaśo\linelabel{_90b}\hfill
\app{\lem[wit={P},alt={bhruvor madhyādhāraḥ}]{bhruvor\skp{-}madhyādhāraḥ}
  \rdg[wit={E}]{bhruvor madhyādhāras}
  \rdg[wit={B,L}]{bhruvor madhye dhāraḥ}
  \rdg[wit={D}]{bhruvor madhye ājñādhāraḥ}
  \rdg[wit={J,K1,N1,N2}]{bhruvor madhye ādhāraḥ}
  \rdg[wit={U1}]{bhruvor madhye ādhāra}
  \rdg[wit={U2}]{bhruvor madhyādhāra}}/\hfill
\app{\lem[wit={ceteri}, alt={tasmin}]{tasmi\skp{n-dṛ}}
  \rdg[wit={N1}]{asmin}
  \rdg[wit={D}]{smin}
  \rdg[wit={U1}]{asin}
}\app{\lem[wit={ceteri}, alt={dṛṣṭeḥ}]{\skm{n-dṛ}ṣṭeḥ}
  \rdg[wit={L}]{dṛṣṭe}
  \rdg[wit={U1}]{na dṛṣṭeḥ}
  \rdg[wit={U2}]{dṛṣṭi°}}\hfill
sthirīkaraṇāt-koṭi\app{\lem[wit={X}]{kiraṇāni}
  \rdg[wit={E,P}]{koṭikiraṇāḥ}
  \rdg[wit={U2}]{koṭikiraṇaḥ}
  \rdg[wit={B,L}]{koṭikiriṇā}}\hfill
sphuranti/\linelabel{_90e}\\
%----------------------------
%ṣoḍaśo  netrādhāraḥ/  ayam aṃgulyagreṇa cālyate/  tadabhyāsāt/ pṛthvīmadhye  yatkiṃcin  tejo  varttate/  \E   %%%p.45
%ṣoḍaśo  netrādhāraḥ   ayam aṃgulyagreṇa cālyate   tadabhyāsāt  pṛthvīmadhye  yatkiṃcit  tejo  vartate... \P
%ṣoḍaśo  netrā//       ayam aṃgulyagreṇa cālyate// tadabhyāsāt  pṛthivīmadhye yatkiṃcit  tejo  vartate//  \B %%%%%%%%%%%%%%%%DSCN7167.jpg Z. 1
%ṣoḍaśo  netrā//       ayam aṃgulyagreṇa cālyate// tadabhyāsāt  pṛthivīmadhye yatkiṃcit  tejo  vartate... \L
%ṣoḍaśaḥ netrādhāraḥ/  ayaṃ agulyagreṇa  cālyate/  tadabhyāsāt  pṛthvīmadhye  yatkiṃcit  tejaḥ varttate/  \N1
%ṣoḍaśaḥ netrādhāraḥ// ayaṃ agulyagreṇa  cālyate// tadabhyāsāt  pṛthvīmadhye  yatkiṃcit  tejaḥ varttate    \D
%ṣoḍaśaḥ netrādhāraḥ// ayaṃ agulyagreṇa  cālyate// tadabhyāsāt  pṛthvīmadhye  yatkiṃcit  tejaḥ varttate    \K1
%ṣoḍaśo  ne  ādhāraḥ   ayaṃ aṃgulyagreṇa  cālyate// tad abhyāsāt//  pṛthvīmadhye  yatkiṃcit varttate//    \J
%ṣoḍaśaḥ netrādhāraḥ/  ayaṃ aṃgugreṇa    cālyate/  tadabhyāsāt  pṛthvīmadhye  yatkiṃcit  tejaḥ varttate/  \N2
%ṣoḍaśo  netrādhāraḥ   ayaṃ aṃgulyagreṇa cālyate   tadābhyāsāt  pṛthvīmadhye  yatkiṃcit        vatate     \U1 %%%%%%%%%%%%%%%%%%285.jpg
%ṣoḍaśo  netrādhāraḥ   ayam aṃgulyagreṇa cālyate// tadabhyāsāt  pṛthivīmadhye yatkiṃcit// tejo vartate//  \U2
%-----------------------------
%[If the gaze] is held at the tip of the finger without wavering, this is the eye support, the sixteenth. Through that practice, some light arises in the middle of the earth.
%The sixteeth, the eye support [is when the gaze] is [held] at the tip of the finger without wavering.  
%-----------------------------
\note[type=source, labelb=_91b, labele=_91e, nosep]{cf. YSv (PT, pp. 840-41): netrādhāraḥ ṣoḍaśo 'yam (\textit{aṅgulyagre na} YK 2.36) aṅgulyagreṇa cālayet | pṛthvīmadhye tu yat kiñcid varttate (\textit{sarvajñaḥ prabhavas tena varddhate} YK 2.36) jaṭharānalaḥ | pratyakṣaṃ tad bhavet sarvaṃ tad ābhyāsān na saṃśayaḥ |}
\note[type=source, labelb=_91b, labele=_91e, nosep]{cf. SSP 2.25 (Ed. p. 37): avaśiṣṭe ṣoḍaśe brahmarandhram ākāśacakram | tatra śrīgurucaraṇāmbujayugmaṃ sadāvalokayet | ākāśavat pūrṇo bhavati |}
\note[type=testium, labelb=_91b, labele=_91e, nosep]{ \approx  \citetitle{hathasamketacandrikajodhpur} (MMPP 2244 f. 99r l. 4): atha ṣoḍaśo netrādhāraḥ 16 ayaṃ aṃgulyagreṇa cālyate tadābhyāsāt pṛthivīmadhye yat kiñcit tejo vartate | tat sarvaṃ tejo dṛṣṭiviṣayaṃ bhavati | taddarśanāt puruṣaḥ sarvajño bhavati | iti pūrvoktaṣoḍaśādhārāṇāṃ spaṣṭo 'rthaḥ |}
\app{\lem[wit={ceteri}, alt={ṣoḍaśo}]{ṣoḍaśo}\linelabel{_91b}
  \rdg[wit={D,K1,N1,N2}]{ṣoḍaśaḥ}}
\app{\lem[wit={ceteri}, alt={netrādhāraḥ}]{netrādhāraḥ}
  \rdg[wit={J}]{ne ādhāraḥ}
  \rdg[wit={B,L}]{netrā}}/
\app{\lem[wit={Y},alt={ayam}]{aya\skp{m-a}}
  \rdg[wit={X}]{ayaṃ}
}\app{\lem[wit={ceteri}, alt={aṅgulyagreṇa}]{\skm{m-a}ṅgulyagreṇa}
  \rdg[wit={D,J,N1}]{agulyagreṇa}
  \rdg[wit={N2}]{aṃgugreṇa}}
cālyate/
tadabhyāsā\skp{t-pṛ}\app{\lem[wit={ceteri},alt={pṛthvī°}]{\skm{t-pṛ}thvī}
  \rdg[wit={L,B,U2}]{pṛthivī°}}madhye
yatkiṃci\skp{t-te}\app{\lem[wit={ceteri}, alt={tejo}]{\skm{t-te}jo}
  \rdg[wit={D,N1,N2}]{tejaḥ}
  \rdg[wit={J,U1}]{\om}}
\app{\lem[wit={ceteri}]{vartate}
  \rdg[wit={U1}]{vatate}}/
%----------------------------
%tatsarvaṃ tejo   dṛṣṭiviṣayaṃ bhavati/  taddarśanāt  puruṣaḥ sarvajño  bhavati// \E
%tatsarvaṃ tejo   dṛṣṭiviṣayaṃ bhavati   tadarśanāt   puruṣaḥ sarvajño  bhavati     \P
%tatsarvaṃ tejo   dṛṣṭiviṣayaṃ bhavatī// taddarśanāt  puruṣaḥ sarvajño  bhavatī// \B
%tatsarvaṃ tejo   dṛṣṭiviṣayaṃ bhavati// taddarśanāt  puruṣaḥ sarvajño  bhavati// \L
%tatsarvvatejo    dṛṣṭiviṣayaṃ bhavati   taddarśanāt  puruṣaḥ sarvvajño bhavati// \N1
%tatsarvatejo     dṛṣṭiviṣayaṃ bhavati   taddarśanāt  puruṣaḥ sarvvajño bhavati// \D
%tatsarvatejo     dṛṣṭiviṣayaṃ bhavati   taddarśanāt  puruṣaḥ sarvvajño bhavati// \K1
%tatsarvaṃ tejo   dṛṣṭiviṣayaṃ bhavati// taddarśanaḥ  puruṣaḥ sarvvajño bhavati// \J
%tatsarvatejo     dṛṣṭiviṣayaṃ bhavati   taddarśanāt  puruṣaḥ sarvajño  bhavati// \N2
%tatsarvaṃ tejo   dṛṣṭīviṣayaṃ bhavati   tatdarśaḥ    puruṣaḥ sarvajño  bhavati \U1
%tatsarvaṃ tajaso dṛṣṭiviṣayaṃ bhavati// taddarśanāt  puruṣaḥ sarvajño  bhavati// \U2
%-----------------------------
%The entire light of it becomes the object of vision. Through its perception, a person becomes all-knowing.
%-----------------------------
\app{\lem[wit={D,N1,N2}]{tatsarvatejo}
  \rdg[wit={ceteri}]{tatsarvaṃ tejo}}
dṛṣṭiviṣayaṃ
\app{\lem[wit={ceteri}]{bhavati}
  \rdg[wit={B}]{bhavatī}}/ 
\app{\lem[wit={ceteri}, alt={taddarśanāt}]{taddarśanā\skp{t-pu}}
  \rdg[wit={P}]{tadarśanāt}
  \rdg[wit={J,U1}]{tatdarśaḥ}}\skm{t-pu}ruṣaḥ
sarvajño 
\app{\lem[wit={ceteri}]{bhavati}
  \rdg[wit={B}]{bhavatī}}\dd{}\linelabel{_91e}
    \end{prose}
  \end{edition}
  \begin{translation}
    \begin{tlate}[p30_06]
  \hspace{1em} The fifteenth support is situated in the middle of the eyebrows. As a result of stabilizing the gaze therein, ten million rays of light sparkle.\footnote{\citetitle{shivayogapradipika} teaches gazing above the brows, which quickly brings about the appearance of light. \citetitle{ssplonavla} calls it the ``support of the forehead'' (\textit{lalāṭādhāra}), in which the practitioner shall visualize a cluster of light. \citetitle{yogatarangini} teaches the centre of the brows. By concentrating on this point, a direct vision of many-rayed light occurs, and one’s mind will merge into the sun-sky (\textit{etasya dṛḍhābhyāse sūryākāśo līyate} |). \citetitle{hathatattvakaumudi} calls it the ``support of ether'' (\textit{vyomādhāra}) and explains that by gazing at it, everything is perceived as light. However, \textit{Netratantra} teaches the \textit{brahmarandhra} as the fifteenth support. \textit{Netroddyota} declares it as the ``support of the fourth state'' (\textit{turyādhāra}). \citetitle{saradaavalon} and \citetitle{jyotsna} also teach the top of the head (\textit{mūrdhan}) as the fifteenth.}
      
  \hspace{1em} The sixteenth is the eye support. It is caused to be rubbed with the fingertips. As a result of that practice, some light arises from the earth[-element].\footnote{Perhaps, \textit{tejas} arises from \textit{pṛthvī}, because its origin is unknown and in Śaiva Tantras the earth as the bottom \textit{tattva} contains the entire \textit{brahmāṇḍa}, cf. \citetitle[2013: 501]{tantrika3}.} That entire light becomes the object of vision. As a result of seeing that, the person becomes omniscient.\footnote{Rāmacandra's description of \textit{netrādhāra} is very similar to \citetitle{yogatarangini}, which also instructs the yogin to rub the eyes with the fingers in order to generate the perception of a light. Other texts have some noteworthy differences: \citetitle{shivayogapradipika} teaches to fix [the gaze] above the eyes. Due to that, the yogin sees a mass of light in the corner of his eyes. \citetitle{hathatattvakaumudi} teaches to meditate upon the eyes. By seeing a mass of light in the corner of the eyes, one soon becomes like Śiva. \citetitle{ssplonavla} teaches to visualize the pair of the lotus feet of the revered teacher (\textit{śrīgurucaraṇāmbujayugmaṃ}) at the \textit{brahmarandhra} in which the \textit{ākāśacakra} is situated. The \citetitle{jyotsna}, too, lists the \textit{brahmarandhra}. \citetitle{saradaavalon}\nocite{avalon} and \textit{Netratantra} teach the \textit{dvādaśānta} as the sixteenth support, cf. \citetitle{tantrika3}, p. 210. \textit{Netroddyota} explains: \textit{nāḍyādhāraḥ paraḥ sūkṣmo ghanavyāptiprabodhakaḥ} || ``The support of the [central?] channel is the highest subtle one which awakens complete pervasion.''}
      \flushpage 
    \end{tlate}
  \end{translation}
\end{alignment}
\hardbreak %after pp. 101-102

\newpage
\selectlanguage{english}
\chapter{Appendix}
\section{Figures}
 
% \begin{landscape}
\clearpage

  \begin{figure}[ht]
	\centering
  \includegraphics[width=1\textwidth]{pics/Wolpertinger.png}
\caption[The \textit{dehasvarūpa} of \textit{ajapāgāyatrī}]{The \textit{dehasvarūpa} of \textit{ajapāgāyatrī}. The image, reminiscent of a hippogriff, is part of an illustrated Sanskrit manuscript written in the Śāradā script. Preserved as a single large scroll under Acc. No. 1334 at the Oriental Institute in Srinagar (Kashmir), it is entitled \textit{Nāḍīcakra}. The manuscript contains a depiction of the yogic body’s \textit{cakra}s and \textit{nāḍī}s. The text surrounding the figure closely corresponds to the additional material found in manuscript \getsiglum{U2} of the \textit{Tattvayogabindu}. The manuscript reads (diplomatic transcription): \textit{oṃ daśame pūrṇagiripīṭhe lalāṭamaṇḍale candro devatā amṛtāśaktiḥ paramātmā ṛṣiḥ dvāviṃśaddalāni amṛtavāsinikalā 4: ambikā 1 lambikā 2 gha(ṃ)ṭkā 3 tālikā 4 dehasvarūpaṃ kākamukhaṃ 1 naranetraṃ 2 gośṛṅgaṃ 3 lalāṭabrahmapara 4 hayagrīvā 5 mayūramuśchaṃ 6 haṃsacārītani 7 sthāna.}}
	\phantomsection\label{fig_wolpertinger}
      \end{figure}

      \clearpage

  \begin{figure}[ht]
	\centering
  \includegraphics[width=1\textwidth]{pics/Vishnu_Vishvarupa_cropped.jpg}
	\caption{Viṣṇu Viśvarūpa, India, Rajasthan, Jaipur, ca. 1800–1820, Opaque watercolor and gold on paper, 38.5 × 28 cm, Victoria and Albert Museum, London, Given by Mrs. Gerald Clark.}
	\label{fig1}
      \end{figure}
\clearpage
  \begin{figure}[ht]
	\centering
  \includegraphics[width=0.5\textwidth]{pics/The_Equivalence_of_Self_and_Universe_(detail),_folio_6_from_the_Siddha_Siddhanta_Paddhati,_(Bulaki),_1824_(Samvat_1881);_122_x_46_cm._Mehrangarh_Museum_Trust..jpg}
	\caption{The Equivalence of Self and Universe (detail), folio 6 from the \textit{Siddhasiddhāntapaddhati} (Bulaki), India, Rajasthan, Jodhpur, 1824 (Samvat 1881), 122 x 46 cm, RJS 2378, Mehragarh Museum Trust.}
	\label{fig2}
      \end{figure}
      % \end{landscape}

      \newpage
      \cleardoublepage
\chapter{Bibliography}
 \label{sec:bibli}
\clearpage
\newpage 
\thispagestyle{empty}
\quad  \addtocounter{page}{-1}

\newrefcontext[sorting=tixel]
\printbibliography[heading=subbibintoc, title=Primary Sources, keyword=primary]

\newrefcontext[sorting=nyt]
\printbibliography[heading=subbibintoc, title=Secondary Literature, keyword=seclit]

\printbibliography[heading=subbibintoc, title=Catalogues, keyword=catalogues]

\printbibliography[heading=subbibintoc, title=Online Sources, keyword=onlinesource]

\end{document}


%%% Local Variables:
%%% mode: latex
%%% TeX-master: t
%%% End:
