%Ultimatives Tool zur Datierung:
%https://www.cc.kyoto-su.ac.jp/~yanom/pancanga/
%skp = ignored in edition
%skm = ignored in xml
%%%---2-DO---%%%:
% - add xml ids for cladistics
% - produce diplomatic transcripts for saktumiva
% - make extra layer in Apparatus for parallels in SVARODAYA, Siddhasiddhantapaddhati and Amanaska
% - check all daṇḍas!!! now I think that it's more likely that many of them were lost in copies. Lectio difficilior! Very unconventional style of the autor! 
% - read Sarvangayogapradipika, Maya Burger! 
% - maybe add second ciritical edition of yogasvarodaya?!
% - Korrekturlesen von \E!! 
% - Verspattern einbauen!
% - add all Testtimonia of SSP & Ysv
% - Sigla alphabetisch ordnen und! daṇḍas mit einkollationieren
% - präambel auslagern wie Jürgen
% - grep-search alle Verse!!!!
% - Mss spreadsheet
% - sort N1,D1,B2 zu N1,N2,D1
% - sort all sigla alphabetically 
% - additions to U2: make footnotes for the bahir mātrā-s: explaining the inventions of female deities and tell that this is "schwer interpretierbar"
% - Belege für source und testimonia einfügen!!!
% - GIVE UNIQUE LABELS for TESTIMONIO AND SOurces
% - Edition mit Sätzen übereinander nennt sich: Synoptische Edition
% - Consider changing Lakṣya to Lakṣa
% - vEREINHEITLICHUNG von source und testium! 
%%%%%%%%%%%%%%%%%%%%%%%%%%%%%%%%%%%%%%%%%
% Don't forget
% Siddhasiddhantapaddhati Yogic Body descriptions are followed by Rāmacandra
% Quotes of the Yogasvarodaya in the Yoga Karṇikā
% Rāmacandra more a compiler than an author!!!
% Identify quotes of YTB in Haṭhasanketacandrikā 
%%%%%%%%%%%%%%%%%%%%%%%%%%%%%%%%%%%%%%%%%%%
%MSS notes
%
%--B: i and ī are not differenciated
%--P: no punctuation no daṇdas nothing
%--U1: dot . serves as daṇḍa 
%--\L and \U2 very similar
%--figure out for U2: // ajapājapaḥ sahasra // 6000 //gha 0 16 pa 0 40// \U2?!?!?!?!?!?
%%%%%%%%%%%%%%%%%%%%%%%%%%%%%%%%%%%%%%%%%%
%
% Einleitung Ideen 
% - sprachliche Simplizität
% - Potenzial als Anfängertext
% - Großartige Einführung in die Textkritik -> Synoptische Edition 
% - Gelegenheit Yogasvarodaya und Yogatattvabindu zu edieren 
% - Historische Evidenz entweder für das königliche Leben in einer Maṭha in der Nähe von Benares während der Muslimischen Herrschaft, oder sogar Lehrtext für die Bildung junger Prinzen  
% - eines der raren Beispiele der engen Verknüfung mehrerer Texte 
% - eines der raren Beispiele der Prosaisierung eines metrischen Textes 
% - Anwendung rezenter Technologie! 
% - How the text was construed -> intermingling of Ysv and SSP
% - Martin Straube: "jeder kleine Dorfhäuptling kann Rāja genannt werden". 
%%%%%%%%%%%%%%%%%%%%%%%%%%%%%%%%%%%%%%%%%%%
%Ich habe dieses Zitat gefunden
%Franz
%Franz Veit
%हठयोगः [Printed book page 5-501-c]
%हठयोगः , पुं, (हठेन योगः ।) योगविशेषः ।
%यथा, —
%“इदानीं हठयोगस्तु कथ्यते हठसिद्धिदः ।
%कृत्वासनं पवनाशं शरीरे रोगहारकम् ॥
%पूरकं कुम्भकञ्चैव रेचकं वायुना भजेत् ।
%इत्थं क्रमोत्क्रमं ज्ञात्वा पवनं सग्धयेत् सदा ॥
%धौत्यादिकर्म्मषट्कञ्च संस्कुर्य्याद्धठसाधकः ।
%एतन्नाड्यान्तु देवेशि ! वायुपूर्णं प्रतिष्ठितम् ॥
%ततो मनो निश्चलं स्यात्तत आनन्द एव हि ।
%हठयोगान्न कालः स्यान्मनः शून्ये भवेद्यदि ॥
%इदानीं हठयोगस्य द्वितीयं भेदवत् शृणु ।
%आकाशे नासिकाग्रे तु सूर्य्यकोटिसमं स्मरेत् ॥
%श्वेतं रक्तं तथा पीतं कृष्णमित्यादिरूपतः ।
%एवं ध्यात्वा चिरायुः स्यादङ्गाजननवर्ज्जितः ॥
%शिवतुल्यो महात्मासौ हठयोगप्रसादतः ।
%हठाज्ज्योतिर्म्मयो भूत्वा ह्यन्तरेण शिव भवेत् ।
%अतोऽयं हठयोगः स्यात् सिद्धिदः सिद्धसेवितः ॥”
%इति योगस्वरोदयः ॥ [ID=41348]

%Now, Haṭhayoga indeed is explained as that which gives the siddhi (accomplishment) of haṭha (persistence).
%One performs the wind-eating/serpent āsanam, which removes illness in the body
%and filling – kumbhaka – emptying may distribute the vāyu/wind.
%In this way, while being aware of the progress and regress of the breath one may feed on the wind continually.
%And with the six karmmas, dhauti etc., the Sādhaka of Haṭha may prepare/embellish himself.
%Thus/thereby, in the channel (nāḍī), Oh supreme Goddess, all of the winds (vāyu) are consecrated/placed.
%Then the mind may be unmoved and then bliss it really is.
%Through Haṭhayoga time will be no more, when the mind in emptiness abides.
%
%Now listen to the second disclosure of haṭhayoga:
%In space, on the tip of the nose indeed, one may remember equal to ten million suns,
%the primal forms: white, red, likewise yellow, dark blue.
%Thus meditating/visualizing, one may have a long life, free of the birth of the body,
%Equal to Śiva, this great soul, due to the blessing of Haṭhayoga,
%shall become through persistence (haṭha) a being of light and internally śiva.
%Therefore this Haṭhayoga grants accomplishment – it’s practiced by the Siddhas (accomplished ones).
%Franz
%Im śabdakalpadruma
%Franz
%Franz Veit
%fj.veit@gmail.com
\documentclass[10pt]{memoir}
\setstocksize{220mm}{155mm} 	        
\settrimmedsize{220mm}{155mm}{*}	
\settypeblocksize{170mm}{116mm}{*}	
\setlrmargins{18mm}{*}{*}
\setulmargins{*}{*}{1.2}
%\setlength{\headheight}{5pt}%
\checkandfixthelayout[lines]
\linespread{1.16}
\flushbottom

%%% Hyphenation settings
\usepackage[htt]{hyphenat}
\hyphenation{he-lio-trope opos-sum}
\tracingparagraphs=1
%Hyphenation in Devanāgarī of the edition still missing? Probably this needs to be modified in babel-iast package? 

%%% babel
\usepackage[english]{babel}
\usepackage{babel-iast/babel-iast}

\babelfont[iast]{rm}[Renderer=Harfbuzz, Scale=1.3]{AdishilaSan}%AdishilaSan}
\babelfont[english]{rm}{Adobe Text Pro}

%%% more functionality
\PassOptionsToPackage{hyphens}{url}
\usepackage{hyperref}
\usepackage{pdflscape}
\usepackage{cleveref}
\usepackage{url}
\usepackage{cleveref}
\usepackage{microtype}
\usepackage{lineno}

%\usepackage{bigfoot}
%%% more functions
\usepackage[dvipsnames]{xcolor}
%\usepackage[para,perpage]{footmisc}

%%%für den Counter von Kapiteln und Sätzen! 
\newcommand{\uproman}[1]{\uppercase\expandafter{\romannumeral#1}}
\newcommand{\lowroman}[1]{\romannumeral#1\relax}

\makeindex
\newfontfamily\sanskritfont[Script=Devanagari,Mapping=RomDev,Scale=1.1]{Sanskrit2003}
\usepackage{pifont,fourier-orns,lettrine,psvectorian,paralist,enumitem,pdfpages,wrapfig,tabulary,lettrine,longtable}
\setlist[enumerate]{itemsep=0mm}
\usepackage[autostyle]{csquotes}
\usepackage[defaultlines=2,all]{nowidow}
\usepackage{ellipsis,adforn,booktabs,longtable,url,tikz}
\lineskiplimit=-3pt          

\makechapterstyle{IeT}{%
  \chapterstyle{default}
  \renewcommand*{\printchapternonum}{\centering}
  \renewcommand*{\clearforchapter}{\cleartorecto} 
  \aliaspagestyle{chapter}{empty}}
\chapterstyle{IeT}
\setsecnumdepth{none}  \openright  \nouppercaseheads
\settocdepth{subsubsection}

%%%% test better pagebreaks
%\def\fussy{%
%  \emergencystretch\z@
%  \tolerance 200%
%  \hfuzz .1\p@
%  \vfuzz\hfuzz}

%\interfootnotelinepenalty=10000\relax

%\usepackage[maxfloats=256]{morefloats}

%\maxdeadcycles=500

%raggedbottomsectiontrue
%%\checkandfixthelayout


%%%%%%%  biblatex
%\newcommand{\noun}[1]{\textsc{#1}}    %  philosophy-verbose
\usepackage[backend=biber, sorting=nyt, style=verbose]{biblatex} %%%%ORIGINAL TiE
\renewcommand*{\mkbibnamefamily}[1]{\textsc{#1}}


\DeclareFieldFormat{url}{%
  \mkbibacro{URL}\addcolon\space
  \href{#1}{\nolinkurl{\thefield{urlraw}}}}

\DeclareFieldFormat{citeurl}{%
  \href{#1}{\nolinkurl{\thefield{urlraw}}}} 


\DeclareFieldFormat{postnote}{#1}
\renewcommand{\postnotedelim}{, }
\addbibresource{bindu.bib}

%%% ekdosis
\usepackage[teiexport=tidy,parnotes=true]{ekdosis}% =tidy cleans up HTML and XML documents by fixing markup errors and upgrading legacy code to modern standards. parnotes=footnotes below or above critical apparatus

\SetLineation{lineation=page, modulo} %lineation=page sets thenumbering to start afresh at the top of each page. =modulo makes every fifth line numbered. {lineation=page} makes every line numbered! 

\renewcommand{\linenumberfont}{\selectlanguage{english}\footnotesize} %sets language of lines to English

\SetTEIxmlExport{autopar=false} %autopar=falseinstructs ekdosis to ignore blank lines in the.tex sourcefile as markers for paragraph boundaries. As a result, each paragraph of the edition must be found within an environment associated with the xml <p> element

\SetHooks{
  lemmastyle=\bfseries,
  %refnumstyle=\selectlanguage{english}\bfseries,
  refnumstyle=\selectlanguage{english}\color{blue}\bfseries,
  appheight=0.8\textheight,
}

\newif\ifinapparatus
\DeclareApparatus{source}[
%bhook=\inapparatustrue,
lang=english,
notelang=english,
% bhook=\selectlanguage{english},
bhook=\selectlanguage{english}\textbf{Sources:},%
%maxentries=4, 
%ehook=.]
%sep={] },
%nosep,
]

\newif\ifinapparatus
\DeclareApparatus{testium}[
%bhook=\inapparatustrue,
lang=english,
notelang=english,
% bhook=\selectlanguage{english},
bhook=\selectlanguage{english}\textbf{Testimonia:},
%maxentries=4, 
%ehook=.]
%nosep, 
]

% Declare \ifinapparatus and set \inapparatustrue at the beginning of
% the apparatus criticus block. Also set the language.  
\newif\ifinapparatus
  \DeclareApparatus{default}[
  %bhook=\inapparatustrue, 
  lang=english,
  %maxentries=33,
  %bhook=\selectlanguage{english},
  sep = {] },
  delim=\hskip 0.75em,
  rule=\rule{0.7in}{0.4pt},
]

\newif\ifinapparatus
\DeclareApparatus{philcomm}[
%bhook=\inapparatustrue,
lang=english,
notelang=english,
bhook=\selectlanguage{english}\textbf{Philological Commentary:},
%bhook=\selectlanguage{english},
sep={: },
]

\ekdsetup{
showpagebreaks,
spbmk = \textcolor{blue}{spb},
hpbmk = \textcolor{red}{hpb}
}

%\usepackage{fnpos}
%\makeFNmid
%\makeFNbottom
\usepackage[bottom]{footmisc}
%%%%%%%%%%%%%%%%%%%%%%%%%%%
\makeatletter
\def\blfootnote{\gdef\@thefnmark{}\@footnotetext}
\makeatother
%%%%%%%%%%%%%%%%%%%%%%%%%


% Macros and Definitions for the Print of Sigla
\def\acpc#1#2#3{{#1}\rlap{\textrm{\textsuperscript{#3}}}\textsubscript{\textrm{#2}}\space}
\def\sigl#1#2{{{#1}}\textsubscript{\textrm{#2}}}
\def\None{{\sigl{N}{1}}} \def\Noneac{\acpc{N}{1}{ac}\,} \def\Nonepc{\acpc{N}{1}{pc}\,}
\def\Ntwo{{\sigl{N}{2}}} \def\Noneac{\acpc{N}{2}{ac}\,} \def\Nonepc{\acpc{N}{2}{pc}\,}
\def\Done{{\sigl{D}{1}}} \def\Doneac{\acpc{D}{1}{ac}\,} \def\Donepc{\acpc{D}{1}{pc}\,}
\def\Dtwo{{\sigl{D}{2}}} \def\Dtwoac{\acpc{D}{2}{ac}\,} \def\Dtwopc{\acpc{D}{2}{pc}\,}
\def\Uone{{\sigl{U}{1}}} \def\Uoneac{\acpc{U}{1}{ac}\,} \def\Uonepc{\acpc{U}{1}{pc}\,}                 
\def\Utwo{{\sigl{U}{2}}} \def\Utwoac{\acpc{U}{2}{ac}\,} \def\Utwopc{\acpc{U}{2}{pc}\,}

%%%%%%%%%%%%%% Tattvabinduyoga - List of Witnesses   %%%%%%%%%%%%%%%%%%%
\DeclareWitness{ceteri}{\selectlanguage{english}cett.}{ceteri}[]   
\DeclareWitness{E}{\selectlanguage{english}E}{Printed Edition}[]    
\DeclareWitness{P}{\selectlanguage{english}P}{Pune BORI 664}[]  
\DeclareWitness{B}{\selectlanguage{english}B}{Bodleian 485}[]       
\DeclareWitness{N1}{\selectlanguage{english}N\textsubscript{1}}{NGMPP 38/31}[]
\DeclareWitness{N2}{\selectlanguage{english}N\textsubscript{2}}{NGMPP B 38/35}[]
\DeclareWitness{L}{\selectlanguage{english}L}{LALCHAND 5876}[]  
\DeclareWitness{D}{\selectlanguage{english}D}{IGNCA 30019}[] 
%\DeclareWitness{D2}{\selectlanguage{english}D\textsubscript{2}}{IGNCA 30020}[]  
\DeclareWitness{U1}{\selectlanguage{english}U\textsubscript{1}}{SORI 1574}[] 
\DeclareWitness{U2}{\selectlanguage{english}U\textsubscript{2}}{SORI 6082}[]
%%%%%%%%%%%%%% Tattvabinduyoga - Groups of Witnesses   %%%%%%%%%%%%%%%%%%%
\DeclareWitness{X}{\selectlanguage{english}\alpha}{Alpha Group: D,N1,N2,U1}[]
\DeclareWitness{Y}{\selectlanguage{english}\beta}{Beta Group: B,E,L,P,U2}[]
%%%%%%%%%%%%% Testimonia
\DeclareWitness{Ysv}{\selectlanguage{english}Ysv}{Yogasvarodaya}[] %%%add infos!  

%%%%%%%%%%%%%%%%%%%%%%%%%%%%%%%%%%%%%%%%%%%
% Macro for Editing Abbrevs.
\def\om{\textrm{\footnotesize \textit{om.}\ }} %prints om. for omitted in apparatus
\def\korr{\textrm{\footnotesize \textit{em.}\ }} %prints em. for emended in apparatus
\def\conj{\textrm{\footnotesize \textit{conj.}\ }} %prints conj. for conjectured in apparatus

% \supplied{text} EDITORIAL ADDITION -> Within \lem oder \rdg
% \surplus{text} EDITORIAL DELETION -> Within \lem oder \rdg
% \sic{text} CRUX
% \gap{text} LACUNAE -> [reason=??, unit=??, quantity=??, extent=??]


%%%%%%%%%%%%%%%%%%%%%%%%%%%%%%%%%%%%%%%%%%% All macros of this list can be used in 
% Macro for Editing Abbrevs.
\def\eyeskip{\textrm{{ab.\,oc. }}}
\def\aberratio{\textrm{{ab.\,oc. }}}
\def\ad{\textrm{{ad}}}
\def\add{\textrm{{add.\ }}}
\def\ann{\textrm{{ann.\ }}}
\def\ante{\textrm{{ante }}} 
\def\post{\textrm{{post }}}
%\def\ceteri{cett.\,}                   
\def\codd{\textrm{{codd.\ }}}

\def\coni{\textrm{{coni.\ }}}
\def\contin{\textrm{{contin.\ }}}
\def\corr{\textrm{{corr.\ }}}
\def\del{\textrm{{del.\ }}}
\def\dub{\textrm{{ dub.\ }}}

\def\expl{\textrm{{explic.\ }}} 
\def\explica t{\textrm{{explic.\ }}}
\def\fol{\textrm{{fol.\ }}}
\def\foll{\textrm{{foll.\ }}}
\def\gloss{\textrm{{glossa ad }}}
\def\ins{\textrm{{ins.\ }}}      
\def\inseruit{\textrm{{ins.\ }}} 
\def\im{{\kern-.7pt\lower-1ex\hbox{\textrm{\tiny{\emph{i.m.}}}\kern0pt}}} %\textrm{\scriptsize{i.m.\ }}}      
\def\inmargine{{\kern-.7pt\lower-.7ex\hbox{\textrm{\tiny{\emph{i.m.}}}\kern0pt}}}%\textrm{\scriptsize{i.m.\ }}}      
\def\intextu{{\kern-.7pt\lower-.95ex\hbox{\textrm{\tiny{\emph{i.t.}}}\kern0pt}}}%\textrm{\scriptsize{i.t.\ }}}           
\def\indist{\textrm{{indis.\ }}}  
\def\indis{\textrm{{indis.\ }}}
\def\iteravit{\textrm{{iter.\ }}} 
\def\iter{\textrm{{iter.\ }}}
\def\lectio{\textrm{{lect.\ }}}   
\def\lec{\textrm{{lect.\ }}}
\def\leginequit{\textrm{{l.n. }}} 
\def\legn{\textrm{{l.n. }}}
\def\illeg{\textrm{{l.n. }}}

\def\primman{\textrm{{pr.m.}}}
\def\prob{\textrm{{prob.}}}
\def\rep{\textrm{{repetitio }}}
\def\secundamanu{\textrm{\scriptsize{s.m.}}}            \def\secm{{\kern-.6pt\lower-.91ex\hbox{\textrm{\tiny{\emph{s.m.}}}\kern0pt}}}%   \textrm{\scriptsize{s.m.}}}
\def\sequentia{\textrm{{seq.\,inv.\ }}}  
\def\seqinv{\textrm{{seq.\,inv.\ }}}
\def\order{\textrm{{seq.\,inv.\ }}}
\def\supralineam{{\kern-.7pt\lower-.91ex\hbox{\textrm{\tiny{\emph{s.l.}}}\kern0pt}}} %\textrm{\scriptsize{s.l.}}}
\def\interlineam{{\kern-.7pt\lower-.91ex\hbox{\textrm{\tiny{\emph{s.l.}}}\kern0pt}}}   %\textrm{\scriptsize{s.l.}}}
\def\vl{\textrm{v.l.}}   \def\varlec{\textrm{v.l.}} \def\varialectio{\textrm{v.l.}}
\def\vide{\textrm{{cf.\ }}}
\def\cf{\textrm{{cf.\ }}} 
\def\videtur{\textrm{{vid.\,ut}}}
\def\crux{\textup{[\ldots]} }
\def\cruxx{\textup{[\ldots]}}
\def\unm{\textit{unm.}}
%%%%%%%%%%%%%%%%%%%%%%%%%%%%%%%%%%%%

% List of Scholars
\DeclareScholar{ego}{ego}[
forename=Nils Jacob,
surname=Liersch]

% Persons:14\DeclareScholar{ego}{ego}[15forename=Robert,16surname=Alessi]17% Useful shorthands:18\DeclareShorthand{codd}{codd.}{V,I,R,H}19\DeclareShorthand{edd}{edd.}{Lit,Erm,Sm}20\DeclareShorthand{egoscr}{\emph{scripsi}}{ego}

%Useful shorthands:
%\DeclareShorthand{codd}{codd.}{V,I,R,H}
%\DeclareShorthand{edd}{edd.}{Lit,Erm,Sm}
\DeclareShorthand{egoscr}{em.}{ego}
\DeclareShorthand{egoscrconj}{conj.}{ego}
\DeclareShorthand{egomute}{\unskip}{ego}

\usepackage{xparse}

\NewDocumentEnvironment{tlg}{O{}O{}}{\setlength{\leftskip}{0pt}\vspace{-1ex}\begin{quotation}}{\hfill #1\ \vspace{-1ex}\end{quotation}\vspace{-1ex}} %verse environment
%\NewDocumentEnvironment{tlg}{O{}O{}}{\begin{verse}}{॥#1\hskip-4pt ॥\\ \end{verse}}
\NewDocumentCommand{\tl}{m}{{\selectlanguage{iast} #1}}

\NewDocumentCommand{\extra}{m}{{\textcolor{gray}{#1}}} %command for additions to U2
\NewDocumentCommand{\crazy}{m}{{\textcolor{red}{#1}}} %totally corrupted passage
\NewDocumentCommand{\coro}{m}{{\textcolor{violet}{#1}}} %colour for sentence counter! 

\NewDocumentEnvironment{prose}{O{}}{\begin{otherlanguage}{iast}}{\end{otherlanguage}}
% \NewDocumentEnvironment{padd}{O{}}{\begin{otherlanguage}{iast}}{\end{otherlanguage}}
\NewDocumentEnvironment{tlate}{O{}}
%\NewDocumentEnvironment{tadd}{O{}}

%Define two commands: \skp ("sanskrit plus"), to be ignored by TeX in
%the edition text, but processed in the TEI output. Conversely, \skm
%("sanskrit minus") is to be processed in the edition text, but
%ignored if found in the apparatus criticus and in the TEI output:

\NewDocumentCommand{\skp}{m}{}
\TeXtoTEIPat{\skp {#1}}{#1}

%\NewDocumentCommand{\skpp}{m}{}
%\TeXtoTEIPat{\skpp {#1}}{#1}

\NewDocumentCommand{\skm}{m}{\unless\ifinapparatus#1-\fi}
\TeXtoTEIPat{\skm {#1}}{}

% \NewDocumentCommand{\dd}{}{/\hskip-4pt/}
\NewDocumentCommand{\dd}{}{\mbox{/\hskip-4pt/}}
\TeXtoTEIPat{\dd {}}{//}


%%% modify environments and commands
%%% TEI mapping
\TeXtoTEIPat{\begin {tlg}}{<lg>} %lg=(Group of verse (s)) contains one or more verses or lines of verse that together form a formal unit (e.g. stanza, chorus).
\TeXtoTEIPat{\end {tlg}}{</lg>}

\TeXtoTEIPat{\begin {prose}}{<p>}
\TeXtoTEIPat{\end {prose}}{</p>}

\TeXtoTEIPat{\begin {tlate}}{<p>}
\TeXtoTEIPat{\end {tlate}}{</p>}

\TeXtoTEIPat{\\}{}
\TeXtoTEIPat{\linebreak}{<br/>}
\TeXtoTEIPat{\noindent}{}
%\TeXtoTEI{tl}{l}
\TeXtoTEI{emph}{hi}
\TeXtoTEI{bigskip}{}
\TeXtoTEI{None}{N1}
\TeXtoTEI{Ntwo}{N2}
\TeXtoTEI{Done}{D1}
\TeXtoTEI{Dtwo}{D2}
\TeXtoTEI{Uone}{U1}
\TeXtoTEI{Utwo}{U2}
%\TeXtoTEIPat{/}{ |}
%\TeXtoTEI{//}{ ||}
\TeXtoTEIPat{\korr}{em. }
\TeXtoTEIPat{\conj}{conj.}
\TeXtoTEIPat{\om}{om.}
\TeXtoTEIPat{english}{}
\TeXtoTEIPat{\hskip}{}
\TeXtoTEIPat{\hskip-4pt}{}
\TeXtoTEIPat{\hskip-2pt}{}
\TeXtoTEIPat{-}{ }
\TeXtoTEIPat{4pt}{}
\TeXtoTEIPat{2pt}{}
\TeXtoTEIPat{\textcolor {#1}{#2}}{<hi rend="#1">#2</hi>} 

% Nullify \selectlanguage in TEI as it has been used in
% \DeclareWitness but should be ignored in TEI.
\TeXtoTEI{selectlanguage}{}



\author{Nils Jacob Liersch}
\title{Yogatattvabindu of Rāmacandra\\ A Critical and Synoptic Edition and Annotated Translation}
\date{\today}

\parindent=15pt
\begin{document}

% Zitiermöglichkeiten:
%\footcite[See][p.\,1]{goldstein01:_tibet_englis_diction_moder_tibet}
%\footnote{\cite{goldstein01:_tibet_englis_diction_moder_tibet}.}

\frontmatter
\thispagestyle{empty}
\begin{center}
  {\Large \emph{The Yogatattvabindu}}\\[3mm]
\end{center}



\newpage

\

\thispagestyle{empty}



\normalsize


\newpage


\begin{center}
\thispagestyle{empty}

\

\vskip 2mm

\begin{otherlanguage}{iast}
\LARGE \sanskritfont{Yogatattvabindu}
\end{otherlanguage}

\vskip .4cm

\Huge Yogatattvabindu \\[7mm]
\Large Critical and Synoptic \\
Edition with annotated Translation


\large

\vspace{3cm}

Von

Nils Jacob Liersch
\small
\vfill

\vfill

Indica et Tibetica Verlag \\ % $\cdot$ 
Marburg 2024

\vskip 6mm

\end{center}

\newpage
\newpage \ \thispagestyle{empty}
\small  \

\noindent

\
\vfill


\small
\noindent \textbf{Bibliographische Information Der Deutschen Bibliothek}

\noindent
Die Deutsche Bibliothek verzeichnet diese Publikation in der Deutschen Nationalbibliographie;
detaillierte bibliographische Informationen sind im Internet über http://dnb.ddb.de abrufbar.

\noindent
\textbf{Bibliographic information published by Die Deutschen Bibliothek}

\noindent
Die Deutsche Bibliothek lists this publication in the Deutsche Nationalbibliographie; detailed
bibliographic data is available in the Internet at http://dnb.ddb.de.  


\vskip 1cm

\noindent
\copyright\ Indica et Tibetica Verlag, Marburg 2024

\medskip

\noindent
Alle Rechte vorbehalten / All rights reserved

\medskip

\noindent
Ohne ausdrückliche Genehmigung des Verlages ist es nicht gestattet, das Werk oder einzelne Teile
daraus nachzudrucken, zu vervielfältigen oder auf Datenträger zu speichern.

\smallskip

\noindent
Apart from any fair dealing for the purpose of private study, research, criticism or review, no
part of this book may be reproduced or translated in any form, by print, photo form, microfilm, or
any other means without written permission. Enquiries should be made to the publishers.

\bigskip

\noindent
Satz: \ \ Nils Jacob Liersch \\
Herstellung: \ \ BoD – Books on Demand GmbH, Norderstedt  \\

\bigskip

\noindent
%\ISBN     

\normalsize

\newpage

%\maketitle
\clearpage
\tableofcontents
\addtocounter{page}{-1}
\thispagestyle{empty}
\clearpage

\chapter{Introduction}
\mainmatter

\chapter{The List of the 15 Yogas}
\label{yogas_list}
The authenticity of the list specifying the fifteen Yogas at the beginning of the text is ambiguous. This is due to the discrepancy between the structure of the Yogas presented in the text and the order presented in the list. For example, the text commences with a description of \textit{kriyāyoga} and goes on to describe \textit{siddhakuṇḍaliniyoga} and then mentions \textit{mantrayoga} without adhering to the order presented in the list. This incongruity raises questions as to why the text structure deviates from the list. However, the reference to \textit{jñānotpattav upāyaḥ} may provide some insight into why \textit{jñānayoga} is included as the second \textit{yoga} in the list. To reconcile these apparent inconsistencies, there are several possible explanations: 1) The text is severely corrupted. 2) The list was added by a different hand at a later time. 3) The term \textit{jñānayoga} is included as a result of the practice of \textit{siddhakuṇḍalinīyoga}, which is said to generate knowledge through the central channel, as stated in the text. These explanations may be combined to provide a comprehensive understanding of the situation.

\chapter{Sources}
\section{The Additions of  SORI 6082 - U\textsubscript{2}}
\label{discussionu2}
Analyse the additions of U\textsubscript{2} and present the \textit{cakra}s and their attriubutes in a table .
\begin{itemize}
\item  Muktabodha-Texte sehe ich 3 Belege für bahiśśakti Muktabodha/krīyakramādyotikā.html 2938 suṣirānte bahiśśaktiṃ vinyasedvyomarūpiṇīm | tasyā madhye tu Muktabodha/sakalāgamasārasaṅgraha.html 2186 suṣirāntabahiśśaktiṃ vyāpinīṃ cintayet tataḥ || Muktabodha/kriyakramadyotikavyākhyā.html 1846 tanmadhye ca bahiśśaktiṃ sudhābindu parisrutim
  \item  Parā\footnote{Im Kaśm. Śiv. °das ewige Wort, in welchem potentiell alle Begriffe und Worte ruhen; vgl. das śabdabrahma des Vyākaraṇa. [B.]― Schmidt S. 246}.
  \end{itemize}

\chapter{Conventions in the Critical Apparatus}
\section{Sigla in the Critical Apparatus}

\begin{itemize}
\item E : Printed Edition
\item P : Pune BORI 664
\item L : Lalchand Research Library LRL5876
\item B : Bodleian Oxford D 4587
\item \None : NGMPP B 38-31
\item \Ntwo : NGMPP B 38-35 / A 1327-14
\item \Done : IGNCA 30019
\item \Uone : SORI 1574
\item \Utwo: SORI 6082
\end{itemize}

The order of the readings in the critical apparatus is arranged according to the quality of readings in decending order. The critical apparatus is positive. Gemitation is not recorded. 


\section{Marking the Reliability of Sources and Testimonia in the Critical Apparatus}
\label{kennz}

To accurately depict information about the textual relationship and estimated degree of relatedness of a passage from the \textit{Yogatattvabindu} in the layers for sources and testimonia of the critical apparatus, a system of sigla was introduced.\footnote{This type of identification system is based on the use of the critical apparatus in \parencite[lii-liii]{steinkellner2005}. It was modified for the text-critical work on the \textit{Yogatattvabindu}.} The sigla are meaningful when a passage is corrupted in all witnesses and can only be reconstructed by means of other texts. The layers of the critical apparatus for sources and testimonia use the following sigla:

\begin{enumerate}
\item[\textbf{Ce}] \textit{citatum ex alio} / quotation from another (text).\footnote{The sigla \textbf{Ce} indicates an identical or largely identical content in the lesser witness and only allows for minor deviations in the wording of the passage.}
\item[\textbf{Cee}] \textit{citatum ex alio modo edendi} / quotation from another (text) with editorial changes.\footnote{The sigla \textbf{Cee} identifies passages with noticeable deviations in the lesser witness.}
\item[\textbf{Ci}] \textit{citatum in alio} / quotation in another (text).\footnote{The sigla \textbf{Ci} indicates an identical or largely identical content in the lesser witness and only allows for minor deviations in the wording of the passage.}
\item[\textbf{Cie}] \textit{citatum in alio modo edendi} / quotation in another (text) with editorial changes.\footnote{The sigla \textbf{Cie} identifies passages in the lesser witness with noticeable deviations that have the intended character of the composer.}
\item[\textbf{Re}] \textit{relatum ex alio} / (content), attested from another text.\footnote{The sigla \textbf{Re} identifies content parallels in the lesser witness that are relevant to the constitution of the critical text. It further indicates in certain cases that the composer might have used this source when composing his text.}
\item[\textbf{Ri}] \textit{relatum in alio} / (content), attested in another text.\footnote{The sigla \textbf{Ri} identifies content parallels in the lesser witness that are relevant to the constitution of the critical text.}
\end{enumerate}

The following acronyms refer to passages that originated from texts that the author of the \textit{Yogatattvabindu} utilized in compiling his work: \textbf{Ce}, \textbf{Cee}, \textbf{Re}. These texts must predate the \textit{Yogatattvabindu}. The other acronyms, such as \textbf{Ci}, \textbf{Cie}, and \textbf{Ri}, are texts that have adopted passages from the \textit{Yogatattvabindu}, or verses or passages that share similar content with the \textit{Yogatattvabindu}, but their relation is given literally, making it impossible to determine who adopted from whom. \textbf{Re} and \textbf{Ri} each refer to passages that are so closely related in content to those of the \textit{Yogatattvabindu} that they are significant in reconstructing a passage.\footnote{\textbf{Ce} and \textbf{Cee} have the highest degree of reliability, \textbf{Ci} and \textbf{Cie} have a moderate degree, and \textbf{Re} and \textbf{Ri} have the lowest.}

\section{Punctuation}

The inconsistent use of punctuation marks in the available witnesses necessitates standardization. Upon close examination, it appears that punctuation has frequently been dropped or added during the transmission of the texts. The neglect or improper handling of punctuation by the copists has resulted in different versions of lists with and without punctuation. In many instances, missing punctuation has led to the addition of case endings, alteration of the text, and the combination of list items into compound formations that were not present in the original text. Although punctuation plays an important role, deviations in punctuation at the end of sentences, lists, and verse-numbering will only be extensively documented in the critical apparatus of the printed edition. This means that emendations of obvious punctuation mistakes will not be recorded in the critical apparatus. However, the digital edition of this work provides a more detailed documentation of deviations in punctuation through diplomatic transcripts of each witness, and even has a function to display sentences cumulatively.

In the printed edition of the \textit{Yogatattvabindu}, standard conventions of punctuation are followed. In verse poetry, a \textit{daṇḍa} (|) marks the end of a half-verse or half of the \textit{śloka}, and a double \textit{daṇḍa} (||) marks the end of a verse. In prose, a single \textit{daṇḍa} indicates the end of a sentence, and a double \textit{daṇḍa} marks the end of a paragraph. Variations in the use of \textit{avagraha} will be recorded, and items in lists will be separated by a double-\textit{daṇḍa}.

\section{Sandhi}

Among the witnesses we see deviating and inconsistent application of \textit{sandhi}. There is no clear evidence that originally \textit{sandhi} was intentionally not applied. This edition will therefore apply \textit{sandhi} consistently throughout the constituted text to provide a readable text sticking to contemporary conventions in Sanskrit. The variant readings concerning \textit{sandhi} are recorded consistently in the apparatus criticus. This is due to various textcritical problems arising from the inconsistent usage of punctuation which results in application or non-application of \textit{sandhi} wheter the respective witness applied a \textit{daṇḍa} or not. This is particularly the case within lists, which frequently occur in our compilation. Items were most likely originally separated by \textit{daṇḍa}. 


\section{Class Nasals}

Due to inconsistent use of class nasals among the witnesses \textit{anusvāra}s have been substituted with the respective class nasals throughout the edition.

\section{Lists}

Lists are a frequent feature in the \textit{Yogatattvabindu}. The text opens with a list of 15 Yogas and there are many more lists utilized throughout its content. To produce a consistent and easily readable edition, all lists have been identified, normalized to the Nominative Singular or Nominative Plural form of the respective item, or in the case of explanatory lists, to the Ablative Singular or Plural. The items are separated by a double \textit{daṇḍa}. Differences in punctuation and simple punctuation emendations, unless they are text-critically or systematically significant, will not be recorded in the apparatus criticus.
\clearpage

\chapter{Critical Edition \& Annotated Translation}
\clearpage
\newpage
\begin{alignment}[
  texts=edition[class="edition"];
  translation[class="translation"],
  ]
  \begin{edition}
    \ekddiv{type=ed}
    \begin{prose}    
%----------------------
%tanmadhye    'gnidhūmākārarekhā     yādṛśy    ādṛśy ekā  puruṣasya mūrttir varttate /  \E
%tanmadhye    'gnidhūmākārarekhā     yādṛśī   tādṛśy ekā  puruṣasya mūrttir varttate /  \P
%tanmadhye    'gnidhūmākārārekhā     yādṛśī   tādṛśy ekā  puruṣasya mūrttir varttate /  \L               
%tanmadhye    'gnidhūmākārārekhā     yādṛśī   tādṛśy ekā  puruṣasya mūrttir varttate /  \B     
%tanmadhye    'gnidhūmākārāreṣā      yādṛśī   tādṛśī ekā  puruṣasya mūrttir varttate /  \N1
%tanmadhye    agnidhūmrākārarekhā    yādṛśī / tādṛśī ekā  puruṣasya mūrttir varttate /  \N2
%tanmadhye    agnidhūmākārāreṣā      yādṛśī   tādṛśī ekā  puruṣasya mūrttir varttate /  \D
%tanmadhye    agnidhūmrākārārekhā    yādṛśī   tādṛśī ekā  puruṣasya mūrtir  vartate     \U1
%tanmadhye    'gnidhūmrākārārekhāyāḥ  etādṛśī         ekā  puruṣasya mūrtir  vartate // \U2
%----------------------      
%In its middle [is] something like a streak having the form of smoke and fire. Such a single [divine] form of the person (\textit{puruṣa}) exists [there].        
%---------------------      
\noindent
    \note[type=testium, labelb=97, lem={\textbf{Cie}}]{\textit{Yogasaṃgraha} IGNCA 30020 folio 2r. l. 3: tanmadhye gnidhūmrāreṣākārā ādimadhyaṃtarahitā puruṣasya mūrttir asti |}
tanmadhye \app{\lem[wit={Y}]{'gnidhūmākārarekhā}
        \rdg[wit={U2}]{'gnidhūmrākārārekhāyāḥ}
        \rdg[wit={N1,D}]{'gnidhūmākārāreṣā}
        \rdg[wit={N2,U1}]{agnidhūmrākārarekhā}}
      \app{\lem[wit={ceteri}]{yādṛśī}
        \rdg[wit={E}]{yādṛśy°}
        \rdg[wit={U2}]{etādṛśī}}/
      \app{\lem[wit={P,L,B},alt={yādṛśy}]{yādṛ\skp{śy-e}}
        \rdg[wit={E}]{ādṛsy}
        \rdg[wit={X}]{yādṛśī}
        \rdg[wit={U2}]{\om}}\skm{śy-e}kā puruṣasya mūrtir-vartate/
%---------------------
%tasyā  nādir nāṃto 'sti / \E
%tasyā  nādināṃ 'to sti / \P
%tasyā  nādir nāṃto sti / \L -> vor dem bei allen anderen vorigen Satz!?!?!?! 
%tasyā  nādir nāṃto sti / \B -> vor dem bei allen anderen vorigen Satz!?!?!?! 
%tasyāḥ nāsty aṃtaḥ ādir-api nāsti / \N1????
%tasyāḥ nāsty aṃtaḥ ādir-api nāsti / \N2
%tasyāḥ nāsty aṃtaḥ ādir api nāsti / \D
%tasyāḥ nāsty aṃtaḥ ādir-api nāsti    \U1
%tasyā  nādir naṃto sti              \U2
%---------------------
% Of her exists no end, nor a beginning.
%---------------------      
\app{\lem[wit={Y}]{tasyā}
  \rdg[wit={X}]{tasyāḥ}}
\app{\lem[alt={nādir nānto 'sti}, wit={ceteri}]{nādir-nānto 'sti}
        \rdg[wit={P}]{nādināṃ 'to sti}
        \rdg[wit={X}]{nāsty aṃtaḥ ādir api nāsti}}/
%---------------------    
%tasyā  mūrtter dhyānakāraṇāt pratyakṣaṃ niraṃtaraṃ  puruṣasyākāśe   gamāgamau   bhavataḥ / \E
%tasyā  mūrtter dhyānakaraṇāt pratyakṣaniraṃtaraṃ    puruṣasyākāśe   gamāgamau   bhavataḥ / \P
%tasyā  mūrtir  dhyānakaraṇāt pratyakṣaniraṃtaraṃ    puruṣasyākāśe   gamāgamau   bhavataḥ / \L         
%tasyā  mūrtir  dhyānakaraṇāt pratyakṣaṃ niraṃtaraṃ  puruṣasyākāśe   gamāgamau   bhavataḥ / \B
%tasyāḥ mūrttir dhyānakaraṇāt pratyakṣaniraṃtaraṃ    puruṣasya ākāśe gamāgamau   bhavataḥ / \N1
%tasyāḥ mūrttir dhyānakaraṇāt pratyakṣaniraṃtaraṃ    puruṣa ākāśe    gamāgame    bhavataḥ / \N2
%tasyāḥ mūrtir  dhyānakaraṇāt pratyakṣaniraṃtaraṃ    puruṣasya ākāśe gamāgamau   bhavataḥ / \D
%tasyāḥ mūrter  dhyānakaraṇāt/ pratyakṣaniraṃtaraṃ   puruṣasya ākāśi gamāmamo   bhavataḥ   \U1
%tasyāḥ mūrter  dhyānakaraṇāt pratyakṣaniraṃtaraṃ    puruṣasyākāśa---gamāgamau bhavata //      \U2
%---------------------    
%BEDEUTUNG DES SATZES BIS JETZT UNKLAR! Idee: Zeilensprung aus übernächstem Satz! Streiche pratyakṣaṃ niraṃtaraṃ und der Satz ergibt Sinn!  
%gamāgamau nom.  dual = coming and going ; bhavataḥ = 3p du ind pres von bhū
%Due to the exercise of meditation on this (divine) form both coming and going of the person in space occurs. 
%Kolloquium: Meinung zu Kompositum pratyakṣaniraṃtaraṃ = macht wenig Sinn oder?
%{\englishnote{\small Even though every single witness at hand transmits the latter reading right after \textit{°karaṇāt}, several considerations make it reasonable to conject that the original sentence is corrupted and was written without it. The main consideration to assume the corruption is that \textit{pratyakṣaṃ nirantaraṃ} is ungrammatical. The second is that the sentence is way more meaningful without it. The third that two sentences later we get the phrase in a meaningful context. Due to the last consideration my best guess is an interlace at an early stage of transmission.}}
%---------------------
\note[type=testium, labelb=98, lem={\textbf{Cie}}]{\textit{Yogasaṃgraha} IGNCA 30020 folio 2r. l. 4: tasyāḥ dhyānakartuḥ}
      \app{\lem[wit={B,E,L,P}]{tasyā}
  \rdg[wit={ceteri}]{tasyāḥ}}
\app{\lem[alt={mūrter},wit={E,P,U1,U2}]{mūrte\skp{r-dhyā}}
  \rdg[wit={B,D,L,N1,N2}]{mūrtir}}
\app{\lem[wit={ceteri}]{dhyānakaraṇāt pratyakṣaniraṃtaraṃ}
  \rdg[wit={B,E}]{dhyānakāraṇāt pratyakṣaṃ niraṃtaraṃ}} 
      \app{\lem[wit={ceteri}]{puruṣasyākāśe}
        \rdg[wit={N2}]{puruṣa ākāśe}
        \rdg[wit={U2}]{puruṣasyākāśa°}
        \rdg[wit={U1}]{puruṣasya ākāśi}}
      gamā\app{\lem[wit={ceteri},alt={°gamau}]{gamau}
        \rdg[wit={U1}]{°gamo}
        \rdg[wit={N2}]{°game}}
        \app{\lem[wit={ceteri}]{bhavataḥ}
          \rdg[wit={U2}]{bhavata}}/
%---------------------     
%pṛthvīmadhye  sthitasyāpi    pṛthvī-------bādho   na bhavati / \E
%pṛthvīmadhye  sthitasyāpi    pṛthaka                 bhavati   \P %Zeilenspringer führt zu Verlust von Zeile in Pune
%pṛthvīmadhye  sthitasyāpi    pṛthvī-------bādho   na bhavati / \L
%pṛthivīmadhye sthitasyāpi // pṛtvī--------bādho   na bhavati // \B
%pṛthvīmadhye  sthitāv-api    pṛthvī kṣato bādho   na bhavati // \N1
%pṛthvīmadhye  sthitāv-api    pṛthvī kṣato bādho   na bhavati // \N2      
%pṛthvīmadhye  sthitāv-api    pṛthvī kṣato bādho   na bhavati // \D
%pṛthvīmadhye  sthitāv-api    pṛthvī kṣato bādho   na bhavati     \U1
%pṛthīvīmadhye sthitasyāpi    pṛthvī       bādhoko na bhati     \U2
%---------------------
%Affliction from the earth-element does not arise [anymore] even if one is situated in the middle of the earth.        
%---------------------
\note[type=testium, labelb=100, lem={\textbf{Cie}}]{\textit{Yogasaṃgraha} IGNCA 30020 folio 2r. ll. 4-5: pṛthivyāṃ sthitāv api pṛthvī kṛtabādho na bhavati | tri?kālikajñānaṃ pratyakṣaṃ bhavati | āyurvṛddiḥ liṃgaśarīreṇa sarvatra pratyakṣaṃ gamāgamo bhavati |} 
        \app{\lem[wit={ceteri}]{pṛthvīmadhye}
          \rdg[wit={B,U2}]{pṛtivīmadhye}}
        \app{\lem[wit={ceteri}]{sthitasyāpi}     
          \rdg[wit={Y},alt={sthitāv api}]{sthitāv\skp{-}api}}
        \app{\lem[wit={E,L}]{pṛthvībādho}
          \rdg[wit={B}]{pṛtvībādho}
          \rdg[wit={P}]{pṛthaka}
          \rdg[wit={U2}]{pṛthvī bādhoko}
          \rdg[wit={X}]{pṛthvī kṣato bādho}}
        \app{\lem[wit={ceteri}]{na bhavati}
          \rdg[wit={P}]{bhavati}}/
%---------------------
%sakalān pratyakṣaṃ niraṃtaraṃ paśyati ca pṛthagbhavati / \E
% \om                                                       \P      
%sakalāḥ pratyakṣaṃ niraṃtara paśyatī  ca pṛthak bhavati // \B
%sakalāḥ pratyakṣaṃ niraṃtara paśyatī  ca pṛthak bhavati / \L
%sakalāpratyakṣaniraṃtaraṃ    paśyati  ca pṛthak ca bhavati // \N1
%sakalapratyakṣaniraṃtaraṃ    paśyati  ca pṛthak ca bhavati    \N2      
%sakalāpratyakṣaniraṃtaraṃ    paśyati  ca pṛthak pṛthak bhavati \D      
%sakalāpratyakṣaniraṃtaraṃ    paśyati  ca/ pṛthak ca bhavati // \U1
%\om                                                     \U2
%---------------------
%He constantly sees everything in front of his eyes and he becomes separated (from the material world).
%---------------------
        \app{\lem[type=emendation, resp=egoscr]{sakalaṃ pratyakṣaṃ nirantaraṃ}
          \rdg[wit={X}]{sakalāpratyakṣaṃ nirantaraṃ}
          \rdg[wit={B,L}]{sakalāḥ pratyakṣaṃ niraṃtara}
          \rdg[wit={E}]{sakalān pratyakṣaṃ niraṃtaraṃ}
          \rdg[wit={P,U2}]{\om}}
        \app{\lem[wit={ceteri}]{paśyati}
          \rdg[wit={L,B}]{paśyatī}
          \rdg[wit={P,U2}]{\om}}
        \app{\lem[wit={E}]{pṛthagbhavati}
          \rdg[wit={B,L}]{ca pṛthak bhavati}
          \rdg[wit={N1,N2,U1}]{ca pṛthak ca bhavati}
          \rdg[wit={D}]{ca pṛthak pṛthak bhavati}
          \rdg[wit={P,U2}]{\om}}/  
%---------------------
%atiśayenāyur vardhate /   \E
%atiśayenāyur vardhate     \P      
%atīśayanāyur vardhayate / \B
%atīśayanāyur vardhayate // \L
%atiśayena āyur varddhate // \N1
%atiśayena āyur varddhate // \N2     
%atiśayena āyur varddhate // \D
%atiśayena āyur varddhate // \U1
%\om                         \U2
%---------------------
% The force of life increases eminently. 
%---------------------
        \app{\lem[alt={atiśayenāyur},wit={E,P}]{atiśayenāyu\skp{r-va}}
          \rdg[wit={B,L}]{atīśayanāyur}
          \rdg[wit={X}]{atiśayena āyur}
          \rdg[wit={U2}]{\om}}\app{\lem[alt={vardhate},wit={ceteri}]{\skm{r-va}rdhate}
          \rdg[wit={B,L}]{vardhayate}}\dd{}
          \vfill
        \end{prose}
        \nolinenumbers
        \smallskip
    \centerline{\textrm{\small{[Ninth Cakra]}}}
    \bigskip
    \linenumbers
    \begin{prose}
      \noindent
%---------------------
%idānīṃ navamacakrasya   bhedāḥ kathyante /  \E
%idānīṃ navamacakrasya   bhedāḥ kathyante /  \P
%idānīṃ navamacakrasya   bhedāḥ kathyate     \L
%idānīṃ navamaṃ cakrasya bhedāḥ kathyate //  \B
%idānīṃ navamacakrasya   bhedāḥ kathyaṃte // \N1
%idānīṃ navamacakrasya   bheda  kathyate  // \N2
%idānīṃ navamacakrasya   bhedāḥ kathyaṃte // \D
%idānīṃ navamaś cakrasya bhedāḥ kathyaṃte    \U1   
%idānīṃ navamacakrasya   bhedaḥ kathyate /   \U2
%---------------------
%Now the divisions/differentiations of the ninth cakra are explained.
%---------------------
\note[type=testium, labelb=101, lem={\textbf{Cie}}]{\textit{Yogasaṃgraha} IGNCA 30020 folio 2r. ll. 6-11:  brahmaraṃdhre eva śatadalacakropari mahāśūnyacakraṃ mahāsiddhacakraṃ pūrṇagiricakraṃ iti saṃjñakaṃ sahasradalaṃ cakraṃ asti | tad upari kiṃcin nāsti | tac cakraṃ atiraktaṃ ūrdhvamukhaṃ sakalaśobhāspadaṃ anekakalyāṇapūrṇaṃ mano vācā ma gocara parimalo petaṃ | tat kamalamadhye trikoṇākarṇikā |}
\note[type=source, labelb=102, lem={\textbf{Re}}]{PT\textsuperscript{qcr \cdot YSV} (Ed. p. 833): navaman tu mahāśūnyaṃ cakran tu tatparātparam | tad upari paraṃ kiñcin nāsti kiñcin mahāparam | mahācakraṃ siddhacakraṃ pūrṇagauryādisaṃjñakam | tanmadhye varttate padmaṃ sahasradalam adbhutam |}
idānīṃ
\app{\lem[wit={ceteri},alt={°navama}]{navama}
  \rdg[wit={B}]{navamaṃ}
  \rdg[wit={U1}]{navamaś°}}cakrasya
\app{\lem[wit={ceteri}]{bhedāḥ}
  \rdg[wit={N2}]{bheda}}
\app{\lem[wit={ceteri}]{kathyante}
  \rdg[wit={L,B,N2,U2}]{kathyate}}/
%------------------------------
%tasya mahāśūnyacakram    iti  saṃjñā /  \E
%tasya mahāśūnyacakram    iti  saṃjñā /  \P
%tasya mahāśūnye cakram   iti  saṃjñā    \L
%tasye mahāśūnye cakram   iti  saṃjñā    \B
%tasya mahāśūnye cakreti       saṃjñā // \N1
%tasya mahāśūnyacakreti        saṃjñā // \N2
%tasya mahāśūnyacakreti        saṃjñā // \D
%tasya mahāśūnyacakreti        saṃjñā /  \U1
%\om /                                   \U2
%---------------------
%The designation of it is ``the \textit{cakra} of the great void (\textit{mahāśūnyacakra})''.
%------------------------------
tasya \app{\lem[wit={ceteri}, alt={mahāśūnya°}]{mahāśūnya}
  \rdg[wit={L,B,N1}]{mahāśūnye}
  \rdg[wit={U2}]{\om}
}\app{\lem[wit={X},alt={°cakreti}]{cakreti}
  \rdg[wit={E,P}]{°cakram iti}
  \rdg[wit={L,B}]{cakram iti}
  \rdg[wit={U2}]{\om}}
\app{\lem[wit={ceteri}]{saṃjñā}
  \rdg[wit={U2}]{\om}}/
%------------------------------
%tadupary aparaṃ kimapi nāsti / \E
%tadupary aparaṃ kimapi nāsti \P
%tadupary        kimapi nāsti \B ??-> auch mögliche Lesart
%tadupari        kimapi nāsti \L
%tadupari aparaṃ kiṃapi nāsti / \N1
%tadupari aparaṃ kiṃapi nāsti / \N2
%tadupari aparaṃ kiṃapi nāsti / \D
%tadupari aparaṃ kiṃapi nāsti   \U1
% \om                           \U2
%---------------------
%kim api: somewhat, to a considerable extent, rather, much more, still, further. Śa
%---------------------
%Above that there is no other. 
%---------------------
\app{\lem[wit={E,P,B},alt={tad upary}]{tad\skp{-}upar\skm{y-a}}
  \rdg[wit={ceteri}]{tad upari}
  \rdg[wit={U2}]{\om}}\app{\lem[wit={ceteri}, alt={aparaṃ}]{\skp{y-a}paraṃ}
  \rdg[wit={B,L,U2}]{\om}}
\app{\lem[wit={ceteri}]{kimapi}
  \rdg[wit={X}]{kiṃ api}
  \rdg[wit={U2}]{\om}} nāsti/
%------------------------------
%tadeva-mahāsiddhacakraṃ kathyate // \E
%tadeva-mahāsiddhacakraṃ kathyate    \P 
%tadeva-mahāsiddhacakraṃ kathyate // \B
%tadeva-mahāsiddhacakraṃ kathyate // \L
%tadeva-mahāsiddhacakraṃ kathyate // \N1
%tadeva-mahāsiddhacakraṃ kathyate // \N2
%tadeva-mahāsiddhacakraṃ kathyate // \D
%tadeva-mahāsiddhacakraṃ kathyate /  \U1
% \om                                \U2
%---------------------
%Therefore it is declared to be the \textit{cakra} of the great perfection (\textit{mahāsiddhacakra}).
%---------------------
tad-eva mahāsiddhacakraṃ kathyate/
%------------------------------
%       tasya           pūrṇagiripīṭha               etadṛśaṃ nāma /  \E 
%       tasya           pūrṇagiripīṭham-iti          etādṛśaṃ nāma    \P
%       tasya           pūrṇagiripīṭham-iti saṃjñā   etādṛsaṃ nāma    \B ->!!! 
%       tasya           pūrṇagiripīṭham-iti saṃjñā   etādṛsaṃ nāma    \L
%       tasya cakrasya  pūrṇagiri                    etādṛśaṃ nāma /  \N1
%       tasya cakrasya  pūrṇagiri                    etādṛśaṃ nāma /  \N2
%       tasya cakrasya  pūrṇagiri                    etādṛśaṃ nāma /  \D
%       tasya cakrasya  pūrṇagire                    etādṛśaṃ nāmaḥ   \U1
%madhye tasya           pūrṇagiripīṭham-iti          ekādaśaṃ nāma // \U2   
%-----------------------------
%Such a name of it is ``(divine) seat of Pūrṇagiri''.   
%------------------------------
\app{\lem[wit={ceteri}]{tasya}
  \rdg[wit={X}]{tasya cakrasya}
  \rdg[wit={U2}]{madhye tasya}}
pūrṇagiri\app{\lem[wit={P,B,L,U2}, alt={°pīṭham}]{pīṭha\skm{m-i}}
  \rdg[wit={E}]{pīṭha}
  \rdg[wit={ceteri}]{\om}
}\app{\lem[wit={P,U2},alt={iti}]{\skp{m-i}ti}
  \rdg[wit={B,L}]{iti saṃjñā}
  \rdg[wit={ceteri}]{\om}}
\app{\lem[wit={ceteri}]{etādṛśaṃ}
  \rdg[wit={E}]{etadṛśaṃ}
  \rdg[wit={U2}]{ekādaśaṃ}}
\app{\lem[wit={ceteri}]{nāma}
  \rdg[wit={U1}]{nāmaḥ}}/
%------------------------------
%tasya mahāśūnyacakrasya madhye ūrdhvamukham iti raktavarṇaṃ sakalaśobhāspadam    \E
%tasya mahāśūnyacakrasya madhye ūrdhvamukham iti raktavarṇa--sakalaśobhāspadaṃ     \P
%tasya mahāśūnyacakrasya madhye ūrdhvamukhem iti raktavarṇaṃ sakalaśobhāspadaṃ // \B    
%tasya mahāśūnyacakrasya madhye ūrdhvamukham iti raktavarṇaṃ sakalaśobhāspadaṃ // \L
%tasya mahāśūnyacakramadhye     ūrdhvamukhaṃ atiraktavarṇaṃ  sakalaśobhāspadaṃ /   \N1 ->!!!
%tasya mahāśūnyacakramadhye     ūrdhvamukhaṃ atiraktavarṇaṃ  sakalaśobhāspadaṃ     \N2
%tasya mahāśūnyacakramadhye     ūrdhvamukhaṃ atiraktavarṇaṃ  sakalaśobhāspadaṃ /   \D
%tasya mahāśūnyacakramadhye     ūrdhvamukhaṃ atiraktavarṇaṃ  sakalaśobhāspadaṃ     \U1
%tasya mahāśūnyacakrasya        urdhvamukham-ativarṇaṃ       sakalaśobhanāsyadaṃ / \U2 
%------------------------------
%anekakalyāṇapūrṇaṃ sahasradalan      ekaṃ kamalaṃ  varttate / \E
%anekakalyāṇapūrṇaṃ sahasradalaṃ      ekaṃ kamalaṃ  vartate    \P
%anekakalyāṇapūrṇa--sahasradalaṃ      ekaṃ kamalaṃ  vartato    \B
%anekakalyāṇapūrṇaṃ sahasradalaṃ      ekaṃ kamalaṃ  vartate    \L
%anekakalyāṇapūrṇaṃ sahasradalaṃ      eka--kamalaṃ  varttate   \D
%anekakalyāṇapūrṇaṃ sahasradalaṃ      ekaṃ kamalaṃ  vartate    \N1
%anekakalyāṇapūrṇa--sahasradalaṃ      ekaṃ kamalaṃ  varttate    \N2
%anekakalyāṇapūrṇaṃ sahasradalaṃ           kamalaṃ  vartate /   \U1
%anekakalyāṇapūrṇaṃ // sahasradalaṃ   ekaṃ kamalaṃ  vartate / \U2
%Fragezeichen in |nepal ... schreiber Einfügung? 
%------------------------------
%In the middle of the \textit{mahāśūnyacakra} exists one lotus facing upward, very red in color with a thousand petals - an abode of brilliance and wholeness.
%------------------------------
tasya mahāśūnya\app{\lem[wit={X},alt={°cakramadhye}]{cakramadhye}
  \rdg[wit={E,P,B,L}]{°cakrasya madhye}
  \rdg[wit={U2}]{°cakrasya}}
\app{\lem[wit={X},alt={ūrdhvamukham}]{ūrdhvamukha\skp{m-a}}
  \rdg[wit={E,P,L}]{ūrdhmukham}
  \rdg[wit={U2}]{urdhvamukham}
  \rdg[wit={B}]{ūrdhvamukhem}}
\app{\lem[wit={X}]{\skm{m-a}tiraktavarṇaṃ}
  \rdg[wit={E,L,B}]{iti raktavarṇaṃ}
  \rdg[wit={P}]{iti raktavarṇa°}
  \rdg[wit={U2}]{ativarṇaṃ}}
sakala\app{\lem[wit={ceteri},alt={°śobhāspadaṃ}]{śobhāspadaṃ}
  \rdg[wit={E}]{°śobhāspadam}
  \rdg[wit={U2}]{°śobhanāsyadaṃ}}
\app{\lem[wit={ceteri}]{anekakalyāṇapūrṇaṃ}
  \rdg[wit={B,N2}]{°pūrṇa°}}
sahasradalaṃ
\app{\lem[wit={ceteri}]{ekaṃ}
  \rdg[wit={D}]{eka°}
  \rdg[wit={U1}]{\om}}
kamalaṃ
\app{\lem[wit={ceteri}]{vartate}
  \rdg[wit={B}]{vartato}}/
\end{prose}
  \end{edition}
  \begin{translation}
    \ekddiv{type=trans}
    \begin{tlate}
      In its middle, appearing as a streak in the form of smoke and fire exists such a unique [divine] form of the soul (\textit{puruṣa}). Of her exists no end nor a beginning. Due to meditation on the form, uninterrupted observation of both the coming and going of the soul in space occurs. Affliction from the earth-element does not arise [anymore] even if one is in the middle of the earth. He constantly sees everything in front of his eyes and becomes separated [from the material world?]. The force of life increases eminently. 
      \end{tlate}
    \bigskip
    \centerline{\textrm{\small{[Ninth Cakra]}}}
    \bigskip
    \begin{tlate}
      Now the divisions of the ninth \textit{cakra} are explained. The designation of it is ``the \textit{cakra} of the great void''. Above that, there is no other. Therefore it is declared to be the \textit{cakra} of the great perfection. [Another] such name is ``(divine) seat of Pūrṇagiri''. In the middle of the \textit{mahāśūnyacakra} exists one lotus facing upward, very red, with a thousand petals - an abode of brilliance and wholeness.
      \vspace*{\fill} 
    \end{tlate}
  \end{translation}
     \ekdpb*{}
\end{alignment}
%%%%%%%%%%%%%%%%%%%%%%%%%%%%%%%%%%%%%%%%%%
%%%%%%%%%%%%%%%%%%%%%%%%%%%%%%%%%%%%%%%%%%
%%%%%%%%PAGEBREAK%%%%%%%PAGEBREAK%%%%%%%%%
%%%%%%%%%%%%%%%%%%%%%%%%%%%%%%%%%%%%%%%%%%
%%%%%%%%%%%%%%%%PAGEBREAK%%%%%%%%%%%%%%%%%
%%%%%%%%%%%%%%%%%%%%%%%%%%%%%%%%%%%%%%%%%%
%%%%%%%%PAGEBREAK%%%%%%%PAGEBREAK%%%%%%%%%
%%%%%%%%%%%%%%%%%%%%%%%%%%%%%%%%%%%%%%%%%%
%%%%%%%%%%%%%%%%%%%%%%%%%%%%%%%%%%%%%%%%%%
%%%%%%%%%%%%%%%%%%%%%%%%%%%%%%%%%%%%%%%%%%
%%%%%%%%%%%%%%%%%%%%%%%%%%%%%%%%%%%%%%%%%%
%%%%%%%%PAGEBREAK%%%%%%%PAGEBREAK%%%%%%%%%
%%%%%%%%%%%%%%%%%%%%%%%%%%%%%%%%%%%%%%%%%%
%%%%%%%%%%%%%%%%PAGEBREAK%%%%%%%%%%%%%%%%%
%%%%%%%%%%%%%%%%%%%%%%%%%%%%%%%%%%%%%%%%%%
%%%%%%%%PAGEBREAK%%%%%%%PAGEBREAK%%%%%%%%%
%%%%%%%%%%%%%%%%%%%%%%%%%%%%%%%%%%%%%%%%%%
%%%%%%%%%%%%%%%%%%%%%%%%%%%%%%%%%%%%%%%%%%
%%%%%%%%%%%%%%%%%%%%%%%%%%%%%%%%%%%%%%%%%%
%%%%%%%%%%%%%%%%%%%%%%%%%%%%%%%%%%%%%%%%%%
%%%%%%%%PAGEBREAK%%%%%%%PAGEBREAK%%%%%%%%%
%%%%%%%%%%%%%%%%%%%%%%%%%%%%%%%%%%%%%%%%%%
%%%%%%%%%%%%%%%%PAGEBREAK%%%%%%%%%%%%%%%%%
%%%%%%%%%%%%%%%%%%%%%%%%%%%%%%%%%%%%%%%%%%
%%%%%%%%PAGEBREAK%%%%%%%PAGEBREAK%%%%%%%%%
%%%%%%%%%%%%%%%%%%%%%%%%%%%%%%%%%%%%%%%%%%
%%%%%%%%%%%%%%%%%%%%%%%%%%%%%%%%%%%%%%%%%%
\begin{alignment}[
  texts=edition[class="edition"];
  translation[class="translation"],
  ]
  \begin{edition}
    \ekddiv{type=ed}
    \begin{prose}
      \noindent
%---------------------
%yasya           parimalo manaso vacaso na gocaraḥ // \E
%yasya           parimalo manasā vacasā na gocaraḥ /  \P
%yasya           parimalo manasā vacasā    gocaraḥ /  \L
%yasya           parimalo manasā vacasā na gocaraḥ /  \B
%yasya           parimalo manasā vacasā na gocaraḥ /  \N1
%yasya           parimalo manasā vacasā na gocara /   \N2
%yasya           parimalo manasā vacasā na gocaraḥ /  \D
%yasya           parimalo vacasā manasā na gocaraḥ    \U1
%yasya kamalasya parimalo manasā vācā   na gocara ..  \U2
%---------------------
%Whose fragrance is not in range by mind and speech. 
%Dessen Duft ist nicht in Reichweite von Geist und Sprache. 
%---------------------
\app{\lem[wit={ceteri}]{yasya}
  \rdg[wit={U2}]{yasya kamalasya}}
\app{\lem[type=conjecture, resp=egoscr]{parimāṇaṃ vaktuṃ}
  \rdg[wit={ceteri}]{parimalo}} 
\app{\lem[wit={B,D,L,P,N1,N2}]{manasā vacasā}
  \rdg[wit={E}]{manaso vacaso}
  \rdg[wit={U1}]{vacasā manasā}
  \rdg[wit={U2}]{manasā vācā}}
\note[type=philcomm, labelb=103, lem={°manaso vacaso}]{All manuscripts and the printed edition share the reading \textit{parimalo} but most of them keep the grammatically incorrect instrumental \textit{manasā vācasā}. Only the variant of the printed edition arrives at a grammatically correct text. However, this seems to be conjectured by the Paṇḍit who edited the text. The source text reveals a more meaningful sentence and provides a plausible conjecture.}
\app{\lem[wit={ceteri}]{na}
  \rdg[wit={L}]{\om}}
\app{\lem[wit={ceteri}]{gocaraḥ}
  \rdg[wit={N2,U2}]{gocara}}/
%---------------------
%tasya kamalasya madhye trikoṇarūpa-ikā karṇikā varttate /    \E
%tasya kamala----madhye trikoṇārūpā ekā karṇikā varttate/ \P
%tasya kamalasya madhye trikoṇarūpā ekā karṇikā varttate/     \L
%tasya kamalasya madhye trikoṇarūpā ekā karṇikā varttate/     \B
%tasya kamalasya madhye trikoṇarūpā eka karṇikā varttate/     \N1
%tasya kamalasya madhye trikoṇarūpā eka karṇikā varttate/     \N2
%tasya kamalasya madhye trikoṇarūpā ekā karṇikā varttate/     \D
%tasya kamalasya madhye trikoṇarūpā ekā karṇikā vartate       \U1
%tasya kamalasya madhye trikoṇarūpā ekā karṇikā vartate //    \U2
%---------------------
%In the middle of this lotus exists one pericarp having the shape of a triangle. 
%------------------------------
tasya
\app{\lem[wit={ceteri}]{kamalasya}
  \rdg[wit={P}]{kamala°}}
madhye
\app{\lem[wit={E}]{trikoṇarūpaikā}
  \rdg[wit={ceteri}]{trikoṇārūpā ekā}
  \rdg[wit={N1,N2}]{trikoṇārūpā eka}}
karṇikā vartate\dd{}
%------------------------------
%tatkarṇikāmadhye saptadaśī         niraṃjanarūpā kalā varttate/ \E
%tatkarṇikāmadhye saptadaśireṇa ekā niraṃjanarūpā kalā vartate// \L
%tatkarṇikāmadhye saptadaśireṇa ekā niraṃjanarūpā kalā vartate// \B
%tatkarṇikāmadhye saptadaśī     ekā niraṃjanarūpā kalā vartate// \P
%tatkarṇikāmadhye saptadaśī     ekā niraṃjanarūpā kalā vartate// \N1
%tatkarṇikāmadhye saptadaśī     ekā niraṃjanarūpā kalā vartate/  \N2
%tatkarṇikāmadhye saptadaśī     ekā niraṃjanarūpā kalā vartate// \D
%tatkarṇikāmadhye saptadaśī     ekā niraṃjanarūpā kalā vartate  \U1
%tatkarṇikāmadhye saptadaśī     eka niraṃjanarūpā kalā varttate/ \U2
%---------------------
%In the middle of the pericarp exists one seventeenth digit in the shape of a immaculé form.
%---------------------
\note[type=source, labelb=104, lem={\textbf{Re}}]{PT\textsuperscript{qcr \cdot YSV} (Ed. p. 833):  ūrddhvavakraṃ mahāvaktre (\textit{mahāvaktraṃ} YK\textsuperscript{ccn \cdot YSV} 1.274 Ed. p. 22) varṇaśobhāpadaṃ mahat | sarvakalyāṇasampūrṇamasya tulyaṃ na vidyate | parimāṇaṃ vaktam (\textit{vaktum} YK\textsuperscript{ccn \cdot YSV} 1.275 Ed. p. 22) asya manasā vacasā na hi | trikoṇakarṇikā tatra (\textit{°tantraṃ} YK\textsuperscript{ccn \cdot YSV} 1.276 Ed. p. 22) varttate jagad īśvari | kalā saptadaśī tatra varttate parameśvari | nirañjanakalā sā tu koṭisūryasamaprabhā | koṭicandraprabhā caiva śītoṣṇādivivarjitā | asya dhyānāt sādhakasya manoduḥkhaṃ bhaven na hi | anantaparamānandasthānaṃ jñeyaṃ tadūrddhvataḥ (\textit{tadarddhataḥ} YK\textsuperscript{ccn \cdot YSV} 1.278 Ed. p. 22) | ūrddhvagatakalā tatra tasya dhyānād bhaved iti | iti siddhirājayogaṃ strīṇāṃ bhogaṃ mahāsukham | gītavādyavinodādi saśivaṃ varddhate kṣitau | dhyānaṃ nirantarañ cāsya puṇyapāpe sthire (\textit{sthirau } YK\textsuperscript{ccn \cdot YSV} 1.280 Ed. p. 22) na hi | nijarūpasya dṛṣṭiḥ syād dūrasyārthañ ca paśyati ||}
\note[type=testium, labelb=101, lem={\textbf{Cie}}]{\textit{Yogasaṃgraha} IGNCA 30020 folio 2r. ll. 9-11: tasyāṃ karṇikāyāṃ saptadaśī niraṃjanarūpā koṭisūryaprabhā satī uṣṇabhava hīnā koṭicandrasamasītalaikākalāsti | tasyāṃ anaṃta paramānaṃtaparamānaṃdānāṃ sthānaṃ tasyāḥ kalāyā dhyānakaraṇāt sādako yadyādi śati tatra bhavati|}
tatkarṇikāmadhye
\app{\lem[wit={ceteri}]{saptadaśī}
  \rdg[wit={L,B}]{saptadaśireṇa}}
\app{\lem[wit={ceteri}]{ekā}
  \rdg[wit={E}]{\om}}
nirañjanarūpā kalā varttate/
%---------------------
%koṭisūryasamaprabhaṃ kalāyās tejo vartate /    \E
%koṭisūryasamaprabhā kalāyās tejo vartate /     \L
%koṭisūryasamaprabhā kalāyās tejo vartate /     \B
%koṭisūryasamaprabha kalāyās tejo vartate /     \P
%koṭisūryasamaprabhaṃ kalāyās tejo vartate /    \N1
%koṭisūryasamaprabhaṃ kalāyā  tejo varttate //  \N2
%koṭisūryasamaprabhaṃ kalāyās tejo vartate /    \D
%koṭisūryasadṛṣaprabhaṃ kalāyās tejo vartate /  \U1
%koṭisūryasamaprabhā // kalāyās tejo varttate / \U2
%---------------------
%A light of the part exists shining like a thousand suns. 
%------------------------------
koṭisūrya\app{\lem[alt={°samaprabhaṃ}, wit={ceteri}]{samaprabhaṃ}
  \rdg[wit={L,B,U2}]{samaprabhā}
  \rdg[wit={P}]{samaprabha}
  \rdg[wit={U1}]{sadṛṣaprabhaṃ}}
kalāyās-tejo vartate/
%------------------------------
%param udbhavo nāsti /     \E
%parim uṣṇabhavo nāsti /   \P
%parim uṣṇabhavo nāsti /   \L
%parim uṣṇabhavo nāsti /   \B
%parim uṣṇabhāvo nāsti /   \N1
%para  uṣṇabhāvo nāsti     \N2
%parim auṣṇabhāvo nāsti /  \D
%paraṃ uṣṇabhāvo nāsti     \U1
%param uṣṇabhāvo nāsti /   \U2
%---------------------
%[But] excessive heat is not arising. 
%------------------------------
\app{\lem[alt={param},wit={E,U1,U2}]{para\skp{m-u}}
  \rdg[wit={U1}]{paraṃ}
  \rdg[wit={N2}]{para}
  \rdg[wit={ceteri}]{parim}
}\app{\lem[wit={ceteri}, alt={uṣṇabhāvo}]{\skm{m-u}ṣṇabhāvo}
  \rdg[wit={P,L,B}]{uṣṇabhavo}
  \rdg[wit={D}]{auṣṇabhāvo}
  \rdg[wit={E}]{udbhavo}
}
nāsti/
%------------------------------
%koṭicandrasamaprabhā    śītalaṃ paraṃ   śītabhāvo   nāsti / \E
%koṭicandrasamaprabhā    śītalaṃ paraṃ   śītabhavo   nāsti / \P
%\om /                                                      \L
%koṭicandrasamaprabhā    śītalaṃ paraṃ   śītabhavo   nāsti / \B
%koṭicandrasamaprabhaṃ   śītalaparaṃ         bhavo   nāsti / \N1
%koṭicandrasamaprabhaṃ   śītalapara----------bhavo   nāsti // \N2
%koṭicaṃdrasamaprabhaṃ   śītalaparaṃ         bhavo   nāsti / \D
%koṭicaṃdrasamaṃ prabhaṃ śītalaṃ paraṃ       bhavo   nāsti / \U1
%koṭicaṃdrasamaprabhā    śītalaṃ paraṃ śītalabhāvo   nāsti / \U2
%---------------------
%Shining like a thousand moons, excess of cold is not arising.
%---------------------
koṭicandra\app{\lem[alt={°samaprabhaṃ},wit={N1,N2,D}]{samaprabhaṃ}
  \rdg[wit={Y}]{°samaprabhā}
  \rdg[wit={U1}]{°samaṃ prabhaṃ}
  \rdg[wit={L}]{\om}}
\app{\lem[wit={N1,D}]{śītalaparaṃ}
  \rdg[wit={ceteri}]{śītalaṃ paraṃ}
  \rdg[wit={N2}]{śītalapara}
  \rdg[wit={L}]{\om}}
\app{\lem[wit={ceteri}]{bhāvo} 
  \rdg[wit={E,P,B}]{śītabhāvo}
  \rdg[wit={U2}]{śītalabhāvo}
  \rdg[wit={L}]{\om}}
nāsti/
%------------------------------
%asyāḥ kalāyā   dhyānayogāt    sādhakasya manasi duḥkhaṃ na bhavati / \E
%asyāḥ kalādhyānayogāt         sādhakasya manasi duḥkhaṃ na bhavati / \P
%asyāḥ kalāyāḥ  dhyānakaraṇāt  sādhakasya manasi duḥkhaṃ na bhavati / N1
%asyā kalāyā    dhyānakaraṇāt  sādhaka----manasi duḥkhaṃ na bhavati / N2
%asyāḥ kalāyāḥ  dhyānakaraṇāt  sādhakasya manasi duḥkhaṃ na bhavati / D
%
%asyāḥ kalāyā   dhyānayogāt    sādhakasya manasi duḥkhaṃ bhavati /B
%asyāḥ kalāyā   dhyānayogāt    sādhakasya manasi duḥkhaṃ bhavati /L
%asyāḥ kalāyā   dhyānakaraṇāt/ sādhakasya manasi duḥkhaṃ na bhavati / U1
%asyā  kalāyāḥ  dhyānayogāt//  sādhakasya manasi duḥkhaṃ na bhavati // \U2
%atrastāne 'haṃ devatā// sohaṃ śaktiḥ// ātmāṛṣiḥ// mokṣamārhaḥ// haṃbhrahmordhaṃ// haṃcakra iti// agnicakre sakaro bhavatī// prāṇīrūḍho bhave jjīva ārohaty avarohati bhavaguhāsthānaṃ pitavarṇaṃ// koṭisūryapratikāśaṃ tejaḥ sadoditaprabhā śīvodevatā// mūlamāyā śaktiḥ// hara ātmālayāvsthā dhvanisthirānādātmako khaṃḍa 'dhvani// adhorāmudrā// mūlamāyā// prakṛtidehaḥ// vāṅmanogocaraḥ// niḥprapaṃcaḥ// niḥsaṃśayaḥ// nistaraṃganirlepalakṣaṃ laya// dhyānasamādhi 
%---------------------
%asyāḥ kalāyā dhyānakaraṇāt\varc{\emend kalāyāḥ dhyānakaraṇāt \nepal \dehlia}{kalāyā dhyānayogāt \nepal \dehlia kalādhyānayogāt \pune} sādhakasya manasi duḥkhaṃ na\varc{na \edprint \pune \nepal \dehlia}{\om \oxford \lalchand} bhavati /
%Due to the exercise of meditation upon the digit suffering does not arise in the mind of the practitioner (anymore). 
%------------------------------
\app{\lem[wit={ceteri}]{asyāḥ}
  \rdg[wit={N2,U2}]{asyā}}
\app{\lem[wit={N2,U1}]{kalāyā}
  \rdg[wit={N1,D}]{kalāyāḥ}
  \rdg[wit={E,B,L}]{kalāyā}
  \rdg[wit={U2}]{kalāyāḥ}
  \rdg[wit={P}]{\om}}
dhyāna\app{\lem[wit={X}, alt={°karaṇāt}]{karaṇāt}
  \rdg[wit={Y}]{°yogāt}}
\app{\lem[wit={ceteri}]{sādhakasya}
  \rdg[wit={N2}]{sādhaka°}}
duḥkhaṃ
\app{\lem[wit={ceteri}]{na}
  \rdg[wit={B,L}]{\om}}
bhavati/
%%%%%%%%%%%%%
%%%%%%%%%%%%
%%%%%%%%%%%%
%%%%%%%%%%%%
%%%%%%%%%%%%
\extra{atra
   \app{\lem[type=emendation, resp=egoscr]{sthāne}
    \rdg[wit={U2}]{stāne}} 'haṃ devatā\dd{}
  sohaṃ śaktiḥ\dd{}
  ātmāṛṣiḥ\dd{}
  \app{\lem[type=emendation, resp=egoscr]{mokṣo}
    \rdg[wit={U2}]{mokṣa°}} mārgaḥ\dd{}
   \app{\lem[type=emendation, resp=egoscr]{ahaṃ brahmordhvaṃ}
    \rdg[wit={U2}]{haṃ brahmordhaṃ}}\dd{}
   \app{\lem[type=emendation, resp=egoscr]{ahaṃ cakra iti}
     \rdg[wit={U2}]{haṃcakra iti}}\dd{}
   agnicakre
   \app{\lem[type=emendation, resp=egoscr]{sakāro}
     \rdg[wit={U2}]{sakaro}}
   \app{\lem[type=emendation, resp=egoscr]{bhavati}
     \rdg[wit={U2}]{bhavatī}}\dd{}
   prāṇī rūḍho bhavej-jīva ārohaty-avarohati\dd{}
bhavaguhā sthānaṃ\dd{}
   \app{\lem[type=emendation, resp=egoscr]{pitaṃ}
     \rdg[wit={U2}]{pita°}} varṇaṃ\dd{}
   koṭisūryapratikāśaṃ tejaḥ\dd{}
   \app{\lem[type=emendation, resp=egoscr]{sadoditā}
     \rdg[wit={U2}]{sadodita°}} prabhā\dd{}
   \app{\lem[type=emendation, resp=egoscr]{śivo}
     \rdg[wit={U2}]{śīvo}} 
   devatā\dd{}
   mūlamāyā śaktiḥ\dd{}
   \app{\lem[type=emendation, resp=egoscr]{harātmālayāvasthā}
     \rdg[wit={U2}]{hara ātmālayāvasthā}}\dd{}
   dhvanisthirānādātmako \app{\lem[type=emendation, resp=egoscr]{'khaṇḍadvaniḥ}
     \rdg[wit={U2}]{khaṃḍadhvani}}\dd{} 
   aghorā mudrā\dd{}
   \app{\lem[type=emendation, resp=egoscr]{mūlā} %macht diese emdendation wirklich Sinn? 
     \rdg[wit={U2}]{mūla°}} māyā\dd{}
   \app{\lem[type=emendation, resp=egoscr,alt={prakṛtir}]{prakṛti\skp{r-de}}
     \rdg[wit={U2}]{prakṛti°}}\skm{r-de}haḥ\dd{}
   vāṅmano 'gocaraḥ\dd{} %%
   niḥprapañcaḥ\dd{}
   niḥsaṃśayaḥ\dd{}
   nistaraṃganirlepalakṣaṃ %%%see pw Vol. 3, S. 229 for nistaranga
  \app{\lem[type=emendation, resp=egoscr]{layo}
     \rdg[wit={U2}]{laya}}
   \app{\lem[type=emendation, resp=egoscr]{dhyānaḥ samādhiḥ}
     \rdg[wit={U2}]{dhyānasamādhi}}\dd{}}
%\extra{Here at this location the ``I''(\textit{aham}) is the deity. The ``he is I'' (\textit{so 'ham}) is the power. This self is the Ṛṣi. The path is liberation. Brahma is the I above. ``I'm a circle''. In the circle of fire is the letter "sa". [There?] life arises, the living soul ascends and decends. The place is the hidden place of being. The colour is yellow. The light is the shine of ten million suns. The shine is always and visible. Śiva is the deity. The power is primordial illusion. The state is the dissolution of the self into Hara\footnote{Epiphet of Śiva.}. The transcendental sound has the nature of a sound with stable resonance. The seal is the ``fearless''. The illusion is the root. The body is the original matter. It is not within reach of speech and mind. It is without delusion. It is without doubt. The unaffected and undefiled goal is dissolution, meditation [and] final absorption.}
    \end{prose}
  \end{edition}
  \begin{translation}
    \ekddiv{type=trans}
    \begin{tlate}
\ldots  It is not possible to express the seize of it with mind and speech. In the middle of this lotus exists one pericarp with a triangle shape. In the middle of the pericarp exists the seventeenth digit in having an immaculé form. There is a light of the digit, shining like a thousand suns., [but] excessive heat is not arising. Shining like a thousand moons, excess of cold is not arising. \extra{Here at this location the ``I''(\textit{aham}) is the deity. The ``he is I'' (\textit{so 'ham}) is the power. This self is the Ṛṣi. The path is liberation. Brahma is the I above. ``I am a circle''. In fire-area is the letter "sa". [There?] life arises, and the soul ascends and descends.\footnote{Find parallels of the hemistich.} The place is the hidden place of being. The colour is yellow. The light is the shine of ten million suns. The shine is always visible. Śiva is the deity. The primordial illusion is the power. The state is the dissolution of the self into Hara\footnote{Epiphet of Śiva.}. The transcendental sound has the nature of a sound with stable resonance. The ``fearless'' is the seal. The illusion is the root. The original matter is the body. Speech and mind are the range. Without delusion, without doubt, the unaffected and undefiled goal is dissolution, meditation [and] final absorption.}
    \end{tlate}
  \end{translation}
\end{alignment}
\begin{alignment}[
  texts=edition[class="edition"];
  translation[class="translation"],
  ]
  \begin{edition}
    \ekddiv{type=ed}
    \begin{prose}
    \noindent
%---------------------
%tatrordhvaśaktiḥ / \E
%tatordhvaśaktiḥ \P
%rdhaśakti ardhaśakti \B
%rdhaśakti ardhaśakti \L
%tatrordhvaśaktiḥ / \N1
%tatra ūrdhva śaktiḥ / \D
%tatra ūrdhva śakti / \N2
%urdhvaśaktir         \U1
%tatrordhvaśaktiḥ// \U2
%---------------------
%There above is \textit{śakti},
%------------------------------
\app{\lem[wit={E,N1,U2}]{tatrordhvaśaktiḥ}
  \rdg[wit={P}]{tatordhvaśaktiḥ}
  \rdg[wit={U1}]{urdhvaśaktir}
  \rdg[wit={D}]{tatra ūrdhva śaktiḥ}
  \rdg[wit={N2}]{tatra ūrdhva śakti}
  \rdg[wit={B,L}]{rdhaśakti ardhaśakti}}/
%------------------------------
%etādṛśī  saṃjñā   ekā kalā vartate / \E
%ekādaśā  saṃjñā   ekā kalā vartate   \P
%etādṛśī  saṃjñā   ekā kalā vartate /  \N1
%etādṛśī  saṃjñā   ekā kalā varttate / \N2
%etādṛsaṃ saṃjñā   ekā kalā vartate / \D
%ekādaśā  saṃjñā   ekā kalā vartate / \B
%ekādaśā  saṃjñā   ekā kalā vartate / \L
%etādṛśī  saṃjñakā ekā kalā vartate /  \U1
%etādṛśā  saṃjñā   ekā kalā vartate/ \U2 
%---------------------
%Being designated as such she is one single digit. 
%------------------------------
\app{\lem[wit={ceteri}]{etādṛśī}
  \rdg[wit={U2}]{etādṛśā}
  \rdg[wit={D}]{etādṛsaṃ}
  \rdg[wit={P,B,L}]{ekādaśā}}
\app{\lem[wit={ceteri}]{saṃjñā}
  \rdg[wit={U1}]{saṃjñakā}}
ekā kalā vartate/ 
%------------------------------
%asyāḥ  kalāyā   dhyānakāraṇāt     puruṣo yadicchati / \E
%asyāḥ  kalāyā   dhyānakāraṇāt     puruṣo yadicchati ?Zeichen? \P
%asyāḥ  kalāyā   dhyānakāraṇāt     puruṣo yadicchati  tad bhavati \N1
%tasyāḥ kalāyāḥ  dhyānakāraṇāt     puruṣo yadicchati  tad bhavati \N2
%asyāḥ  kalāyā   dhyānakāraṇā      puruṣo yadicchati  tad bhavati \D
%asyāḥ  kalāyā   dhyānakāraṇāt /   puruṣo yadicchati / \B
%asyāḥ  kalāyā   dhyānakāraṇāt /   puruṣo yadicchati / \L
%asyā   kalāyā   dhyānakāraṇāt     puruṣo yadicchati tad bhavati vā \U1
%asyāḥ  kalāyāḥ  dhyānakāraṇāt //  puruṣo yadicchati // \U2
%---------------------
%Due to the exercise of meditation on this part the person manifests whatever he wishes for.
%------------------------------
\app{\lem[wit={ceteri}]{asyāḥ}
  \rdg[wit={U1}]{asyā}
  \rdg[wit={N2}]{tasyāḥ}}
\app{\lem[wit={ceteri}]{kalāyā}
  \rdg[wit={N2,U2}]{kalāyāḥ}}
\app{\lem[wit={ceteri}]{dhyānakāraṇāt}
  \rdg[wit={D}]{dhyānakāraṇā}}
puruṣo yad-icchati
\app{\lem[wit={N1,N2,D}, alt={tad bhavati}]{tad-bhavati}
  \rdg[wit={U1}]{tad bhavati vā}
  \rdg[wit={Y}]{\om}}/ 
%------------------------------
%tasya sukhabhogavataḥ / \E
%tasya sukhabhogavataḥ \P
%rājya-sukhabhogavataḥ \N1
%rājya-sukhabhogavataḥ \N2
%rājya-sukhabhogavṛtaḥ \D !!!
%tasya-khaṃ bhogavataṃ / \B
%tasya-sukhaṃ bhogavaṃtaṃ / \L
%rājya-sukhabhogavataḥ \U1
%tasya-sukhabhogavataḥ / \U2
%---------------------
%He is furnished with royal pleasure and enjoyment. 
%------------------------------
\note[type=testium, labelb=107, lem={\textbf{Cie}}]{\textit{Yogasaṃgraha} IGNCA 30020 folio 3v. ll. 1-4: rājyasukhabhogavatah̤ strī vilāsavataḥ saṃgītavinoda prekṣāvato pi sādhakasya śuklapakṣacaṃdravat pratidinaṃ tejaso vapuṣaś ca vṛddiḥ puṇyapāpasya śārbhāvaḥ nijasva rūpaprakāśasāmarthaṃ dūrasthapy arthasya samīpastham iva darśanaṃ ca bhavati | cha | tad uktaṃ tattvajñānapradīpikāyāṃ ||}
\note[type=philcomm, labelb=108, lem={rājyasukhabhoga°}]{Here ends the testimony of the \textit{Yogasaṃgraha}  IGNCA 30020.}
\app{\lem[wit={D}]{rājyasukhabhogavṛtaḥ}
  \rdg[wit={N1,N2,U1}]{rājyasukhabhogavataḥ}
  \rdg[wit={E,P,U2}]{tasya sukhabhogavataḥ}
  \rdg[wit={B}]{tasya khaṃ bhogavataṃ}
  \rdg[wit={L}]{tasya sukhaṃ bhogavaṃtaṃ}}/
%------------------------------
%strīmadhye     vilāsavataḥ    saṃgītavilāsavataḥ vinodaprekṣāvataḥ        puruṣasya pratidinaṃ śuklapakṣe candrakalāvat   kalā     vardhate/   \E
%strīmadhye     vilāsavataḥ    saṃgītavinodaprekṣāvataḥ              eva   puruṣasya pratidinaṃ śuklapakṣe candrakalāvat   kalā     vardhate /  \P
%strīmadhye     vilāsavaṃtaṃ   saṃgītaṃ prekṣāvatāḥ //               evaṃ  puruṣasya pratidinaṃ śuklapakṣe caṃdrakalāvat / kalā     vartate /   \L
%strīmadhye     vilāsavaṃtaṃ   saṃgītaṃ vinodavaṃtaṃ prekṣāvaṃtāḥ // eva   puruṣasya pratidinaṃ śuklapakṣe caṃdrakalāvat / kalā     vartate /   \B
%strīmadhye     vilāsavataḥ    saṃgītavinodaprekṣyāvataḥ             evaṃ  puruṣasya pratidinaṃ śuklapakṣe candrakalā vṛddhivato?   vardhate / \N1
%śrī strīmadhye vilāsavataḥ    saṃgītavinodaprekṣāvataḥ              evaṃ  puruṣasya pratidinaṃ śuklapakṣa candrakalā vṛddhi vaṃto  varttate /  \N2
%strīmadhye     vilāsavataḥ // saṃgītavinodaprekṣyāvataḥ //          evaṃ  puruṣasya pratidinaṃ śuklapakṣe candrakalā vṛddhivato    vardhate / \D
%strīmadhye     vilāśavataḥ    saṃgītavinodaprekṣyāvataḥ             eka   puruṣasya pratidinaṃ śuklapakṣe caṃdrakalā vṛddhir       varddhate / \U1
%strīmadhye     vilāsavata     saṃgītavinodaprekṣāvata//             evaṃ  puruṣasya pratidinaṃ śuklapakṣe candrakalāvat   kalā     varttate/   \U2
%---------------------
%(Selbst) bei einem Menschen, der sich inmitten von Frauen vergnügt, (und) ein Musikvergnügen
%ansieht, wächst täglich die Kraft (kalā = śakti?) wie die "kalā" (Phase) des Mondes in der hellen Monatshälfte.
%The \textit{kalā} of a person grows daily, like the \textit{kalā} of the moon in the bright half of the month, even amusing oneself amongst women and watching a musical pleasure.
%(Even) amusing oneself amongst women, and watching musical pleasures, the \textit{kāla} of the person grows daily like the \textit{kalā} of the moon in the bright half of the month. 
%------------------------------
\app{\lem[wit={ceteri}]{strīmadhye}
  \rdg[wit={N2}]{śrī strīmadhye}}
\app{\lem[wit={ceteri}]{vilāsavataḥ}
  \rdg[wit={U2}]{vilāsavata°}
  \rdg[wit={L,B}]{vilāsavaṃtaṃ}} 
saṃgīta\app{\lem[wit={N1,D,U1},alt={°vinodaprekṣyāvataḥ}]{vinodaprekṣyāvataḥ}
  \rdg[wit={P,N2}]{°vinodaprekṣāvataḥ}
  \rdg[wit={U2}]{°vinodaprekṣāvata}
  \rdg[wit={B}]{°ṃ vinodavaṃtaṃ prekṣāvaṃtāḥ}
  \rdg[wit={E}]{°vilāsavataḥ vinodaprekṣāvataḥ}
  \rdg[wit={L}]{°ṃ prekṣāvatāḥ}}
 \app{\lem[wit={P,B}]{eva}
  \rdg[wit={ceteri}]{evaṃ}
  \rdg[wit={U1}]{eka}}
puruṣasya pratidinaṃ śuklapakṣe
candrakalā\app{\lem[wit={Y},alt={°vat kalā}]{vat kalā}
  \rdg[wit={N1,D}]{vṛddhivato}
  \rdg[wit={N2}]{vṛddhi vaṃto}
  \rdg[wit={U1}]{vṛddhir}}
\app{\lem[wit={D ,E,P,N1,U1}]{vardhate}
  \rdg[wit={ceteri}]{vartate}}/
%------------------------------
%puṇyapāpe  'sya śarīraṃ   na spṛśataḥ /    \E
%\om                                     \P
%puṇyapāpe  asya śarīrena     spṛśataḥ /      \N1
%puṇyapāpe  asya śarīrena     spṛśataḥ /      \N2
%puṇyapāpe  asya śarīrena     spṛśataḥ /      \D
%puṇyapāpe  asya śarīrasya na spṛśataḥ // \B
%puṇyapāpe  asya śarīrasya na spṛśataḥ // \L
%puṇyapāpau asya śarīrena     spṛśāt         \U1
%puṇyapāpe  asya śarīraṃ   na spṛśataḥ // \U2
%---------------------
%puṇyapāpe\varc{puṇyapāpe \edprint \lalchand \oxford \nepal \dehlia}{\om \pune} 'sya\varc{'sya \edprint}{asya \nepal \dehlia \oxford \lalchand \om \pune} śarīrasya\varc{śarīrasya \lalchand \oxford}{śarīraṃ \edprint śarīrena \nepal \dehlia \om \pune} na\varc{na \edprint \oxford \lalchand}{\om \nepal \dehlia \pune} spṛśataḥ\varc{spṛśataḥ \edprint \lalchand \oxford \nepal \dehlia}{\om \pune} /
%---------------------
%His body is not affected by merit and sin. 
%------------------------------
\app{\lem[wit={ceteri}]{puṇyapāpe}
  \rdg[wit={U1}]{puṇyapāpau}
\rdg[wit={P}]{\om}}
\app{\lem[wit={E}]{'sya}
  \rdg[wit={P}]{\om}
  \rdg[wit={ceteri}]{asya}}  
śarīr\app{\lem[wit={B,L}]{śarīrasya}
  \rdg[wit={X}]{śarīrena}
  \rdg[wit={E,U2}]{śarīraṃ}
  \rdg[wit={P}]{\om}}
\app{\lem[wit={E,B,L,U2}]{na}
  \rdg[wit={X,P}]{\om}}
spṛ\app{\lem[wit={ceteri},alt={°śataḥ}]{śataḥ}
  \rdg[wit={U1}]{°śāt}}/
%------------------------------
%                          nirantaradhyānakaraṇāt     nijasvarūpaṃ prakāśanasāmarthyaṃ bhavati / \E
%                          \om until .....            nijasvarūpaprakāśasāmarthyaṃ     bhavati / \P
%                          niraṃtaraṃ dhyānakaraṇāt   nijasvarūpaprakāśasāmarthyaṃ     bhavati / \B
%                          niraṃtaraṃ dhyānakaraṇāt// nijasvarūpaprakāśasāmarthyaṃ     bhavati / \L
%                          nirantaradhyānakaraṇāt /   nijasvarūpaprakāśasāmarthyaṃ     bhavati / \N1 <-----
%                          niraṃtaradhyānakaraṇāt /   nijasvarūpaprakāśasāmarthyaṃ     bhavati // \N2
%                          nirantaradhyānakaraṇāt /   nijasvarūpaprakāśasāmarthyaṃ     bhavati / \D
%                          nirantaradhyānakaraṇāt /   nijasvarūpaprakāśasāmarthyaṃ     bhavati    \U1
%evaṃ puruṣasya pratidinaṃ niraṃtaraṃ dhyānakaraṇāt   nijasvarūpaṃ prakāśanasāmarthyaṃ bhavati// \U2 
%---------------------
%Due to uninterrupted meditation the power of the light of the innate nature arises. 
%------------------------------
\app{\lem[wit={ceteri}]{nirantaradhyānakaraṇāt}
  \rdg[wit={B,L}]{niraṃtaraṃ dhyānakaraṇāt}
  \rdg[wit={U2}]{evaṃ puruṣasya pratidinaṃ niraṃtaraṃ dhyānakaraṇāt}
  \rdg[wit={P}]{\om}}
nijasvarūpa\app{\lem[wit={ceteri},alt={°prakāśa°}]{prakāśa}
  \rdg[wit={E,U2}]{°ṃ prakāśana°}
}sāmarthyaṃ bhavati/
%------------------------------
%dūrasthopi ca dūrasthavastu                   samīpa iva   paśyati // \E
%dūrasthamapi                                  samīpam iva  paśyati // \N1
%dūrasthamapi                                  samīpaṃ iva  paśyati // \N2
%dūrasthamapy-arthaṃ                           samīpa iva   paśyati // \D
%dūrasthamapi padārthaṃ                        samīpa iva   paśyati // \B
%dūrasthamapi parārthaṃ                        samīpa iva   paśyati // \L
%dūrasthamapi padārthaṃ                        samīpa iva   paśyati // \P
%dūrasthamapy-arthaṃ                           samīpam eva  paśyati // \U1
%dūrasthamapi bhavati //dūrasthamapi padārthaṃ samīpa iva   paśyati // \U2
%------------------------------
%dūrasthamapyarthaṃ\varc{dūrasthamapyarthaṃ \dehlia}{dūrasthamapi padārthaṃ \oxford \pune durasthamapi parārthaṃ \lalchand sūrastamapi \nepal ca dūrasthavastu \edprint} samīpa\varc{samīpa \dehlia \edprint \lalchand \oxford \pune}{samīpam \nepal} iva paśyati //
%------------------------------
%He sees remotely located objects as if they'd be near.
%------------------------------
dūra\app{\lem[wit={D,U1},alt={°stham apy arthaṃ}]{stham-apy-arthaṃ}
  \rdg[wit={B,P}]{°stham api padārthaṃ}
  \rdg[wit={L}]{°stham api parārthaṃ}
  \rdg[wit={E}]{°sthopi ca dūrasthavastu}
  \rdg[wit={N1,N2}]{°stham api}
  \rdg[wit={U2}]{°stham api bhavati || dūrastham api padārthaṃ}}
\app{\lem[wit={ceteri}]{samīpa}
  \rdg[wit={N1}]{samīpam}
  \rdg[wit={N2}]{samīpaṃ}
  \rdg[wit={U1}]{samīpam}}
\app{\lem[wit={ceteri}]{iva}
  \rdg[wit={U1}]{eva}} 
paśyati\dd{}
    \end{prose}
  \end{edition}
  \begin{translation}
    \ekddiv{type=trans}
    \begin{tlate}
Above that is the place of infinite supreme bliss. There above is power (\textit{śakti}). Being designated as such, she is one single digit. Due to the meditation exercise on this part, the person manifests whatever he wishes for. He is furnished with royal pleasure and enjoyment. [Even] amusing oneself amongst women and watching musical pleasures, the \textit{kāla} of the person grows daily like the \textit{kalā} of the moon in the bright half of the month. His body is not affected by merit and sin. Due to uninterrupted meditation, the power of the light of innate nature arises. He sees remotely located objects as if they were near. 
    \end{tlate}
  \end{translation}
\end{alignment}
\chapter{Bibliography}
 \label{sec:bibli}
   \clearpage
\newpage 
\thispagestyle{empty}
\quad  \addtocounter{page}{-1}

\printbibliography[heading=subbibintoc, title=Consulted Manuskcipts, keyword=codex]

\printbibliography[heading=subbibintoc, title=Printed Editions, keyword=printsource]

\printbibliography[heading=subbibintoc, title=Secondary Literature, keyword=seclit]

\printbibliography[heading=subbibintoc, title=Online Sources, keyword=onlinesource]


\end{document}
%-----------------------------  
%\begin{alignment}[
%    texts=edition[class="edition"];
%    translation[class="translation"],
%  ]
%\begin{edition}
% \ekddiv{type=ed}
%\begin{prose}homa\end{prose}
%\end{edition}
%\begin{translation}
%  \ekddiv{type=trans}
%  \begin{tlate}\end{tlate}
%   \end{translation}
% \end{alignment}
%
% 
%
%
%
%
%%%%deciphering last folio margin note of %N1!!!  
%\begin{otherlanguage}{english}
\chapter{Translation of the Yogatattvabindu}    
\ekddiv{type=trans}
\centerline{\textrm{\small{[Introduction]}}}
\bigskip
\begin{tlate}
Homage to Śrī Gaṇeśa. Now the methods of Rājayoga are laid down. This is the result of Rājayoga\footnote{This statement seems unconnected to the definition of rājayoga that follows.}: Rājayoga is that by which longterm durability of the body arises even amongst manifold royal pleasures even amongst the manifold royal entertainments and spectacle. This truly is Rājayoga. These are the varieties of this Rājayoga:
\noindent 1. Kriyāyoga, the Yoga of [mental] action; 2. Jñānayoga, the Yoga of knowledge; 3. Caryāyoga, the Yoga of wandering;\footnote{The first three Yogas allude to the four \textit{pāda}s of the Śaiva \textit{āgama}s; namely \textit{kriyā[pāda], caryā[pāda], yoga[padā]} and \textit{jñāna[pāda]}.\parencite[77]{nishvasa2015}.} 4. Haṭhayoga, the Yoga of force; 5. Karmayoga, the Yoga of deeds; 6. Layayoga, the Yoga of absorption; 7. Dhyānayoga, the Yoga of meditation, 8.Mantrayoga, the Yoga of Mantras; 9. Lakṣyayoga, the Yoga of fixation objects, 10. Vāsanāyoga, Yoga of mental residues; 11. Śivayoga, the Yoga of Śiva, 12. Brahmayoga, the Yoga of Brahman; 13. Advaitayoga, the Yoga of non-duality; 14. Siddhayoga, the Yoga of the Siddhas; 15. Rājayoga, the King of Yogas. These are the fifteen \textit{yoga}s.\footnote{At the current stage of research it is not clear if this list is a later addition by another scribe or, if indeed it originally stems from Rāmacandra. The list suggests a text following the order of yogas according to this list. However, the order of the yogas given in the list is not followed closely in the text.}
\end{tlate}
%%%%%%%%%%%%%
 \begin{tlate}
   \ekddiv{type=trans}
      \centerline{\textrm{\small{[Description of \textit{kriyāyoga}]}}}
      \bigskip
Now the characteristic of Kriyāyoga, the Yoga of [mental] action\footnote{In comparison to the Pātañjalean variant of Kriyāyoga, this variat consists of specific mental actions.} are described.
\paragraph{1.} This Yoga is liberation through [mental] action, it bestows success(\textit{siddhi}) in ones own body. Each wave the mind creates at the beginning of an action, of all those one shall withdraw oneself. Then Kriyāyoga ari
\paragraph{2.} Patience, discrimination, equanimity, peace, modesty, desireless: The Yogī who is endowed with these means is said to be a Kriyāyogī. 
\paragraph{3.} Envy, selfishness, cheating, violence, desire and intoxication, pride, lust, anger, fear, laziness, greed, error and impurity. 
\paragraph{4.} Attachment and aversion, indignation and idleness, impatience and dizzyness: Whoever doesn't experience these is called a Kriyāyogī.\footnote{The source of the four verses on Kriyāyoga is unknown.} \bigskip \bigskip

Patience, discrimination, equanimity, peace, contentment etc. are generated in his mind. He alone is called a Yogī of many actions (\textit{bahukriyāyogī})\footnote{The term \textit{bahukriyāyogī} seems to be unique in yoga literature.} Fraud, illusion, property,violence, craving, envy, ego, anger, anxiety, shame, greed, error, impurity, attachment, aversion, idleness, heterodoxy, false view, affection of the senses, sexual desire: He who diminishes these from day to day in is mind, he alone is called a Yogī of many actions (\textit{bahukriyāyogī}).
   \end{tlate}
   %%%%%%%%%%%%%%%%%%%%
   %%%%%%%%%%%%%%%%%%%%
   %%%%%%%%%%%%%%%%%%%%
   %%%%%%%%%%%%%%%%%%%%
  \ekddiv{type=trans}
        \bigskip
    \centerline{\textrm{\small{[Varieties of \textit{rājayoga}: Siddhakuṇḍalinīyoga and Mantrayoga]}}}
    \bigskip
    \begin{tlate}
      Now varieties of Rājayoga will be described. Which are these? One is Siddhakuṇḍalinīyoga\footnote{On the one hand it suprises that we find the term Siddhakuṇḍalinīyoga instead of Siddhayoga as given in the initial list, on the other hand it is suprising that this type of Yoga, given as the second last item in the Yoga taxnomy is introduced as the second type right after Kriyāyoga, which was the first item in the initial list as well as in the following material.What makes this term even more strange is the fact that \textit{kuṇḍaliṇī} is not mentioned at all in the following description of this type of Yoga.} [and one\footnote{It is not entirely clear if those are two different Yogas or one and the same type of Yoga. Just the pretty late witness U2 gives us a sort of description of Mantrayoga. Judging on the basis of U2 only one could translate ``One is Siddhakuṇḍalinīyoga being Mantrayoga.'' Judging by the contents given by the rest of the witnesses this passage leaves a big queastion mark.}] is Mantrayoga\footnote{It seems odd that Mantrayoga is mentioned in the same breath as Sidhdakuṇḍalinīyoga, even though it is not directly expressed in the following. Just the additional descriptions of witness U2, highlighted in a different colour than the main text, indirectly refers to a certain practice of Mantra which is \textit{japājapa} of the \textit{so 'haṃ} for a certain duration of the practioce of meditation that is presrcibed to be performed on every \textit{cakra}.}. These two Rājayogas are described [in the following]. At the location of the root-bulb exists one major vessel in the form of energy. This single vessel reaches to these openings which are \textit{iḍā}, \textit{piṅgalā} and \textit{suṣumnā}. On the left side is the \textit{iḍā}-channel, being a resemblence of the moon. On the right side exists the \textit{piṅgalā}-channel, being a resemblence of the sun. Within the middle path is a lotuspond being very subtle. [It is] made from a web of light [and it] shines like a thousand lightnings. She \extra{emerges as the central channel, assuming the form of benevolence (\textit{śiva}),} is the bestower of enjoyment and liberation. While abiding in (\textit{satyāṃ}) her (\textit{asyāṃ}) knowledge arises [to the point of which] the person becomes all-knowing.
    \end{tlate}
    %%%%%%%%%%%%%%%%%%%
    %%%%%%%%%%%%%%%%%%%
    %%%%%%%%%%%%%%%%%%%
    %%%%%%%%%%%%%%%%%%%
     \ekddiv{type=trans}
      \bigskip
    \centerline{\textrm{\small{[Description of the first Cakra]}}}
    \bigskip
    \begin{tlate}
      The means for the genesis of knowledge in the central channel will now be described. At the beginning\footnote{Supposedly at the beginning of the central channel.} exists the root \textit{cakra} having four petals. \extra{The first \textit{cakra} of support (\textit{ādhāra}) is at the anus [and] is red-colored. Gaṇeśa is the deity. He is success, intelligence and power. A rat is the mount. The Ṛṣi is Kūrma. The seal is contraction. The vitalwind is \textit{apāna}. The \textit{kalā} is the ``wave of consciousness'' (\textit{urmī}). The concentration is ``she who is powerful'' (\textit{ojasvinī}). In the four petals [of it resides] \textit{rajas}, \textit{sattva}, \textit{tamas} and the mind-faculties (\textit{manāṃsi}), [symbolized by the syllables or \textit{bīja}s] vaṃ śaṃ ṣaṃ and saṃ. A trident is situated in the middle of the triangle\footnote{This passage is odd since a triagle wasn't mentioned before.}.} In the middle is a trident, and \textit{kāmapīṭha}\footnote{Discuss the term \textit{kāmapīṭha}.} in the shape of a triangle. In the middle of this seat (\textit{pīṭha}) exists a single form in the shape of a flame. By meditating on this form the whole literature, all \textit{śāstra}s, all poems, dramas etc., everything [related to] elocution, appears in the mind of the person without learning. \extra{[Assigned to it] is external bliss\footnote{Discuss the four blisses.}, yogic bliss, heroic bliss [and] the bliss of coming to rest.}\footnote{It is noteworthy that only the first \textit{cakra} adds a detailled description of mounts, Ṛṣis, gods, seals and so forth among the current majority of witnesses at hand: E, P, L and U2. All other descriptions of the remaining eight \textit{cakra}s leave this out. The only exception is U2, a relatively late witness that adds similar descriptions for the other \textit{cakra}s as well. Since they are interesting for the history of the text I have added them to the edition's text. To indicate the extra status of those passages I have highlighted them in blue color.} An [over] hundredfold recitation of the non-recited [śataḥ = \ldots hundreds of?];  600 [repetitions for]; 9 \textit{ghaṭi}s [and] 40 \textit{palā}s.\footnote{Instructions for the duration of practice are found in all additions of U2 for each \textit{cakra}. It's not entirely clear if either the duration of meditation on the respective cakra, or the duration for the items in the list being visualised by the practitioner are meant here. However, to it seems to be done for the duration of 600 \textit{ajapājapa}, the ritualized repetition of the \textit{ajapā}, which is the voiceless uttering of the ``natural'' \textit{mantra} of the breath: \textit{so 'haṃ} - \textit{haṃ sa}. I suppose this means the practice is to be done for 600 in- and exhalations. The following part of the entry, namely ``\textit{ghaṭi} 9 \textit{palāni} 40'', probably refers to the exact time in which those 600 \textit{ajapājapa}s shall be performed. One \textit{ghaṭi} equals 1/60 of a day, which is 24 minutes. One \textit{pala} equals 1/60 of a \textit{ghaṭi} which is 24 seconds. This would equal 232 minutes or 3 hours and 52 minutes. Dividing the 600 \textit{ajapājapa}s by 232 minutes, this would result in a very slow frequence of breath of 2,586206897 in- and exhalations per minute.}
  \end{tlate}
    %%%%%%%%%%%%%%%%%%%
    %%%%%%%%%%%%%%%%%%%
    %%%%%%%%%%%%%%%%%%%
    %%%%%%%%%%%%%%%%%%%
  \ekddiv{type=trans}
      \bigskip
    \centerline{\textrm{\small{[Description of the second Cakra]}}}
    \bigskip
    \begin{tlate}
      Now the second, the six-petalled \textit{Svādhiṣṭānacakra} known as the seat of \textit{Uḍḍīyāṇa}\footnote{Discuss the term \textit{uḍḍīyāna}.}. \extra{The gender is the location. The color is yellow. The shine is yellow. \textit{Rajas} is the quality. The deity is Brahmā. The speech is \textit{vaikharī}\footnote{vaikharī f. in Kaśm. Śiv. °the 4. form of appearance of \textit{parā}, the empirical speech sound, Utpala's Ṭīkā to Śivadṛṣṭi 2, 7. [B.]― Schmidt p. 337. Welches Buch???} (\textit{vaikharī vāca}). The power is Sāvitrī. The mount is the goose. The \textit{Rṣi} is Vahaṇa. The appearance (\textit{prabhā} is the fire of love (\textit{kāmāgni}). The body is gross, The state is that of being awake. The Veda is Ṛg. The spiritual guide is the characteristic (\textit{liṅga}). The liberation is residing in the world of Brahma. The principle is pure level (\textit{śuddhabhūmikā}). The sphere is smell. The vitalwind is \textit{apāna}. The internal matrix [is]: vaṃ bhaṃ maṃ yaṃ raṃ laṃ. The external matrix: Kāmā ``she who is desire'', Kāmākhyā ``she who is the \textit{tīrtha} of \textit{Kāmākhyā}''\footnote{The Kāmākhyā is situated in Kāmarūpa on the Nīlakūṭa mountain in present day Assam. It's strange that it appears here, since Kāmarūpa appears already as the \textit{tīrtha} associated with the first \textit{cakra}.}, Tejasvinī ``she who is shining'', Ceṣṭikā ``she who is active'', Alasā ``she who is lazy'' [and] Mithunā ``she who is \textit{mithunā}''. A [more than] thousandfold recitation of the non-recited; 6000 [repetitions for]; 16 \textit{ghaṭi}s [and] 40 \textit{palā}s.\footnote{The practice is supposed to be done for the duration of 6000 \textit{ajapājapa}s divided into \textit{ghaṭi}s and 40 \textit{pala}s, resulting in 2320 minutes or 38,67 hours. Again this would result in a frequence of breath of 2,586206897 in- and exhalations per minute.}} In its middle exists extremely red glow. The adept becomes very handsome through meditation on it. \extra{He becomes one who is desired by young women.} The vital force increases from day to day.
    \end{tlate}
   %%%%%%%%%%%%%%%%%%%
    %%%%%%%%%%%%%%%%%%%
    %%%%%%%%%%%%%%%%%%%
    %%%%%%%%%%%%%%%%%%%
  \ekddiv{type=trans}
    \bigskip
    \centerline{\textrm{\small{[Description of the third Cakra]}}}
    \bigskip
    \begin{tlate}
      The third, a lotus with ten petals exists at the location of the navel. \extra{The colour is red (\textit{kapila}). Viṣṇu is the deity. Lakṣmī is the power. Vāyu is the Rṣi. Samāna is the vitalwind. The mount is Garuḍa. The deity is the suble body\footnote{Why another deity is given here?}. The state is sleep. The speech is the inaudible speech (\textit{madhyamāvāg})\footnote{<Śā, Ling>name of the speech which is inaudible and which is of the type of a thought without any definite presence of words making up the expression. Vkp I.143.<Abhyankar 1986: 300>}. The Veda is the Yajurveda. The [fire is the] southern fire. The liberation is ``proximity'' (\textit{samīpatā}).\footnote{What is this exactly?}. Viṣṇu is the characteristic of the teacher (\textit{guruliṅga}). The principle is water. The sphere is athmosphere (\textit{rajo viṣaya}). There are ten petals [and] ten matrices. [The] inner matrix: \textit{ḍaṃ ṭaṃ ṇaṃ taṃ thaṃ daṃ dhaṃ naṃ paṃ phaṃ}. The external matrix: Śānti ``she who peaceful'', Kṣamā ``she who is patient'', Medhā ``she who is insightful'', Tanayā ``the daughter'', Medhavinī ``she who is a learned teacher'', Puṣkarā ``she who is a lotus'', Haṃsagamanā ``she who moves like a swan'', Lakṣyā ``she who is the object aimed at'', Tanmayā ``she who is absorption'' and Amṛtā ``she who is immortality''. A [more than] thousandfold recitation of the non-recited; 6000 [repetitions for]; 16 \textit{ghaṭi}s [and] 40 \textit{palā}s.\footnote{Here we find the same instruction as in the previous description of the second \textit{cakra}. The practice is supposed to be done for the duration of 6000 \textit{ajapājapa}s divided into \textit{ghaṭi}s and 40 \textit{pala}s, resulting in 2320 minutes or 38,67 hours. Again this would result in a frequence of breath of 2,586206897 in- and exhalations per minute.}} In its middle exists a \textit{cakra} with five angles. In its middle is a single [divine] form. It's not possible to describe her shine with speech. Through the execution of meditation on this [divine] form the body of the person is going to be strong.
 \end{tlate}
   %%%%%%%%%%%%%%%%%%%
    %%%%%%%%%%%%%%%%%%%
    %%%%%%%%%%%%%%%%%%%
    %%%%%%%%%%%%%%%%%%%
         \ekddiv{type=trans}
       \bigskip
    \centerline{\textrm{\small{[Description of the fourth Cakra]}}}%%%%%See Jogpradipikaya Edition Page 163 
       \bigskip
         \begin{tlate}
           The fourth lotus having twelve-petals exists in the middle of the heart. \extra{[The] place of the Anāhatacakra is within the heart\footnote{This is redundant.}. The color is white. The quality is Tamas. The deity is Rudra. The power is Umā. The Ṛṣi is Hiraṇyagarbha. The mount is Nandi. The vitalwind is Prāṇa. The body is the cause of digits of light. The state is deep sleep. The speech is Paśyantī\footnote{Add footnote of entry in \textit{Tāntrikābhidhānakośa}.}. [The Veda is] Sāmaveda. The fire is the fire of the householder\footnote{Add explanation.}. The characteristic is Śiva. The level is the ability to attain everything on earth\footnote{Quote \textit{Tantrikābhidhānakośa}.}. The liberation is uniform [with the deity]. [There are] twelve petals, [and] twelve matrices: kaṃ khaṃ gaṃ ghaṃ ṇaṃ caṃ chaṃ jaṃ jhaṃ yaṃ taṃ [and] thaṃ. The external matrix: Rudrāṇī ``she who is Rudra's wife'', Tejasā ``she who is brilliant''\footnote{To be understood as \textit{tejasvinī}.}, Tāpinī ``she who is glow'', Sukhadā ``she who bestows happiness'', Caitanyā ``she who is consciousness'', Śivadā ``she who bestows grace'', Śānti ``she who is peaceful'', Umā ``she who is glorious'', Gaurī ``she who is beautiful'', Mātarā ``she who is bestowing the mother'', Jvalā ``she who is the flame'' [and] Prajvālinī ``she who is blazing''. A [more than] thousandfold recitation of the non-recited; 6000 [repetitions for]; 16 \textit{ghaṭi}s [and] 40 \textit{palā}s.\footnote{The \textit{ajapājapa} for this \textit{cakra} is to be performed 6000 times for a duration of 96 \textit{ghaṭi}s and 40 \textit{pala}s, resulting in 2320 minutes or 38.67 hours. Again this would result in a frequence of breath of 2,586206897 in- and exhalations per minute.}} Due to being made of [such an] intense light [the fourth lotus] is not in the range of sight. In its middle exists a lotus facing downward having eight petals. \extra{The mind resides in the \textit{cakra}. The mind is the deity. The power is external\footnote{n Muktabodha-Texte sehe ich 3 Belege für bahiśśakti Muktabodha/krīyakramādyotikā.html 2938 suṣirānte bahiśśaktiṃ vinyasedvyomarūpiṇīm | tasyā madhye tu Muktabodha/sakalāgamasārasaṅgraha.html 2186 suṣirāntabahiśśaktiṃ vyāpinīṃ cintayet tataḥ || Muktabodha/kriyakramadyotikavyākhyā.html 1846 tanmadhye ca bahiśśaktiṃ sudhābindu parisrutim}, [its] Ṛṣi is the self. In the middle of the navel exists a lotus. Its stalk measures ten \textit{aṅgula}s. The stalk of it is soft (\textit{komala}), pure [and] facing downwards. In its middle is [something] shining like a banana-flower. The mind has no determination of will, giving a firmer direction to man's thoughts for the moment by means of [conscious] submission. [It is] truly changeable in nature.} \extra{While the mind rests on the eastern petal [which is] white in colour clear intellekt arises, which is [endowed with] \textit{dharma}, fame and knowledge etc. While [the mind rests on] the south-east, [which is] reddish in color a mind that is weak due to sleep, laziness and illusion arises. While [the mind is situated] in the right south, [which is] black in color the generation of anger arises. While [the mind is situated] in the southwest, [which is] blue in color a mind of pride arises. While [the mind is situated] in the west, [which is] brown in color a mind that is longing for play, laughing, and celebration arises. While [the mind is situated] in the northwest, [which is] dark in color a mind which is restless by sorrow arises. While [the mind is situated] in the north, [which is] yellow in color a very happy mind with erotic and enjoyment arises. While [the mind is situated] in north-east [which is] whitish in color a mind of unity through knowledge arises.}
           
It's said that in its middle is the place of the \textit{prāṇa}-vitalwind [and] in the middle [of] the eight-petalled lotus is a pericarp (\textit{karṇikā}) in the form of a \textit{liṅga}. The technical designation of her is \textit{kalikā}. In the middle of this \textit{kalikā} exists a single thumbsized [divine] figurine (\textit{puttalikā}) being similiar to a ruby-gem in color. Her technical designation is embodied soul (\textit{jīva}). Not even with a thousand tongues it is possible to talk about her nature and her power. Here it is said [that]: ``Because of the exercise of meditation on this form the inhabitants of the universe [which are] Humans, Gandharvas, Kinnaras, Guhyakas, Vidyādharas and [their] females, in the heavenly world, underworld and open space are obedient to the will of the practicing person.''.
  \end{tlate}
    %%%%%%%%%%%%%%%%%%%
    %%%%%%%%%%%%%%%%%%%
    %%%%%%%%%%%%%%%%%%%
    %%%%%%%%%%%%%%%%%%%
  \ekddiv{type=trans}
      \bigskip
    \centerline{\textrm{\small{[Description of the fifth Cakra]}}}
    \bigskip
    \begin{tlate}
Now the fifth lotus having sixteen petals existing at the location of the throat. \extra{The colour is grey. The deity is the embodied soul (\textit{jīva}). The power is ignorance (\textit{avidyā}). The Ṛṣi is Virāṭ\footnote{Who is this?}. The mount is the wind (\textit{vāyu}). The vitalwind is \textit{udāna}. The digit (\textit{kalā}) is the flame. The binding (\textit{bandha}) is Jālandhara. The body is the primordial cause (\textit{mahākāraṇa}). The state is the fourth state (\textit{tūrya}). The speech is Parā\footnote{Im Kaśm. Śiv. °das ewige Wort, in welchem potentiell alle Begriffe und Worte ruhen; vgl. das śabdabrahma des Vyākaraṇa. [B.]― Schmidt S. 246}. [The Veda is the] Atharvaṇa Veda. The movable characteristic (\textit{jaṅgamaṃ liṅgaṃ}). The earth is Jīvaprāptā\footnote{What is this?}. The liberation is union with the deity (\textit{sāyujyatā}). [There are] sixteen petals [and] sixteen matrices. The internal matrix: aṃ āṃ iṃ īṃ u ūṃ ṛṃ ṝṃ ḷṃ ḹṃ eṃ aiṃ oṃ auṃ aṃ aṃḥ. The external matrix: Vidyā ``she who is knowledge'', Avidyā ``she who is ignorance'', Icchā ``she who is desire'', Śakti ``she who is power'', Jñānaśakti ``she who is the power of knowledge'', Śatalā ``she who is manifold'', Mahāvidyā ``she who is great knowledge'', Mahāmayā ``she who is great illusion'', Buddhi ``she who is intellect'', Tāmasī ``she who is darkness'', Maitrā ``she who is love'', Kumārī ``she who is a young girl'', Maitrāyaṇī ``she who is onb the path of benevolence'', Rudrā ``she who is howling'', Puṣṭā ``she who is abundance'', Siṃhanī ``she who is a lioness''. A thousandfold recitation of the non-recited; 1000 [repetitions for]; 2 \textit{ghaṭi}s, 46 \textit{palā}s. and 40 \textit{akṣara}s.\footnote{It is not entirely clear what kind of measure an \textit{akṣara} is. Maybe see Amanaska 1. Chapter second half in thesis of Jason to clear things up.}} In its middle exists a single person which shines like a thousand moons. Because of the exercise of meditation on this person all diseases which are (otherwise) not possible to be controlled vanish. The person lives up to 1001 years.
    \end{tlate}
   %%%%%%%%%%%%%%%%%%%
    %%%%%%%%%%%%%%%%%%%
    %%%%%%%%%%%%%%%%%%%
    %%%%%%%%%%%%%%%%%%%
\ekddiv{type=trans}
    \bigskip
    \centerline{\textrm{\small{[Description of the sixth Cakra]}}}
    \bigskip
    \begin{tlate}
      Now it exists a sixth \textit{cakra} named Ājñā. \extra{The deity is fire (\textit{agni}). The power is the godess of the centre (\textit{suṣumṇā}). The Ṛṣi is ``the violent'' (\textit{hiṃsa}). The mount is consciousness (\textit{caitanya}). The body is knowledge. The state is understanding. The speech is the ``incomparable'' (\textit{anupama}). The [Veda] is Sāmaveda. The \textit{liṅgaṃ} is intoxication (\textit{pramāda}). The half-matrix: the principle of ether. The gander is the living soul. The origin is the play of conciousness. Twofold matrix: haṃ kṣam is the inner matrix. The external matrix: Sthiti ``she who maintains'' [and] Prabhā ``she who is splendour''. A thousandfold recitation of the non-recited; 1000 [repetitions for]; 2 \textit{ghaṭi}s, 46 \textit{palā}s, and 40 \textit{akṣara}s.\footnote{It's not entirely clear what kind of measure is an \textit{akṣara}.}} This \textit{cakra} is located in the middle of the eyebrows and is two-petalled. In its middle exists a certain object being a form of blazing fire without parts, not being female not being male. Because of the exercise of meditation on it the body of the person becomes non-aging and immortal.
    \end{tlate}
   %%%%%%%%%%%%%%%%%%%
    %%%%%%%%%%%%%%%%%%%
    %%%%%%%%%%%%%%%%%%%
    %%%%%%%%%%%%%%%%%%%
 \ekddiv{type=trans}
    \bigskip
    \centerline{\textrm{\small{[Description of the seventh Cakra]}}}
    \bigskip
    \begin{tlate}
Now the seventh cakra having 64 petals and being full of nectar exists in the middle of the palate. \extra{The forehead is the Maṇḍala. The moon is the deity. The power is the nectar of immortality. The Rṣi is the supreme self. The seventeenth digit is the resident with the nectar of immortality. The wavy stream of nectar is great space. The uvula is the mother. The ornament/rhythm? (\textit{tālikā}) is a small bell. The own form of the body is the unspeakable Gāyatrī, [which has] the face of a crow, the eye of a human, the horn of a cow, a forehead that is Brahmapaṭhā?, a neck like a horse, the face of a peacock [and] limbs like a goose. [This is] the specific nature of the unspeakable Gayatrī.}    
  It is endowed with superabundant beauty. [It is] very bright. In its middle, red in color [is that which is] known as "uvula" (\textit{ghāṃṭikā}). [It] exists as a single pericarp. In its middle is a [certain] site. In the middle of it exists a hidden digit of the moon, being a stream of nectar like a river (\textit{amṛtādhārāsravantī}). Because of the exercise of meditation on this digit death does not come near him. Due to uninterrupted meditation, the stream (\textit{dhārā}) of nectar flows. Then the appearances of emaciation (\textit{kṣayaroga}), fever due to disordered bile (\textit{pittajvara}), heartburn (\textit{hṛdayadāha}), head-disease (\textit{śiroroga}) and tongue insensibility (\textit{jihvājaḍa}) vanish. Also eaten venom doesn't trouble him. If the mind is here, [it] becomes stable.     
  \end{tlate}
  %%%%%%%%%%%%%%%%%%%
    %%%%%%%%%%%%%%%%%%%
    %%%%%%%%%%%%%%%%%%%
    %%%%%%%%%%%%%%%%%%%
 \ekddiv{type=trans}
    \bigskip
    \centerline{\textrm{\small{[Description of the eighth Cakra]}}}
    \bigskip
          \begin{tlate}
Now exists the eigth \textit{cakra} having one hundred petals located at the aperture of Brahman. \extra{The teacher is the deity. Consciousness is the power. Virāṭ is the Ṛṣi, the witness above everything. Made of consciousness is that which is associated with (\textit{bhūta°}) the state beyond the fourth state. It has all colours. It has all matrices. It has all petals. The body is Virāṭ. The state is the standing still. The speech is wisdom.  The "I am that"-[expression] (\textit{sohaṃ}) is the Veda. The place is unsurpassed. A thousandfold recitation of the non-recited; 1000 [repetitions for]; 2 \textit{ghaṭi}s, 46 \textit{palā}s. and 40 \textit{akṣara}s.\footnote{It's not entirely clear what kind of measure is an \textit{akṣara}.} The count is all silent mutterings, [being] 21600. In this way it carries on day and night. He who knows the breath is a learned person. With the sound "sa" he exhales, with the sound "ha" he inhales again: "I'm he, he's I". Because of that the embodied soul constantly utters the Mantra.\footnote{Add intertextual evidence.}} ``The (divine) seat of  Jālaṃdhara'' is the designation of its lotus.\footnote{Find parallels where Jālandhara is situated on top of the head.} [It is] the place of the accomplished person. In its middle looking like a streak [and] having the form of smoke and fire, exists such a single [divine] form of the person (\textit{puruṣa}). Of her exists no end, nor a beginning. Due to the exercise of meditation on this [divine] form both coming and going of the person in space occurs. Affliction from the earth-element doesn't arise [anymore] even if one is situated in the middle of the earth. He constantly sees everything in front of his eyes and he becomes separated [from the material world]. The force of life increases eminently.    
     \end{tlate}
    %%%%%%%%%%%%%%%%%%%
    %%%%%%%%%%%%%%%%%%%
    %%%%%%%%%%%%%%%%%%%
    %%%%%%%%%%%%%%%%%%%
\ekddiv{type=trans}
       \bigskip
    \centerline{\textrm{\small{[Description of the ninth Cakra]}}}
    \bigskip
    \begin{tlate}
Now the divisions/differentiations of the ninth cakra are explained. The designation of it is ``the \textit{cakra} of the great void''. Above that there is no other. Therefore it is declared to be the \textit{cakra} of the great perfection. [Another] such name of it is ``(divine) seat of Pūrṇagiri''. In the middle of the \textit{mahāśūnyacakra} exists one lotus facing upward, very red in colour, with a thousand petals - an abode of brilliance and wholeness, whose fragrance is not in range of mind and speech. In the middle of this lotus exists one pericarp having the shape of a triangle. In the middle of the pericarp exists one seventeenth digit in the shape of a immaculé form. A light of the part exists shining like a thousand suns. [But] excessive heat is not arising. Shining like a thousand moons, excess of cold is not arising. \extra{Here at this location the ``I''(\textit{aham}) is the deity. The ``he is I'' (\textit{so 'ham}) is the power. This self is the Ṛṣi. The path is liberation. Brahma is the I above. ``I'm a circle''. In fire-area is the letter "sa". [There?] life arises, the living soul ascends and decends. The place is the hidden place of being. The colour is yellow. The light is the shine of ten million suns. The shine is always and visible. Śiva is the deity. The power is primordial illusion. The state is the dissolution of the self into Hara\footnote{Epiphet of Śiva.}. The transcendental sound has the nature of a sound with stable resonance. The seal is the ``fearless''. The illusion is the root. The body is the original matter. The range is speech and mind. Without delusion, without doubt, the unaffected and undefiled goal is dissolution, meditation [and] final absorption.} Above that is the place of infinite supreme bliss. There above is power (\textit{śakti}). Being designated as such she is one single digit. Due to the exercise of meditation on this part, the person manifests whatever he wishes for. He is furnished with royal pleasure and enjoyment. [Even] amusing oneself amongst women, and watching musical pleasures, the \textit{kāla} of the person grows daily like the \textit{kalā} of the moon in the bright half of the month. His body is not affected by merit and sin. Due to uninterrupted meditation the power of the light of the innate nature arises. He sees remotely located objects as if they'd be near.
\end{tlate}
    %%%%%%%%%%%%%%%%%%%
    %%%%%%%%%%%%%%%%%%%
    %%%%%%%%%%%%%%%%%%%
    %%%%%%%%%%%%%%%%%%%
  \ekddiv{type=trans}
     \bigskip
    \centerline{\textrm{\small{[Lakṣyayoga, the yoga of fixation]}}}
    \bigskip
 \begin{tlate}
   Now the yoga of fixation (\textit{lakṣyayoga}), which is easily accomplished is explained. Of this yoga of fixation there are five subdivisions:
   1. The upward directed fixation (\textit{ūrdhvalakṣya}),
   2. the downward directed fixation (\textit{adholakṣya}),
   3. the outer fixation (\textit{baḥyalakṣya}),
   4. the central fixation (\textit{madhyalakṣya}),
   5. the inner fixation (\textit{antaralakṣya}).
 \end{tlate}
   %%%%%%%%%%%%%%%%%%%
    %%%%%%%%%%%%%%%%%%%
    %%%%%%%%%%%%%%%%%%%
    %%%%%%%%%%%%%%%%%%%
  \ekddiv{type=trans}
     \bigskip
    \centerline{\textrm{\small{[1. Ūrdhvalakṣya - The upward directed fixation]}}}
    \bigskip    
  \begin{tlate}
At first the upward directed fixation (\textit{ūrdhvalakṣya}) is explained. The gaze (\textit{dṛṣṭi}) [should be] in the middle of the sky. And then having caused the mind to be directed upwards, it is caused to be fixed there. Due to the exercise of stabilizing of this fixation (\textit{lakṣya}) arises unity of the gazing point (\textit{dṛṣṭi}) with the light of the highest lord (\textit{parameśvara}). And then an indefinable invisible object arises in the middle of the sky. It arises in the range of sight of the practitioner. This is truly the upward directed fixation (\textit{ūrdhvalakṣya}).\footnote{Compare \textit{Vijñānabhairava} 84.} 
  \end{tlate}
   %%%%%%%%%%%%%%%%%%%
    %%%%%%%%%%%%%%%%%%%
    %%%%%%%%%%%%%%%%%%%
    %%%%%%%%%%%%%%%%%%%
\ekddiv{type=trans}
   \bigskip
    \centerline{\textrm{\small{[2. Adholakṣya - The downward directed fixation]}}}
    \bigskip
  \begin{tlate}
    Now the downward directed fixation object (\textit{adholakṣya}). One should stabilize the gaze within the circumference (\textit{paryanta}) of twelve \textit{aṅgula}s beyond the nose. Or one should stabilize the gaze onto the tip of the nose. The fixation becomes stable due to firm exercise [on one] of the twofold aims [of fixation]. The breath becomes stable. Vitality increases.
\end{tlate}
 %%%%%%%%%%%%%%%%%%%
    %%%%%%%%%%%%%%%%%%%
    %%%%%%%%%%%%%%%%%%%
    %%%%%%%%%%%%%%%%%%%
\ekddiv{type=trans}
   \bigskip
    \centerline{\textrm{\small{[3. Bāhyalakṣya - The external fixation]}}}
    \bigskip
  \begin{tlate}
    Just as this [aim] is twofold, also the external fixation is said to be [like this]. Internally or externally the aim of fixation is to be done onto the heavenly emptiness. The fear of dying doesn't arise due to the exercise of meditation on the void at all places during ones life - while eating, moving and waking.\footnote{Note that the description of the five types of Lakṣyayoga stops here and the new topic about the body of the Rājayogin is introduced. However, the subject is resumed later on in the text. Even though all witnesses follow this specific and suprising order. Maybe a copist in the early stages of transmission of the text copied the text without noticing the folios of his template to be in the wrong order.}
  \end{tlate}
   %%%%%%%%%%%%%%%%%%%
    %%%%%%%%%%%%%%%%%%%
    %%%%%%%%%%%%%%%%%%%
    %%%%%%%%%%%%%%%%%%%
  \ekddiv{type=trans}
    \bigskip
    \centerline{\textrm{\small{[Description of the Rājayogin's Body]}}}
    \bigskip
      \begin{tlate}
Now it is said that this is the characteristic of the embodied person who is endowed with the royal yoga: Abundance arises at all times. No distance exists on earth. He dwells on earth having pervaded [it]. Birth and death both don't exist. Happiness does'nt exist. Suffering does'nt exist. Impediment does'nt exist. Habit doesn't exist. Place does'nt exist. The manifestation of permanent perception of the connection with god arises in the middle of the mind of this accomplished one. And he is shining - not cold, and not hot, not white [and] not yellow. Neither is there birth of him, nor does he have any attributes. And he is without parts, immacule and uncharacterized. His desire etc. doesn't arise in [situations of] lust [and] is not located within the duality of the result. He attains expanded enjoyment. However, his mind does not suffer attachment in this very state.     
    \end{tlate}
    %%%%%%%%%%%%%%%%%%%%%
    %%%%%%%%%%%%%%%%%%%
    %%%%%%%%%%%%%%%%%%%
    %%%%%%%%%%%%%%%%%%%
    %%%%%%%%%%%%%%%%%%%
 \ekddiv{type=trans}
    \bigskip
    \centerline{\textrm{\small{[Other Attributes]}}}
    \bigskip
  \begin{tlate}
    Another attribute of Rājayoga is described. Even ``of one who is in gain of a kingdom etc.'' [it is said that] perception of success does'nt arise. Even due to loss suffering does'nt arise in the mind. And then desire doesn't arise. And then with regards to an object that has been obtained for whatever reason towards ones object aversion does'nt arise. With regard to this object affection of the mind does'nt arise. Just this is said to be Rājayoga. And then his mind which knows the sacred speech is equal towards a person, friend and enemy. And a neutral view arises. In the mind of one who is entirely situated in the middle of the earth, the pride of authorship does't arise, because of death and rebirth, and because of happiness and enjoyment. Wile wandering the world he doesn't whish to know authorship. This is also said to be Rājayoga. New durable clothes made of silk, or however, old, worn [clothes] with holes smeared with sandalwood and musk, or smeared with mud. In whose mind joy and sorrow are not situated, just he is [in the state of] Rājayoga. Just he is in the state of Rājayoga for whom the mind is neither in abundance nor in lack, being located in a city, a forest, an uninhabited village or a village full of people.    
  \end{tlate}
    %%%%%%%%%%%%%%%%%%%%%
    %%%%%%%%%%%%%%%%%%%
    %%%%%%%%%%%%%%%%%%%
    %%%%%%%%%%%%%%%%%%%
    %%%%%%%%%%%%%%%%%%%
\ekddiv{type=trans}
      \bigskip
    \centerline{\textrm{\small{[Description of Caryāyoga]}}}
      \bigskip
     \begin{tlate}
        Now \textit{caryāyogaḥ}, the Yoga of wandering is explained. Shapeless, unchangeable, permanent [and] unsplitable. Such is the self. It is seen as such by the one whose mind abides in the self without moving. His self is not touched by sin and merit. Just as the leave of the lotus situated in the amidst water doesn't touch the water; likewise the self [is not touched by sin and merit]. Just as the wind wanders according to its own will in space, likewise the mind of one who is absorbed into the universal spirit [wanders according to its own will in space]. This is \textit{caryāyoga}.
      \end{tlate}
    %%%%%%%%%%%%%%%%%%%%%
    %%%%%%%%%%%%%%%%%%%
    %%%%%%%%%%%%%%%%%%%
    %%%%%%%%%%%%%%%%%%%
    %%%%%%%%%%%%%%%%%%%
  \ekddiv{type=trans}
       \bigskip
    \centerline{\textrm{\small{[Description of Haṭhayoga]}}}
      \bigskip
      \begin{tlate}
        Now \textit{haṭhayoga}, the forceful Yoga is explained. The practice of breath shall be done in this manner: "Exhalation, Inhalation [and] Retention etc. And then due to the six practices (\textit{ṣaṭkarma}), like \textit{dhauti} etc. the purification of the body arises. When the full breath abides in the middle of the sun-channel. Then the mind is unmovable. The form of bliss immediately shines through the motionless mind. Due to the execution of Haṭhayoga the mind becomes absorbed into emptiness. The time of death does not approach. Now, the second division of Haṭhayoga is explained. The shine of ten million suns in one's own body beginning from the feet to the top of head is contemplated in any color equal to white, yellow [or] red. Due to the execution of meditation in the entire body disease does'nt arise, fever doesn't arise and vitality grows.
      \end{tlate}
  %%%%%%%%%%%%%%%%%%%%%
    %%%%%%%%%%%%%%%%%%%
    %%%%%%%%%%%%%%%%%%%
    %%%%%%%%%%%%%%%%%%%
    %%%%%%%%%%%%%%%%%%%
  \ekddiv{type=trans}
    \bigskip
        \centerline{\textrm{\small{[Description of \textit{Jñānayoga}]}}}
          \bigskip
    \begin{tlate}
      Now the characteristic of \textit{jñānayoga} is explained.      
  \paragraph{1.} He shall see the world truly as being one, shining in all selves. By applying indistinctness he shall accomplish \textit{Jñānayoga}.
  \paragraph{2.} Wherever the world is established or made of omniscience, who knows thus by means of insight, he is a like an expert of knowledge.
  \paragraph{3.} He always attains the reality of \textit{śāmbhavī} - the goal of eternal non-duality. Just as the seed of the Nyagrodha\footnote{In rituals, the nyagrodha (Ficus indica or India fig or banyan tree) danda, or staff, is assigned to the kshatriya class, along with a mantra, intended to impart physical vitality or 'ojas'.27. Brian K. Smith. Reflections on Resemblance, Ritual, and Religion, Motilal Banarsidass Publishe, 1998} scattered onto the soil [always] becomes a tree.
  \paragraph{4.} The absolute unity (\textit{ekāntaṃ}), is seen as multibel (namely) made up of ten parts by oneself. The rolled up shoots of the branches are the sprouting stalks of the root shoot.
  \paragraph{5.} By virtue of its inherent nature, this branch with its branches, which is the fruit of the flower of love, is in the seed. Certainly, that is pure, eternal, unchanging and immaculate. 
  \paragraph{6.} One, not one and self-existing, existing in manifold ways through its own rule and work, [as] five principles (\textit{tattva}) which are: thinking mind (\textit{manas}), intellect (\textit{buddhi}), illusion (\textit{māya}), individuation (\textit{ahaṃkāra}) and modifications (\textit{vikriyā}).
  \paragraph{7.}In this way, the ten variations fully permeate the world and the non-world. Only one thing is and not something else: Whoever knows this is a connoisseur of reality.
  
   Transmigration is the appearance of the plant world, mountains, trees, earth etc. Transmigration is the appearance of living beings beginning with birds, horses, elephants and humans. And then whoever is one who is a [sense] object of sight is said to be visible. He who is not seen by sight is said to be invisible. In this way the view of separation of one's own self which is subjected to transmigration is to be removed by means of [applying the view of] unity. Only this is Jñānayoga. Because of the execution of it, time does'nt destroy the body.
\end{tlate}
  %%%%%%%%%%%%%%%%%%%%%
    %%%%%%%%%%%%%%%%%%%
    %%%%%%%%%%%%%%%%%%%
    %%%%%%%%%%%%%%%%%%%
    %%%%%%%%%%%%%%%%%%%
\ekddiv{type=trans}
      \bigskip
        \centerline{\textrm{\small{[The Division of the Inherent Nature]}}}
          \bigskip
\begin{tlate}    
  Now the division of the inherent nature is described.\footnote{This refers to the mention of \textit{svabhāva} in verse 5 of the description of Jñānayoga.} Just as the seed of the banyan tree ripens into the shape of the banyan tree, and by its own inherent nature attains such a tenfold division. [Namely]: "Root, shoot, bark, branch, twig, bud, the unfolding flower, flower, fruit and nectar." The division reaches [those] ten parts. In this way, the pure, unchanging, unblemished, attains such [division] precisely because of the inherent nature of the self. [Namely] the division "Earth, Water, Fire, Wind, Space, Mind, Intellect, Illusion, Transformations and Form". Because of the power of Jñānayoga, there arises the certainty that "The Self is verily one." As some particular soil (\textit{ekaika}) sometimes appears soft, sometimes beautiful, sometimes fragrant, sometimes unscented, sometimes golden, sometimes silver, is sometimes made of precious stone, sometimes appearing white, sometimes black, sometimes copper, sometimes yellow, sometimes mottled, sometimes like various fruit, sometimes like flowers, sometimes like the nectar of immortality, [and that only] because of its inherent nature. In this way, the self also takes the form of a human, a bird, a gazelle, an elephant, a vidyādhara, a gandharva, a centaur, great scholar or a great fool, a sick or healthy, an angry or or peaceful person, by virtue of its inherent nature. Because of Jñānayoga, transformation is recognized as formless, Just as the place of origin of the fruit is only one. But the transformation of the fruit is seen as manifold.
  
  One fruit falls onto the ground. It is getting bright. A bee drinks the flower juice of a fruit. The lover [bee] places itself on the flower wreath above the protuberant circular pistil. A bee drinks the juice of a fruit. The lover (bee) places herself on the flower wreath above the upstanding circular pistil. ne fruit throws the nectar over the flower. This is the inherent nature of the matter. In the same way also the one self enjoys the eight pleasures because of its own being.  
\\ \\
What are the eight enjoyments?  \hfill \break
A beautiful dwelling, good clothing, a good bed, a well-trained horse?, a nice place, food and drink.\footnote{The verse only gives 7 enjoyments!} Those are the eight enjoyments of the wise.
\\ \\
  1. Clothes made from silk; \hfill \break
  2. A site of the palace in which there are mainsions endowned with five or seven rooms.\hfill \break
  3. A huge, very soft and lovely bed; \hfill \break
  4. [on which] there is seated a lotus-like youthful, charming and virtuous wife;\hfill \break 
  5. An excellent throne;\hfill \break
  6. An exceptional valuable horse; \hfill \break
  7. Food that pleases the senses; \hfill \break
  8. Various drinks. \hfill \break 

Like the rays of the sun, the butter of milk, the burning of fire, the stupor of poison, the sesame oil from the sesame seed, the shade from the tree, the sweet odor from a fruit, the fire from a scabbard, the sweet sap of Śārkara\footnote{A liquor prepared from Dhātakī with sugar.} and so on, the cold of piles of snow, and so on is the inherent essence of things. In the same way, the course of the world is also in the center of the highest God's own form. And the Most High God is indivisible and all-filling.
\end{tlate}
  %%%%%%%%%%%%%%%%%%%%%
    %%%%%%%%%%%%%%%%%%%
    %%%%%%%%%%%%%%%%%%%
    %%%%%%%%%%%%%%%%%%%
    %%%%%%%%%%%%%%%%%%%
\begin{tlate}
  \ekddiv{type=trans}
   \bigskip
        \centerline{\textrm{\small{[Continuation of \textit{Lakṣyayoga} - Bāhyalakṣya]}}}
          \bigskip
Now the external fixation is taught. Beginning with a four finger wide distance from the tip of the nose, the space[-element] full of light whose appearance is blue shall be made the object of fixation. Or, a six finger wide distance from the tip of the nose, the wind-element whose appearance is greyish shall be made the object of fixation. Or, an eight finger wide distance from the tip of the nose, the very red fire[-element] shall be made the object of fixation. Or, a ten finger wide distance from the tip of the nose, the white water[-element] being fickle shall be made the object of fixation. Or, a twelve finger wide distance from the tip of the nose, the yellow-colored earth-element shall be made the object of fixation. Or beginning at the tip of the nose\footnote{Given the clear instructions of the respective distance of the exercise in the previous sentences, it is surprising that this instruction is lacking the mention of the distance.} the space-element full of fire shining like ten million suns shall be made the object of fixation. After having fixed the gaze on the space[-element?] or above the space[-element?], due to the execution of meditation he sees the sun without the group of thousand rays related to the sun. Or the mass of light situated seventeen fingers wide distance above the head shall be made the fixation object. Diseases of the limbs are removed without medical herbs. All enemies become friends while sleeping. The lifespan increases up to 1000 years. 
\end{tlate}
  %%%%%%%%%%%%%%%%%%%%%
    %%%%%%%%%%%%%%%%%%%
    %%%%%%%%%%%%%%%%%%%
    %%%%%%%%%%%%%%%%%%%
    %%%%%%%%%%%%%%%%%%%
\begin{tlate}
  \ekddiv{type=trans}
     \bigskip
        \centerline{\textrm{\small{[Continuation of \textit{Lakṣyayoga} - Antaralakṣya]}}}
          \bigskip
Now the inner fixation objects are taught. At the location of the root bulp rising from the staff of Brahma up to the aperture of Brahma exists the one white coloured Brahma channel. The interior of the Brahma channel, which equals a pale-red string shining like 10 million suns, goes upwards. A particular manifestation exists as such. Due to the execution of meditation on this manifestation, the eight great supernatural powers of humans beginning with \textit{aṇima} etc.\footnote{Write something about siddhis.} become established after one has entered into [the manufestation's] imminence. Or from the execution of meditation onto the bright light at the centre within the space at the forehead diseases related to the body beginning with leprosy vanish. Lifeforce increases. Or because of executing meditation on the middle of the eyebrows [of which there is] a very subtle and red colored light, he is one who is beloved among all royal people. Having seen this person, everybody's gaze is fixed onto him.
\end{tlate}
  %%%%%%%%%%%%%%%%%%%%%
    %%%%%%%%%%%%%%%%%%%
    %%%%%%%%%%%%%%%%%%%
    %%%%%%%%%%%%%%%%%%%
    %%%%%%%%%%%%%%%%%%%
\begin{tlate}
  \ekddiv{type=trans}
 \bigskip
 \centerline{\textrm{\small{[The Ten Main Bodily Channels]}}}
 \bigskip
 Now the divisions of channels within the body are explained. There are ten primary channels. Among them exists the pair of channels designated Idā and Piṅgalā at the entrance of the nose. The central channel leads from the palate to the door of Brahma. The Sarasvatī[-channel] exists at the centre of the face. The two rivers Gāṃdhārī and Hastjihvā exist within the centre of the two ears. The two rivers Pūṣā and Ālaṃbuṣā are situated at the center of the two eyes. The Śaṃkhinī channel strechtes from the the beginning of the opening of the penis through the Iḍā-channel. In such a way the channels are situated at the 10 openings. The other channels measured as 72000 are situated with a subtle form at the roots of the hairs.
\end{tlate}
  %%%%%%%%%%%%%%%%%%%%%
    %%%%%%%%%%%%%%%%%%%
    %%%%%%%%%%%%%%%%%%%
    %%%%%%%%%%%%%%%%%%%
    %%%%%%%%%%%%%%%%%%%
 \begin{tlate}
  \ekddiv{type=trans}
\bigskip
 \centerline{\textrm{\small{[The Ten Vitalwinds]}}}
 \bigskip
 Now [there are] ten vitalwinds are situated within the body. The Prāṇa vitalwind is located in the middle of the heart and causes inhalation and exhalation. The wish for eating an drinking exists. At the center of the anus the Apāna-Vitalwind exists. He does contraction and checking. At the center of the navel the Samāna[-vitalwind] exists. He causes to dry up all the channels. In this way the channels are caused to thrive, beauty is caused to be generated and the fire is caused to light up. Within the throat the Udāna-vitalwind is situated. This wind swallows food, [and] it drinks water. The Nāga-vitalwind exists in the entire body. Through the vitalwind the body is caused to move. The Kūrma-vitalwind exists within the eyes. It causes [the] opening and closing [of the eyes]. From the Kṛkala-vitalwind gagging arises. From the Devadatta-vitalwind jawning arises. From the Dhanaṃjaya-vitalwind speech arises.
\end{tlate}
  %%%%%%%%%%%%%%%%%%%%%
    %%%%%%%%%%%%%%%%%%%
    %%%%%%%%%%%%%%%%%%%
    %%%%%%%%%%%%%%%%%%%
    %%%%%%%%%%%%%%%%%%%
\begin{tlate}
  \ekddiv{type=trans}
\bigskip
 \centerline{\textrm{\small{[Continuation of \textit{Lakṣyayoga} - Madhyalakṣya]}}}
 \bigskip
Now the central fixation is taught. White-colored, or yellow-colored or red-coloured or smoke-coloured or blue-coloured, like the flame of fire, equal to a lightning, like the orb of the sun, like a half-moon, appearing like flaming space, measured according to ones own body, the fixation shall be directed onto the center of the glowing mind. While abiding in this fixation the burning of the impurity in the center of the mind arises. The Sattva-quality of the mind becomes revealed. After this has happend, the person abides supreme bliss.  
\end{tlate}
  %%%%%%%%%%%%%%%%%%%%%
    %%%%%%%%%%%%%%%%%%%
    %%%%%%%%%%%%%%%%%%%
    %%%%%%%%%%%%%%%%%%%
    %%%%%%%%%%%%%%%%%%%
\begin{tlate}
  \ekddiv{type=trans}
 \bigskip
 \centerline{\textrm{\small{[The Divisions of Space]}}}
 \bigskip
 Now the divisions of space are taught.The fixations of them are taught: Space, beyond space, great space, space of reality, the space of the sun. The fixation onto the pure and formless space \textit{akāśa} shall be done internally as well as externally. Moreover, the fixation of the beyond-space \textit{parākāśa} which is equal to dense darkness shall be done internally and externally. Moreover, the fixation of the great space (\textit{mahākāśa}) which is the plethora of the burning fire of the time of dissolution shall be done internally and externally. Moreover, for whom internally and externally the brightness of millions of blazing lights arises, he shall execute the fixation [directed onto] the reality-space (\textit{tattvakāśa}). After that the fixation of the sun-space (\textit{sūryakāśa}) which is associated with sundisk's appearance of light shall be done internally and externally. From the execution of these fixations contact of diseases does not arise within the body. Thus wrinkles and grey hair, sin or merit do not arise. The nine cakras, the sixteen Adhāras, the three lakṣyas and die five spaces. Who does not know [them?] within ones own body, he is only a Yogin by name.
\end{tlate}
  %%%%%%%%%%%%%%%%%%%%%
    %%%%%%%%%%%%%%%%%%%
    %%%%%%%%%%%%%%%%%%%
    %%%%%%%%%%%%%%%%%%%
    %%%%%%%%%%%%%%%%%%%
\begin{tlate}
  \ekddiv{type=trans}
 \bigskip
 \centerline{\textrm{\small{[The order of Cakras]}}}
 \bigskip
Now the practice of the cakras is explained. At the pelvic floor there is the Brahmacakra. Above the pelvic floor at the root of the gender is the Svadiṣṭhānacakra. At the navel there is the Maṇipūrakacakra. At the heart the Anāhatacakra. Situated within the throat is the Viśuddhicakra. The sixth is the cakra of the palate. In the center of the eyebrows is the Ājñācakra. At the opening of Brahma is the Kālacakra. The ninth is the Ākāśacakra. It is supreme emptiness.
\end{tlate}
  %%%%%%%%%%%%%%%%%%%%%
    %%%%%%%%%%%%%%%%%%%
    %%%%%%%%%%%%%%%%%%%
    %%%%%%%%%%%%%%%%%%%
    %%%%%%%%%%%%%%%%%%%
\begin{tlate}
  \ekddiv{type=trans}
 \bigskip
 \centerline{\textrm{\small{[The sixteen Container]}}}
 \bigskip
 Now the divisions of the container-\textit{cakra}s are taught.
 From the execution of the fixation onto the light at the big toes of the feet, stability of the gaze arises.
 The root container is the second [one]. The heel of the backfoot is caused to be placed at the root of the big toe. As a result, the fire is strengthened.
 One heel is caused to be placed at the Root-container. The heel of the other foot is caused to be placed at the root of the big toe of this foot.
 The fire of it is caused to be kindled.
 The third is the place of the anus container. From the execution of expansion and contraction, a stable vital wind arises. Additionally, death of the person does not arise. Additionally, the person does not die.
 The fourth is the penis container. Due to the execution of repeated practice of contracting the penis in the midst of thereof, the adamantine channel appears in the middle of the staff of the back. From repeated practice again [and again], the transition of both breath and mind into its centre arises. Caused by their transition into the centre, the trinity of knots breaks. There, from the breaking of that, the vital wind after having filled up (the central channel?), resides in the centre of the Brahma-lotus. Then virility and strength arise. The person becomes youthful forever.
 The fifth is Udyāna. From performing \textit{bandha} there, urine and faeces disappear.
 The sixth is the navel container. From the repeated practice of \textit{praṇava}, the unstruck sound arises by itself.
 The seventh is the container of the heart form.
 The throat-support is the eighth. There the contraction of Jālaṃdhara is produced. While abiding therein, the vital wind in the Iḍā-and Piṅgalā-channels becomes stable.
 The ninth is the container of the uvula. There, the tip of the tongue becomes attached [to the uvula]. Then the nectar of immortality flows from the immortality digit. From drinking the nectar of immortality, diseases do not spread in the body.
 The tenth is the container of the palate. After the moving and milking has been done therein while abiding at the door of the uvula, the tongue resides inserted within the palate.
 The eleventh is the tongue container at the surface of the tongue. Within it, the tip of the tongue has to be churned. While doing [that] a sweet drink flows out. And in that manner, the knowledge of areas like poetry, singing, metric and dance is generated.
 On top is the twelfth, the teeth-support, which is situated in between the teeth. At this place, the tip of the tongue is to be positioned with force for the duration of one and a half \textit{ghāṭī}s (24+12 = 36 minutes). Abiding therein, the diseases of the practitioner will entirely disappear!
 The thirteenth is the nose container. While making it into the fixation object the mind becomes stable.
 The fourteenth is the container of breath at the root of the nose. From the execution of stabilizing the gaze onto this, the light of one's own self becomes perceptible within 60 months.
 He breaks the mundane prison through the perceptibility of the light.
 The fifteenth container is situated in the middle of the eyebrows. Due to stabilizing the gaze therein, 10 million rays of light sparkle.
 The sixteenth is the eye container. Without wavering, the gaze [ayam] is to be held at the tip of the finger without wavering. From practising this on earth, any energy exists [for him]. The light of everything that arises as the object of sight. From that sight, the person becomes omniscient. 
\end{tlate}
  %%%%%%%%%%%%%%%%%%%%%
    %%%%%%%%%%%%%%%%%%%
    %%%%%%%%%%%%%%%%%%%
    %%%%%%%%%%%%%%%%%%%
    %%%%%%%%%%%%%%%%%%%
\end{otherlanguage} 
%\section{Bibliography}
% \label{sec:bibli}
%\printshorthands[keyword=critEd]
%\printbibliography[title=Consulted Manuskcipts, keyword=codex]
%\printbibliography[title=Printed Editions, keyword=printsource]
%\printbibliography[title=Secondary Literature, keyword=seclit]
%\end{document}

7.82 recya vāyuṃ svakād dehāt punar ākṛṣya dhārayet 
hṛdaye yaḥ sthito granthis tasya nābhau kṣipen manaḥ 
7.83 mantraṃ caiva tathātmānam ekīkṛtya trayaṃ budhaḥ 
daṇḍākāraṃ nayet'tāvad yāvad brahmabilāntagam 
7.84 tatsthānāt prerayet tūrṇaṃ mahāyānena sundari 
karābhyāṃ caiva tarjanyāṃ pīḍayeta punaḥ punaḥ 
7.85 lalanāghaṇṭike yojya pañcamaṃ sthānam ākramet 
ākramed guhyacakraṃ tu karaṇaṃ cordhvamūlakam 
7.86 lagne granthitrayaṃ devi khagatir nātra saṃśayaḥ 
aṅgaṣaṭkaṃ śṛṇuṣvedaṃ kubjikāyāḥ kuleśvari 

Hallo! Gerne helfe ich Ihnen bei der Übersetzung des Sanskrit-Textes ins Deutsche. Hier ist die Übersetzung:

7.82 Wenn man den Wind (vāyuṃ) aus dem eigenen Körper (svakād dehāt) herauszieht (punar ākṛṣya dhārayet), sollte man den Knoten (granthi) im Herzen (hṛdaye) durch das Werfen (kṣipen) des Geistes (manaḥ) in den Nabel (nābhau) auflösen.

7.83 Der weise (budhaḥ) sollte den Geist (ātmānam) und den Mantra (mantraṃ) vereinen (ekīkṛtya trayaṃ) und bis zum Ende (yāvad brahmabilāntagam) den Daṇḍāsana durchführen (daṇḍākāraṃ nayet).

7.84 Dann sollte man schnell (tūrṇaṃ) durch das mächtige (mahāyānena) Sundari-Chakra (sundari) angetrieben werden (prerayet) und mit den Händen (karābhyāṃ) immer wieder die untere Spitze (tarjanyāṃ) drücken (pīḍayeta punaḥ punaḥ).

7.85 Dann sollte man das fünfte (pañcamaṃ) Chakra (sthānam) aktivieren (ākramet), das mit der Glocke (lalanāghaṇṭike yojya) verbunden ist, und dann das geheime Chakra (guhyacakraṃ) im oberen Teil (cordhvamūlakam) aktivieren.

7.86 Es gibt keinen Zweifel (saṃśayaḥ), dass die Göttin (devi) Khagati (khagatir) die drei Knoten (granthitrayaṃ) im Körper (lagne aṅgaṣaṭkaṃ) auflösen kann. Hören Sie das Śruti (śṛṇuṣvedaṃ) aus dem Kubjikā-Clan (kubjikāyāḥ kuleśvari).

Ich hoffe, das hilft Ihnen weiter!


\