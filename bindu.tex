%Ultimatives Tool zur Datierung:
%https://www.cc.kyoto-su.ac.jp/~yanom/pancanga/
%skp = ignored in edition
%skm = ignored in xml
%%%---2-DO---%%%:
% - add xml ids for cladistics
% - produce diplomatic transcripts for saktumiva
% - make extra layer in Apparatus for parallels in SVARODAYA, Siddhasiddhantapaddhati and Amanaska
% - check all daṇḍas!!! now I think that it's more likely that many of them were lost in copies. Lectio difficilior! Very unconventional style of the autor! 
% - read Sarvangayogapradipika, Maya Burger! 
% - maybe add second ciritical edition of yogasvarodaya?!
% - Korrekturlesen von \E!! 
% - Verspattern einbauen!
% - add all Testtimonia of SSP & Ysv
% - Sigla alphabetisch ordnen und! daṇḍas mit einkollationieren
% - präambel auslagern wie Jürgen 
% - grep-search alle Verse!!!!
% - Mss spreadsheet
% - sort N1,D1,B2 zu N1,N2,D1
% - sort all sigla alphabetically 
% - additions to U2: make footnotes for the bahir mātrā-s: explaining the inventions of female deities and tell that this is "schwer interpretierbar"
% - Belege für source und testimonia einfügen!!!
% - GIVE UNIQUE LABELS for TESTIMONIO AND SOurces
% - Edition mit Sätzen übereinander nennt sich: Synoptische Edition
% - Consider changing Lakṣya to Lakṣa
% - vEREINHEITLICHUNG von source und testium! 
%%%%%%%%%%%%%%%%%%%%%%%%%%%%%%%%%%%%%%%%%
% Don't forget
% Siddhasiddhantapaddhati Yogic Body descriptions are followed by Rāmacandra
% Quotes of the Yogasvarodaya in the Yoga Karṇikā
% Rāmacandra more a compiler than an author!!!
% Identify quotes of YTB in Haṭhasanketacandrikā 
%%%%%%%%%%%%%%%%%%%%%%%%%%%%%%%%%%%%%%%%%%%
%MSS notes
%
%--B: i and ī are not differenciated
%--P: no punctuation no daṇdas nothing
%--U1: dot . serves as daṇḍa 
%--\L and \U2 very similar
%--figure out for U2: // ajapājapaḥ sahasra // 6000 //gha 0 16 pa 0 40// \U2?!?!?!?!?!?
%%%%%%%%%%%%%%%%%%%%%%%%%%%%%%%%%%%%%%%%%%
%
% Einleitung Ideen 
% - sprachliche Simplizität
% - Potenzial als Anfängertext
% - Großartige Einführung in die Textkritik -> Synoptische Edition 
% - Gelegenheit Yogasvarodaya und Yogatattvabindu zu edieren 
% - Historische Evidenz entweder für das königliche Leben in einer Maṭha in der Nähe von Benares während der Muslimischen Herrschaft, oder sogar Lehrtext für die Bildung junger Prinzen  
% - eines der raren Beispiele der engen Verknüfung mehrerer Texte 
% - eines der raren Beispiele der Prosaisierung eines metrischen Textes 
% - Anwendung rezenter Technologie! 
% - How the text was construed -> intermingling of Ysv and SSP
% - Martin Straube: "jeder kleine Dorfhäuptling kann Rāja genannt werden". 
%%%%%%%%%%%%%%%%%%%%%%%%%%%%%%%%%%%%%%%%%%%
%Ich habe dieses Zitat gefunden
%Franz
%Franz Veit
%हठयोगः [Printed book page 5-501-c]
%हठयोगः , पुं, (हठेन योगः ।) योगविशेषः ।
%यथा, —
%“इदानीं हठयोगस्तु कथ्यते हठसिद्धिदः ।
%कृत्वासनं पवनाशं शरीरे रोगहारकम् ॥
%पूरकं कुम्भकञ्चैव रेचकं वायुना भजेत् ।
%इत्थं क्रमोत्क्रमं ज्ञात्वा पवनं सग्धयेत् सदा ॥
%धौत्यादिकर्म्मषट्कञ्च संस्कुर्य्याद्धठसाधकः ।
%एतन्नाड्यान्तु देवेशि ! वायुपूर्णं प्रतिष्ठितम् ॥
%ततो मनो निश्चलं स्यात्तत आनन्द एव हि ।
%हठयोगान्न कालः स्यान्मनः शून्ये भवेद्यदि ॥
%इदानीं हठयोगस्य द्वितीयं भेदवत् शृणु ।
%आकाशे नासिकाग्रे तु सूर्य्यकोटिसमं स्मरेत् ॥
%श्वेतं रक्तं तथा पीतं कृष्णमित्यादिरूपतः ।
%एवं ध्यात्वा चिरायुः स्यादङ्गाजननवर्ज्जितः ॥
%शिवतुल्यो महात्मासौ हठयोगप्रसादतः ।
%हठाज्ज्योतिर्म्मयो भूत्वा ह्यन्तरेण शिव भवेत् ।
%अतोऽयं हठयोगः स्यात् सिद्धिदः सिद्धसेवितः ॥”
%इति योगस्वरोदयः ॥ [ID=41348]

%Now, Haṭhayoga indeed is explained as that which gives the siddhi (accomplishment) of haṭha (persistence).
%One performs the wind-eating/serpent āsanam, which removes illness in the body
%and filling – kumbhaka – emptying may distribute the vāyu/wind.
%In this way, while being aware of the progress and regress of the breath one may feed on the wind continually.
%And with the six karmmas, dhauti etc., the Sādhaka of Haṭha may prepare/embellish himself.
%Thus/thereby, in the channel (nāḍī), Oh supreme Goddess, all of the winds (vāyu) are consecrated/placed.
%Then the mind may be unmoved and then bliss it really is.
%Through Haṭhayoga time will be no more, when the mind in emptiness abides.
%
%Now listen to the second disclosure of haṭhayoga:
%In space, on the tip of the nose indeed, one may remember equal to ten million suns,
%the primal forms: white, red, likewise yellow, dark blue.
%Thus meditating/visualizing, one may have a long life, free of the birth of the body,
%Equal to Śiva, this great soul, due to the blessing of Haṭhayoga,
%shall become through persistence (haṭha) a being of light and internally śiva.
%Therefore this Haṭhayoga grants accomplishment – it’s practiced by the Siddhas (accomplished ones).
%Franz
%Im śabdakalpadruma
%Franz
%Franz Veit
%fj.veit@gmail.com
\input{preamble.tex}
\author{Nils Jacob Liersch}
\title{Yogatattvabindu of Rāmacandra\\ A Critical and Synoptic Edition and Annotated Translation}
\date{\today}

\parindent=15pt
\begin{document}

% Zitiermöglichkeiten:
%\footcite[See][p.\,1]{goldstein01:_tibet_englis_diction_moder_tibet}
%\footnote{\cite{goldstein01:_tibet_englis_diction_moder_tibet}.}

\frontmatter
\thispagestyle{empty}
\begin{center}
  {\Large \emph{The Yogatattvabindu}}\\[3mm]
\end{center}



\newpage

\

\thispagestyle{empty}



\normalsize


\newpage


\begin{center}
\thispagestyle{empty}

\

\vskip 2mm

\begin{otherlanguage}{iast}
\LARGE \sanskritfont{Yogatattvabindu}
\end{otherlanguage}

\vskip .4cm

\Huge Yogatattvabindu \\[7mm]
\Large Critical and Synoptic \\
Edition with annotated Translation


\large

\vspace{3cm}

Von

Nils Jacob Liersch
\small
\vfill

\vfill

Indica et Tibetica Verlag \\ % $\cdot$ 
Marburg 2024

\vskip 6mm

\end{center}

\newpage
\newpage \ \thispagestyle{empty}
\small  \

\noindent

\
\vfill


\small
\noindent \textbf{Bibliographische Information Der Deutschen Bibliothek}

\noindent
Die Deutsche Bibliothek verzeichnet diese Publikation in der Deutschen Nationalbibliographie;
detaillierte bibliographische Informationen sind im Internet über http://dnb.ddb.de abrufbar.

\noindent
\textbf{Bibliographic information published by Die Deutschen Bibliothek}

\noindent
Die Deutsche Bibliothek lists this publication in the Deutsche Nationalbibliographie; detailed
bibliographic data is available in the Internet at http://dnb.ddb.de.  


\vskip 1cm

\noindent
\copyright\ Indica et Tibetica Verlag, Marburg 2024

\medskip

\noindent
Alle Rechte vorbehalten / All rights reserved

\medskip

\noindent
Ohne ausdrückliche Genehmigung des Verlages ist es nicht gestattet, das Werk oder einzelne Teile
daraus nachzudrucken, zu vervielfältigen oder auf Datenträger zu speichern.

\smallskip

\noindent
Apart from any fair dealing for the purpose of private study, research, criticism or review, no
part of this book may be reproduced or translated in any form, by print, photo form, microfilm, or
any other means without written permission. Enquiries should be made to the publishers.

\bigskip

\noindent
Satz: \ \ Nils Jacob Liersch \\
Herstellung: \ \ BoD – Books on Demand GmbH, Norderstedt  \\

\bigskip

\noindent
%\ISBN     

\normalsize

\newpage

%\maketitle
\clearpage
\tableofcontents
\addtocounter{page}{-1}
\thispagestyle{empty}
\clearpage

\chapter{Introduction}
\mainmatter

\chapter{The List of the 15 Yogas}
\label{yogas_list}
The authenticity of the list specifying the fifteen Yogas at the beginning of the text is ambiguous. This is due to the discrepancy between the structure of the Yogas presented in the text and the order presented in the list. For example, the text commences with a description of \textit{kriyāyoga} and goes on to describe \textit{siddhakuṇḍaliniyoga} and then mentions \textit{mantrayoga} without adhering to the order presented in the list. This incongruity raises questions as to why the text structure deviates from the list. However, the reference to \textit{jñānotpattav upāyaḥ} may provide some insight into why \textit{jñānayoga} is included as the second \textit{yoga} in the list. To reconcile these apparent inconsistencies, there are several possible explanations: 1) The text is severely corrupted. 2) The list was added by a different hand at a later time. 3) The term \textit{jñānayoga} is included as a result of the practice of \textit{siddhakuṇḍalinīyoga}, which is said to generate knowledge through the central channel, as stated in the text. These explanations may be combined to provide a comprehensive understanding of the situation.

\section{Lakṣyayoga}

\begin{itemize}
\item origin tantric Traditions -> e.g. Netratantra
\item also check Mālinivijayottara 2004 Vasudeva pp. 256-257
\item also \citetitle{birch2013} 2.10 Śāmbhavī Mudrā
  \end{itemize} 

\chapter{Sources}
\section{The Additions of  SORI 6082 - U\textsubscript{2}}
\label{discussionu2}
Analyse the additions of U\textsubscript{2} and present the \textit{cakra}s and their attriubutes in a table .
\begin{itemize}
\item  Muktabodha-Texte sehe ich 3 Belege für bahiśśakti Muktabodha/krīyakramādyotikā.html 2938 suṣirānte bahiśśaktiṃ vinyasedvyomarūpiṇīm | tasyā madhye tu Muktabodha/sakalāgamasārasaṅgraha.html 2186 suṣirāntabahiśśaktiṃ vyāpinīṃ cintayet tataḥ || Muktabodha/kriyakramadyotikavyākhyā.html 1846 tanmadhye ca bahiśśaktiṃ sudhābindu parisrutim
  \item  Parā\footnote{Im Kaśm. Śiv. °das ewige Wort, in welchem potentiell alle Begriffe und Worte ruhen; vgl. das śabdabrahma des Vyākaraṇa. [B.]― Schmidt S. 246}.
  \end{itemize}

\chapter{Conventions in the Critical Apparatus}
\section{Sigla in the Critical Apparatus}

\begin{itemize}
\item E : Printed Edition
\item P : Pune BORI 664
\item L : Lalchand Research Library LRL5876
\item B : Bodleian Oxford D 4587
\item \None : NGMPP B 38-31
\item \Ntwo : NGMPP B 38-35 / A 1327-14
\item \Done : IGNCA 30019
\item \Uone : SORI 1574
\item \Utwo: SORI 6082
\end{itemize}

The order of the readings in the critical apparatus is arranged according to the quality of readings in decending order. The critical apparatus is positive. Gemitation is not recorded. 

\section{Abbreviations}
\begin{itemize}
  \item qcr: quote cum notatio (quoted with reference)
  \end{itemize}

\section{Marking the Reliability of Sources and Testimonia in the Critical Apparatus}
\label{kennz}

To accurately depict information about the textual relationship and estimated degree of relatedness of a passage from the \textit{Yogatattvabindu} in the layers for sources and testimonia of the critical apparatus, a system of sigla was introduced.\footnote{This type of identification system is based on the use of the critical apparatus in \parencite[lii-liii]{steinkellner2005}. It was modified for the text-critical work on the \textit{Yogatattvabindu}.} The sigla are meaningful when a passage is corrupted in all witnesses and can only be reconstructed by means of other texts. The layers of the critical apparatus for sources and testimonia use the following sigla:

\begin{enumerate}
\item[\textbf{Ce}] \textit{citatum ex alio} / quotation from another (text).\footnote{The sigla \textbf{Ce} indicates an identical or largely identical content in the lesser witness and only allows for minor deviations in the wording of the passage.}
\item[\textbf{Cee}] \textit{citatum ex alio modo edendi} / quotation from another (text) with editorial changes.\footnote{The sigla \textbf{Cee} identifies passages with noticeable deviations in the lesser witness.}
\item[\textbf{Ci}] \textit{citatum in alio} / quotation in another (text).\footnote{The sigla \textbf{Ci} indicates an identical or largely identical content in the lesser witness and only allows for minor deviations in the wording of the passage.}
\item[\textbf{Cie}] \textit{citatum in alio modo edendi} / quotation in another (text) with editorial changes.\footnote{The sigla \textbf{Cie} identifies passages in the lesser witness with noticeable deviations that have the intended character of the composer.}
\item[\textbf{Re}] \textit{relatum ex alio} / (content), attested from another text.\footnote{The sigla \textbf{Re} identifies content parallels in the lesser witness that are relevant to the constitution of the critical text. It further indicates in certain cases that the composer might have used this source when composing his text.}
\item[\textbf{Ri}] \textit{relatum in alio} / (content), attested in another text.\footnote{The sigla \textbf{Ri} identifies content parallels in the lesser witness that are relevant to the constitution of the critical text.}
\end{enumerate}

The following acronyms refer to passages that originated from texts that the author of the \textit{Yogatattvabindu} utilized in compiling his work: \textbf{Ce}, \textbf{Cee}, \textbf{Re}. These texts must predate the \textit{Yogatattvabindu}. The other acronyms, such as \textbf{Ci}, \textbf{Cie}, and \textbf{Ri}, are texts that have adopted passages from the \textit{Yogatattvabindu}, or verses or passages that share similar content with the \textit{Yogatattvabindu}, but their relation is given literally, making it impossible to determine who adopted from whom. \textbf{Re} and \textbf{Ri} each refer to passages that are so closely related in content to those of the \textit{Yogatattvabindu} that they are significant in reconstructing a passage.\footnote{\textbf{Ce} and \textbf{Cee} have the highest degree of reliability, \textbf{Ci} and \textbf{Cie} have a moderate degree, and \textbf{Re} and \textbf{Ri} have the lowest.}

\section{Punctuation}

The inconsistent use of punctuation marks in the available witnesses necessitates standardization. Upon close examination, it appears that punctuation has frequently been dropped or added during the transmission of the texts. The neglect or improper handling of punctuation by the copists has resulted in different versions of lists with and without punctuation. In many instances, missing punctuation has led to the addition of case endings, alteration of the text, and the combination of list items into compound formations that were not present in the original text. Although punctuation plays an important role, deviations in punctuation at the end of sentences, lists, and verse-numbering will only be extensively documented in the critical apparatus of the printed edition. This means that emendations of obvious punctuation mistakes will not be recorded in the critical apparatus. However, the digital edition of this work provides a more detailed documentation of deviations in punctuation through diplomatic transcripts of each witness, and even has a function to display sentences cumulatively.

In the printed edition of the \textit{Yogatattvabindu}, standard conventions of punctuation are followed. In verse poetry, a \textit{daṇḍa} (|) marks the end of a half-verse or half of the \textit{śloka}, and a double \textit{daṇḍa} (||) marks the end of a verse. In prose, a single \textit{daṇḍa} indicates the end of a sentence, and a double \textit{daṇḍa} marks the end of a paragraph. Variations in the use of \textit{avagraha} will be recorded, and items in lists will be separated by a double-\textit{daṇḍa}.

\section{Sandhi}

Among the witnesses we see deviating and inconsistent application of \textit{sandhi}. There is no clear evidence that originally \textit{sandhi} was intentionally not applied. This edition will therefore apply \textit{sandhi} consistently throughout the constituted text to provide a readable text sticking to contemporary conventions in Sanskrit. The variant readings concerning \textit{sandhi} are recorded consistently in the apparatus criticus. This is due to various textcritical problems arising from the inconsistent usage of punctuation which results in application or non-application of \textit{sandhi} wheter the respective witness applied a \textit{daṇḍa} or not. This is particularly the case within lists, which frequently occur in our compilation. Items were most likely originally separated by \textit{daṇḍa}. 


\section{Class Nasals}

Due to inconsistent use of class nasals among the witnesses \textit{anusvāra}s have been substituted with the respective class nasals throughout the edition.

\section{Lists}

Lists are a frequent feature in the \textit{Yogatattvabindu}. The text opens with a list of 15 Yogas and there are many more lists utilized throughout its content. To produce a consistent and easily readable edition, all lists have been identified, normalized to the Nominative Singular or Nominative Plural form of the respective item, or in the case of explanatory lists, to the Ablative Singular or Plural. The items are separated by a double \textit{daṇḍa}. Differences in punctuation and simple punctuation emendations, unless they are text-critically or systematically significant, will not be recorded in the apparatus criticus.
\clearpage

\section{Structural Analysis of the Yogatattvabindu}

\chapter{Critical Edition \& Annotated Translation}
\cleardoublepage
\begin{alignment}[
  texts=edition[class="edition"];
  translation[class="translation"],
  ]
  \begin{edition}
    \ekddiv{type=ed}
    \centerline{\textrm{\small{[\uproman{24}.\textsuperscript{\coro{\lowroman{1}-\lowroman{16}}}Bāhyalakṣya]}}}
    \bigskip
    \begin{prose}
                \noindent
                \note[type=source, labelb=151, lem={\textbf{Re}}]{PT\textsuperscript{qcr \cdot YSV} (Ed. p. 837): idānīṃ vāhyalakṣāṇi siddhidāni śṛṇu priye | dhāraṇākhyā tu caitāni jñātavyāni viśeṣataḥ |}
                \note[type=testium, labelb=151a, lem={\textbf{Ri}}]{SSP 2.28 (Ed. p. 39): atha bahirlakṣyaṃ kathyate| nāsāgrāt bahiraṅgulacatuṣṭaye nīlajyotiḥsaṃkāśaṃ lakṣayet |}
idānīṃ
\app{\lem[wit={P}]{bāhyalakṣyaṃ}
  \rdg[wit={E}]{lakṣyaṃ}
  \rdg[wit={B}]{ṣāhyalakṣa}
  \rdg[wit={L}]{bāhyalakṣa}
  \rdg[wit={N1}]{°lakṣaṃ}
  \rdg[wit={D,N2}]{°lakṣaṇa}
  \rdg[wit={U1}]{°lakṣyaḥ}
  \rdg[wit={U2}]{lakṣaṇaṃ}}
kathyate/
%------------------------------
%nāsāgrād ārabhyāṃgulacatuṣṭaya--pramāṇaṃ nīlākāraṃ tejaḥ   pūrṇam ākāśaṃ  lakṣyaṃ  karttavyam/ \E
%nāsāgrād ārabhyāṃgulacatuṣṭaya--pramāṇaṃ nilākāraṃ tejaḥ   pūrṇam ākāśaṃ lakṣyaṃ  kartavyaṃ  \P
%nāsāgrād ārabhyāṃgulacatuṣṭayaṃ pramāṇaṃ nilākāraṃ   jaḥ   pūrṇam ākāśa--lakṣaṃ   kartavyaṃ//    \B
%nāsāgrād ārabhyāṃgulacatuṣṭayaṃ pramāṇaṃ nilākāraṃ tejaḥ// pūrṇam ākāśaṃ lakṣaṃ   kartavyaṃ// \L
%nāsāgrād ārabhyāṃgulacatuṣṭaya--pramāṇaṃ nīlākāraṃ teja----pūrṇam ākāśa--lakṣaṃ   karttavyaṃ \N1
%nāsāgrād ārabhyāṃgulacatuṣṭaya--pramāṇaṃ nīlākāraṃ teja----pūrṇam ākāśa---lakṣaṃ   karttavyaṃ \D
%nāsāgrād ārabhyāṃgulacatuṣṭaya--pramāṇaṃ nirākāraṃ teja----pūrṇam ākāśa---lakṣaṇaṃ karttavyaṃ// \N2
%nāsāgrād ārabhyāṃgulacatuṣṭaya--pramāṇaṃ nīlākāraṃ tejaḥ   pūrṇam ākāśaṃ lakṣyaṃ  karttavyam \U1
%nāsāgrād ārabhyāṃgulacatuṣṭaya--pramāṇaṃ nīlākāraṃ tejaḥ   pūrṇakām ākāśa-lakṣyaṃ  karttavyaṃ \U2 %%%411.jpg
%------------------------------
%Beginning at a four finger wide distance from the tip of the nose, the space-element, appearing blue, being full of light shall be made the target [of fixation].
%------------------------------
nāsāgrād-ārabhyāṅgula\app{\lem[wit={ceteri}, alt={catuṣṭaya°}]{catuṣṭaya\skp{-}}
  \rdg[wit={B,L}]{catuṣṭayaṃ}
}pramāṇaṃ
\app{\lem[wit={ceteri}]{nīlākāraṃ}
  \rdg[wit={B,L,P}]{nilākaraṃ}
  \rdg[wit={N2}]{nirākāraṃ}}
\app{\lem[wit={D,N1,N2},alt={teja°}]{teja}
  \rdg[wit={ceteri}]{tejaḥ}
  \rdg[wit={B}]{jaḥ}
}\app{\lem[wit={ceteri}, alt={pūrṇam}]{pūrṇa\skp{m-ā}}
  \rdg[wit={U2}]{pūrṇakām}
}\app{\lem[wit={ceteri},alt={ākāśa°}]{\skm{m-ā}kāśa}
    \rdg[wit={E,P,L,U1}]{ākāśaṃ}
}\app{\lem[wit={E,P,U1,U2}]{lakṣyaṃ}
  \rdg[wit={B,D,L,N1}]{lakṣaṃ}
  \rdg[wit={N2}]{lakṣaṇaṃ}}
kartavyaṃ/
%------------------------------
%atha vā nāsāgrād ārabhya ṣaḍaṃgulapramāṇaṃ    pavanatattvaṃ dhūmrākāraṃ        lakṣyaṃ karttavyam// \E
%atha vā nāsāgrād ārabhya ṣaḍaṃgulapramāṇaṃ    pavanatatvaṃ  dhūmrākāraṃ        lakṣyaṃ karttavyam \P
%atha vā nāsāgrād ārabhya ṣaḍaṃgulaṃ pramāṇaṃ ?bi?ṣi?īnāvarṇaṃ .. .. .. ..??.  lakṣyaṃ kartavyam/  \B
%\om \L
%atha vā nāsāgrād ābhya   ṣadaṃgulapramāṇaṃ    pavanatatvaṃ dhūmrākāraṃ        lakṣaṃ karttavyaṃ/  \N1
%atha vā nāsāgrād ābhya   ṣadaṃgulapramāṇaṃ    pavanatatvaṃ dhūmrākāraṃ        lakṣaṃ karttavyaṃ// \D
%atha vā nāsāgrārabhya    ṣadaṃgulapramāṇaṃ    pavanatatvaṃ dhūmrākāraṃ        lakṣaṇaṃ karttavyaṃ// \N2
%atha vā nāsāgrād ārabhya dvadaśaṃgulapramāṇaṃ pavanatatvaṃ dhūmrākāraṃ        lakṣyaṃ karttavyaṃ \U1
%atha vā nāsāgrād ārabhya ṣaḍaṃgulapramāṇaṃ    pavanatatvaṃ dhūmrākāraṃ        lakṣaṃ karttavyaṃ// \U2
%------------------------------
%Or, beginning at a six-finger wide distance from the tip of the nose, the wind-element, appearing greyish, shall be made the target [of fixation]. 
%------------------------------
\note[type=source, labelb=152, lem={\textbf{Re}}]{PT\textsuperscript{qcr \cdot YSV} (Ed. p. 837): līlayā bhāvayel līnaṃ jyotiḥpūrṇaṃ mahāparam | atha vā tatra deveśi dhūmrākāraṃ ṣaḍaṅgulam |}
\note[type=testium, labelb=152a, lem={\textbf{Ri}}]{SSP 2.28 (Ed. p. 39): atha vā nāsāgrād ṣaḍaṅgulam adhovāyutattvaṃ dhūmravarṇaṃ lakṣayet |}
atha vā
\app{\lem[wit={ceteri}]{nāsāgrād\skp{-}ārabhya}
      \rdg[wit={N1,D}]{nāsāgrād ābhya}
      \rdg[wit={N2}]{nāsāgrārabhya}}
    \app{\lem[wit={ceteri}]{ṣaḍaṃgulapramāṇaṃ}
      \rdg[wit={B}]{ṣaḍaṃgulaṃ pramāṇaṃ}
      \rdg[wit={U2}]{dvadaśaṃgulapramāṇaṃ}}
     \app{\lem[wit={E}]{pavanatattvaṃ}
       \rdg[wit={ceteri}]{°tatvaṃ}
       \rdg[wit={B}]{\illeg}}
     \app{\lem[wit={ceteri}]{dhūmrākāraṃ}
       \rdg[wit={B}]{\illeg}}
     \app{\lem[wit={ceteri}]{lakṣyaṃ}
       \rdg[wit={D,N1,U2}]{lakṣaṃ}
       \rdg[wit={N2}]{lakṣaṇaṃ}}
     karttavyaṃ/\note[type=philcomm, labelb=152a, lem={\uproman{24}.\textsuperscript{\lowroman{3}}}]{Sentence is \om in \getsiglum{L}.}
%------------------------------
%\om \E
%\om \P
%\om \B
%\om \L
%atha vā nāsāgrād ārabhyā  ṣaḍaṃgulapramāṇām atiraktaṃ        tejo lakṣaṃ karttavyaṃ \N1
%atha vā nāsāgrād ārabhya  ṣaḍaṃgulapramāṇām atirattaṃ        tejo lakṣaṃ karttavyaṃ// \D
%atha vā nāsāgrād ārabhyaṃ ṣṭāṃgulapramāṇam atirakṭaṃ         tejo lakṣaṇaṃ kartavyaṃ// \N2
%atha    nāsāgrād ārabhyāṣṭaṃgulapramāṇam     itiriktaṃ       tejo lakṣyaṃ karttavyaṃ/ \U1
%atha vā nāsāgrād ārabhyaṃ        ṣṭagulapramāṇaṃ matiraktaṃ  teja lakṣyaṃ karttavyaṃ// \U2 
%------------------------------
% Or, beginning at an eight-finger wide distance from the tip of the nose, the very red fire-element shall be made the target [of fixation]. 
%------------------------------
     \note[type=testium, labelb=153, lem={\textbf{Re}}]{PT\textsuperscript{qcr \cdot YSV} (Ed. p. 837): atha vāṣṭāṅgulaṃ raktaṃ nāsikopari lakṣayet |}
     \note[type=testium, labelb=153a, lem={\textbf{Ri}}]{SSP 2.28 (Ed. p. 39): atha vā aṣṭāṅgula āraktaṃ tejastattvaṃ lakṣayet |}
atha
\app{\lem[wit={ceteri}]{vā}
  \rdg[wit={U1}]{\om}}
nāsāgrā\skp{d-ā}\app{\lem[wit={U1}, alt={ārabhyāṣṭāṃgula°}]{\skm{d-ā}rabhyāṣṭaṃgulapramāṇa\skp{m-a}}
  \rdg[wit={N1}]{ārabhyā ṣaḍaṃgulapramāṇām}
  \rdg[wit={D}]{ārabhya ṣaḍaṃgulapramāṇām}
  \rdg[wit={N2}]{ārabhyaṃ ṣṭāṃgulapramāṇam}
  \rdg[wit={U2}]{ārabhyaṃ ṣṭagulapramāṇaṃ}
}\app{\lem[wit={N1,N2}, alt={atiraktaṃ}]{\skm{m-a}tiraktaṃ}
  \rdg[wit={D}]{atirattaṃ}
  \rdg[wit={U1}]{itiriktaṃ}
  \rdg[wit={U2}]{matiraktaṃ}}
\app{\lem[wit={ceteri}]{tejo}
  \rdg[wit={U2}]{teja°}}
\app{\lem[wit={U1,U2}]{lakṣyaṃ}
  \rdg[wit={N1,N2}]{lakṣaṃ}
  \rdg[wit={N2}]{lakṣaṇaṃ}}
karttavyaṃ/\note[type=philcomm, labelb=153a, lem={\uproman{24}.\textsuperscript{\lowroman{4}}}]{Sentence \om in \getsiglum{Y}, except for \getsiglum{U2}.}
%------------------------------
%\om \E
%\om \P
%\om \B
%\om \L
%atha vā nāsāgrād ārabhya daśāṃgulapramāṇaṃ śuklaṃ caṃcalam    udakaṃ lakṣya   karttavyaṃ/  \N1
%atha vā nāsāgrād ārabhya daśāṃgulapramāṇaṃ śuklaṃ caṃcalam    udakaṃ lakṣya   karttavyaṃ// \D
%atha vā nāsāgrād ārabhya daśāṃgulapramāṇaṃ śuklaṃ caṃdrākāram udakaṃ lakṣyaṃ  kartavyaṃ    \U1
%atha vā nāsāgrād ārabhya daśāṃgulapramāṇaṃ śuklaṃ caṃcalam    udakaṃ lakṣaṇaṃ kartavyaṃ//  \N2 [S.7 Verso, Zeile 1]
%atha vā nāsāgrād ārabhya daśāṃgulapramāṇaṃ śuklaṃ caṃcalam    udakaṃ lakṣaṃ   kartavyaṃ//  \U2
%------------------------------
%Or, beginning at a ten-finger wide distance from the tip of the nose, the white water[-element] being fickle shall be made the target [of fixation]. 
%------------------------------
\note[type=philcomm, labelb=154, lem={daśāṅgulapramāṇaṃ}]{The instruction for a ten-finger wide distance is absent in the surviving testimonia of the \textit{Yogasvarodaya}. However, it can be found in another source text of the \textit{Yogatattvabindu}, the \citetitle{ssplonavla} 2.28 (Ed. p. 39).}
\note[type=testium, labelb=154b, lem={\textbf{Ri}}]{SSP 2.28 (Ed. p. 39): atha vā daśāṅgule kallolavad āpas tattvaṃ lakṣayet |}
atha vā nāsāgrād-ārabhya daśāṅgulapramāṇaṃ śuklaṃ
\app{\lem[wit={ceteri}]{cañcalam}
  \rdg[wit={U1}]{caṃdrākāram}} udakaṃ
\app{\lem[wit={U1}]{lakṣyaṃ}
  \rdg[wit={N1,D}]{lakṣya}
  \rdg[wit={N2}]{lakṣaṇaṃ}
  \rdg[wit={U2}]{lakṣaṃ}} kartavyaṃ/\note[type=philcomm, labelb=153b, lem={\uproman{24}.\textsuperscript{\lowroman{5}}}]{Sentence \om in \getsiglum{Y}, except for \getsiglum{U2}.}\textsuperscript{\begin{otherlanguage}{english}\coro{[\lowroman{5}]}\end{otherlanguage}}
%------------------------------
%atha vā nāsāgrād ārabhya tattvaṃ dvādaśāṃgulapramāṇaṃ   pītavarṇaṃ  pṛthvītattvaṃ lakṣyaṃ  karttavyam/ \E
%atha vā nāsāgrād ārabhya         dvādaśāṃgulapramāṇaṃ   pītavarṇaṃ  pṛthvītatvaṃ  lakṣyaṃ  karttavyaṃ \P
%atha vā nāsāgrād ārabhya         dvadaśāṃgulapramāṇaṃ   pītavarṇaṃ  pṛthvītatvaṃ  lakṣaṃ   kartavyaṃ// \B
%atha vā nāsāgrād ārabhya         dvādaśāṃgulapramāṇaṃ   pītavarṇaṃ  pṛthvītatvaṃ  lakṣaṃ   kartavyaṃ/  \L
%atha vā nāsāgrād ārabhya         dvadaśāṃgulapramāṇaṃ   pītavarṇṇaṃ prthvītatvaṃ  lakṣaṃ   karttavyaṃ/   \N1
%atha vā nāsāgrād ārabhya         dvadaśāṃgulapramāṇaṃ   pītavarṇṇaṃ prthvītatvaṃ  lakṣaṃ   karttavyaṃ/   \D
%atha vā nāsāgrād ārabhya         dvadaśāṃgulapramāṇaṃ   pītavarṇaṃ  prthvītatvaṃ  lakṣaṇaṃ karttavyaṃ//  \N2
%atha vā nāsāgrād ārabhya         dvādaśā aṃgulapramāṇaṃ pītavarṇaṃ  prthvītatvaṃ  lakṣyaṃ  karttavyaṃ   \U1
%atha vā nāsāgrād ārabhya         dvādaśāṃgulapramāṇaṃ   pītavarṇaṃ  pṛthvītatvaṃ  lakṣaṃ   karttavyaṃ//    \U2
%------------------------------
%Or, beginning at a twelve-finger wide distance from the tip of the nose, the yellow-colored earth-element shall be made the target [of fixation].  
%------------------------------
\note[type=source, labelb=155, lem={\textbf{Re}}]{PT\textsuperscript{qcr \cdot YSV} (Ed. p. 837): dvādaśāṅgulamānaṃ vā pṛthvītattvan tu pītabham | lakṣayed atha vā tatra koṭisūryasamaprabham | tejaḥ puñjaṃ mahākāśaṃ tattad dhyānāc chivo bhavet | ākāśamadhye ākāśoparito dṛṣṭis usthiram | kṛtvā dhyānād vinā sūryaṃ caṇḍasūryan tu paśyati | atha vā lakṣam etat tu karttur vahiḥ śivopari |}
\note[type=testium, labelb=155a, lem={\textbf{Ri}}]{SSP 2.28 (Ed. p. 39): atha vā nāsāgrād dvādaśāṅgule pītavarṇaṃ pārthivatattvaṃ lakṣayet |}
atha vā nāsāgrād-ārabhya
\app{\lem[wit={ceteri}]{dvādaśāṃgulapramāṇaṃ}
  \rdg[wit={E}]{tattvaṃ dvādaśāṃgulapramāṇaṃ}
  \rdg[wit={U1}]{dvādaśā aṃgulapramāṇaṃ}}
pītavarṇaṃ pṛthvītattvaṃ
\app{\lem[wit={E,P,U1}]{lakṣyaṃ}
  \rdg[wit={N2}]{lakṣaṇaṃ}
  \rdg[wit={ceteri}]{lakṣaṃ}}
kartavyaṃ/
%------------------------------
%atha vā nāsāgrād ārabhya koṭisūryasamaprabhaṃ tejaḥ/ pūrṇam ākāśatattvaṃ lakṣyaṃ karttavyam/                 \E
%atha vā nāsāgrād ārabhya koṭisūryasamaprabhaṃ tejaḥpūrṇam   ākāśatatvaṃ lakṣyaṃ karttavyaṃ                      \P
%atha vā nāsāgrād ārabhya koṭisūryasamaprabhaṃ tejaḥ/ pūrṇam ākāśatatvaṃ lakṣaṃ kartavyaṃ//                   \B %%%%DSCN7161.JPG letzte 3 Zeilen!
%atha vā nāsāgrād ārabhya koṭisūryasamaprabhāṃ tejaḥpūrṇam   ākāśatatvaṃ lakṣaṃ karttavyaṃ//                   \L
%atha vā nāsāgrād ārabhya koṭisūryasamaprabhaṃ tejaḥpūrṇaṃ   ākāśatatvaṃ lakṣyaṃ karttavyaṃ/                   \N1
%atha vā nāsāgrād ārabhya koṭisūryasamaprabhaṃ tejaḥpūrṇaṃ   ākāśatatvaṃ lakṣyaṃ karttavyaṃ//                   \D
%atha vā nāsāgrād ārabhya koṭisūryasamaprabhaṃ tejaḥpūrṇa    ākāśatatvaṃ lakṣaṇaṃ karttavyaṃ//                    \N2
%atha vā nāsāgrād ārabhya koṭisūryasamaprabhaṃ tejaḥpūrṇaṃ   ākāśatatvaṃ lakṣyaṃ karttavyaṃ                     \U1
%atha vā nāsāgrād ārabhya koṭisūryasamaprabhaṃ tejaḥpūrṇaṃ   ākāśatatvaṃ lakṣaṃ karttavyaṃ//                    \U2
%------------------------------
%Or, beginning at the tip of the nose\footnote{Given the clear instructions of the respective distance of the exercise in the previous sentences, it is surprising that this instruction is lacking the mention of the distance.} the space-element full of fire shining like ten million suns shall be made the target [of fixation].  
%------------------------------
\note[type=testium, labelb=155b, lem={\textbf{Ri}}]{SSP 2.28 (Ed. p. 40): atha vā ākāśamukhaṃ dṛṣṭvā lakṣayat kiraṇākulitaṃ paśyati |}
atha vā nāsāgrād ārabhya koṭisūrya\app{\lem[wit={ceteri}]{samaprabhaṃ}
  \rdg[wit={L}]{°prabhāṃ}}
\app{\lem[wit={ceteri}, alt={tejaḥpūrṇam}]{tejaḥpūrṇa\skp{m-ā}}
  \rdg[wit={E,B}]{tejaḥ | pūrṇaṃ}
  \rdg[wit={X}]{pūrṇaṃ}
  \rdg[wit={N2}]{pūrṇa}
}\skm{m-ā}kāśatattvaṃ
\app{\lem[wit={D,E,P,N1,U1}]{lakṣyaṃ}
  \rdg[wit={B,L,U2}]{lakṣaṃ}
  \rdg[wit={N2}]{lakṣaṇaṃ}}
karttavyaṃ/
    \end{prose}
  \end{edition}
  \begin{translation}
    \ekddiv{type=trans}
\centerline{\textrm{\small{[\uproman{24}.\textsuperscript{\coro{\lowroman{1}-\lowroman{16}}}Bāhyalakṣya]}}}
\bigskip
\begin{tlate}
  \blfootnote{\ldots see in the source, PT\textsuperscript{qcr \cdot YSV} (Ed. p. 837): \begin{quote}``Having gained discrimination of the self, one calms the restless mind. Having seen darkness of ignorance caused by sense objects, one knows the supreme soul and becomes a soul beyond illusion, free from attachment to worldly pleasures, and goes through birth and death in truth, again and again. Abandoning aversion etc. [and] riches, which increase suffering like poison, one is liberated by Jñānayoga, which is as powerful as millions of suns.''\end{quote} This oberservation perfectly suits the initial definition of Rājayoga (cf. \textbf{\uproman{1}.\textsuperscript{\lowroman{3}}}, p.\pageref{intro}) in which it is defined as a practice that works even if the practitioner is leading an exuberant wealthty lifestyle.}
  \noindent
  Now the external fixation is taught. Beginning at a four-finger wide distance from the tip of the nose, the space-element, appearing blue, being full of light, shall be made the target [of fixation]. Or, beginning at a six-finger wide distance from the tip of the nose, the wind-element, appearing greyish, shall be made the target [of fixation]. Or, beginning at an eight-finger wide distance from the tip of the nose, the very red fire-element shall be made the target [of fixation]. Or, beginning at a ten-finger wide distance from the tip of the nose, the white fickle water-element, shall be made the target [of fixation].\textsuperscript{\coro{[\lowroman{5}]}} Or, beginning at a twelve-finger wide distance from the tip of the nose, the yellow-coloured earth-element shall be made the target [of fixation]. Or, beginning at the tip of the nose\footnote{Given the explicit instructions of the respective distance of the exercise in the previous sentences, it is surprising that this instruction is lacking the mention of the distance.} the space-element full of fire shining like ten million suns shall be made the target [of fixation].\footnote{The first five external targets, associated with the five elements can be also identified within \citetitle{sarvangayoga} 3.29-33 (Ed. pp. 104-105).  
    \begin{quote}
bāhya lakṣa aur puni jāṃnahūṃ | paṃca tatva kī lakṣa su ṭhānahuṃ |\\
agra nāsikā aṃgula cārī | nīla varṇa nabha deṣi bicārī || 29 ||\\  
nāsā agra aṃgul chah deṣaiṃ | dhūmrahi varṇ vāyu tat peśai |\\
aṃgul aṣṭa nāsikā āgai | rakta varṇ su vahni tat jāgai || 30 ||\\
nāsā agra aṃgul daś tāṃī | śvet varṇ jal deṣi tahāṃī |\\
nāsā agra su aṃgul bārā | pīta varṇ bhū deṣi apārā || 31 ||\\
bāhya lakṣa aur bahuterī | so jānaṃ jo pāvai serī | \\
sataguru kṛpā karai jau kabahī | dei batāi chinak maiṃ sabahī || 32 ||\\
\end{quote}
\begin{quote}
  (29) Contemplate the external target repeatedly; focusing on the five elements. \ldots
\end{quote}
}
    \end{tlate}
  \end{translation}
\end{alignment}
\ekdpb*{}
%%%%%%%%%%%%%%%%%%%%%%%%%%%%%%%%%%%%%%%%%%
%%%%%%%%%%%%%%%%%%%%%%%%%%%%%%%%%%%%%%%%%%
%%%%%%%%PAGEBREAK%%%%%%%PAGEBREAK%%%%%%%%%
%%%%%%%%%%%%%%%%%%%%%%%%%%%%%%%%%%%%%%%%%%
%%%%%%%%%%%%%%%%PAGEBREAK%%%%%%%%%%%%%%%%%
%%%%%%%%%%%%%%%%%%%%%%%%%%%%%%%%%%%%%%%%%%
%%%%%%%%PAGEBREAK%%%%%%%PAGEBREAK%%%%%%%%%
%%%%%%%%%%%%%%%%%%%%%%%%%%%%%%%%%%%%%%%%%%
%%%%%%%%%%%%%%%%%%%%%%%%%%%%%%%%%%%%%%%%%%
%%%%%%%%%%%%%%%%%%%%%%%%%%%%%%%%%%%%%%%%%%
%%%%%%%%%%%%%%%%%%%%%%%%%%%%%%%%%%%%%%%%%%
%%%%%%%%PAGEBREAK%%%%%%%PAGEBREAK%%%%%%%%%
%%%%%%%%%%%%%%%%%%%%%%%%%%%%%%%%%%%%%%%%%%
%%%%%%%%%%%%%%%%PAGEBREAK%%%%%%%%%%%%%%%%%
%%%%%%%%%%%%%%%%%%%%%%%%%%%%%%%%%%%%%%%%%%
%%%%%%%%PAGEBREAK%%%%%%%PAGEBREAK%%%%%%%%%
%%%%%%%%%%%%%%%%%%%%%%%%%%%%%%%%%%%%%%%%%%
%%%%%%%%%%%%%%%%%%%%%%%%%%%%%%%%%%%%%%%%%%
%%%%%%%%%%%%%%%%%%%%%%%%%%%%%%%%%%%%%%%%%%
%%%%%%%%%%%%%%%%%%%%%%%%%%%%%%%%%%%%%%%%%%
%%%%%%%%PAGEBREAK%%%%%%%PAGEBREAK%%%%%%%%%
%%%%%%%%%%%%%%%%%%%%%%%%%%%%%%%%%%%%%%%%%%
%%%%%%%%%%%%%%%%PAGEBREAK%%%%%%%%%%%%%%%%%
%%%%%%%%%%%%%%%%%%%%%%%%%%%%%%%%%%%%%%%%%%
%%%%%%%%PAGEBREAK%%%%%%%PAGEBREAK%%%%%%%%%
%%%%%%%%%%%%%%%%%%%%%%%%%%%%%%%%%%%%%%%%%%
%%%%%%%%%%%%%%%%%%%%%%%%%%%%%%%%%%%%%%%%%%
\begin{alignment}[
  texts=edition[class="edition"];
  translation[class="translation"],
  ]
  \begin{edition}
    \ekddiv{type=ed}
    \begin{prose}
      \noindent
%------------------------------
%ākāśamadhye  ākāśopari    dṛṣṭiṃ kṛtvā             dhyānakāraṇāt// sūryaṃ vinā sūryasambandhinī  sahasrakiraṇapaṅktīḥ   paśyati/ \E
%             ākāśopari    dṛṣtiṃ kṛtvā             dhyānakaraṇāt   sūryaṃ vinā sūryasaṃbaṃdhīnīṃ sahasrakiraṇāvalīṃ     pati   \P
%             ākāśopari    dṛṣti  kṛtvā ākāśamadhye dhyānakaraṇāt// sūryaṃ vinā sūryasaṃbaṃdhīnī  sahasrakiraṇāvali      paśyatī// \B
%                                       ākāśamadhye dhyānakaraṇāt// sūryaṃ vinā sūryasaṃbaṃdhinī  sahasrakiraṇāvali      paśyati// \L
%ākāśamadhye  ākāśoparī vā dṛṣṭiṃ kṛtvā             dhyānakaraṇāt   sūryaṃ vinā sūryasaṃbaṃdhinī  sahasrāṇy api kīraṇāṇi paśyatī/     \N1
%ākāśamadhye  ākāśopari vā dṛṣṭiṃ kṛtvā             dhyānakaraṇāt   sūryaṃ vinā sūryasaṃbaṃdhinī  sahasrāṇapi   kīraṇāṇi paśyatī//     \D
%ākāśamadhye  ākāśopari vā dṛṣṭiṃ kṛtvā             dhyānakaraṇāt   sūrya  vinā sūryasaṃbaṃdhinī  sahasrāṇapi   kiraṇāṇi paśyate//     \N2
%ākāśamadhye  ākāśopari vā dṛṣṭiṃ kṛtvā             dhyānakaraṇāt   sūryaṃ vinā sūryasaṃbaṃdhinī  sahasrāṇy api kiraṇāṇi paśyaṃti     \U1
%ākāśamadhye  ākāśopari vā dṛṣṭiṃ kṛtvā             dhyānakaraṇāt// sūrya  vinā sūryasaṃbaṃdhinī  sahasrakiraṇāvaliṃ     paśyati//     \U2
%------------------------------ 
%After having fixed the gaze on the space-element or above the space-element, due to the execution of meditation [on either target] he sees the sun without the group of thousand rays related to the sun. 
%------------------------------
\note[type=testium, labelb=155c, lem={\textbf{Ri}}]{SSP 2.28 (Ed. p. 40): evaṃ nirmalīkaraṇam| atha vordhvadṛṣṭayāntarālaṃ lakṣayet | jyotir mukhāni paśyati | atha vā yatra tatrākāśaṃ lakṣayet | ākāśasadṛśaṃ cittaṃ muktipradaṃ bhavati |}
\app{\lem[wit={ceteri}]{ākāśamadhye}
  \rdg[wit={B,L,P}]{\om}}
\app{\lem[wit={ceteri}]{ākāśopari}
  \rdg[wit={N1}]{ākāśoparī}}
\app{\lem[wit={ceteri}]{vā}
  \rdg[wit={B,E,L,P}]{\om}}
\app{\lem[wit={ceteri}]{dṛṣṭiṃ}
  \rdg[wit={B}]{dṛṣṭi}
  \rdg[wit={L}]{\om}}
\app{\lem[wit={ceteri}]{kṛtvā}
  \rdg[wit={B}]{kṛtvā ākāśamadhye}
  \rdg[wit={L}]{ākāśamadhye}}
dhyānakāraṇāt
\app{\lem[wit={ceteri}]{sūryaṃ}
  \rdg[wit={N2,U2}]{sūrya}}
vinā
\app{\lem[wit={P}]{sūryasaṃbaṃdhīnīṃ}
  \rdg[wit={ceteri}]{sūryasaṃbaṃdhinī}}
\app{\lem[wit={P}]{sahasrakiraṇāvalīṃ}
  \rdg[wit={U2}]{sahasrakiraṇāvaliṃ}
  \rdg[wit={B,L}]{sahasrakiraṇāvali}
  \rdg[wit={E}]{sahasrakiraṇapaṅktīḥ}
  \rdg[wit={N1,U1}]{sahasrāṇy api kīraṇāṇi}
  \rdg[wit={D,N2}]{sahasrāṇapi kiraṇāṇi}}
\app{\lem[wit={E,L,U2}]{paśyati}
  \rdg[wit={B,D,N1}]{paśyatī}
  \rdg[wit={N2}]{paśyate}
  \rdg[wit={P}]{pati}
  \rdg[wit={U1}]{paśyaṃti}}/
%-----------------------------
%atha vā śivopari vṛddhaṃ  saptadaśāṃgulapramāṇaṃ  tejaḥpuṃjalakṣyaṃ     karttavyam/ \E
%\om \P
%atha vā śiroparir urdhvaṃ saptadaśāṃgulapramāṇaṃ  tejaḥpūṃjaṃ lakṣaṇaṃ  kartavyaṃ/ \B
%atha vā śiropari ūrdhva---saptadaśāṃgulapramāṇaṃ  tejaḥpūṃjaṃ lakṣaṃ    kartavyaṃ/ \L
%atha kā śiropari ūrddhvaṃ saptadaśāṃgulapramāṇaṃ  tejā  puṃjalakṣaṃ      karttavyaṃ/ \N1
%atha vā śiropari ūrddhvaṃ saptadaśāṃgulapramāṇaṃ  tejā  puṃjalakṣyaṃ     karttavyaṃ// \D
%atha vā śiropari ūrddhvaṃ saptadaśāṃgulaṃ parāṇaṃ tejaḥpuṃjalakṣaṇaṃ    kartavyaṃ// \N2
%atha vā śiropari ūrddhaṃ  saptadaśāṃgulapramāṇaṃ  tejaḥpuṃjakaṃ lakṣyaṃ kartavyaṃ \U1 %%%278.jpg
%atha vā śiropari ūrddhaṃ  saptadaśāṃgulapramāṇa---tejaḥpuṃjaṃ lakṣyaṃ   karttavyaṃ// \U2
%-----------------------------
%Or the mass of light situated seventeen fingers wide distance above the head shall be made the fixation object. 
%-----------------------------
\note[type=source, labelb=156, lem={\textbf{Re}}]{PT\textsuperscript{qcr \cdot YSV} (Ed. p. 837): ūrddhvaṃ saptadaśāṅgulyaṃ pramāṇaṃ tejasā prabham |}
\app{\lem[wit={ceteri}]{atha vā}
  \rdg[wit={N1}]{atha kā}
  \rdg[wit={P}]{\om}}
\app{\lem[type=emendation, resp=egoscr, alt={śiropary}]{śiropar\skp{y-ū}}
  \rdg[wit={ceteri}]{śiropari}
  \rdg[wit={E}]{śivopari}
  \rdg[wit={B}]{śiroparir}
  \rdg[wit={P}]{\om}
}\app{\lem[wit={ceteri}, alt={ūrddhvaṃ}]{\skm{y-ū}rddhvaṃ}
  \rdg[wit={L}]{ūrdhva°}
  \rdg[wit={B}]{urdhvam}
  \rdg[wit={U1,U2}]{ūrddhaṃ}
  \rdg[wit={E}]{vṛddhaṃ}
  \rdg[wit={P}]{\om}}
\app{\lem[wit={ceteri}]{saptadaśāṃgulapramāṇaṃ}
  \rdg[wit={N2}]{saptadaśāṃgulaṃ parāṇaṃ}
  \rdg[wit={U2}]{saptadaśāṃgulapramāṇa°}
  \rdg[wit={P}]{\om}}
\app{\lem[wit={E}]{tejaḥpuṃjalakṣyaṃ}
  \rdg[wit={P}]{tejaḥpūṃjaṃ lakṣaṇaṃ}
  \rdg[wit={L}]{tejaḥpūṃjaṃ lakṣaṃ}
  \rdg[wit={N1}]{tejā puṃjalakṣaṃ}
  \rdg[wit={D}]{tejā puṃjalakṣyaṃ}
  \rdg[wit={N2}]{tejaḥpuṃjalakṣaṇaṃ}
  \rdg[wit={U1}]{tejaḥpuṃjakaṃ lakṣyaṃ}
  \rdg[wit={U2}]{tejaḥpuṃjaṃ lakṣyaṃ}}
karttavyaṃ/
%-----------------------------
%atha vā dṛṣṭer agre tatparaṃ svarṇākāraṃ  pṛthvītattvaṃ  lakṣyaṃ kartavyam/ \E
%atha vā dṛṣṭer agne taptasvarṇavarṇakāraṃ pṛthvītatvaṃ   lakṣyaṃ \P
%atha vā dṛṣṭer agne taptasuvarṇavarṇa-----pṛthivītatvaṃ  lakṣaṃ kartavyaṃ/ \B
%atha vā dṛṣṭer agne taptasuvarṇavarṇa-----pṛthītatvaṃ    lakṣaṃ kartavyaṃ/ \L
%atha vā dṛṣṭer ag..?taptavarṇākāraṃ       pṛthvītatvaṃ   lakṣaṃ karttavyaṃ/ \N1
%atha vā dṛṣṭer agre taptavarṇākāraṃ       pṛthvītatvaṃ   lakṣaṃ karttavyaṃ// \D %%%p.10 beginning
%atha vā dṛṣṭer agre taptavarṇākāraṃ       pṛthvītatvaṃ   lakṣaṇaṃ karttavyaṃ/ \N2
%atha vā dṛṣṭer agre taptavarṇākāraṃ       pṛthvītatvaṃ   lakṣyaṃ karttavyaṃ \U1
%atha vā dṛṣṭer agre taptasvarṇavarṇākāraṃ pṛthvīṃ tatvaṃ lakṣaṃ karttavyaṃ// \U2
%-----------------------------
%Or, at the uppermost part of the [previously mentioned] focal point the earth-element appearing in the color of molten gold shall be made the target [of fixation].  
%-----------------------------
\noindent
\note[type=source, labelb=157, lem={\textbf{Re}}]{PT\textsuperscript{qcr \cdot YSV} (Ed. p. 837): ūrddhvaṃ saptadaśāṅgulyaṃ pramāṇaṃ tejasā prabham | athavā pṛthivītattvaṃ taptakāñcanasannibham | dṛṣṭiragre tu karttavyaṃ lakṣam etad yat ātmanām | uktānāṃ yasya kasyaiva ekaśaḥ karaṇaṃ priye | balīpalitahīnaḥ syādauṣadhena vinā tathā |}
\note[type=testium, labelb=157a, lem={\textbf{Ri}}]{SSP 2.28 (Ed. p. 40): atha vā dṛṣṭyā taptakāñcanasannibhāṃ bhūmiṃ lakṣayet | dṛṣṭiḥ sthirā bhavati | ity anekavidhaṃ bahirlakṣyam |}
atha vā dṛṣṭe\skp{r-a}\app{\lem[wit={ceteri}, alt={agre}]{\skm{r-a}gre}
  \rdg[wit={B,L,P}]{agne}}
\app{\lem[wit={U2}]{taptasvarṇavarṇākāraṃ}
  \rdg[wit={P}]{taptasvarṇavarṇakāraṃ}
  \rdg[wit={E}]{tatparaṃ svarṇākāraṃ}
  \rdg[wit={B,L}]{taptasuvarṇavarṇa}
  \rdg[wit={X}]{taptavarṇākāraṃ}}
\app{\lem[wit={X,E,P}]{pṛthvītattvaṃ}
  \rdg[wit={B}]{pṛthivītatvaṃ}
  \rdg[wit={L}]{pṛthītatvaṃ}
  \rdg[wit={U2}]{pṛthvīṃ tatvaṃ}}
\app{\lem[wit={E,P,U1}]{lakṣyaṃ}
  \rdg[wit={B,D,L,N1,U2}]{lakṣaṃ}
  \rdg[wit={N2}]{lakṣaṇaṃ}}
\app{\lem[wit={ceteri}]{karttavyaṃ}
  \rdg[wit={P}]{\om}}/\textsuperscript{\begin{otherlanguage}{english}\coro{[\lowroman{10}]}\end{otherlanguage}}
%-----------------------------
%uktānāṃ lakṣyāṇāṃ  madhye yasya kasyāpy ekasya lakṣyakaraṇāt     valitapalitā dūre bhavanti/ \E
%uktānāṃ lakṣaṇānāṃ madhye yasya kasyāpy ekasya lakṣyakaraṇāt     valitapalitādidūre bhavati \P
%uktānāṃ lakṣaṇaṃ   madhye yasya kasyāpi kasya  lakṣakaraṇāt//    valitaṃ palitādi dūre bhavatī/ \B
%uktānāṃ lakṣaṇaṃ   madhye yasya kasyāpi kasya  lakṣakaraṇāt//    valitaṃ palitādi dūre bhavati// \L
%uktānāṃ lakṣyaṇāṃ  madhye yasya kasyāpy ekasya lakṣasya karaṇāt  valitapalitādidūre bhavati \N1
%uktānāṃ lakṣyaṇaṃ  madhye yasya kasyāp--ekasya lakṣasya karaṇāt  valitapalitādidūre bhavati// \D
%uktānāṃ lakṣāṇā----madhye yasya lasyāpy elasya lakṣaṇasya karaṇātvalitapalitādidūre bhavati/ \N2
%uktānāṃ lakṣyaṇāṃ  madhye yasya kasyāpi kasya  lakṣyasya karaṇā  valitapalitādidūre bhavati \U1
%uktānāṃ lakṣāṃ     madhye yasya kasyāpy ekasya lakṣyakaraṇāt     valitapalitādidūre bhavaṃti// \U2
%-----------------------------
%From the execution of [the yoga of] targets onto the any of the discussed targets wrinkles and grey hair etc. are removed. 
%-----------------------------
uktānāṃ 
\app{\lem[wit={E}]{lakṣyāṇāṃ}
  \rdg[wit={U1,N1}]{lakṣyaṇāṃ}
  \rdg[wit={D}]{lakṣyaṇaṃ}
  \rdg[wit={P}]{lakṣaṇānāṃ}
  \rdg[wit={B,L}]{lakṣaṇaṃ}
  \rdg[wit={N2}]{lakṣāṇā°}
  \rdg[wit={U2}]{lakṣāṃ}}
madhye yasya
\app{\lem[wit={ceteri},alt={kasyāpy}]{kasyā\skp{py-e}}
  \rdg[wit={B,L,U1}]{kasyāpi}
  \rdg[wit={D}]{kasyāp°}
  \rdg[wit={N2}]{lasyāpy}
}\app{\lem[wit={ceteri}, alt={ekasya}]{\skm{py-e}kasya}
  \rdg[wit={B,L,U1}]{kasya}
  \rdg[wit={N2}]{elasya}}
\app{\lem[wit={ceteri},alt={lakṣya°}]{lakṣya}
  \rdg[wit={B,L}]{lakṣa°}
  \rdg[wit={D,N1}]{lakṣasya}
  \rdg[wit={N2}]{lakṣaṇasya}
  \rdg[wit={U1}]{lakṣyasya}
}\app{\lem[wit={ceteri}, alt={°karaṇāt}]{karaṇāt}
  \rdg[wit={U1}]{karaṇā}}
\app{\lem[wit={E}]{valitapalitā dūre}
  \rdg[wit={B,L}]{valitaṃ palitādi dūre}
  \rdg[wit={ceteri}]{valitapalitādidūre}}
\app{\lem[wit={E,U2}]{bhavanti}
  \rdg[wit={B}]{bhavatī}
  \rdg[wit={ceteri}]{bhavati}}/
%-----------------------------
%aṃgarogāḥ  vinauṣadhaṃ dūrī bhavanti/  samagrāḥ śatravaḥ  svapne pya mitran   nāyāṃti/     \E
%aṃgirogā   vinauṣadhaṃ dūre bhavati    samagrāḥ śatravaḥ  svapne pya mitratām ayāṃti  \P  %%%7646.jpg Z.1 
%aṃgirogādi vinauṣadhaṃ dūro bhavatī    samagrāḥ śatrave   svapne pya mitratām ayāṃti//   \B
%aṃgirogādi vinauṣadhaṃ dūro bhavati    samagrāḥ śatravo   svapne pya mitratām ayāṃtī     \L
%aṃgarogā   vinauṣadhaṃ dūre bhavaṃti/  samagrāḥ śatravaḥ  svapin eva mityaṃ   nāyāti/     \N1
%aṃgarogā   vinauṣadhaṃ dūre bhavaṃti// samagrāḥ śatravaḥ  svacan eva mityaṃ   nāyāti//     \D
%aṃgarogā   vinauṣadhaṃ dūre bhavati//  samagrā  śatravaḥ  svapin evan nityaṃ  nāyāti//    \N2
%aṃgarogā   vinauṣadhaṃ dūre bhavati    samagrāḥ śatravaḥ  svapin eva mitevaṃ  naiyati    sahasravarṣaparyaṃtam āyuṣyaṃ varddhate \U1
%aṃgarogā   vinauṣadhaṃ dūre bhavaṃti// samagra  śatravaḥ  svapne pi mitratām  āyāṃti//  sahasravarṣam āyur varddhate// \U2 ā-yānti= von ā-√yā=in einen Zustand ~, in eine Lage ~, in ein Verhältniss kommen, ~ gerathen; theilhaftig werden, erlangen; mit Acc. 
%-----------------------------
%Diseases of the limbs are removed without medical herbs. All enemies become friends while sleeping.
%-----------------------------
\note[type=source, labelb=158, lem={\textbf{Re}}]{PT\textsuperscript{qcr \cdot YSV} (Ed. p. 837): sarvarogāṇi naśyanti mitravac ca vaśī ripuḥ | jīved varṣasahasran tu sarvalokeṣu pūjitaḥ | jihvāgre prabhaved vidyā vinā śāstrāvalokanāt |}
\app{\lem[wit={ceteri}]{aṅgarogā}
  \rdg[wit={E}]{aṃgarogāḥ}
  \rdg[wit={B,L}]{aṃgirogādi}}
vinauṣadhaṃ
\app{\lem[wit={ceteri}]{dūre}
  \rdg[wit={E}]{dūrī}
  \rdg[wit={B,L}]{dūro}}
\app{\lem[wit={D,E,N1,U2}]{bhavanti}
  \rdg[wit={P,L,N2,U1}]{bhavati}
  \rdg[wit={B}]{bhavatī}}/
\app{\lem[wit={ceteri}]{samagrāḥ}
  \rdg[wit={N2}]{samagrā}
  \rdg[wit={U2}]{samagra°}}
\app{\lem[wit={ceteri}]{svapne}
  \rdg[wit={N1,N2,U1}]{svapin}
  \rdg[wit={D}]{svacan}}
\app{\lem[wit={U2}]{'pi}
  \rdg[wit={B,E,L,P}]{pya}
  \rdg[wit={D,N1,U1}]{eva}
  \rdg[wit={N2}]{evan}}
\app{\lem[wit={B,L,P,U2}]{mitratām}
  \rdg[wit={E}]{mitran}
  \rdg[wit={D,N1}]{mityaṃ}
  \rdg[wit={N2}]{nityaṃ}
  \rdg[wit={U1}]{mitevaṃ}}
\app{\lem[wit={P,B}]{ayāṃti}
  \rdg[wit={L}]{ayāṃtī}
  \rdg[wit={N2}]{āyāṃti}
  \rdg[wit={E}]{nāyāṃti}
  \rdg[wit={D,N1,N2}]{nāyāti}
  \rdg[wit={U1}]{naiyati}}/
%-----------------------------STEMMAPOINT!!!!------------------------------------
% sahasravarṣam āyur bhavati/ \E
% sahasravarṣam āyur varddhate \P  %%%7646.jpg Z.1 
% sahasravarṣam āyur vardhate// \B
% sahasravarṣam āyur vardhate// \L
% sahasravarṣaparyaṃtam āyuṣaṃ varddhate/ \N1
% sahasravarṣaparyaṃtam āyuṣaṃ varddhate// \D
% sahasravarṣaparyaṃtam āyuṣaṃ vardhate// \N2
% sahasravarṣaparyaṃtam āyuṣyaṃ varddhate \U1
% sahasravarṣam āyur varddhate// \U2 ā-yānti= von ā-√yā=in einen Zustand ~, in eine Lage ~, in ein Verhältniss kommen, ~ gerathen; theilhaftig werden, erlangen; mit Acc. 
%-----------------------------
% The lifespan increases up to 1000 years. 
%-----------------------------
\app{\lem[wit={X},alt={sahasravarṣaparyaṃtam}]{sahasravarṣaparyaṃta\skp{m-ā}}
  \rdg[wit={Y}]{sahasravarṣam}
}\app{\lem[wit={D,N1,N2},alt={āyuṣaṃ}]{\skm{m-ā}yuṣaṃ}
  \rdg[wit={U1}]{āyuṣyaṃ}
  \rdg[wit={B,E,P,L,U2}]{āyur}}
vardhate/
%-----------------------------
%apaṭhitaṃ śāstraṃ jihvāgreṇoccarati/  etādṛśaṃ phalaṃ bahutaraṃ bhavati// \E
%apaṭhitaṃ śāstraṃ jihvāgreṇoccarati   etādṛśaṃ mitratāmāyāṃti sahasravarṣam āyur varddhate apaṭhitaṃ śāstraṃ jihvāgreṇoccarati etādṛśaṃ phalaṃ bahutaraṃ bhavati \P
%apaṭhitaṃ śāstraṃ jihvāgreṇoccaratī/  etādṛśaṃ phalaṃ bahutaraṃ bhavatī// \B
%apaṭhitaṃ śāstraṃ jihvāgreṇoccarati   etādṛśaṃ phalaṃ bahutaraṃ bhavaṃtī// \L
%apaṭhitaṃ śāstraṃ jihvāgreṇoccarate// etādṛśaṃ bahutaraṃ phalaṃ bhavati// \N1
%apaṭhitaṃ śāstraṃ jihvāgreṇoccarate// etādṛśaṃ bahutaraṃ phalaṃ bhavati// \D
%apathitaṃ śāstraṃ jihvāgreṇoccarate// etādṛśaṃ bahutaraṃ phalaṃ bhavati// \N2
%apathitaṃ śāstraṃ jihvāgreṇoccarate   etādṛśyaṃ bahutaraṃ phalaṃ bhavati \U1
%apathitaṃ śāstraṃ jihvāgreṇoccarati// etādṛśaṃ phalaṃ bahutaraṃ phalaṃ bhavati// \U2
%-----------------------------
%The non-recited scripture is uttered by the tip of the tongue.  An abundance of such results arise. Such [practice] yields numerous fruits.
%-----------------------------
\app{\lem[wit={ceteri}]{apaṭhitaṃ}
  \rdg[wit={N2,U1,U2}]{apathitaṃ}}
śāstraṃ
jihvāgreṇocca\app{\lem[wit={X}, alt={°rate}]{rate}
  \rdg[wit={B,E,L,U2}]{°rati}
  \rdg[wit={B}]{°ratī}}/\textsuperscript{\begin{otherlanguage}{english}\coro{[\lowroman{15}]}\end{otherlanguage}}
\app{\lem[wit={ceteri}]{etādṛśaṃ}
  \rdg[wit={U1}]{etādṛśyaṃ}}
\app{\lem[wit={X}]{bahutaraṃ phalaṃ}
  \rdg[wit={Y}]{phalaṃ bahutaraṃ}}\dd{}
\note[type=philcomm, labelb=159, lem={\uproman{24}.\textsuperscript{\lowroman{16}}}]{Witness \getsiglum{P} includes a dittography of the previous lines after \uproman{24}.\textsuperscript{\lowroman{16}} and reads: \textit{etādṛśaṃ mitratāmāyāṃti sahasravarṣam āyur varddhate apaṭhitaṃ śāstraṃ jihvāgreṇoccarati etādṛśaṃ phalaṃ bahutaraṃ bhavati |}.}
    \end{prose}
  \end{edition}
  \begin{translation}
    \ekddiv{type=trans}
    \begin{tlate}
      \blfootnote{
  \begin{quote} \ldots Four fingers above the tip of the nose; contemplate the blue-colored space-element.\\
  (30)  Six fingers from the tip of the nose visualize the smoke-coloured air-element.\\
  Eight fingers in front of the nose visualize the red-coloured fire-element.\\
  (31)  Ten fingers from the tip of the nose visualize the white-colored water element.\\
  Twelve fingers in front of the nose visualize the earth-element with a yellow color.\\
  (32) There are many external target, but only a few can attain the ultimate goal.\\
  If the true guru shows mercy at any time, they reveal the secret within.
\end{quote}}
\noindent
After having fixed the gaze on the space-element or above the space-element, due to the execution of meditation [on either target], he sees the sun without the group of thousand rays related to the sun. Or, the mass of light situated seventeen-finger wide distance above the head shall be made the fixation object. Or, at the uppermost part of the [previously mentioned] focal point, the earth-element appearing in the colour of molten gold shall be made the target [of fixation].\textsuperscript{\coro{[\lowroman{10}]}} From the execution of [the yoga of] targets onto any of the discussed targets, wrinkles, grey hair, etc., are removed. Diseases of the limbs are removed without medical herbs. All enemies become friends while sleeping. The lifespan increases up to 1000 years. The non-recited scripture is uttered by the tip of the tongue.\textsuperscript{\coro{[\lowroman{15}]}} Such [practice] yields numerous fruits.
    \end{tlate}
  \end{translation}
\end{alignment}
%%%%%%%%%%%%%%%%%%%%%%%%%%%%%%%%%%%%%%%%%%
%%%%%%%%%%%%%%%%%%%%%%%%%%%%%%%%%%%%%%%%%% 
%%%%%%%%PAGEBREAK%%%%%%%PAGEBREAK%%%%%%%%%
%%%%%%%%%%%%%%%%%%%%%%%%%%%%%%%%%%%%%%%%%% 
%%%%%%%%%%%%%%%%PAGEBREAK%%%%%%%%%%%%%%%%%
%%%%%%%%%%%%%%%%%%%%%%%%%%%%%%%%%%%%%%%%%% 
%%%%%%%%PAGEBREAK%%%%%%%PAGEBREAK%%%%%%%%%
%%%%%%%%%%%%%%%%%%%%%%%%%%%%%%%%%%%%%%%%%% 
%%%%%%%%%%%%%%%%%%%%%%%%%%%%%%%%%%%%%%%%%% 
%%%%%%%%%%%%%%%%%%%%%%%%%%%%%%%%%%%%%%%%%% 
%%%%%%%%%%%%%%%%%%%%%%%%%%%%%%%%%%%%%%%%%% 
%%%%%%%%PAGEBREAK%%%%%%%PAGEBREAK%%%%%%%%%
%%%%%%%%%%%%%%%%%%%%%%%%%%%%%%%%%%%%%%%%%% 
%%%%%%%%%%%%%%%%PAGEBREAK%%%%%%%%%%%%%%%%%
%%%%%%%%%%%%%%%%%%%%%%%%%%%%%%%%%%%%%%%%%% 
%%%%%%%%PAGEBREAK%%%%%%%PAGEBREAK%%%%%%%%%
%%%%%%%%%%%%%%%%%%%%%%%%%%%%%%%%%%%%%%%%%% 
%%%%%%%%%%%%%%%%%%%%%%%%%%%%%%%%%%%%%%%%%% 
%%%%%%%%%%%%%%%%%%%%%%%%%%%%%%%%%%%%%%%%%% 
%%%%%%%%%%%%%%%%%%%%%%%%%%%%%%%%%%%%%%%%%% 
%%%%%%%%PAGEBREAK%%%%%%%PAGEBREAK%%%%%%%%%
%%%%%%%%%%%%%%%%%%%%%%%%%%%%%%%%%%%%%%%%%% 
%%%%%%%%%%%%%%%%PAGEBREAK%%%%%%%%%%%%%%%%%
%%%%%%%%%%%%%%%%%%%%%%%%%%%%%%%%%%%%%%%%%% 
%%%%%%%%PAGEBREAK%%%%%%%PAGEBREAK%%%%%%%%%
%%%%%%%%%%%%%%%%%%%%%%%%%%%%%%%%%%%%%%%%%% 
%%%%%%%%%%%%%%%%%%%%%%%%%%%%%%%%%%%%%%%%%% 
\begin{alignment}[
  texts=edition[class="edition"];
  translation[class="translation"],
  ]
  \begin{edition}
    \ekddiv{type=ed}
    \centerline{\textrm{\small{[\uproman{25}.\textsuperscript{\coro{\lowroman{1}-\lowroman{9}}}Antaralakṣya]}}}
          \bigskip
          \begin{prose}
            \noindent
%-----------------------------
%idānīm anyataraṃ lakṣyaṃ kathyate/ \E
%idānīṃ aṃtaraṃ lakṣyaṃ   kathyate \P
%idānīṃ antaralakṣaṃ      kartavyaṃ// \B
%idānīṃ aṃtaralakṣaṃ      kartavyaṃ// \L
%idānīṃ antaralakṣyakaṃ   kathyate// \N1
%idānīṃ antaralakṣyaṃ     kathyate// \D
%idānīṃ aṇtaralakṣyaṇaṃ   kathyate// \N2
%idānīṃ aṇtaralakṣyaṇaṃ   kathyate \U1
%idānīm ataraṃ lakṣyaṃ    kathyate// \U2
%-----------------------------
%Now the inner target is explained. 
%-----------------------------
\app{\lem[wit={E,U2},alt={idānīm}]{idānī\skp{m-a}}
  \rdg[wit={ceteri}]{idānīṃ}
}\app{\lem[wit={D}, alt={antaralakṣyaṃ}]{\skm{m-a}ntaralakṣyaṃ}
  \rdg[wit={E}]{anyataraṃ lakṣyaṃ}
  \rdg[wit={P}]{aṃtaraṃ lakṣyaṃ}
  \rdg[wit={B,L}]{antaralakṣaṃ}
  \rdg[wit={N1}]{antaralakṣyakaṃ}
  \rdg[wit={N2,U1}]{aṇtaralakṣyaṇaṃ}
  \rdg[wit={U2}]{ataraṃ lakṣyaṃ}}
\app{\lem[wit={ceteri}]{kathyate}
  \rdg[wit={B,L}]{kartavyaṃ}}/    
%-----------------------------
%mūlakandasthāne brahmadaṇḍotpannā nāḍī śvetavarṇā   brahmadaṇḍaparyantam   ekā brahmanāḍī varttate/ \E
%mūlakaṃdasthāne brahmānaṃḍād utpannā   śvetavarṇā   brahmaraṃdhraparyaṃtaṃ ekā brahmanāḍī varttate   \P
%mūlakaṃ sthāne  brahmānaṃḍād utpannā   śvetāvarṇā   brahmaraṃdhraparyaṃtaṃ ekā nāḍī       vartate/     \B
%mūlakaṃdasthāne brahmānaṃdād utpannā   śvetāvarṇā   brahmaraṃdhraparyaṃtaṃ ekanāḍī        vartate/     \L
%mūlakaṃdasthāne brahmadaṃḍa ityannā    śvetavarṇā   brahmaraṃdhraparyaṃtaṃ ekā brahmanāḍī varttate/ \N1
%mūlakaṃdasthāne brahmadaṃḍād utpannā   śvetavarṇā// brahmaraṃdhraparyaṃtaṃ ekā brahmanāḍī varttate// \D
%mūlakaṃdasthāne brahmadaṇḍad ūtpannā   śvetavarṇā   brahmaraṃdhraparyaṃtaṃ ekā brahmanāḍī varttate/ \N2
%mūlakaṃdasthāne brahmadaṇād ūtpannaḥ   śvetavarṇāṃ  brahmaraṃdhraparyaṃtaṃ ekā brahmanāḍī varttate \U1
%mūlakaṃdasthāne brahmadaṇḍād utpannā   śvetavarṇā   brahmaraṃdhraparyaṃtaṃ ekā brahmanāḍī varttate// \U2
%-----------------------------
%Starting from the location of the root-bulp (\textit{mūlakanda}) originating from the staff of Brahma, being white in colour, extending up to the aperture of Brahma [at the top the head] exists the single Brahma-channel.
%-----------------------------
\note[type=source, labelb=160, lem={\textbf{Re}}]{PT\textsuperscript{qcr \cdot YSV} (Ed. p. 838): mūlakandotthatalato brahmanāḍīsamudbhavā | śvetavarṇā brahmarandhraparyantam eva tiṣṭhati | eṣā tu brahmarandhrākhyā tanmadhye varttate parā |}
\note[type=testium, labelb=160a, lem={\textbf{Ri}}]{SSP 2.26 (Ed. pp. 37-38): tatra tāvad antarlakṣyaṃ kathyate | mūlakandād daṇḍalagnāṃ brahmanāḍīṃ śvetavarṇāṃ brahmarandhraparyantaṃ gatāṃ saṃsmaret | tanmadhye kamalatantunibhāṃ vidyutkoṭiprabhām ūrdhvagāminīṃ tāṃ mūrtiṃ manasā lakṣayet | sarvasiddhipradā bhavati|}
\app{\lem[wit={ceteri}]{mūlakandasthāne}
  \rdg[wit={P}]{mūlakaṃ sthāne}}
\app{\lem[wit={ceteri}]{brahmadaṇḍād\skp{-}utpannā}
  \rdg[wit={E}]{brahmadaṇḍotpannā nāḍī}
  \rdg[wit={N1}]{brahmadaṃḍa ityannā}
  \rdg[wit={N2}]{brahmadaṇḍad ūtpannā}
  \rdg[wit={U1}]{brahmadaṇād ūtpannaḥ}}
 śvetavarṇā
\app{\lem[wit={ceteri}]{brahmarandhraparyaṃtaṃ}
  \rdg[wit={E}]{brahmadaṇḍaparyantam}} 
\app{\lem[wit={ceteri}]{ekā brahmanāḍī}
  \rdg[wit={B}]{ekā nāḍī}
  \rdg[wit={L}]{ekanāḍī}}
vartate/
%-----------------------------
%brahmanāḍīmadhye kamalatantusamānākārā koṭisūryavidyutsamaprabhā   ūrdhvaṃ calati/  \E
%brahmanāḍīmadhye kamalataṃ samānākārā  koṭisūryavidyutsamaprabhā   ūrdhvaṃ calati   \P
%brahmanāḍīmadhye kamalataṃtusamānākārā koṭisūryavidyutsabhāprabhā  ūrdhvaṃ calati/  \B
%brahmanāḍīmadhye kamalataṃtusamānākārā koṭisūryavidyutsabhāprabhā  ūrdhvaṃ calati/  \L
%brahmanāḍīmadhye kamalatantusamānākārā koṭisūryavidyutsamaprabhā   ūrdhvaṃ calati/  \N1
%brahmanāḍīmadhye kamalataṃtusamānākārā koṭisūryavidyutsamaprabhā   ūrdhvaṃ calati// \D
%\om                                                                                 \N2
%brahmanāḍīmadhye kamalatantusamānākārā koṭisūryavidyutsamaprabhā    rdhvaṃ ccalati  \U1
%brahmanāḍīmadhye kamalataṃtusamānākārā koṭisūryavidyutsamaprabhā// urdhvaṃ calati   \U2 %%%412.jpg 
%-----------------------------
%Within the Brahma channel [exists something] resembling the shape of a stalk of a lotus-flower shining like 10 million suns [which] goes upwards. 
%-----------------------------
\note[type=source, labelb=161, lem={\textbf{Re}}]{PT\textsuperscript{qcr \cdot YSV} (Ed. p. 838): padmatantusamākārā koṭisūryataḍitprabhā | calaty ūrddhaṃ mahāmūrttir asya dhyānād bhavec chivaḥ | aṇimādy aṣṭasiddhis tu samagreṇa prasīdati |}
\note[type=philcomm, labelb=161a, lem={\uproman{25}.\textsuperscript{\lowroman{3}}}]{Sentence \om in \getsiglum{N2}.}
brahmanāḍīmadhye
\app{\lem[wit={ceteri}]{kamalatantusamānākārā}
  \rdg[wit={P}]{kamalataṃ samānākārā}}
koṭisūryavidyutsa\app{\lem[wit={ceteri},alt={°maprabhā}]{maprabhā}
  \rdg[wit={B,L}]{°bhāprabhā}}
\app{\lem[wit={ceteri}]{ūrdhvaṃ}
  \rdg[wit={U1}]{°rdhvaṃ}
  \rdg[wit={U2}]{urdhvaṃ}}
\app{\lem[wit={ceteri}]{calati}}/
%-----------------------------
%etādṛśy ekā mūrttir varttate/  tan    mūrter dhyānakāraṇāt      aṣṭamahāsiddhayo  'ṇimādayas   tasya                                                                                puruṣasya samīpam āgatya tiṣṭhanti// \E
%etādṛśy ekā mūrttir vartate    tasyā  mūrter dhyānakaraṇāt      aṣṭamahāsiddhayo   ṇimādyāḥ    aṇimā-mahimā-laghīmā-girimā-dure dīya vā            dure  stutvā parakāyapraveśītā   puruṣasya samīm   āgatya tiṣṭhaṃti \P
%etādṛśy ekā mūrttir varttate/  tasyā  mūrte  dhyānakaraṇāt//    aṣṭamahāsiddhayo// aṇimādyāḥ// aṇimā-mahimā-laghimā-girimā-dure vā yadi vā yadi vā dure  śrutvā parakāyāpraveśitā// puruṣasya samīpem āgatya tiṣṭhati// \B
%etādṛśy ekā mūrttir varttate/  tasyā  mūrter dhyānakaraṇāt//    aṣṭamahāsiddhayo   aṇimādyāḥ// aṇimā-mahimā-laghimā-garimā-dure vā yadi         vā ddure śrutvā parakāyāpraveśitā   puruṣasya samīpam āgatya tiṣṭhati// \L
%etādṛśī ekā mūrttir varttate/  tasyāḥ mūrtter dhyānakāraṇāt/    aṇimādīsiddhiḥ                                                                                                      puruṣasya samīpe? āgatya tiṣṭhanti// \N1
%etādṛśī ekā mūrttir varttate// tasyā  mūrtter dhyānakāraṇāt//   aṇimādyaṣṭasiddhiḥ                                                                                                  puruṣasya samīpe  āgatya tiṣṭhati// \D
%\om                            tasyā  mūrtter dhyānakaraṇāc                                                                                                                                                            \N2
%                                                                aṇimādyaṣṭasiddhiḥ                                                                                                  puruṣasya sāmīpe  āgatya tiṣṭhati \U1
%etādṛśy ekā mūrttir varttate// tasyā   mūrter dhyānakaraṇāt//   aṣṭamahāsiddhayo aṇimādyāḥ//                                                                                        puruṣasya samīpam āgamya tiṣṭhati// \U2
%-----------------------------
%[There] exists only one such manifestation. Due to the execution of meditation on this manifestation, the eight great supernatural powers of humans beginning with "becoming as small as the smallest particle of matter" (\textit{aṇima}) etc. become established after one has entered into [the manufestation's] imminence. 
%-----------------------------
\app{\lem[wit={ceteri}]{etādṛśy\skp{-}ekā}
  \rdg[wit={N1,D}]{etādṛśī ekā}
  \rdg[wit={U1,N2}]{\om}}
\app{\lem[wit={ceteri}, alt={mūrtir}]{mūrti\skp{r-va}}
  \rdg[wit={U1,N2}]{\om}
}\app{\lem[wit={ceteri}, alt={vartate}]{\skm{r-va}rtate}
  \rdg[wit={U1,N2}]{\om}}/
\app{\lem[wit={ceteri}]{tasyā}
  \rdg[wit={N1}]{tasyāḥ}
  \rdg[wit={E}]{tan}
  \rdg[wit={U1}]{\om}}
\app{\lem[wit={ceteri}, alt={mūrter}]{mūrte\skp{r-dhyā}}
  \rdg[wit={B}]{mūrte}
  \rdg[wit={U1}]{\om}
}\app{\lem[wit={ceteri}, alt={dhyāna°}]{\skm{r-dhyā}na}
  \rdg[wit={U1}]{\om}
}\app{\lem[resp=egoscr, type=emendation, alt={°kāraṇād}]{kāraṇā\skp{d-a}}
  \rdg[wit={ceteri}]{°karaṇāt}
  \rdg[wit={N2}]{°karaṇāc°}
  \rdg[wit={U1}]{\om}
}\app{\lem[wit={Y}, alt={aṣṭamahāsiddhayo}]{\skm{d-a}ṣṭamahāsiddhayo}
  \rdg[wit={D,U1}]{aṇimādyaṣṭasiddhiḥ}
  \rdg[wit={N1}]{aṇimādīsiddhiḥ}}
\app{\lem[wit={P}]{'ṇimādyāḥ}
  \rdg[wit={E}]{'ṇimādayas tasya}
  \rdg[wit={B,L,U2}]{aṇimādyāḥ}
  \rdg[wit={X}]{\om}}
\note[type=philcomm, labelb=162, lem={\uproman{25}.\textsuperscript{\lowroman{4}}}]{Witnesses \getsiglum{P}, \getsiglum{B} and \getsiglum{L} add a incomplete list of eight supernatural powers right after \uproman{25}.\textsuperscript{\lowroman{4}}: \textit{aṇimāmahimālaghimāgarimādure vā yadi vā dure śrutvā parakāyāpraveśitā} | Since the list is incomplete and corrupt and stemmatically most probable a later addition, it is not included within the edition's text.}
\note[type=philcomm, labelb=162a, lem={\uproman{25}.\textsuperscript{\lowroman{5}-\lowroman{6}}}]{Sentences \om in \getsiglum{N2}.}
puruṣasya
\app{\lem[wit={N1,D}]{samīpe}
  \rdg[wit={U1}]{sāmīpe}
  \rdg[wit={B}]{samīpem}
  \rdg[wit={E,L,U2}]{samīpam}
  \rdg[wit={P}]{samīm}}
\app{\lem[wit={ceteri}]{āgatya}
  \rdg[wit={U2}]{āgamya}}
\app{\lem[wit={E,P,N1}]{tiṣṭhanti}
  \rdg[wit={ceteri}]{tiṣṭhati}}/\textsuperscript{\begin{otherlanguage}{english}\coro{[\lowroman{5}]}\end{otherlanguage}}
%[Aṇima (the ability to reduce size to the size of the smallest particle), Mahimā (the ability to expand one's body to an infinitely large size), Laghimā (the ability to become weightless or lighter than air), Garimā (the ability to become heavy or dense), Dūraśravaṇa (Hearing things far away), Dūradarśanam (Seeing things far away) Parakāya praveśitā: Entering the bodies of others.
\end{prose}
\end{edition}
  \begin{translation}
    \ekddiv{type=trans}
    \begin{tlate}
    \centerline{\textrm{\small{[\uproman{25}.\textsuperscript{\coro{\lowroman{1}-\lowroman{9}}}Antaralakṣya]}}}
    \bigskip
    \noindent
    Now the inner target is explained. Starting from the location of the root-bulb (\textit{mūlakanda})\footnote{There exist diverging concepts about the location of the \textit{kanda} in yogic literature predating Rāmacandra. In the \citetitle{yajnavalkya} 4.16-27 (Ed. p. 33), it is located in the centre of the belly, nine finger widths below the waist:
      \begin{quote}
        kandasthānāṃ manuṣyāṇāṃ dehamadhyān navāṅgulam |\\
        caturaṅgulam utsedham āyāmaś ca tathāvidhaḥ ||16||\\
        aṇḍākṛtivad ākāraṃ bhūṣitaṃ tattvagādhibhiḥ |\\
        catuṣpadāṃ tiraścāṃ ca dvijānāṃ tundamadhyame ||17||
        \end{quote}
``The location of the bulb (\textit{kanda}) of humans is nine finger widths from the waist. [It is] four finger widths in height and has the same [measure in] length. It appears like an egg, and [it is] ornamented with the five elements of creation and sustenance. In quadrupeds, birds and other beings, it is in the centre of the belly.''\\
 \indent In the \citetitle{hathapradipika2024} 3.64cd (gulphadeśasamīpe ca kandaṃ tatra prapīḍayet), the \textit{kanda} is said to be located near the region of the ankles. The \emph{Gorakṣaśataka}, the source for this verse in the \citetitle{hathapradipika2024} mentions pressing the \emph{kanda} with the feet, which could imply that the \emph{kanda} is in the genital region (except one assumes the very challenging posture like \textit{kandapīḍāsana}, cf. \citetitle{encyclopediaasana} Ed. pp. 143-144). Sundaradeva, in \citetitle{yuktabhavadeva} 7.224 and Bhavadevamiśra in \citetitle{hathacandrika} (ms. no. 2244, f. 36r) argue that the \emph{kanda} is near the navel.} originating from the staff of Brahma\footnote{The term \textit{brahmadaṇḍa} in this context refers to the spinal column, which is usually identified with Mt. Meru in tantric texts, cf. \citeauthor[1988: 360]{stupa}.}, being white, extending up to the aperture of Brahma [at the top of the head] exists the single Brahma-channel\footnote{The term \textit{brahmanāḍī} is used as a synonym for the \textit{suṣūmnā}, cf., e.g. \citetitle{hathapradipika2024} 2.67, \citetitle{mallinson2012sataka} 47, \citetitle{yogakundalinyopanisad} 37c-38b, \citetitle{yogacintamani} (Ed. pp. 46, 112, 116, 140, 191), \citetitle{hatharatnavali} 2.8, 2.25, 2.65, 2.69.} Within the Brahma channel [exists something] resembling the shape of a stalk of a lotus flower shining like 10 million suns [which] goes upwards. [There] exists only one such manifestation. Due to the execution of meditation on this manifestation\footnote{A predecessor of this type of meditation is found in \citetitle{bäumer2013} 35 (madhyanāḍī madhyasaṃsthā bisasūtrābharūpayā | dhyātāntarvyomayā devyā tayā devaḥ prakāśate |) ``The central channel, which is situated within the centre of the body [is endowed] with a form resembling the stalk of a lotus. By meditation upon it's the empty inner space god becomes revealed by the goddess [of the middle].''}, the eight supernatural powers of humans beginning with ``becoming as small as the smallest particle of matter'' (\textit{aṇima}) etc.\footnote{For an extensive discussion about the yogic supernatural powers (\textit{siddhi}s), see \citeauthor{yogapowers2017} (2017).} become established after entering [the manifestation's] imminence.\textsuperscript{\coro{[\lowroman{5}]}}
    \end{tlate}
  \end{translation}
\end{alignment}
\ekdpb*{}
%%%%%%%%%%%%%%%%%%%%%%%%%%%%%%%%%%%%%%%%%%
%%%%%%%%%%%%%%%%%%%%%%%%%%%%%%%%%%%%%%%%%% 
%%%%%%%%PAGEBREAK%%%%%%%PAGEBREAK%%%%%%%%%
%%%%%%%%%%%%%%%%%%%%%%%%%%%%%%%%%%%%%%%%%% 
%%%%%%%%%%%%%%%%PAGEBREAK%%%%%%%%%%%%%%%%%
%%%%%%%%%%%%%%%%%%%%%%%%%%%%%%%%%%%%%%%%%% 
%%%%%%%%PAGEBREAK%%%%%%%PAGEBREAK%%%%%%%%%
%%%%%%%%%%%%%%%%%%%%%%%%%%%%%%%%%%%%%%%%%% 
%%%%%%%%%%%%%%%%%%%%%%%%%%%%%%%%%%%%%%%%%% 
%%%%%%%%%%%%%%%%%%%%%%%%%%%%%%%%%%%%%%%%%% 
%%%%%%%%%%%%%%%%%%%%%%%%%%%%%%%%%%%%%%%%%% 
%%%%%%%%PAGEBREAK%%%%%%%PAGEBREAK%%%%%%%%%
%%%%%%%%%%%%%%%%%%%%%%%%%%%%%%%%%%%%%%%%%% 
%%%%%%%%%%%%%%%%PAGEBREAK%%%%%%%%%%%%%%%%%
%%%%%%%%%%%%%%%%%%%%%%%%%%%%%%%%%%%%%%%%%% 
%%%%%%%%PAGEBREAK%%%%%%%PAGEBREAK%%%%%%%%%
%%%%%%%%%%%%%%%%%%%%%%%%%%%%%%%%%%%%%%%%%% 
%%%%%%%%%%%%%%%%%%%%%%%%%%%%%%%%%%%%%%%%%% 
%%%%%%%%%%%%%%%%%%%%%%%%%%%%%%%%%%%%%%%%%% 
%%%%%%%%%%%%%%%%%%%%%%%%%%%%%%%%%%%%%%%%%% 
%%%%%%%%PAGEBREAK%%%%%%%PAGEBREAK%%%%%%%%%
%%%%%%%%%%%%%%%%%%%%%%%%%%%%%%%%%%%%%%%%%% 
%%%%%%%%%%%%%%%%PAGEBREAK%%%%%%%%%%%%%%%%%
%%%%%%%%%%%%%%%%%%%%%%%%%%%%%%%%%%%%%%%%%% 
%%%%%%%%PAGEBREAK%%%%%%%PAGEBREAK%%%%%%%%%
%%%%%%%%%%%%%%%%%%%%%%%%%%%%%%%%%%%%%%%%%% 
%%%%%%%%%%%%%%%%%%%%%%%%%%%%%%%%%%%%%%%%%% 
\begin{alignment}[
  texts=edition[class="edition"];
  translation[class="translation"],
  ]
  \begin{edition}
    \ekddiv{type=ed}
    \begin{prose}
      \noindent
%-----------------------------
%atha vā lalāṭopary ākāśamadhye śuklasadṛśasya tejaso dhyānakāraṇāt       śarīrasambandhinaḥ  kuṣṭhādayo rogā  naśyanti/    āyur vṛddhir bhavati/  \E
%atha vā lalāṭopari ākāśamadhye śuklasadṛśasya tejaso dhyānakāraṇāt       śarīrasaṃbaṃdhinaḥ  kuṣṭhādayo rogāḥ naśyaṃtī     āyur vṛddhir bhavati   \P  %%%7647.jpg
%atha vā lalāṭopari ākāśamadhye śuklasadṛśasya tejaso dhyānakāraṇāt//     charīrasambandhinaḥ kuṣṭhādayo rogā  naśyaṃtī//   āyur vṛddhir bhavatī   \B 
%atha vā lalāṭopari ākāśamadhye śuklasadṛśasya tejaso dhyānakāraṇāt       charīrasambandhinaḥ kuṣṭhādayo rogā  naśyaṃti//   āyur vṛddhir bhavati// \L
%atha vā lalāṭopari ākāśamadhye śuklasadṛśasya tejaso dhyānakāraṇāt       śarīrasambandhī     kuṣṭhādayo rogāḥ naśyaṃti/    āyur vṛddhir bhavati/  \N1
%atha vā lalāṭopari ākāśamadhye śuklasadṛśasya tejaso dhyānakāraṇāt       śarīrasaṃbaṃdhī     kuṣṭādayo  rogāḥ naśyaṃti//   āyur vṛddhir bhavati//  \D
%                                                                         charīrasaṃbaṃdhi----kuṣṭadayo  rogāḥ naśyaṃti     āyur vṛddi   bhavati/  \N2
%atha vā lalāṭoparī ākāśamadhye śuklasadṛśasya tejo   dhyānakāraṇāt       śarīrasambaṃdhī     kuṣṭhādayo rogā  naśyaṃti     āyur vṛddhir bhavati   \U1 %%%279.jpg
%atha vā lalāṭoparī ākāśamadhye śuklasadṛśasya tejaso dhyānakāraṇāt//     śarīrasambaṃdhinaḥ  kuṣṭhādayo rogā  naśyaṃti//   āyur vṛddhir bhavati//  \U2
%-----------------------------
%Or from the execution of meditation onto the bright light at the centre within the space above the forehead diseases related to the body beginning with leprosy vanish. Lifeforce increases.   
%-----------------------------
\note[type=source, labelb=163, lem={\textbf{Re}}]{PT\textsuperscript{qcr \cdot YSV} (Ed. p. 838): lalāṭopari vā dhyātvā candraṃ vā jyotir īśvaram | nāśayet kuṣṭharogādīn mahāyuṣmān śivaḥ paraḥ |}
\note[type=testium, labelb=163a, lem={\textbf{Ri}}]{SSP 2.27 (Ed. p. 38): atha vā lalāṭordhve kollāṭamaṇḍape sphuratkārākāraṃ lakṣayet|}
atha vā
\app{\lem[wit={E}, alt={lalāṭopary}]{lalāṭopa\skp{ry-ā}}
  \rdg[wit={ceteri}]{lalāṭopari}
}\skm{ry-ā}kāśamadhye
śuklasadṛśasya tejaso
\app{\lem[resp=egoscr, type=emendation, alt={dhyānakaraṇāc}]{dhyānakāraṇā\skp{c-cha}}
  \rdg[wit={ceteri}]{dhyānakāraṇāt}
}\app{\lem[wit={B,L}, alt={śarīra°}]{\skm{c-cha}rīra}
  \rdg[wit={ceteri}]{śarīra°}
}\app{\lem[wit={X}, alt={°sambandhinaḥ}]{sambandhinaḥ}
    \rdg[wit={D,N1,U1}]{°sambandhī}}
\app{\lem[wit={ceteri}]{kuṣṭhādayo}
  \rdg[wit={D,N2}]{kuṣṭādayo}}
\app{\lem[wit={ceteri}]{rogā}
  \rdg[wit={D,P,N1,N2}]{rogāḥ}}
\app{\lem[wit={ceteri}]{naśyanti}
  \rdg[wit={B,P}]{naśyaṃtī}}/
āyur-vṛddhir-bhavati/ 
%-----------------------------
%          bhruvor madhye  tiriktavarṇasyātisthūlasya     tejaso dhyānakāraṇād bahulānāṃ   pārthivānāṃ tatpuruṣāṇāṃ ca vallabho bhavati/ jagadvallabho pi bhavati/      \E
%atha vā   bhruvor madhye  tiraktavarṇasyātisthūlasya     tejaso dhyānakaraṇāt   sakalānāṃ pārthivapuruṣāṇāṃ           vallabho bhavati          \P
%atha vā// bhruvor madhye 'tiraktavarṇasyātisthūlasya     tejaso dhyānaṃ karaṇāt-sakalānāṃ pārthivapuruṣāṇāṃ vallabho bhavati/         \B DSCN7163.jpg Z.1
%atha vā// bhruvor madhye 'tiraktavarṇasyātisthūlasya     tejaso dhyānakaraṇāt   sakalānāṃ pārthivapuruṣāṇāṃ vallabho bhavati/         \L
%atha vā   bhruvor madhye 'tiraktavarṇasyātisthūlasya     tejaso dhyānakaraṇāt-sakalānāṃ   pārthivapuruṣāṇāṃ vallabho bhavati/           \N1
%atha vā   bhruvor madhye 'tiraktavarṇasyātisthūlasya     tejaso dhyānakaraṇāt-sakalānā    pārthivapuruṣāṇāṃ vallabho bhavati             \D %%%p.10 verso
%atha vā   bhruvor madhye  tiraktavarṇasyātisthūlasya     tejaso dhyānakaraṇāt-sakālānāṃ   pārthivapuruṣāṇāṃ vallabho bhavati/             \N2
%atha vā   bhruvor madhye  tiraktavarṇasyātī sthalasya    tejaso dhyānakaraṇāt sakalānāṃ   pārthivapuruṣāṇāṃ vallabho bhavati/          \U1
%atha vā   bṛvor   madhye atiraktavarṇasya 'tisthūlasyaḥ  tejāso dhyānakaraṇāt sakalānāṃ   pārthivapuruṣāṇāṃ vallabho bhavati         \U2
%-----------------------------
%Or because of executing meditation on the very subtle and red coloured light in the middle of the eyebrows, he becomes one who is beloved among all royal people.    
%-----------------------------
\note[type=source, labelb=164, lem={\textbf{Re}}]{PT\textsuperscript{qcr \cdot YSV} (Ed. p. 832): bhruvor madhye 'thavā dhyātvā arkantu teja īśvaram | sthiradṛṣṭau rājapūjyo jīvanmuktaḥ śivo yathā | ātmānam ātmarūpaṃ hi dhyātvā yo niṣkriyo bhavet | nirāśīryatatattvo 'yaṃ itaro na nṛpasthitiḥ |}
\app{\lem[wit={ceteri}]{atha vā}
  \rdg[wit={E}]{\om}}
\app{\lem[wit={ceteri}, alt={bhruvor}]{bhruvo\skp{r-ma}}
  \rdg[wit={U2}]{bṛvor}
}\skm{r-ma}dhye
\app{\lem[wit={ceteri}, alt={'tirakta°}]{'tirakta}
  \rdg[wit={U2}]{atirakta°}
  \rdg[wit={E}]{tirikta°}
}\app{\lem[wit={ceteri}]{varṇasyātisthūlasya}
  \rdg[wit={U1}]{varṇasyātī sthalasya}
  \rdg[wit={U2}]{'tisthūlasyaḥ}}
tejaso
\app{\lem[wit={ceteri}, alt={dhyānakaraṇāt}]{dhyānakaraṇā\skp{t-sa}}
  \rdg[wit={B}]{dhyānaṃ karaṇāt}
  \rdg[wit={E}]{dhyānakāraṇād}
}\app{\lem[wit={ceteri},alt={sakālānāṃ}]{\skm{t-sa}kālānāṃ}
  \rdg[wit={D}]{sakalānā}
  \rdg[wit={E}]{bahulānāṃ}}
\app{\lem[wit={ceteri}]{pārthivapuruṣāṇāṃ}
  \rdg[wit={E}]{parthivānāṃ tatpuruṣāṇāṃ ca}}
vallabho\app{\lem[wit={ceteri}]{bhavati}
  \rdg[wit={E}]{bhavati | jagad vallabho pi bhavati}}/
%-----------------------------
%asya puruṣasyāvalokanena sarveṣāṃ dṛṣṭiḥ sthirā bhavati// \E
%taṃ  puruṣaṃ        pratisarveṣāṃ dṛṣṭiḥ sthirā bhavati  \P
%taṃ  puruṣaṃ        pratisarveṣāṃ dṛṣṭisthirā bhavatī// \B
%taṃ  puruṣa         pratisarveṣāṃ dṛṣṭisthirā bhavati// \L
%taṃ  puruṣaṃ dṛṣṭvā      sarveṣāṃ dṛṣṭisthirā bhavati// \N1
%taṃ  puruṣaṃ dṛṣṭvā      sarveṣāṃ dṛṣṭisthirā bhavati// \D
%taṃ  puruṣaṃ dṛṣṭā       sarveṣāṃ dṛṣṭisthirā bhavati// \N2
%taṃ  puruṣaṃ dṛṣṭvā      sarveṣāṃ dṛṣṭisthirā bhavati \U1
%taṃ  puruṣaṃ        pratisarveṣāṃ dṛṣṭisthirā bhavati// \U2
%----------------------------
%Having seen this person, everybody's gaze becomes fixed [onto the person]. 
%-----------------------------
\app{\lem[wit={ceteri}]{taṃ}
  \rdg[wit={E}]{asya}}
\app{\lem[wit={N1,D,U1}]{puruṣaṃ dṛṣṭvā}
  \rdg[wit={N2}]{puruṣaṃ dṛṣṭā}
  \rdg[wit={B,P}]{puruṣaṃ}
  \rdg[wit={L}]{puruṣa°}
  \rdg[wit={E}]{puruṣasyāvalokanena}}
\app{\lem[wit={X,E}]{sarveṣāṃ}
  \rdg[wit={ceteri}]{pratisarveṣāṃ}}
\app{\lem[wit={ceteri}]{dṛṣṭisthirā}
  \rdg[wit={E,P}]{dṛṣṭiḥ sthirā}}
\app{\lem[wit={ceteri}]{bhavati}
  \rdg[wit={B}]{bhavatī}}\dd{}\textsuperscript{\begin{otherlanguage}{english}\coro{[\lowroman{9}]}\end{otherlanguage}}
    \end{prose}
  \end{edition}
  \begin{translation}
    \ekddiv{type=trans}
    \begin{tlate}
      \noindent
      Or, from the execution of meditation onto the bright light within the space above the forehead: diseases related to the body, beginning with leprosy, vanish.\footnote{The \citetitle{ssplonavla} 2.27 (Ed. p. 38) includes three techniques of \textit{antarlakṣya} which are unparalleled in Rāmacandra's system:
        \begin{quote}
          atha vā bhramaraguhāmadhye āraktabhramarākāraṃ lakṣayet | atha vā karṇadvayaṃ tarjanībhyāṃ nirodhayet tataḥ śiromadhye dhūṃ dhūṃ kāraṃ nādaṃ śṛṇoti | atha vā cakṣurmadhye nīlajyotirūpaṃ putalyākāraṃ lakṣayed |
          \end{quote}
          ``Or, one should target the form of a very red bee within the \textit{bhrahmaraguhā}. Or, one should close both ears with the index fingers and listen to the \textit{dhūṃ dhūṃ}-sound in the head. Or, one should target the form of a doll appearing in blue light within the eyes.''} Lifeforce increases. Alternatively, because of executing meditation on the very subtle red light in the middle of the eyebrows, he becomes one who is beloved among all royal people. Having seen this person, everybody's gaze becomes fixed [onto the person].\textsuperscript{\coro{[\lowroman{9}]}}\footnote{Examining the current context, it is puzzling why Rāmacandra, as well as the unknown author of the \textit{Yogasvarodaya}, introduce the ten main channels and ten vital winds immediately after \textit{antaralakṣya} instead of \textit{madhyalakṣya}. The only one verse dedicated to \textit{antarlakṣa} in the \citetitle{sarvangayoga} 3.33 might provide a possible hint to an answer:
  \begin{quote}
    aṃtar lakṣa ju sunahuṃ prakāśā | brahma nāḍikā karahu abhyāsā |\\
    aṣṭa siddhi nava niddhi jahāṃlauṃ | ṭarahiṃ na kabahūṃ jivai jahāṃ lauṃ || 33 ||
  \end{quote}
  \begin{quote}
    "Listen about the light of the inner target. Practice locating the channels within the body. There, you will attain the eight siddhis and nine treasures and never again have to return to the mortal world."
  \end{quote} In Sundardās's discussion of Lakṣyayoga, the channels become the targets of its practice. According to Sundardās, \ldots}
    \end{tlate}
  \end{translation}
\end{alignment}
\ekdpb*{}
%%%%%%%%%%%%%%%%%%%%%%%%%%%%%%%%%%%%%%%%%%
%%%%%%%%%%%%%%%%%%%%%%%%%%%%%%%%%%%%%%%%%% 
%%%%%%%%PAGEBREAK%%%%%%%PAGEBREAK%%%%%%%%%
%%%%%%%%%%%%%%%%%%%%%%%%%%%%%%%%%%%%%%%%%% 
%%%%%%%%%%%%%%%%PAGEBREAK%%%%%%%%%%%%%%%%%
%%%%%%%%%%%%%%%%%%%%%%%%%%%%%%%%%%%%%%%%%% 
%%%%%%%%PAGEBREAK%%%%%%%PAGEBREAK%%%%%%%%%
%%%%%%%%%%%%%%%%%%%%%%%%%%%%%%%%%%%%%%%%%% 
%%%%%%%%%%%%%%%%%%%%%%%%%%%%%%%%%%%%%%%%%% 
%%%%%%%%%%%%%%%%%%%%%%%%%%%%%%%%%%%%%%%%%% 
%%%%%%%%%%%%%%%%%%%%%%%%%%%%%%%%%%%%%%%%%% 
%%%%%%%%PAGEBREAK%%%%%%%PAGEBREAK%%%%%%%%%
%%%%%%%%%%%%%%%%%%%%%%%%%%%%%%%%%%%%%%%%%% 
%%%%%%%%%%%%%%%%PAGEBREAK%%%%%%%%%%%%%%%%%
%%%%%%%%%%%%%%%%%%%%%%%%%%%%%%%%%%%%%%%%%% 
%%%%%%%%PAGEBREAK%%%%%%%PAGEBREAK%%%%%%%%%
%%%%%%%%%%%%%%%%%%%%%%%%%%%%%%%%%%%%%%%%%% 
%%%%%%%%%%%%%%%%%%%%%%%%%%%%%%%%%%%%%%%%%% 
%%%%%%%%%%%%%%%%%%%%%%%%%%%%%%%%%%%%%%%%%% 
%%%%%%%%%%%%%%%%%%%%%%%%%%%%%%%%%%%%%%%%%% 
%%%%%%%%PAGEBREAK%%%%%%%PAGEBREAK%%%%%%%%%
%%%%%%%%%%%%%%%%%%%%%%%%%%%%%%%%%%%%%%%%%% 
%%%%%%%%%%%%%%%%PAGEBREAK%%%%%%%%%%%%%%%%%
%%%%%%%%%%%%%%%%%%%%%%%%%%%%%%%%%%%%%%%%%% 
%%%%%%%%PAGEBREAK%%%%%%%PAGEBREAK%%%%%%%%%
%%%%%%%%%%%%%%%%%%%%%%%%%%%%%%%%%%%%%%%%%% 
%%%%%%%%%%%%%%%%%%%%%%%%%%%%%%%%%%%%%%%%%% 
\begin{alignment}[
  texts=edition[class="edition"];
  translation[class="translation"],
  ]
  \begin{edition}
    \ekddiv{type=ed}
%\nolinenumbers 
%\smallskip
\centerline{\textrm{\small{[\uproman{26}.\textsuperscript{\coro{\lowroman{1}-\lowroman{12}}}The 10 Channels]}}}
\bigskip
%\linenumbers
\begin{prose}
  \noindent
    \note[type=testium, labelb=165, lem={\textbf{Ri}}]{SSP 1.66 (Ed. p. 29): atha nāḍīnāṃ daśadvārāṇi | iḍā piṅgalā ca nāsādvārayor vahataḥ | gāndhārī hastijihvikā ca cakṣurdvārayor vahataḥ | pūṣā yaśasvinī ca karṇadvārayor vahataḥ | alambuṣā ānane vahati | kuhūr gudadvāre vahati | śaṅkhinī liṅgadvāre vahati | suṣumṇā madhyadeśe vahati | sā daṇḍamārgeṇa brahmarandhraparyantaṃ vahati | evaṃ daśanāḍyo daśadvāreṣu vahanti | anyāḥ sarvanāḍyo romakūpeṣu vahanti |}
\note[type=source, labelb=166, lem={\textbf{Re}}]{PT\textsuperscript{qcr \cdot YSV} (Ed. p. 838): idānīṃ śṛṇu nāḍīnāṃ bhedaṃ vakṣyāmi siddhidam | meruvāhye iḍānāmnī piṅgalayā samanvitā | suṣumnā bhānumārgeṇa brahmadvārāvadhi sthitā | sarasvatī sugandhā tu gāndhārī hastijihvakā | jñātavyā karṇayormadhye netrayoś ca tathāntimā | pūṣā cālambuṣā ceti mūlasthā kutracit tathā | liṅgadvārādiḍāmārge brahmasthānāvadhi priye | nāḍyantaṃ pratilomeṣu sahasrāṇāṃ dvisaptatiḥ |}
%-----------------------------   
%idānīṃ śarīramadhye nāḍīnāṃ       bhedāḥ   kathyante  daśamukhyanāḍyaḥ/ \E
%idānīṃ śarīramadhye nāḍīnāṃ       bhedāḥ   kathyaṃte  daśamukhyānāḍyaḥ \P
%idānī  śarīramadhye nāḍī----------bhedaḥ   kathyate// daśamukhyenāḍyā \B
%idānī  śarīramadhye nāḍī----------bhedaḥ   kathyate// daśamukhyānāḍayas... \L
%idānīṃ śarīramadhye nāḍīnām aparo bhedaḥ   kathyate// daśamukhyanādhyaḥ/ \N1
%idānīṃ śarīramadhye nāḍīnām aparo bhedaḥ// kathyaṃte  daśamukhyānādhyaḥ// \D
%idānī  śarīramadhye nāḍīnām aparo bhedāḥ   kathyate// daśamukhyanāḍyaḥ// \N2
%idānīṃ śarīramadhye nāḍīnām aparo bhedāḥ   kathyate   daśamukhyanāḍyas \U1
%idānīṃ śarīramadhye nāḍīnaṃ       bhedaḥ   kathyate   daśamukhyanāḍyaḥ// \U2
%-----------------------------
%Now the divisions of channels within the body are explained. There are ten primary channels. 
%-----------------------------
\app{\lem[wit={ceteri}]{idānīṃ}
  \rdg[wit={B,L,N2}]{idānī}}
śarīramadhye 
\app{\lem[wit={ceteri}]{nāḍīnāṃ}
  \rdg[wit={B,L}]{nāḍī°}
  \rdg[wit={X}]{nāḍīnām aparo}}
\app{\lem[wit={ceteri}]{bhedāḥ}
  \rdg[wit={B,D,L,N1}]{bhedaḥ}}
\app{\lem[wit={E,P,N2,U1}]{kathyante}
  \rdg[wit={ceteri}]{kathyate}}/
\app{\lem[wit={E,N2,U1,U2}]{daśamukhyanāḍyaḥ}
  \rdg[wit={P}]{daśamukhyānāḍyaḥ}
  \rdg[wit={B}]{daśamukhyenāḍyā}
  \rdg[wit={L}]{daśamukhyānāḍayas}
  \rdg[wit={D,N1}]{daśamukhyanādhyaḥ}}/
%----------------------------- 
%tanmadhye dvayam       iḍā  piṃgalāsaṃjñakaṃ       nāsādvāre tiṣṭhati/ \E
%tanmadhye nāḍīdvayaṃ   idāṃ piṃgalāsaṃjñakaṃ       nāsādvāre tiṣṭhati  \P
%tanmadhye nāḍīdvayaṃ/  idāpiṃgalāsaṃjñīkāḥ         nāsādvāre tiṣṭhati//  \B
%tanmadhye nāḍīdvayaṃ   idāpiṃgalāsaṃjñīkāḥ         nāsādvāre tiṣṭhati//  \L
%tanmadhye nāḍīdvayam/  idāpiṃgalāsaṃjñakaṃ         nāsādvāre tiṣṭhati//  \N1
%tanmadhye nāḍīdvayaṃ   idāpiṃgalāsaṃjñakaṃ         nāsānāsādvāre tiṣṭhati//  \D
%tanmadhye nāḍīdvayam/  idānīṃ piṃgalāsaṃjñakaṃ     nāsādvāre tiṣṭhati//  \N2
%tanmadhye nāḍīdvayaṃ   idāpiṃgalāsaṃjñākaṃ         nāsādvāre tiṣṭhati  \U1
%tanmadhye nāḍidvayaṃ// idā// piṃgalā// saṃjñākaṃ// nāsādvāre tiṣṭhati//  \U2
%-----------------------------
%Among them is a pair of channels. [Their] designation is Idā and Piṅgalā [and they] exist at the entrance of the nose. 
%-----------------------------
tanmadhye
\app{\lem[wit={ceteri}]{nāḍīdvayam}
  \rdg[wit={E}]{dvayam}}/
\app{\lem[wit={E}, alt={iḍāpiṅgalā}]{iḍāpiṃgalā}
  \rdg[wit={ceteri}]{idā piṃgalā}
  \rdg[wit={N2}]{idānīṃ piṃgalā}
  \rdg[wit={P}]{idāṃ piṃgalā}
}\app{\lem[wit={ceteri}]{saṃjñakaṃ}
  \rdg[wit={U1,U2}]{saṃjñākaṃ}
  \rdg[wit={B,L}]{saṃjñīkāḥ}}
\app{\lem[wit={ceteri}]{nāsādvāre}
  \rdg[wit={D}]{nāsānāsādvāre}}
tiṣṭhati/
%-----------------------------
%suṣumṇā    tālumārge   brahmadvāraparyantaṃ   vahati tiṣṭhati/ \E
%suṣumṇā    tālumārgeṇa brahmaraṃdhraparyanta--vahati tiṣṭhati... \P
%suṣumṇā    tālumārge   brahmaraṃdhraparyantaṃ vahatī tiṣṭhati... \B
%suṣumṇā    tālumārge   brahmaraṃdhraparyantaṃ vahati tiṣṭhati... \L
%suṣumṇā tu tālumārgeṇa brahmadvāraparyantaṃ   vahatī tiṣṭhati... \N1
%suṣumṇā tu tālumārgeṇa brahmadvāraparyantaṃ   vahatī tiṣṭhati    \D
%suṣumṇā tu tālumārge   brahmadvāraparyantaṃ   vahatī tiṣṭhati// \N2
%suṣumṇā tu tālumārgeṇa brahmadvāraparyantaṃ   vahati tiṣṭhati \U1
%suṣumṇā    tālumārgeṇa brahmadvāraparyantaṃ   vahati// \U2
%-----------------------------
%The Suṣumṇā flows by the path of the sun to the door of Brahma.  
%-----------------------------
\app{\lem[wit={ceteri}]{suṣumṇā}
  \rdg[wit={X}]{suṣumṇā tu}}
\app{\lem[type=conjecture, resp=egoscrconj]{bhānumārgena}
  \rdg[wit={D,P,N1,U1,U2}]{tālumārgeṇa}
  \rdg[wit={B,E,L,N2}]{tālumārge}}
\note[type=philcomm, labelb=166a, lem={bhānumārgena}]{Given the incongruity of \textit{tālu} in the given context, and the availability of a phonetically analogous and semantically superior alternative in the form of the term \textit{bhānu} as proposed by PT\textsuperscript{qcr \cdot YSV}, I have conjectured the latter as the more plausible option.}
brahma\app{\lem[wit={ceteri}, alt={°dvāra°}]{dvāra}
  \rdg[wit={B,L,P}]{°raṃdhra°}
}paryantaṃ
\app{\lem[wit={U2}]{vahati}
  \rdg[wit={E,L,P,U1}]{vahati tiṣṭhati}
  \rdg[wit={ceteri}]{vahatī tiṣṭhati}}/\textsuperscript{\begin{otherlanguage}{english}\coro{[\lowroman{5}]}\end{otherlanguage}}
%-----------------------------
%        sarasvatī mukhamadhye tiṣṭhati/ \E
%        sarasvatī mukhamadhye tiṣṭhati  \P
%        sarasvatī mukhamadhye tiṣṭhatī/ \B
%        sarasvatī mukhamadhye tiṣṭhati/ \L
%        sarasvatī mukhamadhye varttate/ \N1
%        sarasvatī mukhamadhye varttate// \D
%        sarasvatī mukhamadhye varttate/ \N2
%        sarasvatī mukhamadhye varttate \U1
%ti sraḥ sarasvati mukhamadhye tiṣṭhati// \U2
%-----------------------------
%The Sarasvatī[-channel] exists at the centre of the face. 
%-----------------------------
\app{\lem[wit={ceteri}]{sarasvatī}
  \rdg[wit={U2}]{ti sraḥ sarasvati}}
mukhamadhye
\app{\lem[wit={X}]{vartate}
  \rdg[wit={E,L,P,U2}]{tiṣṭhati}
  \rdg[wit={B}]{tiṣṭhatī}}/
%-----------------------------
%gāṃdhārīhastijihvākarṇayor            madhye  vahalyau  tiṣṭhataḥ/    \E
%gāṃdhārīhastinījihve karṇayor         madhye  vahatyau  tiṣṭhataḥ   \P
%gāṃdhārīhastījihve karṇa----------------------vahatyo   tiṣṭhati//                \B
%gāṃdhārīhastijihve karṇa----------------------vahatyo   tiṣṭhati...               \L
%gāṃdhārīhastinījihve karṇayor         madhye  vahatyau  tiṣṭhataḥ// \N1
%gāṃdhārīhastinījihve karṇayor         madhye  vahatyau  tiṣṭhataḥ// \D
%gāṃdhārīhastinījihve karṇayor         madhye  vahatyau  tiṣṭhataḥ// \N2
%gāṃdhādīharratījihvakarṇayor          madhye            tiṣṭhataḥ              \U1
%gāṃdhārī// hastinī// jihve// netrayor madhye  vahaṃtyaḥ//    \U2
%-----------------------------
%The two rivers, Gāṃdhārī and Hastjihvā, exist within the two ears. 
%-----------------------------
gāṃdhārī\app{\lem[wit={E}, alt={hastijihvākarṇayor}]{hastijihvākarṇayo\skp{r-ma}}
  \rdg[wit={D,P,N1,N2}]{hastinījihve karṇayor}
  \rdg[wit={B,L}]{hastījihve karṇa°}
  \rdg[wit={U1}]{harratījihvakarṇayor}
  \rdg[wit={U2}]{hastinī || jihve || netrayor}}
\app{\lem[wit={ceteri}, alt={madhye}]{\skm{r-ma}dhye}
  \rdg[wit={L,B}]{\om}}
\app{\lem[wit={D,P,N1,N2}]{vahatyau}
  \rdg[wit={E}]{vahalyau}
  \rdg[wit={B,L}]{vahatyo}
  \rdg[wit={U2}]{vahaṃtyaḥ}}
\app{\lem[wit={ceteri}]{tiṣṭhataḥ}
  \rdg[wit={B,L}]{tiṣṭhati}
  \rdg[wit={U2}]{\om}}/
%-----------------------------
%pūṣālambusemā         netrayor madhye rvahalyā tiṣṭhataḥ/ \E
%pūṣālaṃbuse           netrayor madhye vahatyau tiṣṭataḥ \P
%pūṣoḍalabuṣe----------netra----madhye vahatyo  tiṣṭhati/ \B
%pūṣo ulabuso          netra----madhye vahatyo  tiṣṭhaṃti// \L
%pūṣāṃalaṃbuṣe         netrayor madhye vahatyo  tiṣṭhataḥ/ \N1
%pūṣāṃalaṃbuṣe         netrayor madhye vahatyau tiṣṭhataḥ// \D
%pūṣāṃalaṃbuṣe         netayor  madhye vahatyo  tiṣṭhataḥ/ \N2
%pūṣālaṃbuṣe           netayor  madhye vahatyau tiṣṭhataḥ \U1
%pūṣāya śakhinī// karṇayor      madhye vahatyo  tiṣṭhata// alaṃbuṣā// bhu?madhye vaṃhatyo tiṣṭhati// \U2
%-----------------------------
%The two rivers Pūṣā and Ālaṃbuṣā are situated at the center of the two eyes. 
%-----------------------------
\app{\lem[resp=egoscr, type=emendation, alt={pūṣālaṃbuṣānetrayor}]{pūṣālaṃbuṣānetrayo\skp{r-ma}}
  \rdg[wit={E}]{pūṣālambusemā netrayor}
  \rdg[wit={P}]{pūṣālaṃbuse netrayor}
  \rdg[wit={B}]{pūṣoḍalabuṣe netra°}
  \rdg[wit={L}]{pūṣo ulabuso netra°}
  \rdg[wit={D,N1}]{pūṣāṃalaṃbuṣe netrayor}
  \rdg[wit={N2}]{pūṣāṃalaṃbuṣe netayor}
  \rdg[wit={U1}]{pūṣālaṃbuṣe netayor}
  \rdg[wit={U2}]{pūṣāya śakhinī || karṇayor}
}\skm{r-ma}dhye
\app{\lem[wit={ceteri}]{vahatyau}
  \rdg[wit={E}]{rvahalyā}
  \rdg[wit={B,L,N1,N2,U2}]{vahatyo}}
\app{\lem[wit={D,E,N1,N2,U1}]{tiṣṭhataḥ}
  \rdg[wit={B}]{tiṣṭhati}
  \rdg[wit={L}]{tiṣṭhaṃti}
  \rdg[wit={P}]{tiṣṭataḥ}
  \rdg[wit={U2}]{tiṣṭhata || alaṃbuṣā || bhrumadhye vaṃhatyo tiṣṭhati ||}}/
%-----------------------------
%śaṃkhinī liṃgadvārād ārabhye--ḍāmārgeṇa     brahmasthānaparyaṃtaṃ tiṣṭhatīti/     \E
%śaṃkhinī liṃgadvārād ārabhya iḍāmārgeṇa     brahmasthānaparyaṃtaṃ tiṣṭhati      \P   %%%%%%%7648.jpg
%śaṃkhinī liṃgadvārād ārabhya iḍāmārgeṇa     brahmasthānaparyaṃtaṃ tiṣṭhati/     \B
%śaṃkhinī liṃgadvārād ārabhya iḍāmārgeṇa     brahmasthānaparyaṃtaṃ tiṣṭhati//    \L 
%śaṃkhanī liṃgadvārād ārabhya iḍāmārgeṇa     brahmasthānaparyaṃtaṃ tiṣṭhati/     \N1
%śaṃkhinī liṃgadvārād ārabhya iḍāmārgeṇa     brahmasthānaparyaṃtaṃ tiṣṭhati//     \D
%śaṃkhinī liṃgadvārād ārabhya iḍānīṃ mārgeṇa brahmasthānaparyaṃtaṃ tiṣṭhati/ \N2
%śaṃkhinī liṃgadvārārabhya    iḍāmārgeṇa     brahmasthānaparyaṃtaṃ tiṣṭhati      \U1
%kuhū     liṃgadvārād ārabhya iḍāmārgeṇa     brahmasthānaparyaṃtaṃ tiṣṭhati// śāṃkhinī mūladvārād arabhya piṃgalamargeṇa brahmasthānaparyaṃtaṃ tiṣṭhati// \U2
%-----------------------------
%Śaṃkhinī stretches from the the beginning of the opening of the gender through the Iḍā-channel up to the Brahmasthāna. Kuhu stretches from the entrance of the root through the Piṅgalā-channel up to the Brahmasthāna.    
%-----------------------------
\app{\lem[wit={ceteri}]{śaṃkhinī}
  \rdg[wit={N1}]{śaṃkhanī}
  \rdg[wit={U2}]{kuhū}}
\app{\lem[wit={ceteri}, alt={liṃgadvārād}]{liṃgadvārā\skp{d-ā}}
  \rdg[wit={U1}]{liṃgadvārā°}}
\app{\lem[wit={ceteri}, alt={ārabhye}]{\skm{d-ā}rabhye}
  \rdg[wit={ceteri}]{ārabhya}
}\app{\lem[wit={E}]{iḍāmārgeṇa}
  \rdg[wit={ceteri}]{iḍāmārgeṇa}
  \rdg[wit={N2}]{iḍānīṃ mārgeṇa}}
brahmasthānaparyaṃtaṃ 
\app{\lem[wit={ceteri}]{tiṣṭhati}
  \rdg[wit={E}]{tiṣṭhatīti}}/
\extra{\app{\lem[type=conjecture, resp=egoscrconj]{kuhū}
    \rdg[wit={U2}]{śāṃkhinī}} mūladvārād-arabhya
\app{\lem[resp=egoscr, type=emendation, alt={piṃgalā°}]{piṅgalā}
  \rdg[wit={U2}]{piṃgala°}}margeṇa brahmasthānaparyaṃtaṃ tiṣṭhati/}
\note[type=philcomm, labelb=177, lem={kuhū}]{Without Kuhū as found in \getsiglum{U2} only, the list would be incomplete. In \getsiglum{U2} Śaṃkhinī and Kuhū are swapped, neither of them is found in PT\textsuperscript{qcr \cdot YSV} but both channels and their proper location are in SSP 1.66.}\textsuperscript{\begin{otherlanguage}{english}\coro{[\lowroman{10}]}\end{otherlanguage}}
%-----------------------------
%etādṛśa  nāḍyo daśasu dvāreṣu tiṣṭhanti/    \E
%etādṛṣā  nāḍyo daśasu dvāreṣu tiṣṭhaṃti      \P
%etādṛṣyā nāḍyo daśasu dvāreṣu tiṣṭhaṃti/    \B
%etādṛṣyā nāḍyo daśa   dvāreṣu    tiṣṭhaṃti/    \L 5876_15.jpg
%etādaśa  nāḍyo daśasu dvāreṣu tiṣṭhaṃti/    \N1
%etādaśa  nāḍyo daśasu dvāreṣu tiṣṭhaṃti//   \D
%etā            daśasu  dvāreṣu tiṣṭhaṃti/                \N2
%etādṛśa  nāḍyo daśasv adhāreṣu  tiṣṭhati    \U1
%etādaśa  nāḍyo daśaśoṣu dvāreṣu tiṣṭhaṃti// \U2 %%%413.jpg
%-----------------------------
%In such a way the channels are situated at the 10 openings. 
%-----------------------------
\app{\lem[wit={P}]{etādṛṣā}
  \rdg[wit={D,E,N1,U1,U2}]{etādṛśa}
  \rdg[wit={B,L}]{etādṛṣyā}
  \rdg[wit={N2}]{etā}}
\app{\lem[wit={ceteri}]{nāḍyo}
  \rdg[wit={N2}]{\om}}
\app{\lem[wit={ceteri}]{daśasu dvāreṣu}
  \rdg[wit={L}]{daśa dvāreṣu}
  \rdg[wit={U1}]{daśasv adhāreṣu}}
\app{\lem[wit={ceteri}]{tiṣṭhanti}
  \rdg[wit={U1}]{tiṣṭhati}}/
%-----------------------------
%anyā dvisaptatisahasraparimitā                      nāḍayo lomnāṃ mūleṣu sūkṣmarūpeṇa tiṣṭanti// \E
%anyā dvisaptatisahasraparimitā                      nāḍyo  lomnā  mūleṣu sūkṣmarūpeṇa tiṣṭaṃti      \P
%anyā dvisaptatīsahasraparimitā                      nāḍyo  lomnā  mūleṣu sūkṣmarūpeṇa tiṣṭaṃti// \B
%anyā dvisaptatisahasraparimitā                      nāḍyo  lomnā  mūleṣu sūkṣmarūpeṇa tiṣṭaṃti// \L
%anyā dvisaptatisahasraparamitā                      nāḍyā  lomnāṃ mūleṣu sūkṣmarūpeṇa tiṣṭaṃti// \N1
%anyā dvisaptatisahasraparamitā                      nāḍyā  lomnāṃ mūleṣu sūkṣmarūpeṇa tiṣṭaṃti// \D
%anyā dvisaptatrisahasraparimitā                     nāḍyā  lomnāṃ mūleṣu sūkṣmarūpeṇa tiṣṭaṃti// \N2
%anyā dvisaptatisahasraparimitāgryo                         lomnā  mūleṣu sūkṣmarūpeṇa tiṣṭaṃti \U1
%anyā hidaśonā dvisatyati sahasraḥ//71110// parimitā nādhyo lomnāṃ mūleṣu sūkṣmarūpeṇa tiṣṭaṃti// \U2
%-----------------------------
%The other channels measured as 72000 are situated with a subtle form at the roots of the hairs.
%-----------------------------
anyā
\app{\lem[wit={ceteri}]{dvisaptatisahasraparimitā}
  \rdg[wit={U1}]{dvisaptatisahasraparimitāgryo}
  \rdg[wit={U2}]{hidaśonā dvisatyati sahasraḥ || 71110 || parimitā}}
\app{\lem[wit={B,L,P}]{nāḍyo}
  \rdg[wit={E}]{nāḍayo}
  \rdg[wit={U2}]{nādhyo}
  \rdg[wit={U1}]{\om}}
\app{\lem[wit={D,E,N1,N2,U2}]{lomnāṃ mūleṣu} %%%lomnāṃ = gen pl neutrum v.loman
  \rdg[wit={B,L,P,U1}]{lomnā}}
sūkṣmarūpeṇa tiṣṭaṃti\dd{}
    \end{prose}
  \end{edition}
  \begin{translation}
    \ekddiv{type=trans}
    \begin{tlate}
 \centerline{\textrm{\small{[\uproman{26}.\textsuperscript{\coro{\lowroman{1}-\lowroman{12}}}The 10 Channels]}}}
 \bigskip
 \noindent
\blfootnote{\ldots the practice bestows \textit{siddhi}s. Even though not directly introduced as a practice, the unknown author of the \textit{Yogasvarodaya} also informs the reader about the \textit{siddhi}-bestowing nature of the channels without any practical instructions (idānīṃ śṛṇu nāḍīnāṃ bhedaṃ vakṣyāmi siddhidam). It seems that in the \textit{Yogasvarodaya}, the ten channels could have been a part of the practice of Antaralakṣya, too. However, both texts, the \textit{Yogasvarodaya}, as well as \textit{Yogatattvabindu}, do not directly indicate a practical usage of the taxonomy of the channels. Rather they present them in textbook style merely to inform about their existance, in the same way, the \citetitle{ssplonavla} 1.66 presents them, but their position remains suspicious.} Now, the divisions of channels within the body are explained.\footnote{Networks of channels are standard in many systems of yoga. The earliest known mention refers to Upaniṣadic literature, such as the \citetitle{brhadaranyaka} 2.1.19. For a detailed depiction of systems of channels in yogic literature, see \citeauthor[2017: 172-174, 184-198]{rootsofyoga2017}.} There are ten primary channels.\footnote{The notion of ten primary channels can already be found in early texts of Haṭhayoga, e.g., the \citetitle{vivekamartandaold} (Central Library, Baroda Acc. No. 4110, 1534 Saṃvat) 17 \begin{quote} teṣu nāḍisahasreṣu dvisaptatir udāhṛtāḥ |\\
pradhānāḥ prāṇavāhinyo bhūyas tatra daśa smṛtāḥ || \end{quote} ``Of those thousand channels, seventy-two have been spoken of, and among them, ten are considered most important. They are the main pathways of the vital breaths.''\\ Also cf. \citetitle{fausta1976} 34. However, the enumeration of ten main channels in yogic literature also has predecessors, e.g., in Śaivasiddhānta, cf. \citetitle{sardha} 10.4-5. Other systems, e.g., \citetitle{yajnavalkya} \uproman{4}.26 or \citetitle{vasisthasamhita} 2.21, enumerate fourteen primary channels.} Among them is a pair of channels. [Their] designation is Idā and Piṅgalā [and they] exist at the entrance of the nose. The Suṣumṇā flows by the path of the sun to the door of Brahma.\footnote{Discuss the course of susumna and explain the door of Brahman.}\textsuperscript{\coro{[\lowroman{5}]}} The Sarasvatī[-channel] exists at the center of the face. The two rivers, Gāndhārī and Hastjihvā, exist within the two ears. The two rivers, Pūṣā and Ālaṃbuṣā, are situated at the center of the two eyes. Śaṃkhinī stretches from the beginning of the opening of the gender through the Iḍā-channel up to the Brahmasthāna\footnote{The \textit{brahmasthāna} probably is a synonym for \textit{brahmarandhra}, ``The aperture of Brahmā/Brahman''; the fontanelle, through which the vital principle of the yogi exists at death, cf. \citetitle[438]{rootsofyoga2017}. In \citetitle{dhyanabind} 65, the \textit{brahmasthāna} is equated with the \textit{sahasrāracakra}.}. Kuhū stretches from the entrance of the root\footnote{The entrance of the root (\textit{mūladvāra}) in this passage refers to the anus, cf. \citetitle{ssplonavla} 1.66 (kuhūr gudadvāre vahati) ``Kuhū conducts through the anus''.} through the Piṅgalā-channel up to the Brahmasthāna.\textsuperscript{\coro{[\lowroman{10}]}} In such a way, the channels are situated at the ten openings. The other channels, measured as 72000, are situated with a subtle form at the roots of the hairs.
\end{tlate}
  \end{translation}
\end{alignment}




\chapter{Bibliography}
 \label{sec:bibli}
   \clearpage
\newpage 
\thispagestyle{empty}
\quad  \addtocounter{page}{-1}

\printbibliography[heading=subbibintoc, title=Consulted Manuskripts, keyword=codex]

\printbibliography[heading=subbibintoc, title=Printed Editions, keyword=printsource]

\printbibliography[heading=subbibintoc, title=Secondary Literature, keyword=seclit]

\printbibliography[heading=subbibintoc, title=Online Sources, keyword=onlinesource]

\end{document}


\centerline{\textrm{\small{[\uproman{27}.\textsuperscript{\coro{\lowroman{1}-\lowroman{23}}}The 10 Vitalwinds]}}}
\bigskip
%-----------------------------
%[p.36]
%idānīṃ śarīramadhye vāyavo daśa tiṣṭhanti/ \E
%idānīṃ śarīramadhye vāyavo daśa tiṣṭhaṃti  \P
%idānīṃ śarīramadhye .....\om               \B
%idānīṃ śarīramadhye .....\om               \L
%idānīṃ śarīramadhye vāyavas tiṣṭhaṃti/     \N1
%idānīṃ śarīramadhye vāyavas tiṣṭhaṃti//    \D
%idānīṃ śarīramadhye vāyavas tiṣṭhaṃti/     \N2
%idānīṃ śarīramadhye vāyavas tiṣṭhaṃti      \U1
%idānīṃ śarīramadhye vāyavo daśa ṣṭaṃti//   \U2
%-----------------------------
%Now ten vitalwinds are situated within the body.  
%-----------------------------
\note[type=source, labelb=178, lem={\textbf{Re}}]{PT\textsuperscript{qcr \cdot YSV} (Ed. p. 838): idānīṃ śṛṇu nāḍīnāṃ bhedaṃ vakṣyāmi siddhidam | meruvāhye iḍānāmnī piṅgalayā samanvitā | suṣumnā bhānumārgeṇa brahmadvārāvadhi sthitā | sarasvatī sugandhā tu gāndhārī hastijihvakā | jñātavyā karṇayor madhye netrayoś ca tathāntimā | pūṣā cālambuṣā ceti mūlasthā kutracit tathā | liṅgadvārādiḍāmārge brahmasthānāvadhi priye | nāḍyantaṃ pratilomeṣu sahasrāṇāṃ dvisaptatiḥ |}
idānīṃ śarīramadhye
\app{\lem[wit={E,P,U2}]{vāyavo}
  \rdg[wit={D,N1,N2,U1}]{vāyavas}
  \rdg[wit={B,L}]{\om}}
\app{\lem[wit={E,P,U2}]{daśa}
  \rdg[wit={ceteri}]{\om}}
\app{\lem[wit={ceteri}]{tiṣṭhanti}
  \rdg[wit={U2}]{ṣṭaṃti}
  \rdg[wit={B,L}]{\om}}/
%-----------------------------
%teṣāṃ nāmāni kāryāṇi kathyante/ \E
%teṣāṃ nāmāni kārmāṇi kathyante/ \P
%\om  \B
%\om \L
%teṣāṃ kāryāṇi kathyante/ \N1
%teṣāṃ kāryāṇi kathyaṃte/ \D
%teṣāṃ kāryāṇi kathyate/ \N2
%teṣāṃ kāryāṇi kathyate \U1 %%%280.jpg
%teṣāṃ kāryāṇi kathyate \U2
%-----------------------------
%their functions are taught. 
%-----------------------------
teṣāṃ
\app{\lem[wit={ceteri}]{kāryāṇi}
  \rdg[wit={E}]{nāmāni kāryāṇi}
  \rdg[wit={P}]{nāmāni kārmāṇi}}
\app{\lem[wit={ceteri}]{kathyante}
  \rdg[wit={N2,U1,U2}]{kathyate}}/
\note[type=philcomm, labelb=178a, lem={\uproman{27}.\textsuperscript{2}}]{Sentence \om in \getsiglum{B} and \getsiglum{L}.}
%-----------------------------
%prāṇavāyur hṛdayamadhye śvāsocchāsaṃ karoti/ \E
%prāṇavāyur hṛdayamadhye śvāsochāsaṃ karoti       \P
%------------------------śvāsośvaroti/               \B
%                        śvāsośvareti...             \L 
%prāṇavāyuhṛdayamadhye  utsvāsaprasvāsasaṃ karoti//   \N1
%prāṇavāyuhṛdayamadhye  utsvāsaprasvāsaṃ karotī   \D
%prāṇavāyuhṛdayamadhye ūrdhvaśvāsapraśvāsaṃ karoti// \N2
%prāṇavāyuhṛdayamadhye ūdhvasaprasase karoti \U1
%prāṇavāyuhṛdayamadhye   svāsochvāsaṃ karoti \U2
%-----------------------------
%The Prāṇa vitalwind is located in the middle of the heart and causes inhalation and exhalation. 
%----------------------------
\app{\lem[wit={E,P}, alt={prāṇavāyur}]{prāṇavāyu\skp{r-hṛ}}
  \rdg[wit={X,U2}]{prāṇavāyu°}
  \rdg[wit={B,L}]{\om}
}\app{\lem[wit={ceteri},alt={hṛdayamadhye}]{\skm{r-hṛ}dayamadhye}
  \rdg[wit={B,L}]{\om}}
\app{\lem[type=emendation, resp=egoscr]{ucchvaśvāsapraśvāsaṃ}
  \rdg[wit={N1}]{utsvāsaprasvāsasaṃ}
  \rdg[wit={N2}]{ūrdhvaśvāsapraśvāsaṃ}
  \rdg[wit={D}]{utsvāsaprasvāsaṃ}
  \rdg[wit={U1}]{ūdhvasaprasase}
  \rdg[wit={E}]{śvāsocchāsaṃ}
  \rdg[wit={P}]{śvāsochāsaṃ}
  \rdg[wit={B}]{śvāsośvaroti}
  \rdg[wit={L}]{śvāsośvareti}}/
\note[type=source, labelb=179, lem={\textbf{Ri}}]{SSP 1.67 (Ed. pp. 23-24): atha daśavāyavaḥ | hṛdaye prāṇavāyur ucchvāsaniḥśvāsakārako hakārasakārātmakaś ca | gude tv apānavāyuḥ recakakumbhakapūrakaś ca | nābhau samānavāyuḥ dīpakaḥ pācakaś ca| kaṇṭhe vyānavāyuḥ śoṣaṇāpy āyanakārakaś ca | tālau udānavāyuḥ grasanavamanajalpakārakaś ca| nāgavāyuḥ sarvāṅgavyāpakaḥ mocakaś cālakaś ca | kūrmavāyuḥ cakṣuṣor unmeṣakārakaś ca| kṛkalaḥ udgārakaḥ kṣutkārakaś ca | devadatto mukhavijṛmbhakaḥ | dhanañjayo nādaghoṣakah ||1.67|| iti daśavāyvavalokanena piṇḍotpattiḥ naranārīrūpam |}
%-----------------------------
%aśanapānecchā bhavati/   gudamadhye                                                            \E
%aśanapānechā  bhavati    gudamadhye 'pānāvāyus  tiṣṭhati    sa āṃkucanastaṃbhanaṃ   karoti     \P
%aśanapānechā  bhavati//  gudamadhye apānāvāyor  tiṣṭhatī    sa āṃkucanastaṃbhanaṃ   karotī/    \B
%aśanapānecha  bhavati//  gudamadhye apānāvāyo   tiṣṭhati    sa āṃkucanastaṃbhanaṃ   karotī/    \L
%asitapittecha bhavati/   guḍamadhye apānavāyu   tiṣṭhati    sa ākuṃcanasthaṃbhanaṃ  karoti/ /  \N2
%aśitapiteccha bhavati/   gudamadhye apānavāyus  tiṣṭhati/   sa ākuṃcanaṃ staṃbhanaṃ karoti/    \N1
%aśitapiteccha bhavati//  gudamadhye apānavāyus  tiṣṭhati/   sa ākuṃcanaṃ staṃbhanaṃ karoti//   \D
%asīte pitechā bhavati    gudamadhye apānavāyu   tiṣṭhati    sa ākuṃcanaṃ staṃbhanaṃ karoti     \U1
%aśanapānechā  bhavati//  gudamadhye apānāvāyo   tiṣṭhati//     āṃkucanastabhanaṃ    karoti/    \U2
%-----------------------------
%The wish for eating an drinking exists. At the center of the anus the Apāna-Vitalwind exists. He does contraction and checking. 
%-----------------------------
\app{\lem[wit={E}]{aśanapānecchā}
  \rdg[wit={B,L,P,U2}]{aśanapānechā}
  \rdg[wit={D,N1}]{aśitapiteccha}
  \rdg[wit={N2}]{asitapittecha}
  \rdg[wit={U1}]{asīte pitechā}}
bhavati/
gudamadhye
\app{\lem[type=emendation, resp=egoscr, alt={'pānavāyus}]{'pānavāyu\skp{s-ti}}
  \rdg[wit={D,N1}]{apānavāyus}
  \rdg[wit={B}]{apānāvāyor}
  \rdg[wit={L,U2}]{apānāvāyo}
  \rdg[wit={N2,U1}]{apānavāyu}
  \rdg[wit={E}]{\om}
}\app{\lem[wit={ceteri}, alt={tiṣṭhati}]{\skm{s-ti}ṣṭhati}
  \rdg[wit={B}]{tiṣṭhatī}
  \rdg[wit={E}]{\om}}/\textsuperscript{\begin{otherlanguage}{english}\coro{[\lowroman{5}]}\end{otherlanguage}}
\app{\lem[wit={ceteri}]{sa}
  \rdg[wit={U2}]{\om}}
\app{\lem[wit={D,N1,U1}]{ākuṃcanaṃ staṃbhanaṃ}
  \rdg[wit={N2}]{ākuṃcanasthaṃbhanaṃ}
  \rdg[wit={B,L,P,U2}]{āṃkucanastaṃbhanaṃ}}
\app{\lem[wit={ceteri}]{karoti}
  \rdg[wit={B}]{karotī}}/
\note[type=philcomm, labelb=179a, lem={\uproman{27}.\textsuperscript{\lowroman{6}}}]{Sentence \om in \getsiglum{E}.}
%-----------------------------
%              samāno vāyur vartate/ sapta samagrā nāḍīḥ śoṣayati/  \E
%  nābhīmadhye samāno varttate       sa samagrā    nāḍīḥ  śoṣayati    \P
%  nābhīmadhye smānā  vartate/       sa samagrā    nāḍī    śoṣayati//  \B
%  nābhīmadhye samānā vartate        sa samagrā    nāḍī    śoṣayatī//  \L
%  nābhimadhye samāno varttate/      sa samāgraṃ   nādhyaṃ śoṣayati/   \N2
%  nābhimadhye samāno varttate/      sa samagraṃ   nādhyaṃ śoṣayati//  \N1
%  nābhimadhye samāno varttate//     sa samagraṃ   nādhyaṃ śoṣayati//  \D
%  nābhimadhye samāno varttate       sa samagrāṃ   nāḍīṃ śoṣayati      \U1
%  nābhipadmamadhye samāno vartate// sa samagrā    nāḍī śoṣayati        \U2
%-----------------------------
%At the center of the navel the Samāna[-vitalwind] exists. He causes to dry up all the channels.
%-----------------------------
\app{\lem[wit={ceteri}]{nābhimadhye}
  \rdg[wit={U2}]{nābhipadmamadhye}
  \rdg[wit={E}]{\om}}
\app{\lem[wit={ceteri}]{samāno}
  \rdg[wit={E}]{samāno vāyur}
  \rdg[wit={B}]{smānā}}
  vartate/
  \app{\lem[wit={ceteri}]{sa}
    \rdg[wit={E}]{sapta}}
  \app{\lem[wit={Y}]{samagrā}
    \rdg[wit={X}]{samāgraṃ}}
  \app{\lem[wit={E,P}]{nāḍīḥ}
    \rdg[wit={B,L,U2}]{nāḍī}
    \rdg[wit={U1}]{nāḍīṃ}
    \rdg[wit={D,N1,N2}]{nādhyaṃ}}
  \app{\lem[wit={ceteri}]{śoṣayati}
    \rdg[wit={L}]{śoṣayatī}}/
%-----------------------------
%tathā nāḍīśoṣaṇāt                      rucim  utpādayati/  vahniṃ dīpayati/ \E
%tathā nāḍīḥ pośayati                   rucim  utpādayati   vahniṃ dīpayatī \P
%tathā       pośayatī/ tathā poṣayatī// rucir  utpādayatī   vahnī  dīpayatī/ \B
%tathā       pośayatī                   rucim  utpādayatī   vahnī  dīpayatī... \L
%tathā nāḍīṃ pośayati/                  kvacit-utpādayati/  āgniṃ  dīpayati \N1
%tathā nāḍīṃ pośayati//                 kvacit-utpādayati// āgniṃ  dīpayati \D %%%p. 11 recto
%tathā nāḍīṃ pośayati/                  kvacit-utpādayati/  āgniṃ  dīpayati \N2
%tathā nāḍīṃ pośa iti                   rucim  utpādayati    agnīṃ  dīpayati \U1
%            ṣoṣayati                   rucim  utpādayati//  vahniṃ dīpayati// \U2
%-----------------------------
%In this way the channels are caused to thrive, beauty is caused to be generated and the fire is caused to light up.  
%-----------------------------
\app{\lem[wit={ceteri}]{tathā}
  \rdg[wit={U2}]{\om}}
\app{\lem[wit={P}]{nāḍīḥ}
  \rdg[wit={E}]{nāḍī}
  \rdg[wit={X}]{nāḍīṃ}
  \rdg[wit={B,L,U2}]{\om}}
\app{\lem[type=emendation, resp=egoscr]{poṣayati}
  \rdg[wit={D,P,N1,N2}]{pośayati}
  \rdg[wit={B}]{pośayatī | tathā poṣayatī}
  \rdg[wit={L}]{pośayatī}
  \rdg[wit={U1}]{pośa iti}
  \rdg[wit={U2}]{ṣoṣayati}
  \rdg[wit={E}]{°śoṣaṇāt}}/
\app{\lem[wit={ceteri}, alt={rucim}]{ruci\skp{m-u}}
  \rdg[wit={B}]{rucir}
  \rdg[wit={D,N1,N2}]{kvacit}}
\app{\lem[wit={ceteri}, alt={utpādayati}]{\skm{m-u}tpādayati}
  \rdg[wit={P}]{utpādayatī}}/
\app{\lem[type=emendation, resp=egoscr]{agniṃ}
  \rdg[wit={D,N1,N2}]{āgniṃ}
  \rdg[wit={U1}]{agnīṃ}
  \rdg[wit={E,P,U2}]{vahniṃ}
  \rdg[wit={B,L}]{vahnī}}
\app{\lem[wit={ceteri}]{dīpayati}
  \rdg[wit={B,L}]{dīpayatī}}/\textsuperscript{\begin{otherlanguage}{english}\coro{[\lowroman{10}]}\end{otherlanguage}}
%-----------------------------
%tālumadhye udāno vāyus-tiṣṭhati/   sa vāyuḥ ratnaṃ līlati/    pānīyaṃ pibati/  nāgavāyuḥ   sarva--śarīre varttate/  tasmād-vāyoḥ śarīraṃ cālayati/ śokam āpnoti// vivilaḥ        \E
%tālumadhye udāno vāyus-tiṣṭhati    sa vāyu  ratnaṃ gilati     pānīyaṃ pībati   nāgavāyuḥ   sakale śarīre varttate   tasmād-vāyo śarīraṃ calayati   śopham āpnoti  vikṛtaḥ        \P %%%7649.jpg
%tālumadhye udānavāyus-tiṣṭhati/    sa vāyur annaṃ  galayatī/  pānīyaṃ pibatī/  nāgavāyuḥ   sakala-śarīre varttate   tasmād-vāyoḥ// śarīre cālatī/  śokam āpnoti   vi??kru??taḥ// \B DSCN7163.JPG Z.11
%tālumadhye udānavāyus tiṣṭhati//   sa vāyur annaṃ  galayati// pānīyaṃ pibatī// nāgavāyu----sakala-śarīre vartate    tasmād vāyoḥ// śarīre cālayatī śokam āpnoti   vikutaḥ...     \L
%tālumadhye udānavāyus-tiṣṭhati/    sa vāyuḥ ratnaṃ śilati/    pānīyaṃ pibati/  nāgavāyuḥ   sakale śarīre varttate// tasmād-vāyoḥ śarīraṃ calati/                                 \N1
%tālumadhye udāno vāyus-tiṣṭhati//  sa vāyur annaṃ  gilati/    pānīyaṃ pibati   nānāgavāyuḥ sakale śarīre varttate// tasmād-vāyoḥ śarīraṃ calati//                                \D
%tālumadhye udānāni vāyus-tiṣṭhati/ sa vāyur-annaṃ  gīlati/    pānīyaṃ pibati/  nāgavāyuḥ   sakale śarīre varttate// tasmād-vāyoḥ śarīraṃ calati/                                 \N2
%tālumadhye udānavāyus-tiṣṭhati     sa vāyur-annaṃ  gilati     pānīyaṃ pibati   nāgavāyu    sakale śarīre varttate   tasmād-vāyoḥ śarīraṃ calati                                  \U1
%tālumadhye udāno vāyus-tiṣṭhati//   sa vāyur annaṃ  gilati//  pānīyaṃ pibati// nāgavāyuḥ   sakale śarīre varttate// tasmād-vāyoḥ śarīraṃ calayati śokam āpnoti vikṛtaḥ//         \U2
%-----------------------------
%Within the throat the Udāna-vitalwind is situated. This wind swallows food, [and] it drinks water. The Nāga-vitalwind exists in the entire body. Through the vitalwind the body is caused to move. 
%em. nāgavāyu = vyānavāyuḥ ....
%-----------------------------
tālumadhye
\app{\lem[wit={B,L,N1,U1}, alt={udānavāyus}]{udānavāyu\skp{s-ti}}
  \rdg[wit={D,E,P,U2}]{udāno vāyus}
  \rdg[wit={N2}]{udānāni vāyus}
}\skm{s-ti}ṣṭhati/
sa \app{\lem[wit={ceteri}, alt={vāyur}]{vāyu\skp{r-a}}
  \rdg[wit={E}]{vāyuḥ}
  \rdg[wit={P}]{vāyu}
}\app{\lem[wit={ceteri}, alt={annaṃ}]{skm{r-a}nnaṃ}
  \rdg[wit={E,P,N1}]{ratnaṃ}}
\app{\lem[wit={ceteri}]{gilati}
  \rdg[wit={E}]{līlati}
  \rdg[wit={B}]{galayatī}
  \rdg[wit={L}]{galayati}
  \rdg[wit={N1}]{śilati}}/
pānīyaṃ \app{\lem[wit={ceteri}]{pibati}
  \rdg[wit={P}]{pībati}
  \rdg[wit={B,L}]{pibatī}}/
\app{\lem[wit={ceteri}]{nāgavāyuḥ}
  \rdg[wit={L}]{nāgavāyu°}
  \rdg[wit={D}]{nānāgavāyuḥ}}
\app{\lem[wit={ceteri}]{sakale}
  \rdg[wit={B,L}]{sakala°}
  \rdg[wit={E}]{sarva°}}
śarīre varttate/ 
%\note[type=philcomm, labelb=s35.z3a, lem={nāgavāyu}]{Only nine of the promised ten vitalwinds are described here. The missing vitalwind is \textit{vyānavāyu}. The description of \textit{nāgavāyu} matches rather the \textit{vyānavāyu}. Witnesses E, P, B, L and U2 preserve a nonsensical fragment after the description of \textit{nāgavāyu}: śokam āpnoti vikṛtaḥ. Possibly the description of \textit{vyānavāyu} was lost due to an eyeskip of a scribe.}
tasmā\skp{d-vā}\app{\lem[wit={ceteri},alt={vāyoḥ}]{skm{d-vā}yoḥ}
  \rdg[wit={P}]{vāyo}}
\app{\lem[wit={ceteri}]{śarīraṃ}
  \rdg[wit={B,L}]{śarīre}}
\app{\lem[type=emendation, resp=egoscr]{calayati}
  \rdg[wit={E}]{cālayati| śokam āpnoti || vivilaḥ}
  \rdg[wit={P}]{calayati śopham āpnoti vikṛtaḥ}
  \rdg[wit={B}]{cālatī | śokam āpnoti vikrutaḥ ||}
  \rdg[wit={L}]{cālayatī śokam āpnoti vikutaḥ}
  \rdg[wit={U2}]{calayati śokam āpnoti vikṛtaḥ ||}
  \rdg[wit={ceteri}]{calati}}/\textsuperscript{\begin{otherlanguage}{english}\coro{[\lowroman{15}]}\end{otherlanguage}}
%-----------------------------
%kūrmavāyur netramadhye tiṣṭhati/ nimeṣonmeṣaṃ karoti/ \E
%kūrmavāyur netramadhye           nimeṣonmeṣaṃ karoti \P
%kūrmavāyoḥ netramadhye           nimeṣonmeṣaṃ karotī/ \B
%kūrmavāyoḥ netramadhye           nimiṣonmeṣaṃ karotī... \L
%kūrmo vāyunetramadhye tiṣṭhati/  unmeṣaṃ nimeṣaṃ karoti/ \N1
%kūrmo vāyunetramadhye tiṣṭhati/  unmeṣaṃ nimeṣaṃ ca karoti// \D
%kūrmo vāyunetramadhye tiṣṭhati/  unmeṣaṃ nimeṣaṃ karoti/ \N2
%\om                                                     \U1
%kūrmavāyur netramadhye           nimiṣonmeṣaṃ karoti//            \U2
%-----------------------------
%The Kūrma-vitalwind exists within the eyes. It causes [the] opening and closing [of the eyes]. 
%-----------------------------
\app{\lem[wit={E,P,U2}, alt={kūrmavāyur}]{kūrmavāyu\skp{r-ne}}
  \rdg[wit={B,L}]{kūrmavāyoḥ}
  \rdg[wit={D,N1,N2}]{kūrmo vāyu}}
\skm{r-ne}tramadhye
\app{\lem[wit={D,E,N1,N2}]{tiṣṭhati}
  \rdg[wit={ceteri}]{\om}}/  
\app{\lem[wit={E,P,B,U2}]{nimeṣonmeṣaṃ}
  \rdg[wit={N1,N2}]{unmeṣaṃ nimeṣaṃ}
  \rdg[wit={D}]{unmeṣaṃ nimeṣaṃ ca}}
\app{\lem[wit={ceteri}]{karoti}
  \rdg[wit={B,L}]{karotī}}/
\note[type=philcomm, labelb=179b, lem={\uproman{27}.\textsuperscript{\lowroman{17}-\lowroman{18}}}]{Sentences \om in \getsiglum{U1}.}
%-----------------------------
%kṛkalakartāvāyur  udgāraṃ karoti      \E
%kṛkalavāyur       udhāraṃ karoti      \P
%kṛkalavāyur       udhāraṃ karotī      \B
%kṛkalavāyur       uhāraṃ karotī        \L
%kṛkalavāyor       ūdgāro bhavati//    \N1
%kṛkalavāyor-------ūdgāto bhavati/      \D
%kṛkaravāyor-------ūdgāro bhavati/      \N2
%                                       \U1
%puṣkaravāyur      udgāraṃ karoti//    \U2
%-----------------------------
%From the Kṛkala-vitalwind gagging arises. 
%-----------------------------
\app{\lem[wit={D,N1,N2},alt={kṛkalavāyor}]{kṛkalavāyo\skp{r-u}}
  \rdg[wit={B,L,P}]{kṛkalavāyur}
  \rdg[wit={E}]{kṛkalakartāvāyur}
  \rdg[wit={U2}]{puṣkaravāyur}
}\app{\lem[type=emendation, resp=egoscr, alt={udgāro}]{\skm{r-u}dgāro}
  \rdg[wit={E,U2}]{udgāraṃ}
  \rdg[wit={B,P}]{udhāraṃ}
  \rdg[wit={L}]{uhāraṃ}
  \rdg[wit={N1,N2}]{ūdgāro}
  \rdg[wit={D}]{ūdgāto}}
\app{\lem[wit={D,N1,N2}]{bhavati}
  \rdg[wit={E,P,U2}]{karoti}
  \rdg[wit={B,L}]{karotī}}/
%-----------------------------
% devadattavāyoḥ  jṛmbhaṇaṃ bhavati/ dhanaṃjayavāyoḥ śabda utpadyate// \E
% devadattavāyor  jumbhā bhavati     dhanaṃjayavāyo  śabdāḥ utpadyete  \P
% devadattavāyor  jumbhā bhavaṃtī    dhanaṃjayavāyoḥ śabda utpadyate// \B
% devadattavāyor  jṛṃbhā bhavatī     dhanaṃjayavāyoḥ śabdaḥ utpadyate// \L
% devadattavāyor  jṛṃbha utpadyate// dhanaṃjayavāyo  śabda utpadyate// \N1
% devadattavāyor  jṛṃbha utpadyate// dhanaṃjayavāyo  śabda utpadyate// \D
% devadattavāyo   jṛṃbhotpadyate/    dhanaṃjayavāyo  śabdotpadyate// \N2
% devadattavāyor  jaṃbhā utpadyate   dhanaṃjayavāyoḥ sabta utpadyate \U1
% devadattavāyo   jṛṃbhā bhavati//   dhanaṃjayavāyoḥ śabda utpadyate// \U2
%-----------------------------
%From the Devadatta-vitalwind jawning arises. From the Dhanaṃjaya-vitalwind speech arises. 
%-----------------------------
\app{\lem[wit={ceteri}, alt={devadattavāyor}]{devadattavāyo\skp{r-jṛ}}
  \rdg[wit={E}]{devadattavāyoḥ}
  \rdg[wit={N2,U2}]{devadattavāyo}
}\app{\lem[wit={D,N1,U2},alt={jṛmbha}]{\skm{r-jṛ}mbha}
  \rdg[wit={E}]{jṛmbhaṇaṃ}
  \rdg[wit={B,P}]{jumbhā}
  \rdg[wit={L}]{jṛṃbhā}
  \rdg[wit={N2}]{jṛṃbho°}
  \rdg[wit={U1}]{jaṃbhā}}
\app{\lem[wit={X}]{utpadyate}
  \rdg[wit={E,P,U2}]{bhavati}
  \rdg[wit={B}]{bhavaṃtī}
  \rdg[wit={L}]{bhavatī}}/
\app{\lem[wit={Y}]{dhanaṃjayavāyoḥ}
  \rdg[wit={X}]{dhanaṃjayavāyo}}
\app{\lem[wit={ceteri}]{śabda}
  \rdg[wit={P}]{śabdāḥ}
  \rdg[wit={L}]{śabdaḥ}
  \rdg[wit={N2}]{śabdo°}
  \rdg[wit={U1}]{sabta}}
utpadyate\dd{}\textsuperscript{\begin{otherlanguage}{english}\coro{[\lowroman{20}]}\end{otherlanguage}}


   \begin{tlate}
  \ekddiv{type=trans}
\bigskip
\centerline{\textrm{\small{[\uproman{26}.\textsuperscript{\coro{\lowroman{1}-\lowroman{23}}}The 10 Vitalwinds]}}}
 \bigskip
 Now [there are] ten vitalwinds are situated within the body. The Prāṇa vitalwind is located in the middle of the heart and causes inhalation and exhalation. The wish for eating an drinking exists. At the center of the anus the Apāna-Vitalwind exists. He does contraction and checking. At the center of the navel the Samāna[-vitalwind] exists. He causes to dry up all the channels. In this way the channels are caused to thrive, beauty is caused to be generated and the fire is caused to light up. Within the throat the Udāna-vitalwind is situated. This wind swallows food, [and] it drinks water. The Nāga-vitalwind exists in the entire body. Through the vitalwind the body is caused to move. The Kūrma-vitalwind exists within the eyes. It causes [the] opening and closing [of the eyes]. From the Kṛkala-vitalwind gagging arises. From the Devadatta-vitalwind jawning arises. From the Dhanaṃjaya-vitalwind speech arises.