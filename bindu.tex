%Ultimatives Tool zur Datierung:
%https://www.cc.kyoto-su.ac.jp/~yanom/pancanga/
%skp = ignored in edition
%skm = ignored in xml
%%%---2-DO---%%%:
% - add xml ids for cladistics
% - produce diplomatic transcripts for saktumiva
% - make extra layer in Apparatus for parallels in SVARODAYA, Siddhasiddhantapaddhati and Amanaska
% - check all daṇḍas!!! now I think that it's more likely that many of them were lost in copies. Lectio difficilior! Very unconventional style of the autor! 
% - read Sarvangayogapradipika, Maya Burger! 
% - maybe add second ciritical edition of yogasvarodaya?!
% - Korrekturlesen von \E!! 
% - Verspattern einbauen!
% - add all Testtimonia of SSP & Ysv
% - Sigla alphabetisch ordnen und! daṇḍas mit einkollationieren
% - präambel auslagern wie Jürgen
% - grep-search alle Verse!!!!
% - Mss spreadsheet
% - sort N1,D1,B2 zu N1,N2,D1
% - sort all sigla alphabetically 
% - additions to U2: make footnotes for the bahir mātrā-s: explaining the inventions of female deities and tell that this is "schwer interpretierbar"
% - Belege für source und testimonia einfügen!!!
% - GIVE UNIQUE LABELS for TESTIMONIO AND SOurces
% - Edition mit Sätzen übereinander nennt sich: Synoptische Edition
% - Consider changing Lakṣya to Lakṣa
% - vEREINHEITLICHUNG von source und testium! 
%%%%%%%%%%%%%%%%%%%%%%%%%%%%%%%%%%%%%%%%%
% Don't forget
% Siddhasiddhantapaddhati Yogic Body descriptions are followed by Rāmacandra
% Quotes of the Yogasvarodaya in the Yoga Karṇikā
% Rāmacandra more a compiler than an author!!!
% Identify quotes of YTB in Haṭhasanketacandrikā 
%%%%%%%%%%%%%%%%%%%%%%%%%%%%%%%%%%%%%%%%%%%
%MSS notes
%
%--B: i and ī are not differenciated
%--P: no punctuation no daṇdas nothing
%--U1: dot . serves as daṇḍa 
%--\L and \U2 very similar
%--figure out for U2: // ajapājapaḥ sahasra // 6000 //gha 0 16 pa 0 40// \U2?!?!?!?!?!?
%%%%%%%%%%%%%%%%%%%%%%%%%%%%%%%%%%%%%%%%%%
%
% Einleitung Ideen 
% - sprachliche Simplizität
% - Potenzial als Anfängertext
% - Großartige Einführung in die Textkritik -> Synoptische Edition 
% - Gelegenheit Yogasvarodaya und Yogatattvabindu zu edieren 
% - Historische Evidenz entweder für das königliche Leben in einer Maṭha in der Nähe von Benares während der Muslimischen Herrschaft, oder sogar Lehrtext für die Bildung junger Prinzen  
% - eines der raren Beispiele der engen Verknüfung mehrerer Texte 
% - eines der raren Beispiele der Prosaisierung eines metrischen Textes 
% - Anwendung rezenter Technologie! 
% - How the text was construed -> intermingling of Ysv and SSP
% - Martin Straube: "jeder kleine Dorfhäuptling kann Rāja genannt werden". 
%%%%%%%%%%%%%%%%%%%%%%%%%%%%%%%%%%%%%%%%%%%
%Ich habe dieses Zitat gefunden
%Franz
%Franz Veit
%हठयोगः [Printed book page 5-501-c]
%हठयोगः , पुं, (हठेन योगः ।) योगविशेषः ।
%यथा, —
%“इदानीं हठयोगस्तु कथ्यते हठसिद्धिदः ।
%कृत्वासनं पवनाशं शरीरे रोगहारकम् ॥
%पूरकं कुम्भकञ्चैव रेचकं वायुना भजेत् ।
%इत्थं क्रमोत्क्रमं ज्ञात्वा पवनं सग्धयेत् सदा ॥
%धौत्यादिकर्म्मषट्कञ्च संस्कुर्य्याद्धठसाधकः ।
%एतन्नाड्यान्तु देवेशि ! वायुपूर्णं प्रतिष्ठितम् ॥
%ततो मनो निश्चलं स्यात्तत आनन्द एव हि ।
%हठयोगान्न कालः स्यान्मनः शून्ये भवेद्यदि ॥
%इदानीं हठयोगस्य द्वितीयं भेदवत् शृणु ।
%आकाशे नासिकाग्रे तु सूर्य्यकोटिसमं स्मरेत् ॥
%श्वेतं रक्तं तथा पीतं कृष्णमित्यादिरूपतः ।
%एवं ध्यात्वा चिरायुः स्यादङ्गाजननवर्ज्जितः ॥
%शिवतुल्यो महात्मासौ हठयोगप्रसादतः ।
%हठाज्ज्योतिर्म्मयो भूत्वा ह्यन्तरेण शिव भवेत् ।
%अतोऽयं हठयोगः स्यात् सिद्धिदः सिद्धसेवितः ॥”
%इति योगस्वरोदयः ॥ [ID=41348]

%Now, Haṭhayoga indeed is explained as that which gives the siddhi (accomplishment) of haṭha (persistence).
%One performs the wind-eating/serpent āsanam, which removes illness in the body
%and filling – kumbhaka – emptying may distribute the vāyu/wind.
%In this way, while being aware of the progress and regress of the breath one may feed on the wind continually.
%And with the six karmmas, dhauti etc., the Sādhaka of Haṭha may prepare/embellish himself.
%Thus/thereby, in the channel (nāḍī), Oh supreme Goddess, all of the winds (vāyu) are consecrated/placed.
%Then the mind may be unmoved and then bliss it really is.
%Through Haṭhayoga time will be no more, when the mind in emptiness abides.
%
%Now listen to the second disclosure of haṭhayoga:
%In space, on the tip of the nose indeed, one may remember equal to ten million suns,
%the primal forms: white, red, likewise yellow, dark blue.
%Thus meditating/visualizing, one may have a long life, free of the birth of the body,
%Equal to Śiva, this great soul, due to the blessing of Haṭhayoga,
%shall become through persistence (haṭha) a being of light and internally śiva.
%Therefore this Haṭhayoga grants accomplishment – it’s practiced by the Siddhas (accomplished ones).
%Franz
%Im śabdakalpadruma
%Franz
%Franz Veit
%fj.veit@gmail.com
\documentclass[10pt]{memoir}
\setstocksize{220mm}{155mm} 	        
\settrimmedsize{220mm}{155mm}{*}	
\settypeblocksize{170mm}{116mm}{*}	
\setlrmargins{18mm}{*}{*}
\setulmargins{*}{*}{1.2}
%\setlength{\headheight}{5pt}%
\checkandfixthelayout[lines]
\linespread{1.16}
\flushbottom

%%% Hyphenation settings
\usepackage[htt]{hyphenat}
\hyphenation{he-lio-trope opos-sum}
\tracingparagraphs=1
%Hyphenation in Devanāgarī of the edition still missing? Probably this needs to be modified in babel-iast package? 

%%% babel
\usepackage[english]{babel}
\usepackage{babel-iast/babel-iast}

\babelfont[iast]{rm}[Renderer=Harfbuzz, Scale=1.3]{AdishilaSan}%AdishilaSan}
\babelfont[english]{rm}{Adobe Text Pro}

%%% more functionality
\PassOptionsToPackage{hyphens}{url}
\usepackage{hyperref}
\usepackage{pdflscape}
\usepackage{cleveref}
\usepackage{url}
\usepackage{cleveref}
\usepackage{microtype}
\usepackage{lineno}

%\usepackage{bigfoot}
%%% more functions
\usepackage[dvipsnames]{xcolor}
%\usepackage[para,perpage]{footmisc}

%%%für den Counter von Kapiteln und Sätzen! 
\newcommand{\uproman}[1]{\uppercase\expandafter{\romannumeral#1}}
\newcommand{\lowroman}[1]{\romannumeral#1\relax}

\makeindex
\newfontfamily\sanskritfont[Script=Devanagari,Mapping=RomDev,Scale=1.1]{Sanskrit2003}
\usepackage{pifont,fourier-orns,lettrine,psvectorian,paralist,enumitem,pdfpages,wrapfig,tabulary,lettrine,longtable}
\setlist[enumerate]{itemsep=0mm}
\usepackage[autostyle]{csquotes}
\usepackage[defaultlines=2,all]{nowidow}
\usepackage{ellipsis,adforn,booktabs,longtable,url,tikz}
\lineskiplimit=-3pt          

\makechapterstyle{IeT}{%
  \chapterstyle{default}
  \renewcommand*{\printchapternonum}{\centering}
  \renewcommand*{\clearforchapter}{\cleartorecto} 
  \aliaspagestyle{chapter}{empty}}
\chapterstyle{IeT}
\setsecnumdepth{none}  \openright  \nouppercaseheads
\settocdepth{subsubsection}

%%%% test better pagebreaks
%\def\fussy{%
%  \emergencystretch\z@
%  \tolerance 200%
%  \hfuzz .1\p@
%  \vfuzz\hfuzz}

%\interfootnotelinepenalty=10000\relax

%\usepackage[maxfloats=256]{morefloats}

%\maxdeadcycles=500

%raggedbottomsectiontrue
%%\checkandfixthelayout


%%%%%%%  biblatex
%\newcommand{\noun}[1]{\textsc{#1}}    %  philosophy-verbose
\usepackage[backend=biber, sorting=nyt, style=verbose]{biblatex} %%%%ORIGINAL TiE
\renewcommand*{\mkbibnamefamily}[1]{\textsc{#1}}


\DeclareFieldFormat{url}{%
  \mkbibacro{URL}\addcolon\space
  \href{#1}{\nolinkurl{\thefield{urlraw}}}}

\DeclareFieldFormat{citeurl}{%
  \href{#1}{\nolinkurl{\thefield{urlraw}}}} 


\DeclareFieldFormat{postnote}{#1}
\renewcommand{\postnotedelim}{, }
\addbibresource{bindu.bib}

%%% ekdosis
\usepackage[teiexport=tidy,parnotes=true]{ekdosis}% =tidy cleans up HTML and XML documents by fixing markup errors and upgrading legacy code to modern standards. parnotes=footnotes below or above critical apparatus

\SetLineation{lineation=page, modulo} %lineation=page sets thenumbering to start afresh at the top of each page. =modulo makes every fifth line numbered. {lineation=page} makes every line numbered! 

\renewcommand{\linenumberfont}{\selectlanguage{english}\footnotesize} %sets language of lines to English

\SetTEIxmlExport{autopar=false} %autopar=falseinstructs ekdosis to ignore blank lines in the.tex sourcefile as markers for paragraph boundaries. As a result, each paragraph of the edition must be found within an environment associated with the xml <p> element

\SetHooks{
  lemmastyle=\bfseries,
  %refnumstyle=\selectlanguage{english}\bfseries,
  refnumstyle=\selectlanguage{english}\color{blue}\bfseries,
  appheight=0.8\textheight,
}

\newif\ifinapparatus
\DeclareApparatus{source}[
%bhook=\inapparatustrue,
lang=english,
notelang=english,
% bhook=\selectlanguage{english},
bhook=\selectlanguage{english}\textbf{Sources:},%
%maxentries=4, 
%ehook=.]
%sep={] },
%nosep,
]

\newif\ifinapparatus
\DeclareApparatus{testium}[
%bhook=\inapparatustrue,
lang=english,
notelang=english,
% bhook=\selectlanguage{english},
bhook=\selectlanguage{english}\textbf{Testimonia:},
%maxentries=4, 
%ehook=.]
%nosep, 
]

% Declare \ifinapparatus and set \inapparatustrue at the beginning of
% the apparatus criticus block. Also set the language.  
\newif\ifinapparatus
  \DeclareApparatus{default}[
  %bhook=\inapparatustrue, 
  lang=english,
  %maxentries=33,
  %bhook=\selectlanguage{english},
  sep = {] },
  delim=\hskip 0.75em,
  rule=\rule{0.7in}{0.4pt},
]

\newif\ifinapparatus
\DeclareApparatus{philcomm}[
%bhook=\inapparatustrue,
lang=english,
notelang=english,
bhook=\selectlanguage{english}\textbf{Philological Commentary:},
%bhook=\selectlanguage{english},
sep={: },
]

\ekdsetup{
showpagebreaks,
spbmk = \textcolor{blue}{spb},
hpbmk = \textcolor{red}{hpb}
}

%\usepackage{fnpos}
%\makeFNmid
%\makeFNbottom
\usepackage[bottom]{footmisc}
%%%%%%%%%%%%%%%%%%%%%%%%%%%
\makeatletter
\def\blfootnote{\gdef\@thefnmark{}\@footnotetext}
\makeatother
%%%%%%%%%%%%%%%%%%%%%%%%%


% Macros and Definitions for the Print of Sigla
\def\acpc#1#2#3{{#1}\rlap{\textrm{\textsuperscript{#3}}}\textsubscript{\textrm{#2}}\space}
\def\sigl#1#2{{{#1}}\textsubscript{\textrm{#2}}}
\def\None{{\sigl{N}{1}}} \def\Noneac{\acpc{N}{1}{ac}\,} \def\Nonepc{\acpc{N}{1}{pc}\,}
\def\Ntwo{{\sigl{N}{2}}} \def\Noneac{\acpc{N}{2}{ac}\,} \def\Nonepc{\acpc{N}{2}{pc}\,}
\def\Done{{\sigl{D}{1}}} \def\Doneac{\acpc{D}{1}{ac}\,} \def\Donepc{\acpc{D}{1}{pc}\,}
\def\Dtwo{{\sigl{D}{2}}} \def\Dtwoac{\acpc{D}{2}{ac}\,} \def\Dtwopc{\acpc{D}{2}{pc}\,}
\def\Uone{{\sigl{U}{1}}} \def\Uoneac{\acpc{U}{1}{ac}\,} \def\Uonepc{\acpc{U}{1}{pc}\,}                 
\def\Utwo{{\sigl{U}{2}}} \def\Utwoac{\acpc{U}{2}{ac}\,} \def\Utwopc{\acpc{U}{2}{pc}\,}

%%%%%%%%%%%%%% Tattvabinduyoga - List of Witnesses   %%%%%%%%%%%%%%%%%%%
\DeclareWitness{ceteri}{\selectlanguage{english}cett.}{ceteri}[]   
\DeclareWitness{E}{\selectlanguage{english}E}{Printed Edition}[]    
\DeclareWitness{P}{\selectlanguage{english}P}{Pune BORI 664}[]  
\DeclareWitness{B}{\selectlanguage{english}B}{Bodleian 485}[]       
\DeclareWitness{N1}{\selectlanguage{english}N\textsubscript{1}}{NGMPP 38/31}[]
\DeclareWitness{N2}{\selectlanguage{english}N\textsubscript{2}}{NGMPP B 38/35}[]
\DeclareWitness{L}{\selectlanguage{english}L}{LALCHAND 5876}[]  
\DeclareWitness{D}{\selectlanguage{english}D}{IGNCA 30019}[] 
%\DeclareWitness{D2}{\selectlanguage{english}D\textsubscript{2}}{IGNCA 30020}[]  
\DeclareWitness{U1}{\selectlanguage{english}U\textsubscript{1}}{SORI 1574}[] 
\DeclareWitness{U2}{\selectlanguage{english}U\textsubscript{2}}{SORI 6082}[]
%%%%%%%%%%%%%% Tattvabinduyoga - Groups of Witnesses   %%%%%%%%%%%%%%%%%%%
\DeclareWitness{X}{\selectlanguage{english}\alpha}{Alpha Group: D,N1,N2,U1}[]
\DeclareWitness{Y}{\selectlanguage{english}\beta}{Beta Group: B,E,L,P,U2}[]
%%%%%%%%%%%%% Testimonia
\DeclareWitness{Ysv}{\selectlanguage{english}Ysv}{Yogasvarodaya}[] %%%add infos!  

%%%%%%%%%%%%%%%%%%%%%%%%%%%%%%%%%%%%%%%%%%%
% Macro for Editing Abbrevs.
\def\om{\textrm{\footnotesize \textit{om.}\ }} %prints om. for omitted in apparatus
\def\korr{\textrm{\footnotesize \textit{em.}\ }} %prints em. for emended in apparatus
\def\conj{\textrm{\footnotesize \textit{conj.}\ }} %prints conj. for conjectured in apparatus

% \supplied{text} EDITORIAL ADDITION -> Within \lem oder \rdg
% \surplus{text} EDITORIAL DELETION -> Within \lem oder \rdg
% \sic{text} CRUX
% \gap{text} LACUNAE -> [reason=??, unit=??, quantity=??, extent=??]


%%%%%%%%%%%%%%%%%%%%%%%%%%%%%%%%%%%%%%%%%%% All macros of this list can be used in 
% Macro for Editing Abbrevs.
\def\eyeskip{\textrm{{ab.\,oc. }}}
\def\aberratio{\textrm{{ab.\,oc. }}}
\def\ad{\textrm{{ad}}}
\def\add{\textrm{{add.\ }}}
\def\ann{\textrm{{ann.\ }}}
\def\ante{\textrm{{ante }}} 
\def\post{\textrm{{post }}}
%\def\ceteri{cett.\,}                   
\def\codd{\textrm{{codd.\ }}}

\def\coni{\textrm{{coni.\ }}}
\def\contin{\textrm{{contin.\ }}}
\def\corr{\textrm{{corr.\ }}}
\def\del{\textrm{{del.\ }}}
\def\dub{\textrm{{ dub.\ }}}

\def\expl{\textrm{{explic.\ }}} 
\def\explica t{\textrm{{explic.\ }}}
\def\fol{\textrm{{fol.\ }}}
\def\foll{\textrm{{foll.\ }}}
\def\gloss{\textrm{{glossa ad }}}
\def\ins{\textrm{{ins.\ }}}      
\def\inseruit{\textrm{{ins.\ }}} 
\def\im{{\kern-.7pt\lower-1ex\hbox{\textrm{\tiny{\emph{i.m.}}}\kern0pt}}} %\textrm{\scriptsize{i.m.\ }}}      
\def\inmargine{{\kern-.7pt\lower-.7ex\hbox{\textrm{\tiny{\emph{i.m.}}}\kern0pt}}}%\textrm{\scriptsize{i.m.\ }}}      
\def\intextu{{\kern-.7pt\lower-.95ex\hbox{\textrm{\tiny{\emph{i.t.}}}\kern0pt}}}%\textrm{\scriptsize{i.t.\ }}}           
\def\indist{\textrm{{indis.\ }}}  
\def\indis{\textrm{{indis.\ }}}
\def\iteravit{\textrm{{iter.\ }}} 
\def\iter{\textrm{{iter.\ }}}
\def\lectio{\textrm{{lect.\ }}}   
\def\lec{\textrm{{lect.\ }}}
\def\leginequit{\textrm{{l.n. }}} 
\def\legn{\textrm{{l.n. }}}
\def\illeg{\textrm{{l.n. }}}

\def\primman{\textrm{{pr.m.}}}
\def\prob{\textrm{{prob.}}}
\def\rep{\textrm{{repetitio }}}
\def\secundamanu{\textrm{\scriptsize{s.m.}}}            \def\secm{{\kern-.6pt\lower-.91ex\hbox{\textrm{\tiny{\emph{s.m.}}}\kern0pt}}}%   \textrm{\scriptsize{s.m.}}}
\def\sequentia{\textrm{{seq.\,inv.\ }}}  
\def\seqinv{\textrm{{seq.\,inv.\ }}}
\def\order{\textrm{{seq.\,inv.\ }}}
\def\supralineam{{\kern-.7pt\lower-.91ex\hbox{\textrm{\tiny{\emph{s.l.}}}\kern0pt}}} %\textrm{\scriptsize{s.l.}}}
\def\interlineam{{\kern-.7pt\lower-.91ex\hbox{\textrm{\tiny{\emph{s.l.}}}\kern0pt}}}   %\textrm{\scriptsize{s.l.}}}
\def\vl{\textrm{v.l.}}   \def\varlec{\textrm{v.l.}} \def\varialectio{\textrm{v.l.}}
\def\vide{\textrm{{cf.\ }}}
\def\cf{\textrm{{cf.\ }}} 
\def\videtur{\textrm{{vid.\,ut}}}
\def\crux{\textup{[\ldots]} }
\def\cruxx{\textup{[\ldots]}}
\def\unm{\textit{unm.}}
%%%%%%%%%%%%%%%%%%%%%%%%%%%%%%%%%%%%

% List of Scholars
\DeclareScholar{ego}{ego}[
forename=Nils Jacob,
surname=Liersch]

% Persons:14\DeclareScholar{ego}{ego}[15forename=Robert,16surname=Alessi]17% Useful shorthands:18\DeclareShorthand{codd}{codd.}{V,I,R,H}19\DeclareShorthand{edd}{edd.}{Lit,Erm,Sm}20\DeclareShorthand{egoscr}{\emph{scripsi}}{ego}

%Useful shorthands:
%\DeclareShorthand{codd}{codd.}{V,I,R,H}
%\DeclareShorthand{edd}{edd.}{Lit,Erm,Sm}
\DeclareShorthand{egoscr}{em.}{ego}
\DeclareShorthand{egoscrconj}{conj.}{ego}
\DeclareShorthand{egomute}{\unskip}{ego}

\usepackage{xparse}

\NewDocumentEnvironment{tlg}{O{}O{}}{\setlength{\leftskip}{0pt}\vspace{-1ex}\begin{quotation}}{\hfill #1\ \vspace{-1ex}\end{quotation}\vspace{-1ex}} %verse environment
%\NewDocumentEnvironment{tlg}{O{}O{}}{\begin{verse}}{॥#1\hskip-4pt ॥\\ \end{verse}}
\NewDocumentCommand{\tl}{m}{{\selectlanguage{iast} #1}}

\NewDocumentCommand{\extra}{m}{{\textcolor{gray}{#1}}} %command for additions to U2
\NewDocumentCommand{\crazy}{m}{{\textcolor{red}{#1}}} %totally corrupted passage
\NewDocumentCommand{\coro}{m}{{\textcolor{violet}{#1}}} %colour for sentence counter! 

\NewDocumentEnvironment{prose}{O{}}{\begin{otherlanguage}{iast}}{\end{otherlanguage}}
% \NewDocumentEnvironment{padd}{O{}}{\begin{otherlanguage}{iast}}{\end{otherlanguage}}
\NewDocumentEnvironment{tlate}{O{}}
%\NewDocumentEnvironment{tadd}{O{}}

%Define two commands: \skp ("sanskrit plus"), to be ignored by TeX in
%the edition text, but processed in the TEI output. Conversely, \skm
%("sanskrit minus") is to be processed in the edition text, but
%ignored if found in the apparatus criticus and in the TEI output:

\NewDocumentCommand{\skp}{m}{}
\TeXtoTEIPat{\skp {#1}}{#1}

%\NewDocumentCommand{\skpp}{m}{}
%\TeXtoTEIPat{\skpp {#1}}{#1}

\NewDocumentCommand{\skm}{m}{\unless\ifinapparatus#1-\fi}
\TeXtoTEIPat{\skm {#1}}{}

% \NewDocumentCommand{\dd}{}{/\hskip-4pt/}
\NewDocumentCommand{\dd}{}{\mbox{/\hskip-4pt/}}
\TeXtoTEIPat{\dd {}}{//}


%%% modify environments and commands
%%% TEI mapping
\TeXtoTEIPat{\begin {tlg}}{<lg>} %lg=(Group of verse (s)) contains one or more verses or lines of verse that together form a formal unit (e.g. stanza, chorus).
\TeXtoTEIPat{\end {tlg}}{</lg>}

\TeXtoTEIPat{\begin {prose}}{<p>}
\TeXtoTEIPat{\end {prose}}{</p>}

\TeXtoTEIPat{\begin {tlate}}{<p>}
\TeXtoTEIPat{\end {tlate}}{</p>}

\TeXtoTEIPat{\\}{}
\TeXtoTEIPat{\linebreak}{<br/>}
\TeXtoTEIPat{\noindent}{}
%\TeXtoTEI{tl}{l}
\TeXtoTEI{emph}{hi}
\TeXtoTEI{bigskip}{}
\TeXtoTEI{None}{N1}
\TeXtoTEI{Ntwo}{N2}
\TeXtoTEI{Done}{D1}
\TeXtoTEI{Dtwo}{D2}
\TeXtoTEI{Uone}{U1}
\TeXtoTEI{Utwo}{U2}
%\TeXtoTEIPat{/}{ |}
%\TeXtoTEI{//}{ ||}
\TeXtoTEIPat{\korr}{em. }
\TeXtoTEIPat{\conj}{conj.}
\TeXtoTEIPat{\om}{om.}
\TeXtoTEIPat{english}{}
\TeXtoTEIPat{\hskip}{}
\TeXtoTEIPat{\hskip-4pt}{}
\TeXtoTEIPat{\hskip-2pt}{}
\TeXtoTEIPat{-}{ }
\TeXtoTEIPat{4pt}{}
\TeXtoTEIPat{2pt}{}
\TeXtoTEIPat{\textcolor {#1}{#2}}{<hi rend="#1">#2</hi>} 

% Nullify \selectlanguage in TEI as it has been used in
% \DeclareWitness but should be ignored in TEI.
\TeXtoTEI{selectlanguage}{}



\author{Nils Jacob Liersch}
\title{Yogatattvabindu of Rāmacandra\\ A Critical and Synoptic Edition and Annotated Translation}
\date{\today}

\parindent=15pt
\begin{document}

% Zitiermöglichkeiten:
%\footcite[See][p.\,1]{goldstein01:_tibet_englis_diction_moder_tibet}
%\footnote{\cite{goldstein01:_tibet_englis_diction_moder_tibet}.}

\frontmatter
\thispagestyle{empty}
\begin{center}
  {\Large \emph{The Yogatattvabindu}}\\[3mm]
\end{center}



\newpage

\

\thispagestyle{empty}



\normalsize


\newpage


\begin{center}
\thispagestyle{empty}

\

\vskip 2mm

\begin{otherlanguage}{iast}
\LARGE \sanskritfont{Yogatattvabindu}
\end{otherlanguage}

\vskip .4cm

\Huge Yogatattvabindu \\[7mm]
\Large Critical and Synoptic \\
Edition with annotated Translation


\large

\vspace{3cm}

Von

Nils Jacob Liersch
\small
\vfill

\vfill

Indica et Tibetica Verlag \\ % $\cdot$ 
Marburg 2024

\vskip 6mm

\end{center}

\newpage
\newpage \ \thispagestyle{empty}
\small  \

\noindent

\
\vfill


\small
\noindent \textbf{Bibliographische Information Der Deutschen Bibliothek}

\noindent
Die Deutsche Bibliothek verzeichnet diese Publikation in der Deutschen Nationalbibliographie;
detaillierte bibliographische Informationen sind im Internet über http://dnb.ddb.de abrufbar.

\noindent
\textbf{Bibliographic information published by Die Deutschen Bibliothek}

\noindent
Die Deutsche Bibliothek lists this publication in the Deutsche Nationalbibliographie; detailed
bibliographic data is available in the Internet at http://dnb.ddb.de.  


\vskip 1cm

\noindent
\copyright\ Indica et Tibetica Verlag, Marburg 2024

\medskip

\noindent
Alle Rechte vorbehalten / All rights reserved

\medskip

\noindent
Ohne ausdrückliche Genehmigung des Verlages ist es nicht gestattet, das Werk oder einzelne Teile
daraus nachzudrucken, zu vervielfältigen oder auf Datenträger zu speichern.

\smallskip

\noindent
Apart from any fair dealing for the purpose of private study, research, criticism or review, no
part of this book may be reproduced or translated in any form, by print, photo form, microfilm, or
any other means without written permission. Enquiries should be made to the publishers.

\bigskip

\noindent
Satz: \ \ Nils Jacob Liersch \\
Herstellung: \ \ BoD – Books on Demand GmbH, Norderstedt  \\

\bigskip

\noindent
%\ISBN     

\normalsize

\newpage

%\maketitle
\clearpage
\tableofcontents
\addtocounter{page}{-1}
\thispagestyle{empty}
\clearpage

\chapter{Introduction}
\mainmatter

\chapter{The List of the 15 Yogas}
\label{yogas_list}
The authenticity of the list specifying the fifteen Yogas at the beginning of the text is ambiguous. This is due to the discrepancy between the structure of the Yogas presented in the text and the order presented in the list. For example, the text commences with a description of \textit{kriyāyoga} and goes on to describe \textit{siddhakuṇḍaliniyoga} and then mentions \textit{mantrayoga} without adhering to the order presented in the list. This incongruity raises questions as to why the text structure deviates from the list. However, the reference to \textit{jñānotpattav upāyaḥ} may provide some insight into why \textit{jñānayoga} is included as the second \textit{yoga} in the list. To reconcile these apparent inconsistencies, there are several possible explanations: 1) The text is severely corrupted. 2) The list was added by a different hand at a later time. 3) The term \textit{jñānayoga} is included as a result of the practice of \textit{siddhakuṇḍalinīyoga}, which is said to generate knowledge through the central channel, as stated in the text. These explanations may be combined to provide a comprehensive understanding of the situation.

\chapter{Sources}
\section{The Additions of  SORI 6082 - U\textsubscript{2}}
\label{discussionu2}
Analyse the additions of U\textsubscript{2} and present the \textit{cakra}s and their attriubutes in a table .
\begin{itemize}
\item  Muktabodha-Texte sehe ich 3 Belege für bahiśśakti Muktabodha/krīyakramādyotikā.html 2938 suṣirānte bahiśśaktiṃ vinyasedvyomarūpiṇīm | tasyā madhye tu Muktabodha/sakalāgamasārasaṅgraha.html 2186 suṣirāntabahiśśaktiṃ vyāpinīṃ cintayet tataḥ || Muktabodha/kriyakramadyotikavyākhyā.html 1846 tanmadhye ca bahiśśaktiṃ sudhābindu parisrutim
  \item  Parā\footnote{Im Kaśm. Śiv. °das ewige Wort, in welchem potentiell alle Begriffe und Worte ruhen; vgl. das śabdabrahma des Vyākaraṇa. [B.]― Schmidt S. 246}.
  \end{itemize}

\chapter{Conventions in the Critical Apparatus}
\section{Sigla in the Critical Apparatus}

\begin{itemize}
\item E : Printed Edition
\item P : Pune BORI 664
\item L : Lalchand Research Library LRL5876
\item B : Bodleian Oxford D 4587
\item \None : NGMPP B 38-31
\item \Ntwo : NGMPP B 38-35 / A 1327-14
\item \Done : IGNCA 30019
\item \Uone : SORI 1574
\item \Utwo: SORI 6082
\end{itemize}

The order of the readings in the critical apparatus is arranged according to the quality of readings in decending order. The critical apparatus is positive. Gemitation is not recorded. 


\section{Marking the Reliability of Sources and Testimonia in the Critical Apparatus}
\label{kennz}

To accurately depict information about the textual relationship and estimated degree of relatedness of a passage from the \textit{Yogatattvabindu} in the layers for sources and testimonia of the critical apparatus, a system of sigla was introduced.\footnote{This type of identification system is based on the use of the critical apparatus in \parencite[lii-liii]{steinkellner2005}. It was modified for the text-critical work on the \textit{Yogatattvabindu}.} The sigla are meaningful when a passage is corrupted in all witnesses and can only be reconstructed by means of other texts. The layers of the critical apparatus for sources and testimonia use the following sigla:

\begin{enumerate}
\item[\textbf{Ce}] \textit{citatum ex alio} / quotation from another (text).\footnote{The sigla \textbf{Ce} indicates an identical or largely identical content in the lesser witness and only allows for minor deviations in the wording of the passage.}
\item[\textbf{Cee}] \textit{citatum ex alio modo edendi} / quotation from another (text) with editorial changes.\footnote{The sigla \textbf{Cee} identifies passages with noticeable deviations in the lesser witness.}
\item[\textbf{Ci}] \textit{citatum in alio} / quotation in another (text).\footnote{The sigla \textbf{Ci} indicates an identical or largely identical content in the lesser witness and only allows for minor deviations in the wording of the passage.}
\item[\textbf{Cie}] \textit{citatum in alio modo edendi} / quotation in another (text) with editorial changes.\footnote{The sigla \textbf{Cie} identifies passages in the lesser witness with noticeable deviations that have the intended character of the composer.}
\item[\textbf{Re}] \textit{relatum ex alio} / (content), attested from another text.\footnote{The sigla \textbf{Re} identifies content parallels in the lesser witness that are relevant to the constitution of the critical text. It further indicates in certain cases that the composer might have used this source when composing his text.}
\item[\textbf{Ri}] \textit{relatum in alio} / (content), attested in another text.\footnote{The sigla \textbf{Ri} identifies content parallels in the lesser witness that are relevant to the constitution of the critical text.}
\end{enumerate}

The following acronyms refer to passages that originated from texts that the author of the \textit{Yogatattvabindu} utilized in compiling his work: \textbf{Ce}, \textbf{Cee}, \textbf{Re}. These texts must predate the \textit{Yogatattvabindu}. The other acronyms, such as \textbf{Ci}, \textbf{Cie}, and \textbf{Ri}, are texts that have adopted passages from the \textit{Yogatattvabindu}, or verses or passages that share similar content with the \textit{Yogatattvabindu}, but their relation is given literally, making it impossible to determine who adopted from whom. \textbf{Re} and \textbf{Ri} each refer to passages that are so closely related in content to those of the \textit{Yogatattvabindu} that they are significant in reconstructing a passage.\footnote{\textbf{Ce} and \textbf{Cee} have the highest degree of reliability, \textbf{Ci} and \textbf{Cie} have a moderate degree, and \textbf{Re} and \textbf{Ri} have the lowest.}

\section{Punctuation}

The inconsistent use of punctuation marks in the available witnesses necessitates standardization. Upon close examination, it appears that punctuation has frequently been dropped or added during the transmission of the texts. The neglect or improper handling of punctuation by the copists has resulted in different versions of lists with and without punctuation. In many instances, missing punctuation has led to the addition of case endings, alteration of the text, and the combination of list items into compound formations that were not present in the original text. Although punctuation plays an important role, deviations in punctuation at the end of sentences, lists, and verse-numbering will only be extensively documented in the critical apparatus of the printed edition. This means that emendations of obvious punctuation mistakes will not be recorded in the critical apparatus. However, the digital edition of this work provides a more detailed documentation of deviations in punctuation through diplomatic transcripts of each witness, and even has a function to display sentences cumulatively.

In the printed edition of the \textit{Yogatattvabindu}, standard conventions of punctuation are followed. In verse poetry, a \textit{daṇḍa} (|) marks the end of a half-verse or half of the \textit{śloka}, and a double \textit{daṇḍa} (||) marks the end of a verse. In prose, a single \textit{daṇḍa} indicates the end of a sentence, and a double \textit{daṇḍa} marks the end of a paragraph. Variations in the use of \textit{avagraha} will be recorded, and items in lists will be separated by a double-\textit{daṇḍa}.

\section{Sandhi}

Among the witnesses we see deviating and inconsistent application of \textit{sandhi}. There is no clear evidence that originally \textit{sandhi} was intentionally not applied. This edition will therefore apply \textit{sandhi} consistently throughout the constituted text to provide a readable text sticking to contemporary conventions in Sanskrit. The variant readings concerning \textit{sandhi} are recorded consistently in the apparatus criticus. This is due to various textcritical problems arising from the inconsistent usage of punctuation which results in application or non-application of \textit{sandhi} wheter the respective witness applied a \textit{daṇḍa} or not. This is particularly the case within lists, which frequently occur in our compilation. Items were most likely originally separated by \textit{daṇḍa}. 


\section{Class Nasals}

Due to inconsistent use of class nasals among the witnesses \textit{anusvāra}s have been substituted with the respective class nasals throughout the edition.

\section{Lists}

Lists are a frequent feature in the \textit{Yogatattvabindu}. The text opens with a list of 15 Yogas and there are many more lists utilized throughout its content. To produce a consistent and easily readable edition, all lists have been identified, normalized to the Nominative Singular or Nominative Plural form of the respective item, or in the case of explanatory lists, to the Ablative Singular or Plural. The items are separated by a double \textit{daṇḍa}. Differences in punctuation and simple punctuation emendations, unless they are text-critically or systematically significant, will not be recorded in the apparatus criticus.
\clearpage

\chapter{Critical Edition \& Annotated Translation}
\clearpage
\newpage
\begin{alignment}[
  texts=edition[class="edition"];
  translation[class="translation"],
  ]
  \begin{edition}
    \ekddiv{type=ed}
    \centerline{\textrm{\small{[2. Adholakṣya]}}}
    \bigskip
    \begin{prose}
      \noindent
%------------------------------
%                            nāsikāyāḥ  upari     dvādaśāṃgulamūlaparyantaṃ dṛṣṭiḥ sthirā karttavyā /   \E
%       athādholakṣaḥ        nāsikāyā   upari     dvādaśāṃgulaparyantaṃ     dṛṣṭiḥ sthirā karttavyā /   \P
%       athādholakṣaḥ //     nāsikāyā   upari     dvādaśāṃgulaparyaṃtaṃ     dṛṣṭiḥ sthirā karttavyā     \L
%       athādholakṣa //      nāsikāyā   upari     dvādaśāṃgulaparyaṃtaṃ     dṛṣṭiḥ sthirā karttavyā     \B
%       atha adholakṣyaḥ //  nāsikāyā   upari     dvādaśaṃgulaparyaṃtaṃ     dṛṣṭiḥ sthirā karttavyā //  \N1
%       atha adholakṣanaḥ // nāsikāyā   upari     dvādaśāṃgulaparyaṃtaṃ     dṛṣṭiḥ sthirā karttavyā //  \N2
%       atha adholakṣaḥ //   nāsikāyā   upari     dvādaśaṃgulaparyaṃtaṃ     dṛṣṭiḥ sthirā karttavyā //  \D
%       atha adholakṣa       nāsikāyā   upari     dvādaśaṃgulaparyaṃtaṃ     dṛṣṭi--sthirā karttavyā     \U1
%                            nāsikāyāḥ  upariṣṭāt    daśāṃgulaparyaṃtaṃ     dṛṣṭiḥ sthirā karttavyā //  \U2
%------------------------------
%Now the downward directed fixation object (\textit{adholakṣya}). One should stabilize the gaze within the circumference (\textit{paryanta}) of twelve \textit{aṅgula}s beyond the nose.
%------------------------------
   \note[type=source, labelb=114, lem={\textbf{Re}}]{YK\textsuperscript{ccn \cdot YSV} 2.4cd-6ab Ed. p. 23: nāsikopari deveśi dvādaśāṅgulamānataḥ ||4|| dṛṣṭisthiran (\textit{dṛṣṭiḥ sthirā} PT\textsuperscript{qcr \cdot YSV} Ed. p. 834) tu karttavyam (\textit{karttavyā} PT\textsuperscript{qcr \cdot YSV} Ed. p. 834) adholakṣam idaṃ bhajet (\textit{bhaja} PT\textsuperscript{qcr \cdot YSV} Ed. p. 834) | tathā ca (\textit{athavā} PT\textsuperscript{qcr \cdot YSV} Ed. p. 834) nāsikāgre tu sthirā dṛṣṭir iyaṃ śṛṇu  (\textit{bhavet} PT\textsuperscript{qcr \cdot YSV} Ed. p. 834) ||5|| yasya bhavet sthirā dṛṣṭiś cirāyuḥ (\textit{sthirā dṛṣṭiś cirāyuḥ syāt tathāsau} PT\textsuperscript{qcr \cdot YSV} Ed. p. 834) sthiradṛṣṭimān |}   
\app{\lem[type=emendation, resp=egoscr]{athādholakṣyaḥ}
  \rdg[wit={N1}]{atha adholakṣyaḥ}
  \rdg[wit={P,L}]{athādholakṣaḥ}
  \rdg[wit={B}]{athādholakṣa}
  \rdg[wit={N2}]{atha adholakṣanaḥ}
  \rdg[wit={D}]{atha adholakṣaḥ}
  \rdg[wit={U1}]{atha adholakṣa}
  \rdg[wit={E,U2}]{\om}}/
\app{\lem[wit={ceteri}]{nāsikāyā}
  \rdg[wit={E,U2}]{nāsikāyāḥ}}
\app{\lem[wit={ceteri}]{upari}
  \rdg[wit={U2}]{upariṣṭāt}}
\app{\lem[wit={ceteri}]{dvādaśāṅgulaparyantaṃ}
  \rdg[wit={E}]{dvādaśāṃgulamūlaparyantaṃ}
  \rdg[wit={U2}]{daśāṃgulaparyaṃtaṃ}}
\app{\lem[wit={ceteri}]{dṛṣṭiḥ}
  \rdg[wit={U1}]{dṛṣṭi°}}
sthirā karttavyā/
%------------------------------
%atha vā nāsikāyā agre dṛṣṭiḥ sthirā karttavyā / \E
%atha vā nāsikāyā agre dṛṣṭiḥ sthirā karttavyā / \P
%\om / \L
%\om / \B
%atha vā nāsikāyā  agre dṛṣṭiḥ sthirā karttavyā // \N1
%atha vā nāsikā    agre dṛṣṭi-sthirā karttavyā      \N2
%atha vā nāsikāyā  agre dṛṣṭiḥ sthirā karttavyā // \D
%atha vā nāśikāyāḥ/ agre dṛṣṭiḥ/ sthirā karttavyā / \U1
%atha vā nāsikāyā  agre dṛṣṭiḥ sthirā karttavyā // \U2
%------------------------------
%Or one should stabilize the gaze onto the tip of the nose.
%------------------------------
\app{\lem[wit={ceteri}]{atha vā}
  \rdg[wit={L,B}]{\om}}
\app{\lem[wit={ceteri}]{nāsikāyā}
  \rdg[wit={U1}]{nāsikāyāḥ}
  \rdg[wit={N2}]{nāsika}}
\app{\lem[wit={ceteri}]{agre}
  \rdg[wit={B,L}]{\om}}
\app{\lem[wit={ceteri}]{dṛṣṭiḥ}
  \rdg[wit={N2}]{dṛṣṭi°}}
\app{\lem[wit={ceteri}]{sthirā}
  \rdg[wit={B,L}]{\om}}
\app{\lem[wit={ceteri}]{karttavyā}
  \rdg[wit={B,L}]{\om}}/ 
%------------------------------
%lakṣadūyasya  dṛḍhīkaraṇāt / dṛṣṭiḥ sthirā bhavati / \E
%lakṣadvayasya dṛṣṭīkaraṇāt / dṛṣṭiḥ sthirā bhavati / \P
%lakṣadvayasya dṛḍhīkaraṇāt   dṛṣṭi--sthiro bhavati / \L
%lakṣadvayasya dṛḍhīkaraṇān---dṛṣṭiḥ sthiro bhavatī   \B
%lakṣadvayasya dṛdhīkaraṇāt   dṛṣṭiḥ sthirā bhavati / \N1
%lakṣadvayasya dṛḍhīkaraṇād---dṛṣṭi--sthirā bhavati / \N2
%lakṣadvayasya dṛḍhīkaraṇāt   dṛṣṭiḥ sthirā bhavati / \D
%lakṣadvayasya dṛḍhīkaraṇāt   dṛṣṭiḥ sthirā bhavati / \U1
%lakṣadvayasya dṛḍhīkaraṇāt   dṛṣṭi--sthirā bhavati // \U2
%------------------------------
%The target becomes stable due to firm exercise [on one] of the twofold aims [of fixation]. 
%------------------------------
\app{\lem[wit={ceteri}]{lakṣadvayasya}   %emend to lakṣyadvayasya??? 
  \rdg[wit={E}]{lakṣadūyasya}} 
\app{\lem[wit={N2}, alt={dṛḍhīkaraṇād}]{dṛḍhīkaraṇā\skm{d-ṛ}}
  \rdg[wit={E,L,N1,D,U1,U2}]{dṛḍhīkaraṇāt}
  \rdg[wit={P}]{dṛṣṭīkaraṇāt}
  \rdg[wit={B}]{dṛḍhīkaraṇān}
}\app{\lem[wit={ceteri}, alt={dṛṣṭiḥ}]{\skp{d-ṛ}ṣṭiḥ}
  \rdg[wit={L,N2,U2}]{dṛṣṭi°}}
\app{\lem[wit={ceteri}]{sthirā}  
  \rdg[wit={B}]{sthiro}
  \rdg[wit={L}]{°sthiro}}
\app{\lem[wit={ceteri}]{bhavati}
  \rdg[wit={B}]{bhavatī}}/
%------------------------------
%pavanaḥ sthiro bhavati / \E
%pavanaḥ sthiro bhavati / \P
%\om                    / \L
%\om                    / \B
%pavanaḥ sthiro bhavati / \N1
%pavana--sthiro bhavati /   \N2
%pavanaḥ sthiro bhavati / \D
%pavana--sthiro bhavati  / \U1
%pavana--sthiro bhavati  / \U2
%------------------------------
%The breath becomes stable. 
%------------------------------
\app{\lem[wit={D,E,P,N1}]{pavanaḥ}
  \rdg[wit={N2,U1,U2}]{pavana°}
  \rdg[wit={B,L}]{\om}}
\app{\lem[wit={ceteri}]{sthiro}
  \rdg[wit={B,L}]{\om}}
\app{\lem[wit={ceteri}]{bhavati}
  \rdg[wit={B,L}]{\om}}/
%------------------------------
%āyurvarddhate / \E
%āyurvarddhate / \P
%āyurvarddhate / \L
%āyurvardhate /  \B
%āyurvardhate /  \N1
%āyurvardhate /  \N2
%āyurvardhate /  \D
%āyurvarddhate   \U1
%āyurvarddhate //  \U2
%------------------------------
%Vitality increases. 
%------------------------------
āyur-varddhate/\vfill
\end{prose}
\nolinenumbers
\bigskip
  \centerline{\textrm{\small{[Statements with reference to other Lakṣyas]}}}
    \bigskip
    \linenumbers
    \begin{prose}
      \noindent
%------------------------------
%etad dūyam       api bāhyalakṣyam eva  bhavati      bāhyāṃtara       ākāśe         śūnyalakṣyaṃ    karttavyaḥ / \E
%etad dvayam      api bāhyalakṣyam eva  bhavati      bāhyābhyaṃtare   ākāśe cet     śūnyalakṣyaṃ    karttavyaḥ / \P
%etad dvayam      api bāhyalakṣam  eva  bhavati//    bāhyābhyaṃtare   ākāśacen      śūnyaṃ lakṣaṃ   karttavyā // \L
%etad dvayadvayam api bāhyalakṣam  eva  bhavatī//    bāhyābhyaṃtare   ākāśacvat     śūnyaṃ lakṣaṃ   karttavyā // \B
%etat advayam     eva bāhyalakṣam  api  kathyate //  bāhyo bhyaṃtaraṃ ākāśavat------śūnyalakṣyaḥ    karttavyaḥ / \N1
%etad dvayam      eva bāhyalakṣam  api  kathyate //  bāhyābhyaṃtaram--ākāśavat------śūnyalakṣaḥ     karttavyaḥ   \N2
%etat advayam     eva bāhyalakṣam  api  kathyate //  bāhyo bhyaṃtaraṃ ākāśavat //   śūnyalakṣyaḥ    karttavyaḥ / \D
%etat dvayam      eva bāhyalakṣyam api  kathyate/    bāhyābhyaṃtare   ākāśavat------śūnyalakṣyaḥ    karttavyaḥ  \U1
%etat dvayam      api bāhyalakṣyam eva  bhavati//    bāhyābhyaṃtare   ākāśe cet     śūnyalakṣyaṃ    karttavyaḥ / \U2
%------------------------------
%This (etad) pair (dvayam) is also called and explained (kathyate) as the outer (bāhya) goal (lakṣyam). The target of emptiness shall be executed internally and externally like space.
%------------------------------
\note[type=source, labelb=115, lem={\textbf{Re}}]{YK\textsuperscript{ccn \cdot YSV} 2.6cd-7 Ed. p. 23: bāhyalakṣaṃ svayaṃ jñeyaṃ yāti tattvanirāsinām (\textit{°nivāsinām} PT\textsuperscript{qcr \cdot YSV} Ed. p. 834) ||6|| kāmināṃ tu bahir dṛṣṭiś cintādiṣu susiddhidā | etad bāhyamadhyalakṣaṃ dṛṣṭicintānirākulaḥ (\textit{iṣṭacintā nirākulam} PT\textsuperscript{qcr \cdot YSV} Ed. p. 834) ||7||}
\app{\lem[wit={L,P,N2},alt={etad dvayam}]{etad-dvaya\skp{m-e}}
  \rdg[wit={E}]{etad dūyam}
  \rdg[wit={B}]{etad dvayadvaya}
  \rdg[wit={D,N2}]{etat advayam}
  \rdg[wit={U1,U2}]{etat dvayam}}\app{\lem[wit={X}, alt={eva}]{\skm{m-e}va}
  \rdg[wit={Y}]{api}} 
\app{\lem[wit={E,P,U1,U2},alt={bāhyalakṣyam}]{bāhyalakṣya\skp{m-a}}
  \rdg[wit={ceteri}]{°lakṣam}}\app{\lem[wit={X},alt={api}]{\skm{m-a}pi}
  \rdg[wit={Y}]{eva}}
\app{\lem[wit={X}]{kathyate}
  \rdg[wit={Y}]{bhavati}
  \rdg[wit={B}]{bhavatī}}/
\app{\lem[wit={N2},alt={bāhyābhyantaram}]{bāhyābhyantara\skm{m-ā}}                %Übersetzung nochmal überdenken! 
  \rdg[wit={D,N1}]{bāhyo bhyaṃtaraṃ}
  \rdg[wit={B,L,P,U1,U2}]{bāhyābhyaṃtare}
  \rdg[wit={E}]{bāhyāṃtara}}\app{\lem[wit={X},alt={ākāśavat}]{\skp{m-ā}kāśavat}
  \rdg[wit={B}]{ākāśacvat}
  \rdg[wit={L}]{ākāśacen}
  \rdg[wit={P,U2}]{ākāśe cet}
  \rdg[wit={E}]{ākāśe}}
\app{\lem[wit={D,N1,U1}]{śūnyalakṣyaḥ}
  \rdg[wit={E,P,U2}]{śūnyalakṣyaṃ}
  \rdg[wit={N2}]{śūnyalakṣaḥ}
  \rdg[wit={B,L}]{śūnyaṃ lakṣaṃ}}
\app{\lem[wit={ceteri}]{karttavyaḥ}
  \rdg[wit={B,L}]{karttavyā}}/
%------------------------------
%jāgraddaśāyāṃ    calanadaśāyāṃ   bhojanadaśāyāṃ   sthitikāle sarvasthāne   śūnyasya dhyānakāraṇāt //                              \E
%jāgraddaśāyāṃ    calanadaśāyāṃ   bhojanaṃ daśāyāṃ sthitikāle sarvasthāne   śūnyasya dhyānakāraṇāt //                              \P
%jāgradādidaśāyāṃ calanadaśāyāṃ// bhojanadaśāyāṃ   sthitikāle sarvasthāneṣu śūnyasya dhyānakāraṇāt //                              \L
%jāgradādidaśāyāṃ calanadaśāyāṃ// bhojanadaśāyāṃ   sthitikāle sarvasthāneṣu śūnyasya dhyānakaraṇāt //                              \B
%jāgraddaśāyāṃ    cakabadaśāyāṃ   bhojanadaśāyāṃ   sthitikāle sarvvasthāne  śūnyasya dhyānakaraṇāt  maraṇatrāso na bhavati//       \N1
%jāyadaśāyāṃ      calanadaśāyāṃ/  bhojanadaśāyāṃ   sthitikāle sarvasthāne   śūnyasya dhyānakaraṇāt  maraṇatrāśo na bhavati//       \N2
%jāgraddaśāyāṃ    calanadaśāyāṃ   bhojanadaśāyāṃ   sthitikāle sarvvasthāne  śūnyasya dhyānakaraṇāt  maraṇatrāso na bhavati// śūnya \D
%jāgraddaśāyāṃ    calanadaśāyāṃ                    sthitikāle sarvasthāne   śūnyasya dhyānakaraṇāt/ maraṇasautrāṃ na bhavati vā    \U1
%jāgṛaddaśāyāṃ    calanadaśāyāṃ   bhojanadaśāyāṃ   sthitikāle sarvasthāne   śūnyasya dhyānakaraṇāt//                               \U2
%------------------------------
%The fear of dying does not arise due to the exercise of meditation on the void at all places during ones life - while eating, moving and waking. 
%------------------------------
\note[type=source, labelb=115a, lem={\textbf{Re}}]{PT\textsuperscript{qcr \cdot YSV} (Ed. p. 834): antarlakṣaṃ śṛṇu śukradigvidigādivarjitam (\textit{subhru°} YK\textsuperscript{ccn \cdot YSV} 2.8a Ed. p. 24) | calaj jāgratsuṣupteṣu bhojaneṣu ca sarvadā | sarvāvasthāsu deveśi cittaṃ śūnye niyojayet | karttā kārayitā śunyaḥ (\textit{śūnyaṃ} YK\textsuperscript{ccn \cdot YSV} 2.10a Ed. p. 24) mūrttimān śūnya īśvaraḥ | harṣaśokaghaṭastho 'yaṃ janmamṛtyū labhet svayam | ghaṭasthā cintyayor mūrttir hatacintāsvarūpadhṛk (\textit{ghaṭasthāṃ cintayen mūrttimitaś} YK\textsuperscript{ccn \cdot YSV} 2.11ab Ed. p. 23) | viṣayaṃ viṣavad duṣṭaṃ (\textit{dṛṣṭvā} YK\textsuperscript{ccn \cdot YSV} 2.11c Ed. p. 23) tyaktvā jñātvā tu mārutam | saṃjñāśūnyamanā bhūtvā puṇyapāpair na lipyate | bāhyam ābhyantaraṃ khaṃ (\textit{yad} YK\textsuperscript{ccn \cdot YSV} 2.12c Ed. p. 24) hi antarlakṣam iti smṛtam | etad dyānāt sadā kiñcid duḥkhaṃ na syāc chivo bhavet | śūnyan tu saccidānandaṃ niḥśabdaṃ brahmaśabditam | saśabdaṃ jñeyam ākāśam (\textit{ākāśa} YK\textsuperscript{ccn \cdot YSV} 2.13c Ed. p. 24) iti bhedadvayan tv iha |}
\app{\lem[wit={ceteri}]{jāgraddaśāyāṃ}
    \rdg[wit={N2}]{jāgṛaddaśāyāṃ}
    \rdg[wit={N2}]{jāyadaśāyāṃ}
    \rdg[wit={B,L}]{jāgradādidaśāyāṃ}}
\app{\lem[wit={ceteri}]{calanadaśāyāṃ}
    \rdg[wit={N1}]{cakabadaśāyāṃ}}
\app{\lem[wit={ceteri}]{bhojanadaśāyāṃ}
    \rdg[wit={P}]{bhojanaṃ daśāyāṃ}
    \rdg[wit={U1}]{\om}}
  sthitikāle
\app{\lem[wit={ceteri}]{sarvasthāne}
    \rdg[wit={B,L}]{sarvasthāneṣu}}
  śūnyasya dhyānakāraṇāt
\app{\lem[wit={D,N1}]{maraṇatrāso}
    \rdg[wit={N2}]{maraṇatrāśo}
    \rdg[wit={U1}]{maraṇasautrāṃ}
    \rdg[wit={B,E,L,P,U2}]{\om}}
\app{\lem[wit={ceteri}]{na}
    \rdg[wit={B,E,P,U2}]{\om}}
\app{\lem[wit={N1,N2}]{bhavati}
    \rdg[wit={D}]{bhavati || śūnya}
    \rdg[wit={U1}]{bhavati vā}
    \rdg[wit={Y}]{\om}}\dd{}
 \end{prose}
  \end{edition}
   \begin{translation}
    \ekddiv{type=trans}
   \bigskip
    \centerline{\textrm{\small{[2. Adholakṣya]}}}
    \bigskip
  \begin{tlate}
    Now the downward-directed target. One should stabilize the gaze within the circumference (\textit{paryanta}) of twelve \textit{aṅgula}s beyond the nose. Alternatively, one should stabilize the gaze onto the tip of the nose. The target becomes stable due to firm exercise [on one] of the twofold targets. The breath becomes stable. Vitality increases.\footnote{In Sundardās's \citetitle{sarvangayoga} 2.26 (Ed. p. 104), one finds the following verse with a very similar concept:
      \begin{quote}
        prathamahīṃ adho lakṣa kauṃ jānaiṃ | nāśā agra dṛṣṭi sthira ānaiṃ |\\
yātoṃ mana pavanā thira hoī | adho lakṣa jo sādhai koī || 26 ||
\end{quote}
 He shares the technique of fixing the gaze onto the nose and the notion of the breath becoming stabilized.} 
\end{tlate}
 %%%%%%%%%%%%%%%%%%%
    %%%%%%%%%%%%%%%%%%%
    %%%%%%%%%%%%%%%%%%%
    %%%%%%%%%%%%%%%%%%%
\ekddiv{type=trans}
   \bigskip
    \centerline{\textrm{\small{[Statements with reference to other Lakṣyas]}}}
    \bigskip
  \begin{tlate}
    This pair is also taught as an external target.\footnote{This short statement associates the two mentioned practices of \textit{adholakṣya} with \textit{bāhyalakṣya}, which is described later on in the text. Indeed, these later descriptions instruct the practitioner to target one of the five elements at various distances in front of the nose or above the head (cf. p. \pageref{bahya2}).} The target of emptiness shall be executed internally and externally like space.\footnote{This description undoubtedly derives from the \textit{Yogasvarodaya} as quoted in the \citetitle{ramatosana} (Ed. p. 834), where it is declared a practice of \textit{antaralakṣ(y)a}.} The fear of dying does not arise due to the exercise of meditation on the void at all places during one's life - while eating, moving and waking.\footnote{Next, Rāmacandra introduces a new topic about the body of the Rājayogin instead of following the expected descriptions on Lakṣyayoga. However, the subject and particularly the descriptions of \textit{bāhya}- \textit{antara}- and \textit{madhyalakṣya} continue later in the text (p. \pageref{bahya2}). One might speculate that an early copyist could have confused folios, but as was noted earlier, Rāmacandra followed the content and structure of \textit{Yogasvarodaya} as quoted in \citetitle{ramatosana}, which presents the exact structure of topics. Judging the structure of the later topics that follow the later \textit{lakṣya}s, rearranging the text would make much more sense! Another noteworthy observation is that in the \textit{Yogasvarodaya} as quoted in \citetitle{ramatosana} \textit{ūrdhvalakṣa}, \textit{adholakṣa}, \textit{bāhyalakṣa} and \textit{madhyalakṣa} are presented rather briefly. After that, \textit{antaralakṣa} is explained in more detail. Rāmacandra follows this template roughly. However, he merges, maybe even confuses? and shortens the descriptions of \textit{bāhya}-, \textit{madhya}- and \textit{antaralakṣ(y)a} into this short description of \textit{bāhyalakṣya}.}\footnote{The concept of five \textit{lakṣ(y)a}s to my current knowledge appears only in the lost \textit{Yogasvarodaya} and from there made its way into \citetitle{ramatosana}, \citetitle{yogakarnika} and \textit{Yogatattvabindu}. The other texts including it are \citetitle{sarvangayoga} and \citetitle{yogacandrika}. In all those texts, this practice becomes a genuine yoga: Lakṣ(y)yoga. If one encounters the concept of three \textit{lakṣ(y)a} as found in \citetitle{ssplonavla} (cf. Ed. pp. 37-41) or various Yoga Upaniṣads, e.g. \citetitle{mandalabrah} (cf. 2.6-2.14) it is never declared as an own type of yoga.} \vspace*{\fill}
    \end{tlate}
  \end{translation}
  \ekdpb*{}
\end{alignment}
%%%%%%%%%%%%%%%%%%%%%%%%%%%%%%%%%%%%%%%%%%
%%%%%%%%%%%%%%%%%%%%%%%%%%%%%%%%%%%%%%%%%%
%%%%%%%%PAGEBREAK%%%%%%%PAGEBREAK%%%%%%%%%
%%%%%%%%%%%%%%%%%%%%%%%%%%%%%%%%%%%%%%%%%%
%%%%%%%%%%%%%%%%PAGEBREAK%%%%%%%%%%%%%%%%%
%%%%%%%%%%%%%%%%%%%%%%%%%%%%%%%%%%%%%%%%%%
%%%%%%%%PAGEBREAK%%%%%%%PAGEBREAK%%%%%%%%%
%%%%%%%%%%%%%%%%%%%%%%%%%%%%%%%%%%%%%%%%%%
%%%%%%%%%%%%%%%%%%%%%%%%%%%%%%%%%%%%%%%%%%
%%%%%%%%%%%%%%%%%%%%%%%%%%%%%%%%%%%%%%%%%%
%%%%%%%%%%%%%%%%%%%%%%%%%%%%%%%%%%%%%%%%%%
%%%%%%%%PAGEBREAK%%%%%%%PAGEBREAK%%%%%%%%%
%%%%%%%%%%%%%%%%%%%%%%%%%%%%%%%%%%%%%%%%%%
%%%%%%%%%%%%%%%%PAGEBREAK%%%%%%%%%%%%%%%%%
%%%%%%%%%%%%%%%%%%%%%%%%%%%%%%%%%%%%%%%%%%
%%%%%%%%PAGEBREAK%%%%%%%PAGEBREAK%%%%%%%%%
%%%%%%%%%%%%%%%%%%%%%%%%%%%%%%%%%%%%%%%%%%
%%%%%%%%%%%%%%%%%%%%%%%%%%%%%%%%%%%%%%%%%%
%%%%%%%%%%%%%%%%%%%%%%%%%%%%%%%%%%%%%%%%%%
%%%%%%%%%%%%%%%%%%%%%%%%%%%%%%%%%%%%%%%%%%
%%%%%%%%PAGEBREAK%%%%%%%PAGEBREAK%%%%%%%%%
%%%%%%%%%%%%%%%%%%%%%%%%%%%%%%%%%%%%%%%%%%
%%%%%%%%%%%%%%%%PAGEBREAK%%%%%%%%%%%%%%%%%
%%%%%%%%%%%%%%%%%%%%%%%%%%%%%%%%%%%%%%%%%%
%%%%%%%%PAGEBREAK%%%%%%%PAGEBREAK%%%%%%%%%
%%%%%%%%%%%%%%%%%%%%%%%%%%%%%%%%%%%%%%%%%%
%%%%%%%%%%%%%%%%%%%%%%%%%%%%%%%%%%%%%%%%%%
\begin{alignment}[
  texts=edition[class="edition"];
  translation[class="translation"],
  ]
  \begin{edition}
    \ekddiv{type=ed}
    \begin{prose}\end{prose}
  \end{edition}
  \begin{translation}
    \ekddiv{type=trans}
    \begin{tlate}\end{tlate}
  \end{translation}
\end{alignment}



\chapter{Bibliography}
 \label{sec:bibli}
   \clearpage
\newpage 
\thispagestyle{empty}
\quad  \addtocounter{page}{-1}

\printbibliography[heading=subbibintoc, title=Consulted Manuskripts, keyword=codex]

\printbibliography[heading=subbibintoc, title=Printed Editions, keyword=printsource]

\printbibliography[heading=subbibintoc, title=Secondary Literature, keyword=seclit]

\printbibliography[heading=subbibintoc, title=Online Sources, keyword=onlinesource]

\end{document}
%%%%%%%%%%%%%%%%%%%%%%%%%%%%%%%%%%%%%%%%%%
%%%%%%%%%%%%%%%%%%%%%%%%%%%%%%%%%%%%%%%%%%
%%%%%%%%END%%%%%%%END%%%%%%%%%%%%%%%%%%%%%
%%%%%%%%%%%%%%%%%%%%%%%%%%%%%%%%%%%%%%%%%%
%%%%%%%%%%%%%%%%END%%%%%%%%%%%%%%%%%%%%%%%
%%%%%%%%%%%%%%%%%%%%%%%%%%%%%%%%%%%%%%%%%%
%%%%%%%%END%%%%%%%END%%%%%%%%%%%%%%%%%%%%%
%%%%%%%%%%%%%%%%%%%%%%%%%%%%%%%%%%%%%%%%%%
%%%%%%%%%%%%%%%%%%%%%%%%%%%%%%%%%%%%%%%%%%
%%%%%%%%%%%%%%%%%%%%%%%%%%%%%%%%%%%%%%%%%%
%%%%%%%%%%%%%%%%%%%%%%%%%%%%%%%%%%%%%%%%%%
%%%%%%%%END%%%%%%%END%%%%%%%%%%%%%%%%%%%%%
%%%%%%%%%%%%%%%%%%%%%%%%%%%%%%%%%%%%%%%%%%
%%%%%%%%%%%%%%%%END%%%%%%%%%%%%%%%%%%%%%%%
%%%%%%%%%%%%%%%%%%%%%%%%%%%%%%%%%%%%%%%%%%
%%%%%%%%END%%%%%%%END%%%%%%%%%%%%%%%%%%%%%
%%%%%%%%%%%%%%%%%%%%%%%%%%%%%%%%%%%%%%%%%%
%%%%%%%%%%%%%%%%%%%%%%%%%%%%%%%%%%%%%%%%%%
%%%%%%%%%%%%%%%%%%%%%%%%%%%%%%%%%%%%%%%%%%
%%%%%%%%%%%%%%%%%%%%%%%%%%%%%%%%%%%%%%%%%%
%%%%%%%%END%%%%%%%END%%%%%%%%%%%%%%%%%%%%%
%%%%%%%%%%%%%%%%%%%%%%%%%%%%%%%%%%%%%%%%%%
%%%%%%%%%%%%%%%%END%%%%%%%%%%%%%%%%%%%%%%%
%%%%%%%%%%%%%%%%%%%%%%%%%%%%%%%%%%%%%%%%%%
%%%%%%%%END%%%%%%%END%%%%%%%%%%%%%%%%%%%%%
%%%%%%%%%%%%%%%%%%%%%%%%%%%%%%%%%%%%%%%%%%
%%%%%%%%%%%%%%%%%%%%%%%%%%%%%%%%%%%%%%%%%%
