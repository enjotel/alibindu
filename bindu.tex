%Ultimatives Tool zur Datierung:
%https://www.cc.kyoto-su.ac.jp/~yanom/pancanga/
%skp = ignored in edition
%skm = ignored in xml
%%%---2-DO---%%%:
% - add xml ids for cladistics
% - produce diplomatic transcripts for saktumiva
% - make extra layer in Apparatus for parallels in SVARODAYA, Siddhasiddhantapaddhati and Amanaska
% - check all daṇḍas!!! now I think that it's more likely that many of them were lost in copies. Lectio difficilior! Very unconventional style of the autor! 
% - read Sarvangayogapradipika, Maya Burger! 
% - maybe add second ciritical edition of yogasvarodaya?!
% - Korrekturlesen von \E!! 
% - Verspattern einbauen!
% - add all Testtimonia of SSP & Ysv
% - Sigla alphabetisch ordnen und! daṇḍas mit einkollationieren
% - präambel auslagern wie Jürgen 
% - grep-search alle Verse!!!!
% - Mss spreadsheet
% - sort N1,D1,B2 zu N1,N2,D1
% - sort all sigla alphabetically 
% - additions to U2: make footnotes for the bahir mātrā-s: explaining the inventions of female deities and tell that this is "schwer interpretierbar"
% - Belege für source und testimonia einfügen!!!
% - GIVE UNIQUE LABELS for TESTIMONIO AND SOurces
% - Edition mit Sätzen übereinander nennt sich: Synoptische Edition
% - Consider changing Lakṣya to Lakṣa
% - vEREINHEITLICHUNG von source und testium! 
%%%%%%%%%%%%%%%%%%%%%%%%%%%%%%%%%%%%%%%%%
% Don't forget
% Siddhasiddhantapaddhati Yogic Body descriptions are followed by Rāmacandra
% Quotes of the Yogasvarodaya in the Yoga Karṇikā
% Rāmacandra more a compiler than an author!!!
% Identify quotes of YTB in Haṭhasanketacandrikā 
%%%%%%%%%%%%%%%%%%%%%%%%%%%%%%%%%%%%%%%%%%%
%MSS notes
%
%--B: i and ī are not differenciated
%--P: no punctuation no daṇdas nothing
%--U1: dot . serves as daṇḍa 
%--\L and \U2 very similar
%--figure out for U2: // ajapājapaḥ sahasra // 6000 //gha 0 16 pa 0 40// \U2?!?!?!?!?!?
%%%%%%%%%%%%%%%%%%%%%%%%%%%%%%%%%%%%%%%%%%
%
% Einleitung Ideen 
% - sprachliche Simplizität
% - Potenzial als Anfängertext
% - Großartige Einführung in die Textkritik -> Synoptische Edition 
% - Gelegenheit Yogasvarodaya und Yogatattvabindu zu edieren 
% - Historische Evidenz entweder für das königliche Leben in einer Maṭha in der Nähe von Benares während der Muslimischen Herrschaft, oder sogar Lehrtext für die Bildung junger Prinzen  
% - eines der raren Beispiele der engen Verknüfung mehrerer Texte 
% - eines der raren Beispiele der Prosaisierung eines metrischen Textes 
% - Anwendung rezenter Technologie! 
% - How the text was construed -> intermingling of Ysv and SSP
% - Martin Straube: "jeder kleine Dorfhäuptling kann Rāja genannt werden". 
%%%%%%%%%%%%%%%%%%%%%%%%%%%%%%%%%%%%%%%%%%%
%Ich habe dieses Zitat gefunden
%Franz
%Franz Veit
%हठयोगः [Printed book page 5-501-c]
%हठयोगः , पुं, (हठेन योगः ।) योगविशेषः ।
%यथा, —
%“इदानीं हठयोगस्तु कथ्यते हठसिद्धिदः ।
%कृत्वासनं पवनाशं शरीरे रोगहारकम् ॥
%पूरकं कुम्भकञ्चैव रेचकं वायुना भजेत् ।
%इत्थं क्रमोत्क्रमं ज्ञात्वा पवनं सग्धयेत् सदा ॥
%धौत्यादिकर्म्मषट्कञ्च संस्कुर्य्याद्धठसाधकः ।
%एतन्नाड्यान्तु देवेशि ! वायुपूर्णं प्रतिष्ठितम् ॥
%ततो मनो निश्चलं स्यात्तत आनन्द एव हि ।
%हठयोगान्न कालः स्यान्मनः शून्ये भवेद्यदि ॥
%इदानीं हठयोगस्य द्वितीयं भेदवत् शृणु ।
%आकाशे नासिकाग्रे तु सूर्य्यकोटिसमं स्मरेत् ॥
%श्वेतं रक्तं तथा पीतं कृष्णमित्यादिरूपतः ।
%एवं ध्यात्वा चिरायुः स्यादङ्गाजननवर्ज्जितः ॥
%शिवतुल्यो महात्मासौ हठयोगप्रसादतः ।
%हठाज्ज्योतिर्म्मयो भूत्वा ह्यन्तरेण शिव भवेत् ।
%अतोऽयं हठयोगः स्यात् सिद्धिदः सिद्धसेवितः ॥”
%इति योगस्वरोदयः ॥ [ID=41348]

%Now, Haṭhayoga indeed is explained as that which gives the siddhi (accomplishment) of haṭha (persistence).
%One performs the wind-eating/serpent āsanam, which removes illness in the body
%and filling – kumbhaka – emptying may distribute the vāyu/wind.
%In this way, while being aware of the progress and regress of the breath one may feed on the wind continually.
%And with the six karmmas, dhauti etc., the Sādhaka of Haṭha may prepare/embellish himself.
%Thus/thereby, in the channel (nāḍī), Oh supreme Goddess, all of the winds (vāyu) are consecrated/placed.
%Then the mind may be unmoved and then bliss it really is.
%Through Haṭhayoga time will be no more, when the mind in emptiness abides.
%
%Now listen to the second disclosure of haṭhayoga:
%In space, on the tip of the nose indeed, one may remember equal to ten million suns,
%the primal forms: white, red, likewise yellow, dark blue.
%Thus meditating/visualizing, one may have a long life, free of the birth of the body,
%Equal to Śiva, this great soul, due to the blessing of Haṭhayoga,
%shall become through persistence (haṭha) a being of light and internally śiva.
%Therefore this Haṭhayoga grants accomplishment – it’s practiced by the Siddhas (accomplished ones).
%Franz
%Im śabdakalpadruma
%Franz
%Franz Veit
%fj.veit@gmail.com
\documentclass[10pt]{memoir}
\setstocksize{220mm}{155mm} 	        
\settrimmedsize{220mm}{155mm}{*}	
\settypeblocksize{170mm}{116mm}{*}	
\setlrmargins{18mm}{*}{*}
\setulmargins{*}{*}{1.2}
%\setlength{\headheight}{5pt}%
\checkandfixthelayout[lines]
\linespread{1.16}
\flushbottom

%%% Hyphenation settings
\usepackage[htt]{hyphenat}
\hyphenation{he-lio-trope opos-sum}
\tracingparagraphs=1
%Hyphenation in Devanāgarī of the edition still missing? Probably this needs to be modified in babel-iast package? 

%%% babel
\usepackage[english]{babel}
\usepackage{babel-iast/babel-iast}

\babelfont[iast]{rm}[Renderer=Harfbuzz, Scale=1.3]{AdishilaSan}%AdishilaSan}
\babelfont[english]{rm}{Adobe Text Pro}

%%% more functionality
\PassOptionsToPackage{hyphens}{url}
\usepackage{hyperref}
\usepackage{pdflscape}
\usepackage{cleveref}
\usepackage{url}
\usepackage{cleveref}
\usepackage{microtype}
\usepackage{lineno}

%\usepackage{bigfoot}
%%% more functions
\usepackage[dvipsnames]{xcolor}
%\usepackage[para,perpage]{footmisc}

%%%für den Counter von Kapiteln und Sätzen! 
\newcommand{\uproman}[1]{\uppercase\expandafter{\romannumeral#1}}
\newcommand{\lowroman}[1]{\romannumeral#1\relax}

\makeindex
\newfontfamily\sanskritfont[Script=Devanagari,Mapping=RomDev,Scale=1.1]{Sanskrit2003}
\usepackage{pifont,fourier-orns,lettrine,psvectorian,paralist,enumitem,pdfpages,wrapfig,tabulary,lettrine,longtable}
\setlist[enumerate]{itemsep=0mm}
\usepackage[autostyle]{csquotes}
\usepackage[defaultlines=2,all]{nowidow}
\usepackage{ellipsis,adforn,booktabs,longtable,url,tikz}
\lineskiplimit=-3pt          

\makechapterstyle{IeT}{%
  \chapterstyle{default}
  \renewcommand*{\printchapternonum}{\centering}
  \renewcommand*{\clearforchapter}{\cleartorecto} 
  \aliaspagestyle{chapter}{empty}}
\chapterstyle{IeT}
\setsecnumdepth{none}  \openright  \nouppercaseheads
\settocdepth{subsubsection}

%%%% test better pagebreaks
%\def\fussy{%
%  \emergencystretch\z@
%  \tolerance 200%
%  \hfuzz .1\p@
%  \vfuzz\hfuzz}

%\interfootnotelinepenalty=10000\relax

%\usepackage[maxfloats=256]{morefloats}

%\maxdeadcycles=500

%raggedbottomsectiontrue
%%\checkandfixthelayout


%%%%%%%  biblatex
%\newcommand{\noun}[1]{\textsc{#1}}    %  philosophy-verbose
\usepackage[backend=biber, sorting=nyt, style=verbose]{biblatex} %%%%ORIGINAL TiE
\renewcommand*{\mkbibnamefamily}[1]{\textsc{#1}}


\DeclareFieldFormat{url}{%
  \mkbibacro{URL}\addcolon\space
  \href{#1}{\nolinkurl{\thefield{urlraw}}}}

\DeclareFieldFormat{citeurl}{%
  \href{#1}{\nolinkurl{\thefield{urlraw}}}} 


\DeclareFieldFormat{postnote}{#1}
\renewcommand{\postnotedelim}{, }
\addbibresource{bindu.bib}

%%% ekdosis
\usepackage[teiexport=tidy,parnotes=true]{ekdosis}% =tidy cleans up HTML and XML documents by fixing markup errors and upgrading legacy code to modern standards. parnotes=footnotes below or above critical apparatus

\SetLineation{lineation=page, modulo} %lineation=page sets thenumbering to start afresh at the top of each page. =modulo makes every fifth line numbered. {lineation=page} makes every line numbered! 

\renewcommand{\linenumberfont}{\selectlanguage{english}\footnotesize} %sets language of lines to English

\SetTEIxmlExport{autopar=false} %autopar=falseinstructs ekdosis to ignore blank lines in the.tex sourcefile as markers for paragraph boundaries. As a result, each paragraph of the edition must be found within an environment associated with the xml <p> element

\SetHooks{
  lemmastyle=\bfseries,
  %refnumstyle=\selectlanguage{english}\bfseries,
  refnumstyle=\selectlanguage{english}\color{blue}\bfseries,
  appheight=0.8\textheight,
}

\newif\ifinapparatus
\DeclareApparatus{source}[
%bhook=\inapparatustrue,
lang=english,
notelang=english,
% bhook=\selectlanguage{english},
bhook=\selectlanguage{english}\textbf{Sources:},%
%maxentries=4, 
%ehook=.]
%sep={] },
%nosep,
]

\newif\ifinapparatus
\DeclareApparatus{testium}[
%bhook=\inapparatustrue,
lang=english,
notelang=english,
% bhook=\selectlanguage{english},
bhook=\selectlanguage{english}\textbf{Testimonia:},
%maxentries=4, 
%ehook=.]
%nosep, 
]

% Declare \ifinapparatus and set \inapparatustrue at the beginning of
% the apparatus criticus block. Also set the language.  
\newif\ifinapparatus
  \DeclareApparatus{default}[
  %bhook=\inapparatustrue, 
  lang=english,
  %maxentries=33,
  %bhook=\selectlanguage{english},
  sep = {] },
  delim=\hskip 0.75em,
  rule=\rule{0.7in}{0.4pt},
]

\newif\ifinapparatus
\DeclareApparatus{philcomm}[
%bhook=\inapparatustrue,
lang=english,
notelang=english,
bhook=\selectlanguage{english}\textbf{Philological Commentary:},
%bhook=\selectlanguage{english},
sep={: },
]

\ekdsetup{
showpagebreaks,
spbmk = \textcolor{blue}{spb},
hpbmk = \textcolor{red}{hpb}
}

%\usepackage{fnpos}
%\makeFNmid
%\makeFNbottom
\usepackage[bottom]{footmisc}
%%%%%%%%%%%%%%%%%%%%%%%%%%%
\makeatletter
\def\blfootnote{\gdef\@thefnmark{}\@footnotetext}
\makeatother
%%%%%%%%%%%%%%%%%%%%%%%%%


% Macros and Definitions for the Print of Sigla
\def\acpc#1#2#3{{#1}\rlap{\textrm{\textsuperscript{#3}}}\textsubscript{\textrm{#2}}\space}
\def\sigl#1#2{{{#1}}\textsubscript{\textrm{#2}}}
\def\None{{\sigl{N}{1}}} \def\Noneac{\acpc{N}{1}{ac}\,} \def\Nonepc{\acpc{N}{1}{pc}\,}
\def\Ntwo{{\sigl{N}{2}}} \def\Noneac{\acpc{N}{2}{ac}\,} \def\Nonepc{\acpc{N}{2}{pc}\,}
\def\Done{{\sigl{D}{1}}} \def\Doneac{\acpc{D}{1}{ac}\,} \def\Donepc{\acpc{D}{1}{pc}\,}
\def\Dtwo{{\sigl{D}{2}}} \def\Dtwoac{\acpc{D}{2}{ac}\,} \def\Dtwopc{\acpc{D}{2}{pc}\,}
\def\Uone{{\sigl{U}{1}}} \def\Uoneac{\acpc{U}{1}{ac}\,} \def\Uonepc{\acpc{U}{1}{pc}\,}                 
\def\Utwo{{\sigl{U}{2}}} \def\Utwoac{\acpc{U}{2}{ac}\,} \def\Utwopc{\acpc{U}{2}{pc}\,}

%%%%%%%%%%%%%% Tattvabinduyoga - List of Witnesses   %%%%%%%%%%%%%%%%%%%
\DeclareWitness{ceteri}{\selectlanguage{english}cett.}{ceteri}[]   
\DeclareWitness{E}{\selectlanguage{english}E}{Printed Edition}[]    
\DeclareWitness{P}{\selectlanguage{english}P}{Pune BORI 664}[]  
\DeclareWitness{B}{\selectlanguage{english}B}{Bodleian 485}[]       
\DeclareWitness{N1}{\selectlanguage{english}N\textsubscript{1}}{NGMPP 38/31}[]
\DeclareWitness{N2}{\selectlanguage{english}N\textsubscript{2}}{NGMPP B 38/35}[]
\DeclareWitness{L}{\selectlanguage{english}L}{LALCHAND 5876}[]  
\DeclareWitness{D}{\selectlanguage{english}D}{IGNCA 30019}[] 
%\DeclareWitness{D2}{\selectlanguage{english}D\textsubscript{2}}{IGNCA 30020}[]  
\DeclareWitness{U1}{\selectlanguage{english}U\textsubscript{1}}{SORI 1574}[] 
\DeclareWitness{U2}{\selectlanguage{english}U\textsubscript{2}}{SORI 6082}[]
%%%%%%%%%%%%%% Tattvabinduyoga - Groups of Witnesses   %%%%%%%%%%%%%%%%%%%
\DeclareWitness{X}{\selectlanguage{english}\alpha}{Alpha Group: D,N1,N2,U1}[]
\DeclareWitness{Y}{\selectlanguage{english}\beta}{Beta Group: B,E,L,P,U2}[]
%%%%%%%%%%%%% Testimonia
\DeclareWitness{Ysv}{\selectlanguage{english}Ysv}{Yogasvarodaya}[] %%%add infos!  

%%%%%%%%%%%%%%%%%%%%%%%%%%%%%%%%%%%%%%%%%%%
% Macro for Editing Abbrevs.
\def\om{\textrm{\footnotesize \textit{om.}\ }} %prints om. for omitted in apparatus
\def\korr{\textrm{\footnotesize \textit{em.}\ }} %prints em. for emended in apparatus
\def\conj{\textrm{\footnotesize \textit{conj.}\ }} %prints conj. for conjectured in apparatus

% \supplied{text} EDITORIAL ADDITION -> Within \lem oder \rdg
% \surplus{text} EDITORIAL DELETION -> Within \lem oder \rdg
% \sic{text} CRUX
% \gap{text} LACUNAE -> [reason=??, unit=??, quantity=??, extent=??]


%%%%%%%%%%%%%%%%%%%%%%%%%%%%%%%%%%%%%%%%%%% All macros of this list can be used in 
% Macro for Editing Abbrevs.
\def\eyeskip{\textrm{{ab.\,oc. }}}
\def\aberratio{\textrm{{ab.\,oc. }}}
\def\ad{\textrm{{ad}}}
\def\add{\textrm{{add.\ }}}
\def\ann{\textrm{{ann.\ }}}
\def\ante{\textrm{{ante }}} 
\def\post{\textrm{{post }}}
%\def\ceteri{cett.\,}                   
\def\codd{\textrm{{codd.\ }}}

\def\coni{\textrm{{coni.\ }}}
\def\contin{\textrm{{contin.\ }}}
\def\corr{\textrm{{corr.\ }}}
\def\del{\textrm{{del.\ }}}
\def\dub{\textrm{{ dub.\ }}}

\def\expl{\textrm{{explic.\ }}} 
\def\explica t{\textrm{{explic.\ }}}
\def\fol{\textrm{{fol.\ }}}
\def\foll{\textrm{{foll.\ }}}
\def\gloss{\textrm{{glossa ad }}}
\def\ins{\textrm{{ins.\ }}}      
\def\inseruit{\textrm{{ins.\ }}} 
\def\im{{\kern-.7pt\lower-1ex\hbox{\textrm{\tiny{\emph{i.m.}}}\kern0pt}}} %\textrm{\scriptsize{i.m.\ }}}      
\def\inmargine{{\kern-.7pt\lower-.7ex\hbox{\textrm{\tiny{\emph{i.m.}}}\kern0pt}}}%\textrm{\scriptsize{i.m.\ }}}      
\def\intextu{{\kern-.7pt\lower-.95ex\hbox{\textrm{\tiny{\emph{i.t.}}}\kern0pt}}}%\textrm{\scriptsize{i.t.\ }}}           
\def\indist{\textrm{{indis.\ }}}  
\def\indis{\textrm{{indis.\ }}}
\def\iteravit{\textrm{{iter.\ }}} 
\def\iter{\textrm{{iter.\ }}}
\def\lectio{\textrm{{lect.\ }}}   
\def\lec{\textrm{{lect.\ }}}
\def\leginequit{\textrm{{l.n. }}} 
\def\legn{\textrm{{l.n. }}}
\def\illeg{\textrm{{l.n. }}}

\def\primman{\textrm{{pr.m.}}}
\def\prob{\textrm{{prob.}}}
\def\rep{\textrm{{repetitio }}}
\def\secundamanu{\textrm{\scriptsize{s.m.}}}            \def\secm{{\kern-.6pt\lower-.91ex\hbox{\textrm{\tiny{\emph{s.m.}}}\kern0pt}}}%   \textrm{\scriptsize{s.m.}}}
\def\sequentia{\textrm{{seq.\,inv.\ }}}  
\def\seqinv{\textrm{{seq.\,inv.\ }}}
\def\order{\textrm{{seq.\,inv.\ }}}
\def\supralineam{{\kern-.7pt\lower-.91ex\hbox{\textrm{\tiny{\emph{s.l.}}}\kern0pt}}} %\textrm{\scriptsize{s.l.}}}
\def\interlineam{{\kern-.7pt\lower-.91ex\hbox{\textrm{\tiny{\emph{s.l.}}}\kern0pt}}}   %\textrm{\scriptsize{s.l.}}}
\def\vl{\textrm{v.l.}}   \def\varlec{\textrm{v.l.}} \def\varialectio{\textrm{v.l.}}
\def\vide{\textrm{{cf.\ }}}
\def\cf{\textrm{{cf.\ }}} 
\def\videtur{\textrm{{vid.\,ut}}}
\def\crux{\textup{[\ldots]} }
\def\cruxx{\textup{[\ldots]}}
\def\unm{\textit{unm.}}
%%%%%%%%%%%%%%%%%%%%%%%%%%%%%%%%%%%%

% List of Scholars
\DeclareScholar{ego}{ego}[
forename=Nils Jacob,
surname=Liersch]

% Persons:14\DeclareScholar{ego}{ego}[15forename=Robert,16surname=Alessi]17% Useful shorthands:18\DeclareShorthand{codd}{codd.}{V,I,R,H}19\DeclareShorthand{edd}{edd.}{Lit,Erm,Sm}20\DeclareShorthand{egoscr}{\emph{scripsi}}{ego}

%Useful shorthands:
%\DeclareShorthand{codd}{codd.}{V,I,R,H}
%\DeclareShorthand{edd}{edd.}{Lit,Erm,Sm}
\DeclareShorthand{egoscr}{em.}{ego}
\DeclareShorthand{egoscrconj}{conj.}{ego}
\DeclareShorthand{egomute}{\unskip}{ego}

\usepackage{xparse}

\NewDocumentEnvironment{tlg}{O{}O{}}{\setlength{\leftskip}{0pt}\vspace{-1ex}\begin{quotation}}{\hfill #1\ \vspace{-1ex}\end{quotation}\vspace{-1ex}} %verse environment
%\NewDocumentEnvironment{tlg}{O{}O{}}{\begin{verse}}{॥#1\hskip-4pt ॥\\ \end{verse}}
\NewDocumentCommand{\tl}{m}{{\selectlanguage{iast} #1}}

\NewDocumentCommand{\extra}{m}{{\textcolor{gray}{#1}}} %command for additions to U2
\NewDocumentCommand{\crazy}{m}{{\textcolor{red}{#1}}} %totally corrupted passage
\NewDocumentCommand{\coro}{m}{{\textcolor{violet}{#1}}} %colour for sentence counter! 

\NewDocumentEnvironment{prose}{O{}}{\begin{otherlanguage}{iast}}{\end{otherlanguage}}
% \NewDocumentEnvironment{padd}{O{}}{\begin{otherlanguage}{iast}}{\end{otherlanguage}}
\NewDocumentEnvironment{tlate}{O{}}
%\NewDocumentEnvironment{tadd}{O{}}

%Define two commands: \skp ("sanskrit plus"), to be ignored by TeX in
%the edition text, but processed in the TEI output. Conversely, \skm
%("sanskrit minus") is to be processed in the edition text, but
%ignored if found in the apparatus criticus and in the TEI output:

\NewDocumentCommand{\skp}{m}{}
\TeXtoTEIPat{\skp {#1}}{#1}

%\NewDocumentCommand{\skpp}{m}{}
%\TeXtoTEIPat{\skpp {#1}}{#1}

\NewDocumentCommand{\skm}{m}{\unless\ifinapparatus#1-\fi}
\TeXtoTEIPat{\skm {#1}}{}

% \NewDocumentCommand{\dd}{}{/\hskip-4pt/}
\NewDocumentCommand{\dd}{}{\mbox{/\hskip-4pt/}}
\TeXtoTEIPat{\dd {}}{//}


%%% modify environments and commands
%%% TEI mapping
\TeXtoTEIPat{\begin {tlg}}{<lg>} %lg=(Group of verse (s)) contains one or more verses or lines of verse that together form a formal unit (e.g. stanza, chorus).
\TeXtoTEIPat{\end {tlg}}{</lg>}

\TeXtoTEIPat{\begin {prose}}{<p>}
\TeXtoTEIPat{\end {prose}}{</p>}

\TeXtoTEIPat{\begin {tlate}}{<p>}
\TeXtoTEIPat{\end {tlate}}{</p>}

\TeXtoTEIPat{\\}{}
\TeXtoTEIPat{\linebreak}{<br/>}
\TeXtoTEIPat{\noindent}{}
%\TeXtoTEI{tl}{l}
\TeXtoTEI{emph}{hi}
\TeXtoTEI{bigskip}{}
\TeXtoTEI{None}{N1}
\TeXtoTEI{Ntwo}{N2}
\TeXtoTEI{Done}{D1}
\TeXtoTEI{Dtwo}{D2}
\TeXtoTEI{Uone}{U1}
\TeXtoTEI{Utwo}{U2}
%\TeXtoTEIPat{/}{ |}
%\TeXtoTEI{//}{ ||}
\TeXtoTEIPat{\korr}{em. }
\TeXtoTEIPat{\conj}{conj.}
\TeXtoTEIPat{\om}{om.}
\TeXtoTEIPat{english}{}
\TeXtoTEIPat{\hskip}{}
\TeXtoTEIPat{\hskip-4pt}{}
\TeXtoTEIPat{\hskip-2pt}{}
\TeXtoTEIPat{-}{ }
\TeXtoTEIPat{4pt}{}
\TeXtoTEIPat{2pt}{}
\TeXtoTEIPat{\textcolor {#1}{#2}}{<hi rend="#1">#2</hi>} 

% Nullify \selectlanguage in TEI as it has been used in
% \DeclareWitness but should be ignored in TEI.
\TeXtoTEI{selectlanguage}{}



\author{Nils Jacob Liersch}
\title{Yogatattvabindu of Rāmacandra\\ A Critical and Synoptic Edition and Annotated Translation}
\date{\today}

\parindent=15pt
\begin{document}

% Zitiermöglichkeiten:
%\footcite[See][p.\,1]{goldstein01:_tibet_englis_diction_moder_tibet}
%\footnote{\cite{goldstein01:_tibet_englis_diction_moder_tibet}.}

\frontmatter
\thispagestyle{empty}
\begin{center}
  {\Large \emph{The Yogatattvabindu}}\\[3mm]
\end{center}



\newpage

\

\thispagestyle{empty}



\normalsize


\newpage


\begin{center}
\thispagestyle{empty}

\

\vskip 2mm

\begin{otherlanguage}{iast}
\LARGE \sanskritfont{Yogatattvabindu}
\end{otherlanguage}

\vskip .4cm

\Huge Yogatattvabindu \\[7mm]
\Large Critical and Synoptic \\
Edition with annotated Translation


\large

\vspace{3cm}

Von

Nils Jacob Liersch
\small
\vfill

\vfill

Indica et Tibetica Verlag \\ % $\cdot$ 
Marburg 2024

\vskip 6mm

\end{center}

\newpage
\newpage \ \thispagestyle{empty}
\small  \

\noindent

\
\vfill


\small
\noindent \textbf{Bibliographische Information Der Deutschen Bibliothek}

\noindent
Die Deutsche Bibliothek verzeichnet diese Publikation in der Deutschen Nationalbibliographie;
detaillierte bibliographische Informationen sind im Internet über http://dnb.ddb.de abrufbar.

\noindent
\textbf{Bibliographic information published by Die Deutschen Bibliothek}

\noindent
Die Deutsche Bibliothek lists this publication in the Deutsche Nationalbibliographie; detailed
bibliographic data is available in the Internet at http://dnb.ddb.de.  


\vskip 1cm

\noindent
\copyright\ Indica et Tibetica Verlag, Marburg 2024

\medskip

\noindent
Alle Rechte vorbehalten / All rights reserved

\medskip

\noindent
Ohne ausdrückliche Genehmigung des Verlages ist es nicht gestattet, das Werk oder einzelne Teile
daraus nachzudrucken, zu vervielfältigen oder auf Datenträger zu speichern.

\smallskip

\noindent
Apart from any fair dealing for the purpose of private study, research, criticism or review, no
part of this book may be reproduced or translated in any form, by print, photo form, microfilm, or
any other means without written permission. Enquiries should be made to the publishers.

\bigskip

\noindent
Satz: \ \ Nils Jacob Liersch \\
Herstellung: \ \ BoD – Books on Demand GmbH, Norderstedt  \\

\bigskip

\noindent
%\ISBN     

\normalsize

\newpage

%\maketitle
\clearpage
\tableofcontents
\addtocounter{page}{-1}
\thispagestyle{empty}
\clearpage

\chapter{Introduction}
\mainmatter

\chapter{The List of the 15 Yogas}
\label{yogas_list}
The authenticity of the list specifying the fifteen Yogas at the beginning of the text is ambiguous. This is due to the discrepancy between the structure of the Yogas presented in the text and the order presented in the list. For example, the text commences with a description of \textit{kriyāyoga} and goes on to describe \textit{siddhakuṇḍaliniyoga} and then mentions \textit{mantrayoga} without adhering to the order presented in the list. This incongruity raises questions as to why the text structure deviates from the list. However, the reference to \textit{jñānotpattav upāyaḥ} may provide some insight into why \textit{jñānayoga} is included as the second \textit{yoga} in the list. To reconcile these apparent inconsistencies, there are several possible explanations: 1) The text is severely corrupted. 2) The list was added by a different hand at a later time. 3) The term \textit{jñānayoga} is included as a result of the practice of \textit{siddhakuṇḍalinīyoga}, which is said to generate knowledge through the central channel, as stated in the text. These explanations may be combined to provide a comprehensive understanding of the situation.

\section{Lakṣyayoga}

\begin{itemize}
\item origin tantric Traditions -> e.g. Netratantra
\item also check Mālinivijayottara 2004 Vasudeva pp. 256-257
\item also \citetitle{birch2013} 2.10 Śāmbhavī Mudrā
  \end{itemize} 

\chapter{Sources}
\section{The Additions of  SORI 6082 - U\textsubscript{2}}
\label{discussionu2}
Analyse the additions of U\textsubscript{2} and present the \textit{cakra}s and their attriubutes in a table .
\begin{itemize}
\item  Muktabodha-Texte sehe ich 3 Belege für bahiśśakti Muktabodha/krīyakramādyotikā.html 2938 suṣirānte bahiśśaktiṃ vinyasedvyomarūpiṇīm | tasyā madhye tu Muktabodha/sakalāgamasārasaṅgraha.html 2186 suṣirāntabahiśśaktiṃ vyāpinīṃ cintayet tataḥ || Muktabodha/kriyakramadyotikavyākhyā.html 1846 tanmadhye ca bahiśśaktiṃ sudhābindu parisrutim
  \item  Parā\footnote{Im Kaśm. Śiv. °das ewige Wort, in welchem potentiell alle Begriffe und Worte ruhen; vgl. das śabdabrahma des Vyākaraṇa. [B.]― Schmidt S. 246}.
  \end{itemize}

\chapter{Conventions in the Critical Apparatus}
\section{Sigla in the Critical Apparatus}

\begin{itemize}
\item E : Printed Edition
\item P : Pune BORI 664
\item L : Lalchand Research Library LRL5876
\item B : Bodleian Oxford D 4587
\item \None : NGMPP B 38-31
\item \Ntwo : NGMPP B 38-35 / A 1327-14
\item \Done : IGNCA 30019
\item \Uone : SORI 1574
\item \Utwo: SORI 6082
\end{itemize}

The order of the readings in the critical apparatus is arranged according to the quality of readings in decending order. The critical apparatus is positive. Gemitation is not recorded. 

\section{Abbreviations}
\begin{itemize}
  \item qcr: quote cum notatio (quoted with reference)
  \end{itemize}

\section{Marking the Reliability of Sources and Testimonia in the Critical Apparatus}
\label{kennz}

To accurately depict information about the textual relationship and estimated degree of relatedness of a passage from the \textit{Yogatattvabindu} in the layers for sources and testimonia of the critical apparatus, a system of sigla was introduced.\footnote{This type of identification system is based on the use of the critical apparatus in \parencite[lii-liii]{steinkellner2005}. It was modified for the text-critical work on the \textit{Yogatattvabindu}.} The sigla are meaningful when a passage is corrupted in all witnesses and can only be reconstructed by means of other texts. The layers of the critical apparatus for sources and testimonia use the following sigla:

\begin{enumerate}
\item[\textbf{Ce}] \textit{citatum ex alio} / quotation from another (text).\footnote{The sigla \textbf{Ce} indicates an identical or largely identical content in the lesser witness and only allows for minor deviations in the wording of the passage.}
\item[\textbf{Cee}] \textit{citatum ex alio modo edendi} / quotation from another (text) with editorial changes.\footnote{The sigla \textbf{Cee} identifies passages with noticeable deviations in the lesser witness.}
\item[\textbf{Ci}] \textit{citatum in alio} / quotation in another (text).\footnote{The sigla \textbf{Ci} indicates an identical or largely identical content in the lesser witness and only allows for minor deviations in the wording of the passage.}
\item[\textbf{Cie}] \textit{citatum in alio modo edendi} / quotation in another (text) with editorial changes.\footnote{The sigla \textbf{Cie} identifies passages in the lesser witness with noticeable deviations that have the intended character of the composer.}
\item[\textbf{Re}] \textit{relatum ex alio} / (content), attested from another text.\footnote{The sigla \textbf{Re} identifies content parallels in the lesser witness that are relevant to the constitution of the critical text. It further indicates in certain cases that the composer might have used this source when composing his text.}
\item[\textbf{Ri}] \textit{relatum in alio} / (content), attested in another text.\footnote{The sigla \textbf{Ri} identifies content parallels in the lesser witness that are relevant to the constitution of the critical text.}
\end{enumerate}

The following acronyms refer to passages that originated from texts that the author of the \textit{Yogatattvabindu} utilized in compiling his work: \textbf{Ce}, \textbf{Cee}, \textbf{Re}. These texts must predate the \textit{Yogatattvabindu}. The other acronyms, such as \textbf{Ci}, \textbf{Cie}, and \textbf{Ri}, are texts that have adopted passages from the \textit{Yogatattvabindu}, or verses or passages that share similar content with the \textit{Yogatattvabindu}, but their relation is given literally, making it impossible to determine who adopted from whom. \textbf{Re} and \textbf{Ri} each refer to passages that are so closely related in content to those of the \textit{Yogatattvabindu} that they are significant in reconstructing a passage.\footnote{\textbf{Ce} and \textbf{Cee} have the highest degree of reliability, \textbf{Ci} and \textbf{Cie} have a moderate degree, and \textbf{Re} and \textbf{Ri} have the lowest.}

\section{Punctuation}

The inconsistent use of punctuation marks in the available witnesses necessitates standardization. Upon close examination, it appears that punctuation has frequently been dropped or added during the transmission of the texts. The neglect or improper handling of punctuation by the copists has resulted in different versions of lists with and without punctuation. In many instances, missing punctuation has led to the addition of case endings, alteration of the text, and the combination of list items into compound formations that were not present in the original text. Although punctuation plays an important role, deviations in punctuation at the end of sentences, lists, and verse-numbering will only be extensively documented in the critical apparatus of the printed edition. This means that emendations of obvious punctuation mistakes will not be recorded in the critical apparatus. However, the digital edition of this work provides a more detailed documentation of deviations in punctuation through diplomatic transcripts of each witness, and even has a function to display sentences cumulatively.

In the printed edition of the \textit{Yogatattvabindu}, standard conventions of punctuation are followed. In verse poetry, a \textit{daṇḍa} (|) marks the end of a half-verse or half of the \textit{śloka}, and a double \textit{daṇḍa} (||) marks the end of a verse. In prose, a single \textit{daṇḍa} indicates the end of a sentence, and a double \textit{daṇḍa} marks the end of a paragraph. Variations in the use of \textit{avagraha} will be recorded, and items in lists will be separated by a double-\textit{daṇḍa}.

\section{Sandhi}

Among the witnesses we see deviating and inconsistent application of \textit{sandhi}. There is no clear evidence that originally \textit{sandhi} was intentionally not applied. This edition will therefore apply \textit{sandhi} consistently throughout the constituted text to provide a readable text sticking to contemporary conventions in Sanskrit. The variant readings concerning \textit{sandhi} are recorded consistently in the apparatus criticus. This is due to various textcritical problems arising from the inconsistent usage of punctuation which results in application or non-application of \textit{sandhi} wheter the respective witness applied a \textit{daṇḍa} or not. This is particularly the case within lists, which frequently occur in our compilation. Items were most likely originally separated by \textit{daṇḍa}. 


\section{Class Nasals}

Due to inconsistent use of class nasals among the witnesses \textit{anusvāra}s have been substituted with the respective class nasals throughout the edition.

\section{Lists}

Lists are a frequent feature in the \textit{Yogatattvabindu}. The text opens with a list of 15 Yogas and there are many more lists utilized throughout its content. To produce a consistent and easily readable edition, all lists have been identified, normalized to the Nominative Singular or Nominative Plural form of the respective item, or in the case of explanatory lists, to the Ablative Singular or Plural. The items are separated by a double \textit{daṇḍa}. Differences in punctuation and simple punctuation emendations, unless they are text-critically or systematically significant, will not be recorded in the apparatus criticus.
\clearpage

\section{Structural Analysis of the Yogatattvabindu}

\chapter{Critical Edition \& Annotated Translation}
\cleardoublepage
\begin{alignment}[
  texts=edition[class="edition"];
  translation[class="translation"],
  ]
  \begin{edition}
    \ekddiv{type=ed}
    \begin{prose}
      \noindent
\note[type=source, labelb=178a, lem={\textbf{Re}}]{PT\textsuperscript{qcr \cdot YSV} (Ed. pp. 838-839): netramadhye kūrmanāmā nimeṣonmeṣakṛdayam | udgāre nāga ākhyātaḥ ūrddhavāyuḥ pracālane | kṛkaraḥ kṣutkaro jñeyo devadatto vijṛmbhaṇe | dhanañcayaḥ saccidākāro mṛtadehaṃ na muñcati | yady api sargakāṇḍe sarvametaduktaṃ tathāpi kāryakāraṇabhāvajñāpanāya punarnirdiṣṭamiti na punaruktam |}
\note[type=source, labelb=179a, lem={\textbf{Ri}}]{SSP 1.67 (Ed. pp. 23-24): kūrmavāyuḥ cakṣuṣor unmeṣakārakaś ca| kṛkalaḥ udgārakaḥ kṣutkārakaś ca | devadatto mukhavijṛmbhakaḥ | dhanañjayo nādaghoṣakah ||1.67|| iti daśavāyvavalokanena piṇḍotpattiḥ naranārīrūpam |}
%-----------------------------
%kūrmavāyur netramadhye tiṣṭhati/ nimeṣonmeṣaṃ karoti/ \E
%kūrmavāyur netramadhye           nimeṣonmeṣaṃ karoti \P
%kūrmavāyoḥ netramadhye           nimeṣonmeṣaṃ karotī/ \B
%kūrmavāyoḥ netramadhye           nimiṣonmeṣaṃ karotī... \L
%kūrmo vāyunetramadhye tiṣṭhati/  unmeṣaṃ nimeṣaṃ karoti/ \N1
%kūrmo vāyunetramadhye tiṣṭhati/  unmeṣaṃ nimeṣaṃ ca karoti// \D
%kūrmo vāyunetramadhye tiṣṭhati/  unmeṣaṃ nimeṣaṃ karoti/ \N2
%\om                                                     \U1
%kūrmavāyur netramadhye           nimiṣonmeṣaṃ karoti//            \U2
%-----------------------------
%The Kūrma vitalwind exists within the eyes. It causes [the] opening and closing [of the eyes]. 
%-----------------------------
\app{\lem[wit={E,P,U2}, alt={kūrmavāyur}]{kūrmavāyu\skp{r-ne}}
  \rdg[wit={B,L}]{kūrmavāyoḥ}
  \rdg[wit={D,N1,N2}]{kūrmo vāyu}
}\skm{r-ne}tramadhye
\app{\lem[wit={D,E,N1,N2}]{tiṣṭhati}
  \rdg[wit={ceteri}]{\om}}/  
\app{\lem[wit={E,P,B,U2}]{nimeṣonmeṣaṃ}
  \rdg[wit={N1,N2}]{unmeṣaṃ nimeṣaṃ}
  \rdg[wit={D}]{unmeṣaṃ nimeṣaṃ ca}}
\app{\lem[wit={ceteri}]{karoti}
  \rdg[wit={B,L}]{karotī}}/
\note[type=philcomm, labelb=179b, lem={\uproman{27}.\textsuperscript{\lowroman{17}-\lowroman{18}}}]{Sentences \om in \getsiglum{U1}.}
%-----------------------------
%kṛkalakartāvāyur  udgāraṃ karoti      \E
%kṛkalavāyur       udhāraṃ karoti      \P
%kṛkalavāyur       udhāraṃ karotī      \B
%kṛkalavāyur       uhāraṃ karotī        \L
%kṛkalavāyor       ūdgāro bhavati//    \N1
%kṛkalavāyor-------ūdgāto bhavati/      \D
%kṛkaravāyor-------ūdgāro bhavati/      \N2
%                                       \U1
%puṣkaravāyur      udgāraṃ karoti//    \U2
%-----------------------------
%From the Kṛkala vitalwind gagging arises. 
%-----------------------------
\app{\lem[wit={D,N1,N2},alt={kṛkalavāyor}]{kṛkalavāyo\skp{r-u}}
  \rdg[wit={B,L,P}]{kṛkalavāyur}
  \rdg[wit={E}]{kṛkalakartāvāyur}
  \rdg[wit={U2}]{puṣkaravāyur}
}\app{\lem[type=emendation, resp=egoscr, alt={udgāro}]{\skm{r-u}dgāro}
  \rdg[wit={E,U2}]{udgāraṃ}
  \rdg[wit={B,P}]{udhāraṃ}
  \rdg[wit={L}]{uhāraṃ}
  \rdg[wit={N1,N2}]{ūdgāro}
  \rdg[wit={D}]{ūdgāto}}
\app{\lem[wit={D,N1,N2}]{bhavati}
  \rdg[wit={E,P,U2}]{karoti}
  \rdg[wit={B,L}]{karotī}}/
%-----------------------------
% devadattavāyoḥ  jṛmbhaṇaṃ bhavati/ dhanaṃjayavāyoḥ śabda utpadyate// \E
% devadattavāyor  jumbhā bhavati     dhanaṃjayavāyo  śabdāḥ utpadyete  \P
% devadattavāyor  jumbhā bhavaṃtī    dhanaṃjayavāyoḥ śabda utpadyate// \B
% devadattavāyor  jṛṃbhā bhavatī     dhanaṃjayavāyoḥ śabdaḥ utpadyate// \L
% devadattavāyor  jṛṃbha utpadyate// dhanaṃjayavāyo  śabda utpadyate// \N1
% devadattavāyor  jṛṃbha utpadyate// dhanaṃjayavāyo  śabda utpadyate// \D
% devadattavāyo   jṛṃbhotpadyate/    dhanaṃjayavāyo  śabdotpadyate// \N2
% devadattavāyor  jaṃbhā utpadyate   dhanaṃjayavāyoḥ sabta utpadyate \U1
% devadattavāyo   jṛṃbhā bhavati//   dhanaṃjayavāyoḥ śabda utpadyate// \U2
%-----------------------------
%From the Devadatta vitalwind jawning arises. From the Dhanaṃjaya vitalwind speech arises. 
%-----------------------------
\app{\lem[wit={ceteri}, alt={devadattavāyor}]{devadattavāyo\skp{r-jṛ}}
  \rdg[wit={E}]{devadattavāyoḥ}
  \rdg[wit={N2,U2}]{devadattavāyo}
}\app{\lem[wit={D,N1,U2},alt={jṛmbha}]{\skm{r-jṛ}mbha}
  \rdg[wit={E}]{jṛmbhaṇaṃ}
  \rdg[wit={B,P}]{jumbhā}
  \rdg[wit={L}]{jṛṃbhā}
  \rdg[wit={N2}]{jṛṃbho°}
  \rdg[wit={U1}]{jaṃbhā}}
\app{\lem[wit={X}]{utpadyate}
  \rdg[wit={E,P,U2}]{bhavati}
  \rdg[wit={B}]{bhavaṃtī}
  \rdg[wit={L}]{bhavatī}}/
\app{\lem[wit={Y}]{dhanaṃjayavāyoḥ}
  \rdg[wit={X}]{dhanaṃjayavāyo}}
\app{\lem[wit={ceteri}]{śabda}
  \rdg[wit={P}]{śabdāḥ}
  \rdg[wit={L}]{śabdaḥ}
  \rdg[wit={N2}]{śabdo°}
  \rdg[wit={U1}]{sabta}}
utpadyate\dd{}\textsuperscript{\begin{otherlanguage}{english}\coro{[\lowroman{20}]}\end{otherlanguage}}\vfill
\end{prose}
\nolinenumbers
\centerline{\textrm{\small{[\uproman{27}.\textsuperscript{\coro{\lowroman{1}-\lowroman{5}}}Madhyalakṣya]}}}
\label{madhyalaksya}
    \bigskip
    \linenumbers
    \begin{prose}
      \noindent
%----------------------------
%\om                               \E
%idānī  madhyalakṣaṃ   kathyate      \P
%idānīṃ madhyalakṣaṇaṃ kathyate//  \B DSCN7164 Z.1
%idānīṃ madhye lakṣaṃ  kathyate//   \L
%idānīṃ madhyalakṣyaṃ  kathyate//   \N1
%idānīṃ madhyalakṣyaṃ  kathyate//   \D
%idānīṃ madhyalakṣaṇaṃ kathyate//  \N2
%idānīṃ madhyalakṣyaṃ  kathyate     \U1
%idānīṃ madhye lakṣyaṃ kathyate//  \U2
%-----------------------------
%Now the central fixation is taught. 
%-----------------------------
\note[type=source, labelb=180, lem={\textbf{Re}}]{PT\textsuperscript{qcr \cdot YSV} (Ed. p. 839): idānīṃ madhyalakṣan tu kathyate siddhikārakam | śvetaṃ raktaṃ tathā pītaṃ dhūmrākāran tu nīlabham |}
\app{\lem[wit={ceteri}]{idānīṃ}
  \rdg[wit={P}]{idānī}}
\app{\lem[wit={D,N1,U1}]{madhyalakṣyaṃ}
  \rdg[wit={B,N2}]{madhyalakṣaṇaṃ}
  \rdg[wit={P}]{madhyalakṣaṃ}
  \rdg[wit={L}]{madhye lakṣaṃ}
  \rdg[wit={U2}]{madhye lakṣyaṃ}}
kathyate/\note[type=philcomm, labelb=180a, lem={\uproman{27}\textsuperscript{\lowroman{1}}}]{Introductory sentence is missing in \getsiglum{E}.}
%-----------------------------SSP. S41!!! almost identical! 
%              aṃtha ca pītavarṇaṃ   raktavarṇaṃ vā dhūmrākāraṃ yan  nīlavarṇaṃ vā   agniśikhāsadṛśaṃ vidyutsamānaṃ   sūryamaṇḍalasadṛśaṃ     arddhacandrasadṛśaṃ jvalad  ākāśasamākāraṃ  \E
%śvetavarṇaṃ   atha     pītavarṇaṃ   raktaṃ vā      dhūmrākāraṃ yan  nīlavarṇaṃ vā   'gniśikhāsadṛśaṃ vidyutsamānaṃ   sūryamaṇdalasadṛśaṃ     arddhacaṃdrasadṛśaṃ jvalad  ākāśasamākāraṃ  \P
%śvetavaraṃ    atha     pītavarṇaṃ// rakta  vā      dhūmrākāraṃ yan  nīlavarṇaṃ vā// agniśikhāsadṛśaṃ vidyutsamānaṃ   sūryamaṇdalasadṛśaṃ/    ūrdhvacaṃdrasadṛśaṃ jvalad  ākāśasamākāraṃ// \B
%śvetavarṇaṃ   atha     pītavarṇaṃ   raktaṃ vā      dhūmrākāraṃ yan  nīlavarṇaṃ vā// agniśikhāsadṛśaṃ vidyutsamāne    sūryamaṇdalasadṛśaṃ//   ardhacaṃdrasadṛśaṃ  jvalad  ākāśasamākāra   \L
%śvetavarṇā/   atha vā  pītavarṇaṃ   raktaṃ vā      dhūmāra     va   nīlavarṇaṃ vā   agniśikhāsadṛśaṃ vidyutsamānaṃ   sūryamaṇdalaṃ sadṛśaṃ/  ūrdhvacaṃdrasadṛśaṃ jvalad  ākāśasamānakāraṃ//  \N1
%śvetavarṇaṃ// atha vā  pītavarṇaṃ   raktaṃ vā      dhūmākāro   vā   nīlavarṇaṃ vā   agniśikhāsadṛśaṃ vidyutsamānaṃ// sūryamaṇdalaṃ sadṛśaṃ// ūrdhvacaṃdrasadṛśaṃ jvalad  ākāśasamānakāraṃ//  \D
%śvetavarṇā    atha vā  pītavarṇa    raktavarṇa     dhūmravarṇa      nīlavarṇaṃ vā   agniśikhāsadṛśaṃ vidyutsamānaṃ   sūryamaṇdalasadṛśaṃ     ūrdhvacaṃdrasadṛśaṃ jvalad  ākāśasamānakāraṃ//  \N2
%svetavarṇaṃ   atha vā  pītavarṇaṃ   raktaṃ vā      dhūmrākāra  van  nīlavarṇaṃ vā   agniśikhāsadṛśaṃ vidyutsamānaṃ   sūryamaṇdalasadṛśaṃ     ārdhacaṃdrasadṛśaṃ  jalad---ā----samānākāraṃ \U1
%svatavarṇaṃ   atha vā  pītavarṇaṃ// raktaṃ vā      dhūmrākāraṃ yan  nīlavarṇaṃ vā   agniśikhāsadṛśaṃ vidyutsamānaṃ   sūryamaṇdalasadṛśaṃ     arddhacaṃdrasadṛśaṃ jvalad--ākāraṃ samākāraṃ \U2
%-----------------------------
%White-colored, or also yellow-colored or red-coloured or smoke-coloured or blue-coloured, like the flame of fire, equal to a lightning, like the orb of the sun, like a half-moon, appearing like flaming space, ...  
%-----------------------------
\note[type=source, labelb=181, lem={\textbf{Re}}]{PT\textsuperscript{qcr \cdot YSV} (Ed. p. 839): agnijvālāsamānābhā vidyutpuñjasamaprabhā | ādityamaṇḍalākāramathavā candramaṇḍalam |}
\note[type=source, labelb=181a, lem={\textbf{Ri}}]{SSP 2.29 (Ed. p. 41): śvetavarṇaṃ vā raktavarṇaṃ vā kṛṣṇavarṇaṃ vā agniśikhākāraṃ vā jyotirūpaṃ vā vidyudākāraṃ sūryamaṇḍalākāraṃ vā arddhacandrākāraṃ vā yatheṣṭasvapiṇḍamātraṃ sthānavarjitaṃ manasā lakṣayet ity anekaviddhaṃ madhyamaṃ lakṣyaṃ |}
\app{\lem[wit={ceteri}, alt={°śveta}]{śveta}
  \rdg[wit={U1}]{sveta°}
  \rdg[wit={U2}]{svata°}
  \rdg[wit={E}]{\om}
}\app{\lem[wit={P,L,U1,U2}, alt={°varṇaṃ}]{varṇaṃ}
  \rdg[wit={D}]{°varṇaṃ ||}
  \rdg[wit={P}]{°varaṃ}
  \rdg[wit={N1}]{°varṇā |}
  \rdg[wit={E}]{\om}}
\app{\lem[wit={ceteri}]{atha}
  \rdg[wit={E}]{aṃtha}}
\app{\lem[wit={ceteri}]{vā}
  \rdg[wit={E}]{ca}
  \rdg[wit={B,L,P}]{\om}}
pīta\app{\lem[wit={ceteri}, alt={°varṇaṃ}]{varṇaṃ}
  \rdg[wit={B,U2}]{°varṇaṃ ||}
  \rdg[wit={N2}]{°varṇa}}
\app{\lem[wit={E}, alt={raktavarṇaṃ}]{raktavarṇaṃ}
  \rdg[wit={N2}]{raktavarṇa}
  \rdg[wit={D,L,N1,U1,U2}]{raktaṃ}
  \rdg[wit={B}]{\om}}
\app{\lem[wit={ceteri}]{vā}
  \rdg[wit={N2}]{\om}}
\app{\lem[type=emendation, resp=egoscr]{dhūmravarṇaṃ}
  \rdg[wit={D}]{dhūmākāro}
  \rdg[wit={N1}]{dhūmāra}
  \rdg[wit={N2}]{dhūmravarṇa}
  \rdg[wit={U1}]{dhūmrākāra}
  \rdg[wit={Y}]{dhūmrākāraṃ}}
%\note[type=philcomm, labelb=182, lem={dhūmra°}]{Due to the repetitive use of the term \textit{varṇaṃ}, both preceding and succeeding the mention of \textit{dhūmra}, as well as its previous occurrence within the same compound, it is highly probable that the original reading was \textit{dhūmravarṇaṃ}.}
\app{\lem[wit={D}]{vā}
  \rdg[wit={N1}]{va}
  \rdg[wit={U1}]{van}
  \rdg[wit={Y}]{yan}
  \rdg[wit={N2}]{\om}}
nīlavarṇaṃ
\app{\lem[wit={ceteri}]{vā}
  \rdg[wit={B,L}]{vā ||}}
\app{\lem[wit={P}, alt={'gni°}]{'gni}
  \rdg[wit={ceteri}]{agni°}
}śikhāsadṛśaṃ vidyut\app{\lem[wit={ceteri},alt={°samānaṃ}]{samānaṃ}
  \rdg[wit={D}]{°samānaṃ ||}
  \rdg[wit={L}]{°samāne}}
sūryamaṇdala\app{\lem[wit={ceteri}, alt={°sadṛśaṃ}]{sadṛśaṃ}
  \rdg[wit={D,N1}]{°ṃ sadṛśaṃ}}
\app{\lem[wit={ceteri},alt={ardha°}]{ardha}
  \rdg[wit={B,D,N1,N2}]{ūrdhva°}
  \rdg[wit={U1}]{ārdha°}
}candrasadṛśaṃ
\app{\lem[wit={ceteri}, alt={jvalad°}]{jvala\skp{d-ā}}
  \rdg[wit={U1}]{jalad}
}\app{\lem[wit={ceteri}, alt={°ākāśa°}]{\skm{d-ā}kāśa}
  \rdg[wit={U1}]{°ā°}
  \rdg[wit={U2}]{°ākāraṃ}
}\app{\lem[wit={ceteri}, alt={°samākāraṃ}]{samākāraṃ}
  \rdg[wit={D,N1,N2,U1}]{°samānakāraṃ}
  \rdg[wit={U2}]{samakāraṃ}
  \rdg[wit={L}]{°samākāra}}/
%-----------------------------
%svaśarīraparimitaṃ      tejomanomadhye tathyaṃ kartavyam// \E
%svaśarīraparimitaṃ      tejomanomadhye lakṣyaṃ karttavyaṃ\P
%svaśarīraparimitaṃ      tejomanomadhye lakṣaṃ kartavyaṃ//  \B
%svaśarīraparimitaṃ      tejomanomadhye lakṣaṃ kartavyaṃ//  \L
%svaśarīraparimitaṃ      tejomanomadhye lakṣyaṃ karttavyaṃ//  \N1
%svaśarīraparimitaṃ      tejomanomadhye lakṣyaṃ karttavyaṃ// \D
%svaśarīraparimitaṃ      tejomanomadhye lakṣaṇaṃ karttavyaṃ//  \N2
%svaśarīraparimanomittaṃ tejomadhye     lakṣyaṃ karttavyaṃ  \U1
%svaśarīraparimitaṃ      tejomanomadhye lakṣaṃ kartavyaṃ//  \U2
%-----------------------------
%measured according to ones own body, the fixation shall be directed onto the center of the glowing mind.  
%-----------------------------
\note[type=source, labelb=183, lem={\textbf{Re}}]{PT\textsuperscript{qcr \cdot YSV} (Ed. p. 839): jvaladākāśatulyaṃvā bhāvayed rūpamātmanaḥ | etaj jyotirmayaṃ dehaṃ manomadhye tu lakṣayet |}
svaśarīrapari\app{\lem[wit={ceteri},alt={°mitaṃ}]{mitaṃ}
  \rdg[wit={U1}]{°manomittaṃ}}
tejo\app{\lem[wit={ceteri},alt={°mano}]{mano}
  \rdg[wit={U1}]{\om}
}madhye
\app{\lem[wit={D,P,N1,U1}]{lakṣyaṃ}
  \rdg[wit={E}]{tathyaṃ}
  \rdg[wit={B,L,U2}]{lakṣaṃ}
  \rdg[wit={N2}]{lakṣaṇaṃ}}
kartavyaṃ/
%-----------------------------
%ekasmin lakṣye    kṛte sati manomadhye sthitasya malasya dāho bhavati/ \E [p.38]%
%etasmil lakṣye    kṛte sati manomadhye sthitasya         dāho bhavati \P
%etasmin lakṣe     kṛte satī manomadhye sthitasya malasya dāho bhavati \B
%etasmil lakṣe     kṛte satī manomadhye sthitasya malasya dāho bhavati... \L
%ekasmin lakṣye    kṛte sati manomadhye sthitasya malasya dāho bhavati/ \N1
%ekasmin lakṣye    kṛte sati manomadhye sthitasya malasya dāho bhavati// \D
%ekasmin lakṣaṇo   kṛte sati manomadhye sthitasya malasya dāho bhavati/ \N2
%etasmin na lakṣye kṛte satī manomadhye sthitasya malasya dāho bhavati \U1 %%%281.jpg
%etasmil lakṣe     kṛte satī manomadhye sthitasya malasya dāho bhavati// \U2
%-----------------------------
%While abiding in this fixation the burning of the impurity in the center of the mind arises. 
%-----------------------------
\note[type=source, labelb=184, lem={\textbf{Re}}]{PT\textsuperscript{qcr \cdot YSV} (Ed. p. 839): eteṣāñ ca kṛte lakṣe nānāduḥkhaṃ praṇaśyati | manas astu malo yāti mahānando bhavet tataḥ |}
\app{\lem[wit={P,L,U2},alt={etasmil}]{etasmi\skp{ll-a}}
  \rdg[wit={U1}]{etasmin}
  \rdg[wit={ceteri}]{ekasmin}
}\app{\lem[wit={ceteri},alt={lakṣye}]{\skm{ll-a}kṣye}
  \rdg[wit={B,L,U2}]{lakṣe}
  \rdg[wit={U1}]{na lakṣye}
  \rdg[wit={N2}]{lakṣaṇo}}
kṛte
\app{\lem[wit={ceteri}]{sati}
  \rdg[wit={B,L,U1,U2}]{satī}}
manomadhye sthitasya
\app{\lem[wit={ceteri}]{malasya}
  \rdg[wit={P}]{\om}}
dāho bhavati/ 
%-----------------------------
%manasaḥ   sattvaguṇaprakāśo   bhavati/     puruṣa ānandamayo bhūtvā tiṣṭhati//   \E 
%manasaḥ   sattvaguṇaḥ prakaṭo bhavati      puruṣa ānandamayo bhūtvā tiṣṭhati    \P   %%%7650.jpg
%manasaḥ// sattvaguṇo  prakaṭo  bhavati//   puruṣa ānandamayo bhūtvā tiṣṭhati// \B
%manasaḥ// sattvaguṇaḥ prakaṭo bhavati      puruṣa ānandamayo bhūtvā tiṣṭhati//  \L
%manasaḥ   sattvaguṇe  prakaṭo  bhavati/    puruṣa ānandamayo bhūtvā tiṣṭhati//    \N1
%manaḥ saḥ sattvaguṇo  prakaṭo  bhavati//   puruṣa ānandamayo bhūtvā tiṣṭhati// \D
%manasaḥ   sattvaguṇo  prakaṭo  bhavati/    puruṣa ānandamayo bhūtvā tiṣṭhati//    \N2
%manasaḥ   sattvaguṇo  prakaṭo  bhavati     puruṣa ānandamayo bhūtvā tiṣṭhati       \U1
%manasaḥ   satvaguṇaprakaśo    bhavati//    puruṣa ānandamayo bhūtvā tiṣṭhati//    \U2 %%414.jpg
%-----------------------------
%The Sattva-quality of the mind becomes revealed. After this has happend the person abides supreme bliss. 
%-----------------------------
mana\app{\lem[wit={ceteri},alt={°saḥ}]{saḥ}
  \rdg[wit={B,L}]{°saḥ ||}
  \rdg[wit={D}]{manaḥ saḥ}}
sattva\app{\lem[wit={B,D,N2,U1},alt={°guṇo}]{guṇo}
  \rdg[wit={N1}]{°guṇe}
  \rdg[wit={E,U2}]{°guṇa°}
  \rdg[wit={P,L}]{°guṇaḥ}}
\app{\lem[wit={ceteri}]{prakaṭo}
  \rdg[wit={E,U2}]{°prakāśo}}
bhavati/ puruṣa ānandamayo bhūtvā tiṣṭhati\dd{}\textsuperscript{\begin{otherlanguage}{english}\coro{[\lowroman{5}]}\end{otherlanguage}}\vfill
\end{prose}
  \end{edition}
  \begin{translation}
    \ekddiv{type=trans}
    \begin{tlate}
\noindent
The Kūrma vital wind exists within the eyes. It causes [the] opening and closing [of the eyes]. From the Kṛkala vital wind gagging arises. From the Devadatta vital wind jawning arises. From the Dhanaṃjaya vital wind speech arises.\textsuperscript{\coro{[\lowroman{20}]}}
\end{tlate}
  \ekddiv{type=trans}
  \bigskip
\centerline{\textrm{\small{[\uproman{26}.\textsuperscript{\coro{\lowroman{1}-\lowroman{5}}}Madhyalakṣya]}}}
\bigskip
\begin{tlate}
Now the central fixation is taught. White-coloured or also yellow-coloured or red-coloured or smoke-coloured or blue-coloured, like the flame of fire, equal to a lightning, like the orb of the sun, like a half-moon, appearing like flaming space, measured according to one's own body, the fixation shall be directed onto the centre of the glowing mind.\footnote{\textit{Śivayogapradīpikā} 4.47cd-48: \begin{quote} śṛṇuṣva madhyalakṣyaṃ ca kathitaṃ pūrvasūribhiḥ || 4.47 \\
 śvetādivarṇanavakhaṇḍasucandrasūryasaudāminīvahniśikhena bimbāt | \\
 jvalan nabho vā sthalahīnam ekaṃ vilakṣayet tat khalu madhyalakṣyam 4.48 ||
 \end{quote}
``(47cd) Hear now the central target taught by the ancient sages. \\
(48) Through the image consisting of rays of fire, lightning, sun, moon, and nine [different] colors such as white, etc., one should fixate on the luminous ether or on that which is locationless. Verily, this is the central goal.''
In the \citetitle{advaya} 7 (Ed. pp. 4-5) one reads: \begin{quote}
atha madhyalakṣyalakṣaṇaṃ prātaścitrādivarṇākhaṇḍasūryacakravat vahnijvālāvalīvat tadvihīnāntarikṣavat paśyati | tadākārākāritayā avatiṣṭhati | tadbhūyodarśanena guṇarahitākāśaṃ bhavati | visphurattārakākāradīpyamānagāḍhatamopamaṃ paramākāśaṃ bhavati | kālānalasamadyotamānaṃ mahākāśaṃ bhavati | sarvotkṛṣṭaparamadyutipradyotamānaṃ tattvākāśaṃ bhavati | koṭisūryaprakāśavaibhavasaṃkāśaṃ sūryākāśaṃ bhavati | evaṃ bāhyābhyantarasthavyomapañcakaṃ tārakalakṣyam | taddarśī vimuktaphalas tādṛgvyomasamāno bhavati | tasmāt tāraka eva lakṣyaṃ amanaskaphalapradaṃ bhavati || 7 ||\end{quote}
\begin{quote}
  Then, the description of Intermediate Introspection: He sees, at daybreak, like the vast disc of the Sun resplendent with variegated and other colours, like a huge conflagration of Fire and like the mid-etherial regions devoid of these. He stands with a form identical with theirs. By seeing them over and over again, there ensues Ether devoid of qualities; there ensues transcendent Ether resembling palpable darkness brought into relief by the splendour of the radiant Tāraka form; there ensues the great Ether shining like the fire of deluge; there ensues the Tattvākāśa (Ether of Verity) effulgent with transcendent lustre excelling; and there ensues the Ether of Sun dazzling with the splendour of a hundred thousand Suns. Thus the five Ethers, external and internal, become visible to the Introspection of a Tāraka-yogin. He who sees it, released from fruits, becomes like such Ether. Hence the Introspectuion of Tāraka alone becomes the bestower of the fruit of non-mindedness.  
(\citeauthor[1938:4]{yogaupaniṣads})
\end{quote}
}
While abiding in the fixation, the burning of the impurity in the centre of the mind arises. The Sattva quality of the mind becomes revealed.\footnote{The generation of the sattvic quality through the practice of \textit{madhyalakṣ(y)a} also appears in \citetitle{sarvangayoga} 3.28: \begin{quote}madhya lakṣa mana madhya bicārai | vapu pramāna koi rūpa nihārai |\\
yāte sātvik upajai āī | madhya lakṣa jo sādhai bhāī || 28 ||\end{quote} ``The central Lakṣa directs the mind to reside at its center, revealing the true form of the body. It produces a sattvic quality in those who practice it.'' (28)} After this has happened, the person abides in supreme bliss.   
\end{tlate}
  \end{translation}
\end{alignment}
\chapter{Bibliography}
 \label{sec:bibli}
   \clearpage
\newpage 
\thispagestyle{empty}
\quad  \addtocounter{page}{-1}

\printbibliography[heading=subbibintoc, title=Consulted Manuskripts, keyword=codex]

\printbibliography[heading=subbibintoc, title=Printed Editions, keyword=printsource]

\printbibliography[heading=subbibintoc, title=Secondary Literature, keyword=seclit]

\printbibliography[heading=subbibintoc, title=Online Sources, keyword=onlinesource]

\end{document}

