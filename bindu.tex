\input{preamble.tex}
\FormatDiv{1}{\begin{center}\Large}{\end{center}}
\FormatDiv{2}{\begin{center}\small}{\end{center}}
\FormatDiv{3}{\bfseries}{.}
\title{Yogatattvabindu of Rāmacandra\\ A Critical Edition and Annotated Translation}
\date{\today}

\parindent=15pt
\begin{document}

% Zitiermöglichkeiten:
%\footcite[See][p.\,1]{goldstein01:_tibet_englis_diction_moder_tibet}
%\footnote{\cite{goldstein01:_tibet_englis_diction_moder_tibet}.}

\frontmatter
\thispagestyle{empty}
\begin{center}
  {\Large \emph{The Yogatattvabindu}}\\[3mm]
\end{center}



\newpage

\

\thispagestyle{empty}



\normalsize


\newpage


\begin{center}
\thispagestyle{empty}

\

\vskip 2mm

\begin{otherlanguage}{iast}
\LARGE \sanskritfont{Yogatattvabindu}
\end{otherlanguage}

\vskip .4cm

\Huge Yogatattvabindu \\[7mm]
\Large Critical Edition\\
with annotated Translation


\large

\vspace{3cm}

Von

Nils Jacob Liersch
\small
\vfill

\vfill

Indica et Tibetica Verlag \\ % $\cdot$ 
Marburg 2024

\vskip 6mm

\end{center}

\newpage
\newpage \ \thispagestyle{empty}
\small  \

\noindent

\
\vfill


\small
\noindent \textbf{Bibliographische Information Der Deutschen Bibliothek}

\noindent
Die Deutsche Bibliothek verzeichnet diese Publikation in der Deutschen Nationalbibliographie;
detaillierte bibliographische Informationen sind im Internet über http://dnb.ddb.de abrufbar.

\noindent
\textbf{Bibliographic information published by Die Deutschen Bibliothek}

\noindent
Die Deutsche Bibliothek lists this publication in the Deutsche Nationalbibliographie; detailed
bibliographic data is available in the Internet at http://dnb.ddb.de.  


\vskip 1cm

\noindent
\copyright\ Indica et Tibetica Verlag, Marburg 2024

\medskip

\noindent
Alle Rechte vorbehalten / All rights reserved

\medskip

\noindent
Ohne ausdrückliche Genehmigung des Verlages ist es nicht gestattet, das Werk oder einzelne Teile
daraus nachzudrucken, zu vervielfältigen oder auf Datenträger zu speichern.

\smallskip

\noindent
Apart from any fair dealing for the purpose of private study, research, criticism or review, no
part of this book may be reproduced or translated in any form, by print, photo form, microfilm, or
any other means without written permission. Enquiries should be made to the publishers.

\bigskip

\noindent
Satz: \ \ Nils Jacob Liersch \\
Herstellung: \ \ BoD – Books on Demand GmbH, Norderstedt  \\

\bigskip

\noindent
%\ISBN     

\normalsize

\newpage

%\maketitle
\clearpage
\tableofcontents
\addtocounter{page}{-1}
\thispagestyle{empty}
\clearpage


\mainmatter

\chapter{Conventions in the Critical Apparatus}
\section{Sigla in the Critical Apparatus}

\begin{itemize}
\item E : Printed Edition
\item P : Pune BORI 664
\item L : Lalchand Research Library LRL5876
\item B : Bodleian Oxford D 4587
\item \None : NGMPP B 38-31
\item \Ntwo : NGMPP B 38-35 / A 1327-14
\item \Done : IGNCA 30019
\item \Uone : SORI 1574
\item \Utwo: SORI 6082
\end{itemize}

\chapter{Critical Edition \& Annotated Translation}
\cleardoublepage
\begin{alignment}[
  texts=edition[class="edition"];
  translation[class="translation"],
  ]
  \begin{edition}
    \ekddiv{
      head={[\uproman{21}. \textbf{jñānayogasya lakṣaṇam}]},
      type=section,
      depth=2, 
      n=XXI
    }
    \xmlhead[h21]{[XXI. jñānayogasya lakṣaṇam]}
    \label{jnanayogastart}
\begin{prose}[p21_01]
%------------------------------
%idānīṃ jñānayogasya lakṣaṇaṃ kathyate/ \E
%idānīṃ jñānayogasya lakṣaṇaṃ kathyate \P
%idānīṃ jñānayogasya lakṣaṇaṃ// \L 5976_0011.jpg 
%idānīṃ jñānayogasya lakṣaṇaṃ// \B
%idānīṃ jñānayogasya lakṣaṇaṃ// \N1 %%%%p.6 verso 
%idānīṃ jñānayogasya lakṣaṇaṃ// \D
%idānīṃ jñānayogasya lakṣaṇaṃ kathyate// \N2
%idānī  jñānayogasya lakṣaṇaṃ kathyate   \U1
%idānīṃ jñānayogasya lakṣaṇaṃ kathyate// \U2
%------------------------------
%Now the characteristic of Jñānayoga is explained. 
%-----------------------------o
\note[type=source, labelb=133, nosep]{cf. YSv (PT p. 835): idānīṃ jñānayogasya lakṣaṇaṃ kathyate śive | yaj jñātvā jñānasampūrṇaḥ śivaḥ syān na punarbhavaḥ |}
\app{\lem[wit={ceteri}]{idānīṃ}
  \rdg[wit={U1}]{idānī}}
jñānayogasya lakṣaṇaṃ
\app{\lem[wit={E,P,N2,U1,U2}]{kathyate}
  \rdg[wit={B,D,L,N1}]{\om}}/
\end{prose}
%--------------------------------------
%ekam eva jagat paśyed viśvāva suvibhāsvaram/
%avikalpatayā yuktyā jñānayogaṃ samācaret//1// \E
%
%ekam eva cayat paśyed viśvātmāsuvibhāsvaram       
%avikalpatayā yuktyā jñānayogaṃ samācaret 1 \P
%
%ekam evā jagat paśyed viśvātmāsuvibhāsvaraṃ//
%avikalpatayā yuktā jñānayogaṃ samācaret// \L
%
%ekam evā jagat paśyad visvātmāsuvibhāsvaraṃ//
%avikalpatayā yuktā jñānayogaṃ samācaret// \B
%
%ekam eva jagat paśyed viśvātmā viśvabhāvanaḥ/
%iti kṛtvā tu vai yukto jñānayogaṃ samācaret// SVARODAYA
%
%ekam eva jagat paśyed dviśvātmāsuvibhāsvaraṃ/
%avikalpatayā yuktyā jñānayogaṃ samācaret//1// \N1
%
%ekam eva jagat paśyed dviśvātmāsuvibhāsvaraṃ//
%avikalpatayā yuktyā jñānayogaṃ samācaret//1// \D
%
%ekam eva jagat paśyed dviśvātmāsuvibhāsvaraṃ//
%avikalpatayā yuktyā jñānayogaṃ samācaret//1// \N2
%
%ekam eva jagataḥ paśyed dviśvātmāsuvibhāsvaraṃ
%āvikalpatayā yuktyā jñānayogaṃ samācaret//1// \U1
%
%ekam eva jagataḥ paśyed dviśvātmāsuvibhāsvaraṃ
%āvikalpatayā yuktyā jñānayogaṃ samācaret// \U2
%------------------------------
%He shall see the world als only 
%------------------------------
\begin{tlg}[21_1] 
  \noindent
  \tl{\note[type=source, labelb=134, labele=_134e, nosep]{ \approx  YSv (PT p. 835): ekam eva jagat paśyed viśvātmā viśvabhāvanaḥ | iti kṛtvā tu vai yukto jñānayogaṃ samācaret |}
eka\skp{m-e}\app{\lem[wit={ceteri}, alt={eva}]{\skm{m-e}va}
  \rdg[wit={B,L}]{evā}}
\app{\lem[wit={ceteri},alt={jagat}]{jaga\skp{t-pa}}
  \rdg[wit={P}]{cayat}
}\app{\lem[wit={ceteri},alt={paśyed}]{\skm{t-pa}śye\skp{d-vi}}
  \rdg[wit={B}]{paśyad}
}\app{\lem[wit={ceteri},alt={viśvātmā°}]{\skm{d-vi}śvātmā}
  \rdg[wit={E}]{viśvāva°}
}suvibhāsvaram/}\\
\tl{\app{\lem[wit={ceteri}]{avikalpatayā}
  \rdg[wit={U1,U2}]{āvikalpatayā}}
\app{\lem[wit={ceteri}]{yuktyā}
  \rdg[wit={B,L}]{yuktā}} 
jñānayogaṃ samācaret\dd{} \begin{otherlanguage}{english}\uproman{21}.1\end{otherlanguage} \dd{}}\linelabel{_134e}
\end{tlg}
%------------------------------
%yatra yatra sthito vāpi sarvajñānamayaṃ jagat/ 
%sa evaṃ vetti bodhena so pi jñānādhikāraṇāt//2// \E 
%
%yatra yatra sthito vāpi sarvajñānamayaṃ jagat  
%ya evaṃ vetti bodhena so pi jñānādhikāravān \P
%
%yatra yatra sthito vāpi sarvajñānamayaṃ jagat//  
%ya evaṃ vetti bodhena so pi jñānādhikāravān// \L
%
%yatra yatra sthito vāpi sarvajñānamayaṃ jagat//  
%ya evaṃ ve bodhena so pi jñānādhikāravān// \B
%
%yatra tatra sthito vāpi sarvajñānamayaṃ jagat/
%ya evam asti bodhena so'pi jñānādhikāravān/ \SVARODAYA
%
%yatra yatra sthito vāpi sarvajñānamayaṃ jagat/
%ya evaṃ vetti bodhena so pi jñānādhikāravān//2//\N1
%
%yatra yatra sthito vāpi sarvajñānamayaṃ jagat//
%ya evaṃ vetti bodhena so pi jñānādhikāravān//2//\D
%
%yatra yatra sthito vāpi sarvajñānamayaṃ jagat//
%ya evaṃ vetti bodhena so pi jñānādhikāravān//2//\N2
%
%yatra yatra sthito vāpi sarvajñānamayaṃ jagat  %%%273.jpg
%evaṃ vette na bodhena so pi jñānādhikāravān 2    \U1
%
%yatra yatra sthito hiṃsa sarvajñānamayaṃ jagat//  
%evaṃ vetti bodhena so pi jñānādhikāravān// 2    \U2
%------------------------------
%Wherever one dwells, the world is essentially (\textit{vāpi}) made of all knowledge. He who grasps this in this way, even possesses ultimate knowledge through [this] realisation.
%------------------------------
\begin{tlg}[21_2]
  \noindent
  \tl{\note[type=source, labelb=135, labele=_135e, nosep]{ \approx  YSv (PT p. 835): yatra tatra sthito vāpi sarvajñānamayaṃ jagat | ya evam asti bodhena so'pi jñānādhikāravān |}
    %\note[type=source, labelb=135, labele=_135ey, nosep]{ \approx Cf. \textit{Netratantra} 8.55cd: yatra yatra sthito vāpi yena yena vratena vā |}
    yatra tatra sthito \app{\lem[wit={ceteri}]{vāpi}
      \rdg[wit={U2}]{hiṃsa°}} sarvajñānamayaṃ jagat/}\linelabel{_135ey}\\
  \tl{\app{\lem[wit={ceteri}]{ya evaṃ}
      \rdg[wit={U1,U2}]{evaṃ}}
    \app{\lem[wit={ceteri}]{vetti}
      \rdg[wit={U1}]{vette na}
      \rdg[wit={B}]{ve}} bodhena so'pi
    \app{\lem[wit={ceteri}]{jñānādhikāravān}
      %\rdg[wit={E}]{jñānādhikāraṇāt}}\dd{} \begin{otherlanguage}{english}\uproman{21}.2\end{otherlanguage}\hskip-2pt \dd{}}\linelabel{_135e}
      \rdg[wit={E}]{jñānādhikāraṇāt}}\dd{} \begin{otherlanguage}{english}\uproman{21}.2\end{otherlanguage} \dd{}}\linelabel{_135e}
\end{tlg}
%------------------------------
%
%\om!!!!!                                                                                                        \E
%
%prāpnoti śāmbhavīmantrān  sadā nityaparāyaṇaḥ/   yathā nyagrodhavījaṃ hi kṣitau   vaptur drumāyate/               \SVARODAYA  
%prāpnoti śāmbhavīṃ sattāṃ sadāṃdvaitaparāyaṇaḥ   yathā nyagrodhabījaṃ hi kṣitāv   uptaṃ drumāyate likāṃ pa..vāḥ 4 \P  7640.jpg last line check word!!!
%prāpnoti śāmbhavīṃ sattān sadādvaitaparāyaṇaḥ//  yathā nyagrodhavīja  hi kṣitāv   utpadyate yathā//               \L
%prāpnoti śāmbhaviṃ sattāṃ sadādvaitaparāyaṇaḥ//  yathā nyagrodhabījāṃ hi kṣitī    utpadyate//                      \B
%prāpnoti sāṃbhavīṃ satta  sadādvaitaparāyaṇaḥ//  yathā nyagrodhavījaṃ hi kṣitāv   uptaṃ drumāyate 3//              \N1
%prāpnoti sāṃbhavīsattāṃ   sadādvaitaparāyaṇaḥ//  yathā nyagrodhavījaṃ hi kṣitāv   uptaṃ drumāyate//                \D
%prāpnoti sāṃbhavīsattā    sadādvaitaparāyaṇaḥ//  yathā nyagrodhavījaṃ hi kṣitāv   uptaṃ drumāyate//                \N2 %drumaayate=denom. wie ein beim  sein 
%prāpnoti sāṃbhavīsattāṃ   sadādvaitaparāyaṇaḥ    yathā nyagrodhabījaṃ hi kṣitāptā ukta drumāyate 3              \U1
%prāpnoti sāṃbhavīsattāṃ   yadādvaitaparāyaṇaḥ//  yathā nyagrodhabījaṃ hi kṣitāv   uptaṃ drumāyate//               \U2
%------------------------------
%He always attains the reality of śāmbhavī - the supreme goal of non-duality.  
%Just as the seed of the Nyagrodha scattered onto the soil [always] becomes a tree.
%------------------------------
\begin{tlg}[21_3]
  \noindent
  \tl{\note[type=source, labelb=136, labele=_136e, nosep]{ \approx  YSv (PT p. 835): prāpnoti śāmbhavīmantrān sadā nityaparāyaṇaḥ | yathā nyagrodhavījaṃ hi kṣitau vaptur drumāyate |}
    \app{\lem[wit={ceteri}]{prāpnoti}
      \rdg[wit={E}]{\om}}
  \app{\lem[type=emendation, resp=egoscr]{śāṃbhavīsattāṃ}
      \rdg[wit={D,U1,U2}]{sāṃbhavīsattāṃ}
      \rdg[wit={B,P}]{śāmbhavīṃ sattāṃ}
      \rdg[wit={L}]{śāmbhavīṃ sattān}
      \rdg[wit={N1}]{sāṃbhavīṃ satta}
      \rdg[wit={N2}]{sāṃbhavīsattā}
      \rdg[wit={E}]{\om}}
    \app{\lem[wit={ceteri},alt={sadādvaita°}]{sadādvaita}
      \rdg[wit={U1}]{sadāṃdvaita°}
      \rdg[wit={E}]{\om}}parāyaṇaḥ/}\\
  \tl{\app{\lem[wit={ceteri}]{yathā}
      \rdg[wit={E}]{\om}}
    \app{\lem[wit={ceteri}]{nyagrodhabījaṃ}
      \rdg[wit={D,N1,N2}]{nyagrodhavījaṃ}
      \rdg[wit={L}]{nyagrodhavīja}
      \rdg[wit={E}]{\om}}
    \app{\lem[wit={ceteri}]{hi}
      \rdg[wit={E}]{\om}}
    \app{\lem[wit={ceteri},alt={kṣitāv}]{kṣitā\skp{v-u}}
      \rdg[wit={B}]{kṣitī}
      \rdg[wit={U1}]{kṣitāptā}
      \rdg[wit={E}]{\om}
 }\app{\lem[wit={ceteri},alt={uptaṃ drumāyate}]{\skm{v-u}ptaṃ drumāyate}
      \rdg[wit={P}]{uptaṃ drumāyate likāṃ pa..vāḥ}
      \rdg[wit={L}]{utpadyate yathā}
      \rdg[wit={B}]{utpadyate}
      \rdg[wit={U1}]{ukta drumāyate}
      \rdg[wit={E}]{\om}}\dd{} \begin{otherlanguage}{english}\uproman{21}.3\end{otherlanguage} \dd{}}\linelabel{_136e}
\end{tlg}
%------------------------------
%ekāntaṃ  naikadā  svena   dṛśyate  daśadhā  kṛtaḥ/  mūlāṅkurasya  coddaṇḍāḥ śākhākuṇḍalapallavāḥ//3//   \E cod?v%on cud? Wurzel in guṇa + daṇḍa? !!! em. zu śaśvadhā = immer wieder, jederzeit 
%\om                                                                                                    \P
%ekāṃte   nekadhā  svena   dṛśyaṃte daśadhāt kṛp?tā/ mūlāṃkurutva kudaṃḍaḥ  śākhākilekālapallavā        \B
%ekāṃte   nekadhā  svena   dṛśyaṃte daśadhāt kṛtaḥ/  mūlāṃkurutva kudaṃḍa   śākhākalikālapallavā        \L
%ekāṃtaṃ  naikadhā śveta   dṛśyate  daśadhā  kṛtā//  mūlāṃkurutva codaṃḍaḥ  śāvārakumbhalapallavaḥ//4// \N1   
%ekāṃtaṃ  naikadhā śvetana dṛśyate  daśadhā  kṛtā//  mūlāṃkurutva codarāṭaḥ śālavākumapadṛtravā//4//    \D
%ekāṃtaṃ  naikadhā śvetana dṛśyet   śadhā    kṛtā//  mūlāṃkurutva codarāṭaḥ śākhākumbhalapallavā//4//   \N2
%yekāṃtaṃ naikadhā svena   dṛśyate  śadhā    kṛtā    mūlāṃkurutva codaṃḍa   śākhākumbhalapallavaḥ       \U1
%ekāṃtaṃ  naikadhā svetana dṛśyate  daśadhā  kṛtiḥ// mūlāṃkurutva codaṃḍaḥ  śākhākusumapallavāḥ//       \U2
%------------------------------
\begin{tlg}[21_4]
  \noindent
  \tl{\note[type=source, labelb=137, labele=_137e, nosep]{ \approx  YSv (PT p. 835): ādāv ekas tato 'nekaḥ svabhāvāc chādanādibhiḥ | varddhate 'harniśaṃ vṛkṣaḥ patrapallavavistṛtaḥ |}
 \app{\lem[wit={ceteri}]{ekāntaṃ}
  \rdg[wit={B,L}]{ekānte}
  \rdg[wit={U1}]{yekāṃtaṃ}
  \rdg[wit={P}]{\om}}
\app{\lem[wit={ceteri}]{naikadhā}
  \rdg[wit={E}]{naikadā}
  \rdg[wit={B,L}]{nekadhā}
  \rdg[wit={P}]{\om}}
\app{\lem[wit={ceteri}]{svena}
  \rdg[wit={N1}]{śveta}
  \rdg[wit={D,N2}]{śvetana}
\rdg[wit={P}]{\om}}
\app{\lem[wit={ceteri}]{dṛśyate}
  \rdg[wit={B,L}]{dṛśyaṃte}
  \rdg[wit={N2}]{dṛśyet}
\rdg[wit={P}]{\om}}
\app{\lem[wit={E,N1,N2}]{daśadhā}  
  \rdg[wit={B,L}]{daśadhāt} 
  \rdg[wit={N2,U1}]{śadhā}
\rdg[wit={P}]{\om}}
\app{\lem[wit={X}]{kṛtā}
  \rdg[wit={E,L}]{kṛtaḥ}
  \rdg[wit={B}]{kṛptā}
  \rdg[wit={U2}]{kṛtiḥ}
\rdg[wit={P}]{\om}}/}\\
 \tl{\app{\lem[wit={E}]{mūlāṅkurasya}
     \rdg[wit={ceteri}]{mūlāṃkurutva}
   \rdg[wit={P}]{\om}}
\app{\lem[wit={E,N1,U2}]{coddaṇḍāḥ}
  \rdg[wit={D,N2}]{codarāṭaḥ}
  \rdg[wit={B}]{kudaṃjaḥ}
  \rdg[wit={L}]{kudaṃḍa}
\rdg[wit={P}]{\om}}
\app{\lem[wit={U2}]{śākhākusumapallavāḥ}
  \rdg[wit={E}]{śākhākuṇḍalapallavāḥ}
  \rdg[wit={B,L}]{śākhākilekālapallavā}
  \rdg[wit={N1,U1}]{śāvārakumbhalapallavaḥ}
  \rdg[wit={N2}]{śākhākumbhalapallavā}
  \rdg[wit={D}]{śālavākumapadṛtravā}
\rdg[wit={P}]{\om}}\dd{} \begin{otherlanguage}{english}\uproman{21}.4\end{otherlanguage} \dd{}}\linelabel{_137e}
\end{tlg}
%------------------------------
%srehapuṇyaphalaṃ   bīje vistaro yaṃ svabhāvataḥ/  tathāsau   nirmalo  nityo nirvikāro niraṃjanaḥ//4// \E
%snehapuṣpaphalaṃ   bīje vistāro yaṃ svabhāvataḥ   tāthāpasau nirmalau nityo nirvikāro niraṃjanaḥ     \P   %%7641.jpg Z.1
%snehe puṣpaphala---bīja-vistāro ya  svabhāvatāḥ   yāthāsau   nirmalo  nityo nirvikāro niraṃjanaḥ//    \B
%snehe puṣpaphala---bīja-vistāro ya  svabhāvatāḥ// tāthāsau   nirmalo  nityo nirvikāro niraṃjanaḥ//    \L
%snehapuṣpaphalaṃ   bīje vistārā yaṃ svabhāvataḥ/  tathāsau   nirmalo  nityo nirvikāro niraṃjanaḥ//5// \N1
%snehapuṣpaphalaṃ   bīje vistārā yasya  bhāvataḥ// tathāsau   nirmalo  nityo nirvikāro niraṃjanaḥ//5// \D
%snehapuṣpaphalaṃ   vīje vistāro yaṃ svabhāvataḥ// tathāsau   nirmalo  nityo nirvikāro niraṃjanaḥ//5// \N2
%snehapuṣpaṃ phalaṃ bīje vistāro yaḥ svabhāvataḥ   tathāsau   nirmalo  nityo nirvikāro niraṃjanaḥ 5 \U1  %%%%274.jpg
%snehapuṣpaphalaṃ   bīje vistāro yaṃ svabhāvataḥ// tathāsau   nirmalo  nityo nirvikāro niraṃjanaḥ// 5  \U2 %%%first Śloka in this series that is numbered in U2 
%------------------------------
%Aufgrund seines inhärenten Wesens ist dieser Ast mit seinen Zweigen, welcher die Frucht der Blüte der Liebe ist, im Samen.
%Gewiss, ist jenes rein, ewig, unveränderlich und makellos. 
%------------------------------
%By virtue of its inherent nature, this branch with its branches, which is the fruit of the flower of love, is in the seed.
%Certainly, that is pure, eternal, unchanging and immaculate.
%------------------------------
    \begin{tlg}[21_5]
\noindent
      \tl{\note[type=source, labelb=138, labele=_138e, nosep]{ \approx  YSv (PT p. 836): snehapuṣpaphalair vījair vistāro 'yaṃ svabhāvataḥ | tathāsau nirmalo nityo nirvikāro nirañjanaḥ |}
    \app{\lem[wit={D,N1,N2,P,U2}]{snehapuṣpaphalaṃ}
  \rdg[wit={B,L}]{snehe puṣpaphala°}
  \rdg[wit={U1}]{snehapuṣpaṃ phala}
  \rdg[wit={E}]{srehapuṇyaphalaṃ}}
\app{\lem[wit={ceteri}]{bīje}
  \rdg[wit={B,L}]{bīja}}
\app{\lem[wit={ceteri}]{vistāro}
  \rdg[wit={D,N1}]{vistārā}
}\app{\lem[wit={E,P,N1,N2,U2}]{'yaṃ}
  \rdg[wit={B,L}]{ya}
  \rdg[wit={U1}]{yaḥ}
  \rdg[wit={D}]{yasya}}
\app{\lem[wit={ceteri}]{svabhāvataḥ}
  \rdg[wit={B,L}]{svabhāvatāḥ}
  \rdg[wit={D}]{bhāvataḥ}}/}\\
\tl{\app{\lem[wit={ceteri}]{tathāsau}
    \rdg[wit={B}]{yathāsau}
    \rdg[wit={P}]{tathāpasau}}
  \app{\lem[wit={ceteri}]{nirmalo}
    \rdg[wit={P}]{nirmalau}}
nityo nirvikāro nirañjanaḥ\dd{} \begin{otherlanguage}{english}\uproman{21}.5\end{otherlanguage} \dd{}} \linelabel{_138e}
\end{tlg}
%\flushpage
  \end{edition}
  \begin{translation}
    \ekddiv{
      head={[\uproman{21}. \textbf{The Characteristic of Jñānayoga}]},
      type=section,
      depth=2, 
      n=XXI.1
    }
    \xmlhead[h21]{[XXI. The Characteristic of Jñānayoga]}
    \label{jnanayogatrans1}
     \begin{tlate}[p21_01]
       \noindent
       \begin{euber}[f14_1]\blfootnote{\hspace{-2.2em}in \citetitle{birch2013} 2.7-8 (\textit{cittaṃ buddhir ahaṅkāra ṛtvijaḥ somapaṃ manaḥ} | \textit{indriyāṇi daśa prāṇāñ juhoti jyotimaṇḍale} || 7 || \textit{ā mūlād bilaparyantaṃ vibhāti jyotimaṇḍalam} | \textit{yogibhiḥ satataṃ dhyeyam aṇimādyaṣṭasiddhidam} || 8 ||). These verses precede or introduce \textit{śāmbhavī mudrā}. Here, thought, intellect and ego are taught the be the officiants, whereas the mind is the sacrificer who sacrifices the senses and the ten vital breaths into the orb of light (2.7). The orb of light (\textit{jyotimaṇḍala}) shines from the root (possibly the root of the body or spine, but \citeauthor[2013:286]{birch2013} suggests the palate) to the aperture at the top of the head. Yoga practitioners should constantly meditate on it to achieve \textit{siddhi}s (2.8).}\end{euber} Now, the characteristic of Jñānayoga is explained.
     \end{tlate}
     \begin{tlate}[21_1]
       \paragraph{\uproman{21}.1} He shall see the world as only one, illumined by the supreme self. By the method of non-dualistic thinking, he shall accomplish \textit{Jñānayoga}.
     \end{tlate}
     \begin{tlate}[21_2]
       \paragraph{\uproman{21}.2} Alternatively, wherever one dwells, the world is made of all knowledge. He who knows thus by realisation is also qualified for gnosis. %%%Cf. NT 8.55cd, vāpi = einfach vā + api = vā (im Prinzip metrischer Füller)
     \end{tlate}
     \begin{tlate}[21_3]
       \paragraph{\uproman{21}.3} The one who is devoted to non-duality always attains the reality of Śāṃbhavī\footnote{Rāmacandra uses the term \textit{śāṃbhavīsattā} as a designation of the ultimate state to be attained by practising Jñānayoga, which he presents as the realization of absolute unity. In medieval Yogatexts, particular in the Rājayoga genre, the feminin noun \textit{śāmbhavī} most often appears in the context of a non-physical \textit{mudrā}, the so-called \textit{śāṃbhavī mudrā}. For a detailed discussion of \textit{śāṃbhavī mudrā}, its influence and all references, see \citeauthor[2013:71-79]{birch2013}. The usage of the feminin noun \textit{śāmbhavī} to qualify a state is uncommon. More frequently one finds the masculine adjective \textit{śāṃbhava} in order to quality an exalted yogic state. See for example \emph{Candrāvalokana} 2, \textit{Haṭhapradīpikā} 4.7, \emph{Anubhavanivedana} 1, \emph{Haṭhatattvakaumudī} 49.27. The idea has its roots in tantric traditions of Śaivism and refers to an meditative state associated with Śambhu (Śiva) (references?).}, just as the seed of the banyan tree scattered onto the ground [always] becomes a tree.
%       \footnote{In rituals the banyan tree (\textit{nygarodha}) is associated with the \textit{kṣatriya} class (\citeauthor[1998:27]{smith1998}).} 
     \end{tlate}
     \begin{tlate}[21_4]
\paragraph{\uproman{21}.4} By nature [the reality of Śāṃbhavī] is not only seen as one [but] has been fabricated tenfold. [Just as] the branches, buds and twigs are [held] up by the stem of the roots and shoots. [\ldots]
\end{tlate}
    \begin{tlate}[21_5]
      \paragraph{\uproman{21}.5} [\ldots] The sap, flower [and] fruit are in the seed. This is the extent [of it] by nature. And so it is pure, eternal, unchanging, and immaculate.
%\flushpage 
    \end{tlate}
\end{translation}
\end{alignment}
\pagebreak %after pp. 41-42
%%%%%%%%%%%%%%%%%%%%%%%%%%%%%%%%%%%%%%%%%% 
%%%%%%%%%%%%%%%%%%%%%%%%%%%%%%%%%%%%%%%%%% 
%%%%%%%%PAGEBREAK%%%%%%%PAGEBREAK%%%%%%%%%
%%%%%%%%%%%%%%%%%%%%%%%%%%%%%%%%%%%%%%%%%% 
%%%%%%%%%%%%%%%%PAGEBREAK%%%%%%%%%%%%%%%%%
%%%%%%%%%%%%%%%%%%%%%%%%%%%%%%%%%%%%%%%%%% 
%%%%%%%%PAGEBREAK%%%%%%%PAGEBREAK%%%%%%%%%
%%%%%%%%%%%%%%%%%%%%%%%%%%%%%%%%%%%%%%%%%% 
%%%%%%%%%%%%%%%%%%%%%%%%%%%%%%%%%%%%%%%%%% 
%%%%%%%%%%%%%%%%%%%%%%%%%%%%%%%%%%%%%%%%%% 
%%%%%%%%%%%%%%%%%%%%%%%%%%%%%%%%%%%%%%%%%% 
%%%%%%%%PAGEBREAK%%%%%%%PAGEBREAK%%%%%%%%%
%%%%%%%%%%%%%%%%%%%%%%%%%%%%%%%%%%%%%%%%%% 
%%%%%%%%%%%%%%%%PAGEBREAK%%%%%%%%%%%%%%%%%
%%%%%%%%%%%%%%%%%%%%%%%%%%%%%%%%%%%%%%%%%% 
%%%%%%%%PAGEBREAK%%%%%%%PAGEBREAK%%%%%%%%%
%%%%%%%%%%%%%%%%%%%%%%%%%%%%%%%%%%%%%%%%%% 
%%%%%%%%%%%%%%%%%%%%%%%%%%%%%%%%%%%%%%%%%% 
%%%%%%%%%%%%%%%%%%%%%%%%%%%%%%%%%%%%%%%%%% 
%%%%%%%%%%%%%%%%%%%%%%%%%%%%%%%%%%%%%%%%%% 
%%%%%%%%PAGEBREAK%%%%%%%PAGEBREAK%%%%%%%%%
%%%%%%%%%%%%%%%%%%%%%%%%%%%%%%%%%%%%%%%%%% 
%%%%%%%%%%%%%%%%PAGEBREAK%%%%%%%%%%%%%%%%%
%%%%%%%%%%%%%%%%%%%%%%%%%%%%%%%%%%%%%%%%%% 
%%%%%%%%PAGEBREAK%%%%%%%PAGEBREAK%%%%%%%%%
%%%%%%%%%%%%%%%%%%%%%%%%%%%%%%%%%%%%%%%%%% 
%%%%%%%%%%%%%%%%%%%%%%%%%%%%%%%%%%%%%%%%%%
\begin{alignment}[
  texts=edition[class="edition"];
  translation[class="translation"],
  ]
  \begin{edition}
%------------------------------
%eko  nekaḥ  svayaṃbhūś ca dhāmnā ca    bahudhā sthitaḥ/   paṃcatattvamanobuddhi-māyāhaṃkāravikriyāḥ //5//   \E
%eko  nekaḥ  svayaṃbhūś ca svadhāmnā    bahudhā sthitāḥ    paṃcatatvamanobuddhir māyāhaṃkāravikriyāḥ   6     \P
%eko  neka   svayaṃbhūś ca dhāmnāya     bahudhā sthitaḥ//  paṃcatatvamanobuddhi--māyāhaṃkāravikriyā  //      \B
%eko  nekaḥ  svayaṃbhūś ca svadhābhāva  bahudhā sthitāḥ//  paṃcatatvamanobuddhi--māyāhaṃkāravikriyā  //      \L
%eko  nekaḥ  svayaṃbhuś ca svayāṃmnā    bahudhā sthitaḥ/   paṃcatatvamanobuddhir māyāhaṃkāravikriyā  //6//   \N1
%eko  nekaḥ  svayaṃbhaś ca svadhā...ṣ   bahudhā sthitāḥ//  paṃcatatvamanobuddhir māyāhaṃkāravikriyā  //6//   \D
%eko  neka   svayaṃbhūś ca svadhāmnāva  bahudhā sthitaḥ//  paṃcatatvamanobuddhir māyāhaṃkāravikriyā  //6//   \N2
%yeko naika/ svayaṃbhūtyā  svabhāvā     bahudhā sthitaḥ    paṃcatatvamanobuddhir māyāhaṃkāravikriyāḥ   6     \U1
%eko  naiko  svayaṃbhūś ca svadhāmnā    bahudhā sthitaḥ//  paṃcatatvamanobuddhir māyāhaṃkāravikriyā  //6//   \U2
%------------------------------
%One, not one, self-existing by it's own power, abiding in multiplicity, as five [gross] elements (\textit{tattva}), thinking mind (\textit{manas}), intellect (\textit{buddhi}), illusion (\textit{māya}), individuation (\textit{ahaṃkāra}), and modifications (\textit{vikriyā}). 
%------------------------------
\begin{tlg}[21_6]
  \noindent
\note[type=source, labelb=139, labele=_139e, nosep]{ \approx  YSv (PT p. 836): eko 'nekaḥ khayaṃ bhūyān sādhanād bahudhā sthitaḥ | pañcatattvamayo buddhimāyāhaṅkāravikriyaḥ |}
  \tl{
\app{\lem[wit={ceteri}]{eko}
  \rdg[wit={U1}]{yeko}}
\app{\lem[type=emendation, resp=egoscr]{naikaḥ}
   \rdg[wit={ceteri}]{nekaḥ}
   \rdg[wit={U1}]{naika}
  \rdg[wit={U2}]{naiko}
  \rdg[wit={B,N2}]{neka}}
\app{\lem[wit={ceteri},alt={svayaṃbhūś ca}]{svayaṃbhūś\skp{-}ca}
  \rdg[wit={U1}]{svayaṃbhūtyā}}
\app{\lem[wit={P,U2}]{svadhāmnā}
  \rdg[wit={E}]{dhāmnā ca}
  \rdg[wit={B}]{dhāmnāya}
  \rdg[wit={L}]{svadhābhāva}
  \rdg[wit={N1}]{svayāṃmnā}
  \rdg[wit={D}]{svadhā..ṣa}
  \rdg[wit={N2}]{svadhāmnāva}
  \rdg[wit={U1}]{svabhāvā}}
bahudhā
\app{\lem[wit={ceteri}]{sthitaḥ}
  \rdg[wit={D,L,P}]{sthitāḥ}}/}\\
\tl{pañcatattvamano\app{\lem[wit={E,P,L},alt={°buddhi°}]{buddhi}
    \rdg[wit={ceteri}]{°buddhir}
  }māyāhaṃkāra\app{\lem[type=emendation, resp=egoscr]{vikriyaḥ}
    \rdg[wit={E,P,U1}]{vikriyāḥ}
    \rdg[wit={B,D,L,N1,N2,U2}]{vikriyā}}\dd{} \begin{otherlanguage}{english}\uproman{21}.6\end{otherlanguage} \dd{}} \linelabel{_139e}
\end{tlg}
%------------------------------ 
%evaṃ daśavidhaṃ viśvaṃ lokālokasavistaram/   eka  eva na cānyo sti yo jānāti sa tattvavit//6// \E
%evaṃ daśavidhaṃ viśvaṃ lokālokasavistaraṃ    eka  eva na cānyo sti yo jānāti sa tatvavit 6 \P
%evaṃ daśavidhā  viśvaṃ lokālokasavistaraṃ//  eka  eva na cānyā sti yo jānāti sa tatvavit// \B
%evaṃ daśavidhā  viśvaṃ lokālokasavistaraṃ//  eka  eva na cānyo sti yo jānāti sa tatvavit// \L
%evaṃ daśavidhaṃ viśvaṃ lokālokasavistarāṃ/   eka  eva na cānyo sti yo nānāti sa tatvavit//7// \N1
%evaṃ daśavidhaṃ viśvaṃ lokālokasavistaraṃ//  eka  eva na cānyo sti yo jānāti sa tatvavit//7// \D
%evaṃ daśavidhā  viśvaṃ lokālokasavistaraṃ//  eka  eva na cānyo sti yo jānāti sa tatvavit//7// \N2
%evaṃ daśavidha--viśvaṃ lokālokasavistaraṃ    eka yeva na cānyo sti yo jānāti sa tatvavit 7 \U1
%evaṃ daśavidhaṃ viśvaṃ lokāloke savistaraṃ// ekam eva na cānyo sti yo jānāti sa tatvavit//7// \U2 %%%409.jpg 
%------------------------------
%Auf diese Weise durchdringen die zehn Variationen die Welt und die Nicht-Welt im vollen Umfang.  
%Nur das Eine ist und nicht etwas anderes: Wer das weiß ist ein Kenner der Realität.  
%------------------------------
%In this way, the ten variations fully permeate the world and the non-world.
%Only one thing is and not something else: Whoever knows this is a connoisseur of reality.
%------------------------------
\begin{tlg}[21_7]
  \noindent
\note[type=source, labelb=140, labele=_140e, nosep]{ \approx  YSv (PT p. 836): evaṃ bahuvidhaṃ viśvaṃ lokālokasuvistaram | ekam eva na cānvo 'sti yo jānāti sa tattvavit |}
   \tl{
     evaṃ
     \app{\lem[wit={D,E,P,N1,U2}]{daśavidhaṃ viśvaṃ}
       \rdg[wit={B,L,N2}]{daśavidhā viśvaṃ}
       \rdg[wit={U1}]{daśavidhaviśvaṃ}}
     \app{\lem[type=emendation, resp=egoscr]{lokālokasuvistaram}
       \rdg[wit={B,D,E,L,P,N2,U1}]{lokālokasavistaram}  
       \rdg[wit={N1}]{lokālokasavistarāṃ}
       \rdg[wit={U2}]{lokāloke savistaraṃ}}/}\\
   \tl{\app{\lem[wit={ceteri}]{eka}
       \rdg[wit={U2}]{ekam}}
       \app{\lem[wit={ceteri}]{eva}
         \rdg[wit={U1}]{yeva}}
       na cānyo'sti yo jānāti sa tattvavit\dd{} \begin{otherlanguage}{english}\uproman{21}.7\end{otherlanguage} \dd{}} \linelabel{_140e}
\vspace{5mm} %5mm vertical space 
   \end{tlg}
    \begin{prose}[p21_02]     
%------------------------------
%pṛthvīvanaspatiparvatādisthārarūpaḥ         saṃsāra---manuṣyahastyaśvapakṣītyādiko    jaṃgamarūpaḥ   saṃsāraḥ// \E
%pṛthvīvanaśpatiparvatādisthāvararūpaḥ       saṃsāraḥ  manuṣyahastyaś ca pakṣītyādiko  jaṃgamarūpaḥ   saṃsāraḥ \P
%pṛthvīvanaspatīparvatādisthāvararūpā        saṃsāraḥ/ manuṣyahasteśvapakṣītyādiko     jaṃgamarūpaḥ   saṃsāraḥ// \B
%pṛthvīvanaspatiparvatādisthāvararūpā        saṃsāraḥ  manuṣyahasteśvapakṣītyādiko     jaṃgamarūpā    saṃsāraḥ// \L
%pṛthvīvanaspatīparvvate tyādisthāvararūpaḥ  saṃsāraḥ  manuṣyahastīaśvapakṣītyādiko    jaṃgamarūpaḥ   saṃsāraḥ// \N1
%pṛthvīvanaspatīparvato tyādisthāṃvararūpaḥ  saṃsāraḥ  manuṣyahastīaśvapakṣītyādiko    jaṃgamaḥ rūpaḥ saṃsāraḥ// \D
%pṛthvīvanaspatiparvate 'thyādisthāvararūpa  saṃsāraḥ  manuṣyahastipakṣītyādiko        jaṃgamarūpaḥ   saṃsāraḥ// \N2
%pṛthivīvanaspatīparvate iyādisthāvararūpaḥ  saṃsāra---manuṣyahastiasvapakṣītyādiko    jagadrūpaḥ     saṃsāro \U1
%pṛthvīvanaspatiparvatādisthāvararūpaḥ       saṃsāraḥ//manuṣyahasttyaś ca pakṣītyādiko jaṃgamarūpaḥ   saṃsāraḥ//8// \U2
%------------------------------
%The mundane existance (\textit{saṃsāra}) exists in the form of stationary [existances] such as earth, tree mountains and so on. The mundane existance (\textit{saṃsāra}) [also] exists in the form of the mobile [existances] such as humans, animals, birds and so on.
%------------------------------
\note[type=source, labelb=141, labele=_141e, nosep]{cf. YSv (PT p. 832): sthāvarāḥ parvatādyā hi jaṅgamāḥ khecarādayaḥ | jaṅgamasthāvarākāraḥ saṃsāraḥ syāt sa īśvaraḥ |}
\app{\lem[wit={ceteri},alt={pṛthvī°}]{pṛthvī}
        \rdg[wit={U1}]{pṛthivī°}
      }\app{\lem[wit={E,N2,U2},alt={°vanaspati°}]{vanaspati}
        \rdg[wit={P}]{vanaśpati}
        \rdg[wit={B,D,L,N1,U1}]{vanaspatī°}
      }\app{\lem[wit={B,L,P,U2}, alt={°parvatādisthāvara°}]{parvatādisthāvara}
        \rdg[wit={E}]{°parvatādisthāra°}
        \rdg[wit={D}]{°parvato tyādisthāṃvara°}
        \rdg[wit={N1}]{°parvvate tyādisthāvara°}
        \rdg[wit={N2}]{°parvate 'thyādisthāvara°}
        \rdg[wit={U1}]{°parvate iyādisthāvara°}
      }\app{\lem[wit={ceteri}]{rūpaḥ}
        \rdg[wit={B,L}]{rūpā}
        \rdg[wit={N2}]{rūpa}}
      \app{\lem[wit={ceteri}]{saṃsāraḥ}
        \rdg[wit={E,U1}]{saṃsāra°}}/
      manuṣya\app{\lem[wit={B,L},alt={°hasteśvapakṣīty ādiko}]{hasteśvapakṣīty\skp{-}ādiko}
          \rdg[wit={E}]{°hasty aśvapakṣīty ādiko}
          \rdg[wit={D,N1}]{°hastīaśvapakṣīty ādiko}
          \rdg[wit={N2}]{°hastipakṣīty ādiko}
          \rdg[wit={U1}]{°hastiasvapakṣīty ādiko}
          \rdg[wit={U2}]{°hasttyaś ca pakṣīty ādiko}}
        \app{\lem[wit={ceteri}]{jaṃgamarūpaḥ}
          \rdg[wit={D}]{jaṃgamaḥ rūpaḥ}
          \rdg[wit={L}]{°rūpā}
          \rdg[wit={U1}]{jagad°}}
        \app{\lem[wit={ceteri}]{saṃsāraḥ}
          \rdg[wit={U1}]{saṃsāro}}/\linelabel{_141e}
%------------
%atha ca   yo  dṛṣṭiviṣayaḥ  sa dṛśya  ucyate/  yo dṛṣṭyā na vīkṣyate sa adṛśya ity  ucyate/ \E
%atha ca   yo  dṛṣṭiviṣayaḥ  sa dṛśya  ucyate   yo dṛṣṭyā na vīkṣyate sa adṛśya ity  ucyate  %%%7641.jog
%atha ca// yo  daṣṭiviṣayaḥ  sa dṛśya  ucyate// yo dṛṣṭyā na vīkṣyate sa adṛśya ty   ucyate// \B
%atha ca   yo ddṛṣṭiviṣayaḥ  sa dṛśya  ucyate// yo dṛṣṭyā na vīkṣyate sa adṛśye ty   ucyate... \L
%atha ca   ya ddṛṣṭiviṣayaḥ  sa dṛśyad ucyate   yo dṛṣṭyā na vīkṣyate sa adṛśya ity  ucyate// \N1
%atha vā   ya dārṣṭiviṣayaḥ  sa dṛśya  ucyate/  yo dṛṣṭyā na vīkṣyate sa adṛśya ity  ucyate// \D
%atha ca   ya  drṣṭiviṣayaḥ  sa dṛśya  ucyate/  yo dyā    na vīkṣyate sa adṛśya śaty ucyate/ \N2
%atha ca   yaḥ drṣṭiviṣayaḥ  sa dṛśy---ucyate   yo dṛṣṭvā na vīkṣyate sa adṛśya ity  ucyate \U1
%atha ca   yo  dṛṣṭiviṣayaḥ  sa dṛśya  ucyate// yo dṛṣṭyā na vīkṣyate sa adṛśya ity  ucyate// \U2
%------------------------------
%Now, that which is the object of sight is called the seen. That which is not seen by sight is called the unseen.
%------------------------------
\note[type=source, labelb=142, labele=_142e, nosep]{cf. YSv (PT p. 836): svabhāvalīlayā bhāti śūnye 'sau śūnyabuddhitaḥ | yad dṛṣṭaṃ viṣayaṃ vastu tad dṛśyam iti kathyate | yo dṛṣṭātītaḥ so 'dṛśyas tadā dṛṣṭaṃ hi manyate | svatanūbhedam evan tu saṃsāraṃ duḥkhasaṅkulam | yatnād dūraṃ parityajya jñānayogo bhavet sudhīḥ | jñānasaṃyoga ekas tu ekas tu jñānayogavān | ato hi jñānato 'bhinnaṃ jñeyaṃ jñānāt pṛthak pṛthak | dūrīkṛtyaiva mā pṛthvī bhedavākyena darśanāt | jñānayogī bhaved yena jñānayogas tu caikakaḥ | evaṃ jñānānmaheśāni kālajit śivatāṃ vrajet |}
atha
      \app{\lem[wit={ceteri}]{ca}
        \rdg[wit={D}]{vā}}
      \app{\lem[wit={ceteri}]{yo}
        \rdg[wit={U1}]{yaḥ}
        \rdg[wit={D,N1,N2}]{ya}}
      \app{\lem[wit={ceteri}, alt={°dṛṣṭi}]{dṛṣṭi}
        \rdg[wit={L,N1}]{°ddṛṣṭi}
        \rdg[wit={B}]{°daṣṭi}
        \rdg[wit={D}]{°dārṣṭi}
}viṣayaḥ sa
\app{\lem[wit={ceteri}]{dṛśya}
  \rdg[wit={N1}]{dṛśyad}
  \rdg[wit={U1}]{dṛṣy°}}
ucyate/
yo
\app{\lem[wit={ceteri}]{dṛṣṭyā}
  \rdg[wit={N2}]{dyā}}
na vīkṣyate sa adṛṣya
\app{\lem[wit={ceteri},alt={ity}]{i\skp{ty-u}}
  \rdg[wit={B,L}]{ty}
  \rdg[wit={N2}]{śaty}
}\skm{ty-u}cyate/
%------------------------------
%evaṃ saṃsārasya svātmano  bhedaṃ dūrīkṛty---aikam eva darśanaṃ sa eva jñānayogaḥ/   \E
%evaṃ saṃsāra----svātmano  bhedaṃ dūrīkṛtya  aikyena   darśanaṃ        jñānayogaḥ    \P
%evaṃ saṃsārasya svātmano  bheda--dūrīkṛtya  aikyona   darśanaṃ        jñānayogaḥ/   \B
%evaṃ saṃsāra----svātmano  bhedaṃ dūrīkṛtya  aikyona   darśanaṃ        jñānayogaḥ... \L
%evaṃ saṃsārasya svātmanaḥ bhedāṃ dūrīkṛtya  ekyena    darśanaṃ        jñānayogaḥ//  \N1
%evaṃ saṃsārasya svātmanaḥ bhedāṃ dūrīkṛtya  ekyena    darśanaṃ        jñānayogaḥ/   \D
%evaṃ saṃsārasya svātmanaḥ bhedaṃ dūrīkṛtya  ekena     darśanaṃ        jñānayogaḥ/   \N2
%evaṃ saṃsārasya svātmanaḥ bhedaṃ dūrīkṛtya  ekānta    darśanaṃ        jñānayogaḥ    \U1
%evaṃ saṃsāra....svātmanoḥ bhedaṃ dūrīkṛtyaṃ ekye?     darśanaṃ        jñānayoga     \U2
%------------------------------
%In this way, the realisation of unity (\textit{aikyena darśanam}) by eliminating the differentiation of the self from the mundane existance is truly Jnānayoga.
%------------------------------
evaṃ
       \app{\lem[wit={ceteri}]{saṃsārasya}
         \rdg[wit={P,L,U2}]{saṃsāra°}}
       \app{\lem[wit={B,E,L,P}]{svātmano}
         \rdg[wit={X}]{svātmanaḥ}
         \rdg[wit={U2}]{svātmanoḥ}}
       \app{\lem[wit={ceteri}]{bhedaṃ}
         \rdg[wit={B}]{bheda}
         \rdg[wit={D,N1}]{bhedāṃ}}
dūrī\app{\lem[wit={U2},alt={°kṛtyaṃ}]{kṛtyaṃ}
  \rdg[wit={ceteri}]{°kṛtya}
  \rdg[wit={E}]{°kṛty}}
\app{\lem[wit={P}]{aikyena}
  \rdg[wit={E}]{aikam eva}
  \rdg[wit={B,L,P}]{aikyona}
  \rdg[wit={D,N1}]{ekyena}
  \rdg[wit={N2}]{ekena}
  \rdg[wit={U1}]{ekānta}
  \rdg[wit={U2}]{ekye}}
darśanaṃ
\app{\lem[wit={E}]{sa eva}
  \rdg[wit={ceteri}]{\om}}
\app{\lem[wit={ceteri}]{jñānayogaḥ}
  \rdg[wit={U2}]{jñānayoga}}/ 
%------------------------------
%tasya         kāraṇāt kālaḥ śarīranāśaṃ na karoti/ \E
%tasya         kāraṇāt kālaḥ śarīranāśaṃ na karoti/ \P
%tasya         karaṇāt kālaḥ śarīranāśaṃ na karoti// \B
%tasya         karaṇāt kālaḥ śarīranāśaṃ na karoti... \L
%tasya         karaṇāt kālaḥ śarīranāśaṃ na karoti// \N1
%tasya         karaṇāt kālaḥ śarīranāśaṃ na karoti// \D
%tasya         karaṇāt kālaḥ śarīranāśaṃ    karoti/ \N2
%gatasya dhyānakaraṇāt kālaḥ śarīranāśaṃ na karoti 8 \U1
%tasya         karaṇāt kālaśarīranāśanaṃ    karoti// \U2
%------------------------------
%Due to this, time does not bring about the destruction of the body.
%------------------------------
\app{\lem[wit={ceteri}]{tasya}
  \rdg[wit={U1}]{gatasya}}
\app{\lem[wit={ceteri}, alt={kāraṇāt}]{kāraṇā\skp{t-kā}}
  \rdg[wit={U1}]{dhyānakaraṇāt}
}\app{\lem[wit={ceteri}, alt={kālaḥ}]{\skm{t-kā}laḥ}
  \rdg[wit={U1}]{kāla°}}
śarīranāśaṃ
\app{\lem[wit={ceteri}]{na}
  \rdg[wit={N2,U2}]{\om}}
karoti\dd{}\linelabel{_142e}
\label{jnanayogaend}
\end{prose}
  \end{edition}
  \begin{translation}
    \begin{tlate}[21_6]
     \paragraph{\uproman{21}.6} One, not one, self-existing by its own power, abiding in multiplicity, as five [gross] elements (\textit{tattva})\footnote{The term \textit{pañcatattva} refers to the five gross elements. The concept of five gross elements emerging from a supreme god is found in tantric works, cf. e.g. \citetitle{sivasvarodaya} 6-8 (\textit{nirañjano nirākara eko devo maheśvaraḥ} | \textit{tasmād ākāśam utpannam ākāśād vāyusambhavaḥ} || 6 || \textit{vayos tejas tataś cāpas tataḥ pṛthvī samudbhavaḥ} | \textit{etāni pañcatattvāni vistīrṇāni ca pañcadhā} || 7 || \textit{tebhyo brahmāṇḍam utpannaṃ tair eva parivartate} | \textit{vilīyate ca tatraiva tatraiva ramate punaḥ} || 8 ||) “Faultless and without a body is the one god, the great supreme ruler. From him, the ether element (\textit{ākāśa}) originated, and from the ether element, the air element came into existence (6). From the air element, the fire element and from the fire element, the water element and from the water element, the earth element. These five elements unfold in a fivefold manner (7). From these five elements, the universe has arisen, evolves and dissolves. [Then] right here, right there [it] enjoys again (8).” } thinking mind (\textit{manas}), intellect (\textit{buddhi}), illusion (\textit{māya}), individuation (\textit{ahaṃkāra}), and modifications (\textit{vikriyā}).\footnote{In the tantric traditions of Śaivism tenfold \textit{tattva}-systems existed from an very early age, cf. \citeauthor[2016:82-85]{goodall2016}. Rāmacandra, as can be seen in the sources of the edition for this passage, is faithful to his source text. However, the exact origin of this specific arrangement of \textit{tattva}s remains unknown. Usually \textit{vikriyā} is not a separate \textit{tattva}, but \textit{ahaṃkāravikrīya} (“transformations of \textit{ahaṃkāra}”) refers to lesser \textit{tattva}s like the \textit{jñānendrīya}s, \textit{karmendrīya}s and \textit{tanmātra}s. Here, it seems the term \textit{vikriyā} is taken as a \textit{tattva} on its own, functioning as a placeholder for the modifications of \textit{ahaṃkāra}.}
% In \citetitle{kumbhaka} 122, a technique of breath retention is dedicated to the five \textit{tattva}s (\textit{tatvādau pūreyed vāyuṃ tat tatvānte virecayet} | \textit{tatvakumbhaḥ sa gaditaḥ pañcadhā tatvabhedataḥ} || 122 ||) “One shall inhale before [the rise] of a particular \textit{tattva} and exhale at the end of that \textit{tattva}. This is called \textit{tattvakumbhaka} being five-fold according to the five divisions of \textit{tattva}.'' The \citetitle{sivasvarodaya} discusses the rise, duration, properties and application of the \textit{tattva}s in greater detail. An overview of this can be seen in \citeauthor[2021: Appendix-\lowroman{3}]{kumbhaka}.
   \end{tlate}
   \begin{tlate}[21_7]
  \paragraph{\uproman{21}.7} Thus, everything is tenfold extending\footnote{The manuscripts read \textit{savistara}, which is nonsensical in this context. Thus, I emended according to the source text’s reading.} as far as the Lokāloka[-mountain].\footnote{\citeauthor{birch2013} suggested to understand \textit{lokāloka} as the mythical mountain that divides the visible world from the regions of darkness. Cf. and see \citeauthor{apte1890practical} (1980: 933) for further references.} There is only one. There is nothing else. One who knows this is a knower of truth.
     \\\\
     \end{tlate}
\begin{tlate}[p21_02]
 The circuit of mundane existence has the nature of stationary [existances] such as earth, trees, mountains, etc. The circuit of mundane existence has the nature of the mobile [existances] such as humans, elephants, horses, birds, etc. Now, that which is the object of sight is called the seen. That which is not seen by sight is called the unseen. In this way, the removal of the distinction of the own self from the circuit of mundane existance is to be done by means ofobservation with identity. Only this is Jñānayoga. From the execution of this, time [as death] does not bring about the destruction of the body.
 \flushpage
    \end{tlate}
  \end{translation}
\end{alignment}
\pagebreak %after pp.43-44
%%%%%%%%%%%%%%%%%%%%%%%%%%%%%%%%%%%%%%%%%%
%%%%%%%%%%%%%%%%%%%%%%%%%%%%%%%%%%%%%%%%%%
%%%%%%%%PAGEBREAK%%%%%%%PAGEBREAK%%%%%%%%%
%%%%%%%%%%%%%%%%%%%%%%%%%%%%%%%%%%%%%%%%%%
%%%%%%%%%%%%%%%%PAGEBREAK%%%%%%%%%%%%%%%%%
%%%%%%%%%%%%%%%%%%%%%%%%%%%%%%%%%%%%%%%%%%
%%%%%%%%PAGEBREAK%%%%%%%PAGEBREAK%%%%%%%%%
%%%%%%%%%%%%%%%%%%%%%%%%%%%%%%%%%%%%%%%%%%
%%%%%%%%%%%%%%%%%%%%%%%%%%%%%%%%%%%%%%%%%%
%%%%%%%%%%%%%%%%%%%%%%%%%%%%%%%%%%%%%%%%%%
%%%%%%%%%%%%%%%%%%%%%%%%%%%%%%%%%%%%%%%%%%
%%%%%%%%PAGEBREAK%%%%%%%PAGEBREAK%%%%%%%%%
%%%%%%%%%%%%%%%%%%%%%%%%%%%%%%%%%%%%%%%%%%
%%%%%%%%%%%%%%%%PAGEBREAK%%%%%%%%%%%%%%%%%
%%%%%%%%%%%%%%%%%%%%%%%%%%%%%%%%%%%%%%%%%%
%%%%%%%%PAGEBREAK%%%%%%%PAGEBREAK%%%%%%%%%
%%%%%%%%%%%%%%%%%%%%%%%%%%%%%%%%%%%%%%%%%%
%%%%%%%%%%%%%%%%%%%%%%%%%%%%%%%%%%%%%%%%%%
%%%%%%%%%%%%%%%%%%%%%%%%%%%%%%%%%%%%%%%%%%
%%%%%%%%%%%%%%%%%%%%%%%%%%%%%%%%%%%%%%%%%%
%%%%%%%%PAGEBREAK%%%%%%%PAGEBREAK%%%%%%%%%
%%%%%%%%%%%%%%%%%%%%%%%%%%%%%%%%%%%%%%%%%%
%%%%%%%%%%%%%%%%PAGEBREAK%%%%%%%%%%%%%%%%%
%%%%%%%%%%%%%%%%%%%%%%%%%%%%%%%%%%%%%%%%%%
%%%%%%%%PAGEBREAK%%%%%%%PAGEBREAK%%%%%%%%%
%%%%%%%%%%%%%%%%%%%%%%%%%%%%%%%%%%%%%%%%%%
%%%%%%%%%%%%%%%%%%%%%%%%%%%%%%%%%%%%%%%%%%
\begin{alignment}[
  texts=edition[class="edition"];
  translation[class="translation"],
  ]
  \begin{edition}
    \ekddiv{
      head={[\uproman{22}. \textbf{svabhāvabhedam}]},
      type=section,
      depth=2, 
      n=XXII
    }
    \xmlhead[h22]{[XXII. svabhāvabhedam]}
    \begin{prose}[p22_01]
      \label{svabhava1}
            \noindent
%------------------------------
%idānīṃ tasya---bhedaḥ    kathyate/   \E
%idānīṃ svabhāvabhedaḥ kathyate    \P
%idānī  svābhāvabhedaḥ kathyate//  \B
%idānīṃ svābhāvabhedaḥ kathyate//  \L
%idānīṃ svabhāvabhedaṃ kathyate//  \N1
%idānīṃ svabhāvabhedaṃ kathyate//  \D
%idānīṃ svabhāvabheda  kathyate//  \N2
%idānīṃ svabhāvabhedāḥ kathyate    \U1
%idānīṃ svabhāvabhedaḥ kathyate//  \U2
%------------------------------
%Now, the division of the inherent being is described. 
%------------------------------  
\note[type=source, labelb=_143i, labele=_145e, nosep]{cf. YSv (PT p. 836): svabhāvabhedam etat śṛṇu devi prayatnataḥ | yac chrutvā sarvabodhaḥ syāt muktidaḥ siddhivāñchitaḥ | ātmano vā pṛthivyādyāḥ svabhāvaḥ kiñcid ucyate | ātmaiva pṛthivī dhātrī komalā ca kvacid dṛḍhā | kvacin manoharā sā ca vimalā ca malāmalā | durgandhā ca sugandhā ca nirgandhā gandhamohinī | svarṇarūpā dhāturūpā citrā ratnamayī parā | kvacit śvetā kvacid raktā kvacit pītā ca kṛṣṇalā | ūrvarā ūrvarā sā tu viṣāmṛtamayī sadā |}
\linelabel{_143i}
\app{\lem[wit={ceteri}]{idānīṃ}
  \rdg[wit={B}]{idānī}}
\app{\lem[wit={ceteri},alt={svabhāva°}]{svabhāva}
  \rdg[wit={B,L}]{svābhāva°}
  \rdg[wit={E}]{tasya}
}\app{\lem[wit={D,N1},alt={°bhedaṃ}]{bhedaṃ}
  \rdg[wit={N2}]{°bheda}
  \rdg[wit={ceteri}]{°bhedaḥ}}
kathyate/
%------------------------------  
%yathā vaṭabījam/ vaṭarūpeṇa pariṇataṃ    sat    daśadhā    bhedaṃ svabhāvata eva prāpnoti/  \E %%%[P.27]
%yathā vaṭabījaṃ  vaṭarūpeṇa pariṇāte     sat    dṛśadhā    bhedaṃ svabhāvata eva prāpnoti   \P
%yathā vaṭabījena rūpeṇa     pariṇamate/  śata   daśadhā    bhedaṃ svābhāva   eva prāpnotī// \B
%yathā vaṭabījena rūpeṇa     pariṇamate   śata   daśadhā    bhedaṃ svābhāva   eva prāpnotī// \L
%yathā vaṭabījaṃ  vaṭarūpeṇa pariṇataṃ//  satṛ   daśadhā    bhedaṃ svabhāvata eva prāpnoti/  \N1
%yathā vaṭabījaṃ  vaṭarūpeṇa pariṇataṃ/   sa     daśadhā    bhedaṃ svabhāvata eva prāpnoti// \D
%yathā vathabījaṃ vaṭarūpeṇa pariṇataṃ/   sa tu  daśadhā    bhedaṃ svabhāvata eva prāpnoti/  \N2
%yathā vaṭabījaṃ  vaṭarūpeṇa pariṇataṃ    sa tat daśadhā    bhedaṃ svabhāvata eva prāpnotī   \U1
%yathā vaṭabīja---vaṭarūpeṇa pariṇamate// sa     dasat                            prāpnoti// \U2
%------------------------------
%Just as the seed of the banyan tree ripens into the shape of the banyan tree, [and] because of its own inherent being develops such a tenfold division. [Namely]:
%------------------------------
yathā
\app{\lem[wit={ceteri},alt={vaṭa°}]{vaṭa}
  \rdg[wit={N2}]{vatha°}
}\app{\lem[wit={D,P,N1,N2,U1},alt={°bījaṃ}]{bījaṃ}
        \rdg[wit={E}]{°bījam}
        \rdg[wit={U2}]{°bīja°}
        \rdg[wit={B,L}]{°bījena}}
      \app{\lem[wit={ceteri}]{vaṭarūpeṇa}
        \rdg[wit={B,L}]{rūpeṇa}}
      \app{\lem[wit={B,L,U2}]{pariṇamate}
        \rdg[wit={P}]{pariṇāte}
        \rdg[wit={X,E}]{pariṇataṃ}}
      \app{\lem[wit={U1},alt={sa tat}]{sa ta\skp{t-da}}
        \rdg[wit={N2}]{sa tu}
        \rdg[wit={N1}]{satṛ}
        \rdg[wit={E,P}]{sat}
        \rdg[wit={B,L}]{śata}
        \rdg[wit={D,U2}]{sa}
      }\app{\lem[wit={ceteri},alt={daśadhā}]{\skm{t-da}śadhā}
        \rdg[wit={P}]{dṛśadhā}
        \rdg[wit={U2}]{dasat}}
      \app{\lem[wit={ceteri}]{bhedaṃ}
        \rdg[wit={U2}]{\om}}
      \app{\lem[wit={ceteri}]{svabhāvata}
        \rdg[wit={B,L}]{svabhāva}
        \rdg[wit={U2}]{\om}}
      \app{\lem[wit={ceteri}]{eva}
        \rdg[wit={U2}]{\om}}
      \app{\lem[wit={ceteri}]{prāpnoti}
        \rdg[wit={B,L,U1}]{prāpnotī}}/
%------------------------------ %%%%STEMMA POINT!!!!
%mūlāṃkura---tvagdaṇḍaśākhā--kalikāpallavapuṣpaphalasnehā                  iti daśabhedān    prāpnoti// \E
%mūla aṃkura-tvakdaṃdaśākhā----kilpikāpallavā puṣpaphalasneha              iti daśabhedān    prāpnotīti \P  %%%7642.jpg
%mūlaṃ aṃkuratvakdaṃdaśākhā----kilakālapallavā// vistāroyaṃ svābhāvataḥ    iti daśabhedān    prāpnoti// \B DSCN7160 Z. 4
%mūlaṃ aṃkuratvakdaṃdaśākhā----kilāpallavā// vistāroyaṃ svābhāvataḥ//      iti daśabhedān    prāpnoti... \L
%mūlāṃ aṃkuratvakdaṃḍaśākhāṃ kalikāpallavapuṣpaphalasneha//                iti bhedo daśadhā prāpnoti// \N1
%mūlāṃkura---tvakdaṇdaśākhāṃ kalikāpallavapuṣpaphalasnehaṃ                 iti bhedo daśadhā prāpnoti// \D
%mūlāṃkura---tvakdaṇdaśākhāṃ kalikāpallavapuṣpaphalasneha/                 iti bhedo daśadhā prāpnoti// \N2
%mūlāṃaṃkura-tvakdaṇdaśākhā--kalikāpallavapuṣpaphalasneha                  iti bhedo daśadhā prāpnoti \U1
%\om                                                                                \U2
%------------------------------
%"Wurzel, Spross, Rinde, Ast, Zweig, Knospe, die sich entfaltende Blüte, Blüte, Frucht und Nektar." Die Auftheilung erreicht [diese] zehn Teile. 
%------------------------------
%"Root, shoot, bark, branch, twig, bud, the unfolding flower, flower, fruit and nectar." The division reaches [those] ten parts.
%------------------------------
\app{\lem[wit={E}]{mūlāṅkuratvagdaṇḍaśākhākalikāpallavapuṣpaphalasnehā}
          \rdg[wit={P}]{mūla aṃkuratvakdaṃdaśākhākilpikāpallavā puṣpaphalasneha}
          \rdg[wit={B}]{mūlaṃ aṃkuratvakdaṃdaśākhākilakālapallavā || vistāroyaṃ svābhāvataḥ}
          \rdg[wit={L}]{mūlaṃ aṃkuratvakdaṃdaśākhākilāpallavā || vistāroyaṃ svābhāvataḥ ||}
          \rdg[wit={N1}]{mūlāṃ aṃkuratvakdaṃḍaśākhāṃ kalikāpallavapuṣpaphalasneha ||}
          \rdg[wit={N2}]{mūlāṃkuratvakdaṇdaśākhāṃ kalikāpallavapuṣpaphalasneha|}
          \rdg[wit={D}]{mūlāṃkuratvakdaṇdaśākhāṃ kalikāpallavapuṣpaphalasnehaṃ}
          \rdg[wit={U1}]{mūlāṃ aṃkuratvakdaṇdaśākhākalikāpallavapuṣpaphalasneha}
          \rdg[wit={U2}]{\om}}
        \app{\lem[wit={ceteri}]{iti}
          \rdg[wit={U2}]{\om}}
        \app{\lem[wit={X}]{bhedo daśadhā}
          \rdg[wit={B,E,L,P}]{daśabhedān}
          \rdg[wit={U2}]{\om}}
        \app{\lem[wit={ceteri}]{prāpnoti}
          \rdg[wit={P}]{prāpnotīti}
          \rdg[wit={U2}]{\om}}/
%------------------------------
%yathā nirmalo  nirvikāraḥ niraṃjana   eka  etādṛśa  ātmā svabhāvād eva/ pṛthivyaptejovāyvākāśamanobuddhimāyāvikārarūpabhedān    prāpnoti/ \E
%tathā nirmalaḥ nirvikāraḥ niraṃjanaḥ  eka  etādṛśa  ātmasvabhāvād eva   pṛthvyetetejo vādvyākāśamanobuddhimāyāvikārarūpabhedāt  prāpnoti \P
%tathā nirmalo  nirvikāraḥ niraṃjanaḥ  eka  etādṛśa  ātmasvabhāvād eva   pṛthvyāpatejovādvyākāśamanobuddhimāyāvikārarūpabhedāna  prāpnoti// \B
%tathā nirmalo  nirvikāraḥ niraṃjanaḥ/ eka  etādṛśa  ātmasvabhāvād eva   pṛthvyāpatejovāybākāśamanobuddhimāyāvikārarūpābhedāna   prāpnoti  \L
%tathā nirmalaḥ nirvikāraḥ niraṃjanaḥ  ekaḥ etādṛśaḥ ātmasvabhāvād eva   pṛthvyāpatejovāybākāśamanobuddhimāyāvikārarūpābhedān    prāpnoti/ \N1
%tathā nirmalaḥ nirvikāraḥ niraṃjanaḥ  eka  etādṛśaḥ ātmasvabhāvād eva   pṛthvīpate/ jīvīkāśamanobuddhir māyāvikārarūpabhedāt    prāpnoti \D
%tathā nirmalaḥ nirvikāraḥ niraṃjanaḥ  ekaḥ etādṛśaḥ ātmasvabhāvād eva   pṛthvīpate/ jīvīkāśamanobuddhir māyāvikārarūpabhedāt    prāpnoti/ \N2
%tathā nirmalaḥ nirvikāraḥ niraṃjanaḥ  ekaḥ etādṛśaḥ ātmascabhāvād eva   pṛthakte jīvāyuvākāśamanobuddhir māyāyāvikārarūpabhedāt prāpnoti \U1 %%%275.jpg
%yathā nirmalaḥ nirvikāraḥ niraṃjanaḥ  eka  etādṛśa  ātmasvabhāvād eva// pṛthvyaptejovāyyākāśa// manobuddhimayāvikārarūpabhedān  prāpnoti/ \U2
%------------------------------
%In dieser Weise erreicht auch das reine, unveränderliche, makellose, eine solche [Auftheilung] eben aufgrund der inhärenten Natur des Selbst. [Nämlich] die Aufteilung "Erde, Wasser, Feuer, Wind, Raum, Geist, Intellektekt, Illusion, Umwandlungen und Gestalt".
%------------------------------
%In this way, the pure, unchanging, unblemished, attains such [a division] precisely, because of the inherent being of the self. [Namely] the division: "Earth, water, fire, wind, space, mind, intellect, illusion, transformations and form".
%------------------------------
        \app{\lem[wit={ceteri}]{tathā}
            \rdg[wit={E,U2}]{yathā}}
          \app{\lem[wit={B,E,L}]{nirmalo}
            \rdg[wit={X,P,U2}]{nirmalaḥ}}
          nirvikāraḥ
          \app{\lem[wit={E}, alt={nirañjana}]{nirañjana}
            \rdg[wit={ceteri}]{niraṃjanaḥ}}
          \app{\lem[wit={ceteri}]{eka}
            \rdg[wit={N1,N2,U1}]{ekaḥ}}
          \app{\lem[wit={E}]{etādṛśa}
            \rdg[wit={N1,N2,U1}]{etādṛśaḥ}}
          \app{\lem[wit={ceteri}, alt={ātmasvabhāvād}]{ātmasvabhāvā\skp{d-e}}
            \rdg[wit={E}]{ātmā svabhāvād}
          }\skm{d-e}va
          \app{\lem[wit={B,L,N1}, alt={pṛthvyāpatejovāybākāśamanobuddhimāyāvikārarūpābhedān}]{pṛthvyāpatejovāybākāśamanobuddhimāyāvikārarūpābhedā\skp{n-prā}}
            \rdg[wit={E}]{pṛthivyapāpatejovāybākāśamanobuddhimāyāvikārarūpābhedān}
            \rdg[wit={P}]{pṛthvyetetejovādvyākāśamanobuddhimāyāvikārarūpābhedān}
            \rdg[wit={D,N2}]{pṛthvīpate | jīvīkāśamanobuddhir māyāvikārarūpabhedāt}
            \rdg[wit={U1}]{pṛthakte jīvāyuvākāśamanobuddhir māyāyāvikārarūpabhedāt}
            \rdg[wit={U2}]{pṛthvyaptejovāyyākāśa || manobuddhimayāvikārarūpabhedā}
          }\skm{n-prā}pnoti/
%------------------------------
%jñānayogaprabhāvād     eka eva  ātmā iti niścayo bhavati// \E
%jñānayogaḥ prabhāvād   eka eka  ātmā iti niścayo bhavati \P
%jñānayogaḥ// prabhāvād eka eka  ātmā iti niścayā bhavatī// \B
%jñānayogaḥ// prabhāvād eka eka  ātmā iti niścayo bhavati// \L
%jñānayogaprabhāvāt     eka eva  ātmā iti niścayo bhavati// \N1
%jñānayogaprabhāvāt     eka eva  ātmā iti niścayo bhavati// \D
%jñānayogaprabhāvāt     eka eva  ātmā iti niścayo bhavati// \N2
%jñānayogaprabhāvāt tu  eka yeva ātmā iti niścayo bhavati \U1
%jñānayogaprabhāvād     eka eva  ātmā iti niścayo bhavati// \U2
%------------------------------
%Because of the power of Jñānayoga, there arises the conviction that "the self is truly one".  
%------------------------------
\app{\lem[wit={E,U2}, alt={jñānayogaprabhāvād}]{jñānayogaprabhāvā\skp{d-e}}
  \rdg[wit={X}]{jñānayogabhavāt}
  \rdg[wit={B,L}]{jñānayogaḥ || prabhāvād°}
  \rdg[wit={P}]{jñānayogaḥ prabhāvād}
}\skm{d-e}ka
\app{\lem[wit={ceteri}]{eva}
  \rdg[wit={B,L,P}]{eka}
  \rdg[wit={U1}]{yeva}}
ātmā iti niścayo bhavati/
    \end{prose}
  \end{edition}
  \begin{translation}
   \ekddiv{
   head={[\uproman{22}. \textbf{Division of the Inherent Being}]},
   type=section,
   depth=2, 
   n=XXII.1 
 }
 \xmlhead[h22]{[XXII. Division of the Inherent Being]}
        \begin{tlate}[p22_01]
            \noindent
   Now, the division of the inherent being\footnote{The term \textit{svabhāvabheda} is a popular term in philosophical systems to signify the division or distinction of essential natures. For a Buddhist example, see \citeauthor{balcerowicz2011dharmakirti} (2011: 6 ff.). In some Nyāya texts, the term is used to signify different essential properties of the self, cf., i.e. \citeauthor{ogawa2023perception} (2023: 162). In the \textit{Netratantantra}, the term \textit{bhāvabheda} occurs frequently in a similar sense as in this passage of the \textit{Yogatattvabindu}. For example cf. \textit{Netratantra} 16.20-21 (\textit{bhaviṣyanti mahādevi kaliḥ kaṣṭataro yataḥ} | \textit{tadarthaṃ paramārtho 'yaṃ mayā te prakaṭīkṛtaḥ} | \textit{paramārthaḥ paratvena mṛtyujit sarvatomukhaḥ} | \textit{bhāvabhedena yaṣṭavyo mokṣasiddhim abhīpsatā} |).} is described. Just as the seed of the banyan tree ripens into the shape of the banyan tree, [and] because of its own inherent being develops such a tenfold division - root, shoot, bark, branch, twig, bud, the unfolding flower, flower, fruit and nectar. The division develops [those] ten parts. In the same way, the pure, unchanging, unblemished [one] attains such [a division] precisely because of the inherent being of the self. [Namely], the division: earth, water, fire, wind, space, mind, intellect, illusion, transformations and form.\footnote{Rāmacandra’s tenfold taxonomy of \textit{tattva}s appears inconsistent. Here, besides the stable list of the five gross elements, \textit{ahaṃkāra} is replaced with \textit{rūpa} and the order of the other elements is changed. None of the tenfold \textit{tattva}-systems known to me equal Rāmacandra's systems exactly. Taxonomies of \textit{tattva}s like \citetitle{bhagavadgita} 7.4 in which Kṛṣṇa presents a list of eight divisions of \textit{prakṛti} are almost identical (\textit{bhūmir āpo 'nalo vāyuḥ khaṃ mano buddhir eva cha} | \textit{ahankāra itīyaṃ me bhinnā prakṛitir aṣhṭadhā} ||7.4||). "Earth, water, fire, air, space, mind, intellect, and ego - these are the eight divisions of the original nature." In this list, we find most of the elements of Rāmacandra's list, except terms like \textit{māyā}, \textit{vikriyā} or \textit{rūpā}. The description of \textit{kuṇḍalinī} in \citetitle{yajnavalkya} 4.21 picks up this system. Another system of ten \textit{tattva}s appears in \textit{Uttarasūtra} 1.9 - 1.13 of the \citetitle{nishvasa2015} in which the tenfold nature of Sadāśiva is homologised with the \textit{mantra}. Here the following list is given: \textit{prakṛti}, \textit{puruṣa}, \textit{niyati}, \textit{kāla}, \textit{māyātattva}, \textit{vidyā}, \textit{Īśvara}, \textit{Sadāśiva}, \textit{dehavyāpin} and \textit{Śakti}, cf. \citeauthor[2016: 83-84]{goodall2016}. There is no similarity between the two tenfold \textit{tattva}-systems. However, since Rājayoga is rooted in ancient Śaivsim (see \citeauthor{birch2019saiva} 2019), and Rāmacandra usually tends to present simplified and transsectarian systems, the choice of a tenfold \textit{tattva}-system might be a remnant of those ancient systems.} Because of the power of Jñānayoga, the conviction arises that `the self is only one'. 
            \flushpage 
            \end{tlate}
  \end{translation}
\end{alignment}
\pagebreak %after pp. 45-46
%%%%%%%%%%%%%%%%%%%%%%%%%%%%%%%%%%%%%%%%%%
%%%%%%%%%%%%%%%%%%%%%%%%%%%%%%%%%%%%%%%%%%
%%%%%%%%PAGEBREAK%%%%%%%PAGEBREAK%%%%%%%%%
%%%%%%%%%%%%%%%%%%%%%%%%%%%%%%%%%%%%%%%%%%
%%%%%%%%%%%%%%%%PAGEBREAK%%%%%%%%%%%%%%%%%
%%%%%%%%%%%%%%%%%%%%%%%%%%%%%%%%%%%%%%%%%%
%%%%%%%%PAGEBREAK%%%%%%%PAGEBREAK%%%%%%%%%
%%%%%%%%%%%%%%%%%%%%%%%%%%%%%%%%%%%%%%%%%%
%%%%%%%%%%%%%%%%%%%%%%%%%%%%%%%%%%%%%%%%%%
%%%%%%%%%%%%%%%%%%%%%%%%%%%%%%%%%%%%%%%%%%
%%%%%%%%%%%%%%%%%%%%%%%%%%%%%%%%%%%%%%%%%%
%%%%%%%%PAGEBREAK%%%%%%%PAGEBREAK%%%%%%%%%
%%%%%%%%%%%%%%%%%%%%%%%%%%%%%%%%%%%%%%%%%%
%%%%%%%%%%%%%%%%PAGEBREAK%%%%%%%%%%%%%%%%%
%%%%%%%%%%%%%%%%%%%%%%%%%%%%%%%%%%%%%%%%%%
%%%%%%%%PAGEBREAK%%%%%%%PAGEBREAK%%%%%%%%%
%%%%%%%%%%%%%%%%%%%%%%%%%%%%%%%%%%%%%%%%%%
%%%%%%%%%%%%%%%%%%%%%%%%%%%%%%%%%%%%%%%%%%
%%%%%%%%%%%%%%%%%%%%%%%%%%%%%%%%%%%%%%%%%%
%%%%%%%%%%%%%%%%%%%%%%%%%%%%%%%%%%%%%%%%%%
%%%%%%%%PAGEBREAK%%%%%%%PAGEBREAK%%%%%%%%%
%%%%%%%%%%%%%%%%%%%%%%%%%%%%%%%%%%%%%%%%%%
%%%%%%%%%%%%%%%%PAGEBREAK%%%%%%%%%%%%%%%%%
%%%%%%%%%%%%%%%%%%%%%%%%%%%%%%%%%%%%%%%%%%
%%%%%%%%PAGEBREAK%%%%%%%PAGEBREAK%%%%%%%%%
%%%%%%%%%%%%%%%%%%%%%%%%%%%%%%%%%%%%%%%%%%
%%%%%%%%%%%%%%%%%%%%%%%%%%%%%%%%%%%%%%%%%%
\begin{alignment}[
  texts=edition[class="edition"];
  translation[class="translation"],
  ]
  \begin{edition}
    \begin{prose}[p22_02]
       \label{svabhava2}
       \noindent
%------------------------------
%yathaikaiva   pṛthvī  kvacit komalarūpā                                                   kvacit parimalarūparahitā kvacit suvarṇarūpā   kvacid raupyarūpā    \E %%%p.28 
%yathā ekaika  pṛthvī  kvacit komalarūpā                                                                                                                       \P   
%yathā ekaika  pṛthvī  kvacit komalarūpā// kvacit manohararūpā//  kvacit parimalarūpayuktā// kvacit parimalarohitā// kvacit suvarṇarūpa                        \B
%yathā ekaika  pṛthvī  kvacit komalarūpā   kvacit manohararūpāḥ// kvacit parimalarūpayuktā// kvacit parimalarahitā// kvacit suvarṇarūpā                        \L
%yathā ekaiva  pṛthivī kvacit komalarūpa/  kvacit manoharā/       kvacit parimalarūpāyuktā// kvacit parimalarahitā/  kvacit suvarṇarūpā/  kvacit rūpyarūpā/    \N1
%yathā ekaiva  pṛthivī kvacit komalarūpa   kvacit manoharā//      kvacit parimalarūpāyuktā/  kvacit parimalarohitā   kvacit suvarṇarūpa// kvacit rūpyarūpa//   \D
%yathā ekaṃ ca pṛthivī kvacit komalarūpa   kvacit manoha?rā       kvacit parimalarūpāyuktaḥ/ kvacit parimalarohitā   kvacit suvarṇarūpā   kvacit rūpyarūpa     \N2
%yathā ekai ca pṛthivī kvacit                                                                                              khavarṇakupā   kvacit rūpyarūpā     \U1
%yathā ekaika  pṛthvī  kvacit komalarūpā// kvacit manohararūpa//  kvacit parimalarūpāyuktā/  kvacit parimalarohitā// kvacit suvarṇarūpā// kvacit rajatarūpā//  \U2
%------------------------------
%As some particular soil (\textit{ekaika}) sometimes appears soft, sometimes appears beautiful, sometimes fragrant, sometimes unscented, sometimes golden, sometimes silver,... 
%------------------------------
\app{\lem[type=emendation, resp=egoscr]{yathaikaikaḥ}
  \rdg[wit={E}]{yathaikaiva}
  \rdg[wit={B,L,P,U2}]{yathā ekaika}
  \rdg[wit={D,N1}]{yathā ekaiva}
  \rdg[wit={N2}]{yathā ekaṃ ca}
  \rdg[wit={U1}]{yathā ekai ca}}
\app{\lem[wit={Y}]{pṛthvī}
  \rdg[wit={X}]{pṛthivī}}
kvacit-komala\app{\lem[wit={Y},alt={°rūpā}]{rūpā}
  \rdg[wit={X}]{°rūpa}}\dd{}
%\note[type=philcomm, labelb=145a, labele=_145e, lem={kvacit manohararūpā \ldots kvacit pītā}]{Section is omitted in \getsiglum{P}.}
\app{\lem[wit={ceteri}, alt={kvacit}]{kvaci\skp{t-ma}}
  \rdg[wit={E,P,U1}]{\om}
}\app{\lem[wit={B}, alt={manohararūpā}]{\skm{t-ma}nohararūpā}
  \rdg[wit={L}]{manohararūpāḥ}
  \rdg[wit={U2}]{manohararūpa}
  \rdg[wit={D,N1,N2}]{manoharā}
  \rdg[wit={E,P,U1}]{\om}}\dd{}
\app{\lem[wit={ceteri}, alt={kvacit}]{kvaci\skp{t-pa}}
  \rdg[wit={E,P,U1}]{\om}
}\app{\lem[wit={ceteri},alt={°parimala}]{\skm{t-pa}rimala}
  \rdg[wit={E,P,U1}]{\om}
}\app{\lem[wit={B,L},alt={°rūpayuktā}]{rūpayuktā}
  \rdg[wit={D,N1}]{°rūpā°}
  \rdg[wit={N2}]{°rūpāyuktaḥ}
  \rdg[wit={E,P,U1}]{\om}}\dd{}
\app{\lem[wit={ceteri}, alt={kvacit}]{kvaci\skp{t-pa}}
  \rdg[wit={P,U1}]{\om}
}\app{\lem[wit={ceteri},alt={°parimala}]{\skm{t-pa}rimala}
  \rdg[wit={E}]{°parimalarūpa°}
  \rdg[wit={P,U1}]{\om}
}\app{\lem[wit={E,L,N1},alt={°rahitā}]{rahitā}
  \rdg[wit={B,N2,U2}]{°rohitā}
  \rdg[wit={D,P,U1}]{\om}}\dd{}
\app{\lem[wit={ceteri}, alt={kvacit}]{kvaci\skp{t-su}}
  \rdg[wit={P,U1}]{\om}
}\app{\lem[wit={E,L,N2,U2}, alt={suvarṇarūpā}]{\skm{t-su}varṇarūpā}
  \rdg[wit={B,D}]{suvarṇarūpa}
  \rdg[wit={U1}]{khavarṇakupā}
  \rdg[wit={P}]{\om}}\dd{}
\app{\lem[wit={ceteri}, alt={kvacit}]{kvaci\skp{t-rau}}
  \rdg[wit={B,L,P}]{\om}
}\app{\lem[wit={E}, alt={raupyarūpā}]{\skm{t-rau}pyarūpā}
  \rdg[wit={N1,U1}]{rūpyarūpā}
  \rdg[wit={D,N2}]{rūpyarūpa}
  \rdg[wit={U2}]{rajatarūpā}
  \rdg[wit={B,L,P}]{\om}}\dd{}
%------------------------------
%kvacid ratnamayī   kvacic ca śvetā                                kvacidraktā   kvacitpītā    \E %%%p.28 
%                                                                                             \P   
%kvacid ratnamaī//  kvacit śverūpā// kvacitkṛṣṇā//                 kvacidraktā/  kvacitpītā//  \B
%kvacid ratnamaī//  kvacit śvetarūpā kvacitkṛṣṇā//                 kvacidraktā// kvacitpītā//  \L
%kvacid ratnamayī/  kvacit śveta/    kvacitkṛṣṇa??/                kvacidrakta/  kvacitpītā/   \N1
%kvacid ratnamayī// kvacit śvetā//   kvacitkṛṣṇā [S8., Z.7]        kvacidrakta   kvacitpītā//  \D
%kvacid ratnamayī   kvacit śveta     kvacitkṛṣṇā// [S6. verso]     kvacidrakta   kvacitpītā    \N2
%kvacid ratnamayī   kvacit śveta     kvacitkṛṣṇā                   kvacidrakta   kvacitpītā    \U1
%kvacid ratnamayī// kvacit śvetā//   kvacitkṛṣṇā//                 kvacidraktā// kvacitpītā//  \U2
%------------------------------
% ... manchmal aus Edelstein gemacht ist, manchmal weiß erscheint, manchmal schwarz, manchmal kupfern, manchmal gelb,
%
%... is sometimes made of precious stone, sometimes appearing white, sometimes black, sometimes copper, sometimes yellow, 
%------------------------------
kvaci\skp{d-ra}\app{\lem[wit={ceteri},alt={ratnamayī}]{\skm{d-ra}tnamayī}
  \rdg[wit={B,L,P}]{ratnamaī}}\dd{}
\app{\lem[wit={ceteri}, alt={kvacit}]{kvaci\skp{t-śve}}
  \rdg[wit={E}]{kvacic ca}
  \rdg[wit={P}]{\om}
}\app{\lem[wit={E,D,U2}, alt={śvetā}]{śvetā}
  \rdg[wit={N1,N2,U1}]{śveta}
  \rdg[wit={L}]{śvetarūpā}
  \rdg[wit={B}]{śverūpā}
  \rdg[wit={P}]{\om}}\dd{}
\app{\lem[wit={ceteri}, alt={kvacit kṛṣṇā}]{kvacit-kṛṣṇā}
  \rdg[wit={N1}]{kṛṣṇa}
  \rdg[wit={E,P}]{\om}}\dd{}
\app{\lem[wit={B,E,L,U2},alt={kvacid raktā}]{kvacid-raktā}
  \rdg[wit={ceteri}]{kvacid rakta}
  \rdg[wit={P}]{\om}}\dd{}
\app{\lem[wit={ceteri}, alt={kvacit pītā}]{kvacit-pītā} 
  \rdg[wit={P}]{\om}}\dd{}\linelabel{_145e}       
%------------------------------
%kvacitkarburā   kvacin nānāvidharūpā        kvacid viṣarūpā    kvacid amṛtarūpamayī svabhāvata eva bhavati//  \E  %%%p.28
%                                                               kvacid amṛtamayī     svabhāvata eva bhavati    \P  %%%rest is \om
%kvacitkarburā// kvacin nānāvidhaphalarūpā   kvacit viṣarūpā//  kvacid amṛtamaī/     svabhāvata eva bhavataḥ// \B
%kvacitkarburā// kvacin nānāvidhāphalarūpā   kvacit viṣarūpā//  kvacid amṛtamaī//    svabhāvata eva bhavataḥ// \L
%kvacitkarburā,  kvacin nānāvidhaphalarūpā/  kvacid puṣparūpā,  kvacid amṛtamayī     svabhāvata eva bhavati/   \N1
%kvacitkarburā   kvacin nānāvidhaphalarūpā// kvacid puṣparūpā// kvacid amṛtamayī/    svabhāvata eva bhavati//  \D
%kvacitkarburā   kvacin nānāvidhaphalarūpā                      kvacid amṛtamayī/    svabhāvata eva bhavati//  \N2
%kvacitkarpurā   kvacin nānāvidhophalarūpā   kvacid ....[rest omitted]                                         \U1
%kvacitkarburā// kvacit nānāvidhaphalarūpā// kvacir viśarūpā//  kvacit amṛtamayī//   svabhāvata eva bhavati//  \U2
%------------------------------
%machmal gesprenkelt, machmal wie verschiedenartige Frucht erscheint, manchmal wie Blumen erscheint, machmal wie der Nektar der Unsterblichkeit erscheint, [und das nur] nur aufgrund seiner inhärenten Natur.
%------------------------------
%sometimes mottled, sometimes appearing like various fruit, sometimes appearing like flowers, sometimes appearing like the nectar of immortality, only because of its inherent being. 
%------------------------------
\app{\lem[wit={ceteri}, alt={kvacit karburā}]{kvacit-karburā}
  \rdg[wit={U1}]{kvacit karpurā}
  \rdg[wit={P}]{\om}}\dd{}
\app{\lem[wit={ceteri}]{kvaci\skp{n-nā}}
  \rdg[wit={U2}]{kvacit}
  \rdg[wit={P}]{\om}
}\app{\lem[wit={ceteri},alt={nānāvidhaphalarūpā}]{\skm{n-nā}nāvidhaphalarūpā}
  \rdg[wit={U1}]{nānāvidhophalarūpā}
  \rdg[wit={E}]{nānāvidharūpā}
  \rdg[wit={P}]{\om}}\dd{}
\app{\lem[wit={B,L},alt={kvacit}]{kvaci\skp{t-pu}}
  \rdg[wit={D,N1,U1}]{kvacid}
  \rdg[wit={U2}]{kvacir}
  \rdg[wit={P,N2}]{\om}
}\app{\lem[wit={D,N1},alt={puṣparūpā}]{\skm{t-pu}ṣparūpā}
\rdg[wit={B,E,L}]{viṣarūpā}
\rdg[wit={U2}]{viśarūpā}
\rdg[wit={U1,P}]{\om}}\dd{}
\app{\lem[wit={ceteri}, alt={kvacid}]{kvaci\skp{d-a}}
  \rdg[wit={U2}]{kvacit}
  \rdg[wit={U1}]{\om}
}\app{\lem[wit={ceteri},alt={amṛtamayī}]{\skm{d-a}mṛtamayī}
  \rdg[wit={E}]{amṛtarūpamayī}
  \rdg[wit={B,L}]{amṛtamaī}
  \rdg[wit={U1}]{\om}}\dd{}
\app{\lem[wit={ceteri}]{svabhāvata}
  \rdg[wit={U1}]{\om}}
\app{\lem[wit={ceteri}]{eva}
  \rdg[wit={U1}]{\om}}
\app{\lem[wit={ceteri}]{bhavati}
  \rdg[wit={B,L}]{bhavataḥ}
  \rdg[wit={U1}]{\om}}\dd{}
%------------------------------
%tathaivātmā   manuṣyapakṣihariṇahastividyādharagandharvakinnaramahāpaṃḍitamahāmūrkha  rogyarogikrodhi---śāṃtarūpaḥ      svabhāvād eva bhavati/ \E
%tathaivātmā   manuṣyapakṣihariṇāhastividyādharagaṃdharvakinnaramahāpiṃḍitamahārmūkha  rogī-----krodhi---śāṃtarūpāḥ      svabhāvād eva bhavati \P
%tathaivātmā// manuṣyapakṣihariṇahastividyādharagaṃdharvakinnaramahāpiṃḍatamahāmūrkha  rogī-----krodhadhiśāṃtarūpaḥ      svabhāvād eva bhavatī/ \B
%tathaivātmā   manuṣyapakṣihariṇahastividyādharagaṃdharvakinnaramahāpaṃḍitamahāmūrkha  rogī-----krodhadhīśāṃtarūpāḥ      svabhāvād eva bhavatī/ \L
%tathātmā//    manuṣyapakṣihariṇahastīvidyādharagandharvakiṃnaramahāpaṃḍitamahāmūrva   rogīarogīkrodhī---śāntarūpa-------svabhāvād eva bhati/ \N1 %%%%%%%CRAZY SWITCH BETWEEN DAṆḌA AND COMMA
%tathātmā//    manuṣyapakṣihariṇahastīvidyādharagandharvakinnaramahāpaṃḍitamahāmūrva   rogīarogīkrodhī---śāṃtarūpa-------svabhāvād eva bhavati/ \D
%tathātmā//    manuṣyapakṣihariṇahastividyādharagandharvakinnaramahāpaṇḍitamahāmūrkha  rogīarogīkrodhī---śāṃtarūpa-------svabhāvād eva bhavati/ \N2
%                                     vidyādharagaṃdharvakinnaramahāpaṇḍitamahāmūrṣa   rogīarogīkrodhī---śāṃtarūpa       evaṃ svabhāvaṃ dharati  \U1
%tathaivātmā   manuṣyapakṣihariṇahastividyādharagaṃdharvakinnaramahāpaṃḍitamahāmūrkha  rogīarogīkrodhi---śāṃtarūpaḥ      svabhāvād eva bhavati// \U2 %%%410.jpg
%------------------------------
%Auf diese Weise nimmt auch das Selbst aufgrund seiner inhärenten Natur die Form eines Menschen, Vogels, einer Gazelle, eines Elefants, eines Vidyādharas, eines Gandharvas, Zentauren, eines großen Gelehrten oder großen Dummkopfes, eines Kranken oder Gesunden, eines Zornigen oder Friedlichen an.
%
%In the same way, the self also takes the form of a human, a bird, a gazelle, an elephant, a vidyādhara, a gandharva, a centaur, great scholar or a great fool, a sick or healthy, an angry or or peaceful person, by virtue of its inherent being.       
%------------------------------      
\note[type=source, labelb=146, labele=_146e, nosep]{cf. YSv (PT p. 836): tathā ca devagandharvakinnarādyāḥ khagādayaḥ | sukhasampiṇḍito rogī tathaiva krodhaśāntadhīḥ | aśeṣarūpabalito nānābuddhirataḥ svayam | devatattvaṃ bhūtaśaktyā jīvasaṃjñā bhramātmikā | jñānayogī nirvikāro nistāpa eka īśvaraḥ | ātmaikamūrttimān bhūtvā nirvikalpo nirañjanaḥ | sukhī duḥkhī mohayukto 'nantacetāḥ svabhāvataḥ |}
\app{\lem[wit={Y}, alt={tathaivātmā}]{tathaivātmā}
  \rdg[wit={X}]{tathātmā}}
\app{\lem[wit={ceteri},alt={manuṣya°}]{manuṣya}
  \rdg[wit={U1}]{\om}
}\app{\lem[wit={ceteri},alt={°pakṣi°}]{pakṣi}
  \rdg[wit={U1}]{\om}
}\app{\lem[wit={ceteri},alt={°hariṇa°}]{hariṇa}
  \rdg[wit={P}]{°hariṇā°}
  \rdg[wit={U1}]{\om}
}\app{\lem[wit={D,N1},alt={°hastī°}]{hastī}
  \rdg[wit={ceteri}]{hasti}
  \rdg[wit={U1}]{\om}
}vidyādharagaṃdharvakinnaramahā\app{\lem[wit={ceteri},alt={°paṇḍita°}]{paṇḍita}
  \rdg[wit={B}]{piṃḍata}
}mahā\app{\lem[wit={ceteri},alt={°mūrkha°}]{mūrkha}
  \rdg[wit={P}]{°rmūkha°}
  \rdg[wit={D,N1}]{°mūrva°}
  \rdg[wit={U1}]{°mūrṣa°}
}\app{\lem[type=emendation, resp=egoscr]{rogyarogī}
  \rdg[wit={E}]{°rogyarogi}
  \rdg[wit={X,U2}]{°rogī arogī}
  \rdg[wit={B,L,P}]{°rogī}
}\app{\lem[wit={ceteri},alt={°krodhī°}]{krodhī}
  \rdg[wit={E,P}]{°krodhi°}
  \rdg[wit={B,L}]{°krodha°}
}\app{\lem[wit={ceteri},alt={°śānta°}]{śānta}
  \rdg[wit={B,L}]{°dhiśānta°}
}\app{\lem[wit={ceteri},alt={°rūpaḥ}]{rūpaḥ}
  \rdg[wit={P,L}]{°rūpāḥ}
  \rdg[wit={X}]{°rūpa}}
\app{\lem[wit={ceteri},alt={svabhāvād eva}]{svabhāvād-eva}
  \rdg[wit={U1}]{evaṃ svabhāvaṃ}}
\app{\lem[wit={ceteri}]{bhavati}
  \rdg[wit={B,L}]{bhavatī}
  \rdg[wit={N1}]{bhati}
  \rdg[wit={D}]{dharati}}\dd{}
%------------------------------      
%jñānayogādhikārarūparahito  jñāyate/  yathā plakṣasyotpattiḥ/ sthānam eva bhavati// \E
%jñānayogādhikārarūparahito  jñāyate   yathā phalasyotpattisthānam ekam eva bhavati \P  %%%7643.jpg        
%jñānayogādhikārarūparahito  jñāyate// yathā phalasyotpattisthānam ekam eva bhavatī// \B
%jñānayogādhikārarūparahito  jñāyate// yathā phalasyotpattisthānam ekam eva bhavati// \L
%jñānayogād vikārarūparahito jñāyate/  yathā phalasyotpattisthānam ekam eva bhavati/ \N1
%jñānayogādhikārarūparahito  jñāyate// yathā phalasyotpattisthānam ekaseva  bhavati// \D
%jñānayogadhikārarūparahito  jñāyate// yathā phalasyotpattisthānam eva kameva bhavati// \N2
%jñānayogāt vikārarūparahito jñāyate   yathā phalasyotpattisthāna  ekam eva ti \U1
%jñānayogādhikārarūparahito  jāyate//  yathā phalasyotpattisthānam ekam eva bhavati// \U2
%------------------------------
%Through Jñānayoga he realizes the emptiness of the mutability of form. Just as the place of origin of the fruit is%%only one.
%------------------------------
\app{\lem[wit={N1,U1}, alt={jñānayogād vikāra}]{jñānayogād-vikāra}
  \rdg[wit={ceteri}]{jñānayogadhikāra}
}rūparahito
\app{\lem[wit={ceteri}]{jñāyate}
  \rdg[wit={U2}]{jāyate}}/
\end{prose}
  \end{edition}
  \begin{translation}
    \begin{tlate}[p22_02]
      \noindent
Just as some particular soil sometimes appears soft, sometimes appears beautiful, sometimes fragrant, sometimes unscented, sometimes golden, sometimes silver, 
is sometimes made of precious stone, sometimes appearing white, sometimes black, sometimes copper, sometimes yellow, sometimes mottled, sometimes like various fruit, 
sometimes like flowers, sometimes like the nectar of immortality, [and that only] because of its inherent being. 
In this way, the self also takes the form of a human, a bird, a gazelle, an elephant, a Vidyādhara, a Gandharva, a centaur, a great scholar or a great fool, 
a sick or healthy, an angry or peaceful person, by virtue of its inherent nature. Through Jñānayoga he realises the emptiness of the mutability of form. 
\flushpage
\end{tlate}
  \end{translation}
\end{alignment}
\pagebreak %after pp. 47-48
%%%%%%%%%%%%%%%%%%%%%%%%%%%%%%%%%%%%%%%%%%
%%%%%%%%%%%%%%%%%%%%%%%%%%%%%%%%%%%%%%%%%%
%%%%%%%%PAGEBREAK%%%%%%%PAGEBREAK%%%%%%%%%
%%%%%%%%%%%%%%%%%%%%%%%%%%%%%%%%%%%%%%%%%%
%%%%%%%%%%%%%%%%PAGEBREAK%%%%%%%%%%%%%%%%%
%%%%%%%%%%%%%%%%%%%%%%%%%%%%%%%%%%%%%%%%%%
%%%%%%%%PAGEBREAK%%%%%%%PAGEBREAK%%%%%%%%%
%%%%%%%%%%%%%%%%%%%%%%%%%%%%%%%%%%%%%%%%%%
%%%%%%%%%%%%%%%%%%%%%%%%%%%%%%%%%%%%%%%%%%
%%%%%%%%%%%%%%%%%%%%%%%%%%%%%%%%%%%%%%%%%%
%%%%%%%%%%%%%%%%%%%%%%%%%%%%%%%%%%%%%%%%%%
%%%%%%%%PAGEBREAK%%%%%%%PAGEBREAK%%%%%%%%%
%%%%%%%%%%%%%%%%%%%%%%%%%%%%%%%%%%%%%%%%%%
%%%%%%%%%%%%%%%%PAGEBREAK%%%%%%%%%%%%%%%%%
%%%%%%%%%%%%%%%%%%%%%%%%%%%%%%%%%%%%%%%%%%
%%%%%%%%PAGEBREAK%%%%%%%PAGEBREAK%%%%%%%%%
%%%%%%%%%%%%%%%%%%%%%%%%%%%%%%%%%%%%%%%%%%
%%%%%%%%%%%%%%%%%%%%%%%%%%%%%%%%%%%%%%%%%%
%%%%%%%%%%%%%%%%%%%%%%%%%%%%%%%%%%%%%%%%%%
%%%%%%%%%%%%%%%%%%%%%%%%%%%%%%%%%%%%%%%%%%
%%%%%%%%PAGEBREAK%%%%%%%PAGEBREAK%%%%%%%%%
%%%%%%%%%%%%%%%%%%%%%%%%%%%%%%%%%%%%%%%%%%
%%%%%%%%%%%%%%%%PAGEBREAK%%%%%%%%%%%%%%%%%
%%%%%%%%%%%%%%%%%%%%%%%%%%%%%%%%%%%%%%%%%%
%%%%%%%%PAGEBREAK%%%%%%%PAGEBREAK%%%%%%%%%
%%%%%%%%%%%%%%%%%%%%%%%%%%%%%%%%%%%%%%%%%%
%%%%%%%%%%%%%%%%%%%%%%%%%%%%%%%%%%%%%%%%%%
\begin{alignment}[
  texts=edition[class="edition"];
  translation[class="translation"],
  ]
  \begin{edition}
    \begin{prose}[p22_03]
      \noindent
      yathā
      \app{\lem[wit={ceteri}]{phalasyotpatti}
        \rdg[wit={E}]{plakṣasyotpattiḥ}
      }\app{\lem[wit={ceteri},alt={°sthānam}]{sthāna\skp{m-e}}
        \rdg[wit={E}]{sthānam}
        \rdg[wit={U1}]{°sthāna}
      }\app{\lem[wit={ceteri},alt={ekam}]{\skm{m-e}ka\skp{m-e}}
        \rdg[wit={D}]{ekas}
        \rdg[wit={N2}]{eva}
        \rdg[wit={E}]{\om}
      }\app{\lem[wit={ceteri},alt={eva}]{\skm{m-e}va}
        \rdg[wit={N2}]{kam eva}}
      \app{\lem[wit={ceteri}]{bhavati}
        \rdg[wit={B}]{bhavatī}
        \rdg[wit={U1}]{ti}}/
      %------------------------------
      %atha ca phalasya gatir bahudhā dṛśyate/ \E
      %atha ca phalasya gati  bahudhā dṛśyate    \P
      %atha ca phalasya gatir bahudhā dṛśyate// \B
      %atha ca phalasya gatir bahudhā dṛśyate// \L
      %atha ca phalasya gatir bahudhā dṛśyate/ \N1
      %atha ca phalasya gatir bahudhā dṛśyate// \D
      %atha ca phalasya gati  bahudhā dṛśyate/ \N2
      %atra ca phalasya gati  bahudhā dṛśyate \U1
      %atha ca phalasya gatir bahudhā dṛśyate// \U2
      %------------------------------
      %But the development of the fruit is seen manifold. 
      %------------------------------
      atha ca phalasya \app{\lem[wit={ceteri},alt={gatir}]{gati\skp{r-ba}}
        \rdg[wit={P,N2,U1}]{gati}
      }\skm{r-ba}hudhā dṛśyate\dd{}\\\\
      %------------------------------ %%%STEMMAPOINT śuklaṃ//śuṣkaṃ
      % ekaṃ phalaṃ pṛthvīmadhye  patati/  śuklaṃ bhavati/   \E
      % ekaṃ phalaṃ pṛthvīmadhye  patati   śuklaṃ bhavati    \P
      % ekaṃ phalaṃ pṛthvīmadhye  patiśuklaṃ      bhavatī//  \B
      % ekaṃ phalaṃ pṛthvīmadhye  patati   śuṣkaṃ bhavatī    \L
      % ekaṃ phala--pṛthvīmadhye  patati/  śuklaṃ bhavati/   \N1 %%%p.7 recto letzte Zeile 
      % ekaṃ phala--pṛthvīmadhye  patati// śuklaṃ bhavati//  \D
      % eva  phala--pṛthvīmadhye  patati   śuklaṃ bhavati//  \N2
      % ekaṃ phalaṃ pṛthivīmadhye patati   śuṣkaṃ bhavati    \U1
      % ekaphalaṃ   pṛthvīmadhye  patati// śuṣkaṃ bhavati//  \U2
      %------------------------------
      %One fruit falls onto the ground, and becomes dry. 
      %------------------------------
      \app{\lem[wit={ceteri}]{ekaṃ}
        \rdg[wit={U2}]{eka°}
        \rdg[wit={N2}]{eva}}
      \app{\lem[wit={ceteri}]{phalaṃ}
        \rdg[wit={D,N1,N2}]{phala°}}
      \app{\lem[wit={ceteri},alt={pṛthvī°}]{pṛthvī}
        \rdg[wit={U1}]{pṛthivī°}
      }madhye patati/
      \app{\lem[wit={L,U1,U2}]{śuṣkaṃ}
        \rdg[wit={ceteri}]{śuklaṃ}}
      \app{\lem[wit={ceteri}]{bhavati}
        \rdg[wit={B}]{bhavatī}}/
      %------------------------------
      % ekasya phalasya makaraṃdaṃ bhramaraḥ  pibati/  \E
      % ekasya phalasya makaraṃdaṃ bhramaraḥ  pibaṃti  \P
      % ekasya            karaṃdaṃ bhramaraṃ  pibatī/  \B
      % ekasya          makaraṃdaṃ bhramaraṃ  pibati   \L
      % ekasya phalasya makaraṃdabhramaraḥ    pibati/  \N1 %%%p.7 recto letzte Zeile 
      % ekasya phalasya makaraṃdabhramaraḥ    pibati/  \D
      % ekasya phalasya makaraṃdaṃ bhramara   pibati/  \N2
      % ekasya phalasya makaraṃdaṃ bhramanaḥ  pibati   \U1
      % ekasya phalasya makaraṃdaṃ bhramaraḥ  pibati// \U2
      % ------------------------------
      % Eine Biene trinkt den Saft der einen Frucht.
      % A bee drinks the juice of the one fruit.     
      %------------------------------
      ekasya
      \app{\lem[wit={ceteri}]{phalasya}
        \rdg[wit={P,L}]{\om}} 
      \app{\lem[wit={E,L,P,N2,U1,U2}]{makarandaṃ}
        \rdg[wit={L,N1}]{makaraṃda°}
        \rdg[wit={B}]{karaṃdaṃ}}
      \app{\lem[wit={ceteri}]{bhramaraḥ}
        \rdg[wit={B,L}]{bhramaraṃ}
        \rdg[wit={N2}]{bhramara}}
      \app{\lem[wit={ceteri}]{pibati}
        \rdg[wit={P}]{pibaṃti}
        \rdg[wit={B}]{pibatī}}/
      %------------------------------
      % ekasya phalasya  mālāṃ kāminī tuṃgakucamaṃḍalopari dadhāti/ \E
      % ekasya phalasya  mālāṃ kāminī tuṃgakucamaṃḍalopari dadhāti \P
      % ekasya phalasya  mālāṃ kāminī tuṃgakucamaṃḍalopari dadhātī// \B
      % ekasya phalasya  mālāṃ kāminī tuṃgakucamaṃḍalopari dadhāti// \L
      % ekasya phalasya  mālāṃ kāminī tuṃgakucamaṃḍalopari dadhāvati/ \N1 %%%p.7 recto letzte Zeile 
      % ekasya phalasya  mālāṃ kāmibī tuṃgakucamaṇḍalopari dadhāti// \D
      % ekasya phalasyaṃ mālākāminī   tuṃgakucamaṇḍalopari dadhovati// \N2
      % ekasya phalasya  mālāṃ kāmini tuṃ  kucamaṃḍalopari dadhāti \U1
      % ekasya phalasya  mālāṃ kāminī tuṃgakucamaṃḍalopari dadhāti// \U2
      %------------------------------
      % of the one fruit Blütenkranz/Girlande die Verliebte (biene) führt ein unmittelbar über dem Kreis des Blütenstempels der wie eine Brust ist ein.  %tu.mga = hervorstehend 
      % Die [nach Blumensaft] Verlangende [Biene] platziert sich auf dem Blütenkranz über dem emportstehenden kreisförmigen Blütenstempel.
      %[Or] a woman places a garland [made of] the one fruit above her voluptuous bosom.   
      %------------------------------
      ekasya
      \app{\lem[wit={ceteri}]{phalasya}
        \rdg[wit={N2}]{phalasyaṃ}}
      \app{\lem[wit={ceteri}]{mālāṃ}
        \rdg[wit={N2}]{mālā°}}
      \app{\lem[wit={ceteri}]{kāminī}
        \rdg[wit={D}]{kāmibī}}
      \app{\lem[wit={ceteri},alt={tuṅga°}]{tuṅga}
        \rdg[wit={U1}]{tuṃ°}
      }kucamaṇḍalopari
      \app{\lem[wit={ceteri}]{dadhāti}
        \rdg[wit={N1}]{dadhāvati}
        \rdg[wit={N2}]{dadhovati}}/\linelabel{_146e}
      %------------------------------ 
      %ekaṃ phalaṃ mṛtamanuṣyopari   kṣipyate/  ayaṃ vastunaḥ svabhāvaḥ/  tathā eka  evātmā   svīyabhāvād evāṣṭau    bhogān  bhunakti/ \E
      %ekaṃ phalaṃ mṛtamanuṣyopari   kṣipyate   ayaṃ vastunaḥ svabhāvaḥ   tathā eka  evātmā   svīyabhāvād evāṣṭau    bhogān  bhunakti \P
      %ekaṃ phalaṃ mṛtamanuṣyopari   kṣapyate// ayaṃ vastunaḥ svabhāvaḥ/  tathā eka  evātmā   svabhāvād   evāṣṭau    bhogān  bhunakte// \B
      %ekaṃ phalaṃ mṛtamanuṣyopari  kṣipyate//  ayaṃ vastunaḥ svabhāvaḥ   tathā eka  evātmā   svabhāvād   evāṣṭau    bhogān  bhunakte// \L
      %ekaphalaṃ   mṛtamanuṣyopari   kṣipyate// ayaṃ vastunaḥ svabhāvaḥ/  tathā eka  evātmā   svīyabhāvād evāṣṭau    bhogān ābhunakti/ \N1
      %ekaphalaṃ   mṛtamanuṣyopari  kṣipyate//  ayaṃ vastunaḥ svabhāvaḥ// tathā eka  evātmā   svīyabhāvād evāṣṭau    bhogān  bhunakti// \D
      %ekaphalaṃ   mṛtamanuṣyopari   kṣipyate/  ayaṃ castunaḥ svabhāvaḥ/  tathā eka  evātmā    svīyabhāvād evāstau   bhogāt  bhunakti/ \N2
      %ekaphalaṃ   mṛtamanuṣyopari   kṣipyate/  ayaṃ castunaḥ svabhāvaḥ/  tathā eka  evātmā    svīyabhāvād evāstau   bhogāt  bhunakti/ \U1 %%%276.jpg
      %ekaṃ phalaṃ mṛtamanuṣyopari   kṣipyate// ayaṃ vastunaḥ svabhāvaḥ// tathā ekam eva ātmā svīyabhāvād evāṣṭabhogān    bhunakti// \U2
      %------------------------------
      %[Or] the one fruit is thrown onto a dead man. Dies ist das inhärente Wesen der Sache. So [ist es das inhärente Wesen der Sache] das [auch] eine Selbst aufgrund des eigenen Wesens die acht Genüsse genießt.  
      %------------------------------
      \note[type=source, labelb=147, nosep]{cf. YSv (PT p. 837): strīpuṃrūpī mahān so hi parasparavimohitaḥ | amanaskaḥ svīyabhāvāt jñānayogī nirākulaḥ | srakcandanādivāmāsu svabhāvād bhogam icchukaḥ |}
      \app{\lem[wit={Y}]{ekaṃ phalaṃ}
        \rdg[wit={X}]{ekaphalaṃ}
      }mṛtamanuṣyopari
      \app{\lem[wit={ceteri}]{kṣipyate}
        \rdg[wit={B}]{kṣapyate}}/
      ayaṃ vastunaḥ svabhāvaḥ/
      tathā \app{\lem[wit={ceteri}]{eka}
        \rdg[wit={U2}]{ekam}}
      \app{\lem[wit={ceteri}]{evātmā}
        \rdg[wit={U2}]{eva ātmā}}
      \app{\lem[wit={ceteri},alt={svīyabhāvād}]{svīyabhāvā\skp{d-e}}
        \rdg[wit={B,L}]{svabhāvād}
      }\app{\lem[wit={ceteri},alt={evāṣṭau}]{\skm{d-e}vāṣṭau}
        \rdg[wit={N2,U1}]{evāstau}
        \rdg[wit={U2}]{evāṣṭa}}
      \app{\lem[wit={ceteri}, alt={bhogān}]{bhogā\skp{n-bhu}}
        \rdg[wit={N2,U1}]{bhogāt}
      }\app{\lem[wit={ceteri}, alt={bhunakti}]{\skm{n-bhu}nakti}
        \rdg[wit={N1}]{ābhunakti}}/
        \app{\lem[wit={ceteri}]{ke te}
      \rdg[wit={B,L}]{\om}
      }\app{\lem[wit={ceteri}]{'ṣṭau}
      \rdg[wit={B,L}]{aṣṭau}
      \rdg[wit={U1}]{ṣṭe}}
      \app{\lem[wit={ceteri}]{bhogāḥ}
      \rdg[wit={P}]{bhobauḥ}
      \rdg[wit={U1,U2}]{bhogā}}\dd{}
    \end{prose}
%------------------------------
%ke te ṣṭau  bhogāḥ – suvāsaś ca   suvastrañ ca  suśayyā    sunitaṃbinī/       susthānañ cānnapānāni    aṣṭau bhogāś ca dhīmatām/      \E
%ke te ṣṭau  bhogauḥ  suvāsaś ca   suvāsaś   ca  suyyā      sunitāṃbinīḥ//     susthānāś cānpanānp------aṣṭau bhogāś ca dhīmatāṃ 1     \P %%%7643.jpg
%      aṣṭau bhogāḥ   suvāsac ca   suvasaś   ca  suśayyāḥ   sūnitaṃbinī/       susthānaś vānnapānāny----aṣṭau bhogāś cā sudhīmatām//1//\B
%      aṣṭau bhogāḥ   suvāsaś ca   suvāsaś   ca  suśayyāḥ   sūnitaṃbinī//      susthānāś cānnapānāny----aṣṭau bhogāś cā sudhīmatāṃ//1// \L
%ke te ṣṭau  bhogāḥ// suvāyaś ca/                suśayyā    sunitaṃbinī/       susthātāś cātmapanasyā----ṣṭau bhogāḥ    sudhipaṇa\N1
%ke te ṣṭau  bhogāḥ// suvāyaś ca//               suśayyā    sunittaṃbinī//     susthātāś cānmanasyā------ṣṭau bhogāḥ    sudhiṣaṇa \D
%ke te ṣṭau  bhogāḥ   suvāyaś ca                 suśayya    sunitaṃbinī/       susthānāś cānmanasyā------ṣṭau bhogāḥ    sudhiyane \N2
%ke te ṣṭe   bhogā –  suvāsaś ca                 suśayyā ca sunītavinīta       susthātāś cānnapānaḥ syādaṣṭau bhogāḥ   sudhiṣaṇāṃ\U1
%ke te aṣṭau bhogā // suvāsaś ca// suvaṃśaś ca// suśayyā//  sunitaṃbinī//      sudehaṃ// sukhasaṃtānaṃ// abhayādicāṣṭakaṃ//  \U2
%------------------------------
%What are the eight enjoyments? A nice perfume, good clothing, a good bed, a beautiful womna, a nice dwelling, food & drink. Those are the eight enjoyments of the wise. Clothes made from silk.
%------------------------------
\begin{tlg}[22_1]
\noindent
\tl{\app{\lem[wit={ceteri},alt={suvāsaś ca}]{suvāsaś-ca}
\rdg[wit={B}]{suvāsac ca}}
\app{\lem[wit={E},alt={suvastrañ ca}]{suvastrañ-ca}
\rdg[wit={U2}]{suvaṃśaś ca}}
\app{\lem[wit={ceteri}]{suśayyā}
\rdg[wit={U1}]{suśayyā ca}
\rdg[wit={B,L}]{suśayyāḥ}
\rdg[wit={P}]{suyyā}}
\app{\lem[wit={ceteri}]{sunitaṃbinī}
\rdg[wit={P}]{sunitāṃbinīḥ}
\rdg[wit={U1}]{sunītavinīta}}/}\\
\tl{\app{\lem[wit={E},alt={susthānañ ca}]{susthāna\skp{ñ-cā}}
\rdg[wit={P,L,N2}]{susthānāś}
\rdg[wit={D,N1,U1}]{susthātāś}
\rdg[wit={U2}]{sudehaṃ}
}\app{\lem[wit={L},alt={°ānnapānāny}]{\skm{ñ-cā}nnapānā\skp{ny-a}}
\rdg[wit={B}]{vānnapānāny}
\rdg[wit={E}]{cānnapānāni}
\rdg[wit={P}]{cānpanānp°}
\rdg[wit={N1}]{cātmapanasyā°}
\rdg[wit={D,N2}]{cānmanasyā°}
\rdg[wit={U1}]{cānnapānaḥ syād°}
\rdg[wit={U2}]{sukhasaṃtānaṃ}
}\app{\lem[wit={E,P},alt={aṣṭau bhogāś ca dhīmatām}]{\skm{ny-a}ṣṭau bhogāś-ca dhīmatām}
\rdg[wit={B,L}]{aṣṭau bhogāś cā sudhīmatām}
\rdg[wit={N1}]{ṣṭau bhogāḥ sudhipaṇa°}
\rdg[wit={D}]{ṣṭau bhogāḥ sudhiṣaṇa°}
\rdg[wit={U1}]{aṣṭau bhogāḥ sudhiṣaṇāṃ}
\rdg[wit={N2}]{aṣṭau bhogāḥ}
\rdg[wit={U2}]{abhayādicāṣṭakaṃ}}\dd{} \begin{otherlanguage}{english}\uproman{22}.1\end{otherlanguage} \dd{}}
\end{tlg}
  \end{edition}
  \begin{translation}
    \begin{tlate}[p22_03]
      Just as the place of origin of the fruit is only one, but the fruit's actions and outcomes are seen as manifold.\\
      
      One fruit falls onto the ground and becomes dry. A bee drinks the fruit's juice; 
      [or] a woman places a garland made of the fruit over her voluptuous bosom; 
      [or] the fruit is thrown onto a dead person. This is the inherent being of the thing. 
      Thus [in the same way], the one self enjoys eight enjoyments due to its own being.
      
      \footnote{The passage appears to describe a single object's multifaceted nature, using fruit as an example. 
      The author suggests that even though the fruit's origin is singular, the actions and outcomes that can arise from it are many and diverse. 
      This can be seen in the various examples, such as the fruit falling onto the ground, a bee drinking the fruit's juice, 
      a woman adorning herself with a garland made of the fruit or the fruit being thrown onto a dead person. 
      The final statement about the self enjoying eight enjoyments suggests that just as the fruit can have different outcomes and experiences, 
      the self can also have different experiences and enjoyments. The passage emphasizes the multifaceted and diverse nature of things and experiences.
       Additionally, the passage suggests that the eight enjoyments are perfectly natural to Rāmacandra and his readership.}
      \\
      What are the eight enjoyments?\footnote{I was not able to allocate the source of the \textit{aṣṭau bhogāḥ} yet.
       In the \textit{Mānasollāsa} of King Someśvara, one finds the mention of twenty royal \textit{upabhoga}s, which, however,
      includes all of the eight pleasures in greater detail \parencite[5]{manasollasa}. 
      This alludes to the possibility of an exceptionally wealthy lifestyle for Rāmacandra's audience.}
      \end{tlate}
    \begin{tlate}[22_1]
      \paragraph{\uproman{22}.1} A good perfume, fine clothing, a good bed, a beautiful woman, a good dwelling (\textit{susthāna})
       food and drink.\footnote{Surprisingly, the verse only gives seven enjoyments. 
       The horse is lacking compared to the list given a little later.} 
       Those are the eight enjoyments of the wise.\footnote{Right after the list presented on the next page, 
       Rāmacandra teaches that the eight enjoyments cause suffering and attachment. 
       However, the end of the verse with \textit{aṣṭau bhogāś ca dhīmatām}, ``the eight enjoyments of the wise/clever person''
        suggests a rather positive connotation.}
      \flushpage
        \end{tlate}
  \end{translation}
\end{alignment}
\pagebreak %after pp. 49-50
%%%%%%%%%%%%%%%%%%%%%%%%%%%%%%%%%%%%%%%%%%
%%%%%%%%%%%%%%%%%%%%%%%%%%%%%%%%%%%%%%%%%%
%%%%%%%%PAGEBREAK%%%%%%%PAGEBREAK%%%%%%%%%
%%%%%%%%%%%%%%%%%%%%%%%%%%%%%%%%%%%%%%%%%%
%%%%%%%%%%%%%%%%PAGEBREAK%%%%%%%%%%%%%%%%%
%%%%%%%%%%%%%%%%%%%%%%%%%%%%%%%%%%%%%%%%%%
%%%%%%%%PAGEBREAK%%%%%%%PAGEBREAK%%%%%%%%%
%%%%%%%%%%%%%%%%%%%%%%%%%%%%%%%%%%%%%%%%%%
%%%%%%%%%%%%%%%%%%%%%%%%%%%%%%%%%%%%%%%%%%
%%%%%%%%%%%%%%%%%%%%%%%%%%%%%%%%%%%%%%%%%%
%%%%%%%%%%%%%%%%%%%%%%%%%%%%%%%%%%%%%%%%%%
%%%%%%%%PAGEBREAK%%%%%%%PAGEBREAK%%%%%%%%%
%%%%%%%%%%%%%%%%%%%%%%%%%%%%%%%%%%%%%%%%%%
%%%%%%%%%%%%%%%%PAGEBREAK%%%%%%%%%%%%%%%%%
%%%%%%%%%%%%%%%%%%%%%%%%%%%%%%%%%%%%%%%%%%
%%%%%%%%PAGEBREAK%%%%%%%PAGEBREAK%%%%%%%%%
%%%%%%%%%%%%%%%%%%%%%%%%%%%%%%%%%%%%%%%%%%
%%%%%%%%%%%%%%%%%%%%%%%%%%%%%%%%%%%%%%%%%%
%%%%%%%%%%%%%%%%%%%%%%%%%%%%%%%%%%%%%%%%%%
%%%%%%%%%%%%%%%%%%%%%%%%%%%%%%%%%%%%%%%%%%
%%%%%%%%PAGEBREAK%%%%%%%PAGEBREAK%%%%%%%%%
%%%%%%%%%%%%%%%%%%%%%%%%%%%%%%%%%%%%%%%%%%
%%%%%%%%%%%%%%%%PAGEBREAK%%%%%%%%%%%%%%%%%
%%%%%%%%%%%%%%%%%%%%%%%%%%%%%%%%%%%%%%%%%%
%%%%%%%%PAGEBREAK%%%%%%%PAGEBREAK%%%%%%%%%
%%%%%%%%%%%%%%%%%%%%%%%%%%%%%%%%%%%%%%%%%%
%%%%%%%%%%%%%%%%%%%%%%%%%%%%%%%%%%%%%%%%%%
\begin{alignment}[
  texts=edition[class="edition"];
  translation[class="translation"],
]
\begin{edition}
  \ekddiv{type=ed}
    \begin{prose}[p22_04]
      %------------------------------
      %padṛsūtramayāni vasrāṇi//  \E
      %padasūtramayāni vastrāṇi?? \P %%%7643.jpg
      %paṭasūtrāmayāni vasrāṇi//  \B
      %paṭasūtrāmayāni vastrāṇi// \L
      %padṛsūtrayāni   vasrāṇi    \N1
      %padṛsūtrayāni   vasrāṇi    \D
      %padṛsūtrayāni   vasrāṇi    \N2
      %padṛsūtramayāni vasrāṇi    \U1
      %paṭasūtramayāni vasrāṇi    \U2
      %------------------------------
      %Clothes made from silk,...
      %------------------------------
      %paṃcasaptā dṛālikā         yuktāni harmyāṇi teṣu vāsaḥ    ativipulā  mṛdutarasukhāsuśayyā/     \E
      %paṃcasaptā dadhikā         yuktāni harmyāṇi teṣu cāsaḥ 2  ativipulā  mṛduttarachadavatīśayyā 2  \P
      %paṃcasatyā dātikā          yuktāni harmyāṇi teṣu vāstu    ativipulā  mṛdutaralāśayyā//2//        \B
      %paṃcasatyā dātikā          yuktāni harmyāṇi teṣu vāstu    ativipulā  mṛdutaralāśayyā//3//        \L
      %paṃca vā sapta vā dṛālikā  yuktāni harmyāṇi/              ativapulā  mṛdu/uttaracchaṃdavatīśayyā/  \N1
      %paṃca vā sapta vā dṛāṃlikā yuktāni harmyāṇi               ativapulā  mṛduuttarachaṃdavatīśayyā/     \D
      %paṃca vā sapta vā tālikā---yuktāni harmyāni               ativipulā  mṛduuttarachaṃdavatīśayyā    \N2
      %paṃca vā sapta vā dālikā---yuktāni harmyāṇi               ativipulāṃ mṛduuttarachadavatiśaiyyā     \U1
      %--------------------------saudhāni harmyāṇi vāsāya kecit// aṣṭau bhogān āha// sugrahaṃ// suvastraṃ// suśayā sustrī//  \U2
      %--------------------------------------------
      %,a site of the palace in which there are mainsions endowned with five or seven rooms. A very large and soft bed with an excellent blanket. 
      %-------------------------------------------
            \noindent
            \note[type=source, labelb=147a, labele=_147e, nosep]{cf. YSv (PT p. 837): ātmāvivekam āgamya calac cittaṃ mahākulam | viṣayāndhatamo dṛṣṭvā no vetti paramātmanaḥ | amāyātmā tattvātītaḥ satsandhānavivarjitaḥ | sukhī duḥkhī janma mṛtyuṃ yāti satyaṃ punaḥ punaḥ | vairāgyādidhanaṃ tyaktvā viṣavad duḥkhakṛddhiyaḥ | koṭisūryasamātmeti jñānayogād vimucyate |}
      \app{\lem[type=emendation, resp=egoscr,alt={paṭṭa°}]{paṭṭa}
        \rdg[wit={B,L,U2}]{paṭa°}
        \rdg[wit={X,E}]{padṛ°}
        \rdg[wit={P}]{pada°}
      }\app{\lem[wit={ceteri},alt={sūtra°}]{sūtra}
        \rdg[wit={B,L}]{sūtrā}
      }\app{\lem[wit={ceteri}, alt={°mayāni}]{mayāni}
        \rdg[wit={D,N1,N2}]{°yāni}}
      \app{\lem[wit={P,L}]{vastrāṇi}
        \rdg[wit={ceteri}]{vasrāṇi}} 1\dd{}
      \app{\lem[wit={X}]{paṃca vā sapta vā}
        \rdg[wit={E,P}]{paṃcasaptā}
        \rdg[wit={L,B}]{paṃcasatyā}}
      \app{\lem[type=emendation, resp=egoscr]{śālikā}
        \rdg[wit={E,N1}]{dṛālikā}
        \rdg[wit={D}]{dṛāṃlikā}
        \rdg[wit={P}]{dadhikā}
        \rdg[wit={B,L}]{dātikā}
        \rdg[wit={N2}]{tālikā}
        \rdg[wit={U1}]{dālikā}
      }\app{\lem[wit={ceteri}]{yuktāni}
        \rdg[wit={U2}]{saudhāni}}
      harmyāṇi
      \app{\lem[wit={L,B}]{teṣu vāstu}
        \rdg[wit={E}]{teṣu vāsaḥ}
        \rdg[wit={P}]{teṣu cāsaḥ}
        \rdg[wit={U2}]{vāsāya kecit}
        \rdg[wit={ceteri}]{\om}} 2\dd{}
      \app{\lem[wit={ceteri},alt={ativipulā°}]{ativipulā}
        \rdg[wit={D,N1}]{ativapulā°}
        \rdg[wit={U1}]{ativipulāṃ}
        \rdg[wit={U2}]{aṣṭau bhogān āha ||}
      }\app{\lem[type=emendation, resp=egoscr,alt={mṛdūttara}]{mṛdūttara}
        \rdg[wit={B,E,L,P}]{mṛdutara°}
        \rdg[wit={X}]{mṛdu | uttara°}
        \rdg[wit={U2}]{sugrahaṃ ||}
      }\app{\lem[wit={P},alt={°chadavatī°}]{chadavatī}
        \rdg[wit={D,N1,N2}]{°chandavatī°}
        \rdg[wit={U1}]{°chadavati°}
        \rdg[wit={U2}]{suvastraṃ ||}
      }\app{\lem[wit={ceteri}, alt={°śayyā}]{śayyā}
        \rdg[wit={U2}]{suśayā sustrī}} 3\dd{}
      %------------------------------
      %padminī tārūṇyavatī  manoharā guṇavatī  tatropaviṣṭā kāṃtā/      \E
      %padminī tārūṇyavatī  manoharā guṇavatī  tatopaviṣṭā  kāṃtā 4     \P
      %padminī tārūnyavatī  manoharā guṇavatī//tatrāpavistā kāṃtā 4     \B
      %padminī tārūnyavatī  manoharā guṇavatī//tatropavistā kāṃtā// 4// \L
      %padmanī tārūṇyavatī  manoharā guṇavatī  tatropavistā//           \N1
      %padminī tārūrāyavatī manoharā guṇavatī  tatropavistā//           \D
      %padminī tārūnyavatī  manoharā guṇavatī  tatropavistā             \N2
      %padminī tārūnyavati  manoharā guṇavati  tatropavistā             \U1
      %                                                                 \U2
      %--------------------------------------------
      %[On which] there is situated [tatropaviṣṭā] a excellent [em. zu tāruṇyavatī] youthful, charming and virtuous wife.
      %-------------------------------------------
      \app{\lem[wit={ceteri}]{padminī}
        \rdg[wit={N1}]{padmanī}
        \rdg[wit={U2}]{\om}}
      \app{\lem[type=emendation, resp=egoscr]{tāruṇyavatī}
        \rdg[wit={ceteri}]{tārūṇyavatī}
        \rdg[wit={N2}]{tārūrāyavatī}
        \rdg[wit={U2}]{\om}}
      \app{\lem[wit={ceteri}]{manoharā guṇavatī}
        \rdg[wit={U2}]{\om}}
      \app{\lem[wit={ceteri}]{tatropavistā}
         \rdg[wit={P}]{tatopavistā}
        \rdg[wit={B}]{tatrāpavistā}
        \rdg[wit={U2}]{\om}}
      \app{\lem[wit={B,E,L,P}]{kāntā}
        \rdg[wit={ceteri}]{\om}} 4\dd{}
      %------------------------------
      %sādhu āśanam/      atimūlyañ ca/         manoramam annaṃ।       tathā vidhaṃ pānam/   \E
      %sādhu āsanaṃ 5     atimūlo 'śvaḥ 6       manoramam annaṃ    7   tathā vidhaṃ pānaṃ 8  \P
      %sādhu āsanaṃ 5     atimūlyo asvaṃ//6     manoramyam attaṃ //7   tathā vidhapānaṃ//8   \B
      %sādhu āsanaṃ// 5// atimūlyo aśvaṃ//6//   manoramyam annaṃ //7// tathā vidhapānaṃ//8// \L
      %sādhyāsanaṃ//      amūlyo svaś ca//      manoramam attaṃ        tathā vidhaṃ pānaṃ/   \N1
      %sādhyāsanaṃ//      amūlyo svaś ca//      manoramam attaṃ        tathā vidhaṃ pānaṃ//  \D
      %sādhyāsanaṃ        amūlyo svaś ca        manotamam annaṃ        tathā vidhapānaṃ//    \N2
      %sādhyāsanaṃ        amolyo svaś ca        manoramam annaṃ        tathā vidhaṃ pānaṃ    \U1
      %sādhu āsanaṃ//           suśvaḥ//        suṣṭu annaṃ//          tathā vidhayānaṃ//    \U2
      %--------------------------------------------
      %good throne/seat; atimūlyo (überaus wertvolles) 'śvaṃ (Pferd), manorama ( die Sinne erfreuendes) Essen, verschiedenes Trinken. 
      %-------------------------------------------
      \app{\lem[type=emendation, resp=egoscr]{sādhvāsanaṃ}
        \rdg[wit={E}]{sādhu āśanam}
        \rdg[wit={B,L,P,U2}]{sādhu āsanaṃ}
        \rdg[wit={D,N1,N2}]{sādhyāsanaṃ}} 5\dd{}
      \app{\lem[type=emendation, resp=egoscr]{atimūlyo'śvaḥ}
        \rdg[wit={E}]{atimūlyañ ca}
        \rdg[wit={P}]{atimūlo 'śvaḥ}
        \rdg[wit={B,L}]{atimūlyo asvaṃ}
        \rdg[wit={X}]{amūlyo svaś ca}
        \rdg[wit={U2}]{suśvaḥ}} 6\dd{}
      \app{\lem[wit={ceteri},alt={manoramam annaṃ}]{manoramam-annaṃ}
        \rdg[wit={B}]{manoramyam attaṃ}
        \rdg[wit={L}]{manoramyam annaṃ}
        \rdg[wit={D,N1}]{manoramam attaṃ}
        \rdg[wit={U2}]{suṣṭu annaṃ}} 7\dd{}
      tathā
      \app{\lem[wit={ceteri}]{vidhaṃ pānaṃ}
        \rdg[wit={B,L,N2}]{vidhapānaṃ}
        \rdg[wit={U2}]{vidhayānaṃ}} 8\dd{}\label{eightenjoyments}
      %------------------------------
      %ete   ṣṭau bhogāḥ   kathitāḥ/   eke  duḥkhaṃ   bhajante/  bhikṣāṃ  yācante// kiñca \E
      %ete   ṣṭau bhogāḥ   kathitā 9   eke  duḥkha    bhajaṃte   bhikṣāṃ  yāṃcaṃte ca  \P
      %ete   ṣṭau bhogāḥ//             eka  duḥkhā    bhajaṃte/  bhikṣā   yāṃcate ca//  \B
      %ete   ṣṭau bhogāḥ//             eka  duḥkhā    bhajaṃte// bhikṣā   yāṃcate ca//  \L
      %ete  aṣṭau bhogā    kathyate/   eke  duḥkhaṃ   bhajaṃte/  bhikṣyāṃ yācate ca/   \N1
      %ete  aṣṭau bhogāḥ   kathyaṃte// ete  duḥkhaṃ   bhajaṃte/  bhikṣyāṃ yācaṃte ca// \D
      %ete  aṣṭau ghogā    kathyate//  ete  duḥkhataṃ bhajate    bhikṣāṃ  yācate ca//  \N2
      %rāte aṣṭau bhogāḥ   kathyate    ete  duḥkhaṃ   bhajate    bhikṣāṃ  pācate ca    \U1
      %ete  ṣṭau  bhogāḥ// kathitāḥ//  ekaṃ duḥkhaṃ   bhajaṃte// bhikṣā   yācaṃte ca// \U2
      %------------------------------
      %The eight enjoyments are being described. They impart suffering, and [make one] begging for their sustenance.
      %------------------------------
      %\note[type=philcomm, labelb=148, lem={'ṣṭau bhogāḥ}]{The eight enjoyments are not attested in any of the sources.}
      \app{\lem[wit={ceteri}]{ete}
        \rdg[wit={U1}]{rāte}
      }\app{\lem[wit={Y}]{'ṣṭau}
        \rdg[wit={X}]{aṣṭau}}
      \app{\lem[wit={ceteri}]{bhogāḥ}
        \rdg[wit={N1,N2}]{bhogā}
        \rdg[wit={U1}]{ghogā}}
      \app{\lem[wit={E,U2}]{kathitāḥ}
        \rdg[wit={P}]{kathitā}
        \rdg[wit={N1,N2,U1}]{kathyate}
        \rdg[wit={D}]{kathyaṃte}
        \rdg[wit={B,L}]{\om}}/
      \app{\lem[wit={D,N2,U1}]{ete}
        \rdg[wit={E,P,N1}]{eke}
        \rdg[wit={B,L}]{eka}
        \rdg[wit={U2}]{ekaṃ}}
      \app{\lem[wit={D,E,N1,U1,U2}]{duḥkhaṃ}
        \rdg[wit={P}]{duḥkha}
        \rdg[wit={B,L}]{duḥkhā}
        \rdg[wit={N2}]{duḥkhataṃ}}
      \app{\lem[wit={ceteri}]{bhajante}
        \rdg[wit={N2,U1}]{bhajate}}/
      \app{\lem[wit={E,P,N2,U1}]{bhikṣāṃ}
        \rdg[wit={D,N1}]{bhikṣyāṃ}
        \rdg[wit={B,L,U2}]{bhikṣā}}
      \app{\lem[wit={ceteri}]{yācante}
        \rdg[wit={P}]{yāṃcaṃte}
        \rdg[wit={B,L}]{yāṃcate}
        \rdg[wit={N2}]{yācate}
        \rdg[wit={U1}]{pācate}}
      \app{\lem[wit={ceteri}]{ca}
        \rdg[wit={E}]{kiñca}}/\linelabel{_147e}
    \end{prose}
    \begin{prose}[p22_05]
      %------------------------------
      %      yathā sūryasya tejaḥ   dugdhasya    ghṛtam   agner jvalanaṃ viṣān mūrchā   tilāttailam/    vṛkṣāc-chāyā/  phalāt parimalaḥ       kāṣṭhād agniḥ    arkarādibhyo   madhuro rasaḥ/ \E
      %      yathā sūryasys tejaḥ   dugdhasya    ghṛtaḥ   agne dvāhaḥ    viṣān mūrchāti tilāttailaṃ     vṛkṣāt-chāyā   phalāsarimalaḥ         kāṣṭād  agniḥ    śarkvarādibhyo madhuro rasaḥ  \P
      %      yathā sūryasye tejāḥ   dugdha-------ghṛtaḥ   agne dvāhaḥ//  viṣān mūrchā   tilāttailaṃ//   vṛkṣā--chāyā   phalāsarimalaḥ         kaṣṭād  agniḥ    śarkadībhyo    madhuro  \B
      %      yathā sūryasya tejāḥ   dugdha-------ghṛtaḥ   agne dvāhaḥ//  viṣān mūrchā   tilātailaṃ//    vṛkṣā--chāyā   phalāt parimalaḥ       kaṣṭād  agniḥ    śarkadībhyo    madhuro  \L
      %      yathā sūryasya tejaḥ/  dugdhasya    ghṛtaṃ/  agne dahiḥ??   viṣān mūrchā   tilāttailaṃ,    vṛkṣāc-chāyā/  phalāt parimalaḥ/      kāṣṭhād āgniḥ/   śarkkarādibhyo madhuro rasaḥ/ \N1
      %      yathā sūryasya tejaḥ// dugdhasya    ghṛtaṃ// agne dadhiḥ    viṣān mūrchā   tilāttailaṃ//   vṛkṣā--chāyā// phalāt palātparimalaḥ//kāṣṭhād āgniḥ//  śarkarādibhyo  madhuro rasaḥ/ \D
      %      yathā sūryasya tejaḥ   dusya        ghṛtaṃ   agne dadhi     viṣān mūrchā   tilatailaṃ      vṛkṣā--chāyā   phalāt parimalaḥ       kāṣṭhād āgniḥ    śarkarādibhyo  madhuro rasaḥ/ \N2
      %      yathā sūryaśca tejaḥ   dugdhasy     ghṛttaṃ  agne dārhaṃ    viṣāt mūrchā   tilātailaṃ      vrakṣā-chāyā   phalāt parimalaḥ       kāṣṭhād āgniḥ    śarkarādibhyo  madhuro rasaḥ \U1
      %      yathā sūryasya tejaḥ// dugdhasya    ghṛtaṃ// agne dāhiḥ//   viṣān mūrchā   tilātailaṃ//    vṛkṣā--chāyā// phalāt parimalaḥ//     kāṣṭād  agniḥ    śarkarādibhyo  madhuro rasaḥ// \U2
      %------------------------------
      %Gleichwie die Strahlen der Sonne, die Butter der Milch, das Brennen des Feuers, die Betäubung aufgrund von Gift, das Sesamöl aus dem Sesamkorn, der Schatten vom Baum, der Wohlgeruch von einer Frucht, das Feuer von einem Holzscheid, der Süße Saft [em. zu śārkara] a liquor prepared from Dhātakī with sugar] und so weiter,   
      %------------------------------
      %Like the rays of the sun, the butter of milk, the burning of fire, the stupor of poison, the sesame oil from the sesame seed, the shade from the tree, the sweet odor from a fruit, the fire from a scabbard, the sweet sap [em . to śārkara] a liquor prepared from Dhātakī with sugar] and so on,
      %------------------------------
      \note[type=source, labelb=149, labele=_149e, nosep]{cf. YSv (PT p. 837): ravī tejo ghṛtaṃ dugdhe tile tailaṃ svabhāvataḥ | śaśam indau kule śākhaṃ kṣāre ca lavaṇaṃ yathā | tathā brahmaṇi saṃsāro hyakhaṇḍaparipūrvake |}
      yathā
      \app{\lem[wit={ceteri}]{sūryasya}
        \rdg[wit={U1}]{sūryaś ca}}
      \app{\lem[wit={ceteri}]{tejaḥ}
        \rdg[wit={B,L}]{tejāḥ}}\dd{}
      \app{\lem[wit={D,E,P,N1,U2}]{dugdhasya}
        \rdg[wit={B,L}]{dugdha°}
        \rdg[wit={N2}]{dusya}
        \rdg[wit={U1}]{dugdhasy}}
      \app{\lem[wit={ceteri}]{ghṛtaṃ}
        \rdg[wit={B,L,P}]{ghṛtaḥ}}\dd{}
      \app{\lem[wit={E}, alt={agner}]{agne\skp{r-dā}}
        \rdg[wit={ceteri}]{agne}
      }\app{\lem[type=emendation, resp=egoscr, alt={dāhaḥ}]{\skm{r-dā}haḥ}
        \rdg[wit={B,L,P}]{dvāhaḥ}
        \rdg[wit={N1}]{dahiḥ}
        \rdg[wit={N2}]{dadhi}
        \rdg[wit={D}]{dadhiḥ}
        \rdg[wit={U1}]{dārhaṃ}
        \rdg[wit={U2}]{dāhiḥ}
        \rdg[wit={E}]{jvalanaṃ}}\dd{}
      \app{\lem[wit={ceteri},alt={viṣān}]{viṣā\skp{n-mū}}
        \rdg[wit={U1}]{viṣāt}
      }\skm{n-mū}rchā\dd{}
      \app{\lem[wit={ceteri},alt={tilāt}]{tilā\skp{t-tai}}
        \rdg[wit={P}]{titilāt}
        \rdg[wit={N2}]{tila}
        \rdg[wit={U1}]{tilā}
      }\skm{t-tai}laṃ\dd{}
      \app{\lem[wit={E,N1}, alt={vṛkṣāt}]{vṛkṣā\skp{c-chā}}
        \rdg[wit={P}]{vṛkṣāt}
        \rdg[wit={B,D,L,N2,U2}]{vṛkṣā}
        \rdg[wit={U1}]{vrakṣā}
      }\skm{c-chā}yā\dd{}
      \app{\lem[wit={ceteri},alt={phalāt}]{phalā\skp{t-pa}}
        \rdg[wit={B,L}]{phalā}
      }\app{\lem[wit={ceteri},alt={parimalaḥ}]{\skm{t-pa}rimalaḥ}
        \rdg[wit={B,L}]{sarimalaḥ}
        \rdg[wit={D}]{palāt parimalaḥ}}\dd{}%\note[type=philcomm, labelb=150, lem={parimalaḥ}]{Clarification: Witness \getsiglum{D} reads \textit{phalāt palāt parimala}.}
      \app{\lem[wit={ceteri}, alt={kāṣṭhād}]{kāṣṭhā\skp{d-a}}
        \rdg[wit={P,U2}]{kāṣṭād}
        \rdg[wit={B,L}]{kaṣṭād}
      }\app{\lem[wit={Y}, alt={agniḥ}]{\skm{d-a}gniḥ}
        \rdg[wit={X}]{āgniḥ}}\dd{}
      \app{\lem[type=emendation, resp=egoscr]{śārkarādibhyo}
        \rdg[wit={E}]{arkarādibhyo}
        \rdg[wit={P}]{śarkvarādibhyo}
        \rdg[wit={L,B}]{śarkadībhyo}}
      madhuro
      \app{\lem[wit={ceteri}]{rasaḥ}
        \rdg[wit={B,L}]{\om}}\dd{}
      %------------------------------
      %himānībhyaḥ   śītam      ityādipadārthānāṃ svabhāvaḥ         tathā    saṃsāro'pi parameśvarasvarūpamadhye      tiṣṭhati/ \E
      %himānībhyaḥ   śītaṃ      ityādipadārthasvabhāva        eva   tathā    saṃsāro'pi parameśvarasvarūpamadhye      tiṣṭhati \P
      %sahīmānībhyaḥ śītaḥ/     ityādipadārthāsvabhāvataḥ// eva     tathā    saṃsāro pi paremesvara svarūpasya madhye tiṣṭhatī/ \B
      %sahimānibhyaḥ śītaḥ//    ityādiphadārthāḥ svabhāvataḥ// eva  tathā    saṃsāro pi paremesvara svarūpasya madhye tiṣṭhati// \L
      %himānibhyaḥ   śaityāṃ    ityādipadārthasvabhāva evā/         tathā    saṃsāro pi parameśvarasvarūpamadhye      tiṣṭhati// \N1
      %himānibhyaḥ   śaityaṃ // ityādipadārthasvabhāva eva//        tathā    saṃsāro pi parameśvarasvarūpamadhye      tiṣṭhati// \D 
      %himānitpa     śaityāś    atyādipadārtharthasvabhāva eva//    tathā    saṃsāro pi parameśvarasvarūpamadhye      tiṣṭhati \N2
      %himānībhyaḥ   śaityaṃ    ityādipadārthasvabhāvaḥ ravaḥ?      tathā vā saṃsāro pi parameśvararūpamadhye         tiṣṭhati/ \U1
      %himānībhyaḥ   śītyaṃ//   ityādipadārthāsvabhāva eva//        tathā    saṃsāro pi parameśvarasvarūpamadhye      tiṣṭhaṃti// \U2
      %------------------------------
      %die Kälte von Schneehaufen, und so weiter ist das inhärente Wesen der Dinge. IN gleicher Weise befindet sich auch der Weltengang im Zentrum der eigenen Gestalt von höchsten Gott.
      %the cold of piles of snow, and so on is the inherent essence of things. In the same way, the course of the world is also in the center of the highest God's own form. 
      %------------------------------
      \app{\lem[wit={ceteri}]{himānībhyaḥ}
        \rdg[wit={B,L}]{sahimānibhyaḥ}
        \rdg[wit={N2}]{himānitpa}}
      \app{\lem[wit={D,U1}]{śaityaṃ}
        \rdg[wit={N1}]{śaityāṃ}
        \rdg[wit={U2}]{śītyaṃ}
        \rdg[wit={N2}]{śaityāś}
        \rdg[wit={E,P}]{śītaṃ}
        \rdg[wit={B,L}]{śītaḥ}}\dd{}
      \app{\lem[wit={D,N1,P}, alt={ityādipadārthasvabhāva}]{ityādipadārthasvabhāva}
        \rdg[wit={U2}]{ityādipadārthā°}
        \rdg[wit={B}]{ityādipadārthāsvabhāvataḥ}
        \rdg[wit={N2}]{atyādipadārtharthasvabhāva}
        \rdg[wit={U1}]{ityādisvabhāvaḥ}
        \rdg[wit={L}]{ityādiphadārthāḥ svabhāvataḥ}
        \rdg[wit={E}]{ityādipadārthānāṃ svabhāvaḥ}}
      \app{\lem[wit={ceteri}]{eva}
        \rdg[wit={N1}]{evā}
        \rdg[wit={U1}]{ravaḥ}
        \rdg[wit={E}]{\om}}\dd{}
      \app{\lem[wit={ceteri}]{tathā}
        \rdg[wit={U1}]{tathā vā}}
      saṃsāro'pi
      \app{\lem[wit={ceteri}]{parameśvarasvarūpamadhye}
        \rdg[wit={B,L}]{paremesvara svarūpasya madhye}
        \rdg[wit={U1}]{parameśvararūpamadhye}}
      \app{\lem[wit={ceteri}]{tiṣṭhati}
        \rdg[wit={B}]{tiṣṭhatī}
        \rdg[wit={U2}]{tiṣṭhaṃti}}\dd{}
      %------------------------------
      %parameśvaro 'khaṇḍa--paripūrṇaḥ/  \E
      %parameśvaro khaṃḍa---paripūrṇaś ca    \P
      %parameśvaro khaṃḍa---paripūrṇaś ca// \B
      %parameśvaro khaṃḍa---paripūrṇaś ca//  \L
      %parameśvaro 'ṣaṃḍa---paripūrṇaś ca//  \N1
      %parameśvaro  ṣaṃḍa---paripūrṇaś ca//  \D %%%S.9 verso
      %parameśvaro yarāṇḍa--paripūrṇaś ca//  \N2
      %parameśvaro khaṃḍaḥ  paripūrṇaś ca   \U1 %%%277.jpg
      %parameśvaro 'khaṃḍa--paripūrṇaś ca//   \U2
      %------------------------------
      %Und der höchste Gott ist unteilbar und das All erfüllend.
      %And the Most High God is indivisible and all-filling.
      %------------------------------
      parameśvaro\app{\lem[wit={ceteri}, alt={'khaṇḍa°}]{'khaṇda}
        \rdg[wit={D,N1}]{'ṣaṃḍa°}
        \rdg[wit={N2}]{yarānda°}
        \rdg[wit={U1}]{khaṃḍaḥ}
      }\app{\lem[wit={ceteri},alt={°paripūrṇaś ca}]{paripūrṇaś\skp{-}ca}
        \rdg[wit={E}]{paripūrṇaḥ}}\dd{}\linelabel{_149e}
   \end{prose}
\end{edition}
\begin{translation}
  \ekddiv{type=trans}
    \begin{tlate}[p22_04]
      \noindent
  1. Clothes made from silk;\footnote{Within the twenty \textit{upabhoga}s of the \textit{Mānasollāsa} there is the topic of \textit{Vastropabhoga} or the ``enjoyment of garments''. Particularly in summer, the king is asked to wear silk or cotton clothes which are thin and charming (\citeauthor[1939:14]{manasollasa}).} 2. A site of the palace in which there are mansions endowed with five or seven rooms.\footnote{The first \textit{adhyāya} of the third \textit{viṃśati} of the \textit{Mānasollāsa} discusses astrology for finding out auspicious moments while building new houses for princes. Here houses with one to four \textit{śālā}s are described (\citeauthor[1939:6-7]{manasollasa}).} 3. A huge and soft bed with an excellent blanket;\footnote{This is found as \textit{Śayyābhoga} within the \textit{Mānasollāsa}. In this section, seven kinds of beds and eight kinds of bed-steads are described (\citeauthor[1939:21]{manasollasa}).} 4. [on which] there is seated an excellent, youthful, charming and virtuous wife;\footnote{This is resembled as \textit{yosidupabhoga} within the \textit{Mānasollāsa}. In this chapter, King Someśvara describes the qualifications of women a king should marry. The two most important qualities he gives are beauty and full youth. Out of the four kinds of women: a) Padminī, b) Citriṇī, c) Śaṅkhinī, and d) Hastinī, he suggests that the latter two kinds are not worth enjoying  (\citeauthor[21]{manasollasa}.)} 5. An excellent seat;\footnote{The \textit{āsanopabhoga} or ``the enjoyment of seats'' within the \textit{Mānasollāsa} describe various kinds of royal seats (\citeauthor[1939:15]{manasollasa}).} 6. An exceptional valuable horse;\footnote{This is resembled as \textit{yānopabhoga} within the \textit{Mānasollāsa}. In this chapter, King Someśvara lists nine kinds of vehicles, including horses (\citeauthor[1939:24]{manasollasa}).} 7. Food that pleases the senses;\footnote{This is resembled as \textit{annabhoga} within the \textit{Mānasollāsa}. In this section, King Someśvara describes the names of various kinds of great food and the directions as to the preparations of various dishes (\citeauthor[1939:21]{manasollasa}).} 8. Various drinks.\begin{buber}[f22_1]\footnote{This is resembled as \textit{pānīyabhoga} within the \textit{Mānasollāsa}. In this chapter, everything related to drinking and drinks is described (\citeauthor[1939:23]{manasollasa}). Instead of following his source text closely, Rāmacandra modifies the passage's meaning. The original ascetic ideal of his source text, which emphasizes renunciation and detachment from wealth, is weakened by introducing the eight enjoyments. Instead of radically abandoning wealth, as we can \ldots}\end{buber} 
  The eight enjoyments are described. They impart suffering and [make one] beg for their sustenance.
    \end{tlate}
    \begin{tlate}[p22_05]
  Like the rays of the sun, the butter of milk, the burning of fire, the stupor of poison, the sesame oil from the sesame seed, 
  the shade from the tree, the sweet odour from a fruit, the fire from a scabbard,
   the sweet sap of Śārkara\footnote{A liquor prepared from Dhātakī with sugar. Beleg?} 
   and so on, the cold of piles of snow, and so on is the inherent essence of things.
   In the same way, the course of the world is also in the centre of the highest God's own form. 
   Moreover, the highest God is indivisible and all-filling.
  \label{endsvabhava}
   \flushpage 
  \end{tlate}
\end{translation}
\end{alignment}
%%%%%%%%%%%%%%%%%%%%%%%%%%%%%%%%%%%%%%%%%%
%%%%%%%%%%%%%%%%%%%%%%%%%%%%%%%%%%%%%%%%%%
%%%%%%%%PAGEBREAK%%%%%%%PAGEBREAK%%%%%%%%%
%%%%%%%%%%%%%%%%%%%%%%%%%%%%%%%%%%%%%%%%%%
%%%%%%%%%%%%%%%%PAGEBREAK%%%%%%%%%%%%%%%%%
%%%%%%%%%%%%%%%%%%%%%%%%%%%%%%%%%%%%%%%%%%
%%%%%%%%PAGEBREAK%%%%%%%PAGEBREAK%%%%%%%%%
%%%%%%%%%%%%%%%%%%%%%%%%%%%%%%%%%%%%%%%%%%
%%%%%%%%%%%%%%%%%%%%%%%%%%%%%%%%%%%%%%%%%%
%%%%%%%%%%%%%%%%%%%%%%%%%%%%%%%%%%%%%%%%%%
%%%%%%%%%%%%%%%%%%%%%%%%%%%%%%%%%%%%%%%%%%
%%%%%%%%PAGEBREAK%%%%%%%PAGEBREAK%%%%%%%%%
%%%%%%%%%%%%%%%%%%%%%%%%%%%%%%%%%%%%%%%%%%
%%%%%%%%%%%%%%%%PAGEBREAK%%%%%%%%%%%%%%%%%
%%%%%%%%%%%%%%%%%%%%%%%%%%%%%%%%%%%%%%%%%%
%%%%%%%%PAGEBREAK%%%%%%%PAGEBREAK%%%%%%%%%
%%%%%%%%%%%%%%%%%%%%%%%%%%%%%%%%%%%%%%%%%%
%%%%%%%%%%%%%%%%%%%%%%%%%%%%%%%%%%%%%%%%%%
%%%%%%%%%%%%%%%%%%%%%%%%%%%%%%%%%%%%%%%%%%
%%%%%%%%%%%%%%%%%%%%%%%%%%%%%%%%%%%%%%%%%%
%%%%%%%%PAGEBREAK%%%%%%%PAGEBREAK%%%%%%%%%
%%%%%%%%%%%%%%%%%%%%%%%%%%%%%%%%%%%%%%%%%%
%%%%%%%%%%%%%%%%PAGEBREAK%%%%%%%%%%%%%%%%%
%%%%%%%%%%%%%%%%%%%%%%%%%%%%%%%%%%%%%%%%%%
%%%%%%%%PAGEBREAK%%%%%%%PAGEBREAK%%%%%%%%%
%%%%%%%%%%%%%%%%%%%%%%%%%%%%%%%%%%%%%%%%%%
%%%%%%%%%%%%%%%%%%%%%%%%%%%%%%%%%%%%%%%%%%
\chapter{Appendix}
\section{Figures}

% \begin{landscape}
\clearpage
  \begin{figure}[ht]
	\centering
  \includegraphics[width=1\textwidth]{pics/Vishnu_Vishvarupa_cropped.jpg}
	\caption{Viṣṇu Viśvarūpa, India, Rajasthan, Jaipur, ca. 1800–1820, Opaque watercolor and gold on paper, 38.5 × 28 cm, Victoria and Albert Museum, London, Given by Mrs. Gerald Clark.}
	\label{fig1}
      \end{figure}
\clearpage
  \begin{figure}[ht]
	\centering
  \includegraphics[width=0.5\textwidth]{pics/The_Equivalence_of_Self_and_Universe_(detail),_folio_6_from_the_Siddha_Siddhanta_Paddhati,_(Bulaki),_1824_(Samvat_1881);_122_x_46_cm._Mehrangarh_Museum_Trust..jpg}
	\caption{The Equivalence of Self and Universe (detail), folio 6 from the \textit{Siddhasiddhāntapaddhati} (Bulaki), India, Rajasthan, Jodhpur, 1824 (Samvat 1881), 122 x 46 cm, RJS 2378, Mehragarh Museum Trust.}
	\label{fig2}
      \end{figure}
      % \end{landscape}


\chapter{Bibliography}
 \label{sec:bibli}
   \clearpage
\newpage 
\thispagestyle{empty}
\quad  \addtocounter{page}{-1}

\printbibliography[heading=subbibintoc, title=Consulted Manuscripts, keyword=codex]

\printbibliography[heading=subbibintoc, title=Printed Editions, keyword=printsource]

\printbibliography[heading=subbibintoc, title=Secondary Literature, keyword=seclit]

\printbibliography[heading=subbibintoc, title=Online Sources, keyword=onlinesource]

\end{document}
