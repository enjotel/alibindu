%Ultimatives Tool zur Datierung:
%https://www.cc.kyoto-su.ac.jp/~yanom/pancanga/
%skp = ignored in edition
%skm = ignored in xml
\input{preamble.tex}
\FormatDiv{1}{\begin{center}\Large}{\end{center}}
\FormatDiv{2}{\begin{center}\small}{\end{center}}
\FormatDiv{3}{\bfseries}{.}
\title{Tattvayogabindu of Rāmacandra\\ A Critical Edition and Annotated Translation\\ and a Comparative Analysis of the \\Complex Early Modern Yoga Yaxonomies }
\date{\today}
\parindent=15pt

\begin{document}

\frontmatter
\thispagestyle{empty} % Verhindert Seitenzahl auf der Seite
%\begin{center}
%\begin{figure}[ht]
%    \centering
%    \includegraphics[width=0.25\textwidth]{pics/siegel.jpg}
%    \phantomsection\label{logo}
%\end{figure}

%\vspace{0.5in}

%\begin{otherlanguage}{iast}
%   \large\sanskritfont{Yogatattvabindu}\\
%\end{otherlanguage}
\vspace*{\fill}
%\vspace{0.25in}
\begin{center}
\huge\MakeUppercase{The Tattvayogabindu \\of Rāmacandra}\\

\vspace{0.2in}

\Large  Critical Edition and Annotated Translation of an Early Modern Text on Rājayoga, with a Comparative Analysis of the Complex Yoga Taxonomies from the Same Period\\ 
\end{center}
%\vspace{0.45in}

% Titel der Arbeit
%\large \textbf{\MakeUppercase{Inaugural-Dissertation}} \\
%zur \\
%Erlangung des Grades eines Doktors der Philosophie (Dr. phil.) \\
%dem \\
%Fachbereich Fremdsprachliche Philologien \\
%der \\
%Philipps-Universität Marburg\\


%\vspace{0.25in}

\vspace*{\fill} % Flexibler Abstand, schiebt den folgenden Text nach unten

% Name des Autors
%\normalsize
%vorgelegt von \\
%\textbf{Nils Jacob Liersch, M. A.} \\
%\vspace{0.25in}
%aus Bad Berleburg\\
%Marburg | Schönau (Altneudorf)\\
%2024
%\vspace{0.5in}
%\end{center}
%\thispagestyle{empty} 
%\newpage
%\noindent
%Vom Fachbereich Fremdsprachliche Philologien der Philipps-Universität Marburg als Dissertation angenommen am:\\\\
%\noindent
%.......................................\\\\
%\noindent
%Gutachter:\\\\
%\noindent
%Prof. Dr. Jürgen Hanneder\\\noindent
%Prof. Dr. James Mallinson\\
%
%\thispagestyle{empty}

%\newpage
%\thispagestyle{empty}
%\mbox{}
%\newpage

\newpage

  \thispagestyle{empty}
  \begin{figure}[p]
    \centering
    \includegraphics[width=0.25\textwidth]{pics/purna.jpg}
  \end{figure}
  
\newpage

\begin{landscape}
\thispagestyle{empty}
  \begin{figure}[p]
	\centering
  \includegraphics[width=1.5\textwidth]{pics/folio1.jpg}
	\caption{Folio 1v of Ms. \getsiglum{N1}.}
	 \phantomsection\label{fig_folio1}
\end{figure}
\end{landscape}

\newpage
%\section*{Deutsche Zusammenfassung}
%\phantomsection
%\addcontentsline{toc}{chapter}{Deutsche Zusammenfassung}
%\thispagestyle{empty} % Verhindert Seitenzahl auf der Seite
%
%\lettrine{D}{as} \textit{Tattvayogabindu} (``Die Essenz des Yoga und der Realität'') von Rāmacandra ist ein frühneuzeitlicher Sanskrit-Text zum Thema Rājayoga, der in der ersten Hälfte des siebzehnten Jahrhunderts verfasst wurde. Das vordergründig auffälligste Merkmal dieses Werkes ist seine hochdifferenzierte Taxonomie verschiedener Yogas. In der Einleitung des \textit{Tattvayogabindu} nennen die meisten Handschriften fünfzehn verschiedene Yogakategorien, die als Methoden des Rājayoga eingeführt werden. Diese lauten wie folgt: 1. Kriyāyoga, 2. Jñānayoga, 3. Caryāyoga, 4. Haṭhayoga, 5. Karmayoga, 6. Layayoga, 7. Dhyānayoga, 8. Mantrayoga, 9. Lakṣyayoga, 10. Vāsanāyoga, 11. Śivayoga, 12. Brahmayoga, 13. Advaitayoga, 14. Siddhayoga und 15. Rājayoga selbst.\footnote{Dies ist ein auffälliger Anstieg der Anzahl der deklarierten Yogas im Vergleich zu der mittelalterlichen Standard-Tetrade von Mantra-, Laya-, Haṭha- und Rājayoga.} Der Text ähnelt einem Kompendium, das in einer Mischung aus hauptsächlich Prosa und 47 Versen im Lehrbuchstil verfasst ist, wobei die 59 Themen des Textes in Abschnitte unterteilt sind, die zumeist durch erkennbare Phrasen eingeleitet werden. Die Abschnitte befassen sich mit den Methoden des Rājayoga und ihren Wirkungen, aber auch mit Themen wie der yogischen Physiologie, dem Avadhūta, der Bedeutung des Gurus, einer Kosmogonie und einem \textit{yogaśāstrarahasya}.
%
%Das \textit{Tattvayogabindu} wurde in der wissenschaftlichen Sekundärliteratur über Yoga bisher nicht ausführlich diskutiert. Die einzige Ausnahme bildet \citeauthor{birch2014} (2014: 415-416), der die Taxonomie der fünfzehn Yogas im Kontext der ``fünfzehn mittelalterlichen Yogas'' kurz beschreibt und feststellt, dass eine ähnliche Taxonomie in Nārāyaṇatīrthas \textit{Yogasiddhāntacandrikā} (17. Jh.) vorkommt, einem Kommentar zum \textit{Pātañjalayogaśāstra}, der fünfzehn Yogakategorien in das \textit{aṣṭāṅga}-Format integriert. Eine andere Darstellung von fünfzehn Yogas findet sich in einem weiteren Sanskrit-Yogatext namens \textit{Yogasvarodaya}, der nur durch Zitate in der \textit{Prāṇatoṣinī}, der \textit{Yogakarṇikā} und dem \emph{Śabdakalpadruma} überliefert ist. Das \textit{Yogasvarodaya} nennt zwar ebenfalls insgesamt fünfzehn Yogas, aber nur acht von ihnen in seinen einleitenden \textit{śloka}s. Das \textit{Yogasvarodaya} war der Hauptquelltext und die wichtigste Textvorlage für Rāmacandras Kompilation des \textit{Tattvayogabindu}. Abgesehen von einigen Passagen folgt Rāmacandra in vielen Fällen dessen Inhalt und Struktur, indem er die Verse des \textit{Yogasvarodaya} in Prosa umschreibt oder diese direkt ohne Zuschreibung zitiert. Aufgrund der unvollständigen Überlieferung des \textit{Yogasvarodaya} ist Rāmacandras \textit{Tattvayogabindu} ein wichtiger Ausgangspunkt für eine eingehende Untersuchung der komplexen frühneuzeitlichen Yogataxonomien, ein Phänomen, das sich zeitlich und, wie ich zeigen werde, auch räumlich sehr genau eingrenzen lässt. Der andere Quelltext, den Rāmacandra verwendete, ist die \textit{Siddhasiddhāntapaddhati}, auf deren Inhalt er vor allem in der zweiten Hälfte seiner Komposition zurückgreift. Ein weiterer Text, der eine ähnliche Taxonomie von zwölf Yogas enthält, die in drei Tetraden unterteilt sind, ist Sundardās' in \textit{Brajbhāṣa} verfasster Yogatext mit dem Titel \textit{Sarvāṅgayogapradīpikā}. 
%
%Diese komplexen Yogataxonomien, die alle im 17. Jahrhundert entstanden, entsprangen sehr unterschiedlichen religiösen Strömungen und wurden von den Autoren der Texte an die spezifischen Bedürfnisse ihrer Traditionen angepasst. Das \textit{Tattvayogabindu} umfasst einen großen Teil der Vielfalt der Yogaformen, die im 17. Jh. von einem breiten Spektrum religiöser Traditionen und Schichten der indischen Gesellschaft praktiziert und diskutiert wurden. Im besonderen Fall des \emph{Tattvayogabindu} gibt es zahlreiche Aussagen im gesamten Text, die eine Strategie offenbaren, den Yoga von seinen typisch asketisch-weltentsagenden Konnotationen zu lösen und Rājayoga als eine Praxis zu stilisieren, die selbst für Praktizierende, die weltliche Vergnügungen und einen extravaganten Lebensstil genießen, den erstrebten soteriologischen Nutzen bringen kann. Textimmanente Belege deuten darauf hin, dass das \textit{Tattvayogabindu} ein wichtiges Beispiel für einen Text ist, der eine frühneuzeitliche Adaption des Rājayoga für ein \textit{kṣatriya}-Publikum in einem höfischen Umfeld darstellt.
%
%Eine Druckausgabe des \textit{Tattvayogabindu} wurde 1905 mitsamt einer Hindi-Übersetzung veröffentlicht und basiert auf einem unbekannten Manuskript.\footnote{\emph{Binduyoga}.\textit{Binduyogaḥ with Bhāṣaṭīkā}. Hrsg. von Jvālāprasāda Miśra. Mumbai, 1905.} Diese Veröffentlichung trägt den Titel ``\textit{Binduyoga}'', was durch das Kolophon des gedruckten Textes bestätigt wird. Wie im Verlauf der Einleitung erörtert wird, war der Text jedoch ursprünglich als \textit{Tattvayogabindu} bekannt. Die konsultierten Manuskripte weisen erhebliche Diskrepanzen, strukturelle Unterschiede und zahlreiche voneinander abweichende Lesarten zwischen ihnen und der gedruckten Ausgabe auf.\footnote{Beispielsweise enthält die Druckausgabe die in den Handschriften präsentierte komplexe Yogataxonomie nicht.} Diese Manuskripte sind vor allem über die nördliche Hälfte des indischen Subkontinents und Nepal verstreut, was darauf schließen lässt, dass der Text weit verbreitet war. Längere Passagen des \textit{Tattvayogabindu} werden ohne Quellenangabe in einem Text namens \textit{Yogasaṃgraha} und Sundaradevas \textit{Haṭhasaṅketacandrikā} zitiert. 
%
%Das erste Kapitel dieser Dissertation beinhaltet eine allgemeine Einführung in Rāmacandras \textit{Tattvayogabindu}. Das Kapitel gibt einen Kurzüberblick über den Inhalt des Textes, befasst sich mit dessen Herkunft, dem Autor und diskutiert das vom Autor intendierte Publikum seines Werkes. Weiterhin werden die Textzeugen, die Quelltexte und Testimonien des \textit{Tattvayogabindu} beschrieben. Im Anschluss wird eine stemmatische Analyse des Textes präsentiert, welche auf manueller philologischer Beobachtung und computergestützter Stemmatik basiert, um ein \textit{stemma codicum} zu präsentieren. Das Kapitel schließt mit einer Darlegung der Editionsrichtlinien ab, welche die Grundlage für das zweite Kapitel dieser Arbeit bildet.
%
%Das zweite Kapitel, das Herzstück dieser Arbeit, ist eine kritische Edition und annotierte Übersetzung des \textit{Tattvayogabindu}. Die kritische Edition dieser Dissertation verbessert den Text signifikant und wirft ein völlig neues Licht auf dessen historische Bedeutung.
%
%Das dritte Kapitel dieser Arbeit beinhaltet eine auf Differenzhermeneutik\footnote{Der Begriff Differenzhermeneutik wird auf S. \pageref{hermeneutics}, Fn. \ref{hemerneutics} näher erläutert.} basierende komparatistische Analyse der komplexen frühneuzeitlichen Yogataxonomien. Anhand der neuen kritischen Edition des \textit{Tattvayogabindu} und der weiter oben genannten Texte, \emph{Yogasvarodaya}, \emph{Yogasiddhāntacandrikā} und \emph{Sarvāṅgayogapradīpikā}, werden die vier komplexen Yogataxonomien detailliert miteinander verglichen. Anhand dieser Komparatistik wurde eine differenzierte Hypothese zur Entstehung der komplexen Yogataxonomien entwickelt und die komplexen Yogataxonomien im breiteren Kontext der historischen Entwicklung der Yogatraditionen verortet. Der Vergleich beinhaltet eine nuancierte Beschreibung der einzelnen von den Yogatraditionen verwendeten Yogakategorien. Zwar operieren die Autoren der vier Texte oftmals mit identischen Bezeichnungen für die einzelnen Yogakategorien, die sie in ihren Taxonomien auflisten, deuten diese Kategorien jedoch in vielen Fällen entsprechend ihres eigenen religiösen Hintergrundes und ihrer eigenen Agenda mit verblüffenden und interessanten Unterschieden aus. Die Kontrastierung der Komparanden, d.h. der Autoren, der Texte, der Yogataxonomien und der zur Verwendung kommenden Yogakategorien, eröffnet daher einen tiefen Einblick in die diskursiven Aushandlungsprozesse der indischen Yogatraditionen des 17. Jahrhunderts.
\cleardoublepage
\tableofcontents
\thispagestyle{empty}
\newpage 
\listoffigures
\thispagestyle{empty}
\newpage
\listoftables
\thispagestyle{empty}
\newpage

%\section{Acknowledgements}
%\phantomsection
%\addcontentsline{toc}{chapter}{Acknowledgements}
%\pagestyle{empty} % Verhindert Seitenzahl auf der Seite
%\lettrine{M}{y} interest in the \textit{Tattvayogabindu} stems from my desire to utilize Sanskrit philology to uncover new source texts of yoga traditions and advance our knowledge of the history of yoga. My passion for philological work was inspired by Dr. Patrick McAllister, whose seminar titled ``Text Criticism in Indological Studies'' I attended with great enthusiasm during the winter semester of 2014/2015 at the University of Heidelberg. As an aspiring Indologist, yoga enthusiast, and yoga teacher, the Haṭha Yoga Project (HYP)\footnote{For more information about the ERC-funded \citetitle{hyp-website} (2015-2020), see \url{http://hyp.soas.ac.uk/} (Alternatively: \url{https://web.archive.org/web/20240516171430/http://hyp.soas.ac.uk/}; saved on archive.org: 04.10.2023).}, which began in 2015, was the most exciting Indological venture I could imagine. This research project, funded by the European Research Council and based at SOAS, University of London, with Prof. Dr. James Mallinson as the principal investigator and his team comprising Dr. Mark Singleton, Dr. Jason Birch, Dr. Daniela Bevilaqua, and Dr. S. V. B. K. V. Gupta, aimed primarily at producing critical editions and annotated translations of ten Sanskrit texts of Haṭhayoga. Consequently, I decided to produce a critical edition of the \emph{Gorakṣayogaśāstra}, another early Haṭhayoga text, as part of my master's thesis.
%During the work on this edition, I met Dr. Jason Birch at the Yoga Studies Summer School (YSSS) at Jagiellonian University in Krakow, held between 21.07.2017 and 05.08.2017. Since then, he has supported my work in every possible way. My gratitude goes especially to him, as my Indological career would have taken a different path without his help and encouragement. After completing my master's thesis, the results of which I presented at the World Sanskrit Conference 2018 in Vancouver, Canada, I also met Dr. James Mallinson. Soon after, I found myself on the island of Procida in southern Italy, fortunate to be invited to the two-week \emph{Amṛtasiddhi} workshop of the HYP. In Procida, Dr. Jason Birch inspired me to work on the \emph{Tattvayogabindu} and provided me with the first manuscripts of the text he had collected. The exploration of the overarching theme of complex yoga taxonomies, presented right at the beginning of the text, seemed very promising for advancing Indological yoga research. At another HYP workshop in spring 2019, focused on the \emph{Yogabīja} at the University of Marburg, I met Prof. Dr. Jürgen Hanneder, who promptly agreed to supervise my dissertation on the \emph{Tattvayogabindu}. I am very grateful for his continuous support, philological expertise, and encyclopedic knowledge.
%
%The funding for my work resulted from my position in the AHRC and DFG-funded research project for creating a critical edition and translation of the \emph{Haṭhapradīpikā} (2021-2024), the most important premodern text on physical yoga. I especially want to thank Prof. Dr. James Mallinson and Prof. Dr. Jürgen Hanneder, the principal investigators, for hiring me for this project. Naturally, I am very grateful to the AHRC and DFG for the funding. Working on the \emph{Haṭhapradīpikā}, a text with a highly complex transmission, was very enriching and provided numerous opportunities for further developing my philological skills, my knowledge about the yoga texts and particularly stemmatology. I have learned a lot from working with this team of outstanding scholars, including Dr. Jason Birch and Dr. Mitsuyo Demoto.
%
%In the final phase of my dissertation project, I read my critical edition of the \emph{Tattvayogabindu} with Dr. Jason Birch and Dr. Sven Sellmer in online meetings, discussing textual criticism issues. Prof. Dr. James Mallinson and Dr. Jürgen Hanneder joined these meetings whenever time allowed. I am deeply grateful for every suggestion I received during these reading sessions. I would also like to extend my gratitude to Maximilian Mehner and Dr. Charles Li, who consistently supported me with technical questions in the field of Digital Humanities, both in my work on the \textit{Haṭhapradīpikā} and my dissertation project. My thanks also go to Dr. Robert Alessi, who took the time to answer my questions about his Lua\LaTeX\ module \textit{ekdosis} and even developed its functionality further specifically for the \textit{Haṭhapradīpikā} and the \emph{Tattvayogabindu}.
%I want to thank Dr. Felix Otter for proofreading the Brajbhāṣā passages translated in this work. My gratitude also extends to Bastian Jantke, with whom I frequently discussed issues of my work and who assisted me with his expertise in the Nepali language in deciphering the colophon of the manuscript \getsiglum{N1}. Thanks to Dr. Dominic Haas for answering my questions about the emergence of the \textit{ajapāgāyatrī} in the supplements of the manuscript \getsiglum{U2}, contributing crucial insights for understanding these passages. I thank Prof. Dr. Shaman Hatley for addressing my inquiries about the mother goddesses in the supplementary material of manuscript \getsiglum{U2}. Prof. Dr. Dominik Goodall answered my questions and provided valuable insights into the tenfold \textit{tattva} system presented in the \emph{Tattvayogabindu}. I am grateful to Prof. Dr. Judit Törzsök for her assistance with my questions regarding the eight-petaled lotus within the twelve-petaled lotus in the heart, the origins of which puzzled me for quite some time. I want to thank Dr. Seth David Powell for sending me a digital copy of his dissertation shortly after its submission. I thank Harshall Bhatt from the University of Baroda for his support in order to obtain a copy of a manuscript of the \emph{Tattvayogabindu}.  
%Additionally, I would like to thank all the previously unmentioned participants of the doctoral colloquium in Marburg, whose helpful advice contributed to discussing various issues I faced during my work. This includes Prof. Dr. Roland Steiner, Dr. Martin Straube, Dr. Stanislav Jager, Prof. Dr. Dragomir Dimitrov, and Janina Kuhn. I hope to return the support I received in my academic career to those who need it in the future.
%
%My special thanks go to my beloved partner and mother of my two children, Melanie Amaya, who supported me in every possible way to finish this dissertation. I also want to thank my two beloved daughters, Luna and Kaya, who motivated me daily to complete this work.

\mainmatter
\pagestyle{defaultstyle}
\counterwithout{footnote}{chapter}
\counterwithout{figure}{chapter}
\counterwithout{table}{chapter}
\renewcommand{\thetable}{\arabic{table}}
%%%tables 
\setsecnumdepth{section}
\maxsecnumdepth{subsubsection}
\newpage
\chapter{Introduction}
\cleardoublepage

\begin{landscape}
\thispagestyle{empty}

\vspace*{\fill}
\begin{center}
  \includegraphics[
    width=0.8\paperheight,
    height=1.2\paperwidth,
    keepaspectratio
  ]{pics/colosynop.png}
\end{center}
\vspace*{\fill}

\captionof{figure}{Synoptic transcription of the manuscripts' final rubrics.}
\label{fig:colosynop}

\end{landscape}

\chapter[Critical Edition \& Annotated Translation of the \emph{Tattvayogabindu}]{The \emph{Tattvayogabindu} of Rāmacandra \\ \huge  
  Critical Edition \& Annotated Translation}
\pagestyle{chapter2style}
\newpage
\begin{criticaledition}

  \end{criticaledition}

\newpage
\selectlanguage{english}
\chapter{Appendix}
\section{Figures}
 
% \begin{landscape}
\clearpage

  \begin{figure}[ht]
	\centering
  \includegraphics[width=1\textwidth]{pics/Wolpertinger.png}
\caption[The \textit{dehasvarūpa} of \textit{ajapāgāyatrī}]{The \textit{dehasvarūpa} of \textit{ajapāgāyatrī}. The image, reminiscent of a hippogriff, is part of an illustrated Sanskrit manuscript written in the Śāradā script. Preserved as a single large scroll under Acc. No. 1334 at the Oriental Institute in Srinagar (Kashmir), it is entitled \textit{Nāḍīcakra}. The manuscript contains a depiction of the yogic body’s \textit{cakra}s and \textit{nāḍī}s. The text surrounding the figure closely corresponds to the additional material found in manuscript \getsiglum{U2} of the \textit{Tattvayogabindu}. The manuscript reads (diplomatic transcription): \textit{oṃ daśame pūrṇagiripīṭhe lalāṭamaṇḍale candro devatā amṛtāśaktiḥ paramātmā ṛṣiḥ dvāviṃśaddalāni amṛtavāsinikalā 4: ambikā 1 lambikā 2 gha(ṃ)ṭkā 3 tālikā 4 dehasvarūpaṃ kākamukhaṃ 1 naranetraṃ 2 gośṛṅgaṃ 3 lalāṭabrahmapara 4 hayagrīvā 5 mayūramuśchaṃ 6 haṃsacārītani 7 sthāna.}}
	\phantomsection\label{fig_wolpertinger}
      \end{figure}

      \clearpage

  \begin{figure}[ht]
	\centering
  \includegraphics[width=1\textwidth]{pics/Vishnu_Vishvarupa_cropped.jpg}
	\caption{Viṣṇu Viśvarūpa, India, Rajasthan, Jaipur, ca. 1800–1820, Opaque watercolor and gold on paper, 38.5 × 28 cm, Victoria and Albert Museum, London, Given by Mrs. Gerald Clark.}
	\label{fig1}
      \end{figure}
\clearpage
  \begin{figure}[ht]
	\centering
  \includegraphics[width=0.5\textwidth]{pics/The_Equivalence_of_Self_and_Universe_(detail),_folio_6_from_the_Siddha_Siddhanta_Paddhati,_(Bulaki),_1824_(Samvat_1881);_122_x_46_cm._Mehrangarh_Museum_Trust..jpg}
	\caption{The Equivalence of Self and Universe (detail), folio 6 from the \textit{Siddhasiddhāntapaddhati} (Bulaki), India, Rajasthan, Jodhpur, 1824 (Samvat 1881), 122 x 46 cm, RJS 2378, Mehragarh Museum Trust.}
	\label{fig2}
      \end{figure}
      % \end{landscape}

      \newpage
      \cleardoublepage
\chapter{Bibliography}
 \label{sec:bibli}
\clearpage
\newpage 
\thispagestyle{empty}
\quad  \addtocounter{page}{-1}

\newrefcontext[sorting=tixel]
\printbibliography[heading=subbibintoc, title=Primary Sources, keyword=primary]

\newrefcontext[sorting=nyt]
\printbibliography[heading=subbibintoc, title=Secondary Literature, keyword=seclit]

\printbibliography[heading=subbibintoc, title=Catalogues, keyword=catalogues]

\printbibliography[heading=subbibintoc, title=Online Sources, keyword=onlinesource]

\end{document}


%%% Local Variables:
%%% mode: latex
%%% TeX-master: t
%%% End:
