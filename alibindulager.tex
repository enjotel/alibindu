\chapter{Critical Edition \& Annotated Translation}
\cleardoublepage
\begin{alignment}[
    texts=edition[class="edition"];
    translation[class="translation"],
  ]
\begin{edition}
 \ekddiv{type=ed}
 \centerline{\textrm{\small{[\uproman{1}.\textsuperscript{\coro{[\lowroman{1}-\lowroman{10}]}} Introduction]}}}
 \label{intro}
    \bigskip
    \begin{prose}
%--------------------------
% śrī gaṇeśāya namaḥ /                                                     rājayogāntargataḥ //  binduyogaḥ   \E 
% śrī gaṇeśāya namaḥ /                                                     atha tattvabiṃduyogaprāraṃbhaḥ     \L
% śrī ṇe ya maḥ /                                                          atha rājayoga         liṣyate      \P
% \om                                                                                                         \B      
% śrī gaṇeśāya namaḥ // śrī gurave namaḥ //                                atha rājayogaprakāro  likhyate //  \N1
% śrī gaṇeśāya namaḥ //                                                //  atha rājayogaprakāro  likhyate //  \N2
% śrī gaṇeśāya namaḥ // śrī sarasvatyai namaḥ // śrī nirañjanāya namaḥ //  atha rājayogaprakāro  likhyate //  \D
% śrī gaṇeśāya namaḥ /  oṃ śrī niraṃjanāya //                              atha rājayogaprakāra  likhyate //  \U1
% śrī gaṇeśāya namaḥ /                                                     atha rājayoga         likhyate //  \U2
%--------------------------
%Homage to Śrī Gaṇeśa. Now the methods of rājayoga are laid down.
%--------------------------          
\noindent \app{\lem[wit={ceteri}]{śrī gaṇeśāya namaḥ}
        \rdg[wit={P}]{śrī ṇe ya maḥ}
        \rdg[wit={N1}]{śrī gaṇeśāya namaḥ || śrī gurave namaḥ ||}
        \rdg[wit={D}]{śrī gaṇeśāya namaḥ || śrī sarasvatyai namaḥ || śrī nirañjanāya namaḥ ||}
        \rdg[wit={U1}]{śrī gaṇeśāya namaḥ || oṃ śrī niraṃjanāya ||}}\dd{}
\app{\lem[wit={N1,N2,D}]{atha rājayogaprakāro likhyate}
        \rdg[wit={U1}]{atha rājayogaprakāra likhyate}
        \rdg[wit={E}]{rājayogāntargataḥ || binduyogaḥ}
        \rdg[wit={L}]{atha tattvabiṃduyogaprāraṃbhaḥ}
        \rdg[wit={P}]{atha rājayoga liṣyate}
        \rdg[wit={U2}]{atha rājayoga likhyate}}\dd{}
%-------------------------- 
% \om                        \E
% \om                        \L
% \om                        \B
% rājayogasyedaṃ phalaṃ      \P
% rājayogasya idaṃ phalaṃ    \N1
% rājayogasya idaṃ phalaṃ    \N2
% rājayogasya idaṃ phalaṃ // \D
% rājayogasya idaṃ phalaṃ    \U1
% rājayogasyedaṃ phalaṃ /    \U2
%--------------------------    
\app{\lem[wit={P,U2}]{rājayogasyedaṃ phalaṃ}
  \rdg[wit={N1,N2,D}]{rājayogasya idaṃ phalaṃ}
  \rdg[wit={E,L}]{\om}}/
%--------------------------
%This is the result of \textit{rājayoga}:
%--------------------------
% \om                                                                                                                                                                         \E
% \om                                                                                                                                                                         \L
% \om                                                                                                                                                                         \B
% yena rājayogenāneka---rājyabhogasamaya   eva   anekapārthivavinodaprekṣaṇasamaya  eva   bahutarakālaṃ  śarīrasthitir  bhavati    sa eva  rājayogaḥ tasyaite     bhedāḥ      \P
% yena rājayogenāneka---rājyabhogasamaya   eva/  anekapārthivavinodaprekṣaṇasamaya  eva/  bahutarakālaṃ  śarīrasthitir  bhavati    sa eva  rājayogaḥ /  tasya ete bhedāḥ /    \N1
% yena rājayogena  anekarājyabhogasamaya   eva// anekapārthivavinodaprekṣaṇasamaya  eva   bahuttarakālaṃ śarīrasthitir  bhavati    sa eva  rājayogaḥ /  tasya ete bhedāḥ /    \N2
% yena rājayogena  anekarājyabhogasamaya   eva// anekapārthivavinodaprekṣaṇasamaya  eva// bahutarakālaṃ  śarīrasthitir  bhavati//  sa eva  rājayogaḥ // tasya ete bhedāḥ /    \D
% yena rājayogena  anekarājyabhogasamaya   eva// anekapārthivavinodaprekṣaṇasamaya  eva// bahutarakālaṃ  śarīrasthitir  bhavati    sa evaṃ rājayogaḥ    tasya ete bhedāḥ //   \U1 
% yena rājayogena  anekarājyabhogasamaya   eva// anekapārthivavinodaprekṣyaṇasamaya eva// bahutarakālaṃ  śarīrasthitir  bhavati//  sa eva  rājayogastaisyaite     bhedāḥ //   \U2
% --------------------------
%\textit{Rājayoga} is that by which longterm durability of the body arises even amongst manifold royal pleasures even amongst the manifold royal entertainments and spectacle. This truly is \textit{rājayoga}. Of this [\textit{rājayoga}] these are the varieties: \end{tlate}
%--------------------------
yena rāja\app{\lem[wit={P,N1}, alt={°yogenāneka°}]{yogenāneka}
  \rdg[wit={D,N2,U1,U2}]{°yogena aneka°}
}rājyabhogasamaya eva/ anekapārthivavinoda
      \app{\lem[wit={ceteri}]{prekṣaṇasamaya}
        \rdg[wit={U2}]{prekṣyaṇasamaya}}
      eva/\textsuperscript{\begin{otherlanguage}{english}\coro{[\lowroman{5}]}\end{otherlanguage}} bahutarakālaṃ śarīrasthitir-bhavati/
      sa \app{\lem[wit={ceteri}]{eva}
        \rdg[wit={U2}]{evaṃ}}
      \app{\lem[wit={ceteri}]{rājayogaḥ}
        \rdg[wit={U2}]{rājayogas}}/ 
      \app{\lem[wit={P,U2}]{tasyaite}
        \rdg[wit={ceteri}]{tasya ete}} bhedāḥ/
%-------------------------
%
% \om                                                                                                                                                                \E
% \om                                                                                                                                                                \L
% \om                                                                                                                                                                \B
% kriyāyogaḥ 1 jñānayogaḥ 2 caryāyogaḥ 3 haṭhayogaḥ 4 karmayogaḥ 5 layayogaḥ 6 dhyānayogaḥ 7 maṃtrayogaḥ 8 lakṣyayogaḥ 9 vāsanāyogaḥ 10 śivayogaḥ 11 brahmayogaḥ 12 advaitayogaḥ 13 siddhayogaḥ 14 rājayogaḥ 15 ete paṃcadaśayogāḥ \P
%
% kriyāyogaḥ / jñānayogaḥ / caryāyogaḥ / haṭhayogaḥ / karmayogaḥ / layayogaḥ / dhyānayogaḥ / maṃtrayogaḥ / lakṣyayogaḥ / vāsanāyogaḥ / śivayogaḥ / brahmayogaḥ / advaitayogaḥ / rājayogaḥ / siddhayogaḥ / ete paṃcadaśayogāḥ // \N1
%
% kriyāyogaḥ jñānayogaḥ caryāyogaḥ haṭhayogaḥ karmayogaḥ layayogaḥ dhyānayogaḥ maṃtrayogaḥ lakṣayogaḥ vāsanāyogaḥ śivayogaḥ brahmayogaḥ advaitayogaḥ rājayogaḥ siddhayogaḥ // ete paṃcadaśayogāḥ // \N2      
%      
% kriyāyogaḥ // jñānayogaḥ // caryāyogaḥ // haṭhayogaḥ // karmayogaḥ // layayogaḥ // dhyānayogaḥ // maṃtrayogaḥ // lakṣyayogaḥ // vāsanāyogaḥ // śivayogaḥ // brahmayogaḥ // advaitayogaḥ // rājayogaḥ // siddhayogaḥ // ete paṃcadaśayogāḥ // \D
% \om                                                                                                                                                                         \D2      
%
% kriyāyogaḥ // jñānayogaḥ // tvaryāyogaḥ // haṭhayogaḥ // karmayogaḥ // layayogaḥ // dhyānayogaḥ maṃtrayogaḥ  lakṣayogaḥ  vāsanāyogaḥ  śivayogaḥ  brahmayogaḥ  advaitayogaḥ  rājayogaḥ  siddhayogaḥ ete paṃcadaśayogāḥ  \U1
%
% kriyāyogaḥ // jñānayogaḥ // caryāyogaḥ // haṭhayogaḥ // karmayogaḥ // nayayogaḥ // dhyānayogaḥ // maṃtrayogaḥ // lakṣyayogaḥ // vāsanāyogaḥ // śivayogaḥ // brahmayogaḥ // advaitayogaḥ // siddhayogaḥ // rājayogaḥ // evaṃ paṃcadaśāyogā bhavaṃti // \U2
%-------------------------
         kriyāyogaḥ 1\dd{}
         jñānayogaḥ 2\dd{}
         \app{\lem[wit={ceteri}]{caryāyogaḥ}
          \rdg[wit={U1}]{tvaryāyogaḥ}} 3\dd{}
        haṭhayogaḥ 4\dd{}
        karmayogaḥ 5\dd{}
        \app{\lem[wit={ceteri}]{layayogaḥ}
          \rdg[wit={U2}]{nayayogaḥ}} 6\dd{}
        dhyānayogaḥ 7\dd{}
        mantrayogaḥ 8\dd{}
        \app{\lem[wit={ceteri}]{lakṣyayogaḥ}
          \rdg[wit={U1}]{lakṣayogaḥ}} 9\dd{}
        vāsanāyogaḥ 10\dd{}
        śivayogaḥ 11\dd{} 
        brahmayogaḥ 12\dd{}
        advaitayogaḥ 13\dd{} 
        \app{\lem[wit={P,U2}]{siddhayogaḥ}
          \rdg[wit={D,N1,N2,U1}]{rājayogaḥ}} 14\dd{}
        \app{\lem[wit={P,U2}]{rājayogaḥ}
          \rdg[wit={ceteri}]{siddhayogaḥ}} 15\dd{}     
        \note[type=philcomm, labelb=1, lem={rājayoga}]{The initial codification of 15 \textit{yoga}s appears in N\textsubscript{1}, N\textsubscript{2}, P, D, U\textsubscript{1} and U\textsubscript{2}. It is ommitted in E, L and B (missing folio). It is also absent in the \textit{Yogasaṃgraha}.}
        \note[type=source, labelb=2, lem={\textbf{Re}}]{PT\textsuperscript{qcr \cdot YSV} (Ed. p. 831): pañcadaśaprakāro 'yaṃ rājayogaḥ || kriyāyogo jñānayogaḥ karmayogo haṭhas tathā | dhyānayogo mantrayoga urayogaś ca vāsanā |  rājaty etad brahmavaśīva ebhiś ca pañcadaśadhā | idānīṃ lakṣaṇañ caiṣāṃ kathayāmi śṛṇu priye |}
        \note[type=testium, labelb=3, lem={\textbf{Ri}}]{\textit{Yogasiddāntacandrikā} (Ed. p. 2): nididhyāsanañcaika tānatādirūpo rājayogāparaparyāyaḥ samādhiḥ | tatsādhanaṃ tu kriyāyogaḥ, caryāyogaḥ, karmayogo, haṭhayogo, mantrayogo, jñānayogaḥ, advaitayogo, lakṣyayogo, brahmayogaḥ, śivayogaḥ, siddhiyogo, vāsanāyogo, layayogo, dhyānayogaḥ, premabhaktiyogaś ca |}
        \app{\lem[wit={D,N1,P,U1}]{ete pañcadaśayogāḥ}
          \rdg[wit={U2}]{evaṃ paṃcadaśāyogā bhavaṃti}}\dd{}\textsuperscript{\begin{otherlanguage}{english}\coro{[\lowroman{10}]}\end{otherlanguage}}
      \end{prose} \vfill
\nolinenumbers    
\smallskip
\centerline{\textrm{\small{[\uproman{2}.\textsuperscript{\coro{\lowroman{1}-\lowroman{10}}} Kriyāyoga]}}}
\label{kriyayoga}
\bigskip
\linenumbers
%--------------------------        
% \om                                      \E
% \om                                      \L
% \om                                      \B
% idānīṃ kriyāyogasya lakṣaṇaṃ kathyate/   \P
% idānīṃ kriyāyogasya lakṣaṇaṃ kathyate/   \N1
% idānī  kriyāyogasya lakṣaṇaṃ kathyate//  \N2
% idānīṃ kriyāyogasya lakṣaṇaṃ kathayate/  \D
% \om                                      \D2
% idānīṃ kriyāyogasya lakṣaṇaṃ kathyate/   \U1
% atha   kriyāyogas   lakṣaṇaṃ          // \U2
%--------------------------
%Now the characteristic of the Yoga of [mental] action (\textit{kriyāyoga}) described.
%--------------------------
\begin{prose}
        \app{\lem[wit={ceteri}]{idānīṃ}
            \rdg[wit={N2}]{idānī}
            \rdg[wit={U2}]{atha}}
          \app{\lem[wit={ceteri}]{kriyāyogasya}
            \rdg[wit={U2}]{kriyāyogas}} lakṣaṇaṃ
          \app{\lem[wit={ceteri}]{kathyate}
            \rdg[wit={D}]{kathayate}
            \rdg[wit={U2}]{\om}}/
\end{prose}
\begin{tlg}
%--------------------------   
% \om                                                    \E
% \om                                                    \L
% \om                                                    \B
% kriyāmuktir    ayaṃ yogaḥ    svapiṇḍe siddhidāyakaḥ    \P
% kriyāmuktir    ayaṃ yogaḥ /  svapiṇḍe siddhidāyakaḥ /  \N1
% kriyāmukti    layaṃ yogaḥ    svapiṇḍe siddhidāyakaḥ /  \N2
% kriyāmuktir    ayaṃ yogaḥ    svapiṇḍe siddhidāyakaḥ /  \D
% \om                                                    \D2
% kriyāyuktir    ayaṃ yogaḥ /  svapiṇḍe siddhidāyakaḥ /  \U1
% kriyāmuktiḥ // ayaṃ yogaḥ    svapiṇḍe siddhidāyakaṃ // \U2 
%--------------------------
%This Yoga is liberation through [mental] action, it bestows success(\textit{siddhi}) in ones own body.
%-------------------------- 
   \tl{\note[type=source, labelb=4, lem=\textbf{Cee}]{PT\textsuperscript{ccn \cdot YSV} (Ed. p. 831): kriyāmuktimayo (\textit{kriyāmuktir ayaṃ} YK\textsuperscript{ccn \cdot YSV} 1.209  Ed. p. 17) yogaḥ sapiṇḍisiddhidāyakaḥ (\textit{sapiṇḍe} YK\textsuperscript{ccn \cdot YSV} 1.210 Ed. p. 17) | yatkāromīti saṅkalpaṃ kāryārambhe manaḥ sadā ||}
     \app{\lem[wit={ceteri}, alt={kriyāmuktir}]{kriyāmukti\skp{r-a}}
    \rdg[wit={N2}]{kriyāmukti}
    \rdg[wit={U2}]{kriyāmuktiḥ ||}
}\app{\lem[wit={ceteri}, alt={ayaṃ}]{\skm{r-a}yaṃ}
  \rdg[wit={N2}]{layaṃ}}
\app{\lem[wit={ceteri}]{yogaḥ}
  \rdg[wit={N1,U1}]{yogaḥ |}} svapiṇḍe
\app{\lem[wit={ceteri}]{siddhidāyakaḥ}
  \rdg[wit={U2}]{siddhidāyakaṃ}}/}\\
%-------------------------
% \om                                                    \E
% \om                                                    \L
% \om                                                    \B
% yaṃ yaṃ karoti kallolaṃ kāryāraṃbhe manaḥ sadā         \P
% yaṃ yaṃ karoti kallolaṃ kāryāraṃbhe manaḥ sadā/        \N1
% yaṃ yaṃ karoti kallolaṃ kāryāraṃbhe manaḥ sadā//1//    \N2
% yaṃ yaṃ karoti kallolaṃ kāryāraṃbhe manaḥ sadā/        \D
% \om                                                    \D2
% yaṃ yaṃ karoti kallolaṃ kāryāraṃbhe manaḥ sadā/ 1      \U1
% yaṃ yaṃ karoti kallolaṃ kāryāraṃbhe manaḥ sadā/        \U2
%--------------------------
%Each wave the mind creates at the beginning of an action,
%-------------------------- 
\tl{yaṃ yaṃ karoti kallolaṃ kāryāraṃbhe manaḥ sadā/}\\
%--------------------------
% \om                                                        \E
% \om                                                        \L
% \om                                                        \B
% tattataḥ   kuñcanaṃ kurvan kriyāyogas tato bhavet            \P
% tattataḥ   kuñcanaṃ kurvan kriyāyogas ato bhava     //       \N1
% tattataḥ   kūrcanaṃ kurvan kriyāyogas ato bhava     //       \N2
% tattataḥ   kuñcanaṃ kurvan kriyāyogas ato bhava     //       \D
% \om                                                          \D2
% taṃkṛ taṃ  kuñcanaṃ kurvan kriyāyogas ato ?va      //1//    \U1
% tatastataḥ kuṃcanaṃ kurvan kriyāyogas tato bhavet //1//     \U2
%--------------------------
%of all those one shall withdraw oneself. Then \textit{kriyāyoga} arises.
%--------------------------
\tl{\note[type=source, labelb=5, lem=\textbf{Cee}]{PT\textsuperscript{ccn \cdot YSV} (Ed. p. 839): tatsāṅgācaraṇaṃ kurvan kriyāyogarato bhavet |}
  \app{\lem[wit={ceteri}]{tattataḥ}
    \rdg[wit={U2}]{tatas tataḥ}
    \rdg[wit={U1}]{taṃkṛ taṃ}}
  \app{\lem[wit={ceteri}]{kuñcanaṃ}
    \rdg[wit={N2}]{kūrcanaṃ}}
  kurvan-kriyāyoga\skp{s-ta}\app{\lem[wit={P,U2}, alt={tato bhavet}]{\skm{s-t}ato bhavet}
    \rdg[wit={D,N1,N2}]{ato bhava}
    \rdg[wit={U1}]{ato va}}\dd{}\begin{otherlanguage}{english}\coro{\uproman{2}.\textsuperscript{\lowroman{2}}}\end{otherlanguage}\hskip-2pt\dd{}}
\end{tlg}
\vfill
\ekdpb*{}
\end{edition}
\begin{translation}
  \ekddiv{type=trans}
  \centerline{\textrm{\small{[\uproman{1}.\textsuperscript{\coro{\lowroman{1}-\lowroman{10}}} Introduction]}}}
  \label{introtrans}
\bigskip
\begin{tlate}
\noindent
Homage to Śrī Gaṇeśa. Now the methods of Rājayoga are written down. \\
This is the result of Rājayoga\footnote{This statement seems unconnected to the definition of rājayoga that follows.}: Rājayoga is that by which long-term durability of the body arises
[and] even amongst manifold royal pleasures even amongst the manifold royal entertainments and spectacle.\textsuperscript{\coro{[\lowroman{4}-\lowroman{8}]}} This truly is Rājayoga. These are the varieties of this Rājayoga:\\\\
\indent \textbf{1.} The Yoga of [mental] action (Kriyāyoga); \textbf{2.} the Yoga of knowledge (Jñānayoga); \textbf{3.} the Yoga of wandering (Caryāyoga);\footnote{The first three Yogas allude to the four \textit{pāda}s of the Śaiva \textit{āgama}s; namely \textit{kriyā}[\textit{pāda}], \textit{caryā}[\textit{pāda}], \textit{yoga}[\textit{padā}] and \textit{jñāna}[\textit{pāda}], see \citeauthor[2015: 77]{nishvasa2015}.} \textbf{4.} the Yoga of force (Haṭhayoga); \textbf{5.} the Yoga of deeds (Karmayoga); \textbf{6.} the Yoga of absorption (Layayoga); \textbf{7.} the Yoga of meditation (Dhyānayoga); \textbf{8.} the Yoga of Mantras (Mantrayoga); \textbf{9.} the Yoga of targets (Lakṣyayoga); \textbf{10.} Yoga of mental residues (Vāsanāyoga); \textbf{11.} the Yoga of Śiva (Śivayoga); \textbf{12.} the Yoga of Brahman (Brahmayoga); \textbf{13.} the Yoga of non-duality (Advaitayoga); \textbf{14.} the Yoga of the Siddhas (Siddhayoga); \textbf{15.} the Yoga of kings (Rājayoga). These are the fifteen Yogas.\textsuperscript{\coro{[\lowroman{10}]}}\footnote{The authenticity of the list of the fifteen Yogas present at the beginning of the text is uncertain. It remains unclear whether the list is a subsequent addition by another scribe or if it is, in fact, a part of the original text composed by Rāmacandra. Despite the suggestion of a sequential arrangement of Yogas in the list, the text only loosely follows the order presented. This raises questions about the reliability of the list and its relationship to the rest of the text. A more detailled investigation of the 15 Yogas can be found at p. \pageref{yogas_list}.}\\\\
      \bigskip 
      \centerline{\textrm{\small{[\uproman{2}.\textsuperscript{\coro{\lowroman{1}-\lowroman{10}}} \textit{Kriyāyoga}]}}}
      \label{kriyayogatrans}
      \bigskip
      \indent Now the characteristic of Kriyāyoga, the Yoga of [mental] action is described. \paragraph{\uproman{2}\textsuperscript{\lowroman{2}}.} This Yoga is liberation through [mental] action. It bestows success(\textit{siddhi}) in one's own body. Each wave the mind creates at the beginning of an action, of all those, one shall withdraw oneself. Then Kriyāyoga arises.\footnote{All four verses on Kriyāyoga were taken from the \textit{Yogsavarodaya} as quotations in the \textit{Prāṇatoṣinī} and \textit{Yogakarṇikā}. No sources for the following prose section can be identified.}\vspace*{\fill}
      \vfill
\end{tlate}
 \ekdpb*{}
   \end{translation}
 \end{alignment}
%%%%%%%%%%%%%%%%%%%%%%%%%%%%%%%%%%%%%%%%%%
%%%%%%%%%%%%%%%%%%%%%%%%%%%%%%%%%%%%%%%%%%
%%%%%%%%PAGEBREAK%%%%%%%PAGEBREAK%%%%%%%%%
%%%%%%%%%%%%%%%%%%%%%%%%%%%%%%%%%%%%%%%%%%
%%%%%%%%%%%%%%%%PAGEBREAK%%%%%%%%%%%%%%%%%
%%%%%%%%%%%%%%%%%%%%%%%%%%%%%%%%%%%%%%%%%%
%%%%%%%%PAGEBREAK%%%%%%%PAGEBREAK%%%%%%%%%
%%%%%%%%%%%%%%%%%%%%%%%%%%%%%%%%%%%%%%%%%%
%%%%%%%%%%%%%%%%%%%%%%%%%%%%%%%%%%%%%%%%%%
%%%%%%%%%%%%%%%%%%%%%%%%%%%%%%%%%%%%%%%%%%
%%%%%%%%%%%%%%%%%%%%%%%%%%%%%%%%%%%%%%%%%%
%%%%%%%%PAGEBREAK%%%%%%%PAGEBREAK%%%%%%%%%
%%%%%%%%%%%%%%%%%%%%%%%%%%%%%%%%%%%%%%%%%%
%%%%%%%%%%%%%%%%PAGEBREAK%%%%%%%%%%%%%%%%%
%%%%%%%%%%%%%%%%%%%%%%%%%%%%%%%%%%%%%%%%%%
%%%%%%%%PAGEBREAK%%%%%%%PAGEBREAK%%%%%%%%%
%%%%%%%%%%%%%%%%%%%%%%%%%%%%%%%%%%%%%%%%%%
%%%%%%%%%%%%%%%%%%%%%%%%%%%%%%%%%%%%%%%%%%
%%%%%%%%%%%%%%%%%%%%%%%%%%%%%%%%%%%%%%%%%%
%%%%%%%%%%%%%%%%%%%%%%%%%%%%%%%%%%%%%%%%%%
%%%%%%%%PAGEBREAK%%%%%%%PAGEBREAK%%%%%%%%%
%%%%%%%%%%%%%%%%%%%%%%%%%%%%%%%%%%%%%%%%%%
%%%%%%%%%%%%%%%%PAGEBREAK%%%%%%%%%%%%%%%%%
%%%%%%%%%%%%%%%%%%%%%%%%%%%%%%%%%%%%%%%%%%
%%%%%%%%PAGEBREAK%%%%%%%PAGEBREAK%%%%%%%%%
%%%%%%%%%%%%%%%%%%%%%%%%%%%%%%%%%%%%%%%%%%
%%%%%%%%%%%%%%%%%%%%%%%%%%%%%%%%%%%%%%%%%%
\begin{alignment}[
    texts=edition[class="edition"];
    translation[class="translation"],
  ]
\begin{edition}
 \ekddiv{type=ed}
    \begin{tlg}
%--------------------------      
% \om                                                                                                 \B
% \om                                                                                                 \L
% kṣamā vivekaṃ vairāgyaṃ śāntiḥ santoṣaniṣpṛhā       etadyuktiyuto  yogī   kriyāyogī nigadyate       \E
% kṣamāvivekavairāgyaṃ    śāntiḥ santoṣanispṛhāḥ      etadyuktiyuto  yogī   kriyāyogī nigadyate       \P
% kṣamāvivekavairāgyaṃ    śāntiḥ santoṣanispṛhā       etat yuktiyuto yogī   kriyāyogī nigadyate       \N1
% kṣamāvivekavairāgyaṃ    śāntiḥ santoṣanispṛhā //2// etat yuktiyuto yo sau kriyāyogī nigadyate//     \N2
% kṣamāvivekavairāgyaṃ    śāntiḥ santoṣanispṛhaḥ      etat yuktiyuto yogī   kriyāyogī nigadyate       \D
% \om                                                                                            \D2
% kṣamāvivekavairāgya---- śāntisantoṣaniḥspṛhī        etad yuktiyuto  yo sau kriyāyogī nigadyate       \U1 
% kṣamā vivekaṃ vairāgyaṃ śāntisaṃtoṣaniṣpṛhāḥ //     etat muktiyuto yogī   kriyāyogī nigadyate //2// \U2
%--------------------------
%Patience, discrimination, equanimity, peace, modesty, desireless: He who is endowed with these means is said to be a \textit{kriyāyogī}.
%--------------------------
% The text of the Printed Edition starts here ---> 
%--------------------------
      \tl{\note[type=source, labelb=6, lem=\textbf{Cee}]{PT\textsuperscript{ccn \cdot YSV} (Ed. p. 831): kṣamāvivekavairāgyaśāntisantoṣanispṛhāḥ | etan muktiyuto yo 'sau (\textit{muktiyutaś cāsau} YK\textsuperscript{ccn \cdot YSV} 1.211 Ed. p. 17) kriyāyogo nigadyate |}
kṣamā\app{\lem[wit={ceteri}, alt={°viveka°}]{viveka}\rdg[wit={E,U2}]{vivekaṃ}}vairāgyaṃ\note[type=philcomm, labelb=8, lem={kṣamā°}]{E begins here.}śāntisantoṣa\app{\lem[wit={P},alt={°nispṛhāḥ}]{nispṛhāḥ}
          \rdg[wit={D}]{°nispṛhaḥ}
          \rdg[wit={E,N1}]{°nispṛhā}
          \rdg[wit={N2}]{°niṣpṛhā ||2||}
          \rdg[wit={U1}]{°niṣpṛhī}
          \rdg[wit={U2}]{°niṣpṛhāḥ ||}}/}\\
      \tl{\app{\lem[wit={E,P,U1},alt={etad}]{eta\skp{d-yu}}
          \rdg[wit={D,N1,N2,U2}]{etat}
}\app{\lem[wit={ceteri}, alt={yuktiyuto}]{\skm{d-yu}ktiyuto}  %%%SANDHI
    \rdg[wit={U2}]{muktiyuto}}
  \app{\lem[wit={N2,U1}]{yo 'sau}  
    \rdg[wit={D,E,P,N1,U2}]{yogī}}
kriyāyogī nigadyate\dd{}\begin{otherlanguage}{english}\coro{\uproman{2}.\textsuperscript{\lowroman{3}}}\end{otherlanguage}\hskip-2pt\dd{}}
\end{tlg}
\begin{tlg}
%-----------------------
% \om                                                \B
% \om                                                \L
% mātsaryaṃ mamatā māyā hiṃsā ca   madagarvitā /     \E
% mātsarya  mamatā māyā hiṃsāśā    madagarvitāḥ      \P
% mātsarya  mamatā māyā hiṃsāḥ //  madagarvatā /     \N1    -> the hiṃsā---''ḥ//'' in \nepal looks like a śā -> indicator that the others copied from \nepal? 
% mātsarya  mamatā māyā hiṃsāśā    madagārvatā //3// \N2
% mātsarya  mamatā māyā hiṃsāśā    madagarvatā /     \D
% \om                                                \D2
% mātsaryaṃ mamatā māyā hiṃsāśā    madagarvatā /     \U1
% mātsaryaṃ mamatā māyā hiṃsāśā    madagarvatā /     \U2
%-----------------------
%Envy, selfishness, cheating, violence, desire and intoxication, pride,
%-----------------------
       \tl{\note[type=source, labelb=9, lem=\textbf{Ce}]{PT\textsuperscript{ccn \cdot YSV} (Ed. p. 831): mātsaryaṃ mamatā māyā hiṃsā ca madagarvitā | kāmaḥ krodho bhayaṃ lajjā lobho mohas tathā 'śuciḥ (\textit{śuciḥ} YK\textsuperscript{ccn \cdot YSV} 1.212 Ed. p. 17) ||}
         \app{\lem[wit={E,U1,U2}]{mātsaryaṃ}
           \rdg[wit={D,N1,P}]{mātsarya}}
         mamatā māyā
         \app{\lem[wit={E}]{hiṃsā ca}
           \rdg[wit={ceteri}]{hiṃsāśā}
           \rdg[wit={N1}]{hiṃsāḥ ||}}
         madagarvatā/}\\
%-----------------------
% \om                                                   \B
% \om                                                   \L
% kāmakrodhabhayaṃ   lajjā lobhamohau tathā śuciḥ //    \E
% kāmakrodhabhayaṃ   lajjā lobhamohau tathā 'śuciḥ      \P
% kāmakrodhabhayaṃ   lajjā lobhamohau tathā 'śuciḥ /    \N1    -> the hiṃsā---''ḥ//'' in \nepal looks like a śā -> indicator that the others copied from \nepal? 
% kāmakrodho bhayaṃ  lajjā lobhamohau tathā śuciḥ //    \N2
% kāmakrodho bhayaṃ  lajjā lobhamohau tathā 'śuciḥ //   \D
% \om                                                   \D2
% kāmakrodhau bhayaṃ lajjā lobhamohau tathā 'śuciḥ      \U1
% kāmakrodhau bhayaṃ lajjā lobhamohau tathā śuciḥ //3// \U2
% -----------------------
% lust, anger, fear, laziness, greed, error and impurity.
%-----------------------
       \tl{kāma\app{\lem[wit={U1,U2}, alt={°krodhau}]{krodhau}
           \rdg[wit={E,N1,P}]{krodha°}
           \rdg[wit={D}]{°krodho}}
         bhayaṃ lajjā lobhamohau tathā
         \app{\lem[wit={ceteri}]{'śuciḥ}
           \rdg[wit={E,N2,U2}]{śuciḥ}}\dd{}\begin{otherlanguage}{english}\coro{\uproman{2}.\textsuperscript{\lowroman{4}}}\end{otherlanguage}\hskip-2pt\dd{}}    %%%AVAGRAHA
\end{tlg}
\begin{tlg}
%-----------------------
%  \om                                                           \B
%  atha dveṣo ghṛṇālasyaṃ bhrāṃtir   daṃbho kṣamā bhramaḥ //     \L
%  rāgadveṣau ghṛṇālasyaṃ bhrāntitvaṃ     mokṣamā bhramaḥ /      \E
%  rāgadveṣau ghṛṇālasyaṃ bhrāṃtir   ddaṃbhokaṣmā bhramaḥ        \P
%  rāgadveṣau ghṛṇālasyaṃ bhrāṃtir   daṃbho kṣamā bhramaḥ //4//  \N1
%  rāgadveṣau ghṛnālasyaṃ bhrāṃtir   daṃbho kṣamā bhramaḥ //4    \N2
%  rāgadveṣau ghṛṇālasyaṃ bhrāṃtir   debho  kṣamā bhramaḥ //     \D
% \om                                                            \D2
%  rāgadoṣau  ghṛṇālasyaṃ bhrāṃti    daṃbha kṣamī bhramaḥ 4      \U1
%  rāgadveṣau ghṛṇālasyaṃ bhrāṃtir   daṃbho kṣamā bhramaḥ //     \U2
%-----------------------
%Attachment and aversion, indignation and idleness, impatience and dizzyness
%-----------------------
        \tl{\note[type=source, labelb=10, lem=\textbf{Ce}]{PT\textsuperscript{ccn \cdot YSV} (Ed. p. 831): rāgadveṣau ghṛṇālasyaśrāntidambhakṣamābhramāḥ (\textit{ghṛṇālasyaṃ bhrāntir dambho 'kṣamā bhramaḥ} YK\textsuperscript{ccn \cdot YSV} 1.213 Ed. p. 17) | yasyai tāni na vidyante kriyāyogī sa ucyate ||}
          \app{\lem[wit={ceteri}]{rāgadveṣau}
            \rdg[wit={U1}]{rāgadoṣau}
            \rdg[wit={L}]{atha dveṣo}}\note[type=philcomm, labelb=11, lem={rāga°}]{L begins here.}
          \app{\lem[wit={ceteri},alt={ghṛṇā°}]{ghṛṇā}
            \rdg[wit={N2}]{ghṛnā°}}lasyaṃ 
          \app{\lem[wit={ceteri}, alt={bhraṃtir daṃbho}]{bhrantir\skp{-}daṃbho}
            \rdg[wit={D}]{bhrāṃtir debho}
            \rdg[wit={E}]{bhrāntitvaṃ}
            \rdg[wit={U1}]{bhrāṃti daṃbha°}}
          \app{\lem[wit={ceteri}]{kṣamā bhramaḥ}
            \rdg[wit={E}]{mokṣam ābhramaḥ}
            \rdg[wit={U1}]{kṣamī bhramaḥ}}/}\\
%-----------------------
%  \om                                               \B
%  yasyai tāni na vidyaṃte kriyāyogī sa ucyate //    \L
%  yasyai tāni ca vidyante kriyāyogī sa ucyate 3     \E
%  yasyai tāni na vidyaṃte kriyāyogī sa ucyate       \P
%  yasyai tāni na vidyaṃte kriyāyogī sa ucyate //    \N1
%  yasyai tāni na vidyaṃte kriyāyogī sa ucyate //    \N2
%  yasyai tāni na vidyaṃte kriyāyogī sa ucyate //    \D
%  yasyai tāni na vidyaṃte kriyāyogī sa ucyate       \U1
%  yasyai tāni na vidyaṃte kriyāyogī sa ucyate //4// \U2 
%  -----------------------
% Whoever doesn't experience these is called a \textit{kriyāyogī}. 
%  -----------------------        
\tl{yasyai tāni \app{\lem[wit={ceteri}]{na}\rdg[wit={E}]{ca}} vidyante kriyāyogī sa ucyate\dd{}\begin{otherlanguage}{english}\coro{\uproman{2}.\textsuperscript{\lowroman{5}}}\end{otherlanguage}\hskip-2pt\dd{}}
      \end{tlg}
      \bigskip
      \begin{prose}
%-----------------------
%  \om                                                                                          \B
%  yasyāntaḥkaraṇe kṣamāvivekavairāgyaśāntisantoṣādīny                         utpadyante //     \E
%  yasyāṃtaḥkaraṇe kṣamāvivekavairāgyaśāṃtisaṃtoṣa         ityādīny            utpādyaṃte        \P
%  tasyāṃtaḥkaraṇe kṣamāvivekavairāgyaśāṃtisaṃtoṣa         ityādīnotpādyaṃte                    \L
%  yasyāṃtaḥkaraṇe kṣamāḥ vivekavairāgya /    śāṃtisaṃtoṣa ityādīni            utpādyaṃte        \N1
%  yasyāṃtaḥkaraṇe kṣamā' vivekavairāgyā      śāṃtisaṃtoṣa ityādīni            utpādyaṃte /      \N2 %see Mss p3 recto vierte Zeile von unten  
%  yasyāṃtaḥkaraṇe kṣamā // vivekavairāgya // śāṃtisaṃtoṣa ityādīni            utpādyaṃte //     \D
%  yasyāṃtaḥkaraṇe kṣamāvivekavairāgyaśāṃtisaṃtoṣa         ityādīna niraṃtaram utyaṃte        \U1
%  yasyāṃtaḥkaraṇe kṣamāvivekavairāgyaśāṃtisaṃtoṣa         ityādayo niraṃtaraṃ utpādyaṃte       \U2
%  -----------------------
%  Patience, discrimination, equanimity, peace, contentment etc. are generated in his mind.
%  -----------------------        
        yasyāntaḥkaraṇe
        \app{\lem[wit={ceteri},alt={kṣamā°}]{kṣamā}
          \rdg[wit={N1}]{kṣamāḥ}
          \rdg[wit={N2}]{kṣamā '}
        }\app{\lem[wit={ceteri}]{vivekavairāgyaśānti}
          \rdg[wit={N1}]{kṣamāḥ vivekavairāgya | śāṃti°}
          \rdg[wit={N2}]{°vairāgyāśānti°}
          \rdg[wit={D}]{kṣamā || vivekavairāgya || śāṃti°}
        }\app{\lem[wit={ceteri}, alt={°santoṣa ityādīny}]{santoṣa ityādī\skp{ny-u}} %the°-problem
          \rdg[wit={E}]{°santoṣādīny}
          \rdg[wit={L}]{°santoṣa ity ādīno°}
          \rdg[wit={U1}]{°santoṣa ity ādīna niraṃtaram}
          \rdg[wit={U2}]{°santoṣa ity ādayo niraṃtaraṃ}
        }\app{\lem[wit={ceteri},alt={utpādyante}]{\skm{ny-u}tpādyante}
          \rdg[wit={E}]{utpadyante}
          \rdg[wit={L}]{°tpādyaṃte}
          \rdg[wit={U1}]{utyaṃte}}/
%-----------------------
% \om \oxford
%  sa eva bahukriyāyogī kathyate /      \E
%  sa eva bahukriyāyogī kathyate        \P
%  sa eva bahukriyāyogī kathyate //     \L
%  sa eva bahukriyāyogī kathyate /      \N1
%  sa eva bahukriyāyogī sa kathyate /   \N2
%  sa eva bahukriyāyogā sa kathyate //  \D
%  sa eva bahukriyāyogī kathyate /      \U1
%  sa eva bahukriyāyogī tkacyate /      \U2
%-----------------------
% He alone is called a \textit{yogī} of many actions (\textit{bahukriyāyogī}).
%-----------------------
        sa eva
        \app{\lem[wit={ceteri}]{bahukriyāyogī}
          \rdg[wit={D}]{bahukriyāyogā}}
        \app{\lem[wit={ceteri}]{kathyate}
          \rdg[wit={D,N2}]{sa kathyate}
          \rdg[wit={U2}]{tkacyate}}/\\
%-----------------------
% \om \B
%               kāpaṭyaṃ      vittaṃ   hiṃsā    tṛṣṇā    mātsaryam    ahaṃkāraḥ    roṣaḥ kṣayaṃ    lajjā lobhamohā      aśucitvaṃ                       pākhaṃḍatvaṃ       bhrāntiḥ indriyavikāraḥ kāmaḥ          ete yasya manasi pratidinaṃ vyunā bhavanti /    \E
%               kāpaṭyaṃ      vittaṃ   hiṃsā    tṛṣṇā    mātsaryaṃ    ahaṃkāraḥ    roṣo bhayaṃ     lajjā lobhaḥ mohaḥ   aśucitvaṃ rāgaḥ dveṣaḥ   ālasyaṃ pākhaṃḍitvaṃ       bhrāṃtiḥ indriyaṃ vikāraḥ kāmaḥ        ete yasya manasi pratidinaṃ nyunā bhavanti     \P
%               kāpayaṃ     //vitaṃ // hiṃsā // tṛṣṇā // mātsaryaṃ // ahaṃkāraḥ // roṣo bhayaṃ //  lajjā lobhaḥ // moha aśucitvaṃ // rājadveṣa  alasyaṃ // pākhaṃḍitvaṃ // bhrāṃtiḥ // itivikāraḥ // kāmaḥ        eta yasya manasi pratidinaṃ nyunā bhavaṃti//    \L
%yasyāṃtakaraṇe kapatyaṃ māyā vitvaṃ   hiṃsā    tṛṣṇā    mātsaryaṃ    ahaṃkāraḥ    roṣo bhayaṃ     lajjā // lobhamohā   asucitvaṃ rāgadveṣaḥ // alasyaṃ pāṣaṃḍitvaṃ      bhraṃtiḥ / iṃdriyaivikāraḥ / kāmaḥ       ete yasya manasi pratidinaṃ nyunā bhavaīti/     \N1
%               kāpaṭyaṃ māyā vitvaṃ   hiṃsā    tṛṣṇā    mātsaryaṃ    ahaṃkāraḥ    e?ṣo bhayaṃ     lajjā/ lobhamoha     asūcitvaṃ rāgadveṣaḥ    ālasyaṃ pārṣaḍitvaṃ        bhrāṃtiḥ iṃdriyavikāraḥ // kāma         ete yasya manasi pratidinaṃ nyunā bhavaṃti //  \N2      
%               kāpaṭyaṃ māya vitvaṃ   hiṃsā    tṛṣṇā    mātsarya     ahaṃkāraḥ    roṣo bhayaṃ     lajjā // lobhamohā   asucitvaṃ rāgadveṣaḥ // ālasyaṃ pāṣaṃḍitvaṃ        bhraṃtiḥ // iṃdriyavikāraḥ // kāmaḥ // ete yasya manasi pratidinaṃ nyunā bhavaṃti //   \D
%               kāpachaṃ yāya vitvaṃ   hiṃsā    tṛṣṇā    mātsarya     ahaṃkāraḥ    roṣaḥ bhayaṃ    lajā     lobhamohā   aśucitvaṃ rāgadveṣaḥ    ālasyaṃ pākhaṃḍitvaṃ       bhraṃtiḥ iṃdriyavīkāraḥ    kāmaḥ       rāte yasya manasi pratidinaṃ nyunā bhavaṃti //  \U1
%               kāpaṭyaṃ pāpā titaṃ    hiṃsā    tṛṣṇā    mātsaryaṃ // ahaṃkāraḥ    roṣo bhayaṃ     lajjā ----mohā       aśucitvaṃ rāgadveṣaḥ    ālasyaṃ pākhaṃḍitvaṃ //    bhraṃtiḥ iṃdriyavikāraḥ //-----        etate yasya manasi pratidinaṃ nyunā bhavaṃti // \U2
%-----------------------
%Fraud, illusion, property, violence, craving, envy, ego, anger, anxiety, shame, greed, error, impurity, attachment, aversion, idleness, heterodoxy, false view, affection of the senses, sexual desire: He who diminishes these from day to day in is mind,
%-----------------------              
\note[type=testium, labelb=12, lem={\textbf{Ci}}]{\textit{Yogasaṃgraha} IGNCA 30020 folio 1r. ll. 1-2: lobhamohau aśucitvaṃ rāgadveṣau ālasyaṃ pāṣaṃḍitvaṃ bhrāṃtiḥ iṃdryiavikāraḥ kāmaḥ ete yasya pratidinaṃ nyunā bhavaṃti}
      \app{\lem[wit={ceteri}]{kāpaṭyaṃ}
        \rdg[wit={L}]{kāpayaṃ}
        \rdg[wit={N1}]{yasyāntaḥkaraṇe kapatyaṃ}
        \rdg[wit={U1}]{kāpachaṃ}}\dd{}
      \app{\lem[wit={N1,N2}]{māyā}
        \rdg[wit={D}]{māya}
        \rdg[wit={U1}]{yāya}
        \rdg[wit={U2}]{pāpa}
        \rdg[wit={E,P,L}]{\om}}\dd{}
        %\rdg[wit={E,P,L}]{\textbf{omitted in}}}
      \app{\lem[wit={E,P}]{vittaṃ}
        \rdg[wit={L}]{vitaṃ}
        \rdg[wit={N1,N2,D,U1}]{vitvaṃ}
        \rdg[wit={U2}]{titaṃ}}\dd{}
      hiṃsā\dd{}
      tṛṣṇā\dd{}
      \app{\lem[wit={ceteri}]{mātsaryaṃ}
        \rdg[wit={E}]{mātsaryam}
        \rdg[wit={D,U1}]{mātsarya}}\dd{}
      ahaṃkāraḥ\dd{}
      \app{\lem[wit={E,U1}]{roṣaḥ}
        \rdg[wit={ceteri}]{roṣo}
        \rdg[wit={N2}]{eṣo}}\dd{}
      \app{\lem[wit={ceteri}]{bhayaṃ}
        \rdg[wit={E}]{kṣayaṃ}}\dd{}
      \app{\lem[wit={ceteri}]{lajjā}
        \rdg[wit={U1}]{lajā}}\dd{}
      \app{\lem[wit={P,L}]{lobhaḥ}
        \rdg[wit={ceteri}]{lobha°}
        \rdg[wit={U2}]{\om}}\dd{}
      \app{\lem[wit={P}]{mohaḥ}
        \rdg[wit={L,N2}]{moha}
        \rdg[wit={ceteri}]{mohā}}\dd{}        
      \app{\lem[wit={ceteri}]{aśucitvaṃ}  %%%Frage: vor daṇḍa wird m zu ṃ??? 
        \rdg[wit={N2}]{aśūcitvaṃ}}\dd{}
      \app{\lem[wit={P}]{rāgaḥ}
        \rdg[wit={ceteri}]{rāga°}
        \rdg[wit={L}]{rāja°}
        \rdg[wit={E}]{\om}}\dd{}
      \app{\lem[wit={ceteri}]{dveṣaḥ}
        \rdg[wit={L}]{dveṣa}
        \rdg[wit={E}]{\om}}\dd{}
      \app{\lem[wit={ceteri}]{ālasyaṃ}
        \rdg[wit={E}]{\om}}\dd{}
      \app{\lem[wit={ceteri}]{pākhaṃḍitvaṃ}
        \rdg[wit={D,N1}]{pāṣaṃḍitvaṃ}
        \rdg[wit={E}]{pākhaṃḍatvaṃ}
        \rdg[wit={N2}]{pārṣaḍitvaṃ}}\dd{}
     bhrāntiḥ\dd{}
     \app{\lem[wit={ceteri}]{indriyavikāraḥ}
        \rdg[wit={P}]{iṃdriyaṃ vīkāraḥ}
        \rdg[wit={L}]{itivikāraḥ}}\dd{}
      \app{\lem[wit={ceteri}]{kāmaḥ}
        \rdg[wit={N2}]{kāma}
        \rdg[wit={U2}]{\om}}\dd{}
      \app{\lem[wit={ceteri}]{ete}
        \rdg[wit={L}]{eta}
        \rdg[wit={U1}]{rāte}
        \rdg[wit={U2}]{etate}}
      yasya manasi pradidinaṃ nyūna
      \app{\lem[wit={ceteri}]{bhavanti}
        \rdg[wit={N1}]{bhavaīti}}/
%-----------------------       
%sa eva bahukriyāyogī kathyate// \E
%sa eva bahukriyāyogī kathyate// \P
%sa eva bahukriyāyogī kathyate// \L
%sa eva bahukriyāyogī kathyate// \N1
%sa eva bahukriyāyogī kathyate// \N2
%sa eva bahukiyāyogī  kathyate//  \D
%sa eva bahukiyāyogī  kathyaṃte// \U1
%sa eva bahukiyāyogī  kathyaṃte// \U2
%-----------------------
%he alone is called a yogī of many actions (\textit{bahukriyāyogī})
%-----------------------
\note[type=testium, labelb=13, lem={\textbf{Cie}}]{\textit{Yogasaṃgraha} IGNCA 30020 folio 1r. l. 2: sa eva kriyāyogī kathyate ||}
%\note[type=philcomm, labelb=14, lem={bahukriyāyogī}]{The term \textit{bahukriyāyogī} currently seems to be unique in Sanskrit literature. The elaborations of Rāmacandra on Kriyāyoga after the quotes from the YSV are either taken from an unknown source or his own creation.}
sa eva \app{\lem[wit={ceteri}]{bahukriyāyogī}
e  \rdg[wit={D,U1,U2}]{bahukiyāyogī}}
      \app{\lem[wit={ceteri}]{kathyate}
        \rdg[wit={U1,U2}]{kathyaṃte}}\dd{}\textsuperscript{\begin{otherlanguage}{english}\coro{[\lowroman{10}]}\end{otherlanguage}}
    \end{prose}
    \ekdpb*{}
\end{edition}
\begin{translation}
\ekddiv{type=trans}
\begin{tlate}
\paragraph{\uproman{2}\textsuperscript{\lowroman{3}}.} Patience, discrimination, equanimity, peace, modesty, desireless:\textsuperscript{\coro{[\lowroman{5}]}}the one who is endowed with these means is said to be a Kriyāyogī. 
\paragraph{\uproman{2}\textsuperscript{\lowroman{4}}.} Envy, selfishness, cheating, violence, desire and intoxication, pride, lust, anger, fear, laziness, greed, error and impurity. 
\paragraph{\uproman{2}\textsuperscript{\lowroman{5}}.} Attachment and aversion, indignation and idleness, impatience and dizzyness: Whoever does not experience these is called a Kriyāyogī.
\\\\
Patience, discrimination, equanimity, peace, contentment etc., are generated in his mind. He alone is called a Yogī of many actions (\textit{bahukriyāyogī})\footnote{The term \textit{bahukriyāyogī} is only found in the \textit{Yogatattvabindu}. It seems to be a neologism of Rāmacandra since the \textit{Yogasvarodaya} and \textit{Yogasaṃgraha} only use the word \textit{kriyāyogī} in its passage on Kriyāyoga to denote its practitioner.}. Fraud, illusion, property, violence, craving, envy, ego, anger, anxiety, shame, greed, error, impurity, attachment, aversion, idleness, heterodoxy, false view, affection of the senses, sexual desire: He who diminishes these from day to day in his mind, he alone is called a Yogī of many actions (\textit{bahukriyāyogī}).\textsuperscript{\coro{[\lowroman{10}]}}\footnote{The most notable mention of the term \textit{kriyāyoga} appears in \textit{Pātañjalayogaśāstra} or \textit{Yogasūtra} 2.1 where is is defined as  
\begin{quote}
tapaḥsvādhyāyeśvarapraṇidhānāni kriyāyogaḥ || 2.1 || (\citeauthor[1983:113]{yogasutra})
\end{quote}
According to the introduction of this \textit{sūtra} in the \textit{Vyāsabhāṣya}, Kriyāyoga is introduced as a means how someone with a distracted mind can also attain Yoga (\textit{vyutthitacitto 'pi yogayuktaḥ}). Yoga, which for Patañjali is \textit{samādhi}, shall be achieved by the three elements of Kriyāyoga, namely mental, moral and physical austerity (\textit{tapas}), repetition of \textit{mantra}s or study of sacred literature (\textit{svadhyāya}) and surrender to god (\textit{īśvarapraṇidhāna}). This trinity of means is supposed to destroy the impurities (\textit{kleśa}s) of \textit{citta}. These are given in \textit{Pātanjalayogaśāstra} 2.3 as ignorance (\textit{avidyā}), egoism (\textit{asmitā}), attachment (\textit{rāga}), aversion (\textit{dveṣa}) and fear of death (\textit{abhiniveśa}), see (\citeauthor[1983:116]{yogasutra}). All three terms of Patañjali's Kriyāyoga are absent in the \textit{Yogatattvabindu}. Nevertheless, the individual elements of the \textit{kleśa}s, along with the aim to reduce these in the yogi's mind, can also be found in the \textit{Yogatattvabindu}. Nārāyaṇatīrtha in this commentary on the \textit{Pātanjalayogaśāstra} titled \textit{Yogasiddhāntacandrikā}, who, like Rāmacandra uses a very similar list of 15 Yogas (possible source for Rāmacandras 15 Yogas), presents Kriyāyoga as the first item of his list and explains its purpose as the generation of \textit{samādhi} and the reduction of \textit{kleśas}, see (\citeauthor[2000:71]{yogacandrika}), whereas the Kriyāyoga of Rāmacandra is said to lead to Rājayoga, which he conceptualizes as bringing about the steadiness of the body.}
\end{tlate}
 \ekdpb*{}
\end{translation}
\end{alignment}
%%%%%%%%%%%%%%%%%%%%%%%%%%%%%%%%%%%%%%%%%%
%%%%%%%%%%%%%%%%%%%%%%%%%%%%%%%%%%%%%%%%%%
%%%%%%%%PAGEBREAK%%%%%%%PAGEBREAK%%%%%%%%%
%%%%%%%%%%%%%%%%%%%%%%%%%%%%%%%%%%%%%%%%%%
%%%%%%%%%%%%%%%%PAGEBREAK%%%%%%%%%%%%%%%%%
%%%%%%%%%%%%%%%%%%%%%%%%%%%%%%%%%%%%%%%%%%
%%%%%%%%PAGEBREAK%%%%%%%PAGEBREAK%%%%%%%%%
%%%%%%%%%%%%%%%%%%%%%%%%%%%%%%%%%%%%%%%%%%
%%%%%%%%%%%%%%%%%%%%%%%%%%%%%%%%%%%%%%%%%%
%%%%%%%%%%%%%%%%%%%%%%%%%%%%%%%%%%%%%%%%%%
%%%%%%%%%%%%%%%%%%%%%%%%%%%%%%%%%%%%%%%%%%
%%%%%%%%PAGEBREAK%%%%%%%PAGEBREAK%%%%%%%%%
%%%%%%%%%%%%%%%%%%%%%%%%%%%%%%%%%%%%%%%%%%
%%%%%%%%%%%%%%%%PAGEBREAK%%%%%%%%%%%%%%%%%
%%%%%%%%%%%%%%%%%%%%%%%%%%%%%%%%%%%%%%%%%%
%%%%%%%%PAGEBREAK%%%%%%%PAGEBREAK%%%%%%%%%
%%%%%%%%%%%%%%%%%%%%%%%%%%%%%%%%%%%%%%%%%%
%%%%%%%%%%%%%%%%%%%%%%%%%%%%%%%%%%%%%%%%%%
%%%%%%%%%%%%%%%%%%%%%%%%%%%%%%%%%%%%%%%%%%
%%%%%%%%%%%%%%%%%%%%%%%%%%%%%%%%%%%%%%%%%%
%%%%%%%%PAGEBREAK%%%%%%%PAGEBREAK%%%%%%%%%
%%%%%%%%%%%%%%%%%%%%%%%%%%%%%%%%%%%%%%%%%%
%%%%%%%%%%%%%%%%PAGEBREAK%%%%%%%%%%%%%%%%%
%%%%%%%%%%%%%%%%%%%%%%%%%%%%%%%%%%%%%%%%%%
%%%%%%%%PAGEBREAK%%%%%%%PAGEBREAK%%%%%%%%%
%%%%%%%%%%%%%%%%%%%%%%%%%%%%%%%%%%%%%%%%%%
%%%%%%%%%%%%%%%%%%%%%%%%%%%%%%%%%%%%%%%%%%
\begin{alignment}[
    texts=edition[class="edition"];
    translation[class="translation"],
  ]
\begin{edition}
 \ekddiv{type=ed}
 \centerline{\textrm{\small{[\uproman{3}.\textsuperscript{\coro{\lowroman{1}-\lowroman{13}}} Siddhakuṇḍalinīyoga and Mantrayoga]}}}
 \label{siddhayoga}
    \bigskip
    \begin{prose}
%-----------------------   
% \om                                   \B
%idānīṃ rājayogasya bhedāḥ kathyante // \E
%idānīṃ rājayogasya bhedāḥ kathyaṃte    \P
%idānīṃ rājayogasya bhedāḥ              \L
%idānīṃ rājayogasya bhedāḥ kathyaṃte    \N1
%idānīṃ rājayogasya bhedā  kathyate//    \N2
%idānīṃ rājayogasya bhedāḥ kathyaṃte // \D
% \om                                   \U1
%idānīṃ rājayogasya bhedāḥ kathyaṃte // \U2
%-----------------------
%Now varieties of \textit{rājayoga} will be described.
%-----------------------
      \noindent idānīṃ rājayogasya
      \note[type=testium, labelb=15, lem={\textbf{Ci}}]{\textit{Yogasaṃgraha} IGNCA 30020 folio 1r. ll. 2-3:  atha rājayogasya bhedau kathyete ||}
       \app{\lem[wit={ceteri}]{bhedāḥ}
         \rdg[wit={N2}]{bhedā}}
       \app{\lem[wit={ceteri}]{kathyante}
         \rdg[wit={N2}]{kathyate}
         \rdg[wit={L}]{\om}}/
       \note[type=philcomm, labelb=16, lem={idānīṃ \ldots kathyante}]{The indroductory sentence is \om in U\textsubscript{1}.}     
%-----------------------
%te ke     \E
%te ke     \P
%te ke     \L
%ke te //  \D
%ke te /   \N1
%kriyate// \N2       
%ke te     \U1
%te ke     \U2
%-----------------------
%Which are these?
%-----------------------       
\app{\lem[wit={D,N1,U1}]{ke te}
         \rdg[wit={ceteri}]{te ke}
         \rdg[wit={N2}]{kriyate}}/ 
%-----------------------
%\om                                       \B
%ekaḥ siddhakuṇḍalinīyogaḥ / mantrayogaḥ / \E
%ekaḥ siddhakuṃḍaṃliṃ yogaḥ maṃtrayogaḥ    \P
%ekaḥ siddhakuṇḍalanīyoga /                \L 
%ekaḥ siddhakuṇḍalinīyogaḥ maṃtrayogaḥ /   \N1
%ekaḥ siddhakuṇḍalanīyogaḥ maṃtrayogaḥ //  \N2
%ekaḥ siddhakuṃḍalanīyogaḥ mantrayogaḥ //  \D
%ekaḥ siddhakuṇḍaliniyogaḥ mantrayogaḥ     \U1
%ekaḥ siddhakuṇḍalinīyoga // mantrayogaḥ   \U2
%-----------------------
%One is \textit{siddhakuṇḍalinīyoga} [and one] is \textit{mantrayoga}.       
%-----------------------
\note[type=testium, labelb=17, lem={\textbf{Ci}}]{\textit{Yogasaṃgraha} IGNCA 30020 folio 1r. l. 3: siddhakuṃḍaliyogaḥ mantrayogaś ceti |}
ekaḥ
\app{\lem[wit={E,N1}]{siddhakuṇḍalinīyogaḥ}
   \rdg[wit={L}]{siddhakuṇḍalanīyoga |}
   \rdg[wit={N2,D}]{siddhakuṃḍalanīyogaḥ}
   \rdg[wit={P}]{siddhakuṃḍaṃliṃ yogaḥ}
   \rdg[wit={U1}]{siddhakuṇḍalinīyogaḥ}
   \rdg[wit={U2}]{siddhakuṇḍalinīyoga ||}}
 \app{\lem[wit={ceteri}]{mantrayogaḥ}
   \rdg[wit={L}]{\om}}
       \note[type=source, labelb=19, lem={\textbf{Re}}]{PT\textsuperscript{ccn \cdot YSV} (Ed. p. 831): jñānayogaṃ pravakṣyāmi tajjñānī śivatāṃ vrajet | paṭhanāt smaraṇād vyānān maṇḍanāt brahmasādhakaḥ | tad bhedasyaikasandhānam aṣṭaiśvaryamayo bhavet | tritīrthaṃ yatra nāḍī ca tripuṇyaṃ parameśvari | \ldots eṣo 'sya viśvarūpasya rājayogo mato budhaiḥ | viśeṣaṃ kathayiṣyāmi śṛṇu caikamanāḥ sati |}
%-----------------------
% \om                         \B
%astu rājayogaḥ kathyate/    \E
%amū  rājayogau kathyete       \P
%amū  rājayogau kathyate//    \L
%amū  rājayogau kathyate       \N1
%amū  rājayogau kathyate//     \N2  %%%p3verso
%amū  rājayogau kathyate//    \D
%amū  rājayogau kathyate       \U1
%amū  rājayogau kathyaṃte//   \U2
%-----------------------
%These two rājayogas are described [in the following].
%-----------------------
       \app{\lem[wit={ceteri}]{amū}
         \rdg[wit={E}]{astu}}
       \app{\lem[wit={ceteri}]{rājayogau}
         \rdg[wit={E}]{rājayogaḥ}}\\
       \app{\lem[wit={P}]{kathyete}
         \rdg[wit={ceteri}]{kathyate}
         \rdg[wit={U2}]{kathyaṃte}}/
%-----------------------
% \om                                                              \B
%mūlakandasthāne    ekā tejorūpā    mahānāḍī varttate /            \E
%mūlaṃ kaṃdasthāne  ekā tejorūpā    mahānāḍī varttate              \P
%mūlakaṃdasthāne    ekā tejorūpā    mahānāḍī vartate               \L
%mūlakaṃdasthāne    eka tejorūpā    mahānāḍī varttate /            \N1
%mūlakaṃdasthāne    eka tejorūpā    mahānāḍī varttate /            \N2
%mūlakaṃdasthāne    ekā tejorūpā    mahānāḍī varttate //           \D
%mūlakaṃdasthāne    ekā tejorūpā    mahānāḍī vartate /             \U1
%mūlakaṃdasthāne // ekā tejorūpā // mahānāḍī pravarttate /         \U2
%-----------------------
%At the location of the root-bulb exists one major vessel in the form of energy.
%-----------------------       
\note[type=testium, labelb=20, lem={\textbf{Ci}}]{\textit{Yogasaṃgraha} IGNCA 30020 folio 1r. ll. 3-4: mūlakandasthāne ekā tejomayā mahānāḍī vartate |}
       \app{\lem[wit={ceteri}]{mūlakandasthāne}
         \rdg[wit={U2}]{mūlakaṃdasthāne ||}
         \rdg[wit={P}]{mūlaṃ kaṃdasthāne}}
       \app{\lem[wit={ceteri}]{ekā}
         \rdg[wit={N1,N2}]{eka}}
       \app{\lem[wit={ceteri}]{tejorūpā}
         \rdg[wit={U2}]{tejorūpā ||}}
       mahānāḍī
       \app{\lem[wit={ceteri}]{vartate}
         \rdg[wit={U2}]{pravartate}}/\textsuperscript{\begin{otherlanguage}{english}\coro{[\lowroman{5}]}\end{otherlanguage}}
       \note[type=source, labelb=21, lem={\textbf{Re}}]{PT\textsuperscript{ccn \cdot YSV} (Ed. p. 831-832): mūlakande sthale caikā nāḍī tejasvatī parā (\textit{tejasvitāparā} YK\textsuperscript{ccn \cdot YSV} 1.246 Ed. p. 20) |}
%-----------------------
% \om                                                            \B
%iyam ekanāḍī /  iḍāpiṃgalāsuṣumṇā      etān bhedān prāpnoti /    \E
%iyaṃ ekanāḍī    iḍāpiṃgalāsuṣumṇā      etān bhedān prāpnoti      \P
%trayaṃ kā nāḍī  iḍāpiṃgalāsuṣumnā //   etān bhedān prāpnoti      \L
%iyaṃ ekā nāḍī   iḍāpiṃgalāsuṣumnān /   ete  bhedān prāpnoti      \N1
%iyaṃ ekā nāḍī   iḍāpiṃgalāsuṣumnān//   ete  bhedān prāpnoti/     \N2
%iyaṃ ekā nāḍī   iḍāpiṃgalasuṣumnān //  ete  bhedān prāpnoti      \D    
%iyaṃ ekā nāḍī   iḍāpiṃgalāsuṣumnā      etān bhedān prāpnoti      \U1
%iyaṃ eka nāḍī   iḍāpiṃgalāsuṣumṇā      etān bhegān prāpnoti      \U2
%-----------------------
%This single vessel reaches to these openings which are \textit{iḍā}, \textit{piṅgalā} and \textit{suṣumnā}.
%-----------------------       
\note[type=testium, labelb=23, lem={\textbf{Ci}}]{\textit{Yogasaṃgraha} IGNCA 30020 folio 1r. l. 4: iyaṃ iḍāpiṃgalasuṣumnā bhedā tridhā |}
\app{\lem[wit={E},alt={iyam}]{iya\skm{m-e}}
         \rdg[wit={ceteri}]{iyaṃ}
         \rdg[wit={L}]{trayaṃ}
}\app{\lem[wit={ceteri}, alt={ekā}]{\skp{m-e}kā}
         \rdg[wit={E}]{eka |}
         \rdg[wit={P}]{eka}
         \rdg[wit={L}]{kā}}
       nāḍī iḍāpiṅgalā\app{\lem[wit={N1,N2,D},alt={°suṣumṇān}]{suṣumṇān}
         \rdg[wit={E,P,U2}]{°suṣumṇā}
         \rdg[wit={L,U1}]{°suṣumnā}}\dd{}
       \app{\lem[wit={Y,U1}]{etān}
         \rdg[wit={N1,N2,D}]{ete}} bhedān prāpnoti/
%-----------------------
%\om                                           \B
%vāmabhāge candrarūpā iḍā nāḍī varttate /      \E
%vāmabhāge caṃdrarūpā iḍā nāḍī varttate        \P
%vāmabhāge caṃdrarūpā iḍā nāḍī varttate //     \L
%vāmabhāge caṃdrarūpā iḍā nāḍī varttate /      \N1
%vāmabhāge caṃdrarūpā iḍā nāḍī varttate //     \N2
%vāmabhāge caṃdrarūpā iḍā nāḍī varttate /      \D
%vāmabhāge caṃdrarūpā iḍā nāḍī vartate         \U1
%vāmabhāge caṃdrarūpā     nāḍī pravarttate //  \U2
%-----------------------
%On the left side is the \textit{iḍā}-channel, being a resemblence of the moon.
%-----------------------        
\note[type=testium, labelb=24, lem={\textbf{Ci}}]{\textit{Yogasaṃgraha} IGNCA 30020 folio 1r. l. 4: vāmabhāge caṃdrarūpā iḍā |}
vāmabhāge candrarūpā
        \app{\lem[wit={ceteri}]{iḍā}
          \rdg[wit={U2}]{\om}} nāḍī
        \app{\lem[wit={ceteri}]{vartate}
          \rdg[wit={U2}]{pravarttate}}/
          \note[type=source, labelb=24x, lem={\textbf{Re}}]{PT\textsuperscript{ccn \cdot YSV} (Ed. p. 832): gudorddhe (\textit{gudordhve} YK\textsuperscript{ccn \cdot YSV} 1.247 Ed. p. 20) sā tribhāgābhūdiḍā (\textit{tridhā bhūyādiḍāvāme} YK\textsuperscript{ccn \cdot YSV} 1.247 Ed. p. 20) nāma śaśiprabhā | śaktirūpā mahānāḍī dhyānāt sarvārthadāyinī | dakṣiṇe 'pi kulākhyeti (\textit{piṅgalākhyeti} YK\textsuperscript{ccn \cdot YSV} 1.248 Ed. p. 20) puṃrūpā sūryavigrahā |}
%-----------------------
% \om                                                \B
%dakṣiṇabhāge  sūryarūpā piṅgalā  nāḍī    varttate /  \E
%dakṣiṇabhāge  sūryarūpā piṃgalā  nāḍī    varttate    \P
%dakṣiṇabhāge  sūryarūpā piṃgalā  nāḍī    varttate // \L
%dakṣiṇabhāge  sūryarūpā piṃgalā  nāḍī    varttate // \N1
%dakṣiṇabhāge  sūryarūpā piṃgalā  nāḍī    varttate/   \N2
%dakṣiṇabhāge  sūryarūpā piṃgalā  nāḍī    varttate // \D       
%dakṣiṇe bhāge sūryarūpā piṃgalā  nāḍī    vartate     \U1
%dakṣiṇabhāge  sūryarūpā piṃgalā  nāḍī pravartate //  \U2
%-----------------------
%On the right side exists the \textit{piṅgalā}-channel, being a resemblence of the sun.        
%-----------------------
\note[type=testium, labelb=25, lem={\textbf{Ci}}]{\textit{Yogasaṃgraha} IGNCA 30020 folio 1r. l. 4: dakṣiṇabhāge sūryarūpā piṃgalā |}
        \app{\lem[wit={ceteri}]{dakṣiṇabhāge}
          \rdg[wit={U1}]{dakṣiṇe bhāge}}
        sūryarūpā piṅgalā nāḍī
        \app{\lem[wit={ceteri}]{vartate}
          \rdg[wit={U2}]{pravarttate}}/
%-----------------------
% \om                                                                   \B
%madhyamārge `tisūkṣmā padminī taṃtusamākārā  koṭividyutsamaprabhā      \E
%madhyamārge `tisūkṣmā padmanī taṃtusamākāra  koṭividyutsamaprabhā      \P
%madhyamārge `tisūkṣmā padmanī taṃtusamākārā  koṭividyutsamaprabhā      \L
%madhyamārge atisūkṣmā padmanī taṃtusamākārā  koṭividyutsamaprabhā //   \N1
%madhyamārge atisūkṣmā padmanī taṃtusamākārā  koṭividyutsamaprabhā //   \N2
%madhyarge   atisūkṣmā padminī taṃtusamākārā  koṭividyutsamaprabhā //   \D
%madhyamārge atisūkṣmā padminī taṃtusamākārā  koṭividyutsamaprabaḥ      \U1
%madhyamārge  tisūkṣmā padminī taṃtusamākārā  koṭividyutsamaprabhā //   \U2
%-----------------------
%Within the middle path is a lotuspond being very subtle. [It is] made from a web of light [and it] shines like a thousand lightnings.
%----------------------- 
\note[type=testium, labelb=26, lem={\textbf{Ci}}]{\textit{Yogasaṃgraha} IGNCA 30020 folio 1r. l. 5: madhyamārge atisūkṣmā visataṃtusamākārā koṭividyutprabhā}
\note[type=source, labelb=26x, lem={\textbf{Re}}]{PT\textsuperscript{ccn \cdot YSV} (Ed. p. 832): madhyabhāge suṣumnākhyā brahmaviṣṇuśivātmikā | śuddhacittena sā vijñā vidyutkoṭisamaprabhā | bhuktimuktipradā dhyānād aṇimādiguṇapradā|}
        \app{\lem[wit={ceteri}]{madhyamārge}
          \rdg[wit={D}]{madhyarge}}
        \app{\lem[wit={Y}]{'tisūkṣmā}
          \rdg[wit={X}]{atisūkṣmā}}
        \app{\lem[wit={ceteri}]{padminī}
          \rdg[wit={L,P,N1,N2}]{padmanī}}/\textsuperscript{\begin{otherlanguage}{english}\coro{[\lowroman{10}]}\end{otherlanguage}}
        \app{\lem[wit={ceteri}]{tantusamākārā}
          \rdg[wit={P}]{taṃtusamākāra°}}
      koṭividyutsama\app{\lem[wit={ceteri},alt={°prabhā}]{prabhā}
        \rdg[wit={U1}]{°prabhaḥ}}/
      \note[type=testium, labelb=27, lem={\textbf{Re}}]{\textit{Siddhasiddāntapaddhati} 2.26 (Ed. p. 38): mūlakandād daṇḍalagnāṃ brahmanāḍīṃ śvetavarṇāṃ brahmarandhraparyantaṃ gatāṃ saṃsmaret | tanmadhye kamalatantunibhāṃ vidyutkoṭiprabhām ūrdhvagāminīṃ tāṃ mūrtiṃ manasā lakṣayet | sarvasiddhipradā bhavati |}
%-----------------------
%\om                                                                                                                                                                 \B
%bhuktimuktipradā                                     'syā jñānotpattau satyaṃ puruṣaḥ sarvajño  bhavati   \E
%bhuktimuktidā                                        asyā jñānotpattau satyāṃ puruṣaḥ sarvajño  bhavati   \P
%bhuktimuktipradā //                                  asyā jñānotpattau satyāṃ puruṣaḥ sarvajño  bhavati   \L
%bhuktimukti--------------------------------------------------dotpanne  sati---puruṣaḥ sarrvajño bhavati   \N1
%bhuktimukti--------------------------------------------------dotpanne  sati---puruṣaḥ sarrvajño bhavati   \N2
%bhuktimukti--------------------------------------------------dotpanne  sati---puruṣaḥ sarrvajño bhavati   \D1 
%bhuktimukti--------------------------------------------------dotpanne  sati---puruṣaḥ sarrvajño bhavati   \U1
%bhuktimuktidā śivarūpiṇī suṣumṇā nāḍī pravarttate // asyā jñānotpattau satyāṃ puruṣa--sarvajño  bhavati   \U2
%-----------------------
%She \extra{emerges as the central channel, assuming the form of benevolence (\textit{śiva}),} is the bestower of enjoyment and liberation. While abiding in (\textit{satyāṃ}) her (\textit{asyāṃ}) knowledge arises [to the point of which]%the person becomes all-knowing.
%-----------------------
\note[type=testium, labelb=27a, lem={\textbf{Ci}}]{\textit{Yogasaṃgraha} IGNCA 30020 folio 1r. ll. 5-6: bhuktimuktipradā suṣumnā nāḍī vartate | yasyāḥ jñāne purusaḥ sarvajño bhavati |}
  \app{\lem[wit={P,U2}]{bhuktimuktidā}
  \rdg[wit={X}]{bhuktimuktido°}
  \rdg[wit={E,L}]{bhuktimuktipradā}}
   %\rdg[wit={U2}]{bhuktimuktidā śivarūpiṇī suṣumṇā nāḍī pravarttate}} %Lesart oder einfach zusätzliches Material? 
   %\textcolor{red}{śivarūpiṇī suṣumṇā nāḍī pravarttate/}
\extra{\app{\lem[wit={U2}]{śivarūpiṇī suṣumṇā nāḍī pravarttate}
    \rdg[wit={ceteri}]{\om}}/}
\note[type=philcomm, labelb=27b, lem={śivarūpiṇī}]{Sentences unlikely to be authorial, but enriching, are included within the edition and marked in another colour.}
\app{\lem[resp=egoscr, type=emendation]{'syāṃ}
      \rdg[wit={E}]{'syā}
      \rdg[wit={P,L,U2}]{asyā}
      \rdg[wit={X}]{\om}}
    \app{\lem[wit={Y}]{jñānotpattau}
      \rdg[wit={X}]{°tpanne}}
    \app{\lem[wit={P,L,U2}]{satyāṃ}
      \rdg[wit={E}]{satyaṃ}
      \rdg[wit={X}]{sati}}
    sarvajño bhavati\dd{}    
    \end{prose}
    % \vspace*{\fill}
\ekdpb*{}
\end{edition}
\begin{translation}
  \ekddiv{type=trans}
   \bigskip
   \centerline{\textrm{\small{[\uproman{3}.\textsuperscript{\coro{\lowroman{1}-\lowroman{13}}} Siddhakuṇḍalinīyoga and Mantrayoga]}}}
   \label{siddhayogatrans}
    \bigskip
    \begin{tlate}
      \noindent Now varieties of Rājayoga will be described. \\\\
      \indent Which are these? One is Siddhakuṇḍalinīyoga\footnote{It is surprising to note the use of the term \textit{siddhakuṇḍalinīyoga} instead of \textit{siddhayoga} as listed initially. Furthermore, it is intriguing that this type of Yoga, which was listed as the second-last item in the Yoga taxonomy, is introduced as the second type right after Kriyāyoga, which was the first item in both the initial list and in the subsequent material. This raises further questions as the term \textit{kuṇḍaliṇī} is not mentioned at all in the subsequent description of this type of Yoga, adding to the mysterious nature of this terminology.} [and one\footnote{The distinction between "Siddhakuṇḍalinīyoga" and "Mantrayoga" is not entirely clear. The witness U\textsubscript{2} provides some description of Mantrayoga, but this raises questions about the relationship between the two. Based on the information from U\textsubscript{2} alone, it could be translated as "Siddhakuṇḍalinīyoga being Mantrayoga." However, given the lack of information from the other witnesses, the contents of this passage remain unclear.}] is Mantrayoga\footnote{The sudden appearance of the term \textit{mantrayoga} in this section is peculiar as the subsequent section does not mention the practice of mantras at all. This discrepancy may be the result of an early scribe's mistake that was subsequently copied by many of the manuscripts. All witnesses except L (L omits the term \textit{mantraygoa}) preserve this reading, and the following sentence supports the reading of \textit{mantrayoga} through the use of dual forms. The structure and content of \textit{Yogatattvabindu} closely follow \textit{Yogasvarodaya}, as quoted in \textit{Prāṇatoṣiṇī} and \textit{Yogakarṇikā}. However, the Yoga introduced in \textit{Yogasvarodaya} at this point is \textit{jñānayoga}, which is taken up by \textit{Yogatattvabindu}. It is also possible that, in the early transmission of the text, folios were lost or became confused, leading to a diffuse arrangement of the five types of Lakṣyayoga and missing Yogas. This issue cannot currently be resolved. Only the additional passages of witness U\textsubscript{2}, highlighted in blue, indirectly allude to a practice of mantra. U\textsubscript{2} prescribes the \textit{japājapa} of \textit{so 'haṃ} during meditation for almost each \textit{cakra}.}. These two Rājayogas are described [in the following]. \\\\
      \indent At the location of the root-bulb exists one major vessel in the form of energy.\textsuperscript{\coro{[\lowroman{5}]}} This single vessel reaches to these openings which are \textit{iḍā}, \textit{piṅgalā} and \textit{suṣumnā}. On the left side is the \textit{iḍā}-channel, being a resemblence of the moon. On the right side exists the \textit{piṅgalā}-channel, being a resemblence of the sun. Within the middle path is a lotuspond being very subtle.\textsuperscript{\coro{[\lowroman{10}]}} [It is] made from a web of light [and it] shines like a thousand lightnings.\\\\
\indent She \extra{emerges as the central channel assuming the form of benevolence (\textit{śiva}), [and]} is the bestower of enjoyment and liberation. While abiding in (\textit{satyāṃ}) her (\textit{asyāṃ}) knowledge arises. The person becomes all-knowing. \vspace*{\fill}
    \end{tlate}
    \ekdpb*{}
   \end{translation}
 \end{alignment}
%%%%%%%%%%%%%%%%%%%%%%%%%%%%%%%%%%%%%%%%%%
%%%%%%%%%%%%%%%%%%%%%%%%%%%%%%%%%%%%%%%%%%
%%%%%%%%PAGEBREAK%%%%%%%PAGEBREAK%%%%%%%%%
%%%%%%%%%%%%%%%%%%%%%%%%%%%%%%%%%%%%%%%%%%
%%%%%%%%%%%%%%%%PAGEBREAK%%%%%%%%%%%%%%%%%
%%%%%%%%%%%%%%%%%%%%%%%%%%%%%%%%%%%%%%%%%%
%%%%%%%%PAGEBREAK%%%%%%%PAGEBREAK%%%%%%%%%
%%%%%%%%%%%%%%%%%%%%%%%%%%%%%%%%%%%%%%%%%%
%%%%%%%%%%%%%%%%%%%%%%%%%%%%%%%%%%%%%%%%%%
%%%%%%%%%%%%%%%%%%%%%%%%%%%%%%%%%%%%%%%%%%
%%%%%%%%%%%%%%%%%%%%%%%%%%%%%%%%%%%%%%%%%%
%%%%%%%%PAGEBREAK%%%%%%%PAGEBREAK%%%%%%%%%
%%%%%%%%%%%%%%%%%%%%%%%%%%%%%%%%%%%%%%%%%%
%%%%%%%%%%%%%%%%PAGEBREAK%%%%%%%%%%%%%%%%%
%%%%%%%%%%%%%%%%%%%%%%%%%%%%%%%%%%%%%%%%%%
%%%%%%%%PAGEBREAK%%%%%%%PAGEBREAK%%%%%%%%%
%%%%%%%%%%%%%%%%%%%%%%%%%%%%%%%%%%%%%%%%%%
%%%%%%%%%%%%%%%%%%%%%%%%%%%%%%%%%%%%%%%%%%
%%%%%%%%%%%%%%%%%%%%%%%%%%%%%%%%%%%%%%%%%%
%%%%%%%%%%%%%%%%%%%%%%%%%%%%%%%%%%%%%%%%%%
%%%%%%%%PAGEBREAK%%%%%%%PAGEBREAK%%%%%%%%%
%%%%%%%%%%%%%%%%%%%%%%%%%%%%%%%%%%%%%%%%%%
%%%%%%%%%%%%%%%%PAGEBREAK%%%%%%%%%%%%%%%%%
%%%%%%%%%%%%%%%%%%%%%%%%%%%%%%%%%%%%%%%%%%
%%%%%%%%PAGEBREAK%%%%%%%PAGEBREAK%%%%%%%%%
%%%%%%%%%%%%%%%%%%%%%%%%%%%%%%%%%%%%%%%%%%
%%%%%%%%%%%%%%%%%%%%%%%%%%%%%%%%%%%%%%%%%%
 \begin{alignment}[
    texts=edition[class="edition"];
    translation[class="translation"],
  ]
\begin{edition}
 \ekddiv{type=ed}
 \centerline{\textrm{\small{[\uproman{4}.\textsuperscript{\coro{\lowroman{1}-\lowroman{10}}} First Cakra]}}}
 \label{cakra1}
    \bigskip
    \begin{prose}
%-----------------------
%\om                                                    \B
%idānīṃ suṣumṇāyāṃ jñānotpattāv---upāyāḥ  kathyante      \E
%idānīṃ suṣumṇāyā  jñānotpattau   upāyāḥ  kathyaṃte      \P
%idānīṃ suṣumnā    jñānotpattau   upāyaḥ  kathyate //    \L
%idānīṃ suṣumnāyāḥ jñanotpanno    'pāyāḥ  kathyaṃte //   \N1
%idānīṃ suṣumnāyāḥ jñanotpanno    upāyāḥ  kathyaṃte //   \N2
%idānīṃ suṣumnāyāḥ jñanotpattau   upāyāḥ  kathyaṃte //   \D
%idānīṃ suṣumnāya--jñanotpattau    upāyāḥ kathyaṃte //   \U1
%idānīṃ suṣumṇāyā  jñānotpattau   upāyā   kathyaṃte //   \U2
%-----------------------
\noindent 
      \note[type=testium, labelb=28, lem={\textbf{Ci}}]{\textit{Yogasaṃgraha} IGNCA 30020 folio 1r. l. 6: atas taj jñānotpattāv upāyā ucyaṃte |}
      \note[type=source, labelb=29, lem={\textbf{Re}}]{PT\textsuperscript{ccn \cdot YSV} (Ed. p. 832): suṣumnāntaḥ samāśritya navacakraṃ yathā śṛṇu | mūlādhāraṃ catuṣpatraṃ gudorddhe (\textit{gudordhve} YK\textsuperscript{ccn \cdot YSV} 1.250 Ed. p. 20) varttate mahat | tanmadhye svarṇapīṭhe tu trikoṇaṃ maṇḍalaṃ (\textit{trikoṇamaṇḍalaṃ} YK\textsuperscript{ccn \cdot YSV} 1.251 Ed. p. 20) param | tatra vahniśikhākārā mūrttiḥ sarvatra siddhidā | asyā dhyānaṃ manomadhye vinā pīṭhena (\textit{pāṭhena} YK\textsuperscript{ccn \cdot YSV} 1.252 Ed. p. 20) vāṅmayam | sarvaśāstrāṇi saṅkarṣaṃ (\textit{saṃkarṣa} YK\textsuperscript{ccn \cdot YSV} 1.252 Ed. p. 20) sadā sphurati yogavit |}
      \note[type=testium, labelb=29a, lem={\textbf{Ri}}]{SSP 2.1 (Ed. p. 29): piṇḍe navacakrāṇi | ādhāre brahmacakraṃ tridhāvartaṃ bhagamaṇḍalākāram | tatra mūlakandaḥ | tatra śaktiṃ pāvakākārāṃ dhyāyet | tatraiva kāmarūpapīṭhaṃ sarvakāmaphalapradaṃ bhavati ||2.1||}
idānīṃ  
    \app{\lem[wit={E}]{suṣumṇāyāṃ}
      \rdg[wit={P,U2}]{suṣumṇāyā}
      \rdg[wit={U1}]{suṣumnāya°}
      \rdg[wit={D,N1,N2}]{suṣumṇāyāḥ}
      \rdg[wit={L}]{suṣumnā°}}
    \app{\lem[wit={E}, alt={jñānotpattāv upāyāḥ}]{jñānotpattāv\skp{-}upāyāḥ}
      \rdg[wit={D,L,P,U1}]{jñānotpattau upāyāḥ}
      \rdg[wit={U2}]{jñānotpattau upāyā}
      \rdg[wit={N1}]{jñānotpanno 'pāyāḥ}
      \rdg[wit={N2}]{jñanotpanno upāyāḥ}}
    \app{\lem[wit={ceteri}]{kathyante}
      \rdg[wit={L}]{kathyate}}/
%-----------------------
%\om                                            \B
%ādau caturdalaṃ mūlaṃ cakraṃ varttate /        \E
%ādau caturddalaṃ mūlaṃ cakraṃ varttate /       \P
%ādau caturdalamūlacakraṃ varttate //           \L
%ādau caturdalaṃ mūlacakraṃ varttate            \N1
%ādau prathamacaturdalamūlacakraṃ pravarttate// \N2      
%ādau caturdalaṃ mūlacakraṃ varttate            \D
%ādau caturdalaṃ mūlaṃ cakraṃ vartate           \U1
%ādau caturdalaṃ mūlacakraṃ pravarttate //      \U2
%-----------------------
%At the beginning\footnote{Supposedly at the beginning of the central channel.} exists the root-cakra having four petals.     
%-----------------------      
\note[type=testium, labelb=30, lem=\textbf{Ci}]{\textit{Yogasaṃgraha} IGNCA 30020 folio 1r. l. 7: gudamūlacakraṃ caturdalaṃ |}
ādau
   \app{\lem[wit={D,N1,U2}]{caturdalaṃ mūlacakraṃ}
        \rdg[wit={E,P,U1}]{caturdalaṃ mūlaṃ cakraṃ}
        \rdg[wit={L}]{caturdalamūlacakraṃ}
        \rdg[wit={N2}]{prathamacaturdalamūlacakraṃ}}
      \app{\lem[wit={ceteri}]{vartate}
        \rdg[wit={U2}]{pravartate}}/
%-----------------------
%
%\om                                       \B
%prathamādhāracakraṃ varttate / gudāsthānaṃ    raktavarṇaṃ    gaṇeśadaivataṃ    siddhibuddhiśaktimuṣakavāhanam       kurmaṛṣiḥ /  ākuṃcamudrā /    apānavāyuḥ                                   caturdaleṣu     rajaḥsattvatamomanāṃsi /  vaṃ śaṃ ṣaṃ saṃ    madhyatrikoṇe triśikhāt    tanmadhye trikoṇākāraṃ kāmapīthaṃ varttate//    \E
%prathamaṃ ādhāracakraṃ         gudāsthānaṃ    raktavarṇaṃ    gaṇeśāṃ daivataṃ  siddhibuddhiśaktir mukhako vāhanam   kurmaṛṣiḥ    ākuṃcanamudrā    apānavāyuś-----------------------------------caturddaleṣu    rajaḥsattvatamomanāṃsi    vaṃ śaṃ ṣaṃ saṃ    madhyatrikoṇe triśikhā     tanmadhye trikoṇākāraṃ kāmapīthaṃ varttate //   \P
%prathamaṃ ādhāracakraṃ         gudāsthānaṃ    raktavarṇaṃ    gaṇeśadaivataṃ    siddhibuddhiśaktimuṣako vāhanaṃ //   kurmaṛṣiḥ    ākuṃcanamudrā    apānavāyuḥ                                   caturddaleṣu    rajaḥsattvatamomanāṃsi // vaṃ śaṃ ṣaṃ saṃ    madhyatrikoṇe triśikhā     tanmadhyatrikoṇākāraṃ kāmapīthaṃ vartate        \L
%---------------------------------------------------------------------------------------------------------------------------------------------------------------------------------------------------------------------------------------------------------------------------------------tanmadhyatrikoṇākāraṃ kāmapiṭhaṃ varttate /     \N1
%---------------------------------------------------------------------------------------------------------------------------------------------------------------------------------------------------------------------------------------------------------------------------------------tanmadhye trikoṇākāraṃ kāmapiṭhaṃ varttate /    \N2
%---------------------------------------------------------------------------------------------------------------------------------------------------------------------------------------------------------------------------------------------------------------------------------------tanmadhye trikoṇākāraṃ kāmapiṭhaṃ varttate /    \D
%---------------------------------------------------------------------------------------------------------------------------------------------------------------------------------------------------------------------------------------------------------------------------------------tanmadhye trikoṇākāraṃ kāmapiṭhaṃ varttate /    \U1
%prathamaṃ ādhāracakraṃ         gudāsthānaṃ // raktavarṇaṃ // gaṇeśadaivataṃ // siddhibuddhiśaktiḥ muṣako vāhanaṃ // kurmaṛṣiḥ // ākuṃcanamudrā // apānavāyu // urmīkalā // ojasvinīdhāraṇā // caturddaleṣu // rajaḥsattvatamomanāṃsi //  vaṃ śaṃ ṣaṃ saṃ // madhyatrikoṇe trirekhā //  tanmadhye trikoṇākāraṃ kāmapīthaṃ varttate //   \U2
%-----------------------
%The first cakra of support (\textit{ādhāra}) is at the anus [and] is red-colored. Gaṇeśa is the deity. He is success, intelligence and power. A rat is the mount. The Ṛṣi is Kūrma. The seal is contraction. The vitalwind is \textit{apāna}. The \textit{kalā} is the wave of consciousness (\textit{urmī}). The concentration is ``she who is powerful'' (\textit{ojasvinī})}. In the four petals [of it resides] \textit{rajas}, \textit{sattva}, \textit{tamas} and the mind-faculties (\textit{manāṃsi}), [symbolized by the syllables or \textit{bīja}s] vaṃ śaṃ ṣaṃ and saṃ. A trident is situated in the middle of the triangle\footnote{This passage is odd since a triagle wasn't mentioned before.}
%-----------------------
\note[type=testium, labelb=31, lem={\textbf{Ci}}]{\textit{Yogasaṃgraha} IGNCA 30020 folio 1r. l. 7: tanmadhye trikoṇākāraṃ kāmapiṭhaṃ |}
      \extra{
          \app{\lem[wit={P,L,U2}]{prathamaṃ ādhāracakraṃ}
            \rdg[wit={E}]{prathamādhāracakraṃ vartate |}}/
                 gudā sthānaṃ\dd{}
                 \app{\lem[type=emendation, resp=egoscr]{raktaṃ}
                   \rdg[wit={Y}]{rakta°}}varṇaṃ\dd{}
            \app{\lem[type=emendation, resp=egoscr]{gaṇeśaṃ daivataṃ}
                 \rdg[wit={E,L,U2}]{gaṇeśadaivataṃ}
                 \rdg[wit={P}]{gaṇeśāṃ daivataṃ}}\dd{}
            siddhibuddhi\app{\lem[type=emendation, resp=egoscr, alt={°śaktiṃ muṣako vāhanaṃ}]{śaktiṃ muṣako vāhanaṃ} %Emendation!!!
                 \rdg[wit={E}]{°śaktimuṣakavāhanam}
                 \rdg[wit={P}]{°śaktir mukhako vāhanam}
                 \rdg[wit={L}]{°śaktimuṣako vāhanaṃ}
                 \rdg[wit={U2}]{°śaktiḥ muṣako vāhanaṃ}}\dd{}
            \app{\lem[type=emendation, resp=egoscr]{kūrma} %%sandhi aḥ vor ṛ wird zu a + ṛ 
                 \rdg[wit={U2}]{kurma}}ṛṣiḥ\dd{}
            \app{\lem[type=emendation, resp=egoscr]{ākuñcanaṃ}
                 \rdg[wit={P,L,U2}]{ākuñcana°}
                 \rdg[wit={E}]{ākuṃca°}}mudrā\dd{}
            apāna\app{\lem[wit={E,L},alt={°vāyuḥ}]{vāyuḥ}
                 \rdg[wit={P}]{°vāyuś}
                 \rdg[wit={U2}]{°vāyu}}\dd{}
               \extra{
                 \app{\lem[type=emendation, resp=egoscr]{ūrmī}
                   \rdg[wit={U2}]{urmī}} kalā\dd{}
                 ojasvinī dhāraṇā\dd{}}
                 caturdaleṣu rajaḥsattvatamomanāṃsi\dd{}
                 vaṃ śaṃ ṣaṃ saṃ\dd{} madhyatrikoṇe
            \app{\lem[wit={P,L}]{triśikhā}
                 \rdg[wit={E}]{triśikhāt}
                 \rdg[wit={U2}]{trirekhā}}\dd{}}\textsuperscript{\begin{otherlanguage}{english}\coro{[\lowroman{5}]}\end{otherlanguage}}
        %%%%%%%%%%%%%%%%%
        %%%%%%%%%%%%%%%%%
        %%%%%%%%%%%%%%%%%
        %%%%%%%%%%%%%%%%%
        %%%%%%%%%%%%%%%%%          
            \app{\lem[wit={ceteri}]{tanmadhye}
                 \rdg[wit={L,N1}]{tanmadhya}}
               trikoṇākāraṃ kāmapiṭhaṃ vartate/
\note[type=philcomm, labelb=32, lem={prathamaṃ \ldots triśikhā}]{The whole section is missing in D, N\textsubscript{1}, N\textsubscript{2} and U\textsubscript{1}. Equally detailled passages for the other \textit{cakra}s which include assigments to various categories like \textit{daivata}, \textit{bīja}s etc. occur in U\textsubscript{2} only. Subsequently these passages were either lost in transmission in all other witnesses and were preserved in U\textsubscript{2} only or the extensive description of the first \textit{cakra} occurred randomly and the additions of U\textsubscript{2} are not authorial. As these passages are of interest for the history and usage of the text, they have been added to the edition and are presented in another colour to indicate their supplementary status.}
%-----------------------
%\om                                                      \B
%tatpīṭhamadhye 'gniśikhākāraikā    mūrtir varttate /        \E
%tatpīṭhamadhye magniśikhākārā ekā  mūrtir varttate /      \P
%tatpīṭhamadhye   jniśikhāka!rāṇakā mūrti varttate //     \L
%tatpīṭhamadhye  agniśikhākārā ekā  mūrttir varttate //    \N1
%tatpīṭhamadhye  agniśikhākārā ekā  mūrttir varttate /     \N2
%tatpīṭhamadhye  agniśikhākārā ekā  mūrttir varttate //    \D
%tatpīṭhamadhye  agniśikhākārā ekā  mūrttir varttate //    \U1
%tatpīṭhamadhye  agniśikhākārā ekā  mūrttir asmi      //    \U2
%-----------------------
%In the middle of this seat (\textit{pīṭha}) exists a single form having the shape of a flame.             
%-----------------------
\note[type=testium, labelb=33, lem={\textbf{Ci}}]{\textit{Yogasaṃgraha} IGNCA 30020 folio 1r. l. 7: tatpīṭhamadhye agniśikhākārā gaṇeśamūrttir varttate |}
tatpīṭhamadhye
\app{\lem[wit={E}]{'gniśikhākāraikā}
  \rdg[wit={X,U2}]{agniśikhākārā ekā}
  \rdg[wit={P}]{magniśikhākārā ekā}
  \rdg[wit={L}]{jñiśikhākarāṇakā}}
murti\skp{r-va}\app{\lem[wit={ceteri}, alt={vartate}]{\skm{r-va}rtate}
  \rdg[wit={U2}]{asmi}}/
%-----------------------%
%\om                                       \B
%tasyāḥ mūrtir  dhyānakāraṇāt sakalaśāstrakāvya-nāṭakādi-sakalavāṅmayaṃ vinābhyāsena puruṣasya manomadhye sphurati,     \E
%tasyā  mūrter  dhyānakaraṇāt sakalaśāstrakāvya-nāṭakādi-sakalavāṅmayaṃ vinābhyāsena puruṣasya manomadhye sphurati      \P
%tasyā  mūrtir  dhyānakāraṇāt sakalaśāstrakāvya-nāṭakādi //----vāṅmayaṃ vinābhyāsena puruṣasya manomadhye sphuraṃti!    \L
%tasyāḥ mūrter  dhyānakaraṇāt sakalaśāstrakāvya-nāṭakādi-sakalavāgmayaṃ vinābhyāsena puruṣasya manomadhye sphurati      \N1
%tasyā  mūrtter dhyānakaraṇāt sakalaśāstrakāvya-nāṭakādi-sakavāgmayaṃ   vinābhyāsena puruṣasya manomadhye sphurati//    \N2
%tasyāḥ mūrter  dhyānakaraṇāt sakalaśāstrakāvya-nāṭakādi-sakalavāgmayaṃ vinābhyāsena puruṣasya manomadhye sphurati      \D
%tasyā  mūrtair dhyānakaraṇāt sakalaśāstrakāvya-nāṭakādi-sakalavāgmayaṃ vinābhyāsena puruṣasya manomadhye sphurati      \U1
%tasyā          dhyānakaraṇāt sakalaśāstrakāvya-nāṭakādi-sakalavāṅmayaṃ vinābhyāsena puruṣasya manomadhye sphurati // asya bahir mānaṃdā // yogānaṃdā virānaṃdā // uparamānaṃdā // ajapājapa śat // 600 // ghaṭi 9 palāni 40 // \U2 %
%-----------------------
%Trough the practice of meditation on this form the whole literature, all \textit{śāstra}s, all poems, dramas etc., everything [related to] elocution, appears in the mind of the person without [prior] learning. \extra{[Assigned to it] is external bliss, yogic bliss, heroic bliss [and] the bliss of coming to rest.}
%-----------------------
\note[type=testium, labelb=34, lem={\textbf{Ci}}]{\textit{Yogasaṃgraha} IGNCA 30020 folio 1r. ll. 8-9: tasyā mūrter dhyānakaraṇāt sakalakāvyanāṭakādisakalavāṅmayaṃ vinābhyāsena puruṣasya manomadhye sphurati |}
\app{\lem[wit={ceteri}]{tasyā}
    \rdg[wit={E,N1,D}]{tasyāḥ}}
\app{\lem[wit={ceteri}, alt={mūrter}]{mūrte\skp{r-dhyā}}
    \rdg[wit={E,L}]{mūrtir}
    \rdg[wit={U1}]{mūrtair}
    \rdg[wit={U2}]{\om}
}\skm{r-dhyā}nakaraṇāt-śāstrakāvya\app{\lem[wit={ceteri}, alt={°nāṭakādi°}]{nāṭakādi}
    \rdg[wit={L}]{°nāṭakādi ||}}\app{\lem[wit={ceteri}, alt={°sakala°}]{sakala}
    \rdg[wit={L}]{\om}
    \rdg[wit={N2}]{°saka°}}\app{\lem[wit={E,P,L,U2},alt={°vāṅmayaṃ}]{vāṅmayaṃ}
    \rdg[wit={X}]{°vāgmayaṃ}} vinābhyāsena puruṣasya manomadhye
\app{\lem[wit={ceteri}]{sphurati}
  \rdg[wit={L}]{sphuraṃti}}/
      \extra{asya
        \app{\lem[type=emendation, resp=egoscr, alt={bahir ānandā}]{bahir\skp{-}ānandā}
          \rdg[wit={U2}]{bahir mānandā}}\dd{}
        yogānandā\dd{}
        \app{\lem[type=emendation, resp=egoscr]{vīrānandā}
          \rdg[wit={U2}]{virānandā}}\dd{}
        uparamānandā\dd{}
        ajapājapaśat\dd{} 600\dd{} ghaṭi 9 palāni 40\dd{}}\textsuperscript{\begin{otherlanguage}{english}\coro{[\lowroman{10}]}\end{otherlanguage}}\vspace*{\fill}
    \end{prose}
    \ekdpb*{}
   \end{edition}
\begin{translation}
  \ekddiv{type=trans}
  \centerline{\textrm{\small{[\uproman{4}.\textsuperscript{\coro{\lowroman{1}-\lowroman{10}}} First Cakra]}}}
  \label{cakra1trans}
    \smallskip
    \begin{tlate}
      \noindent The means for the genesis of knowledge in the central channel will now be described. At the beginning [of the central channel] exists the four-petalled Mūlacakra. \extra{The first \textit{cakra} of support (\textit{ādhāra}) is at the anus [and] is red-colored. Gaṇeśa is the deity - he is success, intelligence and power. The mount is a rat. Kūrma is the seer. Contraction is the seal. Apāna is the vitalwind. Ūrmi is the digit. Ojasvinī is the concentration. In the four petals [exists] \textit{rajas}, \textit{sattva}, \textit{tamas} and the mind-faculties, [as well as] \textit {vaṃ śaṃ ṣaṃ} and \textit{saṃ}. A trident is situated in the middle of the triangle.} In the middle is a trident, and \textit{kāmapīṭha}\footnote{This refers to one of the four \textit{pīṭha}s of tantric Buddhism and the Kaula Yoginī-Tantra named Kāmarūpa, specifically the present-day Kāmākhyā Temple in Assam, which is located in different parts of the yogic body in various yoga traditions. For an in-depth discussion of the term, see \citeauthor[2023: 48-58,129]{liersch2023}, \citeauthor[2020: \textit{et passim}]{rosati2020} and \citeauthor[2021: 119, footnote 144]{asiddhi}.} in the shape of a triangle.\textsuperscript{\coro{[\lowroman{5}]}} In the middle of this seat (\textit{pīṭha}) exists a single form in the shape of a flame. By meditating on this form the whole literature, all \textit{śāstra}s, all poems, dramas etc., everything [related to] elocution, appears in the mind of the person without learning. \extra{[Assigned to it are] external bliss, yogic bliss, heroic bliss [and] the bliss of coming to rest\footnote{Early accounts of "four blisses" can be found in descriptions of sexual yoga in some Vajrayāna works (cf. \citeauthor[2014: 99]{isaac2014} and \citeauthor[2000: 31-33]{sferra2000}). The earliest mention of these blisses is in the \citetitle{hevajra} (1.1.28 \textit{et passim}), which identifies them as \textit{ānanda}, \textit{paramānanda}, \textit{sahajānanda}, and \textit{viramānanda}. The final bliss, \textit{viramānanda}, is known as the "Bliss of Cessation" and refers to the feeling of pleasure experienced by the male partner during sexual ritual at the moment of ejaculation. The concept of the four blisses was later incorporated into the \textit{Amṛtasiddhi}, the earliest text to outline many of the fundamental principles and practices of \textit{haṭhayoga}. However, the \textit{Amṛtasiddhi} contrasts the principles of sexual ritual with the celibate yoga method of male ascetics, which rejected sexual intercourse altogether. The text states that semen (\textit{bindu}) is the source of "the Blisses whose last is Virama" (referring to the four blisses in Vajrayāna) in 7.4, and in 34.3, it asserts that the accomplished yogin delights in the three \textit{ānanda}s (likely \textit{ānanda}, \textit{paramānanda}, and \textit{sahajānanda}) without the bliss of ejaculation, reflecting the celibate yoga taught (cf. \citeauthor[2021: 17]{asiddhi}). In a complex process of adaptation, reconfiguration, and innovation, systems of four blisses were incorporated into texts of the late medieval period, such as the \textit{Yogatattvabindu}. The \textit{Amaraughaprabodha}, one of the earliest texts in the \textit{haṭhayoga} corpus, and other later texts that quote the \textit{Amṛtasiddhi}, modified or removed concepts unique to Buddhism, including technical terms from Vajrayāna sexual yoga (\citeauthor[2019: 21]{birch2019}). The \textit{Amanaska}, the earliest text on Rājayoga, also mentions various blisses such as \textit{ānanda}, \textit{paramānanda}, \textit{sahajānanda}, and \textit{cinmātrānanda} throughout the text (\citeauthor[2013: \textit{et passim}]{birch2013}).}. A hundredfold recitation of the non-recited 600; 9 \textit{ghaṭi}s [and] 40 \textit{palā}s.}\textsuperscript{\coro{[\lowroman{10}]}}\footnote{Instructions for the duration of the practice of meditation are in most of the additions of U\textsubscript{2} \ldots}    
          \end{tlate}
           \ekdpb*{}
        \end{translation}
 \end{alignment}
%%%%%%%%%%%%%%%%%%%%%%%%%%%%%%%%%%%%%%%%%%
%%%%%%%%%%%%%%%%%%%%%%%%%%%%%%%%%%%%%%%%%%
%%%%%%%%PAGEBREAK%%%%%%%PAGEBREAK%%%%%%%%%
%%%%%%%%%%%%%%%%%%%%%%%%%%%%%%%%%%%%%%%%%%
%%%%%%%%%%%%%%%%PAGEBREAK%%%%%%%%%%%%%%%%%
%%%%%%%%%%%%%%%%%%%%%%%%%%%%%%%%%%%%%%%%%%
%%%%%%%%PAGEBREAK%%%%%%%PAGEBREAK%%%%%%%%%
%%%%%%%%%%%%%%%%%%%%%%%%%%%%%%%%%%%%%%%%%%
%%%%%%%%%%%%%%%%%%%%%%%%%%%%%%%%%%%%%%%%%%
%%%%%%%%%%%%%%%%%%%%%%%%%%%%%%%%%%%%%%%%%%
%%%%%%%%%%%%%%%%%%%%%%%%%%%%%%%%%%%%%%%%%%
%%%%%%%%PAGEBREAK%%%%%%%PAGEBREAK%%%%%%%%%
%%%%%%%%%%%%%%%%%%%%%%%%%%%%%%%%%%%%%%%%%%
%%%%%%%%%%%%%%%%PAGEBREAK%%%%%%%%%%%%%%%%%
%%%%%%%%%%%%%%%%%%%%%%%%%%%%%%%%%%%%%%%%%%
%%%%%%%%PAGEBREAK%%%%%%%PAGEBREAK%%%%%%%%%
%%%%%%%%%%%%%%%%%%%%%%%%%%%%%%%%%%%%%%%%%%
%%%%%%%%%%%%%%%%%%%%%%%%%%%%%%%%%%%%%%%%%%
%%%%%%%%%%%%%%%%%%%%%%%%%%%%%%%%%%%%%%%%%%
%%%%%%%%%%%%%%%%%%%%%%%%%%%%%%%%%%%%%%%%%%
%%%%%%%%PAGEBREAK%%%%%%%PAGEBREAK%%%%%%%%%
%%%%%%%%%%%%%%%%%%%%%%%%%%%%%%%%%%%%%%%%%%
%%%%%%%%%%%%%%%%PAGEBREAK%%%%%%%%%%%%%%%%%
%%%%%%%%%%%%%%%%%%%%%%%%%%%%%%%%%%%%%%%%%%
%%%%%%%%PAGEBREAK%%%%%%%PAGEBREAK%%%%%%%%%
%%%%%%%%%%%%%%%%%%%%%%%%%%%%%%%%%%%%%%%%%%
%%%%%%%%%%%%%%%%%%%%%%%%%%%%%%%%%%%%%%%%%%
 \begin{alignment}[
    texts=edition[class="edition"];
    translation[class="translation"],
  ]
\begin{edition}
 \ekddiv{type=ed}
 \centerline{\textrm{\small{[\uproman{5}.\textsuperscript{\coro{\lowroman{1}-\lowroman{6}}}Second Cakra]}}}
 \label{cakra2}
    \bigskip
    \begin{prose}
%-----------------------
% \om                                       \oxford
%idānīṃ dvitīyaṃ svādhiṣṭānacakraṃ   ṣaḍdalaṃ upāyanapīṭhasaṃjñakaṃ bhavati //  \E
%idānīṃ dvitīyaṃ svādhiṣṭānacakraṃ   ṣaṭdalaṃ uḍḍīyānapīṭhaṃ saṃjñakaṃ bhavati  \P
%idānīṃ dvitīyaṃ svādhiṣṭānacakraṃ   ṣaṭdalaṃ uḍḍīyān pīṭhaṃ saṃjñakaṃ bhavati  \L
%idānīṃ dvitīyaṃ svādhiṣṭānacakraṃ   ṣaṭdalaṃ uḍyānapīṭhasaṃjñakaṃ bhavati /    \N1
%idānī  dvitīyaṃ svādhinacakraṃ      ṣaḍḍalaṃ uḍyānapīṭhasaṃjñakaṃ bhavati      \N2
%idānīṃ dvitīyaṃ svādhiṣṭānacakraṃ   ṣaṭdalaṃ uḍyāṇāpīṭhasaṃjñikaṃ bhavati //   \D
%idānīṃ dvitīyaṃ svādhiṣṭhānacakraṃ  ṣaṭdalaṃ uḍāganapīṭasaṃjñakaṃ bhavati      \U1
%idānīṃ dvitīye svādhiṣṭānacakraṃ // ṣaṭdalaṃ // uḍḍīyāṇapīṭhasaṃjñakaṃ bhavati // liṃgasthānaṃ // pītavarṇaṃ // pītaprabhā // rajoguṇa // brahmādevatā // vaikharīvāca // sāvitrīśaktiḥ // haṃsavāhanaṃ // vahaṇaṛṣiḥ // kāmāgniprabhā //sthūladehā // jāgradavasthā // ṛgveda // ācāryaliṃgaṃ // braṃhmasalokatāmokṣaḥ // śuddhabhumikātatvaṃ // gaṃdho viṣayaḥ // apānavāyuḥ // aṃtarmātṛkā // vaṃ bhaṃ maṃ yaṃ raṃ laṃ // bahir mātrā // kāmā // kāmākhyā // tejasī // ceṣṭṛikā // alasā // mithunā // ajapājapaḥ sahasra // 6000 //gha 0 96 pa 0 40// \U2
%-----------------------
%Now the second, the six-petalled \textit{Svādhiṣṭānacakra} known as the seat of \textit{uḍḍīyāṇa}\footnote{Discuss the term \textit{uḍḍīyāna}.}. \extra{The gender is the location. The color is yellow. The shine is yellow. \textit{Rajas} is the quality. The deity is Brahmā. The speech is \textit{vaikharī}\footnote{vaikharī f. in Kaśm. Śiv. °the 4. form of appearacne of \textit{parā}, the empirical speech sound, Utpala's Ṭīkā to Śivadṛṣṭi 2, 7. [B.]― Schmidt p. 337. Welches Buch???} (\textit{vaikharīvāca}). The power is Sāvitrī. The mount is the goose. The \textit{Rṣi} is Vahaṇa. The appearance (\textit{prabhā} is the fire of love (\textit{kāmāgni}). The body is gross, The state is that of being awake. [The Veda associated with it is] the Ṛgveda. The spiritual guide is the \textit{liṅga}. The liberation is residing in the world of Brahma. The level is the pure earth (\textit{śuddhabhumikā}). The sphere is smell. The vitalwind is \textit{apāna}. The internal alphabet [is]: vaṃ bhaṃ maṃ yaṃ raṃ laṃ. The outer alphabet?: desire, the Tīrtha of \textit{Kāmākhyā}\footnote{The Kāmākhyā is situated in Kāmarūpa on the Nīlakūṭa mountain in present day Assam. It's strange that it appears here, since Kāmarūpa appears already as the Tīrtha associated with the first \textit{cakra}.}, beauty of both\footnote{Why dual here?}, \textit{ceṣṭṛikā} (what is that?), lazy [and] copulation.}
%-----------------------      
\noindent
\note[type=testium, labelb=35, lem={\textbf{Ci}}]{\textit{Yogasaṃgraha} IGNCA 30020 folio 1r. l. 9: liṃgo dvitīyaṃ ṣaṭdalaṃ svādhiṣṭānasaṃjñakaṃ kamalaṃ udyānapīṭhasaṃjñakaṃ vartate ||}
\note[type=testium, labelb=35a, lem={\textbf{Ri}}]{SSP 2.2 (Ed. p. 28): dvitīyaṃ svādhiṣṭhānacakram | tanmadhye paścimābhimukhaṃ liṅgaṃ pravālāṅkurasadṛśaṃ dhyāyet | tatraivoḍyānapīṭhaṃ jagadākarṣaṇaṃ bhavati |}
\note[type=source, labelb=36, lem={\textbf{Re}}]{PT\textsuperscript{ccn \cdot YSV} (Ed. p. 832): liṅgamūle tu pīṭhābhaṃ (\textit{raktābhaṃ} YK\textsuperscript{ccn \cdot YSV} 1.253 Ed. p. 20) svādhiṣṭhānan tu ṣaḍdalam | tanmadhye bālasūryābhaṃ mahajjyotiḥ susiddhidam | dhyānāc ca varddhate āyuḥ kandarpasamatāṃ vrajet |}
\app{\lem[wit={ceteri}]{idānīṃ}
          \rdg[wit={N2}]{idānī}}
        \app{\lem[wit={ceteri}]{dvitīyaṃ}
            \rdg[wit={U2}]{dvitīye}}
        \app{\lem[wit={U1}]{svādhiṣṭhānacakraṃ}
            \rdg[wit={E,L,P,D,N1,U2}]{svādhiṣṭānacakraṃ}
            \rdg[wit={N2}]{svādhinacakraṃ}}
        \app{\lem[wit={ceteri}]{ṣaṭdalaṃ}
            \rdg[wit={E}]{ṣaḍdalaṃ}
            \rdg[wit={N2}]{ṣaḍḍalaṃ}}
        \app{\lem[wit={U2},alt={uḍḍīyāṇapīṭha°}]{uḍḍīyāṇapīṭha}
            \rdg[wit={E}]{upāyanapīṭha°}
            \rdg[wit={L}]{uḍḍīyān pīṭhaṃ}
            \rdg[wit={N1,N2}]{uḍyānapīṭha°}
            \rdg[wit={D}]{uḍyāṇāpīṭha°}
            \rdg[wit={U1}]{uḍāganapīṭa°}}saṃjñakaṃ
bhavati/         
      %%%%%%%%%%%%%%%%
      %%%%%%%%%%%%%%%
      %%%%%%%%%%%%%%%%
      %%%%%%%%%%%%%%%
      %%%%%%%%%%%%%%%    
      \extra{\app{\lem[type=emendation, resp=egoscr]{liṅgaṃ}
          \rdg[wit={U2}]{liṅga°}} sthānaṃ\dd{}
        \app{\lem[type=emendation, resp=egoscr]{pītaṃ}
          \rdg[wit={U2}]{pīta°}} varṇaṃ\dd{}
        \app{\lem[type=emendation, resp=egoscr]{pītā}
          \rdg[wit={U2}]{pīta°}} prabhā\dd{}
        rajo \app{\lem[type=emendation, resp=egoscr]{guṇaḥ}
          \rdg[wit={U2}]{guṇa}}\dd{}
        brahmā devatā\dd{}
        vaikharī \app{\lem[type=emendation, resp=egoscr]{vāk}
          \rdg[wit={U2}]{vāca}}\dd{}
        sāvitrī śaktiḥ\dd{}
        \app{\lem[type=emendation, resp=egoscr]{haṃso}
          \rdg[wit={U2}]{haṃsa°}} vāhanaṃ\dd{}
        \app{\lem[type=emendation, resp=egoscr]{vahaṇo}
          \rdg[wit={U2}]{vahaṇa}} ṛṣiḥ\dd{}
        \app{\lem[type=emendation, resp=egoscr, alt={kāmāgnir}]{kāmāgni\skp{r-pra}}
          \rdg[wit={U2}]{kāmāgni°}}\skm{r-pra}bhā\dd{}
        \app{\lem[type=emendation, resp=egoscr]{sthūlo dehaḥ}
          \rdg[wit={U2}]{sthūladehā}}\dd{}
        jāgrad-avasthā\dd{}
        \app{\lem[type=emendation, resp=egoscr]{ṛg vedaḥ}
          \rdg[wit={U2}]{ṛg veda}}\dd{}
        \app{\lem[type=emendation, resp=egoscr]{ācāryaḥ}
          \rdg[wit={U2}]{ācārya°}} liṅgaṃ\dd{}
        brahmasalokatā mokṣaḥ\dd{}
        \app{\lem[type=emendation, resp=egoscr]{śuddhabhumikā}
          \rdg[wit={U2}]{śuddhabhumikā}} tattvaṃ\dd{}
        gaṃdho viṣayaḥ\dd{}
        \app{\lem[type=emendation, resp=egoscr]{apānaḥ}
          \rdg[wit={U2}]{apāna°}} vāyuḥ\dd{}
        aṃtar\skp{-}mātṛkā\dd{}
        vaṃ bhaṃ maṃ yaṃ raṃ laṃ\dd{}
        bahir-mātrā\dd{}
        kāmā\dd{}
        kāmākhyā\dd{}
        \app{\lem[type=emendation, resp=egoscr]{tejasvinī}
          \rdg[wit={U2}]{tejasī}}\dd{}
        ceṣṭikā\dd{}
        alasā\dd{}
        mithunā\dd{}
        ajapājapaḥ \app{\lem[type=emendation, resp=egoscr]{sahasraḥ}
          \rdg[wit={U2}]{sahasra}}\dd{} 6000 \dd{} gha. 16 pa. 40\dd{}}\textsuperscript{\begin{otherlanguage}{english}\coro{[\lowroman{2}]}\end{otherlanguage}}
%-----------------------
%
% \om                                        \B
%tanmadhye atiraktavarṇaṃ tejo varttate /    \E
%tanmadhye 'tiraktavarṇaṃ tejo varttate      \P
%tanmadhye  tiraktavarṇaṃ tejo varttate //   \L
%tanmadhye  atiraktavarṇaṃ tejo varttate     \N1
%tanmadhye  atiraktavarṇatejo varttate      \N2
%tanmadhye  atiraktavarṇaṃ tejo varttate     \D
%tanmadhye  atiraktavarṇatejo varttate       \U1
%tanmadhye 'tiraktavarṇaṃ tejo vartate //    \U2
%-----------------------
%In its middle exists extremely red glow. The adept becomes very handsome by meditation on it.       
%-----------------------          
\note[type=testium, labelb=37, lem={\textbf{Ci}}]{\textit{Yogasaṃgraha} IGNCA 30020 folio 1r. ll. 9-10: tatra atiraktaṃ \sic{yahbhā} saṃjñakaṃ tejaḥ |}
tanmadhye         
        \app{\lem[wit={P,U2}]{'tiraktavarṇaṃ}
            \rdg[wit={ceteri}]{atiraktavarṇaṃ}
            \rdg[wit={U1,N2}]{atiraktavarṇa°}}
tejo vartate/
%-----------------------
% \om                                          \B
%tasya dhyānāt sādhako 'tisundaro bhavati /    \E
%tasya dhyānāt sādhako   tisuṃdaro bhavati      \P
%tasya dhyānāt sādhako   tisuṃdaro bhavati //   \L
%tasya dhyānāt sādhakaḥ  atisuṃdaro bhavati // \N1
%tasya dhyānāt sādhakaḥ  atisuṃdaro bhavati/   \N2
%tasya dhyānāt sādhakaḥ  atisuṃdaro bhavati // \D
%tasyā     nāt sādhakaḥ  atisuṃdarāṃgasan  // \D2
%tasya dhyānāt sādhakaḥ  atisuṃdaro bhavati    \U1
%tasya dhyānāt sādhako  'tisundaro bhavati //   \U2
%-----------------------
%The adept becomes very handsome through meditation on it.
%-----------------------       
\note[type=testium, labelb=38, lem={\textbf{Ci}}]{\textit{Yogasaṃgraha} IGNCA 30020 folio 1r. l. 10: tasyā nāt sādhakaḥ atisuṃdarāṃgasan}
tasya dhyānā\skp{t-sā}
\app{\lem[wit={E,P,L,U2},alt={sādhako}]{\skm{t-sā}dhako}
  \rdg[wit={ceteri}]{sādhakaḥ}}
\app{\lem[wit={Y}]{'tisundaro}
  \rdg[wit={X}]{atisuṃdaro}}
bhavati/ 
%-----------------------
% \om                                  \B
%                                pratidinam-āyur vardhate /             \E
%                                pratidinam-āyur vardhate               \P
%                                pratidinam-āyur vardhate //2//         \L
%                                dinaṃ dinaṃ prati āyurvarddhate // //  \N1
%yuvatīnāṃ ativallabho? bhavati dinadinaṃ prati āyur varddhate//        \N2  %%%3verso
%                                dinaṃ prati āyurvarddhate //2//        \D
%                                dinaṃ dinaṃ prati āyurvarddhate        \U1
%                                pratidinaṃ āyur varddhate //          \U2
%-----------------------
%\extra{He becomes one who is very desired by virgins.} The vital force increases from day to day. \end{tlate}
%-----------------------
\note[type=testium, labelb=39, lem={\textbf{Ci}}]{\textit{Yogasaṃgraha} IGNCA 30020 folio 1r. ll. 10-11: yuvatīnām ativallabhaḥ san pratidinam āyuṣyābhivṛddhimān bhavati | cha |} % \D2 %%%S.2 Z. 11}
\extra{\app{\lem[wit={N2}]{yuvatīnāṃ ativallabho bhavati}
  \rdg[wit={ceteri}]{\om}}/}\textsuperscript{\begin{otherlanguage}{english}\coro{[\lowroman{5}]}\end{otherlanguage}}\note[type=philcomm, labelb=40, lem={yuvatīnāṃ}]{This additional sentence occurs in N\textsubscript{2} and the \textit{Yogasaṃgraha} only.}
\app{\lem[wit={Y}, alt={pratidinam}]{pratidina\skp{m-ā}}
  \rdg[wit={N1,U1}]{dinaṃ dinaṃ prati}
  \rdg[wit={N2}]{dinadinaṃ prati}
  \rdg[wit={D}]{dinaṃ prati}
}\skm{m-ā}yur-vardhate\dd{}\vspace*{\fill}
\end{prose}
\end{edition}
\begin{translation}
  \ekddiv{type=trans}
  \centerline{\textrm{\small{[\uproman{5}.\textsuperscript{\coro{\lowroman{1}-\lowroman{6}}}Second Cakra]}}}
  \label{cakra2trans}
    \bigskip
    \begin{tlate}
      \noindent
      \blfootnote{\ldots for each \textit{cakra}, except the seventh \textit{cakra} at the palate and the ninth \textit{cakra} named \textit{mahāśūnyacakra}. 600 \textit{ajapājapa} refers to the duration of the voiceless uttering of the ``natural'' \textit{mantra} of the breath: \textit{so 'haṃ} (``he is I'') - \textit{haṃ sa} (``I am him''). The same duration of \textit{ajapājapa}s for meditation on \textit{cakra}s is also found in the \textit{Jogpradīpyakā} of Jayatarāma in verses 889-912. As in many other yoga texts the total amount of \textit{ajapājapa} per day is declared to be 21600. If 21600 \textit{ajapājapa} would equals 24 hours, then 600 \textit{ajapājapa} would equal ≈ 40 minutes. In the additions of U\textsubscript{2} one finds the same numbers of \textit{ajapājapa} as in the instructions for meditation onto the seven \textit{cakra}-system of Jayatarāma (cf. \citeauthor[2006: 163]{jogpradipyaka}). Ignoring this discrepancy, the scribe of U\textsubscript{2} applied this system of seven \textit{cakra}s to nine \textit{cakra}s of Rāmacandra. The following instruction of ``\textit{ghaṭi} 9 \textit{palāni} 40'' is not entirely clear. Usually one \textit{ghaṭi} equals 1/60 of a day (cf. \citeauthor[1966: 114]{sircar1966}), which is 24 minutes. One \textit{pala} equals 1/60 of a \textit{ghaṭi}, which is 24 seconds (cf. \citeauthor[1858: 4]{petersburger4}). This conception is explicitly stated in the earliest Rājayoga text, the \textit{Amanaska} in 1.35 (cf. \citeauthor[2013: 231]{birch2013}). For a more detailled tracing of the usage of the system in yogic and tantric literature see \citeauthor[2013: 265, endnote 46]{birch2013}. According to the above mentioned system, 9 \textit{ghaṭi}s and 40 \textit{pala}s would equal 232 minutes. Possibly ``\textit{ajapājapaśat || 600 || ghaṭi 9 palāni 40 ||}'' must then be understood cummulatively, which would equal 272 minutes for the duration of meditation onto the first \textit{cakra}. Other systems are less specific. \citetitle{kumbhaka} 208, i.e. states that ``Six winkings are one \textit{prāṇa}, six \textit{prāṇa}s make up one \textit{pala}. Sixty \textit{pala}s equal the time-period of a \textit{ghaṭikā}.'' (ṣaṇṇimeṣo bhavat prāṇaḥ ṣaḍbhiḥ prāṇaiḥ palaṃ smṛtaṃ | palaiḥ ṣaṣṭibhir-eva syād ghaṭikā-kāla-sammitā || 208 ||).} Now the second, the six-petalled Svādhiṣṭānacakra known as the seat of \textit{Uḍḍīyāṇa}\footnote{The term \textit{uḍḍīyāṇa} originally refers to one of the four \textit{pīṭha}s of tantric Buddhism and the Kaula Yoginī-Tantra, see \citeauthor[1996: 260]{white1996}. According to \citeauthor{urban2010} (2010) and \citeauthor{kowski1988} (1988), Uḍḍiyāna is \ldots} [is described]. \extra{The gender (\textit{liṅga}) is the location. The color is yellow. The shine is yellow. \textit{Rajas} is the quality. Brahmā is the deity. Vaikharī is the speech. Sāvitrī is the power. The mount is the goose. Vahaṇa is the seer. Kāmāgni is the appearance. The body is gross. Being awake is the state. Ṛg is the Veda. The penis (\textit{liṅga}) is the spiritual guide. The liberation is residing in the same world with the Brahman. The pure level (\textit{śuddhabhūmikā}) is the principle. The sphere is smell. Apāna is the vitalwind. The internal matrix [is]: \textit{vaṃ bhaṃ maṃ yaṃ raṃ laṃ}. The external matrix [is]: Kāmā, Kāmākhyā, Tejasvinī, Ceṣṭikā, Alasā [and] Mithunā. A thousandfold recitation of the non-recited; 6000; 16 \textit{ghaṭi}s [and] 40 \textit{palā}s.} In its middle exists extremely red glow. The adept becomes very handsome through meditation on it. \extra{He becomes one who is desired by young women.}\textsuperscript{\coro{[\lowroman{5}]}} The vital force increases from day to day.
\vspace*{\fill}
\end{tlate}
\end{translation}
\end{alignment}
\ekdpb*{}
%%%%%%%%%%%%%%%%%%%%%%%%%%%%%%%%%%%%%%%%%%
%%%%%%%%%%%%%%%%%%%%%%%%%%%%%%%%%%%%%%%%%%
%%%%%%%%PAGEBREAK%%%%%%%PAGEBREAK%%%%%%%%%
%%%%%%%%%%%%%%%%%%%%%%%%%%%%%%%%%%%%%%%%%%
%%%%%%%%%%%%%%%%PAGEBREAK%%%%%%%%%%%%%%%%%
%%%%%%%%%%%%%%%%%%%%%%%%%%%%%%%%%%%%%%%%%%
%%%%%%%%PAGEBREAK%%%%%%%PAGEBREAK%%%%%%%%%
%%%%%%%%%%%%%%%%%%%%%%%%%%%%%%%%%%%%%%%%%%
%%%%%%%%%%%%%%%%%%%%%%%%%%%%%%%%%%%%%%%%%%
%%%%%%%%%%%%%%%%%%%%%%%%%%%%%%%%%%%%%%%%%%
%%%%%%%%%%%%%%%%%%%%%%%%%%%%%%%%%%%%%%%%%%
%%%%%%%%PAGEBREAK%%%%%%%PAGEBREAK%%%%%%%%%
%%%%%%%%%%%%%%%%%%%%%%%%%%%%%%%%%%%%%%%%%%
%%%%%%%%%%%%%%%%PAGEBREAK%%%%%%%%%%%%%%%%%
%%%%%%%%%%%%%%%%%%%%%%%%%%%%%%%%%%%%%%%%%%
%%%%%%%%PAGEBREAK%%%%%%%PAGEBREAK%%%%%%%%%
%%%%%%%%%%%%%%%%%%%%%%%%%%%%%%%%%%%%%%%%%%
%%%%%%%%%%%%%%%%%%%%%%%%%%%%%%%%%%%%%%%%%%
%%%%%%%%%%%%%%%%%%%%%%%%%%%%%%%%%%%%%%%%%%
%%%%%%%%%%%%%%%%%%%%%%%%%%%%%%%%%%%%%%%%%%
%%%%%%%%PAGEBREAK%%%%%%%PAGEBREAK%%%%%%%%%
%%%%%%%%%%%%%%%%%%%%%%%%%%%%%%%%%%%%%%%%%%
%%%%%%%%%%%%%%%%PAGEBREAK%%%%%%%%%%%%%%%%%
%%%%%%%%%%%%%%%%%%%%%%%%%%%%%%%%%%%%%%%%%%
%%%%%%%%PAGEBREAK%%%%%%%PAGEBREAK%%%%%%%%%
%%%%%%%%%%%%%%%%%%%%%%%%%%%%%%%%%%%%%%%%%%
%%%%%%%%%%%%%%%%%%%%%%%%%%%%%%%%%%%%%%%%%%
\begin{alignment}[
    texts=edition[class="edition"];
    translation[class="translation"],
  ]
\begin{edition}
 \ekddiv{type=ed}
 \centerline{\textrm{\small{[\uproman{6}.\textsuperscript{\coro{\lowroman{1}-\lowroman{6}}} Third Cakra]}}}
 \label{cakra3}
    \bigskip
    \begin{prose}
      \noindent
      \note[type=source, labelb=40, lem={\textbf{Re}}]{PT\textsuperscript{ccn \cdot YSV} (Ed. p. 832): tṛtīyaṃ nābhideśe tu digdalaṃ paramādbhutam | mahāmeghaprabhaṃ tat tu koṭividyutsamanvitam | kalpāntāgnisamaṃ (\textit{kalpānto 'gni°} YK\textsuperscript{ccn \cdot YSV} 1.255 Ed. p. 20) jyotis tanmadhye saṃsthitaṃ svayam | tasya (\textit{asya} YK\textsuperscript{ccn \cdot YSV} 1.256 Ed. p. 21) dhyānāc cirāyuḥ syād arogo (\textit{arogī} YK\textsuperscript{ccn \cdot YSV} 1.256 Ed. p. 21) jagatāṃ varaḥ (\textit{jagatāmvaraḥ} YK\textsuperscript{ccn \cdot YSV} 1.256 Ed. p. 21) | sarvapāpavinirmukto jagatkṣobhakaro (\textit{jaganmokṣakaro} YK\textsuperscript{ccn \cdot YSV} 1.256 Ed. p. 21) mahān |}
      \note[type=testium, labelb=40b, lem={\textbf{Ri}}]{SSP 2.3 (Ed. p. 30): tṛtīyaṃ nābhicakraṃ pañcāvartaṃ sarpavat kuṇḍalākāram | tanmadhye kuṇḍalinīṃ śaktiṃ bālārkakoṭisannibhāṃ dhyāyet | sā madhyā śaktiḥ sarvasiddhidā bhavati |}
%-----------------------
% \om                                                 \B
%tṛtīye                      nābhisthāne    daśadalaṃ padmaṃ vartate      \E
%tṛtīyaṃ                     nābhisthāne    daśadalaṃ padmaṃ vartate      \P
%tṛtīyaṃ                     nābhisthāne // daśadalapadme vartate         \L
%tṛtīyaṃ                     nābhisthāne    daśadalaṃ padma varttate //   \N1
%tṛtīyacakraṃ                nābhisthāne    daśadalaṃ padma varttate /    \N2
%tṛtīyaṃ                     nābhisthāne    daśadalaṃ padma varttate //   \D
%tṛtīyaṃ                     nābhisthāne    daśadalakaṃ padmaṃ varttate   \U1
%atha tṛtīyaṃ maṇipūracakraṃ nābhisthāne // kapilavarṇaṃ // viṣṇudevatā // lakṣmīśaktiḥ // vāyuṛṣiḥ // samānavāyuḥ // garuḍavāhanaṃ // sūkṣmaliṃgadevatāha // svapnāvasthā // madhyamāvāk // yajurvedaḥ // dakṣināgniḥ // samipatāmokṣaḥ // guruliṃgaviṣṇuḥ // āpastatvaṃ // rajoviṣayaḥ daśadalāni // daśamātrāḥ // aṃtarmātrā // ḍaṃ ṭaṃ ṇaṃ taṃ thaṃ daṃ dhaṃ naṃ paṃ phaṃ // bahirmātrāḥ // śāṃtiḥ // kṣamā // medhā // tanyā // medhāvinī // puṣkarā // ahaṃsagamanā // lakṣyā //tanmayā // amṛtā // ajapājapa // 6000 gha 016 pa 040 //    \U2
%-----------------------
%\extra{The colour is red (\textit{kapila}). Viṣṇu is the deity. Lakṣmī is the power. Vāyu is the Rṣi. Samāna is the vitalwind. The mount is Garuḍa. The deity is the suble body\footnote{Why another deity is given here?}. The state is sleep. The speech is the inaudible speech (\textit{madhyamāvāg})\footnote{<Śā, Ling>name of the speech which is inaudible and which is of the type of a thought without any definite presence of words making up the expression. Vkp I.143.<Abhyankar 1986: 300>}. The Veda is the Yajurveda. The [fire is the] southern fire. The liberation is ``proximity'' (\textit{samīpatā}).\footnote{What is this exactly?}. Viṣṇu is the characteristic of the teacher (\textit{guruliṅga}). The principle is water. The sphere is athmosphere (\textit{rajo viṣaya}). There are ten petals [and] ten matrices. [The] inner matrix: \textit{ḍaṃ ṭaṃ ṇaṃ taṃ thaṃ daṃ dhaṃ naṃ paṃ phaṃ}. The external matrix : peace, patience, insight, the ``daughter''\textit{tanayā}, the ``learned teacher'', the ``lotus'', \textit{haṃsagamanā}, the ``fixation object'', absorption and immortality.} 
%-----------------------
\note[type=testium, labelb=41, lem={\textbf{Ci}}]{\textit{Yogasaṃgraha} IGNCA 30020 folio 1r. ll. 11: nābhistnāne daśadalaṃ cakraṃ |}
\app{\lem[wit={ceteri}]{tṛtīyaṃ}
      \rdg[wit={E}]{tṛtīye}
      \rdg[wit={U2}]{atha tṛtīyaṃ maṇipūracakraṃ}
      \rdg[wit={N2}]{tṛtīyacakraṃ}}
    nābhisthāne
    \app{\lem[wit={ceteri}]{daśadalaṃ}
      \rdg[wit={L}]{daśadala°}
      \rdg[wit={U1}]{daśadalakaṃ}
      \rdg[wit={U2}]{\om}}
    \app{\lem[wit={E,P,U1}]{padmaṃ}
      \rdg[wit={L}]{°padme}
      \rdg[wit={D,N1,N2}]{padma}
      \rdg[wit={U2}]{\om}}
    \app{\lem[wit={ceteri}]{vartate}
      \rdg[wit={U2}]{\om}}/
    \extra{\app{\lem[type=emendation, resp=egoscr]{kapilaṃ}
        \rdg[wit={U2}]{kapila°}} varṇaṃ\dd{}
      \app{\lem[type=emendation, resp=egoscr,alt={viṣṇur}]{viṣṇu\skp{r-de}}
        \rdg[wit={U2}]{viṣṇu}}\skm{r-de}vatā\dd{}
      lakṣmī śaktiḥ\dd{}
      \app{\lem[type=emendation, resp=egoscr, alt={vāyur}]{vāyu\skp{r-ṛ}}
        \rdg[wit={U2}]{vayu°}}\skm{r-ṛ}ṣiḥ\dd{}}
      \extra{\app{\lem[type=emendation, resp=egoscr]{samāno}
        \rdg[wit={U2}]{samāna°}} vāyuḥ\dd{}
      \app{\lem[type=emendation, resp=egoscr]{garuḍo}
        \rdg[wit={U2}]{garuḍa°}} vāhanaṃ\dd{}
      \app{\lem[type=emendation, resp=egoscr]{sūkṣmaliṅgaṃ devatā}
        \rdg[wit={U2}]{sūkṣmaliṅgadevatāha}}\dd{}
      svapnāvasthā\dd{}
      madhyamā vāk\dd{}
      yajur-vedaḥ\dd{}
      \app{\lem[type=emendation, resp=egoscr]{dakṣiṇo 'gniḥ}
        \rdg[wit={U2}]{dakṣināgniḥ}}\dd{}
      \app{\lem[type=emendation, resp=egoscr]{samīpatā}
        \rdg[wit={U2}]{samipatā}} mokṣaḥ\dd{}
      \app{\lem[type=emendation, resp=egoscr]{guruliṅgo}
        \rdg[wit={U2}]{guruliṅga°}} viṣṇuḥ\dd{}
      āpas-tattvaṃ\dd{}
      rajo viṣayaḥ\dd{}
      daśadalāni\dd{}
      daśamātrāḥ\dd{}
      antar-mātrā\dd{}
      ḍaṃ ṭaṃ ṇaṃ taṃ thaṃ daṃ dhaṃ naṃ paṃ phaṃ\dd{}
      bahir-mātrāḥ\dd{}
      śāṃtiḥ\dd{}
      kṣamā\dd{}
      medhā\dd{}
      tanayā\dd{}
      medhāvinī\dd{}
      puṣkarā\dd{}
      \app{\lem[type=emendation, resp=egoscr]{haṃsagamanā}
        \rdg[wit={U2}]{ahaṃsagamanā}}\dd{}
      lakṣyā\dd{}
      tanmayā\dd{}
      amṛtā\dd{}
      ajapājapaḥ \app{\lem[type=emendation, resp=egoscr]{sahasraḥ}
        \rdg[wit={U2}]{sahasra}}\dd{} 6000\dd{} gha. 16 pa. 40\dd{}}\textsuperscript{\begin{otherlanguage}{english}\coro{[\lowroman{2}]}\end{otherlanguage}}
%-----------------------
% \om                                       \B
%tanmadhye paṃcakoṇaṃ cakraṃ varttate//    \E
%tanmadhye paṃcakoṇaṃ cakraṃ varttate       \P
% \om  \L
%tanmadhye paṃcakoṇaṃ cakraṃ varttate//    \N1
%tanmadhye paṃcakoṇaṃ cakraṃ varttate/    \N2
%tanmadhye paṃcakoṇaṃ cakraṃ varttate//    \D
%tanmadhye paṃcakoṇaṃ cakraṃ varttate       \U1
%tanmadhye paṃcakoṇaṃ cakraṃ vartate//     \U2
%-----------------------
% In its middle exists a \textit{cakra} with five angles.
%-----------------------
\note[type=testium, labelb=61, lem={\textbf{Ci}}]{\textit{Yogasaṃgraha} IGNCA 30020 folio 1r. ll. 11 - 2v. ll. 1: tanmadhye paṃcakoṇaṃ pīṭhe lakṣmīnāparvatī saṃjñakaṃ \sic{guṇā} sahitā śiva saṃjñakā rāmaṇaṃ rūpā}
tanmadhye pancakoṇaṃ cakraṃ vartate/ \note[type=philcomm, labelb=62, lem={tanmadhye \ldots cakraṃ vartate}]{This sentence is \om in L.}
%-----------------------
% \om                                  \B
%tanmadhye ekā mūrtir vartate/         \E
%tanmadhye ekā mūrtir vartate          \P
%\om                                   \L
%tanmadhye ekā mūrttir varttate //     \N1
%tanmadhye ekā mūrttir varttate/       \N2
%tanmadhye ekā mūrttir varttate//      \D
%tanmadhye ekā mūrtir vartate          \U1
%tanmadhye ekā mūrtir asmi//           \U2
%-----------------------
%In its middle is a single (divine) form. 
%-----------------------
\app{\lem[wit={ceteri}]{tanmadhye}
  \rdg[wit={L}]{\om}}
\app{\lem[wit={ceteri}]{ekā}
  \rdg[wit={L}]{\om}}
\app{\lem[wit={ceteri}]{mūrti\skp{r-va}}
  \rdg[wit={L}]{\om}}\app{\lem[wit={ceteri}, alt={vartate}]{\skm{r-va}rtate}
  \rdg[wit={U2}]{asmi}}/
%-----------------------
% \om                                           \B
%tasyās tejo jihvayā kathayituṃ na śakyate /    \E
%tasyās tejo jihvayā kathayituṃ na śakyate      \P
%tasyās tejo jihvayā kathyituṃ  na śakyate       \L
%tasyā  tejo jihvayā kathayituṃ  na śakyate //    \N1
%tasyā  tejo jihvayā kathayituṃ  na śakyate/      \N2
%tasyā  tejo jihvayā kathayituṃ  na śakyate //    \D
%tasyās tejo jihvayā kathatuṃ   na śakyate        \U1
%tasyās tejo jihvayā vaktuṃ     na śakyate //       \U2
%-----------------------
%It's not possible to describe her shine with speech (lit. with the tongue).
%-----------------------
\note[type=testium, labelb=63, lem={\textbf{Ci}}]{\textit{Yogasaṃgraha} IGNCA 30020 folio 2v. ll.1: yasyās tejo jihvayā kathituṃ na śakyate}
\app{\lem[wit={Y,U1}, alt={tasyās}]{tasyā\skp{s-te}}
   \rdg[wit={D,N1,N2}]{tasyā}}\skm{s-te}jo jihvayā
 \app{\lem[wit={ceteri}]{kathayituṃ}
    \rdg[wit={L}]{kathyituṃ}
    \rdg[wit={U1}]{kathatuṃ}
    \rdg[wit={U2}]{vaktuṃ}}
  na śakyate/\textsuperscript{\begin{otherlanguage}{english}\coro{[\lowroman{5}]}\end{otherlanguage}}
%-----------------------
% \om                                                                    \B
%tasyāḥ mūrter dhyānakāraṇāt    puruṣasya śarīraṃ sthiraṃ bhavati //     \E
%tasyā  mūrter dhyānakaraṇāt    -------------------------------------    \P
%tasyā  mūrtir dhyānakaraṇāt // puruṣasya śarīraṃ sthiram bhavati //     \L
%tasyāḥ mūrter dhyānakaraṇāt    puruṣasya śarīraṃ sthiraṃ bhavati /      \N1
%tasyāḥ mūrter dhyānakaraṇāt    puruṣasya śarīraṃ sthiraṃ bhavati//      \N2
%tasyāḥ mūrter dhyānakaraṇāt    puruṣasya śarīraṃ sthiraṃ bhavati /      \D
%tasā          dhyānakaraṇāt    sādhakasya śarīraṃ sthiraṃ bhavati /cha/ \D2
%tasyāḥ mūrter dhyānakaraṇāt    puruṣasya śarīraṃ sthiraṃ bhavati vā     \U1
%tasyāḥ        dhyānakaraṇāt    puruṣasya śarīraṃ sthiraṃ bhavati //     \U2
%-----------------------
%Through the execution of meditation on this (divine) form the body of the person is going to be strong.   
%-----------------------
\note[type=testium, labelb=64, lem={\textbf{Ci}}]{\textit{Yogasaṃgraha} IGNCA 30020 folio 2v. ll. 1-2: tasā dhyānakaraṇāt sādhakasya śarīraṃ sthiraṃ bhavati |cha|}
  \app{\lem[wit={X,E,U2}]{tasyāḥ}
  \rdg[wit={P,L}]{tasyā}}
  \app{\lem[wit={ceteri}, alt={mūrter}]{mūrte\skp{r-dhyā}}
      \rdg[wit={L}]{mūrtir}
      \rdg[wit={U2}]{\om}}\skm{r-dhyā}na\app{\lem[wit={ceteri}, alt={°karaṇāt}]{karaṇāt}
      \rdg[wit={L}]{karaṇāt ||}
      \rdg[wit={E}]{°kāraṇāt}}
\app{\lem[wit={ceteri}]{puruṣasya}
  \rdg[wit={P}]{\om}}
\app{\lem[wit={ceteri}]{śarīraṃ}
  \rdg[wit={P}]{\om}}
\app{\lem[wit={ceteri}]{sthiraṃ}
  \rdg[wit={P}]{\om}}    
  \app{\lem[wit={ceteri}]{bhavati}
    \rdg[wit={U1}]{bhavati vā}
    \rdg[wit={P}]{\om}}\dd{}
\end{prose}
\end{edition}
\begin{translation}
  \ekddiv{type=trans}
  \centerline{\textrm{\small{[\uproman{6}.\textsuperscript{\coro{[\lowroman{1}-\lowroman{6}]}} Third Cakra]}}}
  \label{cakra3trans}
    \bigskip
    \begin{tlate}
      \noindent
\blfootnote{\ldots probably situated in the Swat valley in modern Pakistan. Throughout the text corpus of Haṭhayoga, the \textit{pīṭha}s are repeatedly located differently in the yogic body. Additionally the term refers to a certain yogic technique classified as \textit{mudrā}, see \citeauthor[2017: pp. 228-258]{rootsofyoga2017}. Depending on the text and tradition, there are different models.} The third, a ten petalled lotus exists at the location of the navel. \extra{The colour is red. Viṣṇu is the deity. Lakṣmī is the power. Vāyu is the seer. Samāna is the vitalwind. Garuḍa is the mount. The suble body is the deity\footnote{A second deity seems redundant here.}. Sleep is the state. Madhyamāvāg is the speech. Yajur[veda] is the Veda. The southern fire is the fire. Samīpatā is the liberation. Viṣṇu is the \textit{guruliṅga}\footnote{The phallus of Śiva, considered as one’s teacher or guide, cf. \textit{Śivapurāṇa} 1.18.31 \citetitle[1920]{shivapura} and \citeauthor[1950]{shastri1950}.}. Water is the principle. Rajoviṣaya is the sphere. There are ten petals [and] ten matrices. [The] inner matrix: \textit{ḍaṃ ṭaṃ ṇaṃ taṃ thaṃ daṃ dhaṃ naṃ paṃ phaṃ}. The external matrix: Śānti, Kṣamā, Medhā, Tanayā, Medhavinī, Puṣkarā, Haṃsagamanā, Lakṣyā, Tanmayā and Amṛtā. A thousandfold recitation of the non-recited; 6000; 16 \textit{ghaṭi}s [and] 40 \textit{palā}s.\footnote{The additions of U\textsubscript{2} for each \textit{cakra} are discussed on p. \pageref{discussionu2}.}} In its middle exists a \textit{cakra} with five angles. In its middle is a single [divine] form. It is not possible to describe her shine with speech.\textsuperscript{\coro{[\lowroman{5}]}} Through the execution of meditation on this [divine] form the body of the person becomes strong.
    \end{tlate}
   \end{translation}
 \end{alignment}
   \ekdpb*{}
%%%%%%%%%%%%%%%%%%%%%%%%%%%%%%%%%%%%%%%%%%
%%%%%%%%%%%%%%%%%%%%%%%%%%%%%%%%%%%%%%%%%%
%%%%%%%%PAGEBREAK%%%%%%%PAGEBREAK%%%%%%%%%
%%%%%%%%%%%%%%%%%%%%%%%%%%%%%%%%%%%%%%%%%%
%%%%%%%%%%%%%%%%PAGEBREAK%%%%%%%%%%%%%%%%%
%%%%%%%%%%%%%%%%%%%%%%%%%%%%%%%%%%%%%%%%%%
%%%%%%%%PAGEBREAK%%%%%%%PAGEBREAK%%%%%%%%%
%%%%%%%%%%%%%%%%%%%%%%%%%%%%%%%%%%%%%%%%%%
%%%%%%%%%%%%%%%%%%%%%%%%%%%%%%%%%%%%%%%%%%
%%%%%%%%%%%%%%%%%%%%%%%%%%%%%%%%%%%%%%%%%%
%%%%%%%%%%%%%%%%%%%%%%%%%%%%%%%%%%%%%%%%%%
%%%%%%%%PAGEBREAK%%%%%%%PAGEBREAK%%%%%%%%%
%%%%%%%%%%%%%%%%%%%%%%%%%%%%%%%%%%%%%%%%%%
%%%%%%%%%%%%%%%%PAGEBREAK%%%%%%%%%%%%%%%%%
%%%%%%%%%%%%%%%%%%%%%%%%%%%%%%%%%%%%%%%%%%
%%%%%%%%PAGEBREAK%%%%%%%PAGEBREAK%%%%%%%%%
%%%%%%%%%%%%%%%%%%%%%%%%%%%%%%%%%%%%%%%%%%
%%%%%%%%%%%%%%%%%%%%%%%%%%%%%%%%%%%%%%%%%%
%%%%%%%%%%%%%%%%%%%%%%%%%%%%%%%%%%%%%%%%%%
%%%%%%%%%%%%%%%%%%%%%%%%%%%%%%%%%%%%%%%%%%
%%%%%%%%PAGEBREAK%%%%%%%PAGEBREAK%%%%%%%%%
%%%%%%%%%%%%%%%%%%%%%%%%%%%%%%%%%%%%%%%%%%
%%%%%%%%%%%%%%%%PAGEBREAK%%%%%%%%%%%%%%%%%
%%%%%%%%%%%%%%%%%%%%%%%%%%%%%%%%%%%%%%%%%%
%%%%%%%%PAGEBREAK%%%%%%%PAGEBREAK%%%%%%%%%
%%%%%%%%%%%%%%%%%%%%%%%%%%%%%%%%%%%%%%%%%%
%%%%%%%%%%%%%%%%%%%%%%%%%%%%%%%%%%%%%%%%%%
\begin{alignment}[
    texts=edition[class="edition"];
    translation[class="translation"],
  ]
\begin{edition}
 \ekddiv{type=ed}
 \centerline{\textrm{\small{[\uproman{7}.\textsuperscript{\coro{\lowroman{1}-\lowroman{25}}} Fourth Cakra]}}}
 \label{cakra4}
 \bigskip
    \begin{prose}
    \noindent
\note[type=source, labelb=65, lem={\textbf{Re}}]{PT\textsuperscript{qcr \cdot YSV} (Ed. p. 832): anāhatam aṣṭapīṭhaṃ (\textit{mahāpīṭhaṃ} YK\textsuperscript{ccn \cdot YSV} 1.257  Ed. p. 21) caturthakamalaṃ hṛdi | sūryapatraṃ mahājyotir mahāsūkṣman tu cākṣuṣam | sūryapatraṃ dvādaśadalam (sentence \om in YK\textsuperscript{ccn \cdot YSV}) | tanmadhye 'ṣṭadalaṃ padmam ūrddhavaktraṃ mahāprabham |}
%-----------------------
% \om                                                   \B
%caturthaṃ hṛdayamadhye dvādaśadalaṃ kamalaṃ vartate/   \E
%caturthaṃ hṛdayamadhye dvadaśadalaṃ kamalaṃ varttate/  \P
%caturthaṃ hṛdayamadhye dvadaśadalaṃ kamalaṃ varttate/  \L
%caturthaṃ hṛdayamadhye dvadaśadalaṃ kamalaṃ varttate/  \N1
%caturthacakrakamalaṃ hṛdayamadhye dvadaśadalaṃ bhavati \N2    
%caturthaṃ hṛdayamadhye dvadaśadalaṃ kamalaṃ varttate   \D
%caturthaṃ hṛdayamadhye dvadaśadalaṃ kamalaṃ varttate/  \U1   
%caturthaṃ hṛdayamadhye dvadaśadalaṃ kamalam asti/      \U2
%
% anāhatacakraṃ hṛdayasthānaṃ // śvetavarṇaṃ tamoguṇaḥ // rudrodevatā // umāśaktiḥ // hiraṇyagarbhaṛṣiḥ // naṃdivāhanaṃ // prāṇavāyuḥ // jyotiḥ kalākāraṇaṃ dehe // suṣuptir avasthā // paśyaṃtivācā // sāmavedaḥ // gārhasyatyogniḥ? // śivaliṇgaṃ // prāptibhūmikā // sarūpatāmuktiḥ // dvādaśādalāni //dvādaśamātrā // kaṃ khaṃ gaṃ ghaṃ ṇaṃ caṃ chaṃ jaṃ jhaṃ yaṃ taṃ thaṃ // bahirmātrā // rudrāṇī // tejasā // tāpinī // sukhadā // caitanyā // śivadā // śānti // umā // gaurī // mātara // jvālā // prajvālinī // ajapājapasahasra // 6000 gha. 96 pa. 40 // U2
%-----------------------
%The fourth lotus having twelve-petals exists in the middle at the heart. \extra{[The] Anāhatacakras place is within the heart\footnote{This seems to be redundant.}. The color is white. The quality is \textit{tamas}. The deity is Rudra. The power is Umā. The Ṛṣi is Hiraṇyagarbha. The mount is Nandi. The vitalwind is Prāṇa. In the body it is the light that causes parts (\textit{kalākaraṇa})\footnote{What is this?!}. The state is deep sleep. The speech is \textit{Paśyantī}\footnote{Add footnote of entry in \textit{Tāntrikābhidhānakośa}.}.The [Veda] is Sāmaveda. The fire is Gārhapatya\footnote{Add explanation.}. The Liṅgam is Śivaliṅga. The ability to attain everything on the earth [and] the uniform liberation [are attributed to this \textit{cakra}]. [There are] twelve petals, [and] twelve measures: kaṃ khaṃ gaṃ ghaṃ ṇaṃ caṃ chaṃ jaṃ jhaṃ yaṃ taṃ [and] thaṃ. The external measure: Rudra's wife, light (\textit{tejasā?}), glow, \textit{sphakadā}?, consciousness (\textit{caitanyā}), bestower of grace, peace, Umā, Gaurī, Mātara, the flame [and] Prajvālinī.}
%-----------------------
\note[type=testium, labelb=66, lem={\textbf{Cie}}]{\textit{Yogasaṃgraha} IGNCA 30020 folio 2v. ll. 2: hṛdayamadhye dvadaśadalaṃ}
\note[type=testium, labelb=66a, lem={\textbf{Ri}}]{SSP 2.4 (Ed. p. 30): caturthaṃ hṛdayacakram aṣṭadalakamalam adhomukhaṃ tanmadhye karṇikāyāṃ liṅgākārāṃ jyotīrūpām dhyāyet | saiva haṃsakalā sarvendriyavaśyā bhavati |}
\app{\lem[wit={ceteri}]{caturthaṃ}
      \rdg[wit={N2}]{caturthacakrakamalaṃ}} hṛdayamadhye dvādaśadalaṃ
    \app{\lem[wit={ceteri}]{kamalaṃ}
       \rdg[wit={N2}]{\om}} 
    \app{\lem[wit={ceteri}]{vartate}
       \rdg[wit={U2}]{asti}
       \rdg[wit={N2}]{bhavati}}/
       %%%%%%%%%%%%%%%%%
       %%%%%%%%%%%%%%%%
       %%%%%%%%%%%%%%%%%%
       %%%%%%%%%%%%%%%%%
       %%%%%%%%%%%%%%%%
      \extra{anāhatacakraṃ hṛdayasthānaṃ\dd{}
        \app{\lem[type=emendation, resp=egoscr]{śvetaṃ}
          \rdg[wit={U2}]{śveta°}} varṇaṃ\dd{}
        tamo guṇaḥ\dd{}
        rudro devatā\dd{}
        umā śaktiḥ\dd{}
        hiraṇyagarbha ṛṣiḥ\dd{}
        nandi vāhanaṃ\dd{}
        \app{\lem[type=emendation, resp=egoscr]{prāṇo}
          \rdg[wit={U2}]{prāṇa°}} vāyuḥ\dd{}
        \crazy{\app{\lem[type=emendation, resp=egoscr]{jyotiskalākāraṇaṃ deham}
          \rdg[wit={U2}]{jyotiḥ kalākāraṇaṃ dehe}}\dd{}}
        suṣuptir-avasthā\dd{}
        \app{\lem[type=emendation, resp=egoscr]{paśyantī}
          \rdg[wit={U2}]{paśyaṃti}} vācā\dd{}
        sāmavedaḥ\dd{}
        \app{\lem[type=emendation, resp=egoscr]{gārhapatyo 'gniḥ}
          \rdg[wit={U2}]{gārhasyatyo gniḥ}}\dd{}
        \app{\lem[type=emendation, resp=egoscr]{śivo}
          \rdg[wit={U2}]{śiva°}} liṅgaṃ\dd{}
        \app{\lem[type=emendation, resp=egoscr]{prāptiḥ}
          \rdg[wit={U2}]{prāpti°}} bhūmikā\dd{}
        sarūpatā muktiḥ\dd{}
        dvādaśādalāni\dd{}
        dvādaśamātrā\dd{}
        kaṃ khaṃ gaṃ ghaṃ ṇaṃ caṃ chaṃ jaṃ jhaṃ yaṃ taṃ thaṃ\dd{}
        bahir-mātrā\dd{}
        rudrāṇī\dd{}
        tejasā\dd{}
        tāpinī\dd{}
        sukhadā\dd{}
        caitanyā\dd{}
        śivadā\dd{}
        \app{\lem[type=emendation, resp=egoscr]{śāntiḥ}
          \rdg[wit={U2}]{śānti}}\dd{}
        umā\dd{}
        gaurī\dd{}
        \app{\lem[type=emendation, resp=egoscr]{mātarā} %%%?????
          \rdg[wit={U2}]{mātara}}\dd{}
        jvālā\dd{}
        prajvālinī\dd{}
        \app{\lem[type=emendation, resp=egoscr]{ajapājapaḥ}
          \rdg[wit={U2}]{ajapājapa°}} \app{\lem[type=emendation, resp=egoscr]{sahasraḥ}
          \rdg[wit={U2}]{°sahasra}}\dd{} 6000\dd{} gha. 96 pa. 40\dd{}}\textsuperscript{\begin{otherlanguage}{english}\coro{[\lowroman{2}]}\end{otherlanguage}}
 %%%%%%%%%%%%
 %%%%%%%%%%%%%
 %%%%%%%%%%%%%
 %%%%%%%%%%%%%
 %%%%%%%%%%%%%%
%-----------------------
% \om                                        \B
%atitejomayatvād   dṛṣṭigocaraṃ na bhavati   \E  
%atitejomayatvāt   dṛṣṭigocaraṃ na bhavati   \P
%atitejomayatvād   dṛṣṭigocaraṃ na bhavati// \L
%atitejomayatvāt / dṛṣṭigocaraṃ na bhavati/ \N1
%atitejomayatvāt   dṛṣṭigocaraṃ na bhavati/ \N2
%atitejomayatvāt / dṛṣṭigocaraṃ na bhavati/ \D
%atitejomayatvāt / dṛṣṭigocaraṃ na bhavati/ \U1
%atitejomayatvād   dṛṣṭigocaratāṃ na yāti// \U2 
%-----------------------
%Due to being made of [such an] intense light [the fourth lotus] is not in the range of sight.
%-----------------------
\note[type=testium, labelb=67, lem={\textbf{Cie}}]{\textit{Yogasaṃgraha} IGNCA 30020 folio 2v. ll. 2: tejomayatvāt | dṛṣṭigocaraṃ na bhavaty etādṛśaṃ vartate}
    atitejomayatvād-dṛṣṭi\app{\lem[wit={ceteri}, alt={°gocaraṃ}]{gocaraṃ}  %SANDHI einbauen?! 
       \rdg[wit={U2}]{gocaratāṃ}}
na
\app{\lem[wit={ceteri}]{bhavati}
  \rdg[wit={U2}]{yāti}}/
%-----------------------
% \om                                               \B
%tanmadhye 'ṣṭadalam adhomukhaṃ kamalaṃ varttate // \E  
%tanmadhye 'ṣṭadale  mukhaṃ kamalaṃ varttate //     \P
%tanmadhye ṣṭadalaṃ  adhomukha--kamalaṃ vartate //  \L
%tanmadhye aṣṭadalaṃ adhomukhaṃ kamalaṃ vartate //  \N1
%tanmadhye aṣṭadalaṃ adhomukhaṃ kamalaṃ varttate//  \N2
%tanmadhye aṣṭadalaṃ adhomukhaṃ kamalaṃ vartate //  \D
%tanmadhye aṣṭadalaṃ adhomukhaṃ kamalaṃ vartate /   \U1
%tanmadhye 'ṣṭadalaṃ adhomukhaṃ kamalaṃ asti / manaś-cakre// manodevatā// bahiśaktiḥ// ātmaṛṣih// nābhimadhye sthitaṃ padmaṃ nālaṃ tasya daśāgulaṃ/ komalaṃ tasya tan nālaṃ nirmalaṃ cāpy adhomukhaṃ/ kadalīpuṣpasaṃkāśaṃ tanmadhye ca pratiṣṭhitaṃ/ mana unnaty-asaṃkalpa/ vikalpātmakam-eva ca/ pūrvadale svetavarṇe yadā viśrāmate manaḥ// dharmakīrtividyādi sadbuddhir-bhavati/ agnikoṇe āraktavarṇe nidrā ālasyamāyāmandamatir-bhavati/ dakṣiṇe kṛṣṇavarṇeti tadā krodhotpattir bhavati/ naiṛtye nīlavarṇe mamatāmatir bhavati/ paścime kapilavarṇe/ krīḍāhāsotsavotsāhamatir bhavati/ vāyav ye śāmavarṇe cintodvegamatir bhavati/ uttare pītavarṇe bhogaśṛṇgāramahodayamatir bhavati/ īśāne gauravarṇe jñānasaṃdhāne matir bhavati/} \U2
%-----------------------
\note[type=testium, labelb=68, lem={'ṣṭadalaṃ}]{\textit{Yogasaṃgraha} IGNCA 30020 folio 2v. ll. 3: tanmadhye 'ṣṭadalaṃ adhomukhaṃ kamalaṃ ||}
    tanmadhye \app{\lem[wit={E,U2},alt={'ṣṭadalam}]{'ṣṭadala\skp{m-a}}
      \rdg[wit={P}]{'ṣṭadale}
      \rdg[wit={L}]{ṣṭadalaṃ}
      \rdg[wit={X}]{aṣṭadalaṃ}}\app{\lem[wit={ceteri},alt={adhomukhaṃ kamalaṃ}]{\skp{m-a}dhomukhaṃ kamalaṃ}
        \rdg[wit={L}]{adhomukhakamalaṃ}
        \rdg[wit={P}]{mukhaṃ kamalaṃ}}
      \app{\lem[wit={ceteri}]{vartate}
        \rdg[wit={U2}]{asti}}/
%%%%%%%%%%%%%%%%
%%%%%%%%%%%%%%%
%%%%%%%%%%%%%%%
%%%%%%%%%%%%%%
%%%%%%%%%%%%%%%
  \extra{manaś-cakre\dd{}
    mano devatā\dd{}
        \app{\lem[type=conjecture, resp=egoscrconj]{bahiśśaktiḥ}
          \rdg[wit={U2}]{bahiśaktiḥ}}\dd{}
        \note[type=philcomm, labelb=68a, lem={bahiśśaktiḥ}]{The conjecture is based on the the usage in \citetitle[p. 96]{kriyakrama}. It can also be found in \citetitle[p. 80]{sakalagama}\textsuperscript{ccn \cdot siddhāntaśekhare}. Both texts use the term in the context of \textit{cakra}s, channels, breath-retention and visualization.}
        \app{\lem[type=emendation, resp=egoscr]{ātmā}
          \rdg[wit={U2}]{ātma°}} ṛṣiḥ\dd{}\textsuperscript{\begin{otherlanguage}{english}\coro{[\lowroman{5}]}\end{otherlanguage}}
        nābhimadhye}
\extra{sthitaṃ padmaṃ nālaṃ tasya
        \app{\lem[type=emendation, resp=egoscr]{daśāṅgulaṃ}
          \rdg[wit={U2}]{daśāgulaṃ}}/ %In the middle of the navel [exists] a place, being a lotus, its tube measures ten \textit{aṅgula}s,
        komalaṃ tasya tan-nālaṃ nirmalaṃ cāpy-adhomukhaṃ/ %The fluid (\textit{komala}) of the tube is pure facing upwards.
        kadalīpuṣpasaṃkāśaṃ tanmadhye ca pratiṣṭhitaṃ/ % In its middle is a place shining like a banana-flower.
        mana \app{\lem[type=conjecture, resp=egoscrconj, alt={ānati}]{āna\skp{ty-a}}
          \rdg[wit={U2}]{unnaty}
        }\app{\lem[type=emendation, resp=egoscr,alt={asaṃkalpam}]{\skm{ty-a}saṃkalpam}
          \rdg[wit={U2}]{asaṃkalpa}}/  
        vikalpātmakam-eva ca/}\textsuperscript{\begin{otherlanguage}{english}\coro{[\lowroman{10}]}\end{otherlanguage}}
%%%%%%%%%%
%%%%%%%%%%
%%%%%%%%%%%%
%%%%%%%%%%%%%%
%%%%%%%%%%%%%%%
        \extra{
          pūrvadale \app{\lem[type=emendation, resp=egoscr, alt={°śveta}]{śveta}
            \rdg[wit={U2}]{sveta°}}varṇe yadā \app{\lem[type=emendation, resp=egoscr]{viśramate}
            \rdg[wit={U2}]{viśrāmate}} manaḥ\dd{}
        dharmakīrtividyādisadbuddhir-bhavati/ %While the mind rests on the eastern petal [which is] white in colour clear intellekt arises, which is [endowed with]  \textit{dharma}, fame and knowledge etc. 
        %%%%%
        agnikoṇe āraktavarṇe \app{\lem[type=emendation, resp=egoscr, alt={nidrālasya}]{nidrālasya}
          \rdg[wit={U2}]{nidrā ālasya°}}māyāmandamatir-bhavati/  %While [the mind rests on] the south-east, [which is] reddish in color a mind that is weak due to sleep, laziness and illusion arises.
        %%%%
        dakṣiṇe kṛṣṇavarṇeti tadā krodhotpattir-bhavati/ %While [the mind is situated] in the right south, [which is] black in color the generation of anger arises.
        %%%
        \app{\lem[type=emendation, resp=egoscr]{nairṛtye}
          \rdg[wit={U2}]{naiṛtye}} nīlavarṇe mamatāmatir-bhavati/ %While [the mind is situated] in the southwest, [which is] blue in color a mind of pride arises.
        %%%
        paścime kapilavarṇe krīḍāhāsotsavotsāhamatir-bhavati/\textsuperscript{\begin{otherlanguage}{english}\coro{[\lowroman{15}]}\end{otherlanguage}}% While [the mind is situated] in the west, [which is] brown in color a mind that is longing for play, laughing, and celebration arises.
        %%%
        vāyavye \app{\lem[type=emendation, resp=egoscr, alt={°śyāma}]{śyāma}
          \rdg[wit={U2}]{śāma}}varṇe cintodvegamatir-bhavati/ %While [the mind is situated] in the northwest, [which is] dark in color a mind which is restless by sorrow arises.
        %%%
        uttare pītavarṇe bhogaśṛṅgāramahodayamatir-bhavati/ %While [the mind is situated] in the north, [which is] yellow in color a very happy mind with erotic and enjoyment arises.
        īśāne gauravarṇe
        \app{\lem[type=emendation, resp=egoscr, alt={jñānasaṃdhāna°}]{jñānasaṃdhāna}
          \rdg[wit={U2}]{jñānasaṃdhāne}}matir-bhavati/}%While [the mind is situated] in north-east [which is] whitish in color a mind of unity arises through knowledge arises.
      \vspace*{\fill}
      \ekdpb*{}
  %%%%%%%%%%%%
  %%%%%%%%%%%%
  %%%%%%%%%%%%
  %%%%%%%%%%%%
  %%%%%%%%%%%%
\end{prose}
\end{edition}
\begin{translation}
    \ekddiv{type=trans}
    \centerline{\textrm{\small{[\uproman{7}.\textsuperscript{\coro{\lowroman{1}-\lowroman{25}}} Fourth Cakra]}}}%%%%%See Jogpradipikaya Edition Page 163
    \label{cakra4trans}
       \bigskip
       \begin{tlate}
         \noindent
           The fourth twelve-petalled lotus exists in the middle of the heart.\textsuperscript{\coro{[\lowroman{1}]}}\extra{The place of the Anāhatacakra is within the heart. The color is white. Tamas is the quality. Rudra is the deity. Umā is the power. Hiraṇyagarbha is the Ṛṣi. Nandi is the mount. Prāṇa is the vitalwind. \crazy{The cause of the light digit is the body}. Deep sleep is the state. Paśyantī is the speech. Sāma[veda] is the Veda. The fire is the fire of the householder. Śiva is the \textit{liṅga}. The power to attain anything (\textit{prāpti}) is the level. Sarūpatā is the liberation. [There are] twelve petals, [and] twelve matrices: \textit{kaṃ khaṃ gaṃ ghaṃ ṇaṃ caṃ chaṃ jaṃ jhaṃ yaṃ taṃ} [and] \textit{thaṃ}. The external matrix: Rudrāṇī, Tejasā, Tāpinī, Sukhadā, Caitanyā, Śivadā, Śānti, Umā, Gaurī, Mātarā, Jvalā [and] Prajvālinī. A thousandfold recitation of the non-recited; 6000 ; 16 \textit{ghaṭi}s [and] 40 \textit{palā}s.}\textsuperscript{\coro{[\lowroman{2}]}} Due to being made of [such an] intense light [the fourth lotus] is not in the range of sight. In its middle exists a eight-petalled lotus facing downwards.\\

           \extra{The mind resides in the \textit{cakra}. Manas is the deity. Bahi is the power\footnote{The term \textit{bahiśśaktiḥ} designates the visualization of the external energy infused by inhalation that permeates the body. \citetitle[p. 80]{sakalagama}\textsuperscript{ccn \cdot siddhāntaśekhare}:
      \begin{quote}
 caraṇāṅguṣṭhayoryugmāt sañcintya suṣirāntanau |\\
suṣirāntabahiśśaktiṃ vyāpinīṃ cintayet tataḥ ||
\end{quote}} The Ṛṣi is the self.\textsuperscript{\coro{[\lowroman{5}]}} In the middle of the navel exists a lotus. Its stalk measures ten \textit{aṅgula}s. The stalk is soft, pure [and] facing downwards. In its middle [it is] endowed with the shine of a banana-flower. The mind is unstable, fickle, and full of doubt.\textsuperscript{\coro{[\lowroman{10}]}} While the mind rests on the white eastern petal, clear intellect [endowed with] \textit{dharma}, fame and knowledge etc. arises. While in the south-east, [which is] reddish in color a mind that is weak due to sleep, laziness and illusion arises. While in the right south, [being] black in color, anger is generated. While in the southwest, [being] blue in color, a mind of pride arises. While in the west, [being] brown in color, a mind that is longing for play, laughing, and celebration arises.\textsuperscript{\begin{otherlanguage}{english}\coro{[\lowroman{15}]}\end{otherlanguage}}
While in the northwest, [being] dark in color, a mind restless by sorrow arises. While in the north, [being] yellow in color, a very happy mind with erotic and enjoyment arises. While in north-east [being] whitish in color, a mind of unity through knowledge arises.}
 \vspace*{\fill}
      \ekdpb*{}
\end{tlate}
\end{translation}
\end{alignment}
%%%%%%%%%%%%%%%%%%%%%%%%%%%%%%%%%%%%%%%%%%
%%%%%%%%%%%%%%%%%%%%%%%%%%%%%%%%%%%%%%%%%%
%%%%%%%%PAGEBREAK%%%%%%%PAGEBREAK%%%%%%%%%
%%%%%%%%%%%%%%%%%%%%%%%%%%%%%%%%%%%%%%%%%%
%%%%%%%%%%%%%%%%PAGEBREAK%%%%%%%%%%%%%%%%%
%%%%%%%%%%%%%%%%%%%%%%%%%%%%%%%%%%%%%%%%%%
%%%%%%%%PAGEBREAK%%%%%%%PAGEBREAK%%%%%%%%%
%%%%%%%%%%%%%%%%%%%%%%%%%%%%%%%%%%%%%%%%%%
%%%%%%%%%%%%%%%%%%%%%%%%%%%%%%%%%%%%%%%%%%
%%%%%%%%%%%%%%%%%%%%%%%%%%%%%%%%%%%%%%%%%%
%%%%%%%%%%%%%%%%%%%%%%%%%%%%%%%%%%%%%%%%%%
%%%%%%%%PAGEBREAK%%%%%%%PAGEBREAK%%%%%%%%%
%%%%%%%%%%%%%%%%%%%%%%%%%%%%%%%%%%%%%%%%%%
%%%%%%%%%%%%%%%%PAGEBREAK%%%%%%%%%%%%%%%%%
%%%%%%%%%%%%%%%%%%%%%%%%%%%%%%%%%%%%%%%%%%
%%%%%%%%PAGEBREAK%%%%%%%PAGEBREAK%%%%%%%%%
%%%%%%%%%%%%%%%%%%%%%%%%%%%%%%%%%%%%%%%%%%
%%%%%%%%%%%%%%%%%%%%%%%%%%%%%%%%%%%%%%%%%%
%%%%%%%%%%%%%%%%%%%%%%%%%%%%%%%%%%%%%%%%%%
%%%%%%%%%%%%%%%%%%%%%%%%%%%%%%%%%%%%%%%%%%
%%%%%%%%PAGEBREAK%%%%%%%PAGEBREAK%%%%%%%%%
%%%%%%%%%%%%%%%%%%%%%%%%%%%%%%%%%%%%%%%%%%
%%%%%%%%%%%%%%%%PAGEBREAK%%%%%%%%%%%%%%%%%
%%%%%%%%%%%%%%%%%%%%%%%%%%%%%%%%%%%%%%%%%%
%%%%%%%%PAGEBREAK%%%%%%%PAGEBREAK%%%%%%%%%
%%%%%%%%%%%%%%%%%%%%%%%%%%%%%%%%%%%%%%%%%%
%%%%%%%%%%%%%%%%%%%%%%%%%%%%%%%%%%%%%%%%%%
\begin{alignment}[
   texts=edition[class="edition"];
    translation[class="translation"],
  ]
\begin{edition}
 \ekddiv{type=ed}
 \begin{prose}
   \noindent
%-----------------------
% \om                                                     \B      
%tanmadhye prāṇavāyoḥ sthānam    aṣṭadalakamalamadhye liṃgākārā karṇikā  kathyate/  \E 
%tanmadhye prāṇavāyoḥ sthānam    aṣṭadalakamalamadhye liṃgākārā karṇikā  kathyate/  \P
%tanmadhye prāṇavāyoḥ sthānam    aṣṭadalakamalamadhye liṃgākārā karṇikā  kathyate// \L
%tanmadhye prāṇavāyoḥ sthānam    aṣṭadalakamalamadhye liṃgākārā karṇikā  kathyate// \N1
%tanmadhye prāṇavāyoḥ sthānam/   aṣṭadalakamalamadhye liṃgākārā karṇikā  kathyate// \N2
%tanmadhye prāṇavāyoḥ sthānam // aṣṭadalakamalamadhye liṃgākārā karṇi    kathyate// \D
%ta ca     prāṇavāyoḥ sthānam /  aṣṭadalakamalamadhye liṃgākārā karṇikā              \D2      
%tanmadhye prāṇavāyo  sthānam    aṣṭadalakamalamadhye liṃgākārā karṇikā  kathyate    \U1
%tanmadhye prāṇavāyo  sthānam // aṣṭadalakamalamadhye liṃgākārā karṇikā  kathyate    \U2
%-----------------------
%It's said that in its middle is the place of the \textit{prāṇa}-vitalwind [and] in the middle [of] the eight-petalled lotus is a pericarp (\textit{karṇikā}) in the form of a \textit{liṅga}.
%-----------------------
   \note[type=testium, labelb=69, lem={\textbf{Cie}}]{\textit{Yogasaṃgraha} IGNCA 30020 folio 2v. ll. 3-4: ta ca prāṇavāyoḥ sthānam | aṣṭadalakamalamadhye liṃgākārā karṇikā}
   \note[type=source, labelb=70, lem={\textbf{Re}}]{PT\textsuperscript{qcr \cdot YSV} (Ed. p. 832): prāṇavāyoḥ sthalañcāsya liṅgākāran tu karṇikā | kālikākhyā karṇikeyaṃ asyā madhye tu kuṇḍalī |}
tanmadhye prāṇa\app{\lem[wit={ceteri},alt={°vāyoḥ}]{vāyoḥ}
       \rdg[wit={U1,U2}]{°vāyo}} sthānam-aṣṭadalakamalamadhye liṃgākārā
        \app{\lem[wit={ceteri}]{karṇikā}
          \rdg[wit={U2}]{karṇi}}
kathyate/   
%-----------------------
% \om                                                     \B
%tasyāḥ karṇiketi saṃjñā tatkarṇikāmadhye    padmarāgasamānavarṇāṃ-----------guṣṭhapramāṇaikā     puttalikā varttate //  \E  
%tasyāḥ kaliketi saṃjñā tatkalikāmadhye      padmarāgaratnasamānavarṇāṃ    aṃguṣṭhapramāṇā    ekā puttalikā varttate     \P
%tasyāḥ kalikeli                 madhye      padma    ratnasamānavarṇā //  aṃguṣṭhapramāṇā // ekā puttalikā varttate //  \L
%tasyāḥ kaliketi saṃjñā tatkalikāmadhye      padmarāgaratnasamānavarṇāṃ    aṃguṣṭhapramāṇā    ekā puttalikā varttate     \N1
%tasyāḥ kaliketi saṃjñā/tataḥ kalikāmadhye   padmarāgaratnasamānavarṇa     aṃguṣṭhapramāṇā    ekā putalikā  varttate/    \N2 %%%p4recto
%tasyāḥ kaliketi saṃjñā tatkalikāmadhye      padmarāgaratnasamānavarṇā     aṃguṣṭhapramāṇāt   ekā puttalikā varttate /   \D
%tasyāḥ kaliketi saṃjñā tatkalikāmadhye      padmarāgaratnasamānavarṇā     aṃguṣṭhapramāṇāt   ekā puttalikā varttate /   \U1
%tasyāḥ kaliketi saṃjñā tatkalikāmadhye      padmarāgaratnasamānavarṇā  // aṃguṣṭhapramāṇā    ekā puttalikā varttate /   \U2
%-----------------------
%The technical designation of her is kalikā. In the middle of this kalikā exists a single thumbsized (divine) figurine (puttalikā) being similiar to a ruby-gem in color.
%-----------------------        
\note[type=testium, labelb=71, lem={\textbf{Cie}}]{\textit{Yogasaṃgraha} IGNCA 30020 folio 2v. ll. 4: kaliketi saṃjñikāsti tanmadhye padmarāgaratnasamānavarṇā aṃguṣṭhapramāṇā ekā puttalikā}
\note[type=source, labelb=72, lem={\textbf{Re}}]{PT\textsuperscript{qcr \cdot YSV} (Ed. p. 832): padmavatyāḥ (\textit{padmāvatyāḥ} YK\textsuperscript{ccn \cdot YSV} 1.259 Ed. p. 21) prabhāṅguṣṭhapramāṇā (°\textit{prāmāṇa}° YK\textsuperscript{ccn \cdot YSV} 1.259 Ed. p. 21) ratnasannibhā | tasyā saṅgī (\textit{tasya saṅgī} YK\textsuperscript{ccn \cdot YSV} 1.260 Ed. p. 21) jīva iti ananto balarūpataḥ | asya dhyānaṃ (\textit{dhyānād} YK\textsuperscript{ccn \cdot YSV} 1.260 Ed. p. 21) jagadvaśyaṃ khecarīsarvago bhavet | bhavanti vaśyā devādyāś cintākarttur na (\textit{citta°} YK\textsuperscript{ccn \cdot YSV} 1.260 Ed. p. 21) cānyathā | iṣṭāniṣṭo (\textit{iṣṭāniṣṭa} YK\textsuperscript{ccn \cdot YSV} 1.261 Ed. p. 21) bhaved vaśyaḥ (\textit{vaśyaṃ} YK\textsuperscript{ccn \cdot YSV} 1.261 Ed. p. 21) satyaṃ satyaṃ na saṃśayaḥ | iṣṭasiddhir bhavet tasya sarvajñādiguṇodayaḥ |}
tasyāḥ
\app{\lem[wit={ceteri}]{kaliketi}
  \rdg[wit={L}]{kalikeli}
  \rdg[wit={E}]{karṇiketi}}
\app{\lem[wit={ceteri}]{saṃjñā}
  \rdg[wit={L}]{\om}}/\textsuperscript{\begin{otherlanguage}{english}\coro{[\lowroman{20}]}\end{otherlanguage}}
\app{\lem[wit={ceteri}]{tatkalikāmadhye}
  \rdg[wit={N2}]{tataḥ}
  \rdg[wit={L}]{\om}}
padma\app{\lem[type=emendation, resp=egoscr,alt={°rāgaratnasamānavarṇāṅguṣṭhapramāṇaikā}]{rāgaratnasamānavarṇāṅguṣṭhapramāṇaikā}
  \rdg[wit={E}]{°rāgasamānavarṇāṃguṣṭhapramāṇaikā}
  \rdg[wit={L}]{°ratnasamānavarṇā aṃguṣṭhapramāṇā ekā}
  \rdg[wit={P,N1}]{°rāgaratnasamānavarṇāṃ || aṃguṣṭhapramāṇā || ekā}
  \rdg[wit={N2}]{°rāgaratnasamānavarṇa aṃguṣṭhapramāṇā ekā}
  \rdg[wit={D,U1}]{°rāgaratnasamānavarṇā aṃguṣṭhapramāṇāt ekā}} puttalikā
vartate/
%-----------------------
%
%tasyā  jīvasaṃjñā          tasyā  balamadhyasvarūpaṃ        koṭijihvābhir  vaktuṃ naiva śakyate // \E
%tasyā  jīvasaṃjñā          tasyā  balam atha svarūpaṃ       koṭijihvābhir  vaktuṃ naiva śakyate // \P 
%tasya                             bala sappa svarūpaṃ       koṭijihvāyābhi vaktuṃ na    śakyate // \L 
%tasyāḥ jīveti saṃjñāḥ      tasyāḥ balaṃ atha ca svarūpaṃ    koṭijihvābhir  vaktuṃ na    śakyate // \N1
%tasyāḥ jīveti saṃjñaḥ//    tasyā  balaṃ atha ca svarūpaṃ    koṭijihvābhir  vaktuṃ na    śakyate // \N2
%tasyāḥ jīveti saṃjña/      tasyāḥ balaṃ atha ca svarūpaṃ    koṭijihvābhir  vaktuṃ na    śakyate // \D
%       jīveti saṃjñikāsti/ tasyāḥ balaṃ         svarūpaṃ ca koṭijihvābhir  vaktuṃ na    śakyaṃ  //  D2
%tasyāḥ jīveti saṃjñā       tasyāḥ balaṃ atha ca svarūpaṃ    koṭijihvābhir  vaktuṃ na    śakyate // \U1
%tasyā  jīvasaṃjñā//        tasya  balaṃ tasya atha svarūpaṃ koṭijihvābhir  vaktuṃ na    śakyate // \U2
%-----------------------  
%Her technical designation is embodied soul. Not even with a thousand tongues it is possible to talk about her nature and her power.
%-----------------------
\note[type=testium, labelb=73, lem=\textbf{Cie}]{\textit{Yogasaṃgraha} IGNCA 30020 folio 2v. ll. 5: jīveti saṃjñikāsti | tasyāḥ balaṃ svarūpaṃ ca koṭijihvābhir vaktuṃ na śakyaṃ ||}
\app{\lem[wit={E,P}]{tasyā}
     \rdg[wit={X}]{tasyāḥ}
     \rdg[wit={L,U2}]{tasya}}
\app{\lem[wit={U2}]{jīveti saṃjñā}
     \rdg[wit={N1}]{jīveti saṃjñāḥ}
     \rdg[wit={N2}]{jīveti saṃjñaḥ ||}
     \rdg[wit={D}]{jīveti saṃjña |}
     \rdg[wit={Y}]{jīvasaṃjñā ||}
     \rdg[wit={L}]{\om}}/
\app{\lem[wit={E,N2,P}]{tasyā}
     \rdg[wit={D,N1,U1}]{tasyāḥ}
     \rdg[wit={U2}]{tasya}}
   \app{\lem[wit={ceteri}]{balaṃ atha ca svarūpaṃ}
     \rdg[wit={E}]{balamadhyasvarūpaṃ}
     \rdg[wit={L}]{bala sappa svarūpaṃ}
     \rdg[wit={P}]{balam atha svarūpaṃ}
     \rdg[wit={U2}]{balaṃ tasya atha svarūpaṃ}}
\app{\lem[wit={ceteri}, alt={koṭijihvābhir}]{koṭijihvābhi\skp{r-va}}
    \rdg[wit={L}]{koṭijihvāyābhi}}\skp{r-va}ktuṃ
\app{\lem[wit={ceteri}]{na}
    \rdg[wit={E,P}]{naiva}}
śakyate/
%-----------------------
% \om \B
%asyā  mūrter   dhyānakāraṇāt           svarga-pātāl--ākaśamanuṣyagandharvakinnaraguhyakavidyādharalokasambandhinyaḥ strīyo 'pi--------------------       vaśyā bhavanti / \E
%asyā  mūrter   dhyānakaraṇāt           svarga-pātāl--ākāśamanuṣyagandharvakiṃnaraguhyakavidyādharalokasaṃbaṃdhinyaḥ strīyo 'pi--------------------       vaśyā bhavanti / \P
%asyā  mūrtir   dhyānāt                 svarga-pātāl--ākāśamanuṣyagaṃdharvakinnaraguhyakavidyādharalokasambandhinyaḥ strīyo 'pi--------------------       vaśyā bhavanti /L
%asyāḥ mūrter  dhyānakaraṇāt            svarga-pātāla ākāśamanuṣyagaṃdharvakinnaraguhyakavidyādharalokasaṃbaṃdhinyaḥ strīyaḥ sādhakasya puruṣasya         vaśyā bhavanti // \N1
%asyā  mūrttir dhyānakaraṇāt/           svarga-pātāla ākāśamanuṣya/ gaṃdharvakinnara/ guhyaka/vidyādhara/lokasaṃbaṃdhinyaḥ strīyaḥ sādhakasya puruṣasya   vaśyo bhavati/ \N2
%asyāḥ mūrter  dhyānakaraṇāt            svarga-pātāla ākāśamanuṣyagaṃdharvakiṃnaraguhyakavidyādharalokasaṃbaṃdhinyaḥ strīyaḥ sādhakasya puruṣasya         vaśyā bhavanti // \D
%asyāḥ mūrter  dhyānakaraṇāt            svarga-pātāla ākāśamanuṣyagaṃdharvakiṃnaraguhyakavidyādharalokasaṃbaṃdhinyaḥ strīyaḥ sādhakasya puruṣasya         vaśyā bhavanti // \U1
%tasyāḥ mūrter dhyānaṃ karaṇāt //       svarga-pātāl--ākāśamanuṣyagandharvakinnaraguhyakavidyādharalokasaṃbadhinya---striyo  pi---------------------------vaśyā bhavaṃti // \U2
%-----------------------
%“Because of the exercise of meditation on this form the inhabitants of the universe (which are) Humans, Gandharvas, Kinnaras, Guhyakas, Vidyādharas and (their) females, in the heavenly world, underworld and open space are obedient to the will of the practicing person.”, is what is said here.  
%-----------------------
\note[type=testium, labelb=74, lem=\textbf{Cie}]{\textit{Yogasaṃgraha} IGNCA 30020 folio 2v. ll. 5-6: asyā mūrtter dhyānakaraṇāt sādhakasya svargapātāla ākāśagaṃdharvakiṃnaraguhyakavidyādharastrīyo vaśā bhavati |}
\app{\lem[wit={ceteri}]{asyā}
    \rdg[wit={N1,D,U1}]{asyāḥ}
    \rdg[wit={U2}]{tasyāḥ}}
 \app{\lem[wit={ceteri}, alt={mūrter}]{mūrte\skp{r-dhyā}}
    \rdg[wit={L,N2}]{mūrtir}}\app{\lem[wit={ceteri}, alt={dhyānakāraṇāt}]{\skm{r-dhyā}nakāraṇā\skp{t-sva}}
    \rdg[wit={U2}]{dhyānaṃ karaṇāt ||}
    \rdg[wit={L}]{dhyānāt}
  }\skm{t-sva}rga\app{\lem[wit={Y},alt={°pātālākaśa°}]{pātālākaśa}
    \rdg[wit={X}]{°pātāla ākāśa°}}manuṣyagandharvakinnaraguhyakavidyādhara
  loka\app{\lem[wit={ceteri},alt={°saṃbandhinyaḥ}]{saṃbandhinyaḥ}
    \rdg[wit={U2}]{saṃdadhinya}}
 \app{\lem[wit={X}]{strīyaḥ sādhakasya puruṣasya}
    \rdg[wit={E,P,L}]{strīyo 'pi}
    \rdg[wit={U2}]{striyo pi}}
 \app{\lem[wit={ceteri}]{vaśyā bhavanti}
   \rdg[wit={N2}]{vaśyo bhavati}}/
\note[type=testium, labelb=75, lem={\textbf{Cie}}]{\textit{Yogasaṃgraha} IGNCA 30020 folio 2v. ll. 6-7: pṛthvī loke manuṣyādi striṇāṃ kākathā cha |}
%tanmadhye koṭicaṃdrasamaprabhaḥ ekaḥ puruṣo varttate  \N1bhavanti/\note[type=philcomm, labelb=s16, lem={bhavanti}]{\getsiglum{U1} adds a flawed phrase hereafter: \textit{pṛtvī lokasaṃbaṃdhanyo pi striyaḥ vaśyā bhavaṃti/}. I refrained to include it in the apparatus due to its redundance.}
%-----------------------
% \om \B 
%ityatra kathyate// /E
%ityatra kathyate// \P
%ityatra kathyate// \L
%ityatra kiṃ kathyate // \N1
%ityatra kiṃ kathyate// \N2
%ityaṃtra kiṃ kathyate // \D
%ityatra kiṃ kathyate vā \U1
%ityatra kathyate // \U2
%-----------------------
%is what is said here.  
%-----------------------  
ity-atra  
\app{\lem[wit={X}]{kiṃ}
  \rdg[wit={Y}]{\om}}
\app{\lem[wit={ceteri}]{kathyate}
  \rdg[wit={U1}]{kathyate vā}}\dd{}\textsuperscript{\begin{otherlanguage}{english}\coro{[\lowroman{25}]}\end{otherlanguage}}
\end{prose} \vfill
\nolinenumbers
    \smallskip
    \centerline{\textrm{\small{[\uproman{8}.\textsuperscript{\coro{\lowroman{1}-\lowroman{5}}} Fifth Cakra]}}}
    \label{cakra5}
    \bigskip
\linenumbers
%-----------------------      
%-------------pañcamaṃ kaṇṭhasthāne ṣoḍaśadalaṃ kamalaṃ                         vartate //  \E
%-------------paṃcamaṃ kaṃṭhasthāne ṣoḍaśadalaṃ kamalaṃ                         vartate     \P
%-------------paṃcamaṃ kaṃṭhasthāne ṣoḍaśadalaṃ kamalaṃ                         vartate     \L
%idānīṃ       paṃcamaṃ kamalaṃ      ṣodaśadalaṃ                   kaṃṭhasthāne varttate // \N1
%idānīṃ       paṃcamaṃ kamalaṣodaśadalaṃ                          kaṃṭhasthāne varttate // \N2
%idānīṃ       paṃcamaṃ kamalaṃ      ṣodaśadalaṃ                   kaṃṭhasthāne  varttate // \D --------> Was in diesem Falle machen?
%idānīṃ       paṃcamaṃ kamalaṃ      ṣodaśadalaṃ                   kaṃṭhasthāne        varttate // \U1
%-------------paṃcamaṃ                          viśuddhacakraṃ    kaṃṭhastāne              \U2     
%-----------------------
%Now (follows the description of) the fifth lotus having sixteen petals (which) exists at the location of the throat.
%-----------------------
%U2 continues: dhūmra?varṇe jīvodevatā// avidyāśaktiḥ// virāṭrṣiḥ// vāyurvāhanaṃ// udānavāyuḥ// jvālākalā jālaṃdharobaṃdhaḥ mahākāraṇadeha// tūryāvasthā// parāvācā// atharvaṇavedaḥ// jaṃgamaliṅgaṃ jīvaprāptābhūmikā// sāyujyatāmokṣaḥ// ṣoḍaśadalāni// ṣoḍaśamātrāḥ// atarmātrār-carāḥ// aṃ āṃ iṃ īṃ u ūṃ ṛṃ ṝṃ ḷṃ ḹṃ eṃ aiṃ oṃ auṃ aṃ aṃḥ// bahirmātrā vidyā// avidyā// ichā// śakti// jñānaśaktiḥ// śatalā// mahāvidyā// mahāmāyā// buddhiḥ// tamasī// maitrā?// kumārī// maitrāyaṇī// rudrā// puṣṭa// siṃhanī// ajapājapasahasra/ 1000 gha. 2 pa. 46 akṣara 40//
%-----------------------     
%The colour is smoke-colour. The deity is the embodied soul (\textit{jīva}). The power is ignorance (\textit{avidyā}). The Ṛṣi is Virāṭ\footnote{Who is this?}. The mount is the vitalwind (\textit{vāyu}). The vitalwind is \textit{udāna}. Its Kalā is the flame. The \textit{bandha} is Jālandhara. The body supra-causal (\textit{mahākāraṇa}). The state is the fourth state (\textit{tūrya}). The speech is Parā\footnote{Im Kaśm. Śiv. °das ewige Wort, in welchem potentiell alle Begriffe und Worte ruhen; vgl. das śabdabrahma des Vyākaraṇa. [B.]― Schmidt S. 246}. The [Veda is] Atharvaṇa Veda. The \textit{liṅga} is the living. The level is Jīvaprāptā\footnote{What is this?}. The liberation is absorption into the divine essence (\textit{sāyujyatā}). [There are] sixteen petals [and] sixteen matrices. The internal matrix: aṃ āṃ iṃ īṃ u ūṃ ṛṃ ṝṃ ḷṃ ḹṃ eṃ aiṃ oṃ auṃ aṃ aṃḥ. The external matrix: Vidyā ``she who is knowledge'', Avidyā ``she who is ignorance'', Icchā ``she who is desire'', Śakti ``she who is power'', Jñānaśakti ``she who is the power of knowledge'', Śatalā ``she who is manifold'', Mahāvidyā ``she who is great knowledge'', Mahāmayā ``she who is great illusion'', Buddhi ``she who is intellect'', Tamasī ``she who is darkness'', Maitrā ``she who is love'', Kumārī ``she who is a young girl'', Maitrāyaṇī ``she who is???'', Rudrā ``she who is howling'', Puṣṭā ``she who is abundance'', Siṃhanī ``she who is a lioness''. A thousandfold recitation of the non-recited; 1000 [repetitions for]; 2 \textit{ghaṭi}s, 46 \textit{palā}s. and 40 \textit{akṣara}s.
%-----------------------  
\begin{prose}
  \noindent
\note[type=testium, labelb=76, lem={\textbf{Cie}}]{\textit{Yogasaṃgraha} IGNCA 30020 folio 2v. ll. 7: kaṃṭhasthāne paṃcamaṃ ṣodaśadalaṃ viśudhhasaṃjñakaṃ cakraṃ varttate ||}
\note[type=source, labelb=77, lem={\textbf{Re}}]{PT\textsuperscript{qcr \cdot YSV} (Ed. p. 832) = YK\textsuperscript{ccn \cdot YSV} 1.262 Ed. p. 21: kalāpatraṃ pañcaman tu viśuddhaṃ kaṇṭhadeśataḥ |}
\app{\lem[wit={X}]{idānīṃ}
        \rdg[wit={Y}]{\om}}
pañcamaṃ
\app{\lem[wit={N1,D,U1}]{kamalaṃ ṣodaśadalaṃ kaṇṭhasthāne}
  \rdg[wit={N2}]{kamalaṣodaśadalaṃ kaṇṭhasthāne}
  \rdg[wit={E,P,L}]{kaṇṭhasthāne ṣoḍaśadalaṃ kamalaṃ}
  \rdg[wit={U2}]{viśuddhacakraṃ kaṃṭhastāne}}
vartate/ \extra{\app{\lem[type=emendation, resp=egoscr]{dhūmraṃ varṇaṃ}
          \rdg[wit={U2}]{dhūmravarṇe}}\dd{}
        jīvo devatā\dd{}
        avidyā śaktiḥ\dd{}
        \app{\lem[type=emendation, resp=egoscr,alt={virāṭ}]{virā\skp{ṭ-ṛ}}
          \rdg[wit={U2}]{virāṭha}}\skm{ṭ-ṛ}ṣiḥ\dd{}
vāyur-vāhanaṃ\dd{}
        \app{\lem[type=emendation, resp=egoscr]{udāno}
          \rdg[wit={U2}]{udāna°}} vāyuḥ\dd{}
        jvālā kalā\dd{}
        jālaṃdharo bandhaḥ\dd{}
        \app{\lem[type=emendation, resp=egoscr]{mahākāraṇaḥ dehaḥ}
          \rdg[wit={U2}]{mahākāraṇadeha}}\dd{}
        \app{\lem[type=emendation, resp=egoscr]{tūrya āvasthā}
          \rdg[wit={U2}]{tūryāvasthā}}\dd{}}
\end{prose}
\end{edition}
\begin{translation}
    \ekddiv{type=trans}
    \begin{tlate}
      \noindent
   It is said that in its middle is the place of the \textit{prāṇa}-vitalwind [and] in the middle [of] the eight-petalled lotus is a pericarp (\textit{karṇikā}) in the form of a \textit{liṅga}. The technical designation of her is bud (\textit{kalikā}).\textsuperscript{\coro{[\lowroman{20}]}}\footnote{A similar concept, including the usage of the term \textit{kalikā}, is found in the chapter on creation (\textit{sargakāṇḍa}) of the \citetitle[1898: 54]{ramatosana}. In a quotation attributed to a text called \textit{Śāktānanda} the \textit{jīva} is described as having the shape of a bud of light (\textit{pradīpakalikākāro jīvo}) and always resides in the heart:
        \begin{quote} ādau sañjāyate bījaṃ brahmāṇḍaṃ sahasāṅkuraḥ | tasya madhye sumeruś ca kaṅkāladaṇḍarūpadhṛk | carācarāṇāṃ sarveṣāṃ devādīnāṃ viśeṣataḥ | ālayaḥ savabhūtānāṃ meror abhyantare 'pi ca | pradīpakalikākāro jīvo hṛdi sadā sthitaḥ |
        \end{quote}} In the middle of this bud exists a single thumbsized [divine] figurine (\textit{puttalikā})\footnote{The concept of a \textit{puttalikā} in the heart can be traced back to the Kaula Tantras, e.g. the \citetitle{saradaavalon} 22.126-128: \begin{quote}  puttalikāyā hṛdayaṃ spṛśan prāṇā iha prāṇā jīva iha sthita iti indriyāṇi spṛśan sarvendriyāṇi vāṅmanaścakṣuḥśrotraghrāṇeti sarvāṅgaṃ spṛśan prāṇā ihāyāntu sukhaṃ ciraṃ tiṣṭhan tu iti śiraḥ spṛśan svāheti japet| mantranyāsam iti | \end{quote}} being similiar to a ruby-gem in color. Her technical designation is embodied soul (\textit{jīva}).\footnote{The idea of the thumbsized soul residing in the heart is already present in the oldest strata of yogic literature. See \citetitle{kathaup} 6.17:
\begin{quote}
aṅguṣṭhamātraḥ puruṣo 'ntarātmā\\
    sadā janānāṃ hṛdaye saṃniviṣṭaḥ |\\
taṃ svāc charīrāt pravṛhen muñjād iveṣīkāṃ dhairyeṇa |\\
taṃ vidyāc chukram amṛtaṃ taṃ vidyāc chukram amṛtam iti ||17||\end{quote}
Also cf. \citetitle{hauschild1927} 3.13.}
Not even with a thousand tongues it is possible to talk about her nature and her power. Here it is said [that]: ``Because of the exercise of meditation on this form the inhabitants of the universe [which are] Humans, Gandharvas, Kinnaras, Guhyakas, Vidyādharas and [their] females, in the heavenly world, underworld and open space are obedient to the will of the practicing person.''\textsuperscript{\coro{[\lowroman{25}]}}\vfill
\vfill
     \smallskip
     \centerline{\textrm{\small{[\uproman{8}.\textsuperscript{\coro{\lowroman{1}-\lowroman{5}}} Fifth Cakra]}}}
     \label{cakra5trans}
     \smallskip
 \indent Now the fifth sixteen petalled lotus existing at the location of the throat. \extra{
      The colour is grey. The embodied soul (\textit{jīva}) is the deity. Ignorance is the power. Virāṭ is the Ṛṣi. The wind (\textit{vāyu}) is the mount. Udāna is the vitalwind. The flame is the digit (\textit{kalā}). Jālandhara is the binding (\textit{bandha}). The primordial cause (\textit{mahākāraṇa}) is the body. The fourth state (\textit{tūrya}) is the state.}   
    \end{tlate}
  \end{translation}
  \ekdpb*{}
\end{alignment}
%%%%%%%%%%%%%%%%%%%%%%%%%%%%%%%%%%%%%%%%%%
%%%%%%%%%%%%%%%%%%%%%%%%%%%%%%%%%%%%%%%%%%
%%%%%%%%PAGEBREAK%%%%%%%PAGEBREAK%%%%%%%%%
%%%%%%%%%%%%%%%%%%%%%%%%%%%%%%%%%%%%%%%%%%
%%%%%%%%%%%%%%%%PAGEBREAK%%%%%%%%%%%%%%%%%
%%%%%%%%%%%%%%%%%%%%%%%%%%%%%%%%%%%%%%%%%%
%%%%%%%%PAGEBREAK%%%%%%%PAGEBREAK%%%%%%%%%
%%%%%%%%%%%%%%%%%%%%%%%%%%%%%%%%%%%%%%%%%%
%%%%%%%%%%%%%%%%%%%%%%%%%%%%%%%%%%%%%%%%%%
%%%%%%%%%%%%%%%%%%%%%%%%%%%%%%%%%%%%%%%%%%
%%%%%%%%%%%%%%%%%%%%%%%%%%%%%%%%%%%%%%%%%%
%%%%%%%%PAGEBREAK%%%%%%%PAGEBREAK%%%%%%%%%
%%%%%%%%%%%%%%%%%%%%%%%%%%%%%%%%%%%%%%%%%%
%%%%%%%%%%%%%%%%PAGEBREAK%%%%%%%%%%%%%%%%%
%%%%%%%%%%%%%%%%%%%%%%%%%%%%%%%%%%%%%%%%%%
%%%%%%%%PAGEBREAK%%%%%%%PAGEBREAK%%%%%%%%%
%%%%%%%%%%%%%%%%%%%%%%%%%%%%%%%%%%%%%%%%%%
%%%%%%%%%%%%%%%%%%%%%%%%%%%%%%%%%%%%%%%%%%
%%%%%%%%%%%%%%%%%%%%%%%%%%%%%%%%%%%%%%%%%%
%%%%%%%%%%%%%%%%%%%%%%%%%%%%%%%%%%%%%%%%%%
%%%%%%%%PAGEBREAK%%%%%%%PAGEBREAK%%%%%%%%%
%%%%%%%%%%%%%%%%%%%%%%%%%%%%%%%%%%%%%%%%%%
%%%%%%%%%%%%%%%%PAGEBREAK%%%%%%%%%%%%%%%%%
%%%%%%%%%%%%%%%%%%%%%%%%%%%%%%%%%%%%%%%%%%
%%%%%%%%PAGEBREAK%%%%%%%PAGEBREAK%%%%%%%%%
%%%%%%%%%%%%%%%%%%%%%%%%%%%%%%%%%%%%%%%%%%
%%%%%%%%%%%%%%%%%%%%%%%%%%%%%%%%%%%%%%%%%%
\begin{alignment}[
    texts=edition[class="edition"];
    translation[class="translation"],
]
  \begin{edition}
    \ekddiv{type=ed}
    \begin{prose}
      \noindent
\extra{parā vācā\dd{}
        \app{\lem[type=emendation, resp=egoscr]{atharvaṇo}
          \rdg[wit={U2}]{atharvaṇa}} vedaḥ\dd{}
        \app{\lem[type=emendation, resp=egoscr]{jaṅgamaṃ}
          \rdg[wit={U2}]{jaṃgama°}} liṅgaṃ\dd{}
        jīvaprāptā bhūmikā\dd{}
        sāyujyatā mokṣaḥ\dd{}
        ṣoḍaśadalāni\dd{}
        ṣoḍaśamātrāḥ\dd{}
        \app{\lem[type=emendation, resp=egoscr]{antarmātrā}
          \rdg[wit={U2}]{antarmātrār carāḥ}}\dd{}  %%%what does carā here mean? I emend to the formulation found for the U2 additions in the previous cakra 
        aṃ āṃ iṃ īṃ u ūṃ ṛṃ ṝṃ ḷṃ ḹṃ eṃ aiṃ oṃ auṃ aṃ aṃḥ\dd{}
        bahir-mātrā\dd{}
        vidyā\dd{}
        avidyā\dd{}
        \app{\lem[type=emendation, resp=egoscr]{icchā}
          \rdg[wit={U2}]{ichā}}\dd{}
        \app{\lem[type=emendation, resp=egoscr]{śaktiḥ}
          \rdg[wit={U2}]{śakti}}\dd{}
        jñānaśaktiḥ\dd{}
        śatalā\dd{}
        mahāvidyā\dd{}
        mahāmāyā\dd{}
        buddhiḥ\dd{}
        \app{\lem[type=emendation, resp=egoscr]{tāmasī}
          \rdg[wit={U2}]{tamasī}}\dd{} %%%She who is darkness????
        maitrā\dd{}
        kumārī\dd{}
        maitrāyaṇī\dd{} %%%what's this??? 
        rudrā\dd{}
        \app{\lem[type=emendation, resp=egoscr]{puṣṭā}
          \rdg[wit={U2}]{puṣṭa°}}\dd{}
        siṃhanī\dd{}
        \app{\lem[type=emendation, resp=egoscr]{ajapājapaḥ sahasraḥ}
          \rdg[wit={U2}]{ajapājapasahasra}}\dd{} 1000\dd{} gha. 2 pa. 46 akṣara 40\dd{}}\textsuperscript{\begin{otherlanguage}{english}\coro{[\lowroman{2}]}\end{otherlanguage}}
 %%%%%%%%%%%
 %%%%%%%%%%%
 %%%%%%%%%%%
 %%%%%%%%%%%
 %%%%%%%%%%%
%----------------------- 
%tanmadhye koṭisūryasamāna       ekaḥ puruṣo vartate / \E
%tanmadhye koṭicaṃdrasamaprabhaḥ ekaḥ puruṣo vartate   \P
%tanmadhye koṭicaṃdrasamaprabhā  ekaḥ puruṣo vartate   \L
%tanmadhye koṭicaṃdrasamaprabhaḥ ekaḥ puruṣo varttate  \N1
%tanmadhye koṭicaṃdrasamaprabhaḥ ekaḥ puruṣo varttate  \N2
%tanmadhye koṭicaṃdrasamaprabhā  eka--puruṣo varttate  \D
%tanmadhye koṭicaṃdrasamaprabhaḥ ekaḥ puruṣo varttate  \U1
%tanmadhye koṭicaṃdrasamaprabhaḥ // eka pumān varttate // \U2
%----------------------- 
%In its  middle exists a single person which shines like a thousand moons.
%----------------------- 
\note[type=testium, labelb=78, lem={\textbf{Cie}}]{\textit{Yogasaṃgraha} IGNCA 30020 folio 2v. ll. 7-8: tatra koṭicaṃdraprabha ekaḥ puruṣo sti}
 tanmadhye
koṭicandra\app{\lem[wit={ceteri}, alt={°samaprabhaḥ}]{samaprabhaḥ}
  \rdg[wit={U2}]{°samaprabhaḥ ||}
  \rdg[wit={L,D}]{°samaprabhā}
  \rdg[wit={E}]{°sūryasamāna}}
\app{\lem[wit={ceteri}]{ekaḥ puruṣo}
  \rdg[wit=D]{ekapuruṣo}
  \rdg[wit={U2}]{eka pumān}}
vartate/
%----------------------- 
%tasya puruṣasya dhyānakāraṇād--- asādhyarogā naśyanti // \E
%tasya puruṣasya dhyānakāraṇād--- asādhyarogā naśyanti // \L
%tasya puruṣasya dhyānakāraṇād--- asādhyarogā naśyaṃti // \P
%tasya puruṣasya dhyānakaraṇāt--  asādhyarogā naśyaṃti // \N1
%tasya puruṣasya dhyānakaraṇāt    asādhyarogā naśyaṃti    \N2
%tasya puruṣasya dhyānakaraṇāt /  asādhyarogā naśyaṃti // \D
%tasya puruṣasya dhyānakaraṇāt /  asādhyarogā naśyaṃti    \U1
%tasya puṃsaḥ    dhyānakaraṇāt // asādhyarogā naśyaṃti // \U2
%----------------------- 
%Because of the exercise of meditation on this person all diseases which are (otherwise) not possible to be controlled vanish.
%-----------------------
\note[type=source, labelb=78a, lem={\textbf{Re}}]{PT\textsuperscript{qcr \cdot YSV} (Ed. p. 832) = YK\textsuperscript{ccn \cdot YSV} 1.262 Ed. p. 21: asya madhye pumān ekaḥ koṭicandrasamaprabhaḥ | naśyantya sādhyarogā hi sahasrāyuś ca cintanāt |}
\note[type=testium, labelb=79, lem={\textbf{Cie}}]{\textit{Yogasaṃgraha} IGNCA 30020 folio 2v. l. 8: tasya puruṣasya dhyānakaraṇād asādhyarogā naśyaṃti ||}
tasya
\app{\lem[wit={ceteri}]{puruṣasya}
  \rdg[wit={U2}]{puṃsaḥ}}
\app{\lem[wit={ceteri}, alt={dhyānakāraṇād}]{dhyānakaraṇā\skp{d-a}}
  \rdg[wit={N1,N2}]{dhyānakaraṇāt}
  \rdg[wit={D,U1,U2}]{dhyānakaraṇāt |}}\skm{d-a}sādhyarogā naśyanti/
%----------------------- 
%ekasahasravarṣaparyaṃtaṃ sa puruṣo jīvatīdānīṃ     \E
%ekasahasravarṣaparyaṃtaṃ sa puruṣo jīvati          \P
%ekasahasravarṣa             puruṣo jīvati //       \L
%ekasahasravarṣaparyaṃtaṃ    puruṣo jīvati /        \N1
%ekasahasravarṣaparyaṃta     puruṣo jīvati /        \N2
%ekasahasravarṣaparyaṃtaṃ    puruṣo jīvati /        \D
%ekasahasravarṣaparyaṃtaṃ    puruṣo jīvati / cha    \U1
%ekasahasravarṣaparyaṃtaṃ    puruṣo jīvati //       \U2
%----------------------- 
%The person lives up to 1001 years.
%----------------------- 
\note[type=testium, labelb=80, lem={\textbf{Cie}}]{\textit{Yogasaṃgraha} IGNCA 30020 folio 2v. l. 8: sahasravarṣaṃ jīvati |}
ekasahasravarṣa\app{\lem[wit={ceteri},alt={°paryantaṃ}]{paryantaṃ}
  \rdg[wit={N2}]{°paryaṃta}
  \rdg[wit={L}]{\om}}
\app{\lem[wit={ceteri}]{puruṣo}
  \rdg[wit={E,P}]{sa puruṣo}}
  \app{\lem[wit={ceteri}]{jīvati}
    \rdg[wit={U1}]{jīvati |cha|}
    \rdg[wit={E}]{jīvatīdānīṃ}}\dd{}\textsuperscript{\begin{otherlanguage}{english}\coro{[\lowroman{5}]}\end{otherlanguage}}
  \note[type=testium, labelb=80a, lem={\textbf{Ri}}]{SSP 2.5 (Ed. pp. 30-31): pañcamaṃ kaṇṭhacakraṃ caturaṅgulam | tatra vāma iḍā candranāḍī | dakṣiṇe piṅgalā sūryanāḍī | tanmadhye suṣumnāṃ dhyāyet | saiva anāhatakalā anāhatasiddhidā bhavati ||2.5||} \vfill
\end{prose}
  \nolinenumbers
   \smallskip
   \centerline{\textrm{\small{\uproman{9}. Sixth Cakra]}}}
   \label{cakra6}
    \bigskip
    \linenumbers
%----------------------- 
%īdānīṃ ṣaṣṭhaṃ bhrūmadhye ājñācakraṃ                vartate//   \E
%īdānīṃ ṣaṣṭhaṃ bhrūmadhye ājñācakraṃ                vartate//   \P
%īdānīṃ ṣaṣṭhaḥ bhrūmadhye ājñācakraṃ                vartate//   \L
%idānīṃ ṣaṣṭhacakraṃ       ajñānāmakaṃ               varttate // \N1
%idānīṃ ṣaṣṭhacakraṃ       ajñānāmaka                varttate    \N2
%idānīṃ ṣaṣṭhacakraṃ       ajñānāmakaṃ               varttate // \D
%idānīṃ ṣaṣṭhacakraṃ       ājñānāmakaṃ               vartate     \U1
%idānīṃ ṣaṣṭa   bhrūmadhye ājñācakraṃ raktavarṇaṃ //             \U2
%-----------------------
%āgnirdevatā suṣumṇāśaktiḥ// hiṃsaṛṣiḥ// caitanyavāhanaṃ// jñānadehī// vijñānāvathā// anupamavācā// sāmadevaḥ// pramādaliṃgaṃ// ardhamātrā// ākāśātatvaṃ// jīvahiṃsa// caitanyalīlraṃbhaḥ// dvemātrā// hiṃkṣaṃ// aṃtarmātrā// bahirmātrā//sthiti//prabhā?// ajapājapasahasra// 1000 gha. 2 pa. 46 akṣara 40// \U2
%-----------------------
%The deity is fire. The power is the godess of the centre (\textit{suṣumṇā}). The Ṛṣi is ``the violent'' (\textit{hiṃsa}). The mount is consciousness (\textit{caitanya}. The body is knowledge. The state is understanding. The speech is the ``incomparable'' (\textit{anupama}). The [Veda] is Sāmaveda.The \textit{liṅgaṃ} is intoxication (\textit{pramāda}). The half-measure: the reality of ether, ``the violence of living'' (\textit{jīvahiṃsa}) [and] the origin of the play of Conciousness. Two measures: haṃ kṣam. The inner measure is external measure: maintenance of life (\textit{sthiti}) [and] splendour (\textit{prabhā}).
%-----------------------
    \begin{prose}
      \noindent
    \note[type=source, labelb=81, lem={\textbf{Re}}]{PT\textsuperscript{qcr \cdot YSV} (Ed. p. 832): ājñākhyaṃ ṣaṣṭhakaṃ (\textit{ṣaṭkaṃ} YK\textsuperscript{ccn \cdot YSV} 1.264 Ed. p. 21) cakraṃ bhruvor madhye dvipatrakam | agnijvālānibhaṃ jyotiḥ puṃsaḥ strīto (\textit{pūṃsastrīto} YK\textsuperscript{ccn \cdot YSV} 1.264 Ed. p. 21) vivarjitam | dhyānāc cāsya sarvasiddhirajarāmaratāṃ vrajet |}
    \note[type=testium, labelb=82, lem={\textbf{Cie}}]{\textit{Yogasaṃgraha} IGNCA 30020 folio 2v. ll. 8-9: bhrūvor madhye dvidalaṃ ājñācakraṃ ṣaṣṭhaṃ |}
      \note[type=testium, labelb=82a, lem={\textbf{Ri}}]{SSP 2.7 (Ed. p. 31): saptamaṃ bhrūcakraṃ madhyamāṅguṣṭhamatram | tatra jñānanetraṃ dīpaśikhākāraṃ dhyāyet | tatra vāksiddhir bhavati ||2.7||}
idānīṃ
    \app{\lem[wit={X}]{ṣaṣṭhacakraṃ}
       \rdg[wit={E,P}]{ṣaṣṭhaṃ bhrūmadhye}
       \rdg[wit={L}]{ṣaṣṭhaḥ bhrūmadhye}
       \rdg[wit={U2}]{ṣaṣṭa bhrūmadhye}}
     \app{\lem[wit={ceteri}]{ājñā}
      \rdg[wit={N1,N2,D}]{ajñā}
    }\app{\lem[wit={U1,D,N1}]{nāmakaṃ}
       \rdg[wit={N2}]{nāmaka}
       \rdg[wit={E,P,L}]{cakraṃ}
       \rdg[wit={U2}]{cakraṃ raktavarṇaṃ}}
 \app{\lem[wit={ceteri}]{vartate}
   \rdg[wit={U2}]{\om}}/  
  %%%%%%%%%%%%%%%
  %%%%%%%%%%%%%%
  %%%%%%%%%%%%%%
  %%%%%%%%%%%%%%
  %%%%%%%%%%%%%%
     \extra{\app{\lem[type=emendation, resp=egoscr, alt={agnir}]{agni\skp{r-de}}
         \rdg[wit={U2}]{āgnir}
       }\skm{r-de}vatā\dd{}
       suṣumṇā śaktiḥ\dd{}
       \app{\lem[type=emendation, resp=egoscr]{hiṃso}
         \rdg[wit={U2}]{hiṃsa°}} ṛṣiḥ\dd{}
       \app{\lem[type=emendation, resp=egoscr]{caitanyaṃ}
         \rdg[wit={U2}]{caitanya°}} vāhanaṃ\dd{}
       \app{\lem[type=emendation, resp=egoscr]{jñāno dehaḥ}
         \rdg[wit={U2}]{jñānadehī}}\dd{}
       vijñānāvasthā\dd{}
       \app{\lem[type=emendation, resp=egoscr]{anupamā}
         \rdg[wit={U2}]{anupama°}} vācā\dd{}
       sāmavedaḥ\dd{}
       \app{\lem[type=emendation, resp=egoscr]{pramādaḥ}
         \rdg[wit={U2}]{pramāda°}} liṃgaṃ\dd{}
       \app{\lem[type=emendation, resp=egoscr]{ardhā mātrā}
         \rdg[wit={U2}]{ardhamātrā}}\dd{}
       \app{\lem[type=emendation, resp=egoscr]{ākāśaṃ}
         \rdg[wit={U2}]{ākāśā}}tattvaṃ\dd{}
       \app{\lem[type=emendation, resp=egoscr]{jīvo haṃsaḥ}
         \rdg[wit={U2}]{jīvahiṃsa}}\dd{}
       caitanya\app{\lem[type=emendation, resp=egoscr, alt={°līlā}]{līlā āraṃbhaḥ}
         \rdg[wit={U2}]{°līlāraṃbhaḥ}}\dd{}
       dve mātrā\dd{}
       haṃ kṣaṃ\dd{}
       aṃtar-mātrā\dd{}
       bahir-mātrā\dd{}
       \app{\lem[type=emendation, resp=egoscr]{sthitiḥ}
         \rdg[wit={U2}]{sthiti}}\dd{}
       prabhā\dd{}
       \app{\lem[type=emendation, resp=egoscr]{ajapājapaḥ sahasraḥ}
         \rdg[wit={U2}]{ajapājapasahasra}}\dd{} 1000\dd{} gha. 2 pa. 46 akṣara 40\dd{}}\textsuperscript{\begin{otherlanguage}{english}\coro{[\lowroman{2}]}\end{otherlanguage}}
     \vspace*{\fill}    
 \end{prose}
    \end{edition}
  \begin{translation}
    \ekddiv{type=trans}
    \begin{tlate}
      \noindent
      \extra{Parā is the speech. Atharvaṇa[veda] is the Veda. The movable is the characteristic (\textit{liṅga}). Jīvaprāptā is the earth. The liberation is the union with the deity (\textit{sāyujyatā}). [Associated with it are] sixteen petals [and] sixteen matrices. The internal matrix: aṃ āṃ iṃ īṃ u ūṃ ṛṃ ṝṃ ḷṃ ḹṃ eṃ aiṃ oṃ auṃ aṃ aṃḥ. The external matrix: Vidyā, Avidyā, Icchā, Śakti, Jñānaśakti, Śatalā, Mahāvidyā, Mahāmayā, Buddhi, Tāmasī, Maitrā, Kumārī, Maitrāyaṇī, Rudrā, Puṣṭā, Siṃhanī. A thousandfold recitation of the non-recited; 1000; 2 \textit{ghaṭi}s, 46 \textit{palā}s. 40 \textit{akṣara}s\footnote{According to \citeauthor{birch2013} (2013) the time unit \textit{akṣara} appears in Bhāskara's \citetitle{siddharoma} (17c-d − 18a-b of the \textit{Kālamānādhyāya} in the \textit{Madhyamādhikāra}):
          \begin{quote}
            gurvakṣaraiḥ khendumitair asus taiḥ | ṣaḍbhiḥ palaṃ tair ghaṭikā khaṣaḍbhiḥ || syād vā ghaṭīṣaṣṭir ahaḥ kharāmair māso dinaistair dvikubhiś ca varṣam |
            \end{quote}
            Translation by \citeauthor[2013: p. 265, n. 46]{birch2013}: \begin{quote}
              A breath is ten long syllables, a Pala is six breaths, sixty Palas is one Ghaṭikā, sixty Ghaṭikās is a day, thirty days is a month and twelve months is a year.
            \end{quote}
            If one assumes an \textit{akṣara} to be 1/10 of a breath and 21600 breaths per day, one hour would have 900 breaths, one minute would equal 16 breaths, one breath would equal 4 seconds and one \textit{akṣara} would be 0,4 senconds or 400 milliseconds. Thus, the 10 \textit{akṣara}s given here would equal 16 seconds.}.} In its middle exists a single person shining like a thousand moons. Because of the exercise of meditation on this person, all diseases which are [otherwise] not possible to be controlled vanish. The person lives up to 1001 years.\end{tlate}
    \begin{tlate}
    \bigskip
    \centerline{\textrm{\small{[\uproman{9}.\textsuperscript{\coro{[\lowroman{1}-\lowroman{5}]}} Sixth Cakra]}}}
    \label{cakra6trans}
    \bigskip
     Now exists a sixth \textit{cakra} named Ājñā. \extra{Agni is the deity. The central channel (\textit{suṣumṇā}) is the power. Hiṃsa is the Ṛṣi. Consciousness (\textit{caitanya}) is the mount. Knowledge (\textit{vijñāna}) is the body. Understanding is the stage. The incomparable (\textit{anupama}) is the speech. Sāma[veda] is the Veda. Intoxication (\textit{pramāda}) is the characteristic (\textit{liṅgaṃ}). The half-matrix: the principle of ether. Jīva is the gander, and the play of consciousness the origin, [represent the] twofold matrix. The inner matrix: haṃ kṣam. The external matrix: Sthiti [and] Prabhā. A thousandfold recitation of the non-recited; 1000; 2 \textit{ghaṭi}s, 46 \textit{palā}s, and 40 \textit{akṣara}s.\textsuperscript{\coro{[\lowroman{2}]}}}\vspace*{\fill} 
    \end{tlate}
  \end{translation}
\end{alignment}
\ekdpb*{}
%%%%%%%%%%%%%%%%%%%%%%%%%%%%%%%%%%%%%%%%%%
%%%%%%%%%%%%%%%%%%%%%%%%%%%%%%%%%%%%%%%%%%
%%%%%%%%PAGEBREAK%%%%%%%PAGEBREAK%%%%%%%%%
%%%%%%%%%%%%%%%%%%%%%%%%%%%%%%%%%%%%%%%%%%
%%%%%%%%%%%%%%%%PAGEBREAK%%%%%%%%%%%%%%%%%
%%%%%%%%%%%%%%%%%%%%%%%%%%%%%%%%%%%%%%%%%%
%%%%%%%%PAGEBREAK%%%%%%%PAGEBREAK%%%%%%%%%
%%%%%%%%%%%%%%%%%%%%%%%%%%%%%%%%%%%%%%%%%%
%%%%%%%%%%%%%%%%%%%%%%%%%%%%%%%%%%%%%%%%%%
%%%%%%%%%%%%%%%%%%%%%%%%%%%%%%%%%%%%%%%%%%
%%%%%%%%%%%%%%%%%%%%%%%%%%%%%%%%%%%%%%%%%%
%%%%%%%%PAGEBREAK%%%%%%%PAGEBREAK%%%%%%%%%
%%%%%%%%%%%%%%%%%%%%%%%%%%%%%%%%%%%%%%%%%%
%%%%%%%%%%%%%%%%PAGEBREAK%%%%%%%%%%%%%%%%%
%%%%%%%%%%%%%%%%%%%%%%%%%%%%%%%%%%%%%%%%%%
%%%%%%%%PAGEBREAK%%%%%%%PAGEBREAK%%%%%%%%%
%%%%%%%%%%%%%%%%%%%%%%%%%%%%%%%%%%%%%%%%%%
%%%%%%%%%%%%%%%%%%%%%%%%%%%%%%%%%%%%%%%%%%
%%%%%%%%%%%%%%%%%%%%%%%%%%%%%%%%%%%%%%%%%%
%%%%%%%%%%%%%%%%%%%%%%%%%%%%%%%%%%%%%%%%%%
%%%%%%%%PAGEBREAK%%%%%%%PAGEBREAK%%%%%%%%%
%%%%%%%%%%%%%%%%%%%%%%%%%%%%%%%%%%%%%%%%%%
%%%%%%%%%%%%%%%%PAGEBREAK%%%%%%%%%%%%%%%%%
%%%%%%%%%%%%%%%%%%%%%%%%%%%%%%%%%%%%%%%%%%
%%%%%%%%PAGEBREAK%%%%%%%PAGEBREAK%%%%%%%%%
%%%%%%%%%%%%%%%%%%%%%%%%%%%%%%%%%%%%%%%%%%
%%%%%%%%%%%%%%%%%%%%%%%%%%%%%%%%%%%%%%%%%%
\begin{alignment}[
    texts=edition[class="edition"];
    translation[class="translation"],
 ]
\begin{edition}
 \ekddiv{type=ed}
 \begin{prose}
%----------------------- 
                                       %dvidalaṃ tanmadhye  'gnijvālākārakamalaṃ     kiṃcid vastu vartate/    \E
                                       %dvidalaṃ tanmadhye  agnijvālākārakamalaṃ     kiṃcid vastu vartate/    \P
                                       %dvidalaṃ tanmadhye  agnijvālākārakamalaṃ     kiṃcid vastu vartate/    \L
%                                                           agnijvālākārakamalaṃ     kiṃcid vastu vartate/    \B
%tac cakraṃ bhruvor madhye dvidalakaṃ sthitaṃ // tanmadhye  agnijvālākāraṃ akalaṃ    kiṃcid vastu varttate/   \N1
%tac-cakraṃ bhruvor-madhye dvidalakaṃ sthitaṃ /  tanmadhye  agnijvālākāraṃ akalaṃ    kiṃcid vastu vartate/    \N2
%tac cakraṃ bhruvor madhye dvidalakaṃ sthitaṃ // tanmadhye  agnijvālākāraṃ akalaṃ    kiṃcid vastu varttate/   \D
%tac-cakraṃ bhruvor-madhye dvidalakaṃ sthitaṃ    tanmadhye  agnijvālākāraṃ akala     kiṃcit vastu vartate/    \U1  
%                                                tanmadhye  agnijvālākārakamalaṃ //  kiṃcid vastu varttate/   \U2   
%-----------------------    
\noindent 
   \note[type=testium, labelb=83, lem={\textbf{Cie}}]{\textit{Yogasaṃgraha} IGNCA 30020 folio 2v. l. 9: gnijvālākāraṃ paramātmasaṃjñakaṃ vastv āsti |}
     \app{\lem[wit={X}, alt={tac cakraṃ bhruvor madhye dvidalakaṃ sthitaṃ}]{tac\skp{-}cakraṃ bhruvor-madhye dvidalakaṃ sthitaṃ}
     \rdg[wit={E,P,L}]{dvidalaṃ}
     \rdg[wit={U2}]{\om}}
tanmadhye
\app{\lem[wit={E},alt={'gnijvālā°}]{'gnijvālā}
  \rdg[wit={ceteri}]{agnijvālā°}
}\app{\lem[type=emendation, resp=egoscr,alt={kāram akalaṃ}]{kāram\skp{-}akalaṃ}
  \rdg[wit={D,N1,N2}]{°kāraṃ akalaṃ}
  \rdg[wit={Y}]{°kārakamalaṃ}
  \rdg[wit={U1}]{°kāraṃ akala}}
\note[type=philcomm, labelb=84, lem={agnijvālākāra°}]{Witness B starts here.}
   \app{\lem[wit={ceteri},alt={kiṃcid vastu}]{kiṃcid\skp{-}vastu}
     \rdg[wit={U1}]{kiṃcit vastu}} vartate/
%-----------------------  
%na strī pumān     / tasya dhyānakāraṇāt  puruṣasya  śarīraṃ  ajarāmaraṃ bhavati /     \E
%na strī pumān    // tasyā dhyānakaraṇāt  puruṣasya  śarīraṃ  ajarāmaro  bhavati /     \B
%na strī pumān    // tasyā dhyānakaraṇāt  puruṣasya  śarīraṃ  ajarāmaro  bhavati /     \L
%na strī na pumān // tasyā dhyānakaraṇāt  puruṣasya  śarīraṃ  ajarāmaro  bhavati /     \P
%na strī na pumān /  tasya dhyānakaraṇāt  puruṣasya  śarīraṃ  ajarāmaraṃ bhavati      \N1
%na strī na pumān /  tasya dhyānakaraṇāt  puruṣasya  śarīraṃ  ajarāmaraṃ bhavati //   \N2
%na strī na pumān /  tasya dhyānakaraṇāt  puruṣasya  śarīraṃ  ajarāmaraṃ bhavati      \D
%na strī na pumān    tasya dhyānakaraṇāt  puruṣasya  śarīraṃ  ajarāmaraṃ bhavati vā   \U1
%na strī na pumān /  tasya dhyānakāraṇāt/ puruṣasya--śarīram--ajarāmaraṃ bhavati /    \U2   
%-----------------------
\note[type=testium, labelb=85, lem={na strī}]{\textit{Yogasaṃgraha} IGNCA 30020 folio 2v. ll. 9-10: tac ca na strīpumān | tasya dhyānakaraṇād ajarāmaraḥ sādhako bhavati |cha|}
\app{\lem[wit={ceteri}]{na strī na pumān}
     \rdg[wit={E,B,L}]{na strī pumān}}/
  tasya dhyāna\app{\lem[wit={ceteri},alt={°karaṇāt}]{karaṇā\skp{t-pu}}
    \rdg[wit={U2}]{°karaṇāt |}
}\skm{t-pu}ruṣasya
  \app{\lem[wit={U2}, alt={śarīram ajarāmaraṃ}]{śarīram\skp{-}ajarāmaraṃ}
    \rdg[wit={D,E,N1,N2,U1}]{śarīraṃ ajarāmaraṃ}
    \rdg[wit={B,L,P}]{śarīraṃ ajarāmaro}}
   \app{\lem[wit={ceteri}]{bhavati}
     \rdg[wit={U2}]{bhavati vā}}\dd{}\textsuperscript{\begin{otherlanguage}{english}\coro{[\lowroman{5}]}\end{otherlanguage}}
   \vfill
   \end{prose}
\nolinenumbers
   \bigskip
   \centerline{\textrm{\small{[\uproman{10}.\textsuperscript{\coro{\lowroman{1}-\lowroman{11}}} Seventh Cakra]}}}
   \label{cakra7}
    \bigskip
    \linenumbers
    \begin{prose}
      \noindent
%-----------------------
% idānīṃ saptamaṃ  tālumadhye catuḥṣaṣṭidalaṃ              amṛtapūrṇaṃ vartate / \E
% idānīṃ saptamaṃ  tālumadhye catuḥṣaṣṭhidalaṃ             amṛtapūrṇaṃ vartate / \P
% idānīṃ saptamaṃ  // tāludeśe madhye catuḥṣaṣṭhidala      amṛtapūrṇaṃ vartate / \L
% idānīṃ saptamaṃ  // tāludeśe madhye catuḥṣaṣṭhidala      amṛtapūrṇaṃ vartate / \B
% idānīṃ saptamaṃ  cakraṃ     catuḥṣaṣṭhidalaṃ tālumadhye  amṛtapūrṇaṃ varttate // \N1
% idānīṃ saptamaṃ  cakraṃ     catuṣaṣṭhidalaṃ tālumadhye   amṛtapūrṇa  varttate // \N2      
% idānīṃ saptamaṃ  cakraṃ     catuḥṣaṣṭhidalaṃ tālumadhye  amṛtapūrṇaṃ varttate // \D
% idānīṃ saptamaṃ  cakraṃ     catuḥṣaṣṭhidalaṃ tālumadhye  amṛtapūrṇaṃ varttate // \U1
% idānīṃ saptamaṃ  tālumadhye catuḥṣaṣṭidalaṃ //           amṛtapūrṇaṃ vartate / \U2      
%-----------------------
% Now the seventh cakra having 64 petals and being full of nectar exists in the middle of the palate.
%-----------------------
%U2: \extra{lalāṭa maṃḍalaṃ// caṃdro devatā// amṛtā śaktiḥ// paramātmā ṛṣiḥ// amṛtavāsinīkalāsaptadaśī amṛtakallolanadī// mahākāśa// aṃbikā// laṃbikā// ghaṃṭikā// tālikā// ajapāgāyatrīdehasvarūpaṃ// kākamukhī// naranetrāgośṛṃgālalāṭabrahmapaṭhāhayagrīvā// mayūramukhā// haṃsavadaṃgāni// ajapāgāyatrīsvarūpaṃ// 
%-----------------------
      \note[type=testium, labelb=86, lem={\textbf{Cie}}]{\textit{Yogasaṃgraha} IGNCA 30020 folio 2v. l. 10: tālumadhye catuḥṣaṣṭhidalaṃ amṛtapūrṇaṃ}
      \note[type=source, labelb=87, lem={\textbf{Re}}]{PT\textsuperscript{qcr \cdot YSV} (Ed. pp. 832-833): catuḥṣaṣṭidalaṃ tālumadhye cakran tu madhyamam | pīyūṣapūrṇaṃ (\textit{pīyūṣapūrṇa°} YK\textsuperscript{ccn \cdot YSV} 1.266 Ed. p. 21) koṭīndusannibhaṃ (\textit{°sannibha°} YK\textsuperscript{ccn \cdot YSV} 1.266 Ed. p. 21) cāmṛtasthalī | tanmadhye ghaṭikāsaṃjñā karṇikā raktasannibhā | saha cendukalā tatrāmṛtadhārāṃ (\textit{tāndrā°} YK\textsuperscript{ccn \cdot YSV} 1.267 Ed. p. 21) sravaty asau | etad dhyātvāmṛtaiḥ snātvā sadā yogāt pramucyate |}
idānīṃ saptamaṃ
      \app{\lem[wit={X}]{cakraṃ catuḥṣaṣṭhidalaṃ tālumadhye}
         \rdg[wit={E,P,U2}]{tālumadhye catuḥṣaṣṭidalaṃ}
        \rdg[wit={L,B}]{tāludeśe madhye catuḥṣaṣṭhidala}}
      \app{\lem[type=emendation, resp=egoscr]{'mṛtapūrṇaṃ}
        \rdg[wit={ceteri}]{amṛtapūrṇaṃ}
        \rdg[wit={N2}]{amṛtapūrṇa}}
vartate/
  %%%%%%%%%%%%%%
  %%%%%%%%%%%%%%
  %%%%%%%%%%%%%%
  %%%%%%%%%%%%%%
      \extra{\app{\lem[type=emendation, resp=egoscr]{lalāṭaṃ}
          \rdg[wit={U2}]{lalāṭa°}} maṇḍalaṃ\dd{}
        caṃdro devatā\dd{}
        amṛtā śaktiḥ\dd{}
        paramātmā ṛṣiḥ\dd{}
        amṛtavāsinī kalāsaptadaśī\dd{}
        amṛtakallolanadī \app{\lem[type=emendation, resp=egoscr]{mahākāśā}
          \rdg[wit={U2}]{mahākāśa}}\dd{}
        aṃbikā laṃbikā\dd{}
        ghaṃṭikā tālikā\dd{}
        ajapāgāyatrī dehasvarūpaṃ\dd{}
        kākamukhī\dd{}
        naranetrā\dd{}
        gośṛṃgā\dd{}
        lalāṭabrahmapaṭhā\dd{}
        hayagrīvā\dd{}
        mayūramukhā\dd{}
        haṃsavad-aṃgāni\dd{}  
        ajapāgāyatrī svarūpaṃ\dd{}}\textsuperscript{\begin{otherlanguage}{english}\coro{[\lowroman{2}]}\end{otherlanguage}}
        %%%%%%%%%%%%%%%
        %%%%%%%%%%%%%%%
        %%%%%%%%%%%%%%%
        %%%%%%%%%%%%%%%
        %%%%%%%%%%%%%%%
%-----------------------
%adhikaśobhāyuktam-----atiśvetaṃ          tanmadhye       raktavarṇaṃ ghāṃṭikāsaṃjñaikā      karṇikā varttate / \E 
%adhikataraśobhayuktaṃ atiśvetaṃ          tanmadhye       raktavarṇaṃ ghaṭikāsaṃjñā ekā      karṇikā varttate / \P
%adhikataraśobhayuktaṃ // atiśvetaṃ //    tanmadhye       raktavarṇaṃ ghaṇikāsaṃjñā ekā ekā  karṇikā varttate / \L
%adhikataraśobhayuktaṃ // atiśvetaṃ //    tanmadhye       raktavarṇaṃ ghaṃṭikāsaṃjñā ekā ekā karṇikā varttate / \B
%adhikataraśobhayuktaṃ atiśvetaṃ          tanmadhye       raktavarṇaṃ ghaṃṭikāsaṃjñā ekā     karṇikā varttate / \N1
%adhikataraśobhāyuktaṃ  atiśvetaṃ         tanmadhye       raktavarṇa--ghaṇṭikāsaṃjñā ekā     karṇikā vartate /  \N2
%adhikataraśobhayuktaṃ atiśvetaṃ          tanmadhye       raktavarṇaṃ ghaṃṭikāsaṃjñā ekā     karṇikā varttate / \D
%adhikataraśobhayuktaṃ atiśvetaṃ          tanmadhye       raktavarṇaṃ ghaṃṭikāsaṃjñā ekā     karṇikā varttate / \U1      
%adhikataraprabhāmuktaṃ // atiśvetaṃ //   tanmadhye       raktavarṇaṃ ghaṃṭikāsaṃjñā// ekā   karṇikā varttate / \U2   
%-----------------------
%[It is] endowed with superabundant beauty, very bright [and] in its middle, red in color [is that] known as "uvula" (\textit{ghāṃṭikā}). [She] exists as a single pericarp.  
%-----------------------      
\note[type=testium, labelb=89, lem={\textbf{Cie}}]{\textit{Yogasaṃgraha} IGNCA 30020 folio 2v. l. 11: adhikataraśobhayuktaṃ atiśvetaṃ cakraṃ | tanmadhye raktavarṇaghaṃṭikāsaṃjñā varttate |}
      adhi\app{\lem[wit={ceteri},alt={°kataraśobhayuktaṃ}]{kataraśobhayuktaṃ}
        \rdg[wit={N2}]{°kataraśobhāyuktaṃ}
        \rdg[wit={E}]{°kaśobhāyuktam}
        \rdg[wit={U2}]{°kataraprabhāmuktaṃ}}\dd{}
      \app{\lem[wit={ceteri}]{atiśvetaṃ}
        \rdg[wit={L,B,U2}]{||atiśvetaṃ||}}\dd{}
        tanmadhye
        \app{\lem[wit={ceteri}]{raktavarṇaṃ}
          \rdg[wit={N2}]{raktavarṇa°}}
        \app{\lem[wit={ceteri},alt={ghaṇṭikā°}]{ghaṇṭikā}
          \rdg[wit={E}]{ghāṃṭikā°}
          \rdg[wit={P}]{ghaṭikā°}
          \rdg[wit={L}]{ghaṇikā°}}saṃjñā/
        \app{\lem[wit={ceteri}]{ekā}
          \rdg[wit={L,B}]{ekā ekā}}
          karṇikā vartate/
%-----------------------          
%tanmadhye bhūmiḥ / \E
%tanmadhye bhūmiḥ / \P
%tanmadhye bhūmiḥ / \L     
%tanmadhye bhūmiḥ / \B
%tanmadhye bhūmiḥ / \N1
%tanmadhye bhūmiḥ / \N2
%tanmadhye bhūmiḥ / \D
%tanmadhye bhūmis- / \U1
%tanmadhye bhūmi   / \U2         
%-----------------------
%In its middle is a place. 
%-----------------------        
tanmadhye
\app{\lem[wit={ceteri}]{bhūmiḥ}
     \rdg[wit={U1}]{bhūmis°}
     \rdg[wit={U2}]{bhūmi}}/\textsuperscript{\begin{otherlanguage}{english}\coro{[\lowroman{5}]}\end{otherlanguage}}
%-----------------------  
%tanmadhye prakaṭacandrakalā 'mṛtādhārā bhavati         / \E
%tanmadhye prakaṭacandrakalā 'mṛtādhārā sravati         / \P
%tanmadhye prakaṭacandrakalā 'mṛtādhārā sravaṃti        / \L
%tanmadhye prakaṭacandrakalā 'mṛtādhārā sravaṃti        / \B
%tanmadhye prakaṭacandrakalā amṛtādhārāsravaṃtī varttate/ \N1
%tanmadhye prakaṭacaṃdrakalā amṛtādhārāsravaṃtī varttate/ \N2
%tanmadhye prakaṭacandrakalā 'mṛtādhārāsravaṃtī varttate/ \D %sravantī f. Fluss Nom Sg
%tanmadhye pragaṭacaṃdrakalā amṛtadhārāsravaṃtī varttate  \U1
%tanmadhye-ṃdrakaṭaṃ caṃdrakalā amṛtadhārā sravati       /\U2       
%-----------------------
%In its middle exists a hidden digit of the moon, being a stream of nectar like a river (\textit{amṛtādhārāsravantī}. 
%-----------------------
\note[type=testium, labelb=90, lem={\textbf{Cie}}]{\textit{Yogasaṃgraha} IGNCA 30020 folio 2v. l. 11 - 2r. l.1: tanmadhye prakaṭacandrakalā amṛtādhārāsravaṃtī varttate |}
tanmadhye
       \app{\lem[wit={ceteri},alt={prakaṭa°}]{'prakaṭa}
         \rdg[wit={U1}]{pragaṭa}
         \rdg[wit={U2}]{°ṃdrakaṭaṃ}}candrakalā
       \app{\lem[wit={ceteri}]{amṛtadhārāsravantī}
         \rdg[wit={L,B}]{'mṛtādhārā sravaṃti}
         \rdg[wit={P,U2}]{'mṛtādhārā sravati}
         \rdg[wit={E}]{'mṛtādhārā bhavati}}
       \app{\lem[wit={X}]{vartate}
         \rdg[wit={Y}]{\om}}/
%-----------------------
%tasyāḥ kalāyā     dhyānakāraṇāt tasya samīpe maraṇaṃ nāyāti/     \E -> does not come near to death -> na-ā-yāti
%tasyāḥ kalāyā     dhyānakaraṇāt tasya samīpe maraṇaṃ nāyāti/     \P
%tasyāḥ karṇikāyā  dhyānakaraṇāt tasya samīpe maraṇaṃ na yāti     \L
%tasyāḥ karṇikāyā  dhyānakaraṇāt tasya samīpe maraṇaṃ na yāti     \B
%tasyāḥ kalāyāḥ    dhyānakaraṇāt tasya samīpe maraṇaṃ nāyāti      \N1
%tasyāḥ kalāyāḥ    dhyānakaraṇāt tasya samīpe maraṇaṃ nāyāti/     \N2       
%tasyāḥ kalāyāḥ    dhyānakaraṇāt tasya samīpe maraṇaṃ nāyāti      \D
%tasyāḥ kalāyā     dhyānakaraṇāt tasya samīpe maraṇaṃ nāyāti/     \U1
%tasyāḥ kalāyā     dhyānakāraṇāt// tasya samīpe maraṇaṃ na yāti/  \U2
%-----------------------
%Because of the exercise of meditation on this digit death does not come near him. 
%-----------------------
\note[type=testium, labelb=91, lem={\textbf{Cie}}]{\textit{Yogasaṃgraha} IGNCA 30020 folio 2r. l. 1: tasyāḥ kalāyā nirantaraṃ dhyānakartum maraṇaṃ}
       tasyāḥ
       \app{\lem[wit={ceteri}]{kalāyā}
         \rdg[wit={N1,N2,U1}]{kalāyāḥ} %Sandhi-mistake in apparatus in this case?
         \rdg[wit={L,B}]{karṇikāyā}}
dhyānakaraṇāt tasya samīpe maraṇaṃ
       \app{\lem[wit={ceteri}]{nāyāti}
         \rdg[wit={L,B,U2}]{na yāti}}/
%-----------------------
%nirantaradhyānād        -amṛtadhārāyāḥ sajīvo bhavati /    \E
%niraṃtaradhyānāt---------amṛtadhārā plāvanaṃ   bhavati /   \P
%niraṃtaradhyānakaraṇād   amṛtadhārā           sravati /    \L
%niraṃtaradhyānakaraṇād   amṛtadhārā           sravati /    \B
%niraṃtaradhyānakaraṇāt / amṛtadhārā           sravaṃti /   \N1
%niraṃtaradhyānakaraṇāt   amṛtadhārā            sravaṃti    \N2
%niraṃtaradhyānakaraṇāt / amṛtadhārā           sravaṃti /   \D   
%niraṃtaradhyānakaraṇāt   amṛtadhārā             sravati /  \U1
%niraṃtaradhyānakaraṇāt / amṛtadhārā plavanaṃ  bhavati /    \U2
%-----------------------
%Due to uninterrupted meditation the stream (\textit{dhārā}) of nectar flows. 
%-----------------------
nirantara\app{\lem[wit={ceteri},alt={°dhyānakaraṇād}]{dhyānakaraṇā\skm{d-a}}
        \rdg[wit={E,P}]{°dhyānād}}
      \app{\lem[wit={ceteri}, alt={amṛtadhārā}]{\skp{d-a}mṛtadhārā}
         \rdg[wit={E}]{amṛtadhārāyāḥ sajīvo}
         \rdg[wit={P}]{amṛtadhārā plāvanaṃ}
         \rdg[wit={U2}]{amṛtadhārā plavanaṃ}}
       \app{\lem[wit={L,B,U1}]{sravati}
         \rdg[wit={N1,N2,D}]{sravaṃti}
         \rdg[wit={E,P,U2}]{bhavati}}/\vfill            
 \end{prose}
\end{edition}
\begin{translation}
  \ekddiv{type=trans}
  \begin{tlate}
    This \textit{cakra} is located in the middle of the eyebrows and is two-petalled. In its middle exists a certain object being a form of blazing fire without parts, not being female not being male. Because of the exercise of meditation on it the body of the person becomes non-aging and immortal.\textsuperscript{\coro{[\lowroman{5}]}}\end{tlate}
    \begin{tlate}
    \bigskip
    \centerline{\textrm{\small{[\uproman{10}.\textsuperscript{\coro{\lowroman{1}-\lowroman{11}}} Seventh Cakra]}}}
    \label{cakra7trans}
    \bigskip
    \noindent
Now, the seventh cakra having 64 petals and being full of nectar exists in the middle of the palate. \extra{The forehead is the Maṇḍala. The moon is the deity. The nectar of immortality is the power. The supreme self is the Rṣi. The seventeenth digit is the resident with the nectar of immortality. The wavy stream of nectar is great space. The uvula is the mother. The ornament/rhythm? (\textit{tālikā}) is a small bell. The own form of the body is the unspeakable Gāyatrī, [which has] the face of a crow, the eye of a human, the horn of a cow, a forehead that is Brahmapaṭhā?, a neck like a horse, the face of a peacock [and] limbs like a goose. [This is] the specific nature of the unspeakable Gayatrī.}\textsuperscript{\coro{[\lowroman{2}]}}
It is endowed with superabundant beauty. [It is] very bright, in its middle is that which is known as uvula (\textit{ghāṃṭikā})\footnote{A similar concept of a \textit{cakra} at the uvula can already be identified in \citetitle{kubji1988} 7.85: \begin{quote}
    lalanāghaṇṭike yojya pañcamaṃ sthānam ākramet |\\
    ākramed guhyacakraṃ tu karaṇaṃ cordhvamūlakam ||
  \end{quote}} being red in colour. [It] exists as a single pericarp. In its middle is a site.\textsuperscript{\coro{[\lowroman{5}]}} In the middle of it exists the hidden digit of the moon, being a stream of nectar like a river (\textit{amṛtādhārāsravantī}). Because of the exercise of meditation on this digit death does not reach him. Due to uninterrupted meditation, the stream (\textit{dhārā}) of nectar flows. \vspace*{\fill} 
    \end{tlate}
  \end{translation}
  \ekdpb*{}
\end{alignment}
%%%%%%%%%%%%%%%%%%%%%%%%%%%%%%%%%%%%%%%%%%
%%%%%%%%%%%%%%%%%%%%%%%%%%%%%%%%%%%%%%%%%%
%%%%%%%%PAGEBREAK%%%%%%%PAGEBREAK%%%%%%%%%
%%%%%%%%%%%%%%%%%%%%%%%%%%%%%%%%%%%%%%%%%%
%%%%%%%%%%%%%%%%PAGEBREAK%%%%%%%%%%%%%%%%%
%%%%%%%%%%%%%%%%%%%%%%%%%%%%%%%%%%%%%%%%%%
%%%%%%%%PAGEBREAK%%%%%%%PAGEBREAK%%%%%%%%%
%%%%%%%%%%%%%%%%%%%%%%%%%%%%%%%%%%%%%%%%%%
%%%%%%%%%%%%%%%%%%%%%%%%%%%%%%%%%%%%%%%%%%
%%%%%%%%%%%%%%%%%%%%%%%%%%%%%%%%%%%%%%%%%%
%%%%%%%%%%%%%%%%%%%%%%%%%%%%%%%%%%%%%%%%%%
%%%%%%%%PAGEBREAK%%%%%%%PAGEBREAK%%%%%%%%%
%%%%%%%%%%%%%%%%%%%%%%%%%%%%%%%%%%%%%%%%%%
%%%%%%%%%%%%%%%%PAGEBREAK%%%%%%%%%%%%%%%%%
%%%%%%%%%%%%%%%%%%%%%%%%%%%%%%%%%%%%%%%%%%
%%%%%%%%PAGEBREAK%%%%%%%PAGEBREAK%%%%%%%%%
%%%%%%%%%%%%%%%%%%%%%%%%%%%%%%%%%%%%%%%%%%
%%%%%%%%%%%%%%%%%%%%%%%%%%%%%%%%%%%%%%%%%%
%%%%%%%%%%%%%%%%%%%%%%%%%%%%%%%%%%%%%%%%%%
%%%%%%%%%%%%%%%%%%%%%%%%%%%%%%%%%%%%%%%%%%
%%%%%%%%PAGEBREAK%%%%%%%PAGEBREAK%%%%%%%%%
%%%%%%%%%%%%%%%%%%%%%%%%%%%%%%%%%%%%%%%%%%
%%%%%%%%%%%%%%%%PAGEBREAK%%%%%%%%%%%%%%%%%
%%%%%%%%%%%%%%%%%%%%%%%%%%%%%%%%%%%%%%%%%%
%%%%%%%%PAGEBREAK%%%%%%%PAGEBREAK%%%%%%%%%
%%%%%%%%%%%%%%%%%%%%%%%%%%%%%%%%%%%%%%%%%%
%%%%%%%%%%%%%%%%%%%%%%%%%%%%%%%%%%%%%%%%%%
\begin{alignment}[
  texts=edition[class="edition"];
  translation[class="translation"],
  ]
  \begin{edition}
    \ekddiv{type=ed}
    \begin{prose}
      \noindent
%-----------------------
%tadā  yakṣam-aroga----pittajvarahṛdayadāha-śiroroga-jihvā--jaḍa-bhāvā           naśyanti / \E
%tadā     kṣayaroga----pittajvarahṛdayadāha-śiroroga-jihvā--jaḍa-bhāvān          naśyanti / \P
%tadā     kṣayaroga----pittajvarahṛdayadāha-----roga-jihvāyājaḍa-bhāvān          naśyanti / \L
%tadā     kṣayaroga----pittajvarahṛdayadāha-----roga-jihvāyājaḍa-vān             naśyanti / \B
%         kṣayarogaṃ   pittajvarahṛdayadāha-śiroroga-jihvāyājaḍa-bhāvā           naśyanti / \N1 %besser kṣayarogaṃ emendieren zu vollem Kompositum?
%         kṣayarogaṃ   pittajvarahṛdayadāha-śiroroga-jihvāyājaḍa-bhāvātā         naśyanti / \N2
%         kṣayaṃ rogaṃ pittajvarahṛdayadāha-śiroroga-jihvāyājaḍa-bhāvā           naśyanti / \D
%         kṣayaroga----pittajvarahṛdayadāha-śiroroga-jihvāyājaḍa-bhāvā           naśyanti / \U1  
%tadā     kṣayarogo----ptatti// jvara hṛdayadāha// śiroroga// jihvājaḍatā// dayo naśyanti /cha/ \U2       
%-----------------------
%Then the appearances of emaciation (\textit{kṣayaroga}), fever due to disordered bile (\textit{pittajvara), heartburn (\textit{hṛdayadāha}), head-disease (\textit{śiroroga}) and tongue insensibility (\textit{jihvājaḍa}) vanish. %!!!Krankheiten in Ayurvedabuch checken! medizinische Identifikationen!
%-----------------------
      \note[type=source, labelb=87, lem={\textbf{Re}}]{PT\textsuperscript{qcr \cdot YSV} (Ed. p. 833): unmādajvarapittādidāhaśūlādivedanāḥ (\textit{°śūnyā°} YK\textsuperscript{ccn \cdot YSV} 1.268 Ed. p. 21) | naśyanti ca śiroduḥkhaṃ jāḍyabhāvo 'pi naśyati | sadyodhyānādbhuktaviśvaṃ jihvājāḍyañ ca naśyati (last sentence \om in YK\textsuperscript{ccn \cdot YSV})}
      \note[type=testium, labelb=92, lem={\textbf{Cie}}]{\textit{Yogasaṃgraha} IGNCA 30020 folio 2r. ll. 1-2: kṣayarogaḥ pettajvarahṛdayadāhaśiro..jihvāyājaḍyaṃ ca naśyati |}
       \app{\lem[wit={Y}]{tadā}
         \rdg[wit={X}]{\om}}
       \app{\lem[type=emendation, resp=egoscr]{kṣayarogapittajvarahṛdayadāhaśirorogajihvājaḍyabhāvā}
         \rdg[wit={E}]{yakṣamarogapittajvarahṛdayadāhaśirorogajihvājaḍabhāvā}
         \rdg[wit={P}]{kṣayarogapittajvarahṛdayadāhaśirorogajihvājaḍabhāvān}
         \rdg[wit={L}]{kṣayarogapittajvarahṛdayadāharogajihvāyājaḍabhāvān}
         \rdg[wit={B}]{kṣayarogapittajvarahṛdayadāharogajihvāyājaḍavān}
         \rdg[wit={N1}]{kṣayarogaṃ pittajvarahṛdayadāhaśirorogajihvāyājaḍabhāvā}
         \rdg[wit={N2}]{kṣayarogaṃ pittajvarahṛdayadāhaśirorogajihvāyājaḍabhāvātā}
         \rdg[wit={D}]{kṣayaṃ rogaṃ pittajvarahṛdayadāhaśirorogajihvāyājaḍabhāvā}
         \rdg[wit={U1}]{kṣayarogapittajvarahṛdayadāhaśirorogajihvāyājaḍabhāvā}
         \rdg[wit={U2}]{kṣayarogoptatti || jvara hṛdayadāha || śiroroga || jihvājaḍatā || dayo}}
naśyanti/
%-----------------------       
%bhakṣitam--api   viṣan    na bādhate  / \E
%bhakṣitam--api   viṃṣa    na bādhate  / \P
%bhākṣitam--api   viṣaṃ    na bādhyate / \L
%bhākṣitamār pi   viṣaṃ    na bādhyate / \B
%bhakṣitam        viṣamapi na bādhyate / \N1
%bhakṣitaṃ        viṣamapi na bādhate  / \N2
%bhakṣitāṃ        viṣamapi na bādhyate / \D
%bhakṣitaṃ        viṣamapi na bādhyate   \U1
%bhakṣitam--api   viṣaṃ    na bādhyate / \U2       
%-----------------------       
%Also eaten venom doesn't trouble him. 
%-----------------------
         \app{\lem[wit={N2,U1}]{bhakṣitaṃ}
           \rdg[wit={N1}]{bhakṣitam}
           \rdg[wit={D}]{bhakṣitāṃ}
           \rdg[wit={E,P,L,U2}]{bhakṣitam api}
           \rdg[wit={B}]{bhākṣitamār pi}}
         \app{\lem[wit={X}, alt={viṣam api}]{viṣam\skp{-}api}
           \rdg[wit={L,B,U2}]{viṣaṃ}
           \rdg[wit={E}]{viṣan}
           \rdg[wit={P}]{viṃṣa}}
na \app{\lem[wit={E,P,N2}]{bādhate}
           \rdg[wit={ceteri}]{bādhyate}}/\textsuperscript{\begin{otherlanguage}{english}\coro{[\lowroman{10}]}\end{otherlanguage}}
%-----------------------
%yady-atra manaḥ sthiraṃ   bhavati /  \E
%yady-atra manaḥ sthiraṃ   bhavati /  \P
%yady-atramapi manasthiraṃ bhavati /  \L              %VARIANTE UNSICHER!!!WAS MEINT JÜRGEn??
%yady-atramapi manasthiraṃ bhavati /  \B
%yady-atra     manasthiraṃ bhavati /  \N1
%yadyanna      manasthiraṃ bhavati // \N2
%yadyanna      manasthiraṃ bhavati /  \D
%yadyatra      manasthiraṃ bhavati    \U1
%yadyatra      manasthiraṃ bhavati//  \U2       
%-----------------------
%If here the mind becomes stable.       
%-----------------------
         \app{\lem[wit={ceteri}]{yadyatra}
           \rdg[wit={L,B}]{yady atram api}
           \rdg[wit={N1,D}]{yadyanna}}
         \app{\lem[wit={E,P}]{manaḥ sthiraṃ}
           \rdg[wit={ceteri}]{manasthiraṃ}}
         bhavati\dd{} \vfill
         \end{prose}
\nolinenumbers
    \smallskip
    \centerline{\textrm{\small{[\uproman{11}.\textsuperscript{\coro{\lowroman{1}-\lowroman{16}}} Eighth Cakra]}}}
    \label{cakra8}
    \bigskip
    \linenumbers
    \begin{prose}
      \noindent
%-----------------------
%idānīṃ brahmarandhrasthāne 'ṣṭamaṃ śatadalaṃ cakraṃ varttate / \E
%idānīṃ brahmaraṃdhrasthāne 'ṣṭamaṃ śatadalaṃ cakraṃ vartate / \P
%idānīṃ brahmaraṃdhrasthāne aṣṭamaṃ śatadalaṃ cakraṃ vartate / \L
%idānīṃ brahmaraṃdhrasthāne aṣṭamaṃ śatadalaṃ cakraṃ vartate / \B
%idānīṃ aṣṭamacakraṃ brahmaraṃdhrasthāne śatadalaṃ   vartate / \N1
%idānīṃ aṣṭamacakraṃ brahmaraṃdhrasthāne śatadalaṃ   vartate  \N2
%idānīṃ aṣṭamacakraṃ brahmaraṃdhrasthāne śatadalaṃ   vartate / \D
%idānīṃ aṣṭamaṃ cakraṃ brahmaraṃdhrasthāne śatadalaṃ   vartate . \U1
%idānīṃ brahmaraṃdhrasthāne 'ṣṭamaṃ śatadalaṃ cakraṃ varttate // \U2
%-----------------------
%guru devatā// caitanya śaktiḥ// virāṭ ṛṣiḥ// sarvotkṛṣṭasākṣiḥ// bhūtaturyātītacaitanyātmakaṃ// sarvavarṇāḥ// sarvamātrāḥ// sarvadalāni virāṭdeha sthitāvasthā prajñāvācā sohaṃ veda anupamasthānaṃ// ajapājapasahasra/ 1000 gha 02 pa 046 akṣara 40// sarvajapasaṃkhyā// 21600// ekaviṃśatisahasrāṇiṣaṭśatāni// tathaivaca niśāhevahate// prāṇaḥ yojānātisapaṃḍitaḥ// sakāreṇa bahiryātihakāreṇaviśotpunaḥ// haṃsaḥ sohaṃ// tato maṃtraṃ jīvojapati sarvadā//    
%-----------------------
%Now exists the eigth \textit{cakra} having one hundred petals located at the aperture of Brahman.
%-----------------------
      \note[type=testium, labelb=93, lem={\textbf{Cie}}]{\textit{Yogasaṃgraha} IGNCA 30020 folio 2r. ll. 2-3: brahmaraṃdhre śatadalaṃ jālaṃdharapīṭhasaṃjñakaṃ siddhapuruṣasyānacakraṃ}
      \note[type=source, labelb=94, lem={\textbf{Re}}]{PT\textsuperscript{qcr \cdot YSV} (Ed. p. 833): brahmarandhre 'ṣṭamaṃ cakraṃ śatapatraṃ mahāprabham | jālandharaṃ nāma pīṭhaṃ etat tu parikīrttitam | siddhapuṃsaḥ (\textit{°puṃsa°} YK\textsuperscript{ccn \cdot YSV} 1.270 Ed. p. 22) sthalaṃ jñātvā agnidhūmanibhā śikhā | ādimadhyāntahīnā strīpuṃmūrtti (\textit{°mūrtir} YK\textsuperscript{ccn \cdot YSV} 1.270 Ed. p. 22) varttate parā | antajñānī (\textit{antaryāmī} YK\textsuperscript{ccn \cdot YSV} 1.271 Ed. p. 22) bhaved dhyānād ākāśe 'pi samāgamaḥ | nirantaraṃ sarvavettā ity ūccāno mahān bhavet | jaganmadhye sthito jantur jagadbādhāvivarjitaḥ |}    
idānīṃ
\app{\lem[wit={N1,N2,D}]{aṣṭamacakraṃ brahmaraṃdhrasthāne śatadalaṃ}
    \rdg[wit={U1}]{cakraṃ brahmaraṃdhrasthāne śatadalaṃ}
    \rdg[wit={E,P,U2}]{brahmarandhrasthāne 'ṣṭamaṃ śatadalaṃ cakraṃ}
    \rdg[wit={L,B}]{brahmaraṃdhrasthāne aṣṭamaṃ śatadalaṃ cakraṃ}}
  vartate/
  %%%%%%%%%%
  %%%%%%%%%%%
  %%%%%%%%%%
\extra{\app{\lem[type=emendation, resp=egoscr, alt={gurur}]{guru\skp{r-de}}
          \rdg[wit={U2}]{guru°}}\skm{r-de}vatā\dd{}
        \app{\lem[type=emendation, resp=egoscr]{caitanyaḥ}
          \rdg[wit={U2}]{caitanya°}} śaktiḥ\dd{}
        virāṭ ṛṣiḥ sarvotkṛṣṭasākṣiḥ\dd{}
        \app{\lem[type=emendation, resp=egoscr]{bhūtaturyātītaṃ}
          \rdg[wit={U2}]{bhūtaturyātīta°}} caitanyātmakaṃ\dd{}
        sarvavarṇāḥ\dd{}
        sarvamātrāḥ\dd{}
        sarvadalāni\dd{}
        virāṭ \app{\lem[type=emendation, resp=egoscr]{dehaḥ}
          \rdg[wit={U2}]{deha°}}
        sthitāvasthā\dd{} 
        prajñā vācā\dd{}
        sohaṃ \app{\lem[type=emendation, resp=egoscr]{vedaḥ}
          \rdg[wit={U2}]{veda}}\dd{}
        \app{\lem[type=emendation, resp=egoscr]{anupamaṃ}
          \rdg[wit={U2}]{anupama°}} sthānaṃ\dd{}
         \app{\lem[type=emendation, resp=egoscr]{ajapājapaḥ sahasraḥ}
          \rdg[wit={U2}]{ajapājapasahasra}}\dd{} 1000 ghaṭi 2 palā 46 akṣara 40\dd{}
        \app{\lem[type=emendation, resp=egoscr]{sarvajapaḥ}
          \rdg[wit={U2}]{sarvajapa°}} saṃkhyā\dd{}
        21600\dd{}\textsuperscript{\begin{otherlanguage}{english}\coro{[\lowroman{2}]}\end{otherlanguage}}
        ekaviṃśatisahasrāṇiṣaṭśatāni\dd{}
        tathaiva ca niśāhe vahate\dd{}
        prāṇaḥ yo jānāti sa paṃḍitaḥ\dd{}\textsuperscript{\begin{otherlanguage}{english}\coro{[\lowroman{5}]}\end{otherlanguage}} %%prāṇaḥ = m nom pl
        sakāreṇa bahir-yāti hakāreṇa viśet punaḥ\dd{}  
        haṃsaḥ sohaṃ\dd{}
        tato mantraṃ jīvo japati sarvadā\dd{}}\textsuperscript{\begin{otherlanguage}{english}\coro{[\lowroman{7}]}\end{otherlanguage}}
%The teacher is the deity. Consciousness is the power. Virāṭ is the Ṛṣi, the witness above everything. Made of consciousness is that which is associated with (\textit{bhūta°) the state beyond the fourth state. It has all colours. It has all matrices. It has all petals. The body is Virāṭ. The state is the standing still. The speech is wisdom.  The "I am that"-[expression] (\textit{sohaṃ}) is the Veda. The place is unsurpassed. A thousandfold recitation of the non-recited; 1000 [repetitions for]; 2 \textit{ghaṭi}s, 46 \textit{palā}s. and 40 \textit{akṣara}s.\footnote{It's not entirely clear what kind of measure is an \textit{akṣara}.} The count is all silent mutterings, [being] 21600. Day and night in this way it carries on. He who knows the breath is a learned person. With the sound of "sa" he exhales, with the sound of "ha" he inhales again: "I'm he, he's I". Because of that the embodied soul constantly utters the Mantra.\footnote{Add intertextual evidence.}
  %%%%%%%%%%%%%%%
  %%%%%%%%%%%%%%%
  %%%%%%%%%%%%%%%
  %%%%%%%%%%%%%%
  %%%%%%%%%%%%%%%
%----------------------
%tasya kamala----jātyadharaṇīpīṭha iti saṃjñā / \E
%tasya kamalasya jālaṃdharapīṭha iti saṃjñā / \P
%tasya kamalasya jālaṃdharapīṭha iti saṃjñā ...  \L
%tasya kamalasya jālaṃdharapīṭhasaṃjñā ...  \B
%tasya kamalasya jālaṃdharapīṭha iti saṃjñā ...  \N1
%tasya kamalasya jālaṃdharapīṭha iti saṃjñā ...  \N2
%tasya kamalasya jālaṃdharapīṭha iti saṃjñā ...  \D
%tasya kamalasya jālaṃdharapīṭha iti saṃjñā ...  \U1      
%tasya kamalasya jālaṃdharapīṭha iti saṃjñā //   \U2
%----------------------
%``The (divine) seat of  Jālaṃdhara'' is the designation of the lotus of it. 
%----------------------      
      \note[type=testium, labelb=95b, lem={\textbf{Ri}}]{SSP 2.8 (Ed. pp. 31-32): aṣṭamaṃ brahmarandhraṃ nirvāṇacakraṃ sūcikāgrabhedyam | tatra dhūmaśikhākāraṃ dhyāyet | tatra jālandharapīṭhaṃ mokṣapradaṃ bhavati |}
tasya
\app{\lem[wit={ceteri}]{kamalasya}
  \rdg[wit={E}]{kamala°}}
      \app{\lem[wit={ceteri}]{jālandharapīṭha}
        \rdg[wit={B}]{jālandharapīṭha°}
        \rdg[wit={E}]{jātyadharaṇīpīṭha}}
      \app{\lem[wit={ceteri}]{iti}
        \rdg[wit={B}]{\om}}
      \app{\lem[wit={ceteri}]{saṃjñā}
        \rdg[wit={B}]{°saṃjñā}}/
%---------------------- 
%siddhapuruṣasya sthānam / \E
%siddhapuruṣasya sthānam / \P
%siddhapuruṣasya sthānam mūrti vartate // \L                         %%% schwerer Satz -> wie soll ich hier entscheiden?! 
%siddhapuruṣasya sthānam mūrti vartate // \B %Zeilensprung
%siddhapuruṣasya sthānam // \N1
%siddhapuruṣasya sthānam // \N2
%siddhapuruṣasya sthānam // \D  
%siddhapuruṣasya sthānam    \U1
%siddhapuruṣasya sthānaṃ    \U2
%----------------------      
%[It is] the place of the accomplished person.
%----------------------
      siddha\app{\lem[wit={ceteri},alt={°puruṣasya sthānam}]{puruṣasya\skp{-}sthānaṃ}
        \rdg[wit={L,B}]{sthānam mūrti vartate}}/
    \end{prose}
  \end{edition}
  \begin{translation}
    \ekddiv{type=trans}
    \begin{tlate}
      \noindent
      Then the appearances of emaciation (\textit{kṣayaroga})\footnote{A fever which causes depletion in the body, cf.  \citetitle{ayurveda}, \citeauthor[1968: 441-442]{ayurveda}.}, bilious fever (\textit{pittajvara})\footnote{A fever due to disordered bile, cf. ibid. \citeauthor[1968: 618]{ayurveda}.}, heartburn (\textit{hṛdayadāha})\footnote{The burning sensation in the heart caused by heart disease resulting from disordered bile, cf. ibid. \citeauthor[1968: 1721]{ayurveda}.}, head-disease (\textit{śiroroga}) \footnote{The term refers to disorders of the head. When blood, fat, phlegm or vata decreases, it causes severe pain, cf. ibid. \citeauthor[1968: 1452]{ayurveda}.} and tongue insensibility (\textit{jihvājaḍya})\footnote{Stiffness or numbness of the tongue, cf. ibid. \citeauthor[1968: 1452]{ayurveda}.} vanish. Also eaten venom does not trouble him.\textsuperscript{\coro{[\lowroman{10}]}} If the mind is here, [it] becomes stable.
      \end{tlate}
    \bigskip
    \centerline{\textrm{\small{[\uproman{11}.\textsuperscript{\coro{\lowroman{1}-\lowroman{16}}} Eighth Cakra]}}}
    \label{cakra8trans}
    \bigskip
    \begin{tlate}
    Now [there] exists the eighth \textit{cakra} having one hundred petals located at the aperture of Brahman. \extra{The teacher is the deity. Consciousness is the power. Virāṭ is the Ṛṣi, the witness above everything. That which is made of consciousness is associated with the state beyond the fourth state. It has all colours. It has all matrices. It has all petals. Virāṭ is the body. Standing still is the state. Wisdom is the speech. The "I am that"-[expression] (\textit{sohaṃ}) is the Veda. Unsurpassed is the place. A thousandfold recitation of the non-recited; 1000; 2 \textit{ghaṭi}s, 46 \textit{palā}s, 40 \textit{akṣara}s. The count of all silent mutterings [per day] is 21600. In this way, it carries on day and night. He who knows the breath is a learned person.\textsuperscript{\coro{[\lowroman{5}]}} With the sound "sa", he exhales. With the sound "ha", he inhales again: "I am he, he is I". Because of that, the embodied soul constantly utters the Mantra.}\textsuperscript{\coro{[\lowroman{7}]}} ``The (divine) seat of Jālaṃdhara'' is the designation of its lotus.\footnote{Find parallels where Jālandhara is on top of the head. See for example Saubhagya Upaniṣad or SSP for a similar conception!}  [It is] the place of the accomplished person.\vspace*{\fill} 
    \end{tlate}
  \end{translation}
\end{alignment}
\ekdpb*{}
%%%%%%%%%%%%%%%%%%%%%%%%%%%%%%%%%%%%%%%%%%
%%%%%%%%%%%%%%%%%%%%%%%%%%%%%%%%%%%%%%%%%%
%%%%%%%%PAGEBREAK%%%%%%%PAGEBREAK%%%%%%%%%
%%%%%%%%%%%%%%%%%%%%%%%%%%%%%%%%%%%%%%%%%%
%%%%%%%%%%%%%%%%PAGEBREAK%%%%%%%%%%%%%%%%%
%%%%%%%%%%%%%%%%%%%%%%%%%%%%%%%%%%%%%%%%%%
%%%%%%%%PAGEBREAK%%%%%%%PAGEBREAK%%%%%%%%%
%%%%%%%%%%%%%%%%%%%%%%%%%%%%%%%%%%%%%%%%%%
%%%%%%%%%%%%%%%%%%%%%%%%%%%%%%%%%%%%%%%%%%
%%%%%%%%%%%%%%%%%%%%%%%%%%%%%%%%%%%%%%%%%%
%%%%%%%%%%%%%%%%%%%%%%%%%%%%%%%%%%%%%%%%%%
%%%%%%%%PAGEBREAK%%%%%%%PAGEBREAK%%%%%%%%%
%%%%%%%%%%%%%%%%%%%%%%%%%%%%%%%%%%%%%%%%%%
%%%%%%%%%%%%%%%%PAGEBREAK%%%%%%%%%%%%%%%%%
%%%%%%%%%%%%%%%%%%%%%%%%%%%%%%%%%%%%%%%%%%
%%%%%%%%PAGEBREAK%%%%%%%PAGEBREAK%%%%%%%%%
%%%%%%%%%%%%%%%%%%%%%%%%%%%%%%%%%%%%%%%%%%
%%%%%%%%%%%%%%%%%%%%%%%%%%%%%%%%%%%%%%%%%%
%%%%%%%%%%%%%%%%%%%%%%%%%%%%%%%%%%%%%%%%%%
%%%%%%%%%%%%%%%%%%%%%%%%%%%%%%%%%%%%%%%%%%
%%%%%%%%PAGEBREAK%%%%%%%PAGEBREAK%%%%%%%%%
%%%%%%%%%%%%%%%%%%%%%%%%%%%%%%%%%%%%%%%%%%
%%%%%%%%%%%%%%%%PAGEBREAK%%%%%%%%%%%%%%%%%
%%%%%%%%%%%%%%%%%%%%%%%%%%%%%%%%%%%%%%%%%%
%%%%%%%%PAGEBREAK%%%%%%%PAGEBREAK%%%%%%%%%
%%%%%%%%%%%%%%%%%%%%%%%%%%%%%%%%%%%%%%%%%%
%%%%%%%%%%%%%%%%%%%%%%%%%%%%%%%%%%%%%%%%%%
\begin{alignment}[
  texts=edition[class="edition"];
  translation[class="translation"],
  ]
  \begin{edition}
    \ekddiv{type=ed}
    \begin{prose}    
%----------------------
%tanmadhye    'gnidhūmākārarekhā     yādṛśy    ādṛśy ekā  puruṣasya mūrttir varttate /  \E
%tanmadhye    'gnidhūmākārarekhā     yādṛśī   tādṛśy ekā  puruṣasya mūrttir varttate /  \P
%tanmadhye    'gnidhūmākārārekhā     yādṛśī   tādṛśy ekā  puruṣasya mūrttir varttate /  \L               
%tanmadhye    'gnidhūmākārārekhā     yādṛśī   tādṛśy ekā  puruṣasya mūrttir varttate /  \B     
%tanmadhye    'gnidhūmākārāreṣā      yādṛśī   tādṛśī ekā  puruṣasya mūrttir varttate /  \N1
%tanmadhye    agnidhūmrākārarekhā    yādṛśī / tādṛśī ekā  puruṣasya mūrttir varttate /  \N2
%tanmadhye    agnidhūmākārāreṣā      yādṛśī   tādṛśī ekā  puruṣasya mūrttir varttate /  \D
%tanmadhye    agnidhūmrākārārekhā    yādṛśī   tādṛśī ekā  puruṣasya mūrtir  vartate     \U1
%tanmadhye    'gnidhūmrākārārekhāyāḥ  etādṛśī         ekā  puruṣasya mūrtir  vartate // \U2
%----------------------      
%In its middle [is] something like a streak having the form of smoke and fire. Such a single [divine] form of the person (\textit{puruṣa}) exists [there].        
%---------------------      
\noindent
    \note[type=testium, labelb=97, lem={\textbf{Cie}}]{\textit{Yogasaṃgraha} IGNCA 30020 folio 2r. l. 3: tanmadhye gnidhūmrāreṣākārā ādimadhyaṃtarahitā puruṣasya mūrttir asti |}
tanmadhye \app{\lem[wit={Y}]{'gnidhūmākārarekhā}
        \rdg[wit={U2}]{'gnidhūmrākārārekhāyāḥ}
        \rdg[wit={N1,D}]{'gnidhūmākārāreṣā}
        \rdg[wit={N2,U1}]{agnidhūmrākārarekhā}}
      \app{\lem[wit={ceteri}]{yādṛśī}
        \rdg[wit={E}]{yādṛśy°}
        \rdg[wit={U2}]{etādṛśī}}/\textsuperscript{\begin{otherlanguage}{english}\coro{[\lowroman{10}]}\end{otherlanguage}}
      \app{\lem[wit={P,L,B},alt={yādṛśy}]{yādṛ\skp{śy-e}}
        \rdg[wit={E}]{ādṛsy}
        \rdg[wit={X}]{yādṛśī}
        \rdg[wit={U2}]{\om}}\skm{śy-e}kā puruṣasya mūrtir-vartate/
%---------------------
%tasyā  nādir nāṃto 'sti / \E
%tasyā  nādināṃ 'to sti / \P
%tasyā  nādir nāṃto sti / \L -> vor dem bei allen anderen vorigen Satz!?!?!?! 
%tasyā  nādir nāṃto sti / \B -> vor dem bei allen anderen vorigen Satz!?!?!?! 
%tasyāḥ nāsty aṃtaḥ ādir-api nāsti / \N1????
%tasyāḥ nāsty aṃtaḥ ādir-api nāsti / \N2
%tasyāḥ nāsty aṃtaḥ ādir api nāsti / \D
%tasyāḥ nāsty aṃtaḥ ādir-api nāsti    \U1
%tasyā  nādir naṃto sti              \U2
%---------------------
% Of her exists no end, nor a beginning.
%---------------------      
\app{\lem[wit={Y}]{tasyā}
  \rdg[wit={X}]{tasyāḥ}}
\app{\lem[alt={nādir nānto 'sti}, wit={ceteri}]{nādir-nānto 'sti}
        \rdg[wit={P}]{nādināṃ 'to sti}
        \rdg[wit={X}]{nāsty aṃtaḥ ādir api nāsti}}/
%---------------------    
%tasyā  mūrtter dhyānakāraṇāt pratyakṣaṃ niraṃtaraṃ  puruṣasyākāśe   gamāgamau   bhavataḥ / \E
%tasyā  mūrtter dhyānakaraṇāt pratyakṣaniraṃtaraṃ    puruṣasyākāśe   gamāgamau   bhavataḥ / \P
%tasyā  mūrtir  dhyānakaraṇāt pratyakṣaniraṃtaraṃ    puruṣasyākāśe   gamāgamau   bhavataḥ / \L         
%tasyā  mūrtir  dhyānakaraṇāt pratyakṣaṃ niraṃtaraṃ  puruṣasyākāśe   gamāgamau   bhavataḥ / \B
%tasyāḥ mūrttir dhyānakaraṇāt pratyakṣaniraṃtaraṃ    puruṣasya ākāśe gamāgamau   bhavataḥ / \N1
%tasyāḥ mūrttir dhyānakaraṇāt pratyakṣaniraṃtaraṃ    puruṣa ākāśe    gamāgame    bhavataḥ / \N2
%tasyāḥ mūrtir  dhyānakaraṇāt pratyakṣaniraṃtaraṃ    puruṣasya ākāśe gamāgamau   bhavataḥ / \D
%tasyāḥ mūrter  dhyānakaraṇāt/ pratyakṣaniraṃtaraṃ   puruṣasya ākāśi gamāmamo   bhavataḥ   \U1
%tasyāḥ mūrter  dhyānakaraṇāt pratyakṣaniraṃtaraṃ    puruṣasyākāśa---gamāgamau bhavata //      \U2
%---------------------    
%BEDEUTUNG DES SATZES BIS JETZT UNKLAR! Idee: Zeilensprung aus übernächstem Satz! Streiche pratyakṣaṃ niraṃtaraṃ und der Satz ergibt Sinn!  
%gamāgamau nom.  dual = coming and going ; bhavataḥ = 3p du ind pres von bhū
%Due to the exercise of meditation on this (divine) form both coming and going of the person in space occurs. 
%Kolloquium: Meinung zu Kompositum pratyakṣaniraṃtaraṃ = macht wenig Sinn oder?
%{\englishnote{\small Even though every single witness at hand transmits the latter reading right after \textit{°karaṇāt}, several considerations make it reasonable to conject that the original sentence is corrupted and was written without it. The main consideration to assume the corruption is that \textit{pratyakṣaṃ nirantaraṃ} is ungrammatical. The second is that the sentence is way more meaningful without it. The third that two sentences later we get the phrase in a meaningful context. Due to the last consideration my best guess is an interlace at an early stage of transmission.}}
%---------------------
\note[type=testium, labelb=98, lem={\textbf{Cie}}]{\textit{Yogasaṃgraha} IGNCA 30020 folio 2r. l. 4: tasyāḥ dhyānakartuḥ}
      \app{\lem[wit={B,E,L,P}]{tasyā}
  \rdg[wit={ceteri}]{tasyāḥ}}
\app{\lem[alt={mūrter},wit={E,P,U1,U2}]{mūrte\skp{r-dhyā}}
  \rdg[wit={B,D,L,N1,N2}]{mūrtir}}
\app{\lem[wit={ceteri}]{dhyānakaraṇāt pratyakṣaniraṃtaraṃ}
  \rdg[wit={B,E}]{dhyānakāraṇāt pratyakṣaṃ niraṃtaraṃ}} 
      \app{\lem[wit={ceteri}]{puruṣasyākāśe}
        \rdg[wit={N2}]{puruṣa ākāśe}
        \rdg[wit={U2}]{puruṣasyākāśa°}
        \rdg[wit={U1}]{puruṣasya ākāśi}}
      gamā\app{\lem[wit={ceteri},alt={°gamau}]{gamau}
        \rdg[wit={U1}]{°gamo}
        \rdg[wit={N2}]{°game}}
        \app{\lem[wit={ceteri}]{bhavataḥ}
          \rdg[wit={U2}]{bhavata}}/
%---------------------     
%pṛthvīmadhye  sthitasyāpi    pṛthvī-------bādho   na bhavati / \E
%pṛthvīmadhye  sthitasyāpi    pṛthaka                 bhavati   \P %Zeilenspringer führt zu Verlust von Zeile in Pune
%pṛthvīmadhye  sthitasyāpi    pṛthvī-------bādho   na bhavati / \L
%pṛthivīmadhye sthitasyāpi // pṛtvī--------bādho   na bhavati // \B
%pṛthvīmadhye  sthitāv-api    pṛthvī kṣato bādho   na bhavati // \N1
%pṛthvīmadhye  sthitāv-api    pṛthvī kṣato bādho   na bhavati // \N2      
%pṛthvīmadhye  sthitāv-api    pṛthvī kṣato bādho   na bhavati // \D
%pṛthvīmadhye  sthitāv-api    pṛthvī kṣato bādho   na bhavati     \U1
%pṛthīvīmadhye sthitasyāpi    pṛthvī       bādhoko na bhati     \U2
%---------------------
%Affliction from the earth-element does not arise [anymore] even if one is situated in the middle of the earth.        
%---------------------
\note[type=testium, labelb=100, lem={\textbf{Cie}}]{\textit{Yogasaṃgraha} IGNCA 30020 folio 2r. ll. 4-5: pṛthivyāṃ sthitāv api pṛthvī kṛtabādho na bhavati | tri?kālikajñānaṃ pratyakṣaṃ bhavati | āyurvṛddiḥ liṃgaśarīreṇa sarvatra pratyakṣaṃ gamāgamo bhavati |} 
        \app{\lem[wit={ceteri}]{pṛthvīmadhye}
          \rdg[wit={B,U2}]{pṛtivīmadhye}}
        \app{\lem[wit={ceteri}]{sthitasyāpi}     
          \rdg[wit={Y},alt={sthitāv api}]{sthitāv\skp{-}api}}
        \app{\lem[wit={E,L}]{pṛthvībādho}
          \rdg[wit={B}]{pṛtvībādho}
          \rdg[wit={P}]{pṛthaka}
          \rdg[wit={U2}]{pṛthvī bādhoko}
          \rdg[wit={X}]{pṛthvī kṣato bādho}}
        \app{\lem[wit={ceteri}]{na bhavati}
          \rdg[wit={P}]{bhavati}}/
%---------------------
%sakalān pratyakṣaṃ niraṃtaraṃ paśyati ca pṛthagbhavati / \E
% \om                                                       \P      
%sakalāḥ pratyakṣaṃ niraṃtara paśyatī  ca pṛthak bhavati // \B
%sakalāḥ pratyakṣaṃ niraṃtara paśyatī  ca pṛthak bhavati / \L
%sakalāpratyakṣaniraṃtaraṃ    paśyati  ca pṛthak ca bhavati // \N1
%sakalapratyakṣaniraṃtaraṃ    paśyati  ca pṛthak ca bhavati    \N2      
%sakalāpratyakṣaniraṃtaraṃ    paśyati  ca pṛthak pṛthak bhavati \D      
%sakalāpratyakṣaniraṃtaraṃ    paśyati  ca/ pṛthak ca bhavati // \U1
%\om                                                     \U2
%---------------------
%He constantly sees everything in front of his eyes and he becomes separated (from the material world).
%---------------------
        \app{\lem[type=emendation, resp=egoscr]{sakalaṃ pratyakṣaṃ nirantaraṃ}
          \rdg[wit={X}]{sakalāpratyakṣaṃ nirantaraṃ}
          \rdg[wit={B,L}]{sakalāḥ pratyakṣaṃ niraṃtara}
          \rdg[wit={E}]{sakalān pratyakṣaṃ niraṃtaraṃ}
          \rdg[wit={P,U2}]{\om}}
        \app{\lem[wit={ceteri}]{paśyati}
          \rdg[wit={L,B}]{paśyatī}
          \rdg[wit={P,U2}]{\om}}
        \app{\lem[wit={E}]{pṛthagbhavati}
          \rdg[wit={B,L}]{ca pṛthak bhavati}
          \rdg[wit={N1,N2,U1}]{ca pṛthak ca bhavati}
          \rdg[wit={D}]{ca pṛthak pṛthak bhavati}
          \rdg[wit={P,U2}]{\om}}/\textsuperscript{\begin{otherlanguage}{english}\coro{[\lowroman{15}]}\end{otherlanguage}}
%---------------------
%atiśayenāyur vardhate /   \E
%atiśayenāyur vardhate     \P      
%atīśayanāyur vardhayate / \B
%atīśayanāyur vardhayate // \L
%atiśayena āyur varddhate // \N1
%atiśayena āyur varddhate // \N2     
%atiśayena āyur varddhate // \D
%atiśayena āyur varddhate // \U1
%\om                         \U2
%---------------------
% The force of life increases eminently. 
%---------------------
        \app{\lem[alt={atiśayenāyur},wit={E,P}]{atiśayenāyu\skp{r-va}}
          \rdg[wit={B,L}]{atīśayanāyur}
          \rdg[wit={X}]{atiśayena āyur}
          \rdg[wit={U2}]{\om}}\app{\lem[alt={vardhate},wit={ceteri}]{\skm{r-va}rdhate}
          \rdg[wit={B,L}]{vardhayate}}\dd{}
          \vfill
        \end{prose}
        \nolinenumbers
        \smallskip
        \centerline{\textrm{\small{[\uproman{12}.\textsuperscript{\coro{\lowroman{1}-\lowroman{22}}} Ninth Cakra]}}}
        \label{cakra9}
    \bigskip
    \linenumbers
    \begin{prose}
      \noindent
%---------------------
%idānīṃ navamacakrasya   bhedāḥ kathyante /  \E
%idānīṃ navamacakrasya   bhedāḥ kathyante /  \P
%idānīṃ navamacakrasya   bhedāḥ kathyate     \L
%idānīṃ navamaṃ cakrasya bhedāḥ kathyate //  \B
%idānīṃ navamacakrasya   bhedāḥ kathyaṃte // \N1
%idānīṃ navamacakrasya   bheda  kathyate  // \N2
%idānīṃ navamacakrasya   bhedāḥ kathyaṃte // \D
%idānīṃ navamaś cakrasya bhedāḥ kathyaṃte    \U1   
%idānīṃ navamacakrasya   bhedaḥ kathyate /   \U2
%---------------------
%Now the divisions/differentiations of the ninth cakra are explained.
%---------------------
\note[type=testium, labelb=101, lem={\textbf{Cie}}]{\textit{Yogasaṃgraha} IGNCA 30020 folio 2r. ll. 6-11:  brahmaraṃdhre eva śatadalacakropari mahāśūnyacakraṃ mahāsiddhacakraṃ pūrṇagiricakraṃ iti saṃjñakaṃ sahasradalaṃ cakraṃ asti | tad upari kiṃcin nāsti | tac cakraṃ atiraktaṃ ūrdhvamukhaṃ sakalaśobhāspadaṃ anekakalyāṇapūrṇaṃ mano vācā ma gocara parimalo petaṃ | tat kamalamadhye trikoṇākarṇikā |}
\note[type=source, labelb=102, lem={\textbf{Re}}]{PT\textsuperscript{qcr \cdot YSV} (Ed. p. 833): navaman tu mahāśūnyaṃ cakran tu tatparātparam | tad upari paraṃ kiñcin nāsti kiñcin mahāparam | mahācakraṃ siddhacakraṃ pūrṇagauryādisaṃjñakam | tanmadhye varttate padmaṃ sahasradalam adbhutam |}
\note[type=testium, labelb=102a, lem={\textbf{Ri}}]{SSP 2.9 (Ed. pp. 35): navamam ākāśacakraṃ soḍaśadalakamalam ūrdhvamukham | tanmadhye karṇikāyāṃ trikūṭākārāṃ tad ūrdhvaśaktiṃ tāṃ paramaśunyāṃ dhyāyet | tatraiva pūrṇagiripīṭhaṃ sarveṣṭasiddhir bhavati |}
idānīṃ
\app{\lem[wit={ceteri},alt={°navama}]{navama}
  \rdg[wit={B}]{navamaṃ}
  \rdg[wit={U1}]{navamaś°}}cakrasya
\app{\lem[wit={ceteri}]{bhedāḥ}
  \rdg[wit={N2}]{bheda}}
\app{\lem[wit={ceteri}]{kathyante}
  \rdg[wit={L,B,N2,U2}]{kathyate}}/
%------------------------------
%tasya mahāśūnyacakram    iti  saṃjñā /  \E
%tasya mahāśūnyacakram    iti  saṃjñā /  \P
%tasya mahāśūnye cakram   iti  saṃjñā    \L
%tasye mahāśūnye cakram   iti  saṃjñā    \B
%tasya mahāśūnye cakreti       saṃjñā // \N1
%tasya mahāśūnyacakreti        saṃjñā // \N2
%tasya mahāśūnyacakreti        saṃjñā // \D
%tasya mahāśūnyacakreti        saṃjñā /  \U1
%\om /                                   \U2
%---------------------
%The designation of it is ``the \textit{cakra} of the great void (\textit{mahāśūnyacakra})''.
%------------------------------
tasya \app{\lem[wit={ceteri}, alt={mahāśūnya°}]{mahāśūnya}
  \rdg[wit={L,B,N1}]{mahāśūnye}
  \rdg[wit={U2}]{\om}
}\app{\lem[wit={X},alt={°cakreti}]{cakreti}
  \rdg[wit={E,P}]{°cakram iti}
  \rdg[wit={L,B}]{cakram iti}
  \rdg[wit={U2}]{\om}}
\app{\lem[wit={ceteri}]{saṃjñā}
  \rdg[wit={U2}]{\om}}/
%------------------------------
%tadupary aparaṃ kimapi nāsti / \E
%tadupary aparaṃ kimapi nāsti \P
%tadupary        kimapi nāsti \B ??-> auch mögliche Lesart
%tadupari        kimapi nāsti \L
%tadupari aparaṃ kiṃapi nāsti / \N1
%tadupari aparaṃ kiṃapi nāsti / \N2
%tadupari aparaṃ kiṃapi nāsti / \D
%tadupari aparaṃ kiṃapi nāsti   \U1
% \om                           \U2
%---------------------
%kim api: somewhat, to a considerable extent, rather, much more, still, further. Śa
%---------------------
%Above that there is no other. 
%---------------------
\app{\lem[wit={E,P,B},alt={tad upary}]{tad\skp{-}upar\skm{y-a}}
  \rdg[wit={ceteri}]{tad upari}
  \rdg[wit={U2}]{\om}}\app{\lem[wit={ceteri}, alt={aparaṃ}]{\skp{y-a}paraṃ}
  \rdg[wit={B,L,U2}]{\om}}
\app{\lem[wit={ceteri}]{kimapi}
  \rdg[wit={X}]{kiṃ api}
  \rdg[wit={U2}]{\om}} nāsti/
%------------------------------
%tadeva-mahāsiddhacakraṃ kathyate // \E
%tadeva-mahāsiddhacakraṃ kathyate    \P 
%tadeva-mahāsiddhacakraṃ kathyate // \B
%tadeva-mahāsiddhacakraṃ kathyate // \L
%tadeva-mahāsiddhacakraṃ kathyate // \N1
%tadeva-mahāsiddhacakraṃ kathyate // \N2
%tadeva-mahāsiddhacakraṃ kathyate // \D
%tadeva-mahāsiddhacakraṃ kathyate /  \U1
% \om                                \U2
%---------------------
%Therefore it is declared to be the \textit{cakra} of the great perfection (\textit{mahāsiddhacakra}).
%---------------------
tad-eva mahāsiddhacakraṃ kathyate/
%------------------------------
%       tasya           pūrṇagiripīṭha               etadṛśaṃ nāma /  \E 
%       tasya           pūrṇagiripīṭham-iti          etādṛśaṃ nāma    \P
%       tasya           pūrṇagiripīṭham-iti saṃjñā   etādṛsaṃ nāma    \B ->!!! 
%       tasya           pūrṇagiripīṭham-iti saṃjñā   etādṛsaṃ nāma    \L
%       tasya cakrasya  pūrṇagiri                    etādṛśaṃ nāma /  \N1
%       tasya cakrasya  pūrṇagiri                    etādṛśaṃ nāma /  \N2
%       tasya cakrasya  pūrṇagiri                    etādṛśaṃ nāma /  \D
%       tasya cakrasya  pūrṇagire                    etādṛśaṃ nāmaḥ   \U1
%madhye tasya           pūrṇagiripīṭham-iti          ekādaśaṃ nāma // \U2   
%-----------------------------
%Such a name of it is ``(divine) seat of Pūrṇagiri''.   
%------------------------------
\app{\lem[wit={ceteri}]{tasya}
  \rdg[wit={X}]{tasya cakrasya}
  \rdg[wit={U2}]{madhye tasya}}
pūrṇagiri\app{\lem[wit={P,B,L,U2}, alt={°pīṭham}]{pīṭha\skm{m-i}}
  \rdg[wit={E}]{pīṭha}
  \rdg[wit={ceteri}]{\om}
}\app{\lem[wit={P,U2},alt={iti}]{\skp{m-i}ti}
  \rdg[wit={B,L}]{iti saṃjñā}
  \rdg[wit={ceteri}]{\om}}
\app{\lem[wit={ceteri}]{etādṛśaṃ}
  \rdg[wit={E}]{etadṛśaṃ}
  \rdg[wit={U2}]{ekādaśaṃ}}
\app{\lem[wit={ceteri}]{nāma}
  \rdg[wit={U1}]{nāmaḥ}}/\textsuperscript{\begin{otherlanguage}{english}\coro{[\lowroman{5}]}\end{otherlanguage}}
%------------------------------
%tasya mahāśūnyacakrasya madhye ūrdhvamukham iti raktavarṇaṃ sakalaśobhāspadam    \E
%tasya mahāśūnyacakrasya madhye ūrdhvamukham iti raktavarṇa--sakalaśobhāspadaṃ     \P
%tasya mahāśūnyacakrasya madhye ūrdhvamukhem iti raktavarṇaṃ sakalaśobhāspadaṃ // \B    
%tasya mahāśūnyacakrasya madhye ūrdhvamukham iti raktavarṇaṃ sakalaśobhāspadaṃ // \L
%tasya mahāśūnyacakramadhye     ūrdhvamukhaṃ atiraktavarṇaṃ  sakalaśobhāspadaṃ /   \N1 ->!!!
%tasya mahāśūnyacakramadhye     ūrdhvamukhaṃ atiraktavarṇaṃ  sakalaśobhāspadaṃ     \N2
%tasya mahāśūnyacakramadhye     ūrdhvamukhaṃ atiraktavarṇaṃ  sakalaśobhāspadaṃ /   \D
%tasya mahāśūnyacakramadhye     ūrdhvamukhaṃ atiraktavarṇaṃ  sakalaśobhāspadaṃ     \U1
%tasya mahāśūnyacakrasya        urdhvamukham-ativarṇaṃ       sakalaśobhanāsyadaṃ / \U2 
%------------------------------
%anekakalyāṇapūrṇaṃ sahasradalan      ekaṃ kamalaṃ  varttate / \E
%anekakalyāṇapūrṇaṃ sahasradalaṃ      ekaṃ kamalaṃ  vartate    \P
%anekakalyāṇapūrṇa--sahasradalaṃ      ekaṃ kamalaṃ  vartato    \B
%anekakalyāṇapūrṇaṃ sahasradalaṃ      ekaṃ kamalaṃ  vartate    \L
%anekakalyāṇapūrṇaṃ sahasradalaṃ      eka--kamalaṃ  varttate   \D
%anekakalyāṇapūrṇaṃ sahasradalaṃ      ekaṃ kamalaṃ  vartate    \N1
%anekakalyāṇapūrṇa--sahasradalaṃ      ekaṃ kamalaṃ  varttate    \N2
%anekakalyāṇapūrṇaṃ sahasradalaṃ           kamalaṃ  vartate /   \U1
%anekakalyāṇapūrṇaṃ // sahasradalaṃ   ekaṃ kamalaṃ  vartate / \U2
%Fragezeichen in |nepal ... schreiber Einfügung? 
%------------------------------
%In the middle of the \textit{mahāśūnyacakra} exists one lotus facing upward, very red in color with a thousand petals - an abode of brilliance and wholeness.
%------------------------------
tasya mahāśūnya\app{\lem[wit={X},alt={°cakramadhye}]{cakramadhye}
  \rdg[wit={E,P,B,L}]{°cakrasya madhye}
  \rdg[wit={U2}]{°cakrasya}}
\app{\lem[wit={X},alt={ūrdhvamukham}]{ūrdhvamukha\skp{m-a}}
  \rdg[wit={E,P,L}]{ūrdhmukham}
  \rdg[wit={U2}]{urdhvamukham}
  \rdg[wit={B}]{ūrdhvamukhem}}
\app{\lem[wit={X}]{\skm{m-a}tiraktavarṇaṃ}
  \rdg[wit={E,L,B}]{iti raktavarṇaṃ}
  \rdg[wit={P}]{iti raktavarṇa°}
  \rdg[wit={U2}]{ativarṇaṃ}}
sakala\app{\lem[wit={ceteri},alt={°śobhāspadaṃ}]{śobhāspadaṃ}
  \rdg[wit={E}]{°śobhāspadam}
  \rdg[wit={U2}]{°śobhanāsyadaṃ}}
\app{\lem[wit={ceteri}]{anekakalyāṇapūrṇaṃ}
  \rdg[wit={B,N2}]{°pūrṇa°}}
sahasradalaṃ
\app{\lem[wit={ceteri}]{ekaṃ}
  \rdg[wit={D}]{eka°}
  \rdg[wit={U1}]{\om}}
kamalaṃ
\app{\lem[wit={ceteri}]{vartate}
  \rdg[wit={B}]{vartato}}/
\end{prose}
  \end{edition}
  \begin{translation}
    \ekddiv{type=trans}
    \begin{tlate}
      \noindent
In its middle [is] something like a streak having the form of smoke and fire. Such a single [divine] form of the person (\textit{puruṣa}) exists [there].\textsuperscript{\coro{[\lowroman{10}-\lowroman{11}]}} Of her exists no end nor a beginning. Due to meditation on the form, uninterrupted observation of both the coming and going of the soul in space occurs. Affliction from the earth-element does not arise [anymore] even if one is in the middle of the earth. He constantly sees everything in front of his eyes and becomes separated [from the material world?].\textsuperscript{\coro{[\lowroman{15}]}} The force of life increases eminently. 
      \end{tlate}
    \bigskip
    \centerline{\textrm{\small{[\uproman{12}.\textsuperscript{\coro{\lowroman{1}-\lowroman{22}}} Ninth Cakra]}}}
    \label{cakra9trans}
    \bigskip
    \begin{tlate}
      Now the divisions of the ninth \textit{cakra} are explained. The designation of it is ``the \textit{cakra} of the great void''. Above that, there is no other. Therefore it is declared to be the \textit{cakra} of the great perfection. [Another] such name is ``(divine) seat of Pūrṇagiri''.\textsuperscript{\coro{[\lowroman{5}]}} In the middle of the \textit{mahāśūnyacakra} exists one lotus facing upward, very red, with a thousand petals - an abode of brilliance and wholeness.
      \vspace*{\fill} 
    \end{tlate}
  \end{translation}
     \ekdpb*{}
\end{alignment}
%%%%%%%%%%%%%%%%%%%%%%%%%%%%%%%%%%%%%%%%%%
%%%%%%%%%%%%%%%%%%%%%%%%%%%%%%%%%%%%%%%%%%
%%%%%%%%PAGEBREAK%%%%%%%PAGEBREAK%%%%%%%%%
%%%%%%%%%%%%%%%%%%%%%%%%%%%%%%%%%%%%%%%%%%
%%%%%%%%%%%%%%%%PAGEBREAK%%%%%%%%%%%%%%%%%
%%%%%%%%%%%%%%%%%%%%%%%%%%%%%%%%%%%%%%%%%%
%%%%%%%%PAGEBREAK%%%%%%%PAGEBREAK%%%%%%%%%
%%%%%%%%%%%%%%%%%%%%%%%%%%%%%%%%%%%%%%%%%%
%%%%%%%%%%%%%%%%%%%%%%%%%%%%%%%%%%%%%%%%%%
%%%%%%%%%%%%%%%%%%%%%%%%%%%%%%%%%%%%%%%%%%
%%%%%%%%%%%%%%%%%%%%%%%%%%%%%%%%%%%%%%%%%%
%%%%%%%%PAGEBREAK%%%%%%%PAGEBREAK%%%%%%%%%
%%%%%%%%%%%%%%%%%%%%%%%%%%%%%%%%%%%%%%%%%%
%%%%%%%%%%%%%%%%PAGEBREAK%%%%%%%%%%%%%%%%%
%%%%%%%%%%%%%%%%%%%%%%%%%%%%%%%%%%%%%%%%%%
%%%%%%%%PAGEBREAK%%%%%%%PAGEBREAK%%%%%%%%%
%%%%%%%%%%%%%%%%%%%%%%%%%%%%%%%%%%%%%%%%%%
%%%%%%%%%%%%%%%%%%%%%%%%%%%%%%%%%%%%%%%%%%
%%%%%%%%%%%%%%%%%%%%%%%%%%%%%%%%%%%%%%%%%%
%%%%%%%%%%%%%%%%%%%%%%%%%%%%%%%%%%%%%%%%%%
%%%%%%%%PAGEBREAK%%%%%%%PAGEBREAK%%%%%%%%%
%%%%%%%%%%%%%%%%%%%%%%%%%%%%%%%%%%%%%%%%%%
%%%%%%%%%%%%%%%%PAGEBREAK%%%%%%%%%%%%%%%%%
%%%%%%%%%%%%%%%%%%%%%%%%%%%%%%%%%%%%%%%%%%
%%%%%%%%PAGEBREAK%%%%%%%PAGEBREAK%%%%%%%%%
%%%%%%%%%%%%%%%%%%%%%%%%%%%%%%%%%%%%%%%%%%
%%%%%%%%%%%%%%%%%%%%%%%%%%%%%%%%%%%%%%%%%%
\begin{alignment}[
  texts=edition[class="edition"];
  translation[class="translation"],
  ]
  \begin{edition}
    \ekddiv{type=ed}
    \begin{prose}
      \noindent
%---------------------
%yasya           parimalo manaso vacaso na gocaraḥ // \E
%yasya           parimalo manasā vacasā na gocaraḥ /  \P
%yasya           parimalo manasā vacasā    gocaraḥ /  \L
%yasya           parimalo manasā vacasā na gocaraḥ /  \B
%yasya           parimalo manasā vacasā na gocaraḥ /  \N1
%yasya           parimalo manasā vacasā na gocara /   \N2
%yasya           parimalo manasā vacasā na gocaraḥ /  \D
%yasya           parimalo vacasā manasā na gocaraḥ    \U1
%yasya kamalasya parimalo manasā vācā   na gocara ..  \U2
%---------------------
%Whose fragrance is not in range by mind and speech. 
%Dessen Duft ist nicht in Reichweite von Geist und Sprache. 
%---------------------
\app{\lem[wit={ceteri}]{yasya}
  \rdg[wit={U2}]{yasya kamalasya}}
\app{\lem[type=conjecture, resp=egoscr]{parimāṇaṃ vaktuṃ}
  \rdg[wit={ceteri}]{parimalo}} 
\app{\lem[wit={B,D,L,P,N1,N2}]{manasā vacasā}
  \rdg[wit={E}]{manaso vacaso}
  \rdg[wit={U1}]{vacasā manasā}
  \rdg[wit={U2}]{manasā vācā}}
\note[type=philcomm, labelb=103, lem={°manaso vacaso}]{All manuscripts and the printed edition share the reading \textit{parimalo} but most of them keep the grammatically incorrect instrumental \textit{manasā vācasā}. Only the variant of the printed edition arrives at a grammatically correct text. However, this seems to be conjectured by the Paṇḍit who edited the text. The source text reveals a more meaningful sentence and provides a plausible conjecture.}
\app{\lem[wit={ceteri}]{na}
  \rdg[wit={L}]{\om}}
\app{\lem[wit={ceteri}]{gocaraḥ}
  \rdg[wit={N2,U2}]{gocara}}/
%---------------------
%tasya kamalasya madhye trikoṇarūpa-ikā karṇikā varttate/    \E
%tasya kamala----madhye trikoṇārūpā ekā karṇikā varttate/ \P
%tasya kamalasya madhye trikoṇarūpā ekā karṇikā varttate/     \L
%tasya kamalasya madhye trikoṇarūpā ekā karṇikā varttate/     \B
%tasya kamalasya madhye trikoṇarūpā eka karṇikā varttate/     \N1
%tasya kamalasya madhye trikoṇarūpā eka karṇikā varttate/     \N2
%tasya kamalasya madhye trikoṇarūpā ekā karṇikā varttate/     \D
%tasya kamalasya madhye trikoṇarūpā ekā karṇikā vartate       \U1
%tasya kamalasya madhye trikoṇarūpā ekā karṇikā vartate //    \U2
%---------------------
%In the middle of this lotus exists one pericarp having the shape of a triangle. 
%------------------------------
tasya
\app{\lem[wit={ceteri}]{kamalasya}
  \rdg[wit={P}]{kamala°}}
madhye
\app{\lem[wit={E}]{trikoṇarūpaikā}
  \rdg[wit={ceteri}]{trikoṇārūpā ekā}
  \rdg[wit={N1,N2}]{trikoṇārūpā eka}}
karṇikā vartate\dd{}
%------------------------------
%tatkarṇikāmadhye saptadaśī         niraṃjanarūpā kalā varttate/ \E
%tatkarṇikāmadhye saptadaśireṇa ekā niraṃjanarūpā kalā vartate// \L
%tatkarṇikāmadhye saptadaśireṇa ekā niraṃjanarūpā kalā vartate// \B
%tatkarṇikāmadhye saptadaśī     ekā niraṃjanarūpā kalā vartate// \P
%tatkarṇikāmadhye saptadaśī     ekā niraṃjanarūpā kalā vartate// \N1
%tatkarṇikāmadhye saptadaśī     ekā niraṃjanarūpā kalā vartate/  \N2
%tatkarṇikāmadhye saptadaśī     ekā niraṃjanarūpā kalā vartate// \D
%tatkarṇikāmadhye saptadaśī     ekā niraṃjanarūpā kalā vartate  \U1
%tatkarṇikāmadhye saptadaśī     eka niraṃjanarūpā kalā varttate/ \U2
%---------------------
%In the middle of the pericarp exists one seventeenth digit in the shape of a immaculé form.
%---------------------
\note[type=source, labelb=104, lem={\textbf{Re}}]{PT\textsuperscript{qcr \cdot YSV} (Ed. p. 833):  ūrddhvavakraṃ mahāvaktre (\textit{mahāvaktraṃ} YK\textsuperscript{ccn \cdot YSV} 1.274 Ed. p. 22) varṇaśobhāpadaṃ mahat | sarvakalyāṇasampūrṇamasya tulyaṃ na vidyate | parimāṇaṃ vaktam (\textit{vaktum} YK\textsuperscript{ccn \cdot YSV} 1.275 Ed. p. 22) asya manasā vacasā na hi | trikoṇakarṇikā tatra (\textit{°tantraṃ} YK\textsuperscript{ccn \cdot YSV} 1.276 Ed. p. 22) varttate jagad īśvari | kalā saptadaśī tatra varttate parameśvari | nirañjanakalā sā tu koṭisūryasamaprabhā | koṭicandraprabhā caiva śītoṣṇādivivarjitā | asya dhyānāt sādhakasya manoduḥkhaṃ bhaven na hi | anantaparamānandasthānaṃ jñeyaṃ tadūrddhvataḥ (\textit{tadarddhataḥ} YK\textsuperscript{ccn \cdot YSV} 1.278 Ed. p. 22) | ūrddhvagatakalā tatra tasya dhyānād bhaved iti | iti siddhirājayogaṃ strīṇāṃ bhogaṃ mahāsukham | gītavādyavinodādi saśivaṃ varddhate kṣitau | dhyānaṃ nirantarañ cāsya puṇyapāpe sthire (\textit{sthirau } YK\textsuperscript{ccn \cdot YSV} 1.280 Ed. p. 22) na hi | nijarūpasya dṛṣṭiḥ syād dūrasyārthañ ca paśyati ||}
\note[type=testium, labelb=101, lem={\textbf{Cie}}]{\textit{Yogasaṃgraha} IGNCA 30020 folio 2r. ll. 9-11: tasyāṃ karṇikāyāṃ saptadaśī niraṃjanarūpā koṭisūryaprabhā satī uṣṇabhava hīnā koṭicandrasamasītalaikākalāsti | tasyāṃ anaṃta paramānaṃtaparamānaṃdānāṃ sthānaṃ tasyāḥ kalāyā dhyānakaraṇāt sādako yadyādi śati tatra bhavati|}
tatkarṇikāmadhye
\app{\lem[wit={ceteri}]{saptadaśī}
  \rdg[wit={L,B}]{saptadaśireṇa}}
\app{\lem[wit={ceteri}]{ekā}
  \rdg[wit={E}]{\om}}
nirañjanarūpā kalā varttate/
%---------------------
%koṭisūryasamaprabhaṃ kalāyās tejo vartate /    \E
%koṭisūryasamaprabhā kalāyās tejo vartate /     \L
%koṭisūryasamaprabhā kalāyās tejo vartate /     \B
%koṭisūryasamaprabha kalāyās tejo vartate /     \P
%koṭisūryasamaprabhaṃ kalāyās tejo vartate /    \N1
%koṭisūryasamaprabhaṃ kalāyā  tejo varttate //  \N2
%koṭisūryasamaprabhaṃ kalāyās tejo vartate /    \D
%koṭisūryasadṛṣaprabhaṃ kalāyās tejo vartate /  \U1
%koṭisūryasamaprabhā // kalāyās tejo varttate / \U2
%---------------------
%A light of the part exists shining like a thousand suns. 
%------------------------------
koṭisūrya\app{\lem[alt={°samaprabhaṃ}, wit={ceteri}]{samaprabhaṃ}
  \rdg[wit={L,B,U2}]{samaprabhā}
  \rdg[wit={P}]{samaprabha}
  \rdg[wit={U1}]{sadṛṣaprabhaṃ}}
kalāyās-tejo vartate/\textsuperscript{\begin{otherlanguage}{english}\coro{[\lowroman{10}]}\end{otherlanguage}}
%------------------------------
%param udbhavo nāsti /     \E
%parim uṣṇabhavo nāsti /   \P
%parim uṣṇabhavo nāsti /   \L
%parim uṣṇabhavo nāsti /   \B
%parim uṣṇabhāvo nāsti /   \N1
%para  uṣṇabhāvo nāsti     \N2
%parim auṣṇabhāvo nāsti /  \D
%paraṃ uṣṇabhāvo nāsti     \U1
%param uṣṇabhāvo nāsti /   \U2
%---------------------
%[But] excessive heat is not arising. 
%------------------------------
\app{\lem[alt={param},wit={E,U1,U2}]{para\skp{m-u}}
  \rdg[wit={U1}]{paraṃ}
  \rdg[wit={N2}]{para}
  \rdg[wit={ceteri}]{parim}
}\app{\lem[wit={ceteri}, alt={uṣṇabhāvo}]{\skm{m-u}ṣṇabhāvo}
  \rdg[wit={P,L,B}]{uṣṇabhavo}
  \rdg[wit={D}]{auṣṇabhāvo}
  \rdg[wit={E}]{udbhavo}
}
nāsti/
%------------------------------
%koṭicandrasamaprabhā    śītalaṃ paraṃ   śītabhāvo   nāsti / \E
%koṭicandrasamaprabhā    śītalaṃ paraṃ   śītabhavo   nāsti / \P
%\om /                                                      \L
%koṭicandrasamaprabhā    śītalaṃ paraṃ   śītabhavo   nāsti / \B
%koṭicandrasamaprabhaṃ   śītalaparaṃ         bhavo   nāsti / \N1
%koṭicandrasamaprabhaṃ   śītalapara----------bhavo   nāsti // \N2
%koṭicaṃdrasamaprabhaṃ   śītalaparaṃ         bhavo   nāsti / \D
%koṭicaṃdrasamaṃ prabhaṃ śītalaṃ paraṃ       bhavo   nāsti / \U1
%koṭicaṃdrasamaprabhā    śītalaṃ paraṃ śītalabhāvo   nāsti / \U2
%---------------------
%Shining like a thousand moons, excess of cold is not arising.
%---------------------
koṭicandra\app{\lem[alt={°samaprabhaṃ},wit={N1,N2,D}]{samaprabhaṃ}
  \rdg[wit={Y}]{°samaprabhā}
  \rdg[wit={U1}]{°samaṃ prabhaṃ}
  \rdg[wit={L}]{\om}}
\app{\lem[wit={N1,D}]{śītalaparaṃ}
  \rdg[wit={ceteri}]{śītalaṃ paraṃ}
  \rdg[wit={N2}]{śītalapara}
  \rdg[wit={L}]{\om}}
\app{\lem[wit={ceteri}]{bhāvo} 
  \rdg[wit={E,P,B}]{śītabhāvo}
  \rdg[wit={U2}]{śītalabhāvo}
  \rdg[wit={L}]{\om}}
nāsti/
%------------------------------
%asyāḥ kalāyā   dhyānayogāt    sādhakasya manasi duḥkhaṃ na bhavati / \E
%asyāḥ kalādhyānayogāt         sādhakasya manasi duḥkhaṃ na bhavati / \P
%asyāḥ kalāyāḥ  dhyānakaraṇāt  sādhakasya manasi duḥkhaṃ na bhavati / N1
%asyā kalāyā    dhyānakaraṇāt  sādhaka----manasi duḥkhaṃ na bhavati / N2
%asyāḥ kalāyāḥ  dhyānakaraṇāt  sādhakasya manasi duḥkhaṃ na bhavati / D
%
%asyāḥ kalāyā   dhyānayogāt    sādhakasya manasi duḥkhaṃ bhavati /B
%asyāḥ kalāyā   dhyānayogāt    sādhakasya manasi duḥkhaṃ bhavati /L
%asyāḥ kalāyā   dhyānakaraṇāt/ sādhakasya manasi duḥkhaṃ na bhavati / U1
%asyā  kalāyāḥ  dhyānayogāt//  sādhakasya manasi duḥkhaṃ na bhavati // \U2
%atrastāne 'haṃ devatā// sohaṃ śaktiḥ// ātmāṛṣiḥ// mokṣamārhaḥ// haṃbhrahmordhaṃ// haṃcakra iti// agnicakre sakaro bhavatī// prāṇīrūḍho bhave jjīva ārohaty avarohati bhavaguhāsthānaṃ pitavarṇaṃ// koṭisūryapratikāśaṃ tejaḥ sadoditaprabhā śīvodevatā// mūlamāyā śaktiḥ// hara ātmālayāvsthā dhvanisthirānādātmako khaṃḍa 'dhvani// adhorāmudrā// mūlamāyā// prakṛtidehaḥ// vāṅmanogocaraḥ// niḥprapaṃcaḥ// niḥsaṃśayaḥ// nistaraṃganirlepalakṣaṃ laya// dhyānasamādhi 
%---------------------
%asyāḥ kalāyā dhyānakaraṇāt\varc{\emend kalāyāḥ dhyānakaraṇāt \nepal \dehlia}{kalāyā dhyānayogāt \nepal \dehlia kalādhyānayogāt \pune} sādhakasya manasi duḥkhaṃ na\varc{na \edprint \pune \nepal \dehlia}{\om \oxford \lalchand} bhavati /
%Due to the exercise of meditation upon the digit suffering does not arise in the mind of the practitioner (anymore). 
%------------------------------
\app{\lem[wit={ceteri}]{asyāḥ}
  \rdg[wit={N2,U2}]{asyā}}
\app{\lem[wit={N2,U1}]{kalāyā}
  \rdg[wit={N1,D}]{kalāyāḥ}
  \rdg[wit={E,B,L}]{kalāyā}
  \rdg[wit={U2}]{kalāyāḥ}
  \rdg[wit={P}]{\om}}
dhyāna\app{\lem[wit={X}, alt={°karaṇāt}]{karaṇāt}
  \rdg[wit={Y}]{°yogāt}}
\app{\lem[wit={ceteri}]{sādhakasya}
  \rdg[wit={N2}]{sādhaka°}}
duḥkhaṃ
\app{\lem[wit={ceteri}]{na}
  \rdg[wit={B,L}]{\om}}
bhavati/
%%%%%%%%%%%%%
%%%%%%%%%%%%
%%%%%%%%%%%%
%%%%%%%%%%%%
%%%%%%%%%%%%
\extra{atra
   \app{\lem[type=emendation, resp=egoscr]{sthāne}
    \rdg[wit={U2}]{stāne}} 'haṃ devatā\dd{}
  sohaṃ śaktiḥ\dd{}
  ātmāṛṣiḥ\dd{}
  \app{\lem[type=emendation, resp=egoscr]{mokṣo}
    \rdg[wit={U2}]{mokṣa°}} mārgaḥ\dd{}
   \app{\lem[type=emendation, resp=egoscr]{ahaṃ brahmordhvaṃ}
    \rdg[wit={U2}]{haṃ brahmordhaṃ}}\dd{}
   \app{\lem[type=emendation, resp=egoscr]{ahaṃ cakra iti}
     \rdg[wit={U2}]{haṃcakra iti}}\dd{}
   agnicakre
   \app{\lem[type=emendation, resp=egoscr]{sakāro}
     \rdg[wit={U2}]{sakaro}}
   \app{\lem[type=emendation, resp=egoscr]{bhavati}
     \rdg[wit={U2}]{bhavatī}}\dd{}
   prāṇī rūḍho bhavej-jīva ārohaty-avarohati\dd{}
bhavaguhā sthānaṃ\dd{}
   \app{\lem[type=emendation, resp=egoscr]{pitaṃ}
     \rdg[wit={U2}]{pita°}} varṇaṃ\dd{}
   koṭisūryapratikāśaṃ tejaḥ\dd{}
   \app{\lem[type=emendation, resp=egoscr]{sadoditā}
     \rdg[wit={U2}]{sadodita°}} prabhā\dd{}
   \app{\lem[type=emendation, resp=egoscr]{śivo}
     \rdg[wit={U2}]{śīvo}} 
   devatā\dd{}
   mūlamāyā śaktiḥ\dd{}
   \app{\lem[type=emendation, resp=egoscr]{harātmālayāvasthā}
     \rdg[wit={U2}]{hara ātmālayāvasthā}}\dd{}
   dhvanisthirānādātmako \app{\lem[type=emendation, resp=egoscr]{'khaṇḍadvaniḥ}
     \rdg[wit={U2}]{khaṃḍadhvani}}\dd{} 
   aghorā mudrā\dd{}
   \app{\lem[type=emendation, resp=egoscr]{mūlā} %macht diese emdendation wirklich Sinn? 
     \rdg[wit={U2}]{mūla°}} māyā\dd{}
   \app{\lem[type=emendation, resp=egoscr,alt={prakṛtir}]{prakṛti\skp{r-de}}
     \rdg[wit={U2}]{prakṛti°}}\skm{r-de}haḥ\dd{}
   vāṅmano 'gocaraḥ\dd{} %%
   niḥprapañcaḥ\dd{}
   niḥsaṃśayaḥ\dd{}
   nistaraṃganirlepalakṣaṃ %%%see pw Vol. 3, S. 229 for nistaranga
  \app{\lem[type=emendation, resp=egoscr]{layo}
     \rdg[wit={U2}]{laya}}
   \app{\lem[type=emendation, resp=egoscr]{dhyānaḥ samādhiḥ}
     \rdg[wit={U2}]{dhyānasamādhi}}\dd{}}\textsuperscript{\begin{otherlanguage}{english}\coro{[\lowroman{14}]}\end{otherlanguage}}
%\extra{Here at this location the ``I''(\textit{aham}) is the deity. The ``he is I'' (\textit{so 'ham}) is the power. This self is the Ṛṣi. The path is liberation. Brahma is the I above. ``I'm a circle''. In the circle of fire is the letter "sa". [There?] life arises, the living soul ascends and decends. The place is the hidden place of being. The colour is yellow. The light is the shine of ten million suns. The shine is always and visible. Śiva is the deity. The power is primordial illusion. The state is the dissolution of the self into Hara\footnote{Epiphet of Śiva.}. The transcendental sound has the nature of a sound with stable resonance. The seal is the ``fearless''. The illusion is the root. The body is the original matter. It is not within reach of speech and mind. It is without delusion. It is without doubt. The unaffected and undefiled goal is dissolution, meditation [and] final absorption.}
    \end{prose}
  \end{edition}
  \begin{translation}
    \ekddiv{type=trans}
    \begin{tlate}
      \noindent
\ldots  It is not possible to express the seize of it with mind and speech. In the middle of this lotus exists one pericarp with a triangle shape. In the middle of the pericarp exists the seventeenth digit in having an immaculé form. There is a light of the digit, shining like a thousand suns,\textsuperscript{\coro{[\lowroman{10}]}} [but] excessive heat is not arising. Shining like a thousand moons, excess of cold is not arising. \extra{Here at this location the ``I''(\textit{aham}) is the deity. The ``he is I'' (\textit{so 'ham}) is the power. This self is the Ṛṣi. The path is liberation. Brahma is the I above. ``I am a circle''. In fire-area is the letter "sa". [There?] life arises, and the soul ascends and descends.\footnote{Find parallels of the hemistich.} The place is the hidden place of being. The colour is yellow. The light is the shine of ten million suns. The shine is always visible. Śiva is the deity. The primordial illusion is the power. The state is the dissolution of the self into Hara\footnote{Epiphet of Śiva.}. The transcendental sound has the nature of a sound with stable resonance. The ``fearless'' is the seal. The illusion is the root. The original matter is the body. Speech and mind are the range. Without delusion, without doubt, the unaffected and undefiled goal is dissolution, meditation [and] final absorption.\textsuperscript{\coro{[\lowroman{14}]}}}
    \end{tlate}
  \end{translation}
\end{alignment}
\begin{alignment}[
  texts=edition[class="edition"];
  translation[class="translation"],
  ]
  \begin{edition}
    \ekddiv{type=ed}
    \begin{prose}
    \noindent
%---------------------
%tatrordhvaśaktiḥ / \E
%tatordhvaśaktiḥ \P
%rdhaśakti ardhaśakti \B
%rdhaśakti ardhaśakti \L
%tatrordhvaśaktiḥ / \N1
%tatra ūrdhva śaktiḥ / \D
%tatra ūrdhva śakti / \N2
%urdhvaśaktir         \U1
%tatrordhvaśaktiḥ// \U2
%---------------------
%There above is \textit{śakti},
%------------------------------
\app{\lem[wit={E,N1,U2}]{tatrordhvaśaktiḥ}
  \rdg[wit={P}]{tatordhvaśaktiḥ}
  \rdg[wit={U1}]{urdhvaśaktir}
  \rdg[wit={D}]{tatra ūrdhva śaktiḥ}
  \rdg[wit={N2}]{tatra ūrdhva śakti}
  \rdg[wit={B,L}]{rdhaśakti ardhaśakti}}/\textsuperscript{\begin{otherlanguage}{english}\coro{[\lowroman{15}]}\end{otherlanguage}}
%------------------------------
%etādṛśī  saṃjñā   ekā kalā vartate / \E
%ekādaśā  saṃjñā   ekā kalā vartate   \P
%etādṛśī  saṃjñā   ekā kalā vartate /  \N1
%etādṛśī  saṃjñā   ekā kalā varttate / \N2
%etādṛsaṃ saṃjñā   ekā kalā vartate / \D
%ekādaśā  saṃjñā   ekā kalā vartate / \B
%ekādaśā  saṃjñā   ekā kalā vartate / \L
%etādṛśī  saṃjñakā ekā kalā vartate /  \U1
%etādṛśā  saṃjñā   ekā kalā vartate/ \U2 
%---------------------
%Being designated as such she is one single digit. 
%------------------------------
\app{\lem[wit={ceteri}]{etādṛśī}
  \rdg[wit={U2}]{etādṛśā}
  \rdg[wit={D}]{etādṛsaṃ}
  \rdg[wit={P,B,L}]{ekādaśā}}
\app{\lem[wit={ceteri}]{saṃjñā}
  \rdg[wit={U1}]{saṃjñakā}}
ekā kalā vartate/
%------------------------------
%asyāḥ  kalāyā   dhyānakāraṇāt     puruṣo yadicchati / \E
%asyāḥ  kalāyā   dhyānakāraṇāt     puruṣo yadicchati ?Zeichen? \P
%asyāḥ  kalāyā   dhyānakāraṇāt     puruṣo yadicchati  tad bhavati \N1
%tasyāḥ kalāyāḥ  dhyānakāraṇāt     puruṣo yadicchati  tad bhavati \N2
%asyāḥ  kalāyā   dhyānakāraṇā      puruṣo yadicchati  tad bhavati \D
%asyāḥ  kalāyā   dhyānakāraṇāt /   puruṣo yadicchati / \B
%asyāḥ  kalāyā   dhyānakāraṇāt /   puruṣo yadicchati / \L
%asyā   kalāyā   dhyānakāraṇāt     puruṣo yadicchati tad bhavati vā \U1
%asyāḥ  kalāyāḥ  dhyānakāraṇāt //  puruṣo yadicchati // \U2
%---------------------
%Due to the exercise of meditation on this part the person manifests whatever he wishes for.
%------------------------------
\app{\lem[wit={ceteri}]{asyāḥ}
  \rdg[wit={U1}]{asyā}
  \rdg[wit={N2}]{tasyāḥ}}
\app{\lem[wit={ceteri}]{kalāyā}
  \rdg[wit={N2,U2}]{kalāyāḥ}}
\app{\lem[wit={ceteri}]{dhyānakāraṇāt}
  \rdg[wit={D}]{dhyānakāraṇā}}
puruṣo yad-icchati
\app{\lem[wit={N1,N2,D}, alt={tad bhavati}]{tad-bhavati}
  \rdg[wit={U1}]{tad bhavati vā}
  \rdg[wit={Y}]{\om}}/ 
%------------------------------
%tasya sukhabhogavataḥ / \E
%tasya sukhabhogavataḥ \P
%rājya-sukhabhogavataḥ \N1
%rājya-sukhabhogavataḥ \N2
%rājya-sukhabhogavṛtaḥ \D !!!
%tasya-khaṃ bhogavataṃ / \B
%tasya-sukhaṃ bhogavaṃtaṃ / \L
%rājya-sukhabhogavataḥ \U1
%tasya-sukhabhogavataḥ / \U2
%---------------------
%He is furnished with royal pleasure and enjoyment. 
%------------------------------
\note[type=testium, labelb=107, lem={\textbf{Cie}}]{\textit{Yogasaṃgraha} IGNCA 30020 folio 3v. ll. 1-4: rājyasukhabhogavatah̤ strī vilāsavataḥ saṃgītavinoda prekṣāvato pi sādhakasya śuklapakṣacaṃdravat pratidinaṃ tejaso vapuṣaś ca vṛddiḥ puṇyapāpasya śārbhāvaḥ nijasva rūpaprakāśasāmarthaṃ dūrasthapy arthasya samīpastham iva darśanaṃ ca bhavati | cha | tad uktaṃ tattvajñānapradīpikāyāṃ ||}
\note[type=philcomm, labelb=108, lem={rājyasukhabhoga°}]{Here ends the testimony of the \textit{Yogasaṃgraha}  IGNCA 30020.}
\app{\lem[wit={D}]{rājyasukhabhogavṛtaḥ}
  \rdg[wit={N1,N2,U1}]{rājyasukhabhogavataḥ}
  \rdg[wit={E,P,U2}]{tasya sukhabhogavataḥ}
  \rdg[wit={B}]{tasya khaṃ bhogavataṃ}
  \rdg[wit={L}]{tasya sukhaṃ bhogavaṃtaṃ}}/
%------------------------------
%strīmadhye     vilāsavataḥ    saṃgītavilāsavataḥ vinodaprekṣāvataḥ        puruṣasya pratidinaṃ śuklapakṣe candrakalāvat   kalā     vardhate/   \E
%strīmadhye     vilāsavataḥ    saṃgītavinodaprekṣāvataḥ              eva   puruṣasya pratidinaṃ śuklapakṣe candrakalāvat   kalā     vardhate /  \P
%strīmadhye     vilāsavaṃtaṃ   saṃgītaṃ prekṣāvatāḥ //               evaṃ  puruṣasya pratidinaṃ śuklapakṣe caṃdrakalāvat / kalā     vartate /   \L
%strīmadhye     vilāsavaṃtaṃ   saṃgītaṃ vinodavaṃtaṃ prekṣāvaṃtāḥ // eva   puruṣasya pratidinaṃ śuklapakṣe caṃdrakalāvat / kalā     vartate /   \B
%strīmadhye     vilāsavataḥ    saṃgītavinodaprekṣyāvataḥ             evaṃ  puruṣasya pratidinaṃ śuklapakṣe candrakalā vṛddhivato?   vardhate / \N1
%śrī strīmadhye vilāsavataḥ    saṃgītavinodaprekṣāvataḥ              evaṃ  puruṣasya pratidinaṃ śuklapakṣa candrakalā vṛddhi vaṃto  varttate /  \N2
%strīmadhye     vilāsavataḥ // saṃgītavinodaprekṣyāvataḥ //          evaṃ  puruṣasya pratidinaṃ śuklapakṣe candrakalā vṛddhivato    vardhate / \D
%strīmadhye     vilāśavataḥ    saṃgītavinodaprekṣyāvataḥ             eka   puruṣasya pratidinaṃ śuklapakṣe caṃdrakalā vṛddhir       varddhate / \U1
%strīmadhye     vilāsavata     saṃgītavinodaprekṣāvata//             evaṃ  puruṣasya pratidinaṃ śuklapakṣe candrakalāvat   kalā     varttate/   \U2
%---------------------
%(Selbst) bei einem Menschen, der sich inmitten von Frauen vergnügt, (und) ein Musikvergnügen
%ansieht, wächst täglich die Kraft (kalā = śakti?) wie die "kalā" (Phase) des Mondes in der hellen Monatshälfte.
%The \textit{kalā} of a person grows daily, like the \textit{kalā} of the moon in the bright half of the month, even amusing oneself amongst women and watching a musical pleasure.
%(Even) amusing oneself amongst women, and watching musical pleasures, the \textit{kāla} of the person grows daily like the \textit{kalā} of the moon in the bright half of the month. 
%------------------------------
\app{\lem[wit={ceteri}]{strīmadhye}
  \rdg[wit={N2}]{śrī strīmadhye}}
\app{\lem[wit={ceteri}]{vilāsavataḥ}
  \rdg[wit={U2}]{vilāsavata°}
  \rdg[wit={L,B}]{vilāsavaṃtaṃ}} 
saṃgīta\app{\lem[wit={N1,D,U1},alt={°vinodaprekṣyāvataḥ}]{vinodaprekṣyāvataḥ}
  \rdg[wit={P,N2}]{°vinodaprekṣāvataḥ}
  \rdg[wit={U2}]{°vinodaprekṣāvata}
  \rdg[wit={B}]{°ṃ vinodavaṃtaṃ prekṣāvaṃtāḥ}
  \rdg[wit={E}]{°vilāsavataḥ vinodaprekṣāvataḥ}
  \rdg[wit={L}]{°ṃ prekṣāvatāḥ}}
 \app{\lem[wit={P,B}]{eva}
  \rdg[wit={ceteri}]{evaṃ}
  \rdg[wit={U1}]{eka}}
puruṣasya pratidinaṃ śuklapakṣe
candrakalā\app{\lem[wit={Y},alt={°vat kalā}]{vat kalā}
  \rdg[wit={N1,D}]{vṛddhivato}
  \rdg[wit={N2}]{vṛddhi vaṃto}
  \rdg[wit={U1}]{vṛddhir}}
\app{\lem[wit={D ,E,P,N1,U1}]{vardhate}
  \rdg[wit={ceteri}]{vartate}}/
%------------------------------
%puṇyapāpe  'sya śarīraṃ   na spṛśataḥ /    \E
%\om                                     \P
%puṇyapāpe  asya śarīrena     spṛśataḥ /      \N1
%puṇyapāpe  asya śarīrena     spṛśataḥ /      \N2
%puṇyapāpe  asya śarīrena     spṛśataḥ /      \D
%puṇyapāpe  asya śarīrasya na spṛśataḥ // \B
%puṇyapāpe  asya śarīrasya na spṛśataḥ // \L
%puṇyapāpau asya śarīrena     spṛśāt         \U1
%puṇyapāpe  asya śarīraṃ   na spṛśataḥ // \U2
%---------------------
%puṇyapāpe\varc{puṇyapāpe \edprint \lalchand \oxford \nepal \dehlia}{\om \pune} 'sya\varc{'sya \edprint}{asya \nepal \dehlia \oxford \lalchand \om \pune} śarīrasya\varc{śarīrasya \lalchand \oxford}{śarīraṃ \edprint śarīrena \nepal \dehlia \om \pune} na\varc{na \edprint \oxford \lalchand}{\om \nepal \dehlia \pune} spṛśataḥ\varc{spṛśataḥ \edprint \lalchand \oxford \nepal \dehlia}{\om \pune} /
%---------------------
%His body is not affected by merit and sin. 
%------------------------------
\app{\lem[wit={ceteri}]{puṇyapāpe}
  \rdg[wit={U1}]{puṇyapāpau}
\rdg[wit={P}]{\om}}
\app{\lem[wit={E}]{'sya}
  \rdg[wit={P}]{\om}
  \rdg[wit={ceteri}]{asya}}  
śarīr\app{\lem[wit={B,L}]{śarīrasya}
  \rdg[wit={X}]{śarīrena}
  \rdg[wit={E,U2}]{śarīraṃ}
  \rdg[wit={P}]{\om}}
\app{\lem[wit={E,B,L,U2}]{na}
  \rdg[wit={X,P}]{\om}}
spṛ\app{\lem[wit={ceteri},alt={°śataḥ}]{śataḥ}
  \rdg[wit={U1}]{°śāt}}/\textsuperscript{\begin{otherlanguage}{english}\coro{[\lowroman{20}]}\end{otherlanguage}}
%------------------------------
%                          nirantaradhyānakaraṇāt     nijasvarūpaṃ prakāśanasāmarthyaṃ bhavati / \E
%                          \om until .....            nijasvarūpaprakāśasāmarthyaṃ     bhavati / \P
%                          niraṃtaraṃ dhyānakaraṇāt   nijasvarūpaprakāśasāmarthyaṃ     bhavati / \B
%                          niraṃtaraṃ dhyānakaraṇāt// nijasvarūpaprakāśasāmarthyaṃ     bhavati / \L
%                          nirantaradhyānakaraṇāt /   nijasvarūpaprakāśasāmarthyaṃ     bhavati / \N1 <-----
%                          niraṃtaradhyānakaraṇāt /   nijasvarūpaprakāśasāmarthyaṃ     bhavati // \N2
%                          nirantaradhyānakaraṇāt /   nijasvarūpaprakāśasāmarthyaṃ     bhavati / \D
%                          nirantaradhyānakaraṇāt /   nijasvarūpaprakāśasāmarthyaṃ     bhavati    \U1
%evaṃ puruṣasya pratidinaṃ niraṃtaraṃ dhyānakaraṇāt   nijasvarūpaṃ prakāśanasāmarthyaṃ bhavati// \U2 
%---------------------
%Due to uninterrupted meditation the power of the light of the innate nature arises. 
%------------------------------
\app{\lem[wit={ceteri}]{nirantaradhyānakaraṇāt}
  \rdg[wit={B,L}]{niraṃtaraṃ dhyānakaraṇāt}
  \rdg[wit={U2}]{evaṃ puruṣasya pratidinaṃ niraṃtaraṃ dhyānakaraṇāt}
  \rdg[wit={P}]{\om}}
nijasvarūpa\app{\lem[wit={ceteri},alt={°prakāśa°}]{prakāśa}
  \rdg[wit={E,U2}]{°ṃ prakāśana°}
}sāmarthyaṃ bhavati/
%------------------------------
%dūrasthopi ca dūrasthavastu                   samīpa iva   paśyati // \E
%dūrasthamapi                                  samīpam iva  paśyati // \N1
%dūrasthamapi                                  samīpaṃ iva  paśyati // \N2
%dūrasthamapy-arthaṃ                           samīpa iva   paśyati // \D
%dūrasthamapi padārthaṃ                        samīpa iva   paśyati // \B
%dūrasthamapi parārthaṃ                        samīpa iva   paśyati // \L
%dūrasthamapi padārthaṃ                        samīpa iva   paśyati // \P
%dūrasthamapy-arthaṃ                           samīpam eva  paśyati // \U1
%dūrasthamapi bhavati //dūrasthamapi padārthaṃ samīpa iva   paśyati // \U2
%------------------------------
%dūrasthamapyarthaṃ\varc{dūrasthamapyarthaṃ \dehlia}{dūrasthamapi padārthaṃ \oxford \pune durasthamapi parārthaṃ \lalchand sūrastamapi \nepal ca dūrasthavastu \edprint} samīpa\varc{samīpa \dehlia \edprint \lalchand \oxford \pune}{samīpam \nepal} iva paśyati //
%------------------------------
%He sees remotely located objects as if they'd be near.
%------------------------------
dūra\app{\lem[wit={D,U1},alt={°stham apy arthaṃ}]{stham-apy-arthaṃ}
  \rdg[wit={B,P}]{°stham api padārthaṃ}
  \rdg[wit={L}]{°stham api parārthaṃ}
  \rdg[wit={E}]{°sthopi ca dūrasthavastu}
  \rdg[wit={N1,N2}]{°stham api}
  \rdg[wit={U2}]{°stham api bhavati || dūrastham api padārthaṃ}}
\app{\lem[wit={ceteri}]{samīpa}
  \rdg[wit={N1}]{samīpam}
  \rdg[wit={N2}]{samīpaṃ}
  \rdg[wit={U1}]{samīpam}}
\app{\lem[wit={ceteri}]{iva}
  \rdg[wit={U1}]{eva}} 
paśyati\dd{}
    \end{prose}
  \end{edition}
  \begin{translation}
    \ekddiv{type=trans}
    \begin{tlate}
      \noindent
      Above that is the place of infinite supreme bliss. There above is power (\textit{śakti}).\textsuperscript{\coro{[\lowroman{15}]}} Being designated as such, she is one single digit. Due to the meditation exercise on this part, the person manifests whatever he wishes for. He is furnished with royal pleasure and enjoyment. [Even] amusing oneself amongst women and watching musical pleasures, the \textit{kāla} of the person grows daily like the \textit{kalā} of the moon in the bright half of the month. His body is not affected by merit and sin.\textsuperscript{\coro{[\lowroman{20}]}} Due to uninterrupted meditation, the power of the light of innate nature arises. He sees remotely located objects as if they were near.\footnote{The ninefold \textit{cakra} system can be identified in the \textit{Yogasvarodaya}, the \textit{Siddhasiddhāntapaddhati}, the \textit{Yogakarṇikā}, the \textit{Yogatattvabindu}. Another text that used the same system and probably quoted the \textit{Siddhasiddhāntapaddhati} without reference with a few redactions is the \citetitle{saubhagya}:
        \begin{quote}
          atha hainaṃ devā ūcurnavacakravivekam anubrūhīti | tatheti sa hovāca ādhāre brahmacakraṃ trirāvṛttaṃ bhagamaṇḍalākāram | tatra mūlakande śaktiḥ pāvakākāraṃ dhyāyet | tatraiva kāmarūpapīṭhaṃ sarvakāmapradaṃ bhavati | ity ādhāracakram | dvitīyaṃ svādhiṣṭhānacakraṃ ṣaḍdalam | tanmadhye paścimābhimukhaṃ liṅgaṃ pravālāṅkurasadṛśaṃ dhyāyet | tatraivoḍyāṇapīṭhaṃ jagadākarṣaṇasiddhidaṃ bhavati | tṛtīyaṃ nābhicakraṃ pañcāvartaṃ sarpakuṭilākāram | tanmadhye kuṇḍalinīṃ bālārkakoṭiprabhāṃ tanumadhyāṃ dhyāyet | sāmarthyaśaktiḥ sarvasiddhipradā bhavati | maṇipūracakraṃ hṛdayacakram | aṣṭadalamadhomukham | tanmadhye jyotirmayaliṅgākāraṃ dhyāyet | saiva haṃsakalā sarvapriyā sarvalokavaśyakarī bhavati | kaṇṭhacakraṃ caturaṅgulam | tatra vāme iḍā candranāḍī dakṣiṇe piṅgalā sūryanāḍī tanmadhye suṣumnāṃ śvetavarṇāṃ dhyāyet | ya evaṃ vedānāhatā siddhidā bhavati | tālucakram | tatrāmṛtadhārāpravāhaḥ | ghaṇṭikāliṅgamūlacakrarandhre rājadantāvalambinīvivaraṃ daśadvādaśāram | tatra śūnyaṃ dhyāyet | cittalayo bhavati | saptamaṃ bhūcakramaṅguṣṭhamātram | tatra jñānanetraṃ dīpaśikhākāraṃ dhyāyet | tadeva kapālakandavāksiddhidaṃ bhavati | ājñācakram aṣṭamam | brahmarandhraṃ nirvāṇacakram | tatra sūcikāgṛhetaraṃ dhūmraśikhākāraṃ dhyāyet | tatra jālandharapīṭhaṃ mokṣapradaṃ bhavatīti parabrahmacakram | navamamākāśacakram | tatra ṣoḍaśadalapadmamūrdhvamukhaṃ tanmadhyakarṇikātrikūṭākāram | tanmadhye ūrdhvaśaktiḥ | tāṃ paśyandhyāyet | tatraiva pūrṇagiripīṭhaṃ sarvecchāsiddhisādhanaṃ bhavati |
        \end{quote}}
    \end{tlate}
  \end{translation}
  \ekdpb*{}
\end{alignment}
%%%%%%%%%%%%%%%%%%%%%%%%%%%%%%%%%%%%%%%%%%
%%%%%%%%%%%%%%%%%%%%%%%%%%%%%%%%%%%%%%%%%%
%%%%%%%%PAGEBREAK%%%%%%%PAGEBREAK%%%%%%%%%
%%%%%%%%%%%%%%%%%%%%%%%%%%%%%%%%%%%%%%%%%%
%%%%%%%%%%%%%%%%PAGEBREAK%%%%%%%%%%%%%%%%%
%%%%%%%%%%%%%%%%%%%%%%%%%%%%%%%%%%%%%%%%%%
%%%%%%%%PAGEBREAK%%%%%%%PAGEBREAK%%%%%%%%%
%%%%%%%%%%%%%%%%%%%%%%%%%%%%%%%%%%%%%%%%%%
%%%%%%%%%%%%%%%%%%%%%%%%%%%%%%%%%%%%%%%%%%
%%%%%%%%%%%%%%%%%%%%%%%%%%%%%%%%%%%%%%%%%%
%%%%%%%%%%%%%%%%%%%%%%%%%%%%%%%%%%%%%%%%%%
%%%%%%%%PAGEBREAK%%%%%%%PAGEBREAK%%%%%%%%%
%%%%%%%%%%%%%%%%%%%%%%%%%%%%%%%%%%%%%%%%%%
%%%%%%%%%%%%%%%%PAGEBREAK%%%%%%%%%%%%%%%%%
%%%%%%%%%%%%%%%%%%%%%%%%%%%%%%%%%%%%%%%%%%
%%%%%%%%PAGEBREAK%%%%%%%PAGEBREAK%%%%%%%%%
%%%%%%%%%%%%%%%%%%%%%%%%%%%%%%%%%%%%%%%%%%
%%%%%%%%%%%%%%%%%%%%%%%%%%%%%%%%%%%%%%%%%%
%%%%%%%%%%%%%%%%%%%%%%%%%%%%%%%%%%%%%%%%%%
%%%%%%%%%%%%%%%%%%%%%%%%%%%%%%%%%%%%%%%%%%
%%%%%%%%PAGEBREAK%%%%%%%PAGEBREAK%%%%%%%%%
%%%%%%%%%%%%%%%%%%%%%%%%%%%%%%%%%%%%%%%%%%
%%%%%%%%%%%%%%%%PAGEBREAK%%%%%%%%%%%%%%%%%
%%%%%%%%%%%%%%%%%%%%%%%%%%%%%%%%%%%%%%%%%%
%%%%%%%%PAGEBREAK%%%%%%%PAGEBREAK%%%%%%%%%
%%%%%%%%%%%%%%%%%%%%%%%%%%%%%%%%%%%%%%%%%%
%%%%%%%%%%%%%%%%%%%%%%%%%%%%%%%%%%%%%%%%%%
\begin{alignment}[
  texts=edition[class="edition"];
  translation[class="translation"],
  ]
  \begin{edition}
    \ekddiv{type=ed}
\nolinenumbers
%   \bigskip
\centerline{\textrm{\small{[\uproman{13}.\textsuperscript{\coro{\lowroman{1}-\lowroman{3}}} Lakṣyayoga]}}}
\label{laksyayoga}
    \bigskip
    \linenumbers
    \begin{prose}
      \noindent
%------------------------------
%idānīṃ sukhasādhyo lakṣyayogaḥ kathyate / \E
%idānīṃ sukhasādho  lakṣyayogaḥ kathyate / \P
%idānīṃ sukhasādho  lakṣayogaḥ  kathyate / \B
%idānīṃ sukhasādhe  lakṣayogaḥ  kathyate // \L
%idānīṃ sukhasādhyo lakṣyayogaḥ kathyate / \N1
%idānīṃ sukhasādhya lakṣanayogaḥ kathyate / \N2
%idānīṃ sukhasādhyo lakṣyayogaḥ kathyate / \D
%idānīṃ sukhasādhyopalakṣayogaḥ kathyate / \U1
%idānīṃ sukhasādhyo lakṣyayogaḥ kathyate / \U2
%------------------------------
%Now the yoga of fixation{\textit{lakṣyayoga}}, which is easily accomplished is explained. 
%------------------------------
      \note[type=source, labelb=109, lem={\textbf{Re}}]{YK\textsuperscript{ccn \cdot YSV} 2.1 Ed. p. 23: sukhasādhyaṃ lakṣayogam idānīṃ śrṛṇu pārvati | pañcadhā lakṣayogaś ca ūrdhvalakṣādibhedataḥ (\textit{ūrddha} PT\textsuperscript{qcr \cdot YSV} Ed. p. 833) ||}  
      idānīṃ
      sukha\app{\lem[wit={ceteri},alt={°sādhyo}]{sādhyo}
        \rdg[wit={N2}]{°sādhya}
        \rdg[wit={P,B}]{°sādho}
        \rdg[wit={L}]{°sādhe}
        \rdg[wit={U1}]{°sādhyopa°}}
    \app{\lem[wit={ceteri}]{lakṣyayogaḥ}
        \rdg[wit={B,L}]{lakṣayogaḥ}
        \rdg[wit={U1}]{°lakṣayogaḥ}
        \rdg[wit={N2}]{lakṣanayogaḥ}}
      kathyate/
%------------------------------      
%asya lakṣyayogasya  paṃcabhedā     bhavanti   ūrdhvalakṣyam / adholakṣyam / lakṣyam /      bāhyalakṣyam /  aṃtaralakṣyam /  \E
%asya lakṣyayogasya  paṃcabhedā     bhavanti   ūrdhvalakṣyam   adholakṣyam / madhyalakṣyam  bāhyalakṣyam    aṃtaralakṣyam /  \P
%asya lakṣayogasya   paṃce bhedāḥ   bhavaṃtī   ūrdhvalakṣam//  adholakṣam// bāhyakṣam//                     aṃtaralakṣam //  \B
%asya lakṣayogasya   paṃcabhedāḥ    bhavaṃti   ūrdhvalakṣam    adholakṣam// madhyalakṣam//  bāhyakṣam//     aṃtaralakṣam //  \L
%     lakṣyayogasya  paṃcabhedā     bhavaṃti// urdhvalakṣya    adholakṣya   bāhyalakṣya     madhyalakṣya    antaralakṣya //  \N1
%     lakṣanayogasya paṃcabhedā     bhavati//  urdhvalakṣa     adholakṣa    bāhyalakṣa      madhyalakṣa     antaralakṣa //   \N2
%     lakṣyayogasya  paṃcabhedā     bhavaṃti// urdhvalakṣya    adholakṣya   bāhyalakṣya     madhyalakṣya    antaralakṣya //  \D
%a----lakṣayogasya   paṃcabhedā     bhavati    urdhvalakṣa                  bāhyalakya      madhyalakṣa     aṃtaralakṣya     \U1
%asya lakṣayogasya   paṃcabhedā     bhavaṃti// ūrdhvalakṣam//  adholakṣam/  bāhyalakṣyam /  madhyalakṣaṃ/   sarvalakṣyam /   \U2
%------------------------------
%Of this yoga of fixation (\textit{lakṣyayoga}) there are five subdivisions: 1. The upward directed fixation {\textit{ūrdhvalakṣya}), 2. the downward directed fixation (\textit{adholakṣya}),3. the central fixation (\textit{madhyalakṣya}) 4. the outer fixation (\textit{baḥyalakṣya}), 5. the inner fixation (\textit{antaralakṣya}).
%------------------------------
      \note[type=source, labelb=110, lem={\textbf{Re}}]{YK\textsuperscript{ccn \cdot YSV} 2.2 Ed. p. 23: ūrdhvalakṣam (\textit{ūrddha°} PT\textsuperscript{qcr \cdot YSV} Ed. p. 833) adholakṣaṃ (\textit{°lakṣo} PT\textsuperscript{qcr \cdot YSV} Ed. p. 833) vāhyalakṣaṃ (\textit{bāhya°} PT\textsuperscript{qcr \cdot YSV} Ed. p. 833) tathaiva ca | madhyalakṣaṃ (\textit{°lakṣas} PT\textsuperscript{qcr \cdot YSV} Ed. p. 833) tathā jñeyam (\textit{°lakṣas} PT\textsuperscript{qcr \cdot YSV} Ed. p. 833) antarlakṣaṃ (\textit{°lakṣas} PT\textsuperscript{qcr \cdot YSV} Ed. p. 833) tathaiva ca ||2||}
      \app{\lem[wit={Y}]{asya}
        \rdg[wit={X}]{\om}}
      \app{\lem[wit={ceteri},alt={lakṣya°}]{lakṣya}
        \rdg[wit={B,L,U2}]{lakṣa°}
        \rdg[wit={U1}]{alakṣa°}
        \rdg[wit={N2}]{lakṣana°}}yogasya
            \note[type=philcomm, labelb=111, lem={lakṣyayogasya}]{The designation of this type of yoga is transmitted in various variants. The original reading of the yoga is likely \textit{lakṣyayoga} since it crosses the stemma of the \alpha- and \beta-group. This reading is supported by the usage in the \citetitle{ssplonavla} 2.26-2.32 and \citetitle{yogacandrika} Ed. p. 2. However, \citetitle{ramatosana} (Ed. pp. 833-834)  and \citetitle{yogakarnika} (Ed. pp. 23-24)  as well as \citetitle{sarvangayoga} (Ed. pp. 104-105) use the term \textit{lakṣayoga}, indicating that both designations were common und regularly confused.}
      \app{\lem[wit={ceteri}]{pañcabhedā}
        \rdg[wit={B}]{paṃce bhedāḥ}
        \rdg[wit={L}]{paṃcabhedāḥ}}
     \app{\lem[wit={ceteri}]{bhavanti}
       \rdg[wit={B}]{bhavaṃtī}
       \rdg[wit={N2,U1}]{bhavati}}/
    1 \app{\lem[wit={E,P}]{ūrdhvalakṣyam}
       \rdg[wit={B,L,N2}]{ūrdhvalakṣam}
       \rdg[wit={D,N1}]{urdhvalakṣya}
       \rdg[wit={N2,U1}]{urdhvalakṣa}}/
    2 adho\app{\lem[wit={E,P}, alt={°lakṣyam}]{lakṣyam}
       \rdg[wit={B,L,U2}]{°lakṣam}
       \rdg[wit={D,N1}]{°lakṣya}
       \rdg[wit={N2}]{°lakṣa}
       \rdg[wit={U1}]{\om}}/
    3 \app{\lem[wit={U2}]{bāhyalakṣyam}
       \rdg[wit={D,N1}]{bāhyalakṣya}
       \rdg[wit={N2}]{bāhyalakṣa}
       \rdg[wit={U1}]{bāhyalakya}
       \rdg[wit={B}]{bāhyakṣam}
       \rdg[wit={E}]{lakṣyam}
       \rdg[wit={P}]{madhyalakṣyam}
       \rdg[wit={L}]{madhyalakṣam}}/
    4 \app{\lem[type={emendation}, resp={egoscr}]{madhyalakṣyam}
       \rdg[wit={D,N1}]{madhyalakṣya}
       \rdg[wit={N2,U1}]{madhyalakṣa}
       \rdg[wit={U2}]{madhyalakṣaṃ}
       \rdg[wit={E,P}]{bāhyalakṣyam}
       \rdg[wit={L}]{bāhyakṣam}
       \rdg[wit={B}]{\om}}/
    5 \app{\lem[wit={E,P}]{antaralakṣyam}
       \rdg[wit={D,N1,U1}]{antaralakṣya}
       \rdg[wit={B,L}]{aṃtaralakṣam}
       \rdg[wit={N2}]{antaralakṣa}
       \rdg[wit={U2}]{sarvalakṣyam}}/\textsuperscript{\begin{otherlanguage}{english}\coro{[\lowroman{3}]}\end{otherlanguage}}\vfill
\end{prose}
\nolinenumbers
  % \smallskip
\centerline{\textrm{\small{[\uproman{14}.\textsuperscript{\coro{[\lowroman{1}-\lowroman{7}}]}} Ūrdhvalakṣya]}}
\label{urdhvalaksya}
    \bigskip
    \linenumbers
    \begin{prose}
      \noindent
%------------------------------      
%prathamam ūrdhvalakṣyaṃ kathyate/  \E
%prathamam ūrdhvalakṣyaḥ kathyate/  \P
%atha      ūrdhvalakṣaṃ          // \L
%athama    urdhalakṣaṃ           // \B
%prathamaṃ urdhvalakṣaḥ  kathyate/  \N1
%prathamaṃ urdhvalakṣaḥ  kathyate/  \N2
%prathamaṃ urdhvalakṣaḥ  kathyate/  \D
%prathamaṃ urdhvalakṣya/ kathyate/  \U1
%prathamaṃ urdhvalakṣaṃ  kathyate/  \U2
%------------------------------
%At first the upward directed fixation{\textit{adholakṣya} is explained. 
%------------------------------
  \note[type=source, labelb=112, lem={\textbf{Re}}]{YK\textsuperscript{ccn \cdot YSV} 2.3 Ed. p. 23: lakṣaṇaṃ śrṛṇu caiṣāṃ hi phalaṃ jñātvā maheśvari | ākāśe dṛṣṭim āsthāya mana ūrdhvan (\textit{ūrddhan} PT\textsuperscript{qcr \cdot YSV} Ed. p. 834) tu kārayet ||3||}
  \app{\lem[wit={E,P},alt={prathamam}]{prathama\skp{m-ū}}
       \rdg[wit={D,N1,N2,U1,U2}]{prathamaṃ}
       \rdg[wit={L}]{atha}
       \rdg[wit={B}]{athama}}\app{\lem[wit={E},alt={ūrdhvalakṣyaṃ}]{\skm{m-ū}rdhvalakṣyaṃ}
       \rdg[wit={P}]{ūrdhvalakṣyaḥ}
       \rdg[wit={U1}]{urdhvalakṣya}
       \rdg[wit={L}]{ūrdhvalakṣaṃ}
       \rdg[wit={U2}]{urdhvalakṣaṃ}
       \rdg[wit={D,N1,N2}]{urdhvalakṣaḥ}
       \rdg[wit={B}]{urdhalakṣaṃ}}
     \app{\lem[wit={ceteri}]{kathyate}
       \rdg[wit={L,B}]{\om}}/
%------------------------------     
%ākāśamadhye dṛṣṭiḥ / \E
% \om                 \P
%ākāśamadhye dṛṣṭiḥ / \L
%ākāśamadhye dṛṣṭi    \B
%ākāśamadhye dṛṣṭiḥ / \N1
%ākāśamadhye dṛṣṭiḥ / \N2i
%ākāśamadhye dṛṣṭiḥ / \D
%ākāśamadhye dṛṣṭiḥ / \U1
%ākāśamadhye dṛṣṭiḥ / \U2
%------------------------------
%The gaze (\textit{dṛṣṭi)) is [directed onto] the middle of the sky. 
%------------------------------
  \app{\lem[wit={ceteri}]{ākāśamadhye}
    \rdg[wit={P}]{\om}}
  \app{\lem[wit={ceteri}]{dṛṣṭiḥ}
    \rdg[wit={B}]{dṛṣṭi}
    \rdg[wit={P}]{\om}}/
%------------------------------     
%kadā ca    mana    ūrdhvaṃ      kṛtvā sthāpayati /     \E x
%atha ca    mana    ūrdhvaṃ      kṛtvā sthāpyate /      \P x
%atha vā            ūrdhvaṃ mana kṛtvā sthāpyate        \L
%atha vā            ūrdhvamana   kṛtvā sthāpyate        \B
%atha ca // mana    urdhvaṃ      kṛtvā sthāpyate /      \N1 x
%atha ca mana       ūrdhvaṃ      kṛtvā sthāpyate /      \N2 x
%atha vā mana       ūrdhaṃ       kṛtvā sthāpyate        \D x
%atha ca maner------ddhvaṃ       kṛtvā sthāpyate        \U1
%atha    mana       urdhvaṃ      kṛtvā sthāpyate//      \U2 x
%------------------------------
%And then having caused the mind to be directed upwards, it is caused to be fixed there. 
%------------------------------
  \app{\lem[wit={P,N1,N2,U1}]{atha ca}
    \rdg[wit={B,D,L}]{atha vā}
    \rdg[wit={U2}]{atha}
    \rdg[wit={E}]{kadā ca}}
  \app{\lem[wit={E,P,N2}]{mana ūrdhvaṃ}
    \rdg[wit={D}]{mana ūrdhaṃ}
    \rdg[wit={N1,U2}]{mana urdhvaṃ}
    \rdg[wit={U1}]{manerddhvaṃ}
    \rdg[wit={B}]{ūrdhvamana}
    \rdg[wit={L}]{ūrdhvaṃ mana}}
  kṛtvā
  \app{\lem[wit={ceteri}]{sthāpyate}
    \rdg[wit={E}]{sthāpayati}}/
%------------------------------ 
%etasya lakṣyasya  dṛḍhakaraṇāt   parameśvarasya tejasā saha dṛṣṭer-aikyaṃ  bhavati /  \E
%etasya lakṣyasya  dṛḍhakaraṇāt   parameśvarasya tejasā saha dṛṣṭer-aikyaṃ  bhavati /  \P
%etasya lakṣasya   dṛḍhīkṛtvā//   parameśvarasya teja---saha dṛṣṭair-aikā   bhavati //  \L
%etasya lakṣasya   dṛḍhīkṛtvā//   parameśvarasya teja---saha dṛṣṭair-aikā   bhavati //  \B
%etasya lakṣyasya  dṛḍhīkaraṇāt / parameśvarasya tejasā saha dṛṣteḥ aikyaṃ  bhavati /  \N1
%etasya lakṣaṇasya dṛḍhīkaraṇāt   parameśvarasya tejasā saha dṛṣteḥ ekaṃ    bhavati //  \N2
%etasya lakṣasya   dṛḍhīkaraṇāt// parameśvarasya tejasā saha dṛṣṭeḥ aikyaṃ  bhavati // \D
%etasya lakṣasya   dṛḍhīkaraṇāt/  parameśvarasya tejasā saha dṛṣṭer-aikyaṃ  bhavati/ \U1
%etasya lakṣasya   dṛḍhīkaraṇāt   parameśvarasya tenasā saha dṛṣṭer-aikyaṃ  bhavati // \U2
%------------------------------
%Due to the exercise of stabilizing of this fixation (\textit{lakṣya}) arises unity of the gazing point (\textit{dṛṣṭi}) with the light of the highest lord (\textit{parameśvara}). 
%------------------------------
\note[type=source, labelb=113, lem={\textbf{Re}}]{YK\textsuperscript{ccn \cdot YSV} 2.3-2.4ab Ed. p. 23: ūrdhvalakṣaṃ  (\textit{ūrdha°} PT\textsuperscript{qcr \cdot YSV} Ed. p. 834)  bhaved eṣā parameśasya caikatā |}
  etasya
  \app{\lem[wit={E,P,N1}]{lakṣyasya}
    \rdg[wit={ceteri}]{lakṣasya}
    \rdg[wit={N2}]{lakṣaṇasya}}
  \app{\lem[wit={ceteri},alt={dṛḍhīkaraṇāt}]{dṛḍhīkaraṇā\skp{t-pa}}
    \rdg[wit={E,P}]{dṛḍhakaraṇāt}
    \rdg[wit={B,L}]{dṛḍhīkṛtvā}}\skm{t-pa}rameśvarasya
\app{\lem[wit={ceteri}]{tejasā}
  \rdg[wit={U2}]{tenasā}
  \rdg[wit={B,L}]{teja°}}
saha
\app{\lem[wit={E,P,U1,U2},alt={dṛṣṭer aikyaṃ}]{dṛṣṭer\skp{-}aikyaṃ}
  \rdg[wit={D,N1}]{dṛṣṭeḥ aikyaṃ}
  \rdg[wit={N2}]{dṛṣteḥ ekaṃ}
  \rdg[wit={B,L}]{dṛṣṭair aikā}}
bhavati/  
%------------------------------
%atha cākāśa----madhye    yaḥ kaścidadṛṣṭaḥ   padārtho bhavati /  \E x
%atha cākāśa----madhye    yaḥ kaścidadṛṣṭaḥ   padārtho bhavati /  \P  x
%atha vākāśa----madhye    yaḥ kacciddṛṣṭaḥ    padārtho bhavati    \L  x
%athā cākāśa----madhye    yaḥ kaccit dṛṣṭaḥ   padārtho bhavati    \B   x
%atha ca ākāśa--madhye    yaḥ kaścitadṛṣtaḥ   padārthe bhavati /  \N1   x
%atha// ākāśa---madhye    yaḥ kaścita adṛṣtaḥ padārtha bhavati /  \N2  x 
%atha ca ākāśa--madhye    yaḥ kaścitadṛsṭaḥ   padārtho bhavati /  \D    x
%atha ca/ ākāśa-madhye    yaḥ kaścidadṛsṭaḥ   padārtho bhavati    \U1    x
%atha cākāśa----madhye    yaḥ kaściddṛsṭa-----padārtho bhavati /  \U2
%------------------------------
%And then an indefinable invisible object arises in the middle of the sky.
%------------------------------
\app{\lem[wit={ceteri}]{atha}
  \rdg[wit={B}]{athā}}
\app{\lem[wit={E,P,B,U2},alt={cākāśa°}]{cākāśa}
  \rdg[wit={D,N1,U1}]{ca ākāśa°}
  \rdg[wit={L}]{vākāśa°}
  \rdg[wit={N2}]{ākāśa°}}madhye
yaḥ
\app{\lem[wit={ceteri},alt={kaścid adṛṣṭaḥ}]{kaścid\skp{-}adṛṣṭaḥ}
  \rdg[wit={B}]{kaccit dṛṣṭaḥ}
  \rdg[wit={B}]{kaccit dṛṣṭaḥ}
  \rdg[wit={N2}]{kaścita adṛṣtaḥ}
  \rdg[wit={U2}]{kaścid dṛsṭa°}}
\app{\lem[wit={ceteri}]{padārtho}
  \rdg[wit={N1}]{padārthe}
  \rdg[wit={N2}]{padārtha}}
bhavati/\textsuperscript{\begin{otherlanguage}{english}\coro{[\lowroman{5}]}\end{otherlanguage}}
%------------------------------
%sa sādhakasya dṛṣṭigocaro bhavati//  \E
%sa sādhakasya dṛṣṭigocaro bhavati//  \P
%   sādhakasya dṛṣṭigocaro bhavati//  \L
%   sādhakasya dṛṣṭigocaro bhavatī    \B
%sa sādhakasya dṛṣṭigocare bhavati // \D  saḥ-Sonderregel -> ḥ fällt aus vor allen Konsonanten
%sa sādhakasya dṛṣṭigocare bhavati // \N1
%   sādhakasya dṛṣṭigocarā bhavati // \N2
%sa sādhakasya dṛṣṭigocaro bhavati    \U1
%   sādhakasya dṛṣṭigocare bhavati // \U2
%------------------------------
%It arises in the range of sight of the practitioner.  
%------------------------------
\app{\lem[wit={ceteri}]{sa}
  \rdg[wit={B,L,N2,U2}]{\om}}
sādhakasya
\app{\lem[wit={D,N1,U2}]{dṛṣṭigocare}
  \rdg[wit={ceteri}]{dṛṣṭigocaro}
  \rdg[wit={N2}]{dṛṣṭigocarā}}
\app{\lem[wit={ceteri}]{bhavati}
  \rdg[wit={B}]{bhavatī}}/
%------------------------------
%ayam evordhvalakṣyaḥ      \E
%ayam evordhvalakṣyaḥ      \P
%ayam evordhvalakṣaḥ  //   \L
%ayam evordhalakṣaḥ  //    \B
%ayam evordhvalakṣya  //   \N1
%ayam eva vodhalakṣaṇam // \N2
%ayam evordhvalakṣyaḥ //   \D
%ayam evordhvalakṣyaḥ      \U1
%ayam evordhvalakṣya //    \U2
%------------------------------
%This is truly the upward directed fixation (\textit{ūrdhvalakṣya}).
%------------------------------
aya\skp{m-e}\app{\lem[wit={D,E,P,U1},alt={evordhvalakṣyaḥ}]{\skm{m-e}vordhvalakṣayaḥ}
  \rdg[wit={L}]{evordhvalakṣaḥ}
  \rdg[wit={B}]{evordhalakṣaḥ}
  \rdg[wit={N1,U2}]{evordhvalakṣya}
  \rdg[wit={N2}]{eva vodhalakṣaṇam}}/
    \end{prose}
  \end{edition}
  \begin{translation}
    \ekddiv{type=trans}
     %\bigskip
    \centerline{\textrm{\small{[\uproman{13}.\textsuperscript{\coro{\lowroman{1}-\lowroman{3}}} Lakṣyayoga]}}}
    \label{laksyayogatrans}
    \bigskip
    \begin{tlate}
      \noindent
   Now the yoga of targets (\textit{lakṣyayoga}), which is easily accomplished\footnote{The emphasis on the easiness of Lakṣ(y)ayoga is not just shared with the \textit{Yogasvarodaya} but also with Sundardās's \citetitle{sarvangayoga} 3.25a (Ed. p. 104): lakṣayoga hai sugam upāī |}, is explained. Of this yoga of targets, there are five subdivisions:
   1. The upward directed target (\textit{ūrdhvalakṣya}),
   2. the downward directed target (\textit{adholakṣya}),
   3. the outer target (\textit{baḥyalakṣya}),
   4. the central target (\textit{madhyalakṣya}),
   5. the inner target (\textit{antaralakṣya}).\textsuperscript{\coro{[\lowroman{3}]}}\footnote{The concepts and practices of Lakṣ(y)ayoga in Sundardās's \citetitle{sarvangayoga} 3.35 - 3.36 (Ed. pp. 104-105) are identical except that the descriptions a more concise, presented in a different order and subsumed under the category of Haṭhayoga. The \citetitle{ssplonavla}, one of Rāmacandra's central sources, particularly for the second half of his text, only describes three \textit{lakṣya}s in 2.26 - 2.31: \textit{antarlakṣya}, \textit{bahirlakṣya} and \textit{madhyamaṃ lakṣyaṃ}. Nevertheless, the practices are almost identical, and since he used the \citetitle{ssplonavla} as a source, one has to assume that the text influenced his descriptions. Lakṣyayoga in Nārāyaṇatīrtha's \citetitle{yogacandrika} refers to keeping the goal of liberation in mind during the practice of yoga. According to Nārāyaṇatīrtha, Lakṣyayoga also directs conscious thinking towards a specific goal. When the mind is focused on a goal, it can be focused on a direct experience of subtle divine scents and other sensory perceptions related to that goal. This focus is called \textit{viṣayavatī} and is the key to achieving stable mental concentration, cf. Ed. p. 54. Several categories of the five practices of Rāmacandra are also found in the \citetitle{yogacandrika} such as targeting various distances associated with the five elements in front of the nose, cf. Ed. pp. 62-63.}
 \end{tlate}
  \ekddiv{type=trans}
     \smallskip
     \centerline{\textrm{\small{[\uproman{14}.\textsuperscript{\coro{\lowroman{1}-\lowroman{7}}} Ūrdhvalakṣya]}}}
     \label{urdhvalaksyatrans}
    \bigskip    
  \begin{tlate}
    At first, the upward directed target is explained. The gaze [is aimed] into the middle of the sky. And then, having caused the mind to be directed upwards, it is caused to be fixed there. Due to the exercise of stabilizing this target arises unity of the gaze with the light of the highest lord. And then an indefinable invisible object arises in the middle of the sky.\textsuperscript{\coro{[\lowroman{5}]}} It arises in the range of sight of the practitioner. This is truly the upward directed target.\footnote{Sundardās shares the concept of \textit{ūrdhvalakṣ(y)a} as fixing the gaze in the sky is in his \citetitle{sarvangayoga} 3.27: 
\begin{quote}
ūrddha lakṣa karai ihīṃ bhāṃtī | duṣṭyākāśa rahai dina rātī |\\ 
bibidh prakāra hoi ujiyārā | gopi padāratha dīsahiṃ sārā || 27 ||
\end{quote}
A very similar practice appears already in \citetitle{bäumer2013} 84:
      \begin{quote}
ākāśaṃ vimalam paśyan kṛtvā dṛṣṭiṃ nirantarām |  \\
stabdhātmā tatkṣaṇād devi bhairavaṃ vapur āpnuyāt ||
\end{quote}
Although the term \textit{lakṣya} is not used, the central elements of the practice are found here: the gaze is constantly fixed on the sky, establishing a connection with the divine.} \vspace*{\fill} 
  \end{tlate}
  \end{translation}
  \ekdpb*{}
\end{alignment}
%%%%%%%%%%%%%%%%%%%%%%%%%%%%%%%%%%%%%%%%%%
%%%%%%%%%%%%%%%%%%%%%%%%%%%%%%%%%%%%%%%%%%
%%%%%%%%PAGEBREAK%%%%%%%PAGEBREAK%%%%%%%%%
%%%%%%%%%%%%%%%%%%%%%%%%%%%%%%%%%%%%%%%%%%
%%%%%%%%%%%%%%%%PAGEBREAK%%%%%%%%%%%%%%%%%
%%%%%%%%%%%%%%%%%%%%%%%%%%%%%%%%%%%%%%%%%%
%%%%%%%%PAGEBREAK%%%%%%%PAGEBREAK%%%%%%%%%
%%%%%%%%%%%%%%%%%%%%%%%%%%%%%%%%%%%%%%%%%%
%%%%%%%%%%%%%%%%%%%%%%%%%%%%%%%%%%%%%%%%%%
%%%%%%%%%%%%%%%%%%%%%%%%%%%%%%%%%%%%%%%%%%
%%%%%%%%%%%%%%%%%%%%%%%%%%%%%%%%%%%%%%%%%%
%%%%%%%%PAGEBREAK%%%%%%%PAGEBREAK%%%%%%%%%
%%%%%%%%%%%%%%%%%%%%%%%%%%%%%%%%%%%%%%%%%%
%%%%%%%%%%%%%%%%PAGEBREAK%%%%%%%%%%%%%%%%%
%%%%%%%%%%%%%%%%%%%%%%%%%%%%%%%%%%%%%%%%%%
%%%%%%%%PAGEBREAK%%%%%%%PAGEBREAK%%%%%%%%%
%%%%%%%%%%%%%%%%%%%%%%%%%%%%%%%%%%%%%%%%%%
%%%%%%%%%%%%%%%%%%%%%%%%%%%%%%%%%%%%%%%%%%
%%%%%%%%%%%%%%%%%%%%%%%%%%%%%%%%%%%%%%%%%%
%%%%%%%%%%%%%%%%%%%%%%%%%%%%%%%%%%%%%%%%%%
%%%%%%%%PAGEBREAK%%%%%%%PAGEBREAK%%%%%%%%%
%%%%%%%%%%%%%%%%%%%%%%%%%%%%%%%%%%%%%%%%%%
%%%%%%%%%%%%%%%%PAGEBREAK%%%%%%%%%%%%%%%%%
%%%%%%%%%%%%%%%%%%%%%%%%%%%%%%%%%%%%%%%%%%
%%%%%%%%PAGEBREAK%%%%%%%PAGEBREAK%%%%%%%%%
%%%%%%%%%%%%%%%%%%%%%%%%%%%%%%%%%%%%%%%%%%
%%%%%%%%%%%%%%%%%%%%%%%%%%%%%%%%%%%%%%%%%%
\begin{alignment}[
  texts=edition[class="edition"];
  translation[class="translation"],
  ]
  \begin{edition}
    \ekddiv{type=ed}
    \centerline{\textrm{\small{[\uproman{15}.\textsuperscript{\coro{\lowroman{1}-\lowroman{6}}} Adholakṣya]}}}
    \label{adholaksya}
    \bigskip
    \begin{prose}
      \noindent
%------------------------------
%                            nāsikāyāḥ  upari     dvādaśāṃgulamūlaparyantaṃ dṛṣṭiḥ sthirā karttavyā /   \E
%       athādholakṣaḥ        nāsikāyā   upari     dvādaśāṃgulaparyantaṃ     dṛṣṭiḥ sthirā karttavyā /   \P
%       athādholakṣaḥ //     nāsikāyā   upari     dvādaśāṃgulaparyaṃtaṃ     dṛṣṭiḥ sthirā karttavyā     \L
%       athādholakṣa //      nāsikāyā   upari     dvādaśāṃgulaparyaṃtaṃ     dṛṣṭiḥ sthirā karttavyā     \B
%       atha adholakṣyaḥ //  nāsikāyā   upari     dvādaśaṃgulaparyaṃtaṃ     dṛṣṭiḥ sthirā karttavyā //  \N1
%       atha adholakṣanaḥ // nāsikāyā   upari     dvādaśāṃgulaparyaṃtaṃ     dṛṣṭiḥ sthirā karttavyā //  \N2
%       atha adholakṣaḥ //   nāsikāyā   upari     dvādaśaṃgulaparyaṃtaṃ     dṛṣṭiḥ sthirā karttavyā //  \D
%       atha adholakṣa       nāsikāyā   upari     dvādaśaṃgulaparyaṃtaṃ     dṛṣṭi--sthirā karttavyā     \U1
%                            nāsikāyāḥ  upariṣṭāt    daśāṃgulaparyaṃtaṃ     dṛṣṭiḥ sthirā karttavyā //  \U2
%------------------------------
%Now the downward directed fixation object (\textit{adholakṣya}). One should stabilize the gaze within the circumference (\textit{paryanta}) of twelve \textit{aṅgula}s beyond the nose.
%------------------------------
   \note[type=source, labelb=114, lem={\textbf{Re}}]{YK\textsuperscript{ccn \cdot YSV} 2.4cd-6ab Ed. p. 23: nāsikopari deveśi dvādaśāṅgulamānataḥ ||4|| dṛṣṭisthiran (\textit{dṛṣṭiḥ sthirā} PT\textsuperscript{qcr \cdot YSV} Ed. p. 834) tu karttavyam (\textit{karttavyā} PT\textsuperscript{qcr \cdot YSV} Ed. p. 834) adholakṣam idaṃ bhajet (\textit{bhaja} PT\textsuperscript{qcr \cdot YSV} Ed. p. 834) | tathā ca (\textit{athavā} PT\textsuperscript{qcr \cdot YSV} Ed. p. 834) nāsikāgre tu sthirā dṛṣṭir iyaṃ śṛṇu  (\textit{bhavet} PT\textsuperscript{qcr \cdot YSV} Ed. p. 834) ||5|| yasya bhavet sthirā dṛṣṭiś cirāyuḥ (\textit{sthirā dṛṣṭiś cirāyuḥ syāt tathāsau} PT\textsuperscript{qcr \cdot YSV} Ed. p. 834) sthiradṛṣṭimān |}   
\app{\lem[type=emendation, resp=egoscr]{athādholakṣyaḥ}
  \rdg[wit={N1}]{atha adholakṣyaḥ}
  \rdg[wit={P,L}]{athādholakṣaḥ}
  \rdg[wit={B}]{athādholakṣa}
  \rdg[wit={N2}]{atha adholakṣanaḥ}
  \rdg[wit={D}]{atha adholakṣaḥ}
  \rdg[wit={U1}]{atha adholakṣa}
  \rdg[wit={E,U2}]{\om}}/
\app{\lem[wit={ceteri}]{nāsikāyā}
  \rdg[wit={E,U2}]{nāsikāyāḥ}}
\app{\lem[wit={ceteri}]{upari}
  \rdg[wit={U2}]{upariṣṭāt}}
\app{\lem[wit={ceteri}]{dvādaśāṅgulaparyantaṃ}
  \rdg[wit={E}]{dvādaśāṃgulamūlaparyantaṃ}
  \rdg[wit={U2}]{daśāṃgulaparyaṃtaṃ}}
\app{\lem[wit={ceteri}]{dṛṣṭiḥ}
  \rdg[wit={U1}]{dṛṣṭi°}}
sthirā karttavyā/
%------------------------------
%atha vā nāsikāyā agre dṛṣṭiḥ sthirā karttavyā / \E
%atha vā nāsikāyā agre dṛṣṭiḥ sthirā karttavyā / \P
%\om / \L
%\om / \B
%atha vā nāsikāyā  agre dṛṣṭiḥ sthirā karttavyā // \N1
%atha vā nāsikā    agre dṛṣṭi-sthirā karttavyā      \N2
%atha vā nāsikāyā  agre dṛṣṭiḥ sthirā karttavyā // \D
%atha vā nāśikāyāḥ/ agre dṛṣṭiḥ/ sthirā karttavyā / \U1
%atha vā nāsikāyā  agre dṛṣṭiḥ sthirā karttavyā // \U2
%------------------------------
%Or one should stabilize the gaze onto the tip of the nose.
%------------------------------
\app{\lem[wit={ceteri}]{atha vā}
  \rdg[wit={L,B}]{\om}}
\app{\lem[wit={ceteri}]{nāsikāyā}
  \rdg[wit={U1}]{nāsikāyāḥ}
  \rdg[wit={N2}]{nāsika}}
\app{\lem[wit={ceteri}]{agre}
  \rdg[wit={B,L}]{\om}}
\app{\lem[wit={ceteri}]{dṛṣṭiḥ}
  \rdg[wit={N2}]{dṛṣṭi°}}
\app{\lem[wit={ceteri}]{sthirā}
  \rdg[wit={B,L}]{\om}}
\app{\lem[wit={ceteri}]{karttavyā}
  \rdg[wit={B,L}]{\om}}/ 
%------------------------------
%lakṣadūyasya  dṛḍhīkaraṇāt / dṛṣṭiḥ sthirā bhavati / \E
%lakṣadvayasya dṛṣṭīkaraṇāt / dṛṣṭiḥ sthirā bhavati / \P
%lakṣadvayasya dṛḍhīkaraṇāt   dṛṣṭi--sthiro bhavati / \L
%lakṣadvayasya dṛḍhīkaraṇān---dṛṣṭiḥ sthiro bhavatī   \B
%lakṣadvayasya dṛdhīkaraṇāt   dṛṣṭiḥ sthirā bhavati / \N1
%lakṣadvayasya dṛḍhīkaraṇād---dṛṣṭi--sthirā bhavati / \N2
%lakṣadvayasya dṛḍhīkaraṇāt   dṛṣṭiḥ sthirā bhavati / \D
%lakṣadvayasya dṛḍhīkaraṇāt   dṛṣṭiḥ sthirā bhavati / \U1
%lakṣadvayasya dṛḍhīkaraṇāt   dṛṣṭi--sthirā bhavati // \U2
%------------------------------
%The target becomes stable due to firm exercise [on one] of the twofold aims [of fixation]. 
%------------------------------
\app{\lem[wit={ceteri}]{lakṣadvayasya}   %emend to lakṣyadvayasya??? 
  \rdg[wit={E}]{lakṣadūyasya}} 
\app{\lem[wit={N2}, alt={dṛḍhīkaraṇād}]{dṛḍhīkaraṇā\skm{d-ṛ}}
  \rdg[wit={E,L,N1,D,U1,U2}]{dṛḍhīkaraṇāt}
  \rdg[wit={P}]{dṛṣṭīkaraṇāt}
  \rdg[wit={B}]{dṛḍhīkaraṇān}
}\app{\lem[wit={ceteri}, alt={dṛṣṭiḥ}]{\skp{d-ṛ}ṣṭiḥ}
  \rdg[wit={L,N2,U2}]{dṛṣṭi°}}
\app{\lem[wit={ceteri}]{sthirā}  
  \rdg[wit={B}]{sthiro}
  \rdg[wit={L}]{°sthiro}}
\app{\lem[wit={ceteri}]{bhavati}
  \rdg[wit={B}]{bhavatī}}/
%------------------------------
%pavanaḥ sthiro bhavati / \E
%pavanaḥ sthiro bhavati / \P
%\om                    / \L
%\om                    / \B
%pavanaḥ sthiro bhavati / \N1
%pavana--sthiro bhavati /   \N2
%pavanaḥ sthiro bhavati / \D
%pavana--sthiro bhavati  / \U1
%pavana--sthiro bhavati  / \U2
%------------------------------
%The breath becomes stable. 
%------------------------------
\app{\lem[wit={D,E,P,N1}]{pavanaḥ}
  \rdg[wit={N2,U1,U2}]{pavana°}
  \rdg[wit={B,L}]{\om}}
\app{\lem[wit={ceteri}]{sthiro}
  \rdg[wit={B,L}]{\om}}
\app{\lem[wit={ceteri}]{bhavati}
  \rdg[wit={B,L}]{\om}}/\textsuperscript{\begin{otherlanguage}{english}\coro{[\lowroman{5}]}\end{otherlanguage}}
%------------------------------
%āyurvarddhate / \E
%āyurvarddhate / \P
%āyurvarddhate / \L
%āyurvardhate /  \B
%āyurvardhate /  \N1
%āyurvardhate /  \N2
%āyurvardhate /  \D
%āyurvarddhate   \U1
%āyurvarddhate //  \U2
%------------------------------
%Vitality increases. 
%------------------------------
āyur-varddhate\dd{}\vfill
\end{prose}
\nolinenumbers
\bigskip
\centerline{\textrm{\small{[\uproman{16}.\textsuperscript{\coro{\lowroman{1}-\lowroman{3}}} Statements with Reference to other Lakṣyas]}}}
\label{laksyastatements}
    \bigskip
    \linenumbers
    \begin{prose}
      \noindent
%------------------------------
%etad dūyam       api bāhyalakṣyam eva  bhavati      bāhyāṃtara       ākāśe         śūnyalakṣyaṃ    karttavyaḥ / \E
%etad dvayam      api bāhyalakṣyam eva  bhavati      bāhyābhyaṃtare   ākāśe cet     śūnyalakṣyaṃ    karttavyaḥ / \P
%etad dvayam      api bāhyalakṣam  eva  bhavati//    bāhyābhyaṃtare   ākāśacen      śūnyaṃ lakṣaṃ   karttavyā // \L
%etad dvayadvayam api bāhyalakṣam  eva  bhavatī//    bāhyābhyaṃtare   ākāśacvat     śūnyaṃ lakṣaṃ   karttavyā // \B
%etat advayam     eva bāhyalakṣam  api  kathyate //  bāhyo bhyaṃtaraṃ ākāśavat------śūnyalakṣyaḥ    karttavyaḥ / \N1
%etad dvayam      eva bāhyalakṣam  api  kathyate //  bāhyābhyaṃtaram--ākāśavat------śūnyalakṣaḥ     karttavyaḥ   \N2
%etat advayam     eva bāhyalakṣam  api  kathyate //  bāhyo bhyaṃtaraṃ ākāśavat //   śūnyalakṣyaḥ    karttavyaḥ / \D
%etat dvayam      eva bāhyalakṣyam api  kathyate/    bāhyābhyaṃtare   ākāśavat------śūnyalakṣyaḥ    karttavyaḥ  \U1
%etat dvayam      api bāhyalakṣyam eva  bhavati//    bāhyābhyaṃtare   ākāśe cet     śūnyalakṣyaṃ    karttavyaḥ / \U2
%------------------------------
%This (etad) pair (dvayam) is also called and explained (kathyate) as the outer (bāhya) goal (lakṣyam). The target of emptiness shall be executed internally and externally like space.
%------------------------------
\note[type=source, labelb=115, lem={\textbf{Re}}]{YK\textsuperscript{ccn \cdot YSV} 2.6cd-7 Ed. p. 23: bāhyalakṣaṃ svayaṃ jñeyaṃ yāti tattvanirāsinām (\textit{°nivāsinām} PT\textsuperscript{qcr \cdot YSV} Ed. p. 834) ||6|| kāmināṃ tu bahir dṛṣṭiś cintādiṣu susiddhidā | etad bāhyamadhyalakṣaṃ dṛṣṭicintānirākulaḥ (\textit{iṣṭacintā nirākulam} PT\textsuperscript{qcr \cdot YSV} Ed. p. 834) ||7||}
\app{\lem[wit={L,P,N2},alt={etad dvayam}]{etad-dvaya\skp{m-e}}
  \rdg[wit={E}]{etad dūyam}
  \rdg[wit={B}]{etad dvayadvaya}
  \rdg[wit={D,N2}]{etat advayam}
  \rdg[wit={U1,U2}]{etat dvayam}}\app{\lem[wit={X}, alt={eva}]{\skm{m-e}va}
  \rdg[wit={Y}]{api}} 
\app{\lem[wit={E,P,U1,U2},alt={bāhyalakṣyam}]{bāhyalakṣya\skp{m-a}}
  \rdg[wit={ceteri}]{°lakṣam}}\app{\lem[wit={X},alt={api}]{\skm{m-a}pi}
  \rdg[wit={Y}]{eva}}
\app{\lem[wit={X}]{kathyate}
  \rdg[wit={Y}]{bhavati}
  \rdg[wit={B}]{bhavatī}}/
\app{\lem[wit={N2},alt={bāhyābhyantaram}]{bāhyābhyantara\skm{m-ā}}                %Übersetzung nochmal überdenken! 
  \rdg[wit={D,N1}]{bāhyo bhyaṃtaraṃ}
  \rdg[wit={B,L,P,U1,U2}]{bāhyābhyaṃtare}
  \rdg[wit={E}]{bāhyāṃtara}}\app{\lem[wit={X},alt={ākāśavat}]{\skp{m-ā}kāśavat}
  \rdg[wit={B}]{ākāśacvat}
  \rdg[wit={L}]{ākāśacen}
  \rdg[wit={P,U2}]{ākāśe cet}
  \rdg[wit={E}]{ākāśe}}
\app{\lem[wit={D,N1,U1}]{śūnyalakṣyaḥ}
  \rdg[wit={E,P,U2}]{śūnyalakṣyaṃ}
  \rdg[wit={N2}]{śūnyalakṣaḥ}
  \rdg[wit={B,L}]{śūnyaṃ lakṣaṃ}}
\app{\lem[wit={ceteri}]{karttavyaḥ}
  \rdg[wit={B,L}]{karttavyā}}/
%------------------------------
%jāgraddaśāyāṃ    calanadaśāyāṃ   bhojanadaśāyāṃ   sthitikāle sarvasthāne   śūnyasya dhyānakāraṇāt //                              \E
%jāgraddaśāyāṃ    calanadaśāyāṃ   bhojanaṃ daśāyāṃ sthitikāle sarvasthāne   śūnyasya dhyānakāraṇāt //                              \P
%jāgradādidaśāyāṃ calanadaśāyāṃ// bhojanadaśāyāṃ   sthitikāle sarvasthāneṣu śūnyasya dhyānakāraṇāt //                              \L
%jāgradādidaśāyāṃ calanadaśāyāṃ// bhojanadaśāyāṃ   sthitikāle sarvasthāneṣu śūnyasya dhyānakaraṇāt //                              \B
%jāgraddaśāyāṃ    cakabadaśāyāṃ   bhojanadaśāyāṃ   sthitikāle sarvvasthāne  śūnyasya dhyānakaraṇāt  maraṇatrāso na bhavati//       \N1
%jāyadaśāyāṃ      calanadaśāyāṃ/  bhojanadaśāyāṃ   sthitikāle sarvasthāne   śūnyasya dhyānakaraṇāt  maraṇatrāśo na bhavati//       \N2
%jāgraddaśāyāṃ    calanadaśāyāṃ   bhojanadaśāyāṃ   sthitikāle sarvvasthāne  śūnyasya dhyānakaraṇāt  maraṇatrāso na bhavati// śūnya \D
%jāgraddaśāyāṃ    calanadaśāyāṃ                    sthitikāle sarvasthāne   śūnyasya dhyānakaraṇāt/ maraṇasautrāṃ na bhavati vā    \U1
%jāgṛaddaśāyāṃ    calanadaśāyāṃ   bhojanadaśāyāṃ   sthitikāle sarvasthāne   śūnyasya dhyānakaraṇāt//                               \U2
%------------------------------
%The fear of dying does not arise due to the exercise of meditation on the void at all places during ones life - while eating, moving and waking. 
%------------------------------
\note[type=source, labelb=115a, lem={\textbf{Re}}]{PT\textsuperscript{qcr \cdot YSV} (Ed. p. 834): antarlakṣaṃ śṛṇu śukradigvidigādivarjitam (\textit{subhru°} YK\textsuperscript{ccn \cdot YSV} 2.8a Ed. p. 24) | calaj jāgratsuṣupteṣu bhojaneṣu ca sarvadā | sarvāvasthāsu deveśi cittaṃ śūnye niyojayet | karttā kārayitā śunyaḥ (\textit{śūnyaṃ} YK\textsuperscript{ccn \cdot YSV} 2.10a Ed. p. 24) mūrttimān śūnya īśvaraḥ | harṣaśokaghaṭastho 'yaṃ janmamṛtyū labhet svayam | ghaṭasthā cintyayor mūrttir hatacintāsvarūpadhṛk (\textit{ghaṭasthāṃ cintayen mūrttimitaś} YK\textsuperscript{ccn \cdot YSV} 2.11ab Ed. p. 23) | viṣayaṃ viṣavad duṣṭaṃ (\textit{dṛṣṭvā} YK\textsuperscript{ccn \cdot YSV} 2.11c Ed. p. 23) tyaktvā jñātvā tu mārutam | saṃjñāśūnyamanā bhūtvā puṇyapāpair na lipyate | bāhyam ābhyantaraṃ khaṃ (\textit{yad} YK\textsuperscript{ccn \cdot YSV} 2.12c Ed. p. 24) hi antarlakṣam iti smṛtam | etad dyānāt sadā kiñcid duḥkhaṃ na syāc chivo bhavet | śūnyan tu saccidānandaṃ niḥśabdaṃ brahmaśabditam | saśabdaṃ jñeyam ākāśam (\textit{ākāśa} YK\textsuperscript{ccn \cdot YSV} 2.13c Ed. p. 24) iti bhedadvayan tv iha |}
\app{\lem[wit={ceteri}]{jāgraddaśāyāṃ}
    \rdg[wit={N2}]{jāgṛaddaśāyāṃ}
    \rdg[wit={N2}]{jāyadaśāyāṃ}
    \rdg[wit={B,L}]{jāgradādidaśāyāṃ}}
\app{\lem[wit={ceteri}]{calanadaśāyāṃ}
    \rdg[wit={N1}]{cakabadaśāyāṃ}}
\app{\lem[wit={ceteri}]{bhojanadaśāyāṃ}
    \rdg[wit={P}]{bhojanaṃ daśāyāṃ}
    \rdg[wit={U1}]{\om}}
  sthitikāle
\app{\lem[wit={ceteri}]{sarvasthāne}
    \rdg[wit={B,L}]{sarvasthāneṣu}}
  śūnyasya dhyānakāraṇāt
\app{\lem[wit={D,N1}]{maraṇatrāso}
    \rdg[wit={N2}]{maraṇatrāśo}
    \rdg[wit={U1}]{maraṇasautrāṃ}
    \rdg[wit={B,E,L,P,U2}]{\om}}
\app{\lem[wit={ceteri}]{na}
    \rdg[wit={B,E,P,U2}]{\om}}
\app{\lem[wit={N1,N2}]{bhavati}
    \rdg[wit={D}]{bhavati || śūnya}
    \rdg[wit={U1}]{bhavati vā}
    \rdg[wit={Y}]{\om}}\dd{}
 \end{prose}
  \end{edition}
   \begin{translation}
    \ekddiv{type=trans}
   \bigskip
   \centerline{\textrm{\small{[\uproman{16}.\textsuperscript{\coro{\lowroman{1}-\lowroman{6}}} Adholakṣya]}}}
   \label{adholaksyatrans}
    \bigskip
    \begin{tlate}
      \noindent
    Now the downward-directed target. One should stabilize the gaze within the circumference (\textit{paryanta}) of twelve \textit{aṅgula}s beyond the nose. Alternatively, one should stabilize the gaze onto the tip of the nose. The target becomes stable due to firm exercise [on one] of the twofold targets. The breath becomes stable.\textsuperscript{\coro{[\lowroman{5}]}} Vitality increases.\footnote{In Sundardās's \citetitle{sarvangayoga} 2.26 (Ed. p. 104), one finds the following verse with a very similar concept:
      \begin{quote}
        prathamahīṃ adho lakṣa kauṃ jānaiṃ | nāśā agra dṛṣṭi sthira ānaiṃ |\\
yātoṃ mana pavanā thira hoī | adho lakṣa jo sādhai koī || 26 ||
\end{quote}
 He shares the technique of fixing the gaze onto the nose and the notion of the breath becoming stabilized.} 
\end{tlate}
 %%%%%%%%%%%%%%%%%%%
    %%%%%%%%%%%%%%%%%%%
    %%%%%%%%%%%%%%%%%%%
    %%%%%%%%%%%%%%%%%%%
\ekddiv{type=trans}
   \bigskip
   \centerline{\textrm{\small{[\uproman{17}.\textsuperscript{\coro{\lowroman{1}-\lowroman{3}}} Statements with Reference to other Lakṣyas]}}}
   \label{laksyastatementstrans}
    \bigskip
  \begin{tlate}
    This pair is also taught as an external target.\footnote{This short statement associates the two mentioned practices of \textit{adholakṣya} with \textit{bāhyalakṣya}, which is described later on in the text. Indeed, these later descriptions instruct the practitioner to target one of the five elements at various distances in front of the nose or above the head (cf. p. \pageref{bahya2}).} The target of emptiness shall be executed internally and externally like space.\footnote{This description undoubtedly derives from the \textit{Yogasvarodaya} as quoted in the \citetitle{ramatosana} (Ed. p. 834), where it is declared a practice of \textit{antaralakṣ(y)a}.} The fear of dying does not arise due to the exercise of meditation on the void at all places during one's life - while eating, moving and waking.\footnote{Next, Rāmacandra introduces a new topic about the attributes of the body of the Rājayogin instead of following the missing descriptions on \textit{bāhya}-, \textit{madhya}- and \textit{antaralakṣya} as one would exspect. However, the subject and particularly the descriptions of the three missing \textit{lakṣyayoga}s continue later in the text (p. \pageref{bahya2}). One might speculate that an early copyist could have confused folios, but as was noted earlier, Rāmacandra followed the content and structure of \textit{Yogasvarodaya} as quoted in \citetitle{ramatosana}, which presents the exact structure of topics. Judging the structure of the later topics that follow the later \textit{lakṣya}s, rearranging the text would make much more sense! Another noteworthy observation is that in the \textit{Yogasvarodaya} as quoted in \citetitle{ramatosana} \textit{ūrdhvalakṣa}, \textit{adholakṣa}, \textit{bāhyalakṣa} and \textit{madhyalakṣa} are presented rather briefly. After that, \textit{antaralakṣa} is explained in more detail. Rāmacandra follows this template roughly. However, he merges, maybe even confuses? and shortens the descriptions of \textit{bāhya}-, \textit{madhya}- and \textit{antaralakṣ(y)a} into this short description of \textit{bāhyalakṣya}.}\footnote{The concept of five \textit{lakṣ(y)a}s to my current knowledge appears only in the lost \textit{Yogasvarodaya} and from there made its way into \citetitle{ramatosana}, \citetitle{yogakarnika} and \textit{Yogatattvabindu}. The other texts including it are \citetitle{sarvangayoga} and \citetitle{yogacandrika}. In all those texts, this practice becomes a genuine yoga: Lakṣ(y)yoga. If one encounters the concept of three \textit{lakṣ(y)a} as found in \citetitle{ssplonavla} (cf. Ed. pp. 37-41) or various Yoga Upaniṣads, e.g. \citetitle{mandalabrah} (cf. 2.6-2.14) it is never declared as an own type of yoga.} \vspace*{\fill}
    \end{tlate}
  \end{translation}
  \ekdpb*{}
\end{alignment}
%%%%%%%%%%%%%%%%%%%%%%%%%%%%%%%%%%%%%%%%%%
%%%%%%%%%%%%%%%%%%%%%%%%%%%%%%%%%%%%%%%%%%
%%%%%%%%PAGEBREAK%%%%%%%PAGEBREAK%%%%%%%%%
%%%%%%%%%%%%%%%%%%%%%%%%%%%%%%%%%%%%%%%%%%
%%%%%%%%%%%%%%%%PAGEBREAK%%%%%%%%%%%%%%%%%
%%%%%%%%%%%%%%%%%%%%%%%%%%%%%%%%%%%%%%%%%%
%%%%%%%%PAGEBREAK%%%%%%%PAGEBREAK%%%%%%%%%
%%%%%%%%%%%%%%%%%%%%%%%%%%%%%%%%%%%%%%%%%%
%%%%%%%%%%%%%%%%%%%%%%%%%%%%%%%%%%%%%%%%%%
%%%%%%%%%%%%%%%%%%%%%%%%%%%%%%%%%%%%%%%%%%
%%%%%%%%%%%%%%%%%%%%%%%%%%%%%%%%%%%%%%%%%%
%%%%%%%%PAGEBREAK%%%%%%%PAGEBREAK%%%%%%%%%
%%%%%%%%%%%%%%%%%%%%%%%%%%%%%%%%%%%%%%%%%%
%%%%%%%%%%%%%%%%PAGEBREAK%%%%%%%%%%%%%%%%%
%%%%%%%%%%%%%%%%%%%%%%%%%%%%%%%%%%%%%%%%%%
%%%%%%%%PAGEBREAK%%%%%%%PAGEBREAK%%%%%%%%%
%%%%%%%%%%%%%%%%%%%%%%%%%%%%%%%%%%%%%%%%%%
%%%%%%%%%%%%%%%%%%%%%%%%%%%%%%%%%%%%%%%%%%
%%%%%%%%%%%%%%%%%%%%%%%%%%%%%%%%%%%%%%%%%%
%%%%%%%%%%%%%%%%%%%%%%%%%%%%%%%%%%%%%%%%%%
%%%%%%%%PAGEBREAK%%%%%%%PAGEBREAK%%%%%%%%%
%%%%%%%%%%%%%%%%%%%%%%%%%%%%%%%%%%%%%%%%%%
%%%%%%%%%%%%%%%%PAGEBREAK%%%%%%%%%%%%%%%%%
%%%%%%%%%%%%%%%%%%%%%%%%%%%%%%%%%%%%%%%%%%
%%%%%%%%PAGEBREAK%%%%%%%PAGEBREAK%%%%%%%%%
%%%%%%%%%%%%%%%%%%%%%%%%%%%%%%%%%%%%%%%%%%
%%%%%%%%%%%%%%%%%%%%%%%%%%%%%%%%%%%%%%%%%%
\begin{alignment}[
  texts=edition[class="edition"];
  translation[class="translation"],
  ]
  \begin{edition}
    \ekddiv{type=ed}
    \centerline{\textrm{\small{[\uproman{18}.\textsuperscript{\coro{\lowroman{1}-\lowroman{17}}} The Sign of a Rājayogin's Body]}}}
    \label{rajabody}
    \bigskip
    \begin{prose}
      \noindent
%------------------------------  
%idānīṃ rājayogayuktasya           śarīre yaccihnaṃ  tat    kathyate / \E
%idānīṃ rājayogayuktasya puruṣasya yaccharīracihnaṃ         kathyate / \P
%idānīṃ rājayogayuktasya puruṣasya          cinhnaṃ         kathyate / \L
%idānīṃ rājayogayuktasya puruṣasya          cinhnaṃ         kathyate // \B
%idānīṃ rājayogayuktasya puruṣasya yaccarīracihnaṃ   tat    kathyate / \N1
%idānīṃ rājayogayuktasya puruṣasya yaccharīracihūṃ   tat    kathyate// \N2
%idānīṃ rājayogayuktasya puruṣasya yaccarīracihnaṃ   tat    kathyate / \D
%idānīṃ rājayogayuktasya puruṣasya yaccharīre cinhaṃ tata   kathyate \U1
%idānīṃ rājayogayuktasya puruṣasya yat śarīracinhaṃ         kathyate / \U2
%------------------------------
%Now the sign of the body of the person who is endowed with Rājayoga is taught.
%------------------------------
\note[type=source, labelb=116, lem={\textbf{Re}}]{PT\textsuperscript{qcr \cdot YSV} (Ed. p. 834): 
  idānīṃ kathayiṣyāmi rājayogasya lakṣaṇam | rājayoge kṛte puṃbhiḥ siddhicihnaṃ bhaved iti |}
      idānīṃ rājayogayuktasya
  \app{\lem[wit={ceteri}]{puruṣasya}
    \rdg[wit={E}]{\om}}
  \app{\lem[wit={D,N1,P},alt={yac carīracihnaṃ}]{yac-carīracihnaṃ}
    \rdg[wit={U2}]{yat śarīracinhaṃ}
    \rdg[wit={E}]{śarīre yac cihnaṃ}
    \rdg[wit={U1}]{yac charīre cinhaṃ}
    \rdg[wit={N2}]{yac charīracihūṃ}
    \rdg[wit={B,L}]{cinhnaṃ}}
  \app{\lem[wit={D,E,N1,N2}]{tat}
    \rdg[wit={U1}]{tata}
    \rdg[wit={ceteri}]{\om}}
  kathyate/
%------------------------------  
%tatsarvatra pūrṇo bhavati / \E
%tatsarvatra pūrṇā bhavati / \P
%tatsarvatra pūrṇo bhavati / \L
%tatsarvatra pūrṇo bhavatī / \B
%  sarvvatra pūrṇo bhavati / \N1
%  sarvvatra pūrṇā bhavati  \N2
%  sarvvatra pūrṇo bhavati  \D
%  sarvvatra pūrṇo bhavati   \U1
%tatsarvatra pūrṇo bhavati// \U2
%------------------------------
%Abundance arises at all times. %Alternative=permanent Abundance arises because of that.   
%------------------------------
\note[type=source, labelb=117, lem={\textbf{Re}}]{PT\textsuperscript{qcr \cdot YSV} (Ed. p. 834): paripūrṇaṃ bhavec cittaṃ jagatstho 'pi jagadbahiḥ |}
  \app{\lem[wit={X},alt={sarvatra°}]{sarvatra}
  \rdg[wit={Y}]{tatsarvatra°}}
\app{\lem[wit={ceteri}, alt={°pūrṇo}]{pūrṇo}
  \rdg[wit={P,N2}]{pūrṇā}}
\app{\lem[wit={ceteri}]{bhavati}
  \rdg[wit={B}]{bhavatī}}/
%------------------------------  
%pṛthivyāḥ dūre tiṣṭhati / \E
%pṛthivyāḥ hare tiṣṭhati / \P
%\om                      \L
%\om                      \B
%pṛthivyāḥ dūre  tiṣṭhati / \N1
%pṛthivyāḥ dūra  tiṣṭhati / \N2
%pṛthivyāḥ dūre  tiṣṭhati / \D
%pṛthivyāḥ ddūre tiṣṭhati / \U1 %emend to na tiṣṭhati? 
%pṛthivyā dūraṃ  tiṣṭhati // \U2 !!dūraṃ
%------------------------------
%No distance exist on earth.
%------------------------------
\app{\lem[type=conjecture, resp=egoscrconj]{pṛthivyāṃ}
  \rdg[wit={ceteri}]{pṛthivyāḥ}
  \rdg[wit={U2}]{pṛthivyā}} 
\app{\lem[wit={U2}]{dūraṃ}
  \rdg[wit={D,E,N1}]{dūre}
  \rdg[wit={U1}]{ddūre}
  \rdg[wit={N2}]{dūra}}
\app{\lem[type=conjecture, resp=egoscrconj]{na tiṣṭhati}
  \rdg[wit={ceteri}]{tiṣṭhati}}/
\note[type=philcomm, labelb=117a, lem={pṛthivyāḥ dūraṃ tiṣṭhati}]{The sentence is missing in B and L.}
%------------------------------
%pṛthivyāṃ vyāpya tiṣṭhati / \E
%pṛthi-----vyāpya tiṣṭhati / \P
%\om                         \L
%\om                         \B
%pṛthvāṃ vyāpya   tiṣṭhati /   \N1
%pṛthvīṃ vyāpya   tiṣṭhati /   \N2
%pṛthvīṃ vyāpya   tiṣṭhati /   \D  %geht auch pṛthu für Erde? 
%\om   \U1
%pṛthivyā vyāti   tiṣṭhati     \U2
%------------------------------
%He dwells on earth having pervaded [it]. 
%------------------------------
\app{\lem[type=emendation, resp=egoscr]{pṛthivīṃ}
  \rdg[wit={E}]{pṛthivyāṃ}
  \rdg[wit={P}]{pṛthi°}
  \rdg[wit={N1}]{pṛthvāṃ}
  \rdg[wit={D,N2}]{pṛthvīṃ}
  \rdg[wit={U2}]{pṛthivyā}}
\app{\lem[wit={D,E,P,N1,N2}]{vyāpya}
  \rdg[wit={U2}]{vyāti}} 
tiṣṭhati/
\note[type=philcomm, labelb=117a, lem={pṛthivīṃ vyāpya tiṣṭhati}]{The sentence is missing in B, L and U\textsubscript{1}.}
%------------------------------
% yasya janmamaraṇe  na staḥ sukhaṃ na bhavati /  \E
% yasya janmamaraṇe  na staḥ sukhaṃ na bhavati /  \P
% \om                                            \L
% \om                                            \B
% yasya janmamaraṇe  na staḥ sukhaṃ na bhavati /  \N1
% yasya janmamaraṇe  na staḥ sukhaṃ na bhavati /  \N2
% yasya janmamaraṇe  na staḥ sukhaṃ na bhavati /  \D
% \om                                            \U1
% yasya jananamaraṇe na staḥ sukhaṃ na bhavati /  \U2 maraṇe nom/acc dual! staḥ von as 3. dual 
%------------------------------
% Birth and death both do not exist. Happiness does not exist. 
% ------------------------------
\note[type=source, labelb=118, lem={\textbf{Re}}]{PT\textsuperscript{qcr \cdot YSV} (Ed. p. 832): na kṣobho janma mṛtyuś ca na duḥkhaṃ na sukhaṃ tathā |}
yasya janmamaraṇe  na staḥ sukhaṃ na bhavati/\textsuperscript{\begin{otherlanguage}{english}\coro{[\lowroman{5}]}\end{otherlanguage}}
\note[type=philcomm, labelb=118b, lem={yasya \ldots na bhavati}]{The sentence is \om in B,L and U\textsubscript{1}.}
% ------------------------------
% \om                 \E
% \om                 \P
% \om                 \L
% \om                  \B
% duḥkhaṃ na bhavati / \N1
% duḥkhaṃ na bhavati / \N2
% duḥkham na bhavati / \D
% \om                  \U1
% \om                  \U2
% ------------------------------
%Suffering does not exist. 
%------------------------------
duḥkhaṃ na bhavati/
\note[type=philcomm, labelb=118c, lem={duḥkham na bhavati}]{The sentence is \om in X and U\textsubscript{1}.}
%------------------------------
% \om               \E
% kalaṃ na bhavati  \L
% kulaṃ na bhavatī// \B
% kūlaṃ na bhavati / \P
% kūlaṃ na bhavati / \N1
% kūlaṃ na bhavati / \N2
% kūlaṃ na bhavati / \D
% \om               \U1
% kulaṃ na bhavatī// \U2
%------------------------------
%Impediment does not exist.
%------------------------------
\note[type=source, labelb=119, lem={kūlaṃ}]{PT\textsuperscript{qcr \cdot YSV} (Ed. p. 834): bhedābhedau manaḥsthau na jñānaṃ śīlaṃ kulaṃ tathā |}
\note[type=philcomm, labelb=119b, lem={kūlaṃ na bhavati}]{The sentence is \om in E and U\textsubscript{1}.}
\app{\lem[wit={D,P,N1,N2}]{kūlaṃ}
  \rdg[wit={B,U2}]{kulaṃ}
  \rdg[wit={L}]{kalaṃ}}
na \app{\lem[wit={ceteri}]{bhavati}
  \rdg[wit={B,U2}]{bhavatī}}/
%------------------------------
% \om                  \E
% śītalaṃ na bhavati / \P
% \om                  \L
% \om                  \B
% śīlaṃ na bhavati /   \N1
% śīlaṃ na bhavati /   \N2
% śīlaṃ na bhavati /   \D
% śīlaṃ na bhavati /   \U1
% śīlaṃ na bhavati /   \U2
%------------------------------
% Custom does not exist. 
% ------------------------------
\app{\lem[wit={ceteri}]{śīlaṃ}
  \rdg[wit={P}]{śītalaṃ}}
na bhavati/
\note[type=philcomm, labelb=119c, lem={śīlaṃ na bhavati}]{The sentence is \om in E,L and B.}
%------------------------------
% \om                 \E
% sthānaṃ na bhavati / \P
% \om                  \L
% \om                  \B
% sthānaṃ na bhavati / \N1
% sthānaṃ na bhavati / \N2
% sthānaṃ na bhavati / \D
% sthānaṃ na bhavati / \U1
% sthānaṃ na bhavati / \U2
%------------------------------
% Place does not exist. 
%------------------------------
kiñcic cihnaṃskiñcic cihnaṃthānaṃ na bhavati/
\note[type=philcomm, labelb=119d, lem={sthānaṃ na bhavati}]{The sentence is \om in E,L and B.}
%------------------------------
% \om                                                                             \E
%asya siddhasya manomadhye īśvarasaṃbaṃdhī prakāśo niraṃtaraṃ     pratyakṣo bhavati  \P
%asya siddhasya manomadhye īśvarasaṃbaṃdhi prakāśo  niraṃtaraṃ    pratyakṣo bhavati  \L
%asya siddhasya manomadhye īśvaraṃ saṃbaṃdhī prakāśo  niraṃtaraṃ  pratyakṣo bhavatī//  \B
%asya siddhasya manomadhye īśvarasaṃbaṃdhī prakāśaḥ niraṃtaraṃ    pratyakṣa bhavati  \N1
%asya siddhasya manomadhye īśvarasaṃbaṃdhī prakāśaḥ niraṃtaraṃ    pratyakṣa bhavati/  \N2
%asya siddhasya manomadhye īśvarasaṃbaṃdhi prakāśaḥ niraṃtaraṃ    pratyakṣo bhavati  \D
%asya siddhasyaṃ pṛthivī vyāpya tiṣṭhati yasya yanma maraṇai na saḥ sukhaṃ na bhati kulaṃ na bhavati śīlaṃ na bhavati sthānaṃ na bhavati ..... asya siddhasya manomadhye īśvarasaṃbaṃdhī prakāśaḥ niraṃtaraṃ pratyakṣo bhavati  \U1
%asya siddhasya manomadhye īśvarasaṃbaṃdhī prakāśo nirattaraṃ  pratyakṣo bhavati//  \U2
%------------------------------
%The manifestation of permanent perception of the connection with god arises within the mind of the accomplished one. 
%------------------------------
\note[type=source, labelb=120, lem={\textbf{Re}}]{PT\textsuperscript{qcr \cdot YSV} (Ed. p. 834): prakāśakuśasambandhiprasaṅgo 'yaṃ nirantaram | sarvaprakāśako 'sau tu naṣṭabhedādir eva ca |}
\note[type=philcomm, labelb=120b, lem={asya siddhasya \ldots bhavati}]{The sentence is \om in E.}
asya
\app{\lem[wit={ceteri}]{siddhasya}
  \rdg[wit={U1}]{siddhasyaṃ pṛthivī vyāpya tiṣṭhati yasya yanma maraṇai na saḥ sukhaṃ na bhati kulaṃ na bhavati śīlaṃ na bhavati sthānaṃ na bhavati asya siddhasya}}
%\note[type=philcomm, labelb=s34.z3, lem={asya siddhasyaṃ}]{U\textsubscript{1} repeats the whole section from \textit{pṛthivī} to \ldots \textit{sthānaṃ na bhavati} due to an eyeskip in the process of copying.}
manomadhye
\app{\lem[wit={ceteri}]{īśvarasaṃbandhī}
  \rdg[wit={B}]{īśvaraṃ saṃbaṃdhī}}
\app{\lem[wit={Y}]{prakāśo}
  \rdg[wit={X}]{prakāśaḥ}}
\app{\lem[wit={ceteri}]{nirantaraṃ}
  \rdg[wit={U2}]{nirattaraṃ}}
\app{\lem[wit={ceteri}]{pratyakṣo}
  \rdg[wit={N1}]{prakyakṣa}}
\app{\lem[wit={ceteri}]{bhavati}
  \rdg[wit={B}]{bhavatī}}/\textsuperscript{\begin{otherlanguage}{english}\coro{[\lowroman{10}]}\end{otherlanguage}}
%------------------------------
%sa ca prakāśo na śīto na coṣṇo na śveto na pīto bhavati/ \E
%sa ca prakāśo na śīto na coṣṇo na śveto na pīto bhavati/ \P
%sa ca prakāśo na śīto na coṣṇo na śveto na pīto bhavatī// \L
%sa ca prakāśo na śīto na coṣṇo na śveto na pīto bhavatī// \B
%sa ca prakāśo na śīto na coṣṇo na śveto na pīto bhavati/ \N1
%sa ca prakāśo na śīto na coṣṇo na śveto na pīto bhavati    \D
%sa ca prakāśo na śīto na coṣṇo na kheto na pīto bhavati/ \N2
%sa ca prakāśo na śīto na ?hbho?na kheto na pīto bhavati // \U1
%sa ca prakāśo// na śīto na coṣṇo na śveto pīto na bhavati // \U2
%------------------------------
%And he is shining - not cold, and not hot, not white [and] not yellow. 
%------------------------------
sa ca prakāśo na śīto na
\app{\lem[wit={ceteri}]{coṣṇo}
  \rdg[wit={U1}]{...o}}
na
\app{\lem[wit={ceteri}]{śveto}
  \rdg[wit={N2,U1}]{kheto}}
\app{\lem[wit={ceteri}]{na pīto}
  \rdg[wit={U2}]{pīto na}}
\app{\lem[wit={ceteri}]{bhavati}
  \rdg[wit={B,L}]{bhavatī}}/
%------------------------------
%tasya na jātir  na kiñcic cihnam  \E
%tasya na jātir  na kiñcic cihnaṃ  \P
%tasya na jātir  na kiṃcic cinhaṃ  \L
%tasya na jātir  na kiṃcic cinhaṃ  \B
%tasya na jātir  na kiṃcic cihūṃ  \N1
%tasya na jāti   na kiṃcic cihūṃ//  \D
%tasya na jāti   na kiṃcic cihūṃ  \N2
%tasya na jātir  na kiṃcit khecha cinhaṃ  \U1
%tasya na jānāti na kiṃcit cinhaṃ //  \U2
%------------------------------
%He does not have a caste, nor does he have any attributes.
%------------------------------
\note[type=source, labelb=121, lem={\textbf{Re}}]{asya jāterna cihnañ ca niṣkalo 'yaṃ nirañjanaḥ | ananto 'yaṃ mahājyotir vāñchāṃ bhogaṃ dadāti ca |}
tasya na
\app{\lem[wit={ceteri}, alt={jātir}]{jāti\skp{r-na}}
  \rdg[wit={D,N2}]{jāti}
  \rdg[wit={U2}]{jānāti}
}\skm{r-na}
\app{\lem[wit={ceteri}, alt={kiñcic cihnaṃ}]{kiñcic\skp{-}cihnaṃ}
  \rdg[wit={E}]{kiñcic cihnam}
  \rdg[wit={D,N1,N2}]{kiñcic cihūṃ}
  \rdg[wit={U1}]{kiṃcit khecha cinhaṃ}
  \rdg[wit={U2}]{na kiṃcit cinhaṃ}}/
%------------------------------
%ayaṃ   ca niṣkalo   niraṃjanaḥ   alakṣyaś ca bhavati \E
%ayaṃ   ca niṣkalo   niraṃjanaḥ   alakṣyaś ca bhavati \P
%vyayaṃ ca niṣkalo   niraṃjanaṃ// alakṣaś  ca bhavati// \L
%vyayaṃ ca nīṣkalo   niraṃjanaṃ// alakṣaś  ca bhavatī// \B
%ayaṃ   ca niṣkalo   niraṃjanaḥ// alakṣyaś ca bhavati// \D
%ayaṃ   ca nīṣkalo   niraṃjanaḥ   alakṣaś  ca bhavati// \N1
%ayaṃ   ca niṣkalo   niraṃjanaḥ   alakṣaś  ca bhavati// \N2
%ayaṃ   ca niḥkalo   niraṃjanaḥ   alakṣyaḥ    bhavati/ \U1
%ayaṃ   ca nīṣkalo   niraṃjanaḥ// alakṣyaḥ    bhavati// \U2
%------------------------------
%And he is without parts, immacule and uncharacterized.  
%------------------------------
\app{\lem[wit={ceteri}]{ayaṃ}
  \rdg[wit={B,L}]{vyayaṃ}}
ca
\app{\lem[wit={ceteri}]{niṣkalo}
  \rdg[wit={B,U2}]{nīṣkalo}
  \rdg[wit={U1}]{niḥkalo}}
nirañjanaḥ/
\app{\lem[wit={ceteri}, alt={alakṣyaś}]{alakṣya\skp{ś-ca}}
  \rdg[wit={U1,U2}]{alakṣyaḥ}
  \rdg[wit={B,L,N1,N2}]{alakṣaś}
}\app{\lem[wit={ceteri}, alt={ca}]{\skm{ś-ca}}
  \rdg[wit={U1,U2}]{\om}}
\app{\lem[wit={ceteri}]{bhavati}
  \rdg[wit={B}]{bhavati}}/
%------------------------------
%atha ca phaladvaṃde  na         kāminy āder   yasyecchā         na bhavati // \E
%atha ca phalacaṃda   na         kāminy āder   yasyochā          na bhavati    \P
%atha ca phalavaṃda   na         kāminy ādir   yasya             na bhavati    \L
%atha ca phalaṃ jaṃda na         kāminy ādar   yasye             na bhavatī    \B
%atha ca phalacaṃdra  na         kāminy āder   yasya  yasyeccha     bhavati/   \N1
%atha ca phalacaṃda   na         kāminy āde    yasya  yasyechā      bhavati//  \D
%atha ca phalaṃ/caṃdra           kāminy āder   yasya  yasyeccha     bhavati/   \N2
%atha ca phalaṃ caṃda na         kāminy āder   yasya  yaṃ           bhavati    \U1
%atha ca phalacaṃda   na         kāminy āder   yasyechā             bhavati//  \U2
%------------------------------
%And his desire etc. doesn't arise in [situations of] lust [and] is not located within the duality of the result.  
%------------------------------
atha ca
\app{\lem[wit={E}]{phaladvande}
     \rdg[wit={D,P,U2}]{phalacaṃda}
     \rdg[wit={U1}]{phalaṃ caṃda}
     \rdg[wit={L}]{phalavaṃda}
     \rdg[wit={B}]{phalaṃ jaṃda}
     \rdg[wit={N1}]{phalacaṃdra}
     \rdg[wit={N2}]{phalaṃ/ caṃdra}}
\app{\lem[wit={ceteri}]{na}
     \rdg[wit={N2}]{\om}}
kāmi\skp{ny-ā}\app{\lem[wit={ceteri}, alt={āder}]{\skm{ny-ā}de\skp{r-ya}}
     \rdg[wit={D}]{āde}
     \rdg[wit={B}]{ādar}
     \rdg[wit={L}]{ādir}
}\app{\lem[wit={E},alt={yasyecchā}]{\skm{r-ya}syecchā}
     \rdg[wit={P}]{yasyochā}
     \rdg[wit={L}]{yasya}
     \rdg[wit={B}]{yasye}
     \rdg[wit={N1,N2}]{yasya yasyeccha}
     \rdg[wit={D}]{yasya yasyechā}
     \rdg[wit={U1}]{yasya yaṃ}
     \rdg[wit={U2}]{yasye chā}}
\app{\lem[wit={B,E,L,P}]{na}
     \rdg[wit={ceteri}]{\om}}
\app{\lem[wit={ceteri}]{bhavati}
  \rdg[wit={B}]{bhavatī}}/\textsuperscript{\begin{otherlanguage}{english}\coro{[\lowroman{15}]}\end{otherlanguage}}
%------------------------------
% \om                      \E
% \om                      \P
% \om                      \L
% \om                      \B
%taṃ taṃ bhogaṃ prāpnoti   \D
%taṃ taṃ bhogaṃ prāpnoti   \N1
%taṃ taṃ bhogaṃ prāpnoti// \N2
%tataṃ bhogaṃ prāpnoti     \U1
% \om                      \U2
%------------------------------
%He attains expanded enjoyment. 
%------------------------------
\app{\lem[wit={D,N1,N2}]{taṃ taṃ}
  \rdg[wit={U1}]{tataṃ}} bhogaṃ prāpnoti/
\note[type=philcomm, labelb=122b, lem={taṃ taṃ \ldots prāpnoti}]{The sentence is \om in the \beta-group.}
%------------------------------
% \om                      \P
% \om                      \L
% \om                      \B
%atha vā yasya    mana    eva   sthāne 'nurāgaṃ        na prāpnoti// \D
%atha vāsya/vātya mana   eva   sthāne 'nurāgaṃ        na prāpnoti/ \N1
%atha vā syamana         eva   sthāne 'nurāgaṃ        na prāpnoti/ \N2
%atha vā svāmana         etata sthāne  nurāgaṃ         na prāpnoti/ \U1
% \om                      \U2
%------------------------------
%However, his mind does not suffer attachment in this very state.  
%------------------------------
atha 
\app{\lem[wit={D}]{vā yasya}
  \rdg[wit={N1}]{vāsya}
  \rdg[wit={N2}]{vā sya}
  \rdg[wit={U1}]{vā svā}}
mana
\app{\lem[wit={D,N1,N2,U1}]{eva}
  \rdg[wit={U1}]{etata}}
sthāne 'nurāgaṃ na prāpnoti\dd{}
\note[type=philcomm, labelb=122c, lem={atha vā yasya mana \ldots na prāpnoti}]{The sentence is \om in the \beta-group.}
\end{prose}
  \end{edition}
  \begin{translation}
    \ekddiv{type=trans}
    \centerline{\textrm{\small{[\uproman{18}.\textsuperscript{\coro{\lowroman{1}-\lowroman{17}}} The Sign of a Rājayogin's Body]}}}
    \label{rajabodytrans}
    \bigskip
    \begin{tlate}
      \noindent
        Now the sign of the body of the person who is in the state of Rājayoga is taught. Abundance arises at all times. No distance exists on earth.\footnote{This statement probably refers to the so-called Bhūcara Siddhi, which is common in texts of Rājayoga. This term designates the ability to instantly travel anywhere around the world, cf. \citetitle{birch2013} 1.65: (dvādaśāhalayenāpi bhūcaratvaṃ hi sidhyati | nimiṣārdhapramāṇena paryaṭaty eva bhūtalam || 65 ||) Similar abilities are also mentioned in earlier texts such as \citetitle{datta2015} 81ab−82cd and \citetitle{mallinson2007} 3.52. For a detailled discussion see \citeauthor[2013: 275, endnote 91]{birch2013}.}
  He dwells on earth having pervaded [it]. Both, birth and death do not exist.\footnote{Cf. \citetitle{birch2013} 1.27. Here the yogin in \textit{samādhi} is described as neither alive nor dead like a lifeless like a piece of wood (na ca jīvan mṛto vāpi na paśyati na mīlati | nirjīvaḥ kāṣṭhavat tiṣṭhel layasthaś cābhidhīyate || 27 ||); also Cf. \citetitle{sarvangayoga} 19d (jarā na vyāpai kāla na ṣāī |) ``he does not know old age and death '' and 20c (ajar amar ati bajraśarīrā |) ``\ldots non-ageing, immortal supreme diamond body.''}\textsuperscript{\coro{[\lowroman{5}]}} Happiness does not exist. Suffering does not exist.\footnote{Equanimity towards happiness and suffering or other opposites in the state of Rājayoga are commonly found among texts that teach Rājayoga, cf. e.g. \citetitle{birch2013} 1.26ab (sukhaṃ duḥkhaṃ na jānāti śītoṣṇaṃ ca na vindati |); \citetitle{hp} 4.111 ≈ \citetitle{nadabindu} 53ab−54cd: (na vijānāti śītoṣṇaṃ na duḥkhaṃ na sukhaṃ tathā | na mānaṃ nopamānaṃ ca yogī yuktaḥ samādhinā ||); also cf. \citetitle{sarvangayoga} 3.18cd: (jākaiṃ dukh aru sukh nahiṃ koī | harṣ śok vyāpai nahiṃ koī || 18 ||) ``for whom neither sorrow nor joy matters, and neither joy nor sorrow overwhelms him.''} Impediment does not exist.\footnote{Cf. \citetitle{sarvangayoga} 3.22: (icchā parai tahāṃ so jāī | tīni lok mahiṃ aṭak na kāī | svarg jāī devani mahiṃ baithai | nāgalok pātāl su paiṭhai || 22 ||) ``He goes where he wants without being stopped in the three worlds, he goes to heaven and sits with the gods, he goes to the underworld and subdues the demons.''} Moral conduct (\textit{śīla}) does not exist.\footnote{Cf. \citetitle{datta2015} 162. Here, the Yogin can do anything, behaving as he likes.} Place does not exist.\textsuperscript{\coro{[\lowroman{10}]}} The manifestation of a permanent perception of the connection with god arises within the mind of the accomplished one. Moreover, he is shining - not cold, hot, white or yellow.\footnote{Cf. \citetitle{birch2013} 1.51: (vāsarārdhalayenāpi svātmajyotiḥ prakāśate | sūryo gobhir ivoddīpto yogī viśvaṃ prakāśate || 51 ||); Cf. \citetitle{sarvangayoga} 3.13cd: (rājayog sab ūpara chājai | jo sādhai so adhik birājai || 13 ||) ``Rajayog is supreme and those who practice it shine even more.'' and Cf. \citetitle{sarvangayoga} 3.23cd: (hṛdai prakāś rahai din rātī | deśai jyoti tel bin vātī || 23 ||) ``The light in his heart remains bright day and night, without oil.''}
He does not have a caste nor does he have any attribute. Furthermore, he is without parts, immaculate and uncharacterized. His desire etc. does not arise in [situations of] lust\footnote{The emphasis on desirelessness as a result of practicing Rājayoga is seen e.g. in \citetitle{birch2013} 1.44: (palāṣṭakalayenāpi kāmas tasya nivartate | kadāpi naiva jāyeta kāminyāliṅgitasya ca || 44 ||)} [and] is not located within the duality of the result.\textsuperscript{\coro{[\lowroman{15}]}} He attains expanded enjoyment. However, his mind does not suffer attachment in this very state.
\end{tlate}
\end{translation}
\end{alignment}
%\ekdpb*{}
%%%%%%%%%%%%%%%%%%%%%%%%%%%%%%%%%%%%%%%%%% 
%%%%%%%%%%%%%%%%%%%%%%%%%%%%%%%%%%%%%%%%%%
%%%%%%%%PAGEBREAK%%%%%%%PAGEBREAK%%%%%%%%%
%%%%%%%%%%%%%%%%%%%%%%%%%%%%%%%%%%%%%%%%%%
%%%%%%%%%%%%%%%%PAGEBREAK%%%%%%%%%%%%%%%%%
%%%%%%%%%%%%%%%%%%%%%%%%%%%%%%%%%%%%%%%%%%
%%%%%%%%PAGEBREAK%%%%%%%PAGEBREAK%%%%%%%%%
%%%%%%%%%%%%%%%%%%%%%%%%%%%%%%%%%%%%%%%%%%
%%%%%%%%%%%%%%%%%%%%%%%%%%%%%%%%%%%%%%%%%%
%%%%%%%%%%%%%%%%%%%%%%%%%%%%%%%%%%%%%%%%%%
%%%%%%%%%%%%%%%%%%%%%%%%%%%%%%%%%%%%%%%%%%
%%%%%%%%PAGEBREAK%%%%%%%PAGEBREAK%%%%%%%%%
%%%%%%%%%%%%%%%%%%%%%%%%%%%%%%%%%%%%%%%%%%
%%%%%%%%%%%%%%%%PAGEBREAK%%%%%%%%%%%%%%%%%
%%%%%%%%%%%%%%%%%%%%%%%%%%%%%%%%%%%%%%%%%%
%%%%%%%%PAGEBREAK%%%%%%%PAGEBREAK%%%%%%%%%
%%%%%%%%%%%%%%%%%%%%%%%%%%%%%%%%%%%%%%%%%%
%%%%%%%%%%%%%%%%%%%%%%%%%%%%%%%%%%%%%%%%%%
%%%%%%%%%%%%%%%%%%%%%%%%%%%%%%%%%%%%%%%%%%
%%%%%%%%%%%%%%%%%%%%%%%%%%%%%%%%%%%%%%%%%%
%%%%%%%%PAGEBREAK%%%%%%%PAGEBREAK%%%%%%%%%
%%%%%%%%%%%%%%%%%%%%%%%%%%%%%%%%%%%%%%%%%%
%%%%%%%%%%%%%%%%PAGEBREAK%%%%%%%%%%%%%%%%%
%%%%%%%%%%%%%%%%%%%%%%%%%%%%%%%%%%%%%%%%%%
%%%%%%%%PAGEBREAK%%%%%%%PAGEBREAK%%%%%%%%%
%%%%%%%%%%%%%%%%%%%%%%%%%%%%%%%%%%%%%%%%%%
%%%%%%%%%%%%%%%%%%%%%%%%%%%%%%%%%%%%%%%%%%
\begin{alignment}[
  texts=edition[class="edition"];
  translation[class="translation"],
  ]
  \begin{edition}
    \ekddiv{type=ed}
 \centerline{\textrm{\small{[\uproman{19}.\textsuperscript{\coro{\lowroman{1}-\lowroman{15}}} Another Sign]}}}
    \label{anjacihna}    
    \bigskip
    \begin{prose}
      \noindent
%------------------------------
%anyad  rājayogasya cihnaṃ kathyate   \E
% \om                                 \P
%anyata rājayogasya cinhaṃ kathyate// \L
%anyata rājayogasya cinhaṃ kathyate// \B
%anyat  rājayogasya cinhaṃ kathyate// \N1 yasyecchā bhavati??? taṃ taṃ bhogaṃ prāpnoti/ atha vāsya mana eva sthāne 'nu rāgaṃ na prāpnoti/ anyat rājayogasya cinhaṃ kathate//
%anyat  rājayogasya cihuṃ  kathyate// \D
%anyad  rājayogasya ciṃhuṃ kathyate// \N2
%anyat  rājayogacinhaṃ     kathyate/  \U1
%anyat  rājayogasya cinhaṃ kathyate// \U2
%------------------------------
% Another sign of Rājayoga is described. 
%------------------------------
\note[type=source, labelb=122, lem={\textbf{Re}}]{PT\textsuperscript{qcr \cdot YSV} (Ed. pp. 834-835): asya citte nānurāgo virāgo na bhaved iti | rājya prāpte 'pi no harṣo hānau duḥkhaṃ bhaven na hi | kvacid vastuni deśasya niḥsvane keṣu kutracit |}
\note[type=philcomm, labelb=122d, lem={anyad rājayogasya \ldots anicchā na bhavati}]{\uproman{19}.\lowroman{1}-\lowroman{5} are \om in P.}
\app{\lem[wit={E,N2},alt={anyad}]{anya\skm{d-rā}}
  \rdg[wit={X}]{anyat}
  \rdg[wit={B,L}]{anyate}
}\app{\lem[wit={ceteri},alt={rājayogasya}]{\skp{d-rā}jayogasya}
  \rdg[wit={U1}]{rājayoga°}}
\app{\lem[wit={E}]{cihnaṃ}
  \rdg[wit={B,L,N1,U2}]{cinhaṃ} %????
  \rdg[wit={N2}]{ciṃhuṃ}
  \rdg[wit={D}]{cihuṃ}}
kathyate/
%------------------------------
%yasya rājyādilābhe 'pi    phalalābho na bhavati/ \E
% \om                                            \P
%yasya rājādilābhe   ty     aphalalābho       na bhavatī \L
%yasya rājādilābhe   ty     aphalalābho       na bhavatī \B
%yasya rājyādilābhe  pi     phalalābho       ba bhavati/ \N1
%yasya rājyādilābhe  pi     phalalābho       na bhavati// \D
%yasya rājyādilobhe  pi ca  phalalābho       na bhavati// \N2
%yasya rājyādilābe  'pi ca  palalābho        na bhavati/ \U1
%yasya rājyādilābho                          na bhavati/ \U2
%------------------------------
%Even ``of one who is in gain of a kingdom etc.'' [it is said that] perception of a gain does'nt arise.
%------------------------------
yasya
\app{\lem[wit={ceteri},alt={rājyādi°}]{rājyādi}
  \rdg[wit={B,L}]{rāja°}}\app{\lem[wit={D,E,N1},alt={°lābhe}]{lābhe}
   \rdg[wit={N2}]{°lobhe}
  \rdg[wit={U1}]{°lābe}
  \rdg[wit={U2}]{°lābho}
 \rdg[wit={B,L}]{lābhety}}
\app{\lem[wit={D,E,N1}]{'pi}
  \rdg[wit={N2,U1}]{'pi ca}
  \rdg[wit={U2}]{\om}}
\app{\lem[wit={D,E,N1,N2}]{phalalābho}
  \rdg[wit={U1}]{pala°}
  \rdg[wit={B,L}]{aphala°}
  \rdg[wit={U2}]{\om}}
\app{\lem[wit={D,E,N2,U1,U2}]{na bhavati}
  \rdg[wit={B,L}]{na bhavatī}
  \rdg[wit={N1}]{ba bhavati}}/
%------------------------------
%hānāv api manomadhye duḥkhaṃ na bhavati/ \E
% \om                                      \P
%hananād pi mānomadhye duḥkahṃ na bhavatī/ \L
%hananād pi mānomadhye duḥkahṃ na bhavatī/ \B
%hānāv api manomadhye duḥkhaṃ na bhavati/ \N1 %emend to hānau loc. sg. of hāni -> abandonment
%hānāv api manomadhye duḥkhaṃ na bhavati// \D
%hānāv  pi manomadhye duḥkhaṃ na bhavati// \N2
%hānāv api manomadhye duḥkhaṃ na bhavati/  \U1
%hānād api manomadhye duḥkhaṃ na bhavati// \U2
%------------------------------
%Even due to loss suffering does'nt arise in the mind.  
%------------------------------
\app{\lem[wit={ceteri},alt={hānāv}]{hānā\skp{v-a}}
  \rdg[wit={U2}]{hānād}
  \rdg[wit={B,L}]{hananād}
}\app{\lem[wit={ceteri},alt={api}]{\skm{v-a}pi}
  \rdg[wit={B,L,N2}]{pi}}
manomadhye duḥkhaṃ na
\app{\lem[wit={ceteri}]{bhavati}
  \rdg[wit={B,L}]{bhavatī}}/
%------------------------------
%atha ca tṛṣṇā na bhavati/ \E
% \om                      \P
%atha ca tṛṣṇā na bhavati/ \L
%atha ca tṛṣṇā na bhavatī/ \B
%atha ca tṛṣṇā na bhavati/ \N1
%atha ca tṛṣṇā na bhavati  \D
%atha ca tṛṣṇā na bhavati/ \N2
%atha ca tṛṣṇā na bhavati/ \U1
%atha ca tṛṣṇā na bhavati/ \U2
%------------------------------
%And then desire doesn't arise. 
%------------------------------
atha ca tṛṣṇā na
\app{\lem[wit={ceteri}]{bhavati}
  \rdg[wit={B}]{bhavatī}}/
%------------------------------
%atha ca kasmin                                  padārthasyopary   anicchā na bhavati/ \E
% \om                                                                                       \P
%atha ca kasmin na    padārtho   prāpte kasyāpi  padārthasyopari   ānīcha  na  bhavati//    \L
%atha ca kasmin na    padārthau  prāpte kasyāpi  padārthāsyopari   ānīchā  ni  bhavati//    \B
%atha ca kasminn pi   padārthe   prāpta kasyāpi  padārthasya upari anusthā na  bhavaṃti//   \N1 
%atha ca kasminn api  padārthe   prāpte kasyāpi  padārthasya upari anichā      bhavaṃti     \D
%atha ca kasminn pi   padārthe   prāpte kasyāpi  padārthasya upari anisthā na  bhavati//    \N2
%atha ca kasminn api  padārthe   prātpe kasyāpi  padārthasya upari aniṣṭā  na  bhavati      \U1
%atha ca kasmin   adhipadārtha   prāpte kābhyādi padārthopari      anicha  na  bhavati//    \U2 %%%407.jpg
%------------------------------
%And then with regards to some object that has been obtained for whatever reason, aversion does'nt arise towards the object.    
%------------------------------
atha ca
kasmi\skp{n-na}\app{\lem[wit={D,U1},alt={api}]{\skm{n-na}pi}
  \rdg[wit={B,L}]{na}
  \rdg[wit={N1,N2}]{pi}
  \rdg[wit={U2}]{adhi}
  \rdg[wit={E,P}]{\om}}
\app{\lem[wit={ceteri}]{padārthe}
  \rdg[wit={B}]{padārthau}
  \rdg[wit={L}]{padārtho}
  \rdg[wit={U2}]{padārtha°}
  \rdg[wit={E}]{\om}}
\app{\lem[wit={ceteri}]{prāpte}
  \rdg[wit={N1}]{prāpta}
  \rdg[wit={E}]{\om}}
\app{\lem[wit={ceteri}]{kasyāpi}
  \rdg[wit={U2}]{kābhyādi}
  \rdg[wit={E}]{\om}}
\app{\lem[wit={E},alt={padārthasyopary}]{padārthasyopa\skp{ry-a}}
  \rdg[wit={B,L}]{padārthasyopari}
  \rdg[wit={U2}]{padārthopari}
  \rdg[wit={X}]{padārthasya upari}
}\app{\lem[wit={E},alt={anicchā}]{\skm{ry-a}nicchā}
  \rdg[wit={B}]{ānīchā}
  \rdg[wit={L}]{ānīcha}
  \rdg[wit={D}]{anichā}
  \rdg[wit={N1}]{anusthā}
  \rdg[wit={N2}]{anisthā}
  \rdg[wit={U1}]{aniṣṭā}
  \rdg[wit={U2}]{anicha}}
\app{\lem[wit={ceteri}]{na}
  \rdg[wit={B}]{ni}
  \rdg[wit={D}]{\om}}
\app{\lem[wit={ceteri}]{bhavati}
  \rdg[wit={N1,D}]{bhavaṃti}}/\textsuperscript{\begin{otherlanguage}{english}\coro{[\lowroman{5}]}\end{otherlanguage}}
%------------------------------
%kasmin    padārthe manaso   nurāgo na bhavati/    \E
%asminnapi padārthe manaso   nurāgo na bhavati... ayam api padārthe manasonurāgo na bhavati... \P
%asminn    padārthe manaso   nurāgo na bhavatī/    \L
%asminn    padārthe manaso   nurāgo na bhavatī/    \B
%asminnapi padārthe manasaḥ anurāgo    bhavati/    \N1
%asminnapi padārthe manasaḥ anurāgo    bhavati//   \D
%asminnapi padārthe manasaḥ anurāgo    bhavati/    \N2
%asminnapi padārthe manasa  anurāgo    bhavati     \U1 
%kasminnpi padārthe         anurāgo na bhavati// ayam api padārthe anurāgo na bhavati//  \U2
%------------------------------
%With regard to this object also affection of the mind does'nt arise. 
%------------------------------
\app{\lem[wit={ceteri},alt={asminn}]{asmi\skp{n-a}}
  \rdg[wit={E,U2}]{kasmin}
}\app{\lem[wit={ceteri},alt={api}]{\skm{n-a}pi}
  \rdg[wit={B,E,L}]{\om}} 
padārthe
\app{\lem[wit={B,E,L,P}]{manaso}
  \rdg[wit={X}]{manasaḥ}
  \rdg[wit={U1}]{manasa}
  \rdg[wit={U2}]{\om}}
\app{\lem[wit={B,E,L,P}]{'nurāgo}
  \rdg[wit={ceteri}]{anurāgo}}
\app{\lem[wit={E}]{na bhavati}
  \rdg[wit={B,L}]{na bhavatī}
  \rdg[wit={P}]{na bhavati ayam api padārthe manasonurāgo na bhavati}
  \rdg[wit={U2}]{na bhavati || ayam api padārthe anurāgo na bhavati}
  \rdg[wit={X}]{bhavati}}/
%------------------------------
%ayam  api rājayogaḥ kathyate/  \E
%atham api rājayogaḥ kathyate   \P
%atha  samarājayogaḥ kathyate/  \L
%ayam  api rājayogaḥ kathyate/  \B
%ayam  api rājayogaḥ kathyate/  \N1
%ayam  api rājayogaḥ kathyate// \D
%ayam  api rājayoga  kathyate// \N2
%ayam  api rājayogaḥ kathyate/  \U1
%ayam  api rājayoga  kathyate// \U2
%------------------------------
%Just this is said to be Rājayoga. 
%------------------------------
\app{\lem[wit={ceteri},alt={ayam}]{aya\skp{m-a}}
  \rdg[wit={P}]{atham}
  \rdg[wit={L}]{atha}
}\app{\lem[wit={ceteri},alt={api}]{\skm{m-a}pi}
  \rdg[wit={L}]{sama}}
\app{\lem[wit={ceteri}]{rājayogaḥ}
  \rdg[wit={N2,U2}]{rājayoga}}
kathyate/
%------------------------------ %%%%split in stemma?! maitre mitre!!!
%atha caḥ yasya manaḥ   munividvat  puruṣeṣu maitre        ca samaṃ bhavati/ \E
%atha ca  yasya manaḥ   śunividvat  puruṣe   maitre śatrau ca samaṃ bhavati \P
%atha ca  yasya manaḥ   bhunividvat puruṣe   maitre śatrau ca samaṃ bhavati/ \L
%atha ca  yasya manaḥ   śrunividvat puruṣe   maitre śatro  ca samaṃ bhavatī/ \B
%atha ca  yasya manaḥ/  śrutividyut puruṣe   mitre  śatrau ca samaṃ bhavati/ \N1
%atha ca  yamanaḥ       śrutividyut puruṣe   mitre  śatrau ca samaṃ bhavati// \D
%atha ca  yasya manaḥ   śrutividyut puruṣe   mitre  śatrau ca samaṃ bhavati/ \N2
%atha ca  yasya mana    śrunividvat puruṣe   mitre  śatrau ca samaṃ bhavati \U1
%atha ca  yasya manaḥ   śuciviśuddhapuruṣe   mitre  śatrau ca samaṃ bhavati// \U2
%------------------------------
%And then his mind which knows the sacred speech is equal towards a person, friend and enemy.  
%------------------------------
\note[type=source, labelb=124, lem={\textbf{Re}}]{PT\textsuperscript{qcr \cdot YSV} (Ed. p. 835): vidyāvidyāmitraśatrau samā dṛṣṭiś ca sarvaśaḥ | bhogāsaktādikarttṛtvena mano no bhavet khavat |}
atha
\app{\lem[wit={ceteri}]{ca}
  \rdg[wit={E}]{caḥ}}
\app{\lem[wit={ceteri}]{yasya}
  \rdg[wit={D}]{ya}}
manaḥ
\app{\lem[resp=egoscr, type=emendation]{śrutividvat}
  \rdg[wit={E}]{munividvat}
  \rdg[wit={P}]{śunividvat}
  \rdg[wit={L}]{bhunividvat}
  \rdg[wit={B,U1}]{śrunividvat}
  \rdg[wit={D,N1,N2}]{śrutividyut}
  \rdg[wit={U2}]{śuciviśuddha°}
}\app{\lem[wit={ceteri}]{puruṣe}
  \rdg[wit={E}]{puruṣeṣu}}
\app{\lem[wit={ceteri}]{mitre}
  \rdg[wit={B,E,L,P}]{maitre}}
 \app{\lem[wit={ceteri}]{śatrau}
   \rdg[wit={B}]{śatro}
   \rdg[wit={E}]{\om}}
 ca samaṃ bhavati/
%------------------------------
%dṛṣṭiś ca samā bhavati/   \E
%dṛṣṭiś ca namnā bhavati   \P
% \om                      \L
% \om                      \B
%dṛṣṭiś ca samā bhavati//  \N1
%dṛṣṭiś ca samā bhavati//  \D
%dṛṣṭiś ca samā bhavati//  \N2
%dṛṣṭiś ca samā bhavati/   \U1
%dṛṣṭiś ca samā bhavati/   \U2
%------------------------------
%And a neutral view arises. 
%------------------------------
\note[type=philcomm, labelb=124b, lem={dṛṣṭiś ca \ldots bhavati}]{\uproman{19}.\lowroman{1} is \om in B and L}
dṛṣṭiś-ca
\app{\lem[wit={ceteri}]{samā}
  \rdg[wit={P}]{namnā}}
bhavati/
%------------------------------
%sakalapṛthvīmadhye gamanavataḥ       sukhabhogavataḥ      yasya manasi karttṛtvābhimāno   nāsti/ \E
%sakalapṛthvīmadhye gamanāgamanavataḥ sukhabhogavataḥ      yasya manasi kartṛtvābhimāno    nāsti/ \P
%sakalapṛtvīmadhye  gamanāgamanataḥ   sukhabogho bhavataḥ  yasya manasi kartutvābhimano   nāsti/ \L
%sakalapṛthvīmadhye gamanāgamanataḥ   sukhabogho bhavataḥ  yasya manasi kartutvābhimano    nāsti// \B
%sakalapṛthvīmadhye gamanavataḥ//     sukhabhogavataḥ/     yasya manasi kartṛtvādyabhimāno nāsti/  \N1
%sakalapṛthvīmadhye gamanaṃvataḥ//    sukhabhogavataḥ      yasya manasi kartṛtvādyabhimāno nāsti// \D
%sakalapṛthvīmadhye gamavataḥ         sukhabhogavataḥ      yasya manasi kartṛtvādyabhimāno nāsti// \N2
%sakalapṛthvīmadhye gamanavataḥ       sukho bhogavataḥ     yasya manasi kartṛtvābhimāno    nāsti   \U1
%sakalapṛthvīmadhye gamanāgamanavat// sukhabhogavat        yasya manasi kartṛtvābhimāno    nāsti// \U2
%------------------------------
%Amidst the whole earth, which is like the comings and goings of joys and pleasures, pride of authorship does not arise in his mind.
%------------------------------
\app{\lem[wit={ceteri}]{sakalapṛthvīmadhye}
  \rdg[wit={L}]{°pṛtvī°}}
\app{\lem[wit={P}]{gamanāgamanavataḥ}
  \rdg[wit={U2}]{gamanāgamanavat}
  \rdg[wit={B,L}]{gamanāgamanataḥ}
  \rdg[wit={E,N1,U1}]{gamanavataḥ}
  \rdg[wit={D}]{gamanaṃ vataḥ}
  \rdg[wit={U1}]{gamavataḥ}}
\app{\lem[wit={ceteri}]{sukhabhogavataḥ}
  \rdg[wit={B,L}]{sukhabogho bhavataḥ}
  \rdg[wit={U1}]{sukho bhogavataḥ}
  \rdg[wit={U2}]{sukhabhogavat}}
yasya manasi
\app{\lem[wit={E,P,U1,U2}]{kartṛtvābhimāno}
  \rdg[wit={B,L}]{kartutvābhimano}
  \rdg[wit={D,N1,N2}]{kartṛtvādyabhimāno}}
nāsti/\textsuperscript{\begin{otherlanguage}{english}\coro{[\lowroman{10}]}\end{otherlanguage}}
%------------------------------
%atha ca lokamadhye gamanavataḥ sukhabhogavataḥ yasya manasi karttṛtvābhimāno nāsti/....
%atha ca lokamadhye kartṛtvaṃ na jñāpayati/ \E
%anucalokamadhye    kartṛtvaṃ na jñāpayati/ \P
%anucaralokamadhya  kartṛtvābhimano nāsti \L
%anucaralokamadhya--kartṛtvābhimano nāsti// \B
%anucalokamadhye    kartṛtvaṃ    jñāpayati// \N1
%anucalokamadhye    kartṛtvaṃ na jñātvā payati/ \D
%anucalokamadhye    kartṛtvaṃ na jñāpayati/ \N2
%anucalokamadhye    kartṛtvaṃ    jñātva payati \U1
%anucalokamadhye    kartṛtvaṃ na jñāpayati \U2
%------------------------------
%While wandering the world he does not whish to know authorship. 
%------------------------------
\note[type=source, labelb=125, lem={\textbf{Re}}]{PT\textsuperscript{qcr \cdot YSV} (Ed. p. 835): lokamadhye bhavet karttā manomadhye 'pi niṣkriyaḥ |}
\app{\lem[wit={L,B}, alt={anucara°}]{anucara}
  \rdg[wit={D,N1,N2,U1,U2,P}]{anuca°}
  \rdg[wit={E}]{atha ca}
}loka\app{\lem[wit={ceteri},alt={°madhye}]{madhye}
  \rdg[wit={B,L}]{°madhya}}
\app{\lem[wit={D,E,P,N2,U2}]{kartṛtvaṃ na}
  \rdg[wit={B,L}]{kartṛtvābhimano}
  \rdg[wit={N1,U1}]{kartṛtvaṃ}}
\app{\lem[wit={E,P,N1,N2,U2}]{jñāpayati}
  \rdg[wit={D,U1}]{jñātva payati}
  \rdg[wit={B,L}]{nāsti}}/
%------------------------------
%so  pi  rājayogaḥ kathyate// \E
%so  pi  rājayogaḥ kathyate   \P
%so  pi  rājayoga  kathyate/   \L
%so  pi  rājayoga  kathyate/   \B
%so  pi  rājayogaḥ kathyate//  \N1
%so  pi  rājayoga  kathyate//  \D
%so 'pi  rājayoga  kathyate// \N2
%so  pi  rājayoga  kathyate/   \U1
%so  pi  rājayoga  kathyate    \U2
%------------------------------
%This is also said to be Rājayoga. 
%------------------------------
\note[type=source, labelb=126, lem={\textbf{Re}}]{PT\textsuperscript{qcr \cdot YSV} (Ed. p. 835): eṣo 'pi rājayogīti sukhe duḥkhe samas tathā |}
so 'pi
\app{\lem[wit={E,P,N1}]{rājayogaḥ}
  \rdg[wit={ceteri}]{rājayoga}}
kathyate/
%------------------------------
%navīnāni         paṭṭasūtramaya     dhṛtāni vastrāṇi   \E
%navīnāni         paṭasūtramayāni    dhṛtāni vastrāṇi   \P
%navinīnīśpī      paṭṭasūtramayāni   dhṛtāni vastrāṇi// \L
%navinīnīr api    paṭṭasūtramayāni   dhṛtāni vastrāṇi// \B
%navīnāni         paṭasūtramayāni    dhṛtāni vastrāṇi/  \N1
%navīnāni         paṭasūtramayāni    dhṛtāni vastrāṇi// \D
%navīnāni         paṭasūtramayāni    dhṛtāni vastrāṇi/  \N2
%navīnāni         padasūtramayāni       tāni vastrāṇi   \U1
%navīnāni      paṭ(h)asūtramayāni    dhṛtāni            \U2
%------------------------------
%New durable clothes made of silk,  
%------------------------------
\app{\lem[wit={ceteri}]{navīnāni}
  \rdg[wit={B}]{navinīnīr api}
  \rdg[wit={L}]{navīnīnīś pī}}
\app{\lem[wit={B,E,L}, alt={paṭṭa°}]{paṭṭa}
  \rdg[wit={D,P,N1,N2,U2}]{paṭa°}
  \rdg[wit={U1}]{pada°}
}sūtra\app{\lem[wit={ceteri},alt={°mayāni}]{mayāni}
  \rdg[wit={E}]{°maya}}
\app{\lem[wit={ceteri}]{dhṛtāni}
  \rdg[wit={U1}]{tāni}}
\app{\lem[wit={ceteri}]{vastrāṇi}
  \rdg[wit={U2}]{\om}}
%------------------------------ %%%%KOLLOQUIUM: was hier tun? kastūrī/kasturikā = gleichwertig 
%atha vā jīrṇāni chidrāṇi    dhṛtāni    kastūrīcandanalepair   vā  kardamalepena   yasya manasi harṣaśokau  na staḥ/ \E
%atha vā jīrṇāni sachadrāṇi  dhūtāni    kastūrīcaṃdanalepo     vā  karddamalepo vā yasya manasi harṣaśokau na staḥ/ \P
%atha vā jīrṇāni svachidrāṇi dhṛtāni    kasturīcaṃdanalepo     cā  kardamalepo  vā yasya manasi harṣaśokau na sthaḥ// \L
%atha vā jīrṇāni svachidrāṇi dhṛtāni    kastūrīcaṃdanalepo     vā  kardamalepo  vā yasya manasi harṣaśokau na sthaḥ// \B
%atha vā jīrṇāni sacchidrāṇi dhṛtāni/   kasturikā caṃdanalepo vā/ kardamalepo  vā yasya manasi harṣaśoko  na sthaḥ  \N1
%atha vā jīrṇāni sacchidrāṇi dhṛtāni//  kasturikā caṃdanalepo vā/ kardamalepo  vā yasya manasi harṣaśoko  na sthaḥ  \D
%atha vā jīrṇāni sacchidrāṇi dhṛtāni // kasturikā caṃdanalepo vā/ kardamalepo  vā yasya manasi harṣaśoka  na sthāḥ \N2
%atha vā jīrṇāni sachidrāṇi  dhvatāni   kasturikā caṃdanalepo vā  kardamalepo  vā yasya manasi harṣaśokau na sthāḥ \U1 %%272.jg
%                                       kastūrīcaṃdanalepo     vā                  yasya manasi harṣaśoko  na sta// \U2
%------------------------------
%or however, old, worn (clothes) with holes smeared with sandalwood and musk, or smeared with mud. In whose mind joy and sorrow are not situated,
%------------------------------
atha vā jīrṇāni
\app{\lem[wit={N1,N2,D}]{sacchidrāṇi}
  \rdg[wit={U2}]{sachidrāṇi}
  \rdg[wit={P}]{sachadrāṇi}
  \rdg[wit={L,B}]{svachidrāṇi}
  \rdg[wit={E}]{chidrāṇi}}
\app{\lem[wit={ceteri}]{dhṛtāni}
  \rdg[wit={U2}]{dhvātāni}
  \rdg[wit={P}]{dhūtāni}}
\begin{otherlanguage}{english}\ldots\end{otherlanguage}
\end{prose} 
  \end{edition}
  \begin{translation}
 \ekddiv{type=trans}
  \centerline{\textrm{\small{[\uproman{19}.\textsuperscript{\coro{\lowroman{1}-\lowroman{15}}} Another Sign]}}}
    \label{anjacihnatrans}   
    \bigskip
    \begin{tlate}
      \noindent
   Another sign of Rājayoga is described. Even [when] attaining a kingdom\footnote{The term \textit{rājya} could also designate a royal title. In the context of the source \citetitle{ramatosana}, Ed. pp. 834-835 it seems more convincing that a kingdom is meant here.} etc., the perception that a gain has taken place\footnote{I have understood \textit{phalalābhaḥ} as a genitive \textit{tatpuruṣa} in which I took \textit{lābhaḥ} as perception in the sense of the German word ``Auffassung'' (cf. \citetitle{petersburger5}, p. 223) and \textit{phala} in the sense of gain, win or profit, in order to resemble the great equanimity of the Rājayogin, which I think shall be expressed here.} does not arise. Even in loss [of it], suffering does not arise in the mind, neither does desire arise. And then, with regard to whatever object that has been obtained, aversion does not arise towards the object.\textsuperscript{\coro{[\lowroman{5}]}} Concerning this object, affection of the mind does not arise. Just this is said to be Rājayoga. And then the mind of him, who knows the Veda (\textit{śruti}), is equal towards [any] person - [be it] friend and enemy. A neutral view arises.\footnote{The impartial view onto all things is expressed, e.g. in \citetitle{birch2013} 1.24 (yadā sarvasamo jāto bhaved vyāpāravarjitaḥ | parabrahmaṇi sambaddho yogī prāptalayas tadā || 24 ||) ``When he has become equanimous towards all living beings and is free from activity, then the yogin, who has united oneself into the highest Brahman, has reached absorption.'' Also cf. \citetitle{birch2013} 1.26cd (vicāraṃ cendriyārthānāṃ na vetti hi layaṃ gataḥ || 26 ||). ``The yogin who was reached absorption gives no thought to sense objects.''}
Amidst the whole earth, which is like the comings and goings of joys and pleasures, pride of authorship does not arise in his mind.\textsuperscript{\coro{[\lowroman{10}]}} While wandering the world, he does not wish to know authorship. This is also said to be Rājayoga. New durable clothes made of silk, or old, worn [clothes] with holes smeared with sandalwood and \ldots 
 \end{tlate}
\end{translation}
\end{alignment}
\ekdpb*{}
%%%%%%%%%%%%%%%%%%%%%%%%%%%%%%%%%%%%%%%%%% 
%%%%%%%%%%%%%%%%%%%%%%%%%%%%%%%%%%%%%%%%%%
%%%%%%%%PAGEBREAK%%%%%%%PAGEBREAK%%%%%%%%%
%%%%%%%%%%%%%%%%%%%%%%%%%%%%%%%%%%%%%%%%%%
%%%%%%%%%%%%%%%%PAGEBREAK%%%%%%%%%%%%%%%%%
%%%%%%%%%%%%%%%%%%%%%%%%%%%%%%%%%%%%%%%%%%
%%%%%%%%PAGEBREAK%%%%%%%PAGEBREAK%%%%%%%%%
%%%%%%%%%%%%%%%%%%%%%%%%%%%%%%%%%%%%%%%%%%
%%%%%%%%%%%%%%%%%%%%%%%%%%%%%%%%%%%%%%%%%%
%%%%%%%%%%%%%%%%%%%%%%%%%%%%%%%%%%%%%%%%%%
%%%%%%%%%%%%%%%%%%%%%%%%%%%%%%%%%%%%%%%%%%
%%%%%%%%PAGEBREAK%%%%%%%PAGEBREAK%%%%%%%%%
%%%%%%%%%%%%%%%%%%%%%%%%%%%%%%%%%%%%%%%%%%
%%%%%%%%%%%%%%%%PAGEBREAK%%%%%%%%%%%%%%%%%
%%%%%%%%%%%%%%%%%%%%%%%%%%%%%%%%%%%%%%%%%%
%%%%%%%%PAGEBREAK%%%%%%%PAGEBREAK%%%%%%%%%
%%%%%%%%%%%%%%%%%%%%%%%%%%%%%%%%%%%%%%%%%%
%%%%%%%%%%%%%%%%%%%%%%%%%%%%%%%%%%%%%%%%%%
%%%%%%%%%%%%%%%%%%%%%%%%%%%%%%%%%%%%%%%%%%
%%%%%%%%%%%%%%%%%%%%%%%%%%%%%%%%%%%%%%%%%%
%%%%%%%%PAGEBREAK%%%%%%%PAGEBREAK%%%%%%%%%
%%%%%%%%%%%%%%%%%%%%%%%%%%%%%%%%%%%%%%%%%%
%%%%%%%%%%%%%%%%PAGEBREAK%%%%%%%%%%%%%%%%%
%%%%%%%%%%%%%%%%%%%%%%%%%%%%%%%%%%%%%%%%%%
%%%%%%%%PAGEBREAK%%%%%%%PAGEBREAK%%%%%%%%%
%%%%%%%%%%%%%%%%%%%%%%%%%%%%%%%%%%%%%%%%%%
%%%%%%%%%%%%%%%%%%%%%%%%%%%%%%%%%%%%%%%%%%
\begin{alignment}[
  texts=edition[class="edition"];
  translation[class="translation"],
  ]
  \begin{edition}
    \ekddiv{type=ed}
    \begin{prose}
      \noindent
      \app{\lem[wit={X}]{kasturikā}
  \rdg[wit={B,E,P,U2}]{kastūrī}
  \rdg[wit={L}]{kasturī}
}candana\app{\lem[wit={E},alt={lepair}]{lepai\skp{r-vā}}
  \rdg[wit={ceteri}]{lepo}
}\app{\lem[wit={ceteri},alt={vā}]{\skm{r-vā}}
  \rdg[wit={L}]{cā}}
\app{\lem[wit={E}]{kardamalepena}
  \rdg[wit={ceteri}]{kardamalepo}}
\app{\lem[wit={ceteri}]{vā}
  \rdg[wit={E}]{\om}}
yasya manasi
harṣa\app{\lem[wit={ceteri},alt={°śokau}]{śokau}
  \rdg[wit={D,N1,U2}]{°śoko}
  \rdg[wit={N2}]{°śoka}}
na
\app{\lem[type=emendation, resp=egoscr]{sthau}
  \rdg[wit={ceteri}]{sthaḥ}
  \rdg[wit={N2,U1}]{sthā}
  \rdg[wit={U2}]{sta}}
%------------------------------
%sa evātra tiṣṭhati/         \E
%sa eva rājayogaḥ            \P
%sa eva rājayogaḥ// idānīṃ// \L
%sa eva rājayogaḥ// idānīṃ// \B
%sa eva rājayogaḥ//          \N1
%sa eva rājayogaḥ//          \D
%sa eva rājayogaḥ//          \N2
%sa eva rājayogaḥ            \U1
%sa eva rājayoga             \U2
%------------------------------
%just he is in the state of Rājayoga. 
%------------------------------
%yasya janmamaraṇe na staḥ sukhaṃ na bhavati/ kulaṃ na bhavati śīlaṃ na bhavati/ sthānaṃ na bhavati/ \E
%\om \P
%\om \L
%\om \B
%\om \N1
%\om \D
%\om \N2
%\om \U1
%\om \U2
%------------------------------
%One who is not situated in birth and death has no happiness, has no family, and cold does not arise, place does not arise.?!?!!?
%----------------------------
\app{\lem[wit={ceteri}]{sa eva}
  \rdg[wit={E}]{sa evātra}}
\app{\lem[wit={ceteri}]{rājayogaḥ}
  \rdg[wit={U2}]{rājayoga}
  \rdg[wit={B,L}]{rājayogaḥ || idānīṃ ||}
  \rdg[wit={E}]{tiṣṭhati | yasya janmamaraṇe na staḥ sukhaṃ na bhavati | kulaṃ na bhavati śīlaṃ na bhavati | sthānaṃ na bhavati |}}/
%---------
%rājayogaḥ naramadhye      atha ca vanamadhye             yuddhe saṃgrāmamadhye                        vā yasya manaḥ        bhayapūrṇaṃ vā  na bhavati/  so pi rājayogaḥ kathyate// \E
%          nagaramadhye    'tha ca vanamadhye                  utasaṃgrāmamadhye                       vā yasya mana      ūnaṃ    pūrṇaṃ vāṃ na bhavati   so pi rājayogaḥ            \P
%          nagaramadhye     tha ca vanamadhye                 udvastagrāmamadhye                       vā yasya manaḥ     unaṃ    pūrṇaṃ vā  na bhavati   so pi rājayogaḥ//          \L
%          nagaramadhye  (')tha ca vanamadhye                udvastagrāmaṃmadhye                       vā yasya manaḥ     unaṃ    pūrṇaṃ vā  na bhavatī   so pi rājayogaḥ//          \B
%          nagaramadhye    atha ca vanamadhye/                 udvesūgrāmamadhye .. ..pūrṇagrāmamadhye vā yasya manaḥ     ūnaṃ na pūrṇaṃ vā  na bhavati// so pi rājayogaḥ//          \N1
%          ṣagaramadhye    atha ca vanamadhye//                udvesūgrāmamadhye svetapūrṇagrāmamadhye vā yasya manaḥ     ūnan na pūrṇaṃ vā  na bhavati/  so pi rājayogaḥ//          \D
%          nagaramadhye    atha ca vanamadhye//                udvesūgrāmamadhye svetapūrṇagrāmamadhye vā yasya manaḥ     ūnan na pūrṇaṃ vā  na bhavati/  so pi rājayogaḥ//          \N2
%       vā nagaramadhye    atha ca vanamadhye                 udassaṃgrāmamadhye  lokapūrṇagrāmamadhye vā yasya manaḥ     unaṃ    pūrṇaṃ     na bhavati   so pi rājayogaḥ            \U1
%          nagaramadhye    'tha vā vanamadhye                  udvasagrāmamadhye                       vā yasya mana      ūnaṃ    pūrṇaṃ vāṃ na bhavati   so pi rājayogaḥ            \U2
%------------------------------
%Just he is in the state of Rājayoga for whom the mind is neither in abundance nor in lack, being located in a city, a forest, an uninhabited village or a village full of people. 
%----------------------------
\note[type=source, labelb=127b, lem={\textbf{Re}}]{PT\textsuperscript{qcr \cdot YSV} (Ed. p. 835): harṣaśokau na jātveṣāṃ nodvego lokasaṅgame | nityollāse nirākāre nirāsane nirātmani | manasā niścalo bhūtvā sadā tiṣṭhet samo 'pi ca |}
\note[type=philcomm, labelb=127c, lem={nagaramadhye \ldots}]{Corresponding prose version of the original with extensive editorial changes in \uproman{20}.\lowroman{13}-\lowroman{15}.}
\app{\lem[wit={ceteri}]{nagaramadhye}
  \rdg[wit={E}]{rājayogaḥ nagaramadhye}
  \rdg[wit={D}]{ṣagaramadhye}
  \rdg[wit={U1}]{vā nagaramadhye}}
\app{\lem[wit={P,L,B,U2}]{'tha ca}
  \rdg[wit={D,E,N1,N2,U1}]{atha ca}}
vanamadhye
\app{\lem[wit={U2},alt={udvasa°}]{udvasa}
  \rdg[wit={E}]{yuddhe saṃ°}
  \rdg[wit={P}]{utasaṃ°}
  \rdg[wit={B,L}]{udvasta°}
  \rdg[wit={D,N1,N2}]{udvesū°}
  \rdg[wit={U1}]{udassaṃ°}
}\app{\lem[wit={ceteri}]{grāmamadhye}
  \rdg[wit={B}]{grāmaṃ madhye}}
\app{\lem[wit={U1}]{lokapūrṇagrāmamadhye}
  \rdg[wit={N1}]{....pūrṇagrāmamadhye}
  \rdg[wit={D,N2}]{svetapūrṇagrāmamadhye}}
vā yasya
\app{\lem[wit={P,U2}]{mana}
  \rdg[wit={ceteri}]{manaḥ}}
\app{\lem[wit={P,N1,N2,U2}]{ūnaṃ}
  \rdg[wit={D,N2}]{ūnan}
  \rdg[wit={B,L,U1}]{unaṃ}
  \rdg[wit={E}]{bhaya°}}
\app{\lem[wit={D,N1,N2}]{na}
  \rdg[wit={ceteri}]{\om}}
pūrṇaṃ
\app{\lem[wit={ceteri}]{vā}
  \rdg[wit={P,U2}]{vāṃ}
  \rdg[wit={U1}]{\om}}
na bhavati/ so
\app{\lem[type=emendation, resp=egoscr]{'pi}
  \rdg[wit={ceteri}]{pi}}
\app{\lem[wit={ceteri}]{rājayogaḥ}
  \rdg[wit={E}]{rājayogaḥ kathyate}}\dd{}\textsuperscript{\begin{otherlanguage}{english}\coro{[\lowroman{15}]}\end{otherlanguage}}
\end{prose} \vfill
%\ekddiv{type=ed}
  \nolinenumbers
%  \smallskip
  \centerline{[\uproman{20}.\textsuperscript{\coro{\lowroman{1}-\lowroman{9}}}\textrm{\small{Caryāyoga]}}}
  \label{caryayoga}
  \bigskip
  \linenumbers
  \begin{prose}
    \noindent
%----------------------------
%idānīṃ      yogaḥ  kathyate/ \E
%idānīṃ caryāyogaḥ  kathyate   \P
%idānīṃ caryāyogaḥ  kathyate// \L
%idānīṃ caryāyogaḥ  kathyate// \B
%idānīṃ caryāyoga   kathyate// \N1
%idānīṃ caryāyogaḥ  kathyate// \D [S.7, Z.7]
%idānīṃ caryāyoga   kathyate// \N2
%idānīṃ tvaryāyogaḥ kathyate \U1
%idānīṃ caryāyoga   kathyate// \U2
%------------------------------
%Now \textit{caryāyogaḥ}, the Yoga of wandering is explained.
%----------------------------
idānīṃ
\app{\lem[wit={ceteri}]{caryāyogaḥ}
     \rdg[wit={U1}]{tvaryāyogaḥ}
     \rdg[wit={E}]{yogaḥ}} kathyate/
%----------------------------
%nirākāro         nityo 'bhedyaḥ    sa etādṛśaḥ ātmani                  mano   yasya  niścalaṃ tiṣṭhati/  \E
%nirākāro  'calo  nityo  bhedhyaḥ   sa etādṛa   ātmā    etādṛśo  ātmani mano   yasya  niścala  tiṣṭhati   \P %%%7639.jpg
%nirākāro  calo   nityo  bhedhyaḥ   sa etādṛa   ātmā sa etādṛśe  ātmani               niścala  tiṣṭhati/  \L     %daṇḍa nach ātmā besser -> emend? oder in weiteren Hss?
%nirākāro  calo   nityo  bhedhyaḥ   sa etādṛa   ātmā sa etādṛśye ātmani               niścalaṃ tiṣṭhati/  \B
%nirākālo  nityo   calo 'bhedhyaḥ/  sa etādṛśaḥ ātmā    etādṛśe  ātmani manaḥ  yasya  niścalaṃ tiṣṭhati   \N1
%nirākālo  nityo   calo 'bhedhyaḥ// sa etādṛśaḥ ātmā    etādṛśe  ātmani manaḥ  yasya  niścalaṃ tiṣṭhati   \D
%nirākālo  nityo   calo 'bhedhyaḥ   sa etādṛśaḥ ātmā    etādṛśa  ātmani manaḥ  yasya  niścala  tiṣṭhati/  \N2
%nirākāro  nityo   calo abhedhyaḥ   sa etādṛśaḥ ātmā    etādṛśo  ātmani mano   yasya  niścalaṃ bhavati    \U1
%nirvikāro  'calo nityo 'bhedhya    sa etādṛśā  ātmani                  mano   yasya  niścalaṃ tiṣṭhati// \U2
%------------------------------
%Shapeless, unchangeable, permanent [and] unsplitable - such is the self. It is seen as such by the one whose mind abides in the self without moving. 
%------------------------------
   \note[type=source, labelb=128, lem={\textbf{Re}}]{PT\textsuperscript{qcr \cdot YSV} (Ed. p. 835): harṣaśokau na jātveṣāṃ nodvego lokasaṅgame | nityollāse nirākāre nirāsane nirātmani | manasā niścalo bhūtvā sadā tiṣṭhet samo 'pi ca |}
   \note[type=philcomm, labelb=129, lem={caryāyogaḥ}]{Caryāyoga is not mentioned in PT\textsuperscript{qcr \cdot YSV}, Rāmacandra however, utilizes this passage to construe this type of yoga. Due to its brevity it must be seen as an attempt to do justice to the list of Yogas provided in the beginning, which obviously fails in the course of his compilation.}
\app{\lem[wit={B,E,L,P,U1}]{nirākāro}
  \rdg[wit={D,N1,N2}]{nirākālo}
  \rdg[wit={U2}]{nirvikāro}}
\app{\lem[wit={P,U2}]{'calo}
  \rdg[wit={B,L}]{calo}
  \rdg[wit={X}]{nityo}
  \rdg[wit={E}]{\om}}
\app{\lem[wit={Y}]{nityo}
  \rdg[wit={X}]{calo}}
\app{\lem[wit={D,E,N1,N2}]{'bhedyaḥ}
  \rdg[wit={B,L,P}]{bhedhyaḥ}
  \rdg[wit={U1}]{abhedhyaḥ}
  \rdg[wit={U2}]{'bhedyha}}
   sa
\app{\lem[wit={B,L,P}]{etādṛśa}
  \rdg[wit={D,E,N1,N2,U1}]{etādṛśaḥ}
  \rdg[wit={U2}]{etādṛśā}}
\app{\lem[wit={ceteri}]{ātmā}
  \rdg[wit={E,U2}]{ātmani}}/
\app{\lem[wit={L,B}]{sa}
  \rdg[wit={ceteri}]{\om}}
\app{\lem[wit={N2}]{etādṛśa}
  \rdg[wit={P,U1}]{etādṛśo}
  \rdg[wit={D,L,N1}]{etādṛśe}
  \rdg[wit={B}]{etādṛśye}
  \rdg[wit={E,U2}]{\om}}
\app{\lem[wit={ceteri}]{ātmani}
  \rdg[wit={E,U2}]{\om}}
\app{\lem[wit={E,P,U1,U2}]{mano}
  \rdg[wit={D,N1,N2}]{manaḥ}
  \rdg[wit={B,L}]{\om}}
\app{\lem[wit={ceteri}]{yasya}
  \rdg[wit={B,L}]{\om}}
\app{\lem[wit={ceteri}]{niścalaṃ}
  \rdg[wit={P,L,N2}]{niścala}}
\app{\lem[wit={ceteri}]{tiṣṭhati}
  \rdg[wit={U1}]{bhavati}}/
%------------------------------
%tasyātmanaḥ puṇyapāpasparśo na bhavati/ \E
%tasyātmanaḥ puṇyapāpasparśo na bhavati  \P
%tasyātmanaḥ puṇyapāpasparśo na bhavati/ \L
%tasyātmanaḥ puṇyapāpasparśo na bhavatī/ \B
%tasyātmanaḥ punyapāpasparśo na bhavati/  \N1
%tasyātmanaḥ punyapāpasparśo na bhavati// \D
%tasyātmanaḥ puṇyapāpasparśo na bhavati/ \N2
%tasya ātmanaḥ puṇyapāsya sparśo na bhavati  \U1
%tasya ātmanaḥ puṇyapāsya sparśo na bhavati//  \U2
%------------------------------
%His self is not touched by sin and merit. 
%------------------------------
\app{\lem[wit={ceteri}]{tasyātmanaḥ}
  \rdg[wit={U1,U2}]{tasya ātmanaḥ}}
\app{\lem[wit={ceteri}]{puṇyapāpasparśo}
  \rdg[wit={U1,U2}]{puṇyapāsya sparśo}}
na bhavati/
%------------------------------
%udakamadhye sthitasya padmapatre       yathodakasya sparśo    bhavati/  tathaivātmani   \E
%udakamadhye sthitasya padmanī patrasya yathodakasya sparśo na bhavati   tathaivātmani   \P
%udakamadhye sthitasya padmanī patrasya yathodakasya sparśo na bhavati/  tathaivātmani   \L
%udakamadhye sthitasya padmanī patrasya yathodakasya sparśā na bhavatī/  tathaivātmani   \B
%udakamadhye sthitasya padminī patrasya yathā/ udakasparśo  na bhavati/  tathaivātmani   \N1
%udakamadhye sthitasya padminī patrasya yathā  udakasparśo  na bhavati// tathaivātmani   \D
%udakamadhye sthitasya padminī patrasya yathā  udakasparśo  na bhavati/  tathaivātmani   \N2
%udakamadhye sthitasya padminī patrasya yathā  udakasparśo  na bhavati   tathaivātmani   \U1
%udakamadhye sthitasya padminī patrasya yathodakasparśo     na bhavati// tathaivātmani   \U2
%------------------------------
%Just as the leave of the lotus situated in the amidst water doesn't touch the water; likewise the self [is not touched by sin and merit].
%------------------------------
udakamadhye sthitasya
\app{\lem[wit={ceteri}]{padminī patrasya}
  \rdg[wit={B,L,P}]{padmanī patrasya}
  \rdg[wit={E}]{padmapatre}}
\app{\lem[wit={E,P,L}]{yathodakasya sparśo}
  \rdg[wit={B}]{yathodakasya sparśā}
  \rdg[wit={X}]{yathā udakasparśo}
  \rdg[wit={U2}]{yathodakasparśo}}
na
\app{\lem[wit={ceteri}]{bhavati}
  \rdg[wit={B}]{bhavatī}}
tathaivātmani/\textsuperscript{\begin{otherlanguage}{english}\coro{[\lowroman{5}]}\end{otherlanguage}}
%------------------------------
%yathākāśamadhye   pavanaḥ svecchayā bhramati/ \E
%yathākāśamadhye   pavanaḥ svechayā  bhramati \P
%yathā ākāśamadhye pavanaḥ svechayā  bhramati/ \L
%yathā ākāśamadhye pavanaḥ svechayā  bhramatī/ \B
%yathā ākāśamadhye pavanasvachayā    bhramati/ \N1
%yathā ākāśamadhye pavanasvachayā    bhramati \D
%yathā ākāśamadhye pavanasvachayā    bhramati/ \N2
%yathā ākāśamadhye pavanaḥ svechayā  bhramayati \U1
%yathā 'kāśamadhye pavanaḥ svechayā  bhramati// \U2
%------------------------------
%Just as the wind wanders according to its own will in space,...  
%------------------------------
\note[type=source, labelb=130, lem={pavanaḥ}]{PT\textsuperscript{qcr \cdot YSV} (Ed. p. 835): yathākāśe bhraman vāyur ākāśaṃ vrajate svayam | tathākāśe mano līnaṃ rājayogakriyā matā | jagatsaṃsarganirlepaṃ padmapatrajalaṃ yathā |}
yathā\app{\lem[wit={E,P}]{kāśamadhye}
  \rdg[wit={U2}]{'kāśamadhye}
  \rdg[wit={ceteri}]{ākāśamadhye}}
\app{\lem[wit={ceteri}]{pavanaḥ svechayā}
  \rdg[wit={D,N1,N2}]{pavanasvachayā}}
\app{\lem[wit={ceteri}]{bhramati}
  \rdg[wit={U1}]{brahmayati}}/
%------------------------------
%tathā yasya manaḥ nirākāramadhye līnaṃ bhavati/  sa eva caryāyogaḥ// \E
%tathā yasya manaḥ nirākāramadhye līnaṃ bhavati   sa eva caryāyogaḥ   \P
%tathā yasya manaḥ nirākāramadhye līnaṃ bhavati   sa eva caryāyogaḥ// \L
%tathā yasya manaḥ nirākāramadhye līnaṃ bhavatī   sa eva caryāyogaḥ// \B
%tathā yamanaḥ     nirākāramadhye līnaṃ bhavati/  sa eva kriyāyogaḥ// \N1
%tathā yasya manaḥ nirākāramadhye līnaṃ bhavati/  sa eva kriyāyogaḥ// \D !!!!!Stemma point!!!!!!
%tathā       pavananirākāramadhye līnaṃ bhavati/  sa eva kriyāyogaḥ// \N2
%tathā yasya manaḥ nirākāramadhye līnaṃ bhavati   sa eva kriyāyogaḥ   \U1 
%tathā yasya manaḥ nirākāramadhye līnaṃ bhavati// sa eva caryāyogaḥ// \U2
%------------------------------
%Likewise is the mind of whom is absorbed into the universal spirit [wanders according to its own will in space]. This is \textit{\caryāyoga}.  
%------------------------------
tathā
\app{\lem[wit={ceteri}]{yasya manaḥ}
  \rdg[wit={D}]{yamanaḥ}
  \rdg[wit={N2}]{pavana°}}
nirākāramadhye līnaṃ
\app{\lem[wit={ceteri}]{bhavati}
  \rdg[wit={B}]{bhavatī}}/
sa eva
\app{\lem[wit={ceteri}]{caryāyogaḥ}
  \rdg[wit={X}]{kriyāyogaḥ}}\dd{}\textsuperscript{\begin{otherlanguage}{english}\coro{[\lowroman{9}]}\end{otherlanguage}}
\vspace*{\fill}
\end{prose}
  \end{edition}
  \begin{translation}
    \ekddiv{type=trans}
    \begin{tlate}
      \noindent
\ldots musk, or smeared with mud. In whose mind joy and sorrow are not situated, he is [in the state of] Rājayoga. He is in the state of Rājayoga, for whom the mind is neither in abundance nor in lack, situated in a city, a forest, an uninhabited village or a village full of people.\textsuperscript{\coro{[\lowroman{15}]}}
\end{tlate}
      %\smallskip
  \centerline{[\uproman{20}.\textsuperscript{\coro{\lowroman{1}-\lowroman{9}}}\textrm{\small{Caryāyoga]}}}
      \label{caryayogatrans}
      \smallskip
     \begin{tlate}
       Now Caryāyogaḥ is explained. Shapeless, unchangeable, permanent [and] unsplitable - such is the self. It is seen as such by the one whose mind abides in the self without moving. His self is not touched by sin and merit. Just as the leave of the lotus situated in the amidst water doesn't touch the water; likewise the self [is not touched by sin and merit].\textsuperscript{\coro{[\lowroman{5}]}} Just as the wind wanders according to its own will in space, likewise the mind of one who is absorbed into the universal spirit [wanders according to its own will in space]. This is Caryāyoga.\textsuperscript{\coro{[\lowroman{9}]}}\footnote{Identifying the source verses quoted in the PT\textsuperscript{qcr \cdot YSV} (Ed. p. 835), it seems that Rāmacandra added his Caryāyoga to do justice to the list and inventing his straightforward concept of this yoga. Rāmacandra emphasizes the cultivation of detachment towards sin (\textit{pāpa}) and merit (\textit{puṇya}). Parallels can be identified with the concept of Caryāyoga as presented in the \citetitle{yogacandrika} (Ed. pp. 2, 52-53, 100-101, 150). Here, it appears that Caryāyoga is a spiritual practice or discipline that aims to purify the mind. Nārāyaṇatīrtha introduces Caryāyoga in the context of Yogasūtra 1.33, Ed. p. 52 (maitrīkaruṇāmuditopekṣāṇāṃ sukhaduḥkhapuṇyāpuṇyaviṣayāṇāṃ bhāvanātaś cittaprasādanam). According to Nārāyaṇatīrtha's commentary, the practice of it involves cultivating specific mental attitudes, such as \textit{maitrī} (loving-kindness), \textit{karuṇā} (compassion), \textit{muditā} (sympathetic joy), and \textit{upekṣā} (equanimity), towards different objects or situations, such as happiness, suffering, merit, and demerit. The practice of Caryāyoga is said to lead to eradicating mental impurities and attaining a calm and tranquil state of mind. Nārāyaṇatīrtha characterizes the practice as the renunciation of worldly attachments and desires and the performance of selfless actions or Karmayoga. Nārayaṇatīrtha states that Caryāyoga is the "primary discipline of detachment (\textit{vairāgya})," which suggests that it emphasizes the cultivation of detachment or dispassion towards worldly objects and desires as a means of achieving spiritual liberation. Moreover, it is interesting that Rāmacandra begins his short discourse by describing characteristics of the \textit{ātman}. Within \citetitle{sarvangayoga} (2.40-51, Ed. pp. 96-98), we find a similar sounding term among the four types of Bhaktiyog going by the name of Cārcāyog. Sundardās describes Cārcāyog as a type of Bhaktiyog which is \textit{bhakti} towards unmanifest consciousness (\textit{avyakta puruṣa}) in a specific manner of delightful devotion based on the concepts he presents in his verses. The practice results in a beautiful inner being (50-51). He first describes the unmanifest consciousness (\textit{avyakta puruṣa}) as being formless and eternal and so on (40), as beginningless and endless, bringing about wonderful deeds. It is beyond words and so on (41). Next, Sundardās describes the various layers of creation emanating from \textit{oṃ} (42-45). He says the unmanifest consciousness illumines every corner of existence (46), being the inner knower of all (47). Then, in the light of the previously concepts of the unmanifest consciousness and since it is infinite, divine, all-knowing and incomprehensible (48-49), Sundardās expresses the importance of deep awe towards the unmanifest consciousness, which is the key component of his Cārcāyog type of \textit{bhakti}.}\end{tlate}
   \end{translation}
   \ekdpb*{}
 \end{alignment}
 %%%%%%%%%%%%%%%%%%%%%%%%%%%%%%%%%%%%%%%%%%
%%%%%%%%%%%%%%%%%%%%%%%%%%%%%%%%%%%%%%%%%%
%%%%%%%%PAGEBREAK%%%%%%%PAGEBREAK%%%%%%%%%
%%%%%%%%%%%%%%%%%%%%%%%%%%%%%%%%%%%%%%%%%%
%%%%%%%%%%%%%%%%PAGEBREAK%%%%%%%%%%%%%%%%%
%%%%%%%%%%%%%%%%%%%%%%%%%%%%%%%%%%%%%%%%%%
%%%%%%%%PAGEBREAK%%%%%%%PAGEBREAK%%%%%%%%%
%%%%%%%%%%%%%%%%%%%%%%%%%%%%%%%%%%%%%%%%%%
%%%%%%%%%%%%%%%%%%%%%%%%%%%%%%%%%%%%%%%%%%
%%%%%%%%%%%%%%%%%%%%%%%%%%%%%%%%%%%%%%%%%%
%%%%%%%%%%%%%%%%%%%%%%%%%%%%%%%%%%%%%%%%%%
%%%%%%%%PAGEBREAK%%%%%%%PAGEBREAK%%%%%%%%%
%%%%%%%%%%%%%%%%%%%%%%%%%%%%%%%%%%%%%%%%%%
%%%%%%%%%%%%%%%%PAGEBREAK%%%%%%%%%%%%%%%%%
%%%%%%%%%%%%%%%%%%%%%%%%%%%%%%%%%%%%%%%%%%
%%%%%%%%PAGEBREAK%%%%%%%PAGEBREAK%%%%%%%%%
%%%%%%%%%%%%%%%%%%%%%%%%%%%%%%%%%%%%%%%%%%
%%%%%%%%%%%%%%%%%%%%%%%%%%%%%%%%%%%%%%%%%%
%%%%%%%%%%%%%%%%%%%%%%%%%%%%%%%%%%%%%%%%%%
%%%%%%%%%%%%%%%%%%%%%%%%%%%%%%%%%%%%%%%%%%
%%%%%%%%PAGEBREAK%%%%%%%PAGEBREAK%%%%%%%%%
%%%%%%%%%%%%%%%%%%%%%%%%%%%%%%%%%%%%%%%%%%
%%%%%%%%%%%%%%%%PAGEBREAK%%%%%%%%%%%%%%%%%
%%%%%%%%%%%%%%%%%%%%%%%%%%%%%%%%%%%%%%%%%%
%%%%%%%%PAGEBREAK%%%%%%%PAGEBREAK%%%%%%%%%
%%%%%%%%%%%%%%%%%%%%%%%%%%%%%%%%%%%%%%%%%%
%%%%%%%%%%%%%%%%%%%%%%%%%%%%%%%%%%%%%%%%%%
\begin{alignment}[
  texts=edition[class="edition"];
  translation[class="translation"],
  ]
  \begin{edition}
    \ekddiv{type=ed}
\bigskip
\centerline{[\uproman{21}.\textsuperscript{\coro{\lowroman{1}-\lowroman{12}}}\textrm{\small{Haṭhayoga]}}}
\label{hathayoga}
    \bigskip
    \begin{prose}
      \noindent
%------------------------------
%idānīṃ grahayogaḥ kathyate/  \E %[p.23]
%idānīṃ haṭhayogaḥ kathyate   \P
%idānīṃ haṭhayogaḥ kathyate/  \L
%idānīṃ haṭayoga   kathyate/  \B
%idānīṃ haṭhayogaḥ kathyate//  \N1
%idānīṃ haṭhayogaḥ kathyate/  \D
%idānīṃ haṭhayoga  kathyate// \N2
%idānīṃ haṭhayogaḥ kathyate   \U1
%idānīṃ haṭhayoga  kathyate   \U2
%------------------------------
%Now \textit{haṭhayoga} is explained. 
%------------------------------
idānīṃ
\app{\lem[wit={D,L,P,N1,U1}]{haṭhayogaḥ}
     \rdg[wit={B}]{haṭayoga}
     \rdg[wit={E}]{grahayogaḥ}
     \rdg[wit={U2}]{haṭhayoga}} kathyate/
\note[type=source, labelb=131, lem={\textbf{Re}}]{PT\textsuperscript{qcr \cdot YSV} (Ed. p. 835): idānīṃ haṭhayogas tu kathyate haṭhasiddhidaḥ | kṛtvāsanaṃ pavanāśaṃ śarīre rogahārakam | pūrakaṃ kumbhakañcaiva recakaṃ vāyunā bhajet | itthaṃ kramotkramaṃ jñātvā pavanaṃ sādhayet sadā | dhauty ādikarmaṣaṭkañ ca prakuryād haṭhasādhakaḥ | etan nāḍyān tu deveśi vāyupūrṇaṃ pratiṣṭhitam | tato mano niścalaṃ syāt tata ānanda eva hi | haṭhayogān na kālaḥ syān manonāśo bhaved yadi |}
%------------------------------
%recakapūrakakumbhaka  ityādiprakāreṇa   pavanasādhanaṃ     kartavyam/ \E
%recakapūrakakuṃbhaka  ityādiprakāreṇa   pavanasādhanaṃ     karttavyaṃ \P
%recakapūrakakumbhaka  ityādiprakāreṇa   pavanasya sādhanaṃ kartavyam// \L
%recakapūrakakuṃbhaka  ityādiprakāreṇa// pavanasya sādhanaṃ kartavyam \B
%recakapūrakakuṃbhaka/ ityādiprakāreṇa   pavanasya sādhanaṃ kartavyaṃ/ \N1
%recakapūrakakuṃbhaka  ityādiprakāreṇa   pavanasya sādhanaṃ kartavyaṃ// \D
%recakapūrakakuṃbhaka  ityādhiprakāreṇa  pavanasya sādhanaṃ kartavyaṃ// \N2
%recakapūrakakuṃbhaka  ityādiprakāreṇa   pavanasya sādhanaṃ kartavyaṃ \U1
%recakapūrakakuṃbhaka  ityādiprakāreṇa   pavanasya sādhanaṃ kartavyaṃ// \U2
%------------------------------
%The practice of breath shall be done in this manner: "Exhalation, Inhalation [and] Retention etc.
%------------------------------        
 recakapūrakakuṃbhaka
        \app{\lem[wit={ceteri}, alt={ityādi°}]{ityādi}
          \rdg[wit={N2}]{ityādhi°}
        }prakāreṇa
        \app{\lem[wit={ceteri}]{pavanasya sādhanaṃ}
          \rdg[wit={E,P}]{pavanasādhanaṃ}}
 \app{\lem[wit={B,E,L}]{kartavyam}
   \rdg[wit={ceteri}]{kartavyaṃ}}/
%------------------------------
%atha ca dhautyādiṣaṭkarmakāraṇāt   śarīrasya śuddhir bhavati/ \E
%atha ca dhautyādiṣaṭkarmakāraṇāt   śarīrasya śuddhir bhavati \P
%atha ca dhautyādiṣaṭkarmakāraṇāt// śarīrasya śuddhir bhavati \L
%atha ca  dhotyādiṣaṭkarmakaraṇāt// śarīrasya śuddhir bhavatī \B
%atha ca dhautyādiṣaṭkarmakaraṇāt/  śarīrasya śuddhir bhavati/ \N1
%atha ca dhautyādiṣaṭkarmakaraṇāt   śarīrasya śuddhir bhavati// \D
%atha ca dhautyādiṣaṭkarmakaraṇāt// śarīrasya śuddhir bhavati// \N2
%atha   vidhotyādiṣaṭkarmakaraṇāt   śarīrasya śuddhir bhavati/ \U1
%atha ca dhautyādiṣaṭkarmakaraṇāt// śarīrasya śuddhir bhavati// \U2 %%%408.jpg 
%------------------------------
%And then due to the six practices(\textit{ṣaṭkarma}), like \textit{dhauti} etc. the purification of the body arises. 
%------------------------------        
 atha
 \app{\lem[wit={ceteri}]{ca}
   \rdg[wit={U1}]{\om}}
 \app{\lem[wit={ceteri}]{dhautyādi}
   \rdg[wit={B}]{dhotyādi}
   \rdg[wit={U1}]{vidhotyādi}
 }ṣaṭkarmakāraṇāt śarīrasya śuddhir\skp{-}bhavati/
 %------------------------------
%sūryanāḍīmadhye       pavanaḥ pūrṇo yadā tiṣṭati/   \E %!
%sūryanāḍīmadhye       pavanaḥ pūrṇo yadā tiṣṭati    \P
%sūryanāḍīmadhye       pavanapūrṇo   yadāti/         \L
%sarvasūryanāḍīmadhye  pavanapūrṇo   yadāti/         \B
%sūryanāḍīmadhye       pavanaḥ pūrṇo yadā tiṣṭhati/  \N1
%sūryanāḍīmadhye       pavanaḥ pūrṇo yadā tiṣṭhati   \D
%sūryanāḍīmadhye       pvanaḥ  pūrṇo yadā tiṣṭhati/  \N2
%sūryanāḍīmadhye       pavanaḥ pūrṇo yadā tiṣṭhati/  \U1
%sūryanāḍīmadhye       pavanaḥ sūryo yadā tiṣṭhati// \U2
%------------------------------
%When the full breath abides in the middle of the sun-channel, ... 
%------------------------------
 \app{\lem[wit={ceteri}]{sūryanāḍīmadhye}
   \rdg[wit={B}]{sarvasūryanāḍīmadhye}}
 \app{\lem[wit={ceteri}]{pavanaḥ pūrṇo}
   \rdg[wit={B,L}]{pavanapūrṇo}
   \rdg[wit={N2}]{pvanaḥ pūrṇo}}
 \app{\lem[wit={ceteri}]{yadā tiṣṭhati}
   \rdg[wit={B,L}]{yadāti}}
%------------------------------
%tadā mano  niścalaṃ bhavati/  \E
%tadā mano  niścalo  bhavati   \P
%tadā mano  niścalo  bhavati/  \L
%tadā mano  niścalo  bhavatī// \B
%tadā manaḥ niścalaṃ bhavati/  \N1
%tadā manaḥ niścalaṃ bhavati   \D
%tadā manaḥ niścalaṃ bhavati   \N2
%tadā manaḥ niścalaṃ bhavati   \U1
%tadā mano  niścalaṃ bhavati// \U2
%------------------------------
%Then the mind is unmovable. 
%------------------------------
 tadā
 \app{\lem[wit={Y}]{mano}
   \rdg[wit={X}]{manaḥ}}
\app{\lem[wit={ceteri}]{niścalaṃ}
  \rdg[wit={B,L,P}]{niścalo}}
bhavati/
%------------------------------
%manaso  niścalatvena ānandarūpaṃ      pratyakṣaṃ bhāsate/  \E
%manaso  niścalatve   ānandaṃ svarūpa--pratyakṣaṃ bhāsate   \P %%%%7640.jpg
%manaso  niścalatve   ānandaṃ svarūpaṃ pratyakṣaṃ bhāsate/  \L
%manaso  niścalatve   ānaṃdaṃ svarūpaṃ pratyakṣaṃ bhāsate// \B
%manasaḥ niścalatve   ānaṃdasvarūpaṃ   pratyakṣaṃ bhāsate/  \N1
%manasaḥ niścalatve   ānaṃdasvarūpaṃ   pratyakṣaṃ bhāsate/  \D
%manasaḥ niścalatve   ānaṃdasvarūpaṃ   pratyakṣaṃ bhāṣate/  \N2
%manasaḥ niścalatve   ānaṃdasvarūpaṃ   pratyakṣaṃ bhāṣate/  \U1 %%%273.jpg
%manaso  niścalatve   ānaṃdasvarūpaṃ   pratyakṣaṃ bhāsate// \U2
%------------------------------
%The form of bliss immediately shines through the motionless mind.  
%------------------------------
\app{\lem[wit={Y}]{manaso}
  \rdg[wit={X}]{manasaḥ}}
\app{\lem[wit={ceteri}]{niścalatve}
  \rdg[wit={E}]{niścalatvena}}
\app{\lem[wit={ceteri}]{ānandasvarūpaṃ}
  \rdg[wit={B,L}]{ānaṃdaṃ svarūpaṃ}
  \rdg[wit={P}]{ānandaṃ svarūpa°}
  \rdg[wit={E}]{ānandarūpaṃ}}
pratyakṣaṃ
\app{\lem[wit={ceteri}]{bhāsate}
  \rdg[wit={N2,U1}]{bhāṣate}}/\textsuperscript{\begin{otherlanguage}{english}\coro{[\lowroman{5}]}\end{otherlanguage}}
%------------------------------
%haṭhayogakāraṇāt  manaḥ   śūnyamadhye līnaṃ   bhavati/  kālaḥ samīpe   nāgacchati/  \E
%haṭhayogakāraṇāt  manaḥ   śūnyamadhye līnaṃ   bhavati   kālaḥ samīpe   nāgacchati   \P %%%%7640.jpg
%haṭhayogakāraṇāt  manaḥ   śūnyamadhye līnaṃ   bhavati/  kālaḥ samīpe   nāgacchati// \L
%haṭayogākāraṇāt   manaḥ// śūnyamadhye līnaṃ   bhavatī/  kālāsamīpe nāma gacchati//  \B
%haṭhayogakaraṇāt  manaḥ   śūnyamadhye līnaṃ   bhavati/  kālaḥ samīpe   nāgachati//  \N1
%haṭhayogakaraṇāt  manaḥ   śūnyamadhye līnaṃ   bhavati// kālaḥ samīpe   nāgachaṃti// \D
%haṭhayogakaraṇāt  mana----śūnyamadhye līnaṃ   bhavati/  kālasamīpe     nāgachati//  \N2
%haṭhayogakaraṇāt/ manaḥ   śūnyamadhye līnaṃ   bhavati/  kālasamīpe ti  nāgachati    \U1 %%%273.jpg
%haṭhayogakaraṇāt  manaḥ   śūnyamadhye sthānaṃ bhavati// kāsaḥ samīpe   nāgachati//  \U2
%------------------------------
%Due to the execution of haṭhayoga the mind becomes absorbed into emptiness. The time of death does not approach.
%------------------------------
\app{\lem[wit={ceteri}, alt={haṭha°}]{haṭha}
  \rdg[wit={B}]{haṭa°}
}\app{\lem[wit={ceteri},alt={yoga°}]{yoga}
  \rdg[wit={B}]{yogā°}
}\app{\lem[wit={ceteri}]{karaṇāt}
  \rdg[wit={B,E,L,P}]{kāraṇāt}}
\app{\lem[wit={ceteri}]{manaḥ}
  \rdg[wit={N2}]{mana}}
śūnyamadhye
\app{\lem[wit={ceteri}]{līnaṃ}
  \rdg[wit={U2}]{sthānaṃ}}
bhavati/
\app{\lem[wit={ceteri}]{kālaḥ}
  \rdg[wit={B}]{kālā°}
  \rdg[wit={N2,U1}]{kāla°}
  \rdg[wit={U2}]{kāsaḥ}}
samīpe
\app{\lem[wit={ceteri}]{nāgacchati}
  \rdg[wit={B}]{nāma gacchati}
  \rdg[wit={D}]{nāgachaṃti}
  \rdg[wit={U1}]{ti nāgachati}}\dd{}\\
%------------------------------
%idānīṃ haṭhayogasya dvitīyo  bhedaḥ kathyate/   \E
%idānīṃ haṭhayoga----dvitīya--bhedaḥ kathyate    \P
%idānīṃ haṭhayogasya dvitīya--bhedāḥ kathyante/  \L
%idānīṃ haṭayogasya  dvitīyaṃ bhedāḥ kathyaṃte// \B
%idānīṃ haṭhayogasya dvitīyo  bhedaḥ kathyate//  \N1
%idānīṃ haṭhayogasya dvitīya--bhedaḥ kathyate    \D
%idānīṃ haṭayogasya  dvitīyo  bhedaḥ kathyate    \U1
%idānīṃ haṭhayogasya dvitīyo  bhedaḥ kathyate//  \U2 
%------------------------------
%Now, the second division of haṭhayoga is explained.
%------------------------------
idānīṃ
\app{\lem[wit={ceteri}]{haṭhayogasya}
  \rdg[wit={B,U1}]{haṭayogasya}
  \rdg[wit={P}]{haṭhayoga°}}
\app{\lem[wit={ceteri}]{dvitīyo}
  \rdg[wit={D,L,P}]{dvitīya°}
  \rdg[wit={B}]{dvitīyaṃ}}
\app{\lem[wit={ceteri}]{bhedaḥ}
  \rdg[wit={B,L}]{bhedāḥ}}
\app{\lem[wit={ceteri}]{kathyate}
  \rdg[wit={B,L}]{kathyante}}/ \note[type=source, labelb=132, lem={\textbf{Re}}]{PT\textsuperscript{qcr \cdot YSV} (Ed. p. 835): idānīṃ haṭhayogasya dvitīyaṃ bhedam acchṛṇu | ākāśe nāsikāgre tu sūryakoṭisamaṃ smaret | śvetaṃ raktaṃ tathā pītaṃ kṛṣṇam ity ādirūpataḥ | evaṃ dhyātvā cirāyus syād aṅgājananavarjitam (\textit{°varjitaḥ} YK\textsuperscript{ccn \cdot YSV} 12.25 Ed. p. 108; possibly em. to aṅgajaraṇavarjitaḥ or aṅgajvaranavarjitaḥ) | śivatulyo mahātmāsau haṭhayogaprasādataḥ (\textit{°prasaṅgataḥ} YK\textsuperscript{ccn \cdot YSV} 12.25 Ed. p. 108) | haṭhāj jyotir (\textit{haṭha°} YK\textsuperscript{ccn \cdot YSV} 12.26 Ed. p. 108) mayo bhūtvā hyantareṇa śivo bhavet | ato 'yaṃ haṭhayogaḥ syāt siddhidaḥ siddhasevitaḥ |}
%------------------------------
%pādādārabhya śiraḥ paryaṃtaṃ    svaśarīre  koṭisūryatejaḥ   samānaṃ śvetaṃ pītaṃ       raktaṃ kiṃcidvarṇaṃ ciṃtyate/  \E
%pādādārabhya śiraḥ paryaṃtaṃ    svaśarīre  koṭisūryatejaḥ   samānaṃ śvetaṃ pītaṃ nīlaṃ raktaṃ kiṃdrupaṃ    cityate    \P
%pādādārabhya śira--paryaṃtaṃ    svaśarīre  koṭisūryatejaḥ   samānaśvetaṃ nīlaṃ         raktaṃ tiṃdrupaṃ    ciṃtate/   \L
%pādādārabhya śira--paryaṃtaṃ    svaśarīre  koṭisūryatejaḥ// samānaśvetanīlaṃ           raktaṃ kiṃdrupaṃ    ciṃtate//  \B
%pādādārabhyā śiraḥ paryentaṃ    svaśarīre  koṭisūryatejaḥ   samānaṃ śvetaṃ pītaṃ nīlaṃ laktaṃ kiṃcidrūpaṃ  ciṃtyate   \N1 
%pādādārabhyā śiraḥ paryaṃtaṃ    svaśarīre  koṭisūryatejaḥ   samānaṃ śvetaṃ pītaṃ nīlaṃ raktaṃ kiṃcidrūpaṃ  ciṃtyate   \D
%pādādārabhya śiraḥ pariyataṃ    svaśarīraṃ koṭisūryatejaḥ   samānaṃ śvetaṃ pītaṃ nīlaṃ raktaṃ ciṃrūpaṃ     ciṃtyate   \U1
%pādādārabhya śiro  paryaṃtaṃ    svaśarīre  koṭisūryye tejaḥ samānaṃ śvetaṃ pītaṃ nīlaṃ raktaṃ kiṃcidrūpaṃ  ciṃtyate// \U2
%------------------------------
%The shine of ten million suns in one's own body beginning from the feet to the top of head is contemplated in any color equal to white, yellow [or] red.
%------------------------------
\app{\lem[wit={ceteri}]{pādādārabhya}
  \rdg[wit={N1,D}]{pādādārabhyā}}
\app{\lem[wit={ceteri}]{śiraḥ}
  \rdg[wit={B,L}]{śira°}
  \rdg[wit={U2}]{śiro}}
\app{\lem[wit={ceteri}]{paryantaṃ}
  \rdg[wit={N1}]{paryentaṃ}
  \rdg[wit={U1}]{pariyataṃ}}
\app{\lem[wit={ceteri}]{svaśarīre}
  \rdg[wit={U1}]{svaśarīraṃ}}
\app{\lem[wit={ceteri}]{koṭisūryatejaḥ}
  \rdg[wit={U2}]{koṭisūryye tejaḥ}}
\app{\lem[wit={ceteri}]{samānaṃ}
  \rdg[wit={B,L}]{samāna°}}
  \app{\lem[wit={ceteri}]{śvetaṃ}
  \rdg[wit={B}]{śveta°}}
\app{\lem[wit={ceteri}]{pītaṃ}
  \rdg[wit={B,L}]{\om}}
nīlaṃ
\app{\lem[wit={ceteri}]{raktaṃ}
  \rdg[wit={N1}]{laktaṃ}}
\app{\lem[wit={D,N1,U2}]{kiṃcidrūpaṃ}
  \rdg[wit={B,P}]{kiṃdrupaṃ}
  \rdg[wit={L}]{tiṃdrupaṃ}
  \rdg[wit={U1}]{ciṃrūpaṃ}
  \rdg[wit={E}]{kiṃcidvarṇaṃ}}
\app{\lem[wit={ceteri}]{cintyate}
  \rdg[wit={P}]{cityate}
  \rdg[wit={B,L}]{ciṃtate}}/
%------------------------------
%ttad  dhyānakāraṇāt     sakalaṃ   rogajvalanaṃ     bhavati/                      āyur          vardhate/          \E
%tad   dhyānakāraṇāt     sakalāṃge rogajvalanaṃ  na bhavati                       āyur vṛddhir  bhavati   \P
%tad   dhyānakāraṇāt     sakalaṃge rogajvalanaṃ  na bhavati/                      āyur          vardhate/          \L
%tat   dhyānakāraṇāt     sakalaṃge rogajvalanaṃ  na bhavati/                      āyur vṛddhir  bhavatī/  \B
%na    dhyānaṃ kāraṇāt/  sakalāṃge roga          na bhavati/  jvalanaṃ na bhavati āyur vṛddhir  bhavati/  \N1
%ta    dhyānaṃ karaṇāt// sakalāṃge rogajvalanaṃ  na bhavati//                                             \D
%tad---dhyānaṃ karaṇāt / sakalāṃge roga          na bhavati   jvaranaṃ na bhavati āyu--vṛddhir  bhavati// \N2
%ta    dhyānaṃ karaṇāt   sakalāṃge roga kṣataṃ?  na bhavati                       āyur vṛddhir  bhavati   \U1
%tat   dhyānakāraṇāt     sakalāṃge rogajvalanaṃ     bhavati//                     āyur vṛddhir  bhavati// \U2
%------------------------------
%aDue to the execution of meditation in the entire body disease does'nt arise, fever doesn't arise and vitality grows.  
%------------------------------
\app{\lem[wit={E,L,P,N2},alt={tad}]{ta\skp{d-dhyā}}
  \rdg[wit={B,U2}]{tat}
  \rdg[wit={D,U1}]{ta}
  \rdg[wit={N1}]{na}
}\app{\lem[wit={Y},alt={dhyānakāraṇāt}]{\skm{d-dhyā}nakāraṇāt}
  \rdg[wit={X}]{dhyānaṃ karaṇāt}}
\app{\lem[wit={X,P,U2}]{sakalāṅge}
  \rdg[wit={B,L}]{sakalaṃge}
  \rdg[wit={E}]{sakalaṃ}}
\app{\lem[type=emendation, resp=egoscr]{rogaḥ}
\rdg[wit={N1,N2}]{roga}
\rdg[wit={B,D,E,L,D,P,U2}]{rogajvalanaṃ}
\rdg[wit={U1}]{roga kṣataṃ}}
\app{\lem[wit={ceteri}]{na}
  \rdg[wit={E,U2}]{\om}}
bhavati/\textsuperscript{\begin{otherlanguage}{english}\coro{[\lowroman{10}]}\end{otherlanguage}}
\app{\lem[wit={N2}]{jvaranaṃ na bhavati}
  \rdg[wit={N1}]{jvalanaṃ na bhavati}
  \rdg[wit={ceteri}]{\om}}/
\app{\lem[wit={ceteri}, alt={āyur}]{āyu\skp{r-vṛ}}
  \rdg[wit={N2}]{āyu°}
  \rdg[wit={D}]{\om}
}\app{\lem[wit={ceteri},alt={vṛddhir}]{\skm{r-vṛ}ddhi\skp{r-bha}}
  \rdg[wit={D,E,L}]{\om}
}\app{\lem[wit={ceteri},alt={bhavati}]{\skm{r-bha}vati}
  \rdg[wit={B}]{bhavatī}
  \rdg[wit={E,L}]{vardhate}
  \rdg[wit={D}]{\om}}\dd{}\vspace*{\fill}
\end{prose}
  \end{edition}
  \begin{translation}
  \ekddiv{type=trans}
       \bigskip
       \centerline{[\uproman{21}.\textsuperscript{\coro{\lowroman{1}-\lowroman{12}}}\textrm{\small{Haṭhayoga]}}}
       \label{hathayogatrans}
      \bigskip
      \begin{tlate}
        \noindent
        Now Haṭhayoga is explained. The practice of breath shall be done in this manner: "Exhalation, Inhalation [and] Retention etc. And then due to the six practices (\textit{ṣaṭkarma}), like \textit{dhauti} etc., the purification of the body arises. When the full breath abides in the middle of the sun channel\footnote{Usually the \textit{sūryanāḍi} is the \textit{piṅgalā}-channel or right nostril, as previously declared in \uproman{3}.\lowroman{7}.In the light of the context it appears more likely that \textit{sūryanaḍī} must refer to the central channel, the \textit{suṣūṃṇā}.}, then the mind is unmovable. The form of bliss immediately shines through the motionless mind.\textsuperscript{\coro{[\lowroman{5}]}} Due to the execution of Haṭhayoga, the mind becomes absorbed into emptiness. The time of death does not approach.\\
        \indent Now, the second division of Haṭhayoga is explained.\footnote{In \citetitle{yogakarnika}\textsuperscript{ccn \cdot YSV} 12.23 Ed. p. 107 the section which includes the verses used for the second division of Rāmacandra's Haṭhayoga is introduced by a verse which is missing in the quotes of \citetitle{ramatosana}: (susthāsanaṃ samāsīno nīrajāyatalocanaḥ | cintayet paramātmānaṃ yo vadet sa bhaviṣyati || 23 ||) ``Well-seated in a stable position, with eyes extended like lotus petals, he who contemplates the supreme self, he will become that whom he proclaims.'' This verse again confirms that Rāmacandra used a version of the \textit{Yogasvarodaya} which is very close or similar to the one we find in the \citetitle{ramatosana}, since it is not resembled in Rāmacandra's prose.} The shine of ten million suns in one's body from the feet to the top of the head is contemplated in any colour equal to white, yellow [or] red. Due to the execution of meditation, disease does not arise in the entire body.\textsuperscript{\coro{[\lowroman{10}]}}  Fever does not arise. Vitality grows.\footnote{It is interesting to compare this passage with PT\textsuperscript{qcr \cdot Ysv} (Ed. p. 835) as presented in \textbf{sources} for \uproman{21}.\lowroman{9}-\lowroman{13}, l.5 p.\pageref{hathayoga}: ''Now, listen to the second variation of Haṭhayoga. Contemplate the space at the tip of the nose as if it were equal to the radiance of ten million suns in colours such as white, red, yellow, black, and other colours of that nature. By meditating in this way, one can achieve a long life because one is freed from the process of ageing (aṅgajaraṇavarjitaḥ ] conj. aṅgājananavarjitaṃ). Through the devoted practice of Haṭhayoga, one whose self is great becomes like Śiva. Having become like the light, one truly becomes one with Shiva inside. Therefore, the path of Haṭhayoga will bring forth supernatural abilities and is followed by the Siddhas.'' Rāmacandras transfer obviously misses various details. We can speculate if the omission of details was intentional or simply the result of sloppiness. The original second type appears like Lakṣyayoga, which includes the visualisation of intense light, whereas Rāmacandra's technique involves visualising the whole body in colourful lights. Rāmacandra also decided to leave out the sectarian details. A superficially related but more complex light-based technique of Haṭhayoga, which is classified as a technique of \textit{dhyāna} involves visualising equally intense light at the navel, heart and head and results in igniting this light in all six \textit{cakra}s and ultimately leading to liberation from the fetters of birth (\textit{mucyante janmabandhanāt}) can be found in \citetitle{liersch2023} 33-50. Another interesting similarity appears in \ldots}
        \end{tlate}
      \end{translation}
    \end{alignment}
          \ekdpb*{}
%%%%%%%%%%%%%%%%%%%%%%%%%%%%%%%%%%%%%%%%%%
%%%%%%%%%%%%%%%%%%%%%%%%%%%%%%%%%%%%%%%%%%
%%%%%%%%PAGEBREAK%%%%%%%PAGEBREAK%%%%%%%%%
%%%%%%%%%%%%%%%%%%%%%%%%%%%%%%%%%%%%%%%%%%
%%%%%%%%%%%%%%%%PAGEBREAK%%%%%%%%%%%%%%%%%
%%%%%%%%%%%%%%%%%%%%%%%%%%%%%%%%%%%%%%%%%%
%%%%%%%%PAGEBREAK%%%%%%%PAGEBREAK%%%%%%%%%
%%%%%%%%%%%%%%%%%%%%%%%%%%%%%%%%%%%%%%%%%%
%%%%%%%%%%%%%%%%%%%%%%%%%%%%%%%%%%%%%%%%%%
%%%%%%%%%%%%%%%%%%%%%%%%%%%%%%%%%%%%%%%%%%
%%%%%%%%%%%%%%%%%%%%%%%%%%%%%%%%%%%%%%%%%%
%%%%%%%%PAGEBREAK%%%%%%%PAGEBREAK%%%%%%%%%
%%%%%%%%%%%%%%%%%%%%%%%%%%%%%%%%%%%%%%%%%%
%%%%%%%%%%%%%%%%PAGEBREAK%%%%%%%%%%%%%%%%%
%%%%%%%%%%%%%%%%%%%%%%%%%%%%%%%%%%%%%%%%%%
%%%%%%%%PAGEBREAK%%%%%%%PAGEBREAK%%%%%%%%%
%%%%%%%%%%%%%%%%%%%%%%%%%%%%%%%%%%%%%%%%%%
%%%%%%%%%%%%%%%%%%%%%%%%%%%%%%%%%%%%%%%%%%
%%%%%%%%%%%%%%%%%%%%%%%%%%%%%%%%%%%%%%%%%%
%%%%%%%%%%%%%%%%%%%%%%%%%%%%%%%%%%%%%%%%%%
%%%%%%%%PAGEBREAK%%%%%%%PAGEBREAK%%%%%%%%%
%%%%%%%%%%%%%%%%%%%%%%%%%%%%%%%%%%%%%%%%%%
%%%%%%%%%%%%%%%%PAGEBREAK%%%%%%%%%%%%%%%%%
%%%%%%%%%%%%%%%%%%%%%%%%%%%%%%%%%%%%%%%%%%
%%%%%%%%PAGEBREAK%%%%%%%PAGEBREAK%%%%%%%%%
%%%%%%%%%%%%%%%%%%%%%%%%%%%%%%%%%%%%%%%%%%
%%%%%%%%%%%%%%%%%%%%%%%%%%%%%%%%%%%%%%%%%%
\begin{alignment}[
  texts=edition[class="edition"];
  translation[class="translation"],
  ]
  \begin{edition}
    \ekddiv{type=ed}
    \centerline{[\uproman{22}.\textsuperscript{\coro{\lowroman{1}-\lowroman{14}}}\textrm{\small{Jñānayoga]}}}
    \label{jnanayoga}
          \bigskip
          \begin{prose}
%------------------------------
%idānīṃ jñānayogasya lakṣaṇaṃ kathyate/ \E
%idānīṃ jñānayogasya lakṣaṇaṃ kathyate \P
%idānīṃ jñānayogasya lakṣaṇaṃ// \L 5976_0011.jpg 
%idānīṃ jñānayogasya lakṣaṇaṃ// \B
%idānīṃ jñānayogasya lakṣaṇaṃ// \N1 %%%%p.6 verso 
%idānīṃ jñānayogasya lakṣaṇaṃ// \D
%idānīṃ jñānayogasya lakṣaṇaṃ kathyate// \N2
%idānī  jñānayogasya lakṣaṇaṃ kathyate   \U1
%idānīṃ jñānayogasya lakṣaṇaṃ kathyate// \U2
%------------------------------
%Now the characteristic of jñānayoga is explained. 
%-----------------------------
\note[type=source, labelb=133, lem={\textbf{Re}}]{PT\textsuperscript{qcr \cdot YSV} (Ed. p. 835): idānīṃ jñānayogasya lakṣaṇaṃ kathyate śive | yaj jñātvā jñānasampūrṇaḥ śivaḥ syān na punarbhavaḥ |}
\app{\lem[wit={ceteri}]{idānīṃ}
  \rdg[wit={U1}]{idānī}}
jñānayogasya lakṣaṇaṃ
\app{\lem[wit={E,P,N2,U1,U2}]{kathyate}
  \rdg[wit={B,D,L,N1}]{\om}}/
\end{prose}
%--------------------------------------
%ekam eva jagat paśyed viśvāvasu vibhāsvaram/
%avikalpatayā yuktyā jñānayogaṃ samācaret//1// \E
%
%ekam eva cayat paśyed viśvātmāsu vibhāsvaram       
%avikalpatayā yuktyā jñānayogaṃ samācaret 1 \P
%
%ekam evā jagat paśyed viśvātmāsu vibhāsvaraṃ//
%avikalpatayā yuktā jñānayogaṃ samācaret// \L
%
%ekam evā jagat paśyad visvātmāsu vibhāsvaraṃ//
%avikalpatayā yuktā jñānayogaṃ samācaret// \B
%
%ekam eva jagat paśyed viśvātmā viśvabhāvanaḥ/
%iti kṛtvā tu vai yukto jñānayogaṃ samācaret// SVARODAYA
%
%ekam eva jagat paśyed dviśvātmāsu vibhāsvaraṃ/
%avikalpatayā yuktyā jñānayogaṃ samācaret//1// \N1
%
%ekam eva jagat paśyed dviśvātmāsu vibhāsvaraṃ//
%avikalpatayā yuktyā jñānayogaṃ samācaret//1// \D
%
%ekam eva jagat paśyed dviśvātmāsu vibhāsvaraṃ//
%avikalpatayā yuktyā jñānayogaṃ samācaret//1// \N2
%
%ekam eva jagataḥ paśyed dviśvātmāsu vibhāsvaraṃ
%āvikalpatayā yuktyā jñānayogaṃ samācaret//1// \U1
%
%ekam eva jagataḥ paśyed dviśvātmāsu vibhāsvaraṃ
%āvikalpatayā yuktyā jñānayogaṃ samācaret// \U2
%------------------------------
%He shall see the world truly as being one, shining in all selves. 
%By applying indistinctness he shall accomplish \textit{jñānayoga}.   
%------------------------------
\begin{tlg}
  \tl{\note[type=source, labelb=134, lem={\textbf{Re}}]{PT\textsuperscript{qcr \cdot YSV} (Ed. p. 835): ekam eva jagat paśyed viśvātmā viśvabhāvanaḥ | iti kṛtvā tu vai yukto jñānayogaṃ samācaret ||}
eka\skp{m-e}\app{\lem[wit={ceteri}, alt={eva}]{\skm{m-e}va}
  \rdg[wit={B,L}]{evā}}
\app{\lem[wit={ceteri},alt={jagat}]{jaga\skp{t-pa}}
  \rdg[wit={P}]{cayat}
}\app{\lem[wit={ceteri},alt={paśyed}]{\skm{t-pa}śye\skp{d-vi}}
  \rdg[wit={B}]{paśyad}
}\app{\lem[wit={B,L,P},alt={viśvātmāsu}]{\skm{d-vi}śvātmāsu}
  \rdg[wit={E}]{viśvāvasu}
  \rdg[wit={X,U2}]{dviśvātmāsu}}
vibhāsvaraṃ/}\\
\tl{\app{\lem[wit={ceteri}]{avikalpatayā}
  \rdg[wit={U1,U2}]{āvikalpatayā}}
\app{\lem[wit={ceteri}]{yuktyā}
  \rdg[wit={B,L}]{yuktā}} 
jñānayogaṃ samācaret\dd{}\begin{otherlanguage}{english}\coro{\uproman{22}.\textsuperscript{\lowroman{2}}}\end{otherlanguage}\hskip-2pt\dd{}}
\end{tlg}
%------------------------------
%yatra yatra sthito vāpi sarvajñānamayaṃ jagat/ 
%sa evaṃ vetti bodhena so pi jñānādhikāraṇāt//2// \E 
%
%yatra yatra sthito vāpi sarvajñānamayaṃ jagat  
%ya evaṃ vetti bodhena so pi jñānādhikāravān \P
%
%yatra yatra sthito vāpi sarvajñānamayaṃ jagat//  
%ya evaṃ vetti bodhena so pi jñānādhikāravān// \L
%
%yatra yatra sthito vāpi sarvajñānamayaṃ jagat//  
%ya evaṃ ve bodhena so pi jñānādhikāravān// \B
%
%yatra tatra sthito vāpi sarvajñānamayaṃ jagat/
%ya evam asti bodhena so'pi jñānādhikāravān/ \SVARODAYA
%
%yatra yatra sthito vāpi sarvajñānamayaṃ jagat/
%ya evaṃ vetti bodhena so pi jñānādhikāravān//2//\N1
%
%yatra yatra sthito vāpi sarvajñānamayaṃ jagat//
%ya evaṃ vetti bodhena so pi jñānādhikāravān//2//\D
%
%yatra yatra sthito vāpi sarvajñānamayaṃ jagat//
%ya evaṃ vetti bodhena so pi jñānādhikāravān//2//\N2
%
%yatra yatra sthito vāpi sarvajñānamayaṃ jagat  %%%273.jpg
%evaṃ vette na bodhena so pi jñānādhikāravān 2    \U1
%
%yatra yatra sthito hiṃsa sarvajñānamayaṃ jagat//  
%evaṃ vetti bodhena so pi jñānādhikāravān// 2    \U2
%------------------------------
%Wherever the world is established or made of omniscience,
%who knows thus by means of insight, he is a like an expert of knowledge.      
%------------------------------
\begin{tlg}
  \tl{\note[type=testium, labelb=135, lem={\textbf{Re}}]{PT\textsuperscript{qcr \cdot YSV} (Ed. p. 835): yatra tatra sthito vāpi sarvajñānamayaṃ jagat | ya evam asti bodhena so'pi jñānādhikāravān ||}
    yatra tatra sthito \app{\lem[wit={ceteri}]{vāpi}
      \rdg[wit={U2}]{hiṃsa°}} sarvajñānamayaṃ jagat/}\\
  \tl{\app{\lem[wit={ceteri}]{ya evaṃ}
      \rdg[wit={U1,U2}]{evaṃ}}
    \app{\lem[wit={ceteri}]{vetti}
      \rdg[wit={U1}]{vette na}
      \rdg[wit={B}]{ve}} bodhena so pi
    \app{\lem[wit={ceteri}]{jñānādhikāravān}
      \rdg[wit={E}]{jñānādhikāraṇāt}}\dd{}\begin{otherlanguage}{english}\coro{\uproman{22}.\textsuperscript{\lowroman{3}}}\end{otherlanguage}\hskip-2pt\dd{}}
\end{tlg}
%------------------------------
%
%\om!!!!!                                                                                                        \E
%
%prāpnoti śāmbhavīmantrān  sadā nityaparāyaṇaḥ/   yathā nyagrodhavījaṃ hi kṣitau   vaptur drumāyate/               \SVARODAYA  
%prāpnoti śāmbhavīṃ sattāṃ sadāṃdvaitaparāyaṇaḥ   yathā nyagrodhabījaṃ hi kṣitāv   uptaṃ drumāyate likāṃ pa..vāḥ 4 \P  7640.jpg last line check word!!!
%prāpnoti śāmbhavīṃ sattān sadādvaitaparāyaṇaḥ//  yathā nyagrodhavīja  hi kṣitāv   utpadyate yathā//               \L
%prāpnoti śāmbhaviṃ sattāṃ sadādvaitaparāyaṇaḥ//  yathā nyagrodhabījāṃ hi kṣitī    utpadyate//                      \B
%prāpnoti sāṃbhavīṃ satta  sadādvaitaparāyaṇaḥ//  yathā nyagrodhavījaṃ hi kṣitāv   uptaṃ drumāyate 3//              \N1
%prāpnoti sāṃbhavīsattāṃ   sadādvaitaparāyaṇaḥ//  yathā nyagrodhavījaṃ hi kṣitāv   uptaṃ drumāyate//                \D
%prāpnoti sāṃbhavīsattā    sadādvaitaparāyaṇaḥ//  yathā nyagrodhavījaṃ hi kṣitāv   uptaṃ drumāyate//                \N2 %drumaayate=denom. wie ein beim  sein 
%prāpnoti sāṃbhavīsattāṃ   sadādvaitaparāyaṇaḥ    yathā nyagrodhabījaṃ hi kṣitāptā ukta drumāyate 3              \U1
%prāpnoti sāṃbhavīsattāṃ   yadādvaitaparāyaṇaḥ//  yathā nyagrodhabījaṃ hi kṣitāv   uptaṃ drumāyate//               \U2
%------------------------------
%He always attains the reality of śāmbhavī - the supreme goal of non-duality.  
%Just as the seed of the Nyagrodha scattered onto the soil [always] becomes a tree.
%------------------------------
\begin{tlg}
  \tl{\note[type=source, labelb=136, lem={\textbf{Re}}]{PT\textsuperscript{qcr \cdot YSV} (Ed. p. 835): prāpnoti śāmbhavīmantrān sadā nityaparāyaṇaḥ | yathā nyagrodhavījaṃ hi kṣitau vaptur drumāyate ||}
    \app{\lem[wit={ceteri}]{prāpnoti}
      \rdg[wit={E}]{\om}}
    \app{\lem[wit={D,U1,U2}]{sāṃbhavīsattāṃ}
      \rdg[wit={B,P}]{śāmbhavīṃ sattāṃ}
      \rdg[wit={L}]{śāmbhavīṃ sattān}
      \rdg[wit={N1}]{sāṃbhavīṃ satta}
      \rdg[wit={N2}]{sāṃbhavīsattā}
      \rdg[wit={E}]{\om}}
    \app{\lem[wit={ceteri},alt={sadādvaita°}]{sadādvaita}
      \rdg[wit={U1}]{sadāṃdvaita°}
      \rdg[wit={E}]{\om}}parāyaṇaḥ/}\\
  \tl{\app{\lem[wit={ceteri}]{yathā}
      \rdg[wit={E}]{\om}}
    \app{\lem[wit={ceteri}]{nyagrodhabījaṃ}
      \rdg[wit={D,N1,N2}]{nyagrodhavījaṃ}
      \rdg[wit={L}]{nyagrodhavīja}
      \rdg[wit={E}]{\om}}
    \app{\lem[wit={ceteri}]{hi}
      \rdg[wit={E}]{\om}}
    \app{\lem[wit={ceteri},alt={kṣitāv}]{kṣitā\skp{v-u}}
      \rdg[wit={B}]{kṣitī}
      \rdg[wit={U1}]{kṣitāptā}
      \rdg[wit={E}]{\om}
 }\app{\lem[wit={ceteri},alt={uptaṃ drumāyate}]{\skm{v-u}ptaṃ drumāyate}
      \rdg[wit={P}]{uptaṃ drumāyate likāṃ pa..vāḥ}
      \rdg[wit={L}]{utpadyate yathā}
      \rdg[wit={B}]{utpadyate}
      \rdg[wit={U1}]{ukta drumāyate}
      \rdg[wit={E}]{\om}}\dd{}\begin{otherlanguage}{english}\coro{\uproman{22}.\textsuperscript{\lowroman{4}}}\end{otherlanguage}\hskip-2pt\dd{}}
\end{tlg}
%------------------------------
%ekāntaṃ  naikadā  svena   dṛśyate  daśadhā  kṛtaḥ/  mūlāṅkurasya  coddaṇḍāḥ śākhākuṇḍalapallavāḥ//3//   \E cod?v%on cud? Wurzel in guṇa + daṇḍa? !!! em. zu śaśvadhā = immer wieder, jederzeit 
%\om                                                                                                    \P
%ekāṃte   nekadhā  svena   dṛśyaṃte daśadhāt kṛp?tā/ mūlāṃkurutva kudaṃḍaḥ  śākhākilekālapallavā        \B
%ekāṃte   nekadhā  svena   dṛśyaṃte daśadhāt kṛtaḥ/  mūlāṃkurutva kudaṃḍa   śākhākalikālapallavā        \L
%ekāṃtaṃ  naikadhā śveta   dṛśyate  daśadhā  kṛtā//  mūlāṃkurutva codaṃḍaḥ  śāvārakumbhalapallavaḥ//4// \N1   
%ekāṃtaṃ  naikadhā śvetana dṛśyate  daśadhā  kṛtā//  mūlāṃkurutva codarāṭaḥ śālavākumapadṛtravā//4//    \D
%ekāṃtaṃ  naikadhā śvetana dṛśyet   śadhā    kṛtā//  mūlāṃkurutva codarāṭaḥ śākhākumbhalapallavā//4//   \N2
%yekāṃtaṃ naikadhā svena   dṛśyate  śadhā    kṛtā    mūlāṃkurutva codaṃḍa   śākhākumbhalapallavaḥ       \U1
%ekāṃtaṃ  naikadhā svetana dṛśyate  daśadhā  kṛtiḥ// mūlāṃkurutva codaṃḍaḥ  śākhākusumapallavāḥ//       \U2
%------------------------------
\begin{tlg}
  \tl{\note[type=source, labelb=137, lem={\textbf{Re}}]{PT\textsuperscript{qcr \cdot YSV} (Ed. p. 835): ādāv ekas tato 'nekaḥ svabhāvāc chādanādibhiḥ | varddhate 'harniśaṃ vṛkṣaḥ patrapallavavistṛtaḥ|}
    \note[type=philcomm, labelb=137a, lem={\uproman{22}.\textsuperscript{\lowroman{5}}}]{The verse is omitted in P.}
 \app{\lem[wit={ceteri}]{ekāntaṃ}
  \rdg[wit={B,L}]{ekānte}
  \rdg[wit={U1}]{yekāṃtaṃ}}
\app{\lem[wit={ceteri}]{naikadhā}
  \rdg[wit={E}]{naikadā}
  \rdg[wit={B,L}]{nekadhā}}
\app{\lem[wit={ceteri}]{svena}
  \rdg[wit={N1}]{śveta}
  \rdg[wit={D,N2}]{śvetana}}
\app{\lem[wit={ceteri}]{dṛśyate}
  \rdg[wit={B,L}]{dṛśyaṃte}
  \rdg[wit={N2}]{dṛśyet}}
\app{\lem[wit={E,N1,N2}]{daśadhā}    %%[type=conjecture, resp=egoscr]{śaśvadhā}????
  \rdg[wit={B,L}]{daśadhāt}
  \rdg[wit={N2,U1}]{śadhā}}
\app{\lem[type=emendation, resp=egoscr]{kṛtaṃ}
  \rdg[wit={E,L}]{kṛtaḥ}
  \rdg[wit={X}]{kṛtā}
  \rdg[wit={B}]{kṛptā}
  \rdg[wit={U2}]{kṛtiḥ}}/}\\
 \tl{\app{\lem[wit={E}]{mūlāṅkurasya}
  \rdg[wit={ceteri}]{mūlāṃkurutva}}
\app{\lem[wit={E,N1,U2}]{coddaṇḍāḥ}
  \rdg[wit={D,N2}]{codarāṭaḥ}
  \rdg[wit={B}]{kudaṃjaḥ}
  \rdg[wit={L}]{kudaṃḍa}}
\app{\lem[wit={U2}]{śākhākusumapallavāḥ}
  \rdg[wit={E}]{śākhākuṇḍalapallavāḥ}
  \rdg[wit={B,L}]{śākhākilekālapallavā}
  \rdg[wit={N1,U1}]{śāvārakumbhalapallavaḥ}
  \rdg[wit={N2}]{śākhākumbhalapallavā}
  \rdg[wit={D}]{śālavākumapadṛtravā}}\dd{}\begin{otherlanguage}{english}\coro{\uproman{22}.\textsuperscript{\lowroman{5}}}\end{otherlanguage}\hskip-2pt\dd{}}
\end{tlg}%\vspace*{\fill}
\ekdpb*{}
  \end{edition}
  \begin{translation}
  \ekddiv{type=trans}
  \centerline{[\uproman{22}.\textsuperscript{\coro{\lowroman{1}-\lowroman{14}}}\textrm{\small{Jñānayoga]}}}
    \label{jnanayogatrans}
     \bigskip
     \begin{tlate}
       \noindent
      \blfootnote{\ldots in \citetitle{birch2013} 2.7-8. These verses precede or introduce \textit{śāmbhavī mudrā}. Here, tought, intellect and ego are taught the be the officiants, whereas the mind is the sacrificer who sacrifices the senses and the ten vitalbreaths into the orb of light (2.7). The orb of light (\textit{jyotimaṇḍala}) shines from the root (possibly the root of the body or spine, but \citeauthor[2013:286]{birch2013} suggests the palate) to the aperture at the top of the head. Practitioners of yoga should constantly meditate on it to achieve \textit{siddhi}s (2.8). (cittaṃ buddhir ahaṅkāra ṛtvijaḥ somapaṃ manaḥ | indriyāṇi daśa prāṇāñ juhoti jyotimaṇḍale || 7 || ā mūlād bilaparyantaṃ vibhāti jyotimaṇḍalam | yogibhiḥ satataṃ dhyeyam aṇimādyaṣṭasiddhidam || 8 ||)} Now the characteristic of Jñānayoga is explained.      
 \paragraph{\uproman{22}\textsuperscript{\lowroman{2}}.} He shall see the world as one, shining in all selves [of the world]. By the method of non-dualistic thinking he shall accomplish \textit{Jñānayoga}.
 \paragraph{\uproman{22}\textsuperscript{\lowroman{3}}.} Wherever one sees the all-knowing consciousness pervading the world, and who understands this with wisdom, possesses knowledge superior to that of mere learning.
 \paragraph{\uproman{22}\textsuperscript{\lowroman{4}}.} By always being totally directed towards non-duality, one attains the being of Śiva (\textit{śāmbhavī})\footnote{In medieval Yogatexts, particular in the Rājayoga genre, the term \textit{śāmbhavī} most often appears in the context of a \textit{mudrā}, the so-called \textit{śāṃbhavī mudrā}. The two earliest references for \textit{śāṃbhavī mudrā} are \textit{Candrāvalokana} 1 = \textit{Amanaska} 2.10, who share the same verse. The practice of \textit{śāṃbhavī mudrā} involves focusing the mind internally at an internal orb of light \citetitle{birch2013} 2.7-8. At the same time, the gaze is directed outwards and free from closing and opening the eyes citetitle{birch2013} (2.10). In \citetitle{birch2013} 2.14, the practice is said to bring about \textit{siddhi}s and the no-mind-state (\textit{unmani}) and according to \citetitle{birch2013} 2.14 liberation while alive (\textit{jiivanmukti}). For a detailed discussion of \textit{śāṃbhavī mudrā}, its influence and all references, see \citeauthor[2013:71-79]{birch2013}. Here, Rāmacandra uses the term \textit{śāmbhavī} as a designation of the ultimate state to be attained by practicing Rājayoga, which he presents as the realization of absolute unity. The \textit{śāmbhavī}-state certainly has roots in the ancient tantric traditions of Śaivism. See references entry \textit{Tantriābhadhānakośa} Vol 3?}, just as the seed of the banyan tree\footnote{In rituals the banyan tree (\textit{nygarodha}) is assicoated with the \textit{kṣatriya} class (\citeauthor[1998:27]{smith1998}).} scattered in the ground becomes a tree.
 \paragraph{\uproman{22}\textsuperscript{\lowroman{5}}.} The absolute unity (\textit{ekāntaṃ}) is perceived not uniform by the self [as it is] made of ten parts. [Just as] the stems, branches, buds and leaves of the original shoot.
  \vfill
\end{tlate}
\end{translation}
\end{alignment}
%%%%%%%%%%%%%%%%%%%%%%%%%%%%%%%%%%%%%%%%%% 
%%%%%%%%%%%%%%%%%%%%%%%%%%%%%%%%%%%%%%%%%% 
%%%%%%%%PAGEBREAK%%%%%%%PAGEBREAK%%%%%%%%%
%%%%%%%%%%%%%%%%%%%%%%%%%%%%%%%%%%%%%%%%%% 
%%%%%%%%%%%%%%%%PAGEBREAK%%%%%%%%%%%%%%%%%
%%%%%%%%%%%%%%%%%%%%%%%%%%%%%%%%%%%%%%%%%% 
%%%%%%%%PAGEBREAK%%%%%%%PAGEBREAK%%%%%%%%%
%%%%%%%%%%%%%%%%%%%%%%%%%%%%%%%%%%%%%%%%%% 
%%%%%%%%%%%%%%%%%%%%%%%%%%%%%%%%%%%%%%%%%% 
%%%%%%%%%%%%%%%%%%%%%%%%%%%%%%%%%%%%%%%%%% 
%%%%%%%%%%%%%%%%%%%%%%%%%%%%%%%%%%%%%%%%%% 
%%%%%%%%PAGEBREAK%%%%%%%PAGEBREAK%%%%%%%%%
%%%%%%%%%%%%%%%%%%%%%%%%%%%%%%%%%%%%%%%%%% 
%%%%%%%%%%%%%%%%PAGEBREAK%%%%%%%%%%%%%%%%%
%%%%%%%%%%%%%%%%%%%%%%%%%%%%%%%%%%%%%%%%%% 
%%%%%%%%PAGEBREAK%%%%%%%PAGEBREAK%%%%%%%%%
%%%%%%%%%%%%%%%%%%%%%%%%%%%%%%%%%%%%%%%%%% 
%%%%%%%%%%%%%%%%%%%%%%%%%%%%%%%%%%%%%%%%%% 
%%%%%%%%%%%%%%%%%%%%%%%%%%%%%%%%%%%%%%%%%% 
%%%%%%%%%%%%%%%%%%%%%%%%%%%%%%%%%%%%%%%%%% 
%%%%%%%%PAGEBREAK%%%%%%%PAGEBREAK%%%%%%%%%
%%%%%%%%%%%%%%%%%%%%%%%%%%%%%%%%%%%%%%%%%% 
%%%%%%%%%%%%%%%%PAGEBREAK%%%%%%%%%%%%%%%%%
%%%%%%%%%%%%%%%%%%%%%%%%%%%%%%%%%%%%%%%%%% 
%%%%%%%%PAGEBREAK%%%%%%%PAGEBREAK%%%%%%%%%
%%%%%%%%%%%%%%%%%%%%%%%%%%%%%%%%%%%%%%%%%% 
%%%%%%%%%%%%%%%%%%%%%%%%%%%%%%%%%%%%%%%%%% 
\begin{alignment}[
  texts=edition[class="edition"];
  translation[class="translation"],
  ]
  \begin{edition}
    \ekddiv{type=ed}
%------------------------------
%srehapuṇyaphalaṃ   bīje vistaro yaṃ svabhāvataḥ/  tathāsau   nirmalo  nityo nirvikāro niraṃjanaḥ//4// \E
%snehapuṣpaphalaṃ   bīje vistāro yaṃ svabhāvataḥ   tāthāpasau nirmalau nityo nirvikāro niraṃjanaḥ     \P   %%7641.jpg Z.1
%snehe puṣpaphala---bīja-vistāro ya  svabhāvatāḥ   yāthāsau   nirmalo  nityo nirvikāro niraṃjanaḥ//    \B
%snehe puṣpaphala---bīja-vistāro ya  svabhāvatāḥ// tāthāsau   nirmalo  nityo nirvikāro niraṃjanaḥ//    \L
%snehapuṣpaphalaṃ   bīje vistārā yaṃ svabhāvataḥ/  tathāsau   nirmalo  nityo nirvikāro niraṃjanaḥ//5// \N1
%snehapuṣpaphalaṃ   bīje vistārā yasya  bhāvataḥ// tathāsau   nirmalo  nityo nirvikāro niraṃjanaḥ//5// \D
%snehapuṣpaphalaṃ   vīje vistāro yaṃ svabhāvataḥ// tathāsau   nirmalo  nityo nirvikāro niraṃjanaḥ//5// \N2
%snehapuṣpaṃ phalaṃ bīje vistāro yaḥ svabhāvataḥ   tathāsau   nirmalo  nityo nirvikāro niraṃjanaḥ 5 \U1  %%%%274.jpg
%snehapuṣpaphalaṃ   bīje vistāro yaṃ svabhāvataḥ// tathāsau   nirmalo  nityo nirvikāro niraṃjanaḥ// 5  \U2 %%%first Śloka in this series that is numbered in U2 
%------------------------------
%Aufgrund seines inhärenten Wesens ist dieser Ast mit seinen Zweigen, welcher die Frucht der Blüte der Liebe ist, im Samen.
%Gewiss, ist jenes rein, ewig, unveränderlich und makellos. 
%------------------------------
%By virtue of its inherent nature, this branch with its branches, which is the fruit of the flower of love, is in the seed.
%Certainly, that is pure, eternal, unchanging and immaculate.
%------------------------------
\begin{tlg}\tl{\note[type=source, labelb=138, lem={\textbf{Re}}]{PT\textsuperscript{qcr \cdot YSV} (Ed. p. 836): snehapuṣpaphalair vījair vistāro 'yaṃ svabhāvataḥ | tathāsau nirmalo nityo nirvikāro nirañjanaḥ |}
    \app{\lem[wit={D,N1,N2,P,U2}]{snehapuṣpaphalaṃ}
  \rdg[wit={B,L}]{snehe puṣpaphala°}
  \rdg[wit={U1}]{snehapuṣpaṃ phala}
  \rdg[wit={E}]{srehapuṇyaphalaṃ}}
\app{\lem[wit={ceteri}]{bīje}
  \rdg[wit={B,L}]{bīja}}
\app{\lem[wit={ceteri}]{vistāro}
  \rdg[wit={D,N1}]{vistārā}}
\app{\lem[wit={E,P,N1,N2,U2}]{'yaṃ}
  \rdg[wit={B,L}]{ya}
  \rdg[wit={U1}]{yaḥ}
  \rdg[wit={D}]{yasya}}
\app{\lem[wit={ceteri}]{svabhāvataḥ}
  \rdg[wit={B,L}]{svabhāvatāḥ}
  \rdg[wit={D}]{bhāvataḥ}}/}\\
\tl{\app{\lem[wit={ceteri}]{tathāsau}
    \rdg[wit={B}]{yathāsau}
    \rdg[wit={P}]{tathāpasau}}
  \app{\lem[wit={ceteri}]{nirmalo}
    \rdg[wit={P}]{nirmalau}}
nityo nirvikāro niraṃjanaḥ\dd{}\begin{otherlanguage}{english}\coro{\uproman{22}.\textsuperscript{\lowroman{6}}}\end{otherlanguage}\hskip-2pt\dd{}}
\end{tlg}
%------------------------------
%eko  nekaḥ  svayaṃbhūś ca dhāmnā ca    bahudhā sthitaḥ/   paṃcatattvamanobuddhi-māyāhaṃkāravikriyāḥ //5//   \E
%eko  nekaḥ  svayaṃbhūś ca svadhāmnā    bahudhā sthitāḥ    paṃcatatvamanobuddhir māyāhaṃkāravikriyāḥ   6     \P
%eko  neka   svayaṃbhūś ca dhāmnāya     bahudhā sthitaḥ//  paṃcatatvamanobuddhi--māyāhaṃkāravikriyā  //      \B
%eko  nekaḥ  svayaṃbhūś ca svadhābhāva  bahudhā sthitāḥ//  paṃcatatvamanobuddhi--māyāhaṃkāravikriyā  //      \L
%eko  nekaḥ  svayaṃbhuś ca svayāṃmnā    bahudhā sthitaḥ/   paṃcatatvamanobuddhir māyāhaṃkāravikriyā  //6//   \N1
%eko  nekaḥ  svayaṃbhaś ca svadhā...ṣ   bahudhā sthitāḥ//  paṃcatatvamanobuddhir māyāhaṃkāravikriyā  //6//   \D
%eko  neka   svayaṃbhūś ca svadhāmnāva  bahudhā sthitaḥ//  paṃcatatvamanobuddhir māyāhaṃkāravikriyā  //6//   \N2
%yeko naika/ svayaṃbhūtyā  svabhāvā     bahudhā sthitaḥ    paṃcatatvamanobuddhir māyāhaṃkāravikriyāḥ   6     \U1
%eko  naiko  svayaṃbhūś ca svadhāmnā    bahudhā sthitaḥ//  paṃcatatvamanobuddhir māyāhaṃkāravikriyā  //6//   \U2
%------------------------------
%One, not one, self-existing by it's own power, abiding in multiplicity, as five [gross] elements (\textit{tattva}), thinking mind (\textit{manas}), intellect (\textit{buddhi}), illusion (\textit{māya}), individuation (\textit{ahaṃkāra}), and modifications (\textit{vikriyā}). 
%------------------------------
\begin{tlg}
\note[type=source, labelb=139, lem={\textbf{Re}}]{PT\textsuperscript{qcr \cdot YSV} (Ed. p. 836): eko 'nekaḥ khayaṃ bhūyān sādhanād bahudhā sthitaḥ | pañcatattvamayo buddhimāyāhaṅkāravikriyaḥ |}
  \tl{
\app{\lem[wit={ceteri}]{eko}
  \rdg[wit={U1}]{yeko}}
\app{\lem[type=emendation, resp=egoscr]{naikaḥ}
   \rdg[wit={ceteri}]{nekaḥ}
   \rdg[wit={U1}]{naika}
  \rdg[wit={U2}]{naiko}
  \rdg[wit={B,N2}]{neka}}
\app{\lem[wit={ceteri},alt={svayaṃbhūś ca}]{svayaṃbhūś\skp{-}ca}
  \rdg[wit={U1}]{svayaṃbhūtyā}}
\app{\lem[wit={P,U2}]{svadhāmnā}
  \rdg[wit={E}]{dhāmnā ca}
  \rdg[wit={B}]{dhāmnāya}
  \rdg[wit={L}]{svadhābhāva}
  \rdg[wit={N1}]{svayāṃmnā}
  \rdg[wit={D}]{svadhā..ṣa}
  \rdg[wit={N2}]{svadhāmnāva}
  \rdg[wit={U1}]{svabhāvā}}
bahudhā
\app{\lem[wit={D,L,P}]{sthitāḥ}
  \rdg[wit={ceteri}]{sthitaḥ}}/}\\
\tl{pañcatattvamano\app{\lem[wit={E,P,L},alt={°buddhi°}]{buddhi}
    \rdg[wit={ceteri}]{°buddhir}
  }māyāhaṃkāra\app{\lem[wit={E,P,U1},alt={°vikriyāḥ}]{vikriyāḥ}
    \rdg[wit={ceteri},alt={°vikriyā}]{vikriyā}}\dd{}\begin{otherlanguage}{english}\coro{\uproman{22}.\textsuperscript{\lowroman{7}}}\end{otherlanguage}\hskip-2pt\dd{}}
\end{tlg}
%------------------------------ 
%evaṃ daśavidhaṃ viśvaṃ lokālokasavistaram/   eka  eva na cānyo sti yo jānāti sa tattvavit//6// \E
%evaṃ daśavidhaṃ viśvaṃ lokālokasavistaraṃ    eka  eva na cānyo sti yo jānāti sa tatvavit 6 \P
%evaṃ daśavidhā  viśvaṃ lokālokasavistaraṃ//  eka  eva na cānyā sti yo jānāti sa tatvavit// \B
%evaṃ daśavidhā  viśvaṃ lokālokasavistaraṃ//  eka  eva na cānyo sti yo jānāti sa tatvavit// \L
%evaṃ daśavidhaṃ viśvaṃ lokālokasavistarāṃ/   eka  eva na cānyo sti yo nānāti sa tatvavit//7// \N1
%evaṃ daśavidhaṃ viśvaṃ lokālokasavistaraṃ//  eka  eva na cānyo sti yo jānāti sa tatvavit//7// \D
%evaṃ daśavidhā  viśvaṃ lokālokasavistaraṃ//  eka  eva na cānyo sti yo jānāti sa tatvavit//7// \N2
%evaṃ daśavidha--viśvaṃ lokālokasavistaraṃ    eka yeva na cānyo sti yo jānāti sa tatvavit 7 \U1
%evaṃ daśavidhaṃ viśvaṃ lokāloke savistaraṃ// ekam eva na cānyo sti yo jānāti sa tatvavit//7// \U2 %%%409.jpg 
%------------------------------
%Auf diese Weise durchdringen die zehn Variationen die Welt und die Nicht-Welt im vollen Umfang.  
%Nur das Eine ist und nicht etwas anderes: Wer das weiß ist ein Kenner der Realität.  
%------------------------------
%In this way, the ten variations fully permeate the world and the non-world.
%Only one thing is and not something else: Whoever knows this is a connoisseur of reality.
%------------------------------
\begin{tlg}
\note[type=source, labelb=140, lem={\textbf{Re}}]{PT\textsuperscript{qcr \cdot YSV} (Ed. p. 836): evaṃ bahuvidhaṃ viśvaṃ lokālokasuvistaram | ekam eva na cānvo 'sti yo jānāti sa tattvavit |}
   \tl{
     evaṃ
     \app{\lem[wit={B,L,N2}]{daśavidhā viśvaṃ}
       \rdg[wit={D,E,P,N1,U2}]{daśavidhaṃ viśvaṃ}
       \rdg[wit={U1}]{daśavidhaviśvaṃ}}
     \app{\lem[wit={ceteri}]{lokālokasavistaram}  
       \rdg[wit={N1}]{lokālokasavistarāṃ}
       \rdg[wit={U2}]{lokāloke savistaraṃ}}/}\\
   \tl{\app{\lem[wit={ceteri}]{eka}
       \rdg[wit={U2}]{ekam}}
       \app{\lem[wit={ceteri}]{eva}
         \rdg[wit={U1}]{yeva}}
       na cānyo 'sti yo jānāti sa tattvavit\dd{}\begin{otherlanguage}{english}\coro{\uproman{22}.\textsuperscript{\lowroman{8}}}\end{otherlanguage}\hskip-2pt\dd{}}
\vspace{5mm} %5mm vertical space 
   \end{tlg}
    \begin{prose}     
%------------------------------
%pṛthvīvanaspatiparvatādisthārarūpaḥ         saṃsāra---manuṣyahastyaśvapakṣītyādiko    jaṃgamarūpaḥ   saṃsāraḥ// \E
%pṛthvīvanaśpatiparvatādisthāvararūpaḥ       saṃsāraḥ  manuṣyahastyaś ca pakṣītyādiko  jaṃgamarūpaḥ   saṃsāraḥ \P
%pṛthvīvanaspatīparvatādisthāvararūpā        saṃsāraḥ/ manuṣyahasteśvapakṣītyādiko     jaṃgamarūpaḥ   saṃsāraḥ// \B
%pṛthvīvanaspatiparvatādisthāvararūpā        saṃsāraḥ  manuṣyahasteśvapakṣītyādiko     jaṃgamarūpā    saṃsāraḥ// \L
%pṛthvīvanaspatīparvvate tyādisthāvararūpaḥ  saṃsāraḥ  manuṣyahastīaśvapakṣītyādiko    jaṃgamarūpaḥ   saṃsāraḥ// \N1
%pṛthvīvanaspatīparvato tyādisthāṃvararūpaḥ  saṃsāraḥ  manuṣyahastīaśvapakṣītyādiko    jaṃgamaḥ rūpaḥ saṃsāraḥ// \D
%pṛthvīvanaspatiparvate 'thyādisthāvararūpa  saṃsāraḥ  manuṣyahastipakṣītyādiko        jaṃgamarūpaḥ   saṃsāraḥ// \N2
%pṛthivīvanaspatīparvate iyādisthāvararūpaḥ  saṃsāra---manuṣyahastiasvapakṣītyādiko    jagadrūpaḥ     saṃsāro \U1
%pṛthvīvanaspatiparvatādisthāvararūpaḥ       saṃsāraḥ//manuṣyahasttyaś ca pakṣītyādiko jaṃgamarūpaḥ   saṃsāraḥ//8// \U2
%------------------------------
%The mundane existance (\textit{saṃsāra}) exists in the form of stationary [existances] such as earth, tree mountains and so on. The mundane existance (\textit{saṃsāra}) [also] exists in the form of the mobile [existances] such as humans, animals, birds and so on.
%------------------------------
\note[type=source, labelb=141, lem={\textbf{Re}}]{PT\textsuperscript{qcr \cdot YSV} (Ed. p. 832): sthāvarāḥ parvatādyā hi jaṅgamāḥ khecarādayaḥ | jaṅgamasthāvarākāraḥ saṃsāraḥ syāt sa īśvaraḥ |}
\app{\lem[wit={ceteri},alt={pṛthvī°}]{pṛthvī}
        \rdg[wit={U1}]{pṛthivī°}
      }\app{\lem[wit={E,N2,U2},alt={°vanaspati°}]{vanaspati}
        \rdg[wit={P}]{vanaśpati}
        \rdg[wit={B,D,L,N1,U1}]{vanaspatī°}
      }\app{\lem[wit={P,B,L,U2}, alt={°parvatādisthāvara°}]{parvatādisthāvara}
        \rdg[wit={E}]{°parvatādisthāra°}
        \rdg[wit={D}]{°parvato tyādisthāṃvara°}
        \rdg[wit={N1}]{°parvvate tyādisthāvara°}
        \rdg[wit={N2}]{°parvate 'thyādisthāvara°}
        \rdg[wit={U1}]{°parvate iyādisthāvara°}
      }\app{\lem[wit={ceteri}]{rūpaḥ}
        \rdg[wit={B,L}]{rūpā}
        \rdg[wit={N2}]{rūpa}}
      \app{\lem[wit={ceteri}]{saṃsāraḥ}
        \rdg[wit={E,U1}]{saṃsāra°}}/
      manuṣya\app{\lem[wit={B,L},alt={°hasteśvapakṣīty ādiko}]{hasteśvapakṣīty\skp{-}ādiko}
          \rdg[wit={E}]{°hasty aśvapakṣīty ādiko}
          \rdg[wit={D,N1}]{°hastīaśvapakṣīty ādiko}
          \rdg[wit={N2}]{°hastipakṣīty ādiko}
          \rdg[wit={U1}]{°hastiasvapakṣīty ādiko}
          \rdg[wit={U2}]{°hasttyaś ca pakṣīty ādiko}}
        \app{\lem[wit={ceteri}]{jaṃgamarūpaḥ}
          \rdg[wit={D}]{jaṃgamaḥ rūpaḥ}
          \rdg[wit={L}]{°rūpā}
          \rdg[wit={U1}]{jagad°}}
        \app{\lem[wit={ceteri}]{saṃsāraḥ}
          \rdg[wit={U1}]{saṃsāro}}/\textsuperscript{\begin{otherlanguage}{english}\coro{[\lowroman{10}]}\end{otherlanguage}}
%------------
%atha ca   yo  dṛṣṭiviṣayaḥ  sa dṛśya  ucyate/  yo dṛṣṭyā na vīkṣyate sa adṛśya ity  ucyate/ \E
%atha ca   yo  dṛṣṭiviṣayaḥ  sa dṛśya  ucyate   yo dṛṣṭyā na vīkṣyate sa adṛśya ity  ucyate  %%%7641.jog
%atha ca// yo  daṣṭiviṣayaḥ  sa dṛśya  ucyate// yo dṛṣṭyā na vīkṣyate sa adṛśya ty   ucyate// \B
%atha ca   yo ddṛṣṭiviṣayaḥ  sa dṛśya  ucyate// yo dṛṣṭyā na vīkṣyate sa adṛśye ty   ucyate... \L
%atha ca   ya ddṛṣṭiviṣayaḥ  sa dṛśyad ucyate   yo dṛṣṭyā na vīkṣyate sa adṛśya ity  ucyate// \N1
%atha vā   ya dārṣṭiviṣayaḥ  sa dṛśya  ucyate/  yo dṛṣṭyā na vīkṣyate sa adṛśya ity  ucyate// \D
%atha ca   ya  drṣṭiviṣayaḥ  sa dṛśya  ucyate/  yo dyā    na vīkṣyate sa adṛśya śaty ucyate/ \N2
%atha ca   yaḥ drṣṭiviṣayaḥ  sa dṛśy---ucyate   yo dṛṣṭvā na vīkṣyate sa adṛśya ity  ucyate \U1
%atha ca   yo  dṛṣṭiviṣayaḥ  sa dṛśya  ucyate// yo dṛṣṭyā na vīkṣyate sa adṛśya ity  ucyate// \U2
%------------------------------
%Now, whatever is the object of perception, it is called visible. Whatever is not seen by perception, it is called invisible. 
%------------------------------
\note[type=source, labelb=142, lem={\textbf{Re}}]{PT\textsuperscript{qcr \cdot YSV} (Ed. p. 836): svabhāvalīlayā bhāti śūnye'sau śūnyabuddhitaḥ | yad dṛṣṭaṃ viṣayaṃ vastu tad dṛśyam iti kathyate | yo dṛṣṭātītaḥ so'dṛśyas tadā dṛṣṭaṃ hi manyate | svatanūbhedam evan tu saṃsāraṃ duḥkhasaṅkulam |}
atha
      \app{\lem[wit={ceteri}]{ca}
        \rdg[wit={D}]{vā}}
      \app{\lem[wit={ceteri}]{yo}
        \rdg[wit={U1}]{yaḥ}
        \rdg[wit={D,N1,N2}]{ya}}
      \app{\lem[wit={ceteri}]{dṛṣṭi}
        \rdg[wit={L,N1}]{ddṛṣṭi}
        \rdg[wit={B}]{daṣṭi}
        \rdg[wit={D}]{dārṣṭi}
}viṣayaḥ sa
\app{\lem[wit={ceteri}]{dṛśya}
  \rdg[wit={N1}]{dṛśyad}
  \rdg[wit={U1}]{dṛṣy°}}
ucyate/
yo
\app{\lem[wit={ceteri}]{dṛṣṭyā}
  \rdg[wit={N2}]{dyā}}
na vīkṣyate sa adṛṣya
\app{\lem[wit={ceteri},alt={ity}]{i\skp{ty-u}}
  \rdg[wit={B,L}]{ty}
  \rdg[wit={N2}]{śaty}
}\skm{ty-u}cyate/
%------------------------------
%evaṃ saṃsārasya svātmano  bhedaṃ dūrīkṛty---aikam eva darśanaṃ sa eva jñānayogaḥ/   \E
%evaṃ saṃsāra----svātmano  bhedaṃ dūrīkṛtya  aikyena   darśanaṃ        jñānayogaḥ    \P
%evaṃ saṃsārasya svātmano  bheda--dūrīkṛtya  aikyona   darśanaṃ        jñānayogaḥ/   \B
%evaṃ saṃsāra----svātmano  bhedaṃ dūrīkṛtya  aikyona   darśanaṃ        jñānayogaḥ... \L
%evaṃ saṃsārasya svātmanaḥ bhedāṃ dūrīkṛtya  ekyena    darśanaṃ        jñānayogaḥ//  \N1
%evaṃ saṃsārasya svātmanaḥ bhedāṃ dūrīkṛtya  ekyena    darśanaṃ        jñānayogaḥ/   \D
%evaṃ saṃsārasya svātmanaḥ bhedaṃ dūrīkṛtya  ekena     darśanaṃ        jñānayogaḥ/   \N2
%evaṃ saṃsārasya svātmanaḥ bhedaṃ dūrīkṛtya  ekānta    darśanaṃ        jñānayogaḥ    \U1
%evaṃ saṃsāra....svātmanoḥ bhedaṃ dūrīkṛtyaṃ ekye?     darśanaṃ        jñānayoga     \U2
%------------------------------
%In this way, the realisation of unity (\textit{aikyena darśanam}) by eliminating the differentiation of the self from the mundane existance is truly Jnānayoga.
%------------------------------
evaṃ
       \app{\lem[wit={ceteri}]{saṃsārasya}
         \rdg[wit={P,L,U2}]{saṃsāra°}}
       \app{\lem[wit={B,E,L,P}]{svātmano}
         \rdg[wit={X}]{svātmanaḥ}
         \rdg[wit={U2}]{svātmanoḥ}}
       \app{\lem[wit={ceteri}]{bhedaṃ}
         \rdg[wit={B}]{bheda}
         \rdg[wit={D,N1}]{bhedāṃ}}
dūrī\app{\lem[wit={U2},alt={°kṛtyaṃ}]{kṛtyaṃ}
  \rdg[wit={ceteri}]{°kṛtya}
  \rdg[wit={E}]{°kṛty}}
\app{\lem[wit={P}]{aikyena}
  \rdg[wit={E}]{aikam eva}
  \rdg[wit={B,L,P}]{aikyona}
  \rdg[wit={D,N1}]{ekyena}
  \rdg[wit={N2}]{ekena}
  \rdg[wit={U1}]{ekānta}
  \rdg[wit={U2}]{ekye}}
darśanaṃ
\app{\lem[wit={E}]{sa eva}
  \rdg[wit={ceteri}]{\om}}
\app{\lem[wit={ceteri}]{jñānayogaḥ}
  \rdg[wit={U2}]{jñānayoga}}/ 
%------------------------------
%tasya         kāraṇāt kālaḥ śarīranāśaṃ na karoti/ \E
%tasya         kāraṇāt kālaḥ śarīranāśaṃ na karoti/ \P
%tasya         karaṇāt kālaḥ śarīranāśaṃ na karoti// \B
%tasya         karaṇāt kālaḥ śarīranāśaṃ na karoti... \L
%tasya         karaṇāt kālaḥ śarīranāśaṃ na karoti// \N1
%tasya         karaṇāt kālaḥ śarīranāśaṃ na karoti// \D
%tasya         karaṇāt kālaḥ śarīranāśaṃ    karoti/ \N2
%gatasya dhyānakaraṇāt kālaḥ śarīranāśaṃ na karoti 8 \U1
%tasya         karaṇāt kālaśarīranāśanaṃ    karoti// \U2
%------------------------------
%Due to this, time does not bring about the destruction of the body.
%------------------------------
\app{\lem[wit={ceteri}]{tasya}
  \rdg[wit={U1}]{gatasya}}
\app{\lem[wit={ceteri}]{kāraṇāt}
  \rdg[wit={U1}]{dhyānakaraṇāt}}
\app{\lem[wit={ceteri}]{kālaḥ}
  \rdg[wit={U1}]{kāla°}}
śarīranāśaṃ
\app{\lem[wit={ceteri}]{na}
  \rdg[wit={N2,U2}]{\om}}
karoti\dd{}\textsuperscript{\begin{otherlanguage}{english}\coro{[\lowroman{14}]}\end{otherlanguage}}\vspace*{\fill}
\end{prose}
\ekdpb*{}
  \end{edition}
  \begin{translation}
    \ekddiv{type=trans}
    \begin{tlate}
      \paragraph{\uproman{22}\textsuperscript{\lowroman{6}}.} By virtue of its inherent nature, the branch of a tree with its new shoots (\textit{vistāra}), being the fruit of the flower of love, is in the seed. Certainly, that [the inherent nature?] is pure, eternal, unchanging, and immaculate.
      \paragraph{\uproman{22}\textsuperscript{\lowroman{7}}.} One, not one, self-existing by it's own power, abiding in multiplicity, as five [gross] elements (\textit{tattva})\footnote{The term \textit{pañcatattva} refers to the five gross elements. The concept of five gross elements emerging from a supreme god is found in some other tantric works, cf. e.g. \citetitle{sivasvarodaya} 6-8. \begin{quote}
nirañjano nirākara eko devo maheśvaraḥ |\\
tasmād ākāśam utpannam ākāśād vāyusambhavaḥ||6||\\
vayos tejas tataś cāpas tataḥ pṛthvī samudbhavaḥ |\\
etāni pañcatattvāni vistīrṇāni ca pañcadhā ||7||\\
tebhyo brahmāṇḍam utpannaṃ tair eva parivartate |\\
vilīyate ca tatraiva tatraiva ramate punaḥ ||8|| \end{quote}
\begin{quote}
``Faultless and without body is the one god, the great supreme ruler. From him the ether element (\textit{ākāśa}) originated, and from the ether element the air element came into existance (6). From the air element, the fire element and from the fire element, the water element and from the water element, the earth element. These five elements unfold in a fivefold manner (7). From these five elements the universe has arisen, evolves and dissolves. [Then] right here, right there [it] enjoys again (8).''\end{quote}} thinking mind (\textit{manas}), intellect (\textit{buddhi}), illusion (\textit{māya}), individuation (\textit{ahaṃkāra}), and modifications (\textit{vikriyā}).\footnote{In the tantric traditions of Śaivism tenfold \textit{tattva}-systems existed \ldots.}
  \paragraph{\uproman{22}\textsuperscript{\lowroman{8}}.} Thus, the ten variations permeate the whole world and non-world. There is only one, there is no other. One who knows this is a knower of the truth."\\

The mundane existance (\textit{saṃsāra}) exists in the form of stationary [existances] such as earth, tree mountains and so on. The mundane existance (\textit{saṃsāra}) [also] exists in the form of the mobile [existances] such as humans, animals, birds and so on.\textsuperscript{\coro{[\lowroman{10}]}} Now, whatever is the object of perception, it is called visible. Whatever is not seen by perception, it is called invisible. In this way, the realisation of unity (\textit{aikyena darśanam}) by eliminating the differentiation of the self from the mundane existance is truly Jnānayoga. Due to this, time does not bring about the destruction of the body.\textsuperscript{\coro{[\lowroman{14}]}}
    \end{tlate}
  \end{translation}
\end{alignment}
%%%%%%%%%%%%%%%%%%%%%%%%%%%%%%%%%%%%%%%%%%
%%%%%%%%%%%%%%%%%%%%%%%%%%%%%%%%%%%%%%%%%%
%%%%%%%%PAGEBREAK%%%%%%%PAGEBREAK%%%%%%%%%
%%%%%%%%%%%%%%%%%%%%%%%%%%%%%%%%%%%%%%%%%%
%%%%%%%%%%%%%%%%PAGEBREAK%%%%%%%%%%%%%%%%%
%%%%%%%%%%%%%%%%%%%%%%%%%%%%%%%%%%%%%%%%%%
%%%%%%%%PAGEBREAK%%%%%%%PAGEBREAK%%%%%%%%%
%%%%%%%%%%%%%%%%%%%%%%%%%%%%%%%%%%%%%%%%%%
%%%%%%%%%%%%%%%%%%%%%%%%%%%%%%%%%%%%%%%%%%
%%%%%%%%%%%%%%%%%%%%%%%%%%%%%%%%%%%%%%%%%%
%%%%%%%%%%%%%%%%%%%%%%%%%%%%%%%%%%%%%%%%%%
%%%%%%%%PAGEBREAK%%%%%%%PAGEBREAK%%%%%%%%%
%%%%%%%%%%%%%%%%%%%%%%%%%%%%%%%%%%%%%%%%%%
%%%%%%%%%%%%%%%%PAGEBREAK%%%%%%%%%%%%%%%%%
%%%%%%%%%%%%%%%%%%%%%%%%%%%%%%%%%%%%%%%%%%
%%%%%%%%PAGEBREAK%%%%%%%PAGEBREAK%%%%%%%%%
%%%%%%%%%%%%%%%%%%%%%%%%%%%%%%%%%%%%%%%%%%
%%%%%%%%%%%%%%%%%%%%%%%%%%%%%%%%%%%%%%%%%%
%%%%%%%%%%%%%%%%%%%%%%%%%%%%%%%%%%%%%%%%%%
%%%%%%%%%%%%%%%%%%%%%%%%%%%%%%%%%%%%%%%%%%
%%%%%%%%PAGEBREAK%%%%%%%PAGEBREAK%%%%%%%%%
%%%%%%%%%%%%%%%%%%%%%%%%%%%%%%%%%%%%%%%%%%
%%%%%%%%%%%%%%%%PAGEBREAK%%%%%%%%%%%%%%%%%
%%%%%%%%%%%%%%%%%%%%%%%%%%%%%%%%%%%%%%%%%%
%%%%%%%%PAGEBREAK%%%%%%%PAGEBREAK%%%%%%%%%
%%%%%%%%%%%%%%%%%%%%%%%%%%%%%%%%%%%%%%%%%%
%%%%%%%%%%%%%%%%%%%%%%%%%%%%%%%%%%%%%%%%%%
\begin{alignment}[
  texts=edition[class="edition"];
  translation[class="translation"],
  ]
  \begin{edition}
    \ekddiv{type=ed}
     \centerline{\textrm{\small{[\uproman{23}.\textsuperscript{\coro{\lowroman{1}-\lowroman{30}}}Division of the Inherent Being]}}}
          \bigskip
          \begin{prose}
            \noindent
%------------------------------
%idānīṃ tasya---bhedaḥ    kathyate/   \E
%idānīṃ svabhāvabhedaḥ kathyate    \P
%idānī  svābhāvabhedaḥ kathyate//  \B
%idānīṃ svābhāvabhedaḥ kathyate//  \L
%idānīṃ svabhāvabhedaṃ kathyate//  \N1
%idānīṃ svabhāvabhedaṃ kathyate//  \D
%idānīṃ svabhāvabheda  kathyate//  \N2
%idānīṃ svabhāvabhedāḥ kathyate    \U1
%idānīṃ svabhāvabhedaḥ kathyate//  \U2
%------------------------------
%Now the division of the inherent being is described. 
%------------------------------  
\note[type=source, labelb=143, lem={\textbf{Re}}]{PT\textsuperscript{qcr \cdot YSV} (Ed. p. 836): svabhāvabhedam etat śṛṇu devi prayatnataḥ |}
\app{\lem[wit={ceteri}]{idānīṃ}
  \rdg[wit={B}]{idānī}}
\app{\lem[wit={ceteri},alt={svabhāva°}]{svabhāva}
  \rdg[wit={B,L}]{svābhāva°}
  \rdg[wit={E}]{tasya}
}\app{\lem[wit={D,N1},alt={°bhedaṃ}]{bhedaṃ}
  \rdg[wit={N2}]{°bheda}
  \rdg[wit={ceteri}]{°bhedaḥ}}
kathyate/
%------------------------------  
%yathā vaṭabījam/ vaṭarūpeṇa pariṇataṃ    sat    daśadhā    bhedaṃ svabhāvata eva prāpnoti/  \E %%%[P.27]
%yathā vaṭabījaṃ  vaṭarūpeṇa pariṇāte     sat    dṛśadhā    bhedaṃ svabhāvata eva prāpnoti   \P
%yathā vaṭabījena rūpeṇa     pariṇamate/  śata   daśadhā    bhedaṃ svābhāva   eva prāpnotī// \B
%yathā vaṭabījena rūpeṇa     pariṇamate   śata   daśadhā    bhedaṃ svābhāva   eva prāpnotī// \L
%yathā vaṭabījaṃ  vaṭarūpeṇa pariṇataṃ//  satṛ   daśadhā    bhedaṃ svabhāvata eva prāpnoti/  \N1
%yathā vaṭabījaṃ  vaṭarūpeṇa pariṇataṃ/   sa     daśadhā    bhedaṃ svabhāvata eva prāpnoti// \D
%yathā vathabījaṃ vaṭarūpeṇa pariṇataṃ/   sa tu  daśadhā    bhedaṃ svabhāvata eva prāpnoti/  \N2
%yathā vaṭabījaṃ  vaṭarūpeṇa pariṇataṃ    sa tat daśadhā    bhedaṃ svabhāvata eva prāpnotī   \U1
%yathā vaṭabīja---vaṭarūpeṇa pariṇamate// sa     dasat                            prāpnoti// \U2
%------------------------------
%Just as the seed of the banyan tree ripens into the shape of the banyan tree, [and] because of its own inherent being develops such a tenfold division. [Namely]:
%------------------------------
yathā
\app{\lem[wit={ceteri},alt={vaṭa°}]{vaṭa}
  \rdg[wit={N2}]{vatha°}
}\app{\lem[wit={D,P,N1,N2,U1},alt={°bījaṃ}]{bījaṃ}
        \rdg[wit={E}]{°bījam}
        \rdg[wit={U2}]{°bīja°}
        \rdg[wit={B,L}]{°bījena}}
      \app{\lem[wit={ceteri}]{vaṭarūpeṇa}
        \rdg[wit={B,L}]{rūpeṇa}}
      \app{\lem[wit={B,L,U2}]{pariṇamate}
        \rdg[wit={P}]{pariṇāte}
        \rdg[wit={X,E}]{pariṇataṃ}}
      \app{\lem[wit={U1}]{sa tat}
        \rdg[wit={N2}]{sa tu}
        \rdg[wit={N1}]{satṛ}
        \rdg[wit={E,P}]{sat}
        \rdg[wit={B,L}]{śata}
        \rdg[wit={D,U2}]{sa}}
      \app{\lem[wit={ceteri}]{daśadhā}
        \rdg[wit={P}]{dṛśadhā}
        \rdg[wit={U2}]{dasat}}
      \app{\lem[wit={ceteri}]{bhedaṃ}
        \rdg[wit={U2}]{\om}}
      \app{\lem[wit={ceteri}]{svabhāvata}
        \rdg[wit={B,L}]{svabhāva}
        \rdg[wit={U2}]{\om}}
      \app{\lem[wit={ceteri}]{eva}
        \rdg[wit={U2}]{\om}}
      \app{\lem[wit={ceteri}]{prāpnoti}
        \rdg[wit={B,L,U1}]{prāpnotī}}/
%------------------------------ %%%%STEMMA POINT!!!!
%mūlāṃkura---tvagdaṇḍaśākhā--kalikāpallavapuṣpaphalasnehā                  iti daśabhedān    prāpnoti// \E
%mūla aṃkura-tvakdaṃdaśākhā----kilpikāpallavā puṣpaphalasneha              iti daśabhedān    prāpnotīti \P  %%%7642.jpg
%mūlaṃ aṃkuratvakdaṃdaśākhā----kilakālapallavā// vistāroyaṃ svābhāvataḥ    iti daśabhedān    prāpnoti// \B DSCN7160 Z. 4
%mūlaṃ aṃkuratvakdaṃdaśākhā----kilāpallavā// vistāroyaṃ svābhāvataḥ//      iti daśabhedān    prāpnoti... \L
%mūlāṃ aṃkuratvakdaṃḍaśākhāṃ kalikāpallavapuṣpaphalasneha//                iti bhedo daśadhā prāpnoti// \N1
%mūlāṃkura---tvakdaṇdaśākhāṃ kalikāpallavapuṣpaphalasnehaṃ                 iti bhedo daśadhā prāpnoti// \D
%mūlāṃkura---tvakdaṇdaśākhāṃ kalikāpallavapuṣpaphalasneha/                 iti bhedo daśadhā prāpnoti// \N2
%mūlāṃaṃkura-tvakdaṇdaśākhā--kalikāpallavapuṣpaphalasneha                  iti bhedo daśadhā prāpnoti \U1
%\om                                                                                \U2
%------------------------------
%"Wurzel, Spross, Rinde, Ast, Zweig, Knospe, die sich entfaltende Blüte, Blüte, Frucht und Nektar." Die Auftheilung erreicht [diese] zehn Teile. 
%------------------------------
%"Root, shoot, bark, branch, twig, bud, the unfolding flower, flower, fruit and nectar." The division reaches [those] ten parts.
%------------------------------
\app{\lem[wit={E}]{mūlāṃkuratvagdaṇḍaśākhākalikāpallavapuṣpaphalasnehā}
          \rdg[wit={P}]{mūla aṃkuratvakdaṃdaśākhākilpikāpallavā puṣpaphalasneha}
          \rdg[wit={B}]{mūlaṃ aṃkuratvakdaṃdaśākhākilakālapallavā || vistāroyaṃ svābhāvataḥ}
          \rdg[wit={L}]{mūlaṃ aṃkuratvakdaṃdaśākhākilāpallavā || vistāroyaṃ svābhāvataḥ ||}
          \rdg[wit={N1}]{mūlāṃ aṃkuratvakdaṃḍaśākhāṃ kalikāpallavapuṣpaphalasneha ||}
          \rdg[wit={N2}]{mūlāṃkuratvakdaṇdaśākhāṃ kalikāpallavapuṣpaphalasneha|}
          \rdg[wit={D}]{mūlāṃkuratvakdaṇdaśākhāṃ kalikāpallavapuṣpaphalasnehaṃ}
          \rdg[wit={U1}]{mūlāṃaṃkuratvakdaṇdaśākhākalikāpallavapuṣpaphalasneha}
          \rdg[wit={U2}]{\om}}
        \app{\lem[wit={ceteri}]{iti}
          \rdg[wit={U2}]{\om}}
        \app{\lem[wit={X}]{bhedo daśadhā}
          \rdg[wit={B,E,L,P}]{daśabhedān}
          \rdg[wit={U2}]{\om}}
        \app{\lem[wit={ceteri}]{prāpnoti}
          \rdg[wit={P}]{prāpnotīti}
          \rdg[wit={U2}]{\om}}/
%------------------------------
%yathā nirmalo  nirvikāraḥ niraṃjana   eka  etādṛśa  ātmā svabhāvād eva/ pṛthivyaptejovāyvākāśamanobuddhimāyāvikārarūpabhedān    prāpnoti/ \E
%tathā nirmalaḥ nirvikāraḥ niraṃjanaḥ  eka  etādṛśa  ātmasvabhāvād eva   pṛthvyetetejo vādvyākāśamanobuddhimāyāvikārarūpabhedāt  prāpnoti \P
%tathā nirmalo  nirvikāraḥ niraṃjanaḥ  eka  etādṛśa  ātmasvabhāvād eva   pṛthvyāpatejovādvyākāśamanobuddhimāyāvikārarūpabhedāna  prāpnoti// \B
%tathā nirmalo  nirvikāraḥ niraṃjanaḥ/ eka  etādṛśa  ātmasvabhāvād eva   pṛthvyāpatejovāybākāśamanobuddhimāyāvikārarūpābhedāna   prāpnoti  \L
%tathā nirmalaḥ nirvikāraḥ niraṃjanaḥ  ekaḥ etādṛśaḥ ātmasvabhāvād eva   pṛthvyāpatejovāybākāśamanobuddhimāyāvikārarūpābhedān    prāpnoti/ \N1
%tathā nirmalaḥ nirvikāraḥ niraṃjanaḥ  eka  etādṛśaḥ ātmasvabhāvād eva   pṛthvīpate/ jīvīkāśamanobuddhir māyāvikārarūpabhedāt    prāpnoti \D
%tathā nirmalaḥ nirvikāraḥ niraṃjanaḥ  ekaḥ etādṛśaḥ ātmasvabhāvād eva   pṛthvīpate/ jīvīkāśamanobuddhir māyāvikārarūpabhedāt    prāpnoti/ \N2
%tathā nirmalaḥ nirvikāraḥ niraṃjanaḥ  ekaḥ etādṛśaḥ ātmascabhāvād eva   pṛthakte jīvāyuvākāśamanobuddhir māyāyāvikārarūpabhedāt prāpnoti \U1 %%%275.jpg
%yathā nirmalaḥ nirvikāraḥ niraṃjanaḥ  eka  etādṛśa  ātmasvabhāvād eva// pṛthvyaptejovāyyākāśa// manobuddhimayāvikārarūpabhedān  prāpnoti/ \U2
%------------------------------
%In dieser Weise erreicht auch das reine, unveränderliche, makellose, eine solche [Auftheilung] eben aufgrund der inhärenten Natur des Selbst. [Nämlich] die Aufteilung "Erde, Wasser, Feuer, Wind, Raum, Geist, Intellektekt, Illusion, Umwandlungen und Gestalt".
%------------------------------
%In this way, the pure, unchanging, unblemished, attains such [a division] precisely, because of the inherent being of the self. [Namely] the division: "Earth, water, fire, wind, space, mind, intellect, illusion, transformations and form".
%------------------------------
        \app{\lem[wit={ceteri}]{tathā}
            \rdg[wit={E,U2}]{yathā}}
          \app{\lem[wit={B,E,L}]{nirmalo}
            \rdg[wit={X,P,U2}]{nirmalaḥ}}
          nirvikāraḥ
          \app{\lem[wit={E}]{niraṃjana}
            \rdg[wit={ceteri}]{niraṃjanaḥ}}
          \app{\lem[wit={ceteri}]{eka}
            \rdg[wit={N1,N2,U1}]{ekaḥ}}
          \app{\lem[wit={E}]{etādṛśa}
            \rdg[wit={N1,N2,U1}]{etādṛśaḥ}}
          \app{\lem[wit={ceteri}]{ātmasvabhāvād}
            \rdg[wit={E}]{ātmā svabhāvād}}
          eva
          \app{\lem[wit={N1,B,L}]{pṛthvyāpatejovāybākāśamanobuddhimāyāvikārarūpābhedān}
            \rdg[wit={E}]{pṛthivyapāpatejovāybākāśamanobuddhimāyāvikārarūpābhedān}
            \rdg[wit={P}]{pṛthvyetetejovādvyākāśamanobuddhimāyāvikārarūpābhedān}
            \rdg[wit={D,N2}]{pṛthvīpate | jīvīkāśamanobuddhir māyāvikārarūpabhedāt}
            \rdg[wit={U1}]{pṛthakte jīvāyuvākāśamanobuddhir māyāyāvikārarūpabhedāt}
            \rdg[wit={U2}]{pṛthvyaptejovāyyākāśa || manobuddhimayāvikārarūpabhedā}}
          prāpnoti/\textsuperscript{\begin{otherlanguage}{english}\coro{[\lowroman{5}]}\end{otherlanguage}}
%------------------------------
%jñānayogaprabhāvād     eka eva  ātmā iti niścayo bhavati// \E
%jñānayogaḥ prabhāvād   eka eka  ātmā iti niścayo bhavati \P
%jñānayogaḥ// prabhāvād eka eka  ātmā iti niścayā bhavatī// \B
%jñānayogaḥ// prabhāvād eka eka  ātmā iti niścayo bhavati// \L
%jñānayogaprabhāvāt     eka eva  ātmā iti niścayo bhavati// \N1
%jñānayogaprabhāvāt     eka eva  ātmā iti niścayo bhavati// \D
%jñānayogaprabhāvāt     eka eva  ātmā iti niścayo bhavati// \N2
%jñānayogaprabhāvāt tu  eka yeva ātmā iti niścayo bhavati \U1
%jñānayogaprabhāvād     eka eva  ātmā iti niścayo bhavati// \U2
%------------------------------
%Because of the power of Jñānayoga, there arises the conviction that "the self is truly one".  
%------------------------------
\app{\lem[wit={E,U2}, alt={jñānayogaprabhāvād}]{jñānayogaprabhāvā\skp{d-e}}
  \rdg[wit={X}]{jñānayogabhavāt}
  \rdg[wit={B,L}]{jñānayogaḥ || prabhāvād°}
  \rdg[wit={P}]{jñānayogaḥ prabhāvād}
}\skm{d-e}ka
\app{\lem[wit={ceteri}]{eva}
  \rdg[wit={B,L,P}]{eka}
  \rdg[wit={U1}]{yeva}}
ātmā iti niścayo bhavati/
%------------------------------
%yathaikaiva   pṛthvī  kvacit komalarūpā                                                   kvacit parimalarūparahitā kvacit suvarṇarūpā   kvacid raupyarūpā    \E %%%p.28 
%yathā ekaika  pṛthvī  kvacit komalarūpā                                                                                                                       \P   
%yathā ekaika  pṛthvī  kvacit komalarūpā// kvacit manohararūpā//  kvacit parimalarūpayuktā// kvacit parimalarohitā// kvacit suvarṇarūpa                        \B
%yathā ekaika  pṛthvī  kvacit komalarūpā   kvacit manohararūpāḥ// kvacit parimalarūpayuktā// kvacit parimalarahitā// kvacit suvarṇarūpā                        \L
%yathā ekaiva  pṛthivī kvacit komalarūpa/  kvacit manoharā/       kvacit parimalarūpāyuktā// kvacit parimalarahitā/  kvacit suvarṇarūpā/  kvacit rūpyarūpā/    \N1
%yathā ekaiva  pṛthivī kvacit komalarūpa   kvacit manoharā//      kvacit parimalarūpāyuktā/  kvacit parimalarohitā   kvacit suvarṇarūpa// kvacit rūpyarūpa//   \D
%yathā ekaṃ ca pṛthivī kvacit komalarūpa   kvacit manoha?rā       kvacit parimalarūpāyuktaḥ/ kvacit parimalarohitā   kvacit suvarṇarūpā   kvacit rūpyarūpa     \N2
%yathā ekai ca pṛthivī kvacit                                                                                              khavarṇakupā   kvacit rūpyarūpā     \U1
%yathā ekaika  pṛthvī  kvacit komalarūpā// kvacit manohararūpa//  kvacit parimalarūpāyuktā/  kvacit parimalarohitā// kvacit suvarṇarūpā// kvacit rajatarūpā//  \U2
%------------------------------
%As some particular soil (\textit{ekaika}) sometimes appears soft, sometimes appears beautiful, sometimes fragrant, sometimes unscented, sometimes golden, sometimes silver,... 
%------------------------------
\note[type=source, labelb=145, lem={\textbf{Re}}]{PT\textsuperscript{qcr \cdot YSV} (Ed. p. 836): ātmano vā pṛthivyādyāḥ svabhāvaḥ kiñcid ucyate | ātmaiva pṛthivī dhātrī komalā ca kvacid dṛḍhā | kvacin manoharā sā ca vimalā ca malāmalā | durgandhā ca sugandhā ca nirgandhā gandhamohinī | svarṇarūpā dhāturūpā citrā ratnamayī parā | kvacit śvetā kvacid raktā kvacit pītā ca kṛṣṇalā | ūrvarā ūrvarā sā tu viṣāmṛtamayī sadā |}
\app{\lem[type=emendation, resp=egoscr]{yathaikaikaḥ}
  \rdg[wit={E}]{yathaikaiva}
  \rdg[wit={B,L,P,U2}]{yathā ekaika}
  \rdg[wit={D,N1}]{yathā ekaiva}
  \rdg[wit={N2}]{yathā ekaṃ ca}
  \rdg[wit={U1}]{yathā ekai ca}}
\app{\lem[wit={Y}]{pṛthvī}
  \rdg[wit={X}]{pṛthivī}}
kvacit
komala\app{\lem[wit={Y},alt={°rūpā}]{rūpā}
    \rdg[wit={X}]{°rūpa}}\dd{}
\app{\lem[wit={ceteri}]{kvacit}
  \rdg[wit={E,P,U1}]{\om}}
\app{\lem[wit={B}]{manohararūpā}
  \rdg[wit={L}]{°rūpāḥ}
  \rdg[wit={U2}]{°rūpa}
  \rdg[wit={D,N1,N2}]{manoharā}
  \rdg[wit={E,P,U1}]{\om}}\dd{}
\app{\lem[wit={ceteri}]{kvacit}
  \rdg[wit={E,P,U1}]{\om}}
\app{\lem[wit={ceteri},alt={°parimala}]{parimala}
  \rdg[wit={E,P,U1}]{\om}
}\app{\lem[wit={B,L},alt={°rūpayuktā}]{rūpayuktā}
  \rdg[wit={D,N1}]{°rūpā°}
  \rdg[wit={N2}]{°rūpāyuktaḥ}
  \rdg[wit={E,U1}]{\om}}\dd{}
\app{\lem[wit={ceteri}]{kvacit}
  \rdg[wit={P,U1}]{\om}}
\app{\lem[wit={ceteri},alt={°parimala}]{parimala}
  \rdg[wit={E}]{°parimalarūpa°}
  \rdg[wit={P,U1}]{\om}
}\app{\lem[wit={E,L,N1},alt={°rahitā}]{rahitā}
  \rdg[wit={B,N2,U2}]{°rohitā}
  \rdg[wit={D,P,U1}]{\om}}\dd{}
\app{\lem[wit={ceteri}]{kvacit}
  \rdg[wit={P,U1}]{\om}}
\app{\lem[wit={E,L,N2,U2}]{suvarṇarūpā}
  \rdg[wit={B,D}]{suvarṇarūpa}
  \rdg[wit={U1}]{khavarṇakupā}
  \rdg[wit={P}]{\om}}\dd{}
\app{\lem[wit={ceteri}]{kvacit}
  \rdg[wit={B,L,P}]{\om}}
\app{\lem[wit={E}]{raupyarūpā}
  \rdg[wit={N1,U1}]{rūpyarūpā}
  \rdg[wit={D,N2}]{rūpyarūpa}
  \rdg[wit={U2}]{rajatarūpā}
  \rdg[wit={B,L,P}]{\om}}\dd{}
%------------------------------
%kvacid ratnamayī   kvacic ca śvetā                                kvacidraktā   kvacitpītā    \E %%%p.28 
%                                                                                             \P   
%kvacid ratnamaī//  kvacit śverūpā// kvacitkṛṣṇā//                 kvacidraktā/  kvacitpītā//  \B
%kvacid ratnamaī//  kvacit śvetarūpā kvacitkṛṣṇā//                 kvacidraktā// kvacitpītā//  \L
%kvacid ratnamayī/  kvacit śveta/    kvacitkṛṣṇa??/                kvacidrakta/  kvacitpītā/   \N1
%kvacid ratnamayī// kvacit śvetā//   kvacitkṛṣṇā [S8., Z.7]        kvacidrakta   kvacitpītā//  \D
%kvacid ratnamayī   kvacit śveta     kvacitkṛṣṇā// [S6. verso]     kvacidrakta   kvacitpītā    \N2
%kvacid ratnamayī   kvacit śveta     kvacitkṛṣṇā                   kvacidrakta   kvacitpītā    \U1
%kvacid ratnamayī// kvacit śvetā//   kvacitkṛṣṇā//                 kvacidraktā// kvacitpītā//  \U2
%------------------------------
% ... manchmal aus Edelstein gemacht ist, manchmal weiß erscheint, manchmal schwarz, manchmal kupfern, manchmal gelb,
%
%... is sometimes made of precious stone, sometimes appearing white, sometimes black, sometimes copper, sometimes yellow, 
%------------------------------
\note[type=philcomm, labelb=145a, lem={kvacid}]{Sentence ???? is \om in P.}
kvaci\skp{d-ra}\app{\lem[wit={ceteri},alt={ratnamayī}]{\skm{d-ra}tnamayī}
  \rdg[wit={B,L}]{ratnamaī}}\dd{}
\app{\lem[wit={ceteri}]{kvacit}
  \rdg[wit={E}]{kvacic ca}}
\app{\lem[wit={E,D,U2}]{śvetā}
  \rdg[wit={N1,N2,U1}]{śveta}
  \rdg[wit={L}]{śvetarūpā}
  \rdg[wit={B}]{śverūpā}}\dd{}
\app{\lem[wit={ceteri}]{kvacit kṛṣṇā}
  \rdg[wit={N1}]{kṛṣṇa}
  \rdg[wit={E}]{\om}}\dd{}
kvaci\skp{d-ra}\app{\lem[wit={B,E,L,U2},alt={raktā}]{\skm{d-ra}ktā}
  \rdg[wit={ceteri}]{°rakta}}\dd{}
kvacit pītā\dd{}
    \end{prose}
  \end{edition}
  \begin{translation}
    \ekddiv{type=trans}
 \centerline{\textrm{\small{[\uproman{23}.\textsuperscript{\coro{\lowroman{1}-\lowroman{30}}}Division of the Inherent Being]}}}
          \bigskip
          \begin{tlate}
\blfootnote{\noindent \ldots from an very early age, cf. \citeauthor[2016:82-85]{goodall2016}. Rāmacandra, as can be seen in the sources of the edition for this passage, is faithful to his template. However, the exact origin of this specific arrangement of \textit{tattva}s remains unknown. Usually \textit{vikriyā} is not a separate \textit{tattva}, but \textit{ahaṃkāravikrīya} (``transformations of \textit{ahaṃkāra}'') refers to lesser \textit{tattva}s like the \textit{jñānendrīya}s, \textit{karmendrīya}s and \textit{tanmātra}s. Here, it seems the term \textit{vikriyā} is taken as a \textit{tattva} on its own, functioning as a placeholder for the modifications of \textit{ahaṃkāra}.} Now, the division of the inherent being is described. Just as the seed of the banyan tree ripens into the shape of the banyan tree, [and] because of its own inherent being develops such a tenfold division. [Namely]: root, shoot, bark, branch, twig, bud, the unfolding flower, flower, fruit and nectar. The division develops [those] ten parts. In this way, the pure, unchanging, unblemished, attains such [a division] precisely, because of the inherent being of the self. [Namely], the division: earth, water, fire, wind, space, mind, intellect, illusion, transformations and form.\footnote{Add a note and explain various tenfold \textit{tattva}-systems.}\textsuperscript{\coro{[\lowroman{5}]}} Because of the power of Jñānayoga, there arises the conviction that "the self is truly one". Just as some particular soil (\textit{ekaika}) sometimes appears soft, sometimes appears beautiful, sometimes fragrant, sometimes unscented, sometimes golden, sometimes silver, is sometimes made of precious stone, sometimes appearing white, sometimes black, sometimes copper, sometimes yellow, \ldots
            \end{tlate}
  \end{translation}
\end{alignment}
\ekdpb*{}
%%%%%%%%%%%%%%%%%%%%%%%%%%%%%%%%%%%%%%%%%%
%%%%%%%%%%%%%%%%%%%%%%%%%%%%%%%%%%%%%%%%%%
%%%%%%%%PAGEBREAK%%%%%%%PAGEBREAK%%%%%%%%%
%%%%%%%%%%%%%%%%%%%%%%%%%%%%%%%%%%%%%%%%%%
%%%%%%%%%%%%%%%%PAGEBREAK%%%%%%%%%%%%%%%%%
%%%%%%%%%%%%%%%%%%%%%%%%%%%%%%%%%%%%%%%%%%
%%%%%%%%PAGEBREAK%%%%%%%PAGEBREAK%%%%%%%%%
%%%%%%%%%%%%%%%%%%%%%%%%%%%%%%%%%%%%%%%%%%
%%%%%%%%%%%%%%%%%%%%%%%%%%%%%%%%%%%%%%%%%%
%%%%%%%%%%%%%%%%%%%%%%%%%%%%%%%%%%%%%%%%%%
%%%%%%%%%%%%%%%%%%%%%%%%%%%%%%%%%%%%%%%%%%
%%%%%%%%PAGEBREAK%%%%%%%PAGEBREAK%%%%%%%%%
%%%%%%%%%%%%%%%%%%%%%%%%%%%%%%%%%%%%%%%%%%
%%%%%%%%%%%%%%%%PAGEBREAK%%%%%%%%%%%%%%%%%
%%%%%%%%%%%%%%%%%%%%%%%%%%%%%%%%%%%%%%%%%%
%%%%%%%%PAGEBREAK%%%%%%%PAGEBREAK%%%%%%%%%
%%%%%%%%%%%%%%%%%%%%%%%%%%%%%%%%%%%%%%%%%%
%%%%%%%%%%%%%%%%%%%%%%%%%%%%%%%%%%%%%%%%%%
%%%%%%%%%%%%%%%%%%%%%%%%%%%%%%%%%%%%%%%%%%
%%%%%%%%%%%%%%%%%%%%%%%%%%%%%%%%%%%%%%%%%%
%%%%%%%%PAGEBREAK%%%%%%%PAGEBREAK%%%%%%%%%
%%%%%%%%%%%%%%%%%%%%%%%%%%%%%%%%%%%%%%%%%%
%%%%%%%%%%%%%%%%PAGEBREAK%%%%%%%%%%%%%%%%%
%%%%%%%%%%%%%%%%%%%%%%%%%%%%%%%%%%%%%%%%%%
%%%%%%%%PAGEBREAK%%%%%%%PAGEBREAK%%%%%%%%%
%%%%%%%%%%%%%%%%%%%%%%%%%%%%%%%%%%%%%%%%%%
%%%%%%%%%%%%%%%%%%%%%%%%%%%%%%%%%%%%%%%%%%
\begin{alignment}[
  texts=edition[class="edition"];
  translation[class="translation"],
  ]
  \begin{edition}
    \ekddiv{type=ed}
    \begin{prose}
    \noindent
%------------------------------
%kvacitkarburā   kvacin nānāvidharūpā        kvacid viṣarūpā    kvacid amṛtarūpamayī svabhāvata eva bhavati//  \E  %%%p.28
%                                                               kvacid amṛtamayī     svabhāvata eva bhavati    \P  %%%rest is \om
%kvacitkarburā// kvacin nānāvidhaphalarūpā   kvacit viṣarūpā//  kvacid amṛtamaī/     svabhāvata eva bhavataḥ// \B
%kvacitkarburā// kvacin nānāvidhāphalarūpā   kvacit viṣarūpā//  kvacid amṛtamaī//    svabhāvata eva bhavataḥ// \L
%kvacitkarburā,  kvacin nānāvidhaphalarūpā/  kvacid puṣparūpā,  kvacid amṛtamayī     svabhāvata eva bhavati/   \N1
%kvacitkarburā   kvacin nānāvidhaphalarūpā// kvacid puṣparūpā// kvacid amṛtamayī/    svabhāvata eva bhavati//  \D
%kvacitkarburā   kvacin nānāvidhaphalarūpā                      kvacid amṛtamayī/    svabhāvata eva bhavati//  \N2
%kvacitkarpurā   kvacin nānāvidhophalarūpā   kvacid ....[rest omitted]                                         \U1
%kvacitkarburā// kvacit nānāvidhaphalarūpā// kvacir viśarūpā//  kvacit amṛtamayī//   svabhāvata eva bhavati//  \U2
%------------------------------
%machmal gesprenkelt, machmal wie verschiedenartige Frucht erscheint, manchmal wie Blumen erscheint, machmal wie der Nektar der Unsterblichkeit erscheint, [und das nur] nur aufgrund seiner inhärenten Natur.
%------------------------------
%sometimes mottled, sometimes appearing like various fruit, sometimes appearing like flowers, sometimes appearing like the nectar of immortality, only because of its inherent being. 
%------------------------------
kvavit
\app{\lem[wit={ceteri}]{karburā}
  \rdg[wit={U1}]{karpurā}}\dd{}
\app{\lem[wit={ceteri}]{kvaci\skp{n-nā}}
  \rdg[wit={U2}]{kvacit}
  \rdg[wit={P}]{\om}
}\app{\lem[wit={ceteri},alt={nānāvidhaphalarūpā}]{\skm{n-nā}nāvidhaphalarūpā}
  \rdg[wit={U1}]{nānāvidhophalarūpā}
  \rdg[wit={E}]{nānāvidharūpā}
  \rdg[wit={P}]{\om}}\dd{}
\app{\lem[wit={ceteri},alt={kvacid}]{kvaci\skp{d-pu}}
  \rdg[wit={B,L}]{kvacit}
  \rdg[wit={U2}]{kvacir}
  \rdg[wit={P,N2}]{\om}
}\app{\lem[wit={D,N1},alt={puṣparūpā}]{\skm{d-pu}ṣparūpā}
\rdg[wit={B,E,L}]{viṣarūpā}
\rdg[wit={U2}]{vśarūpā}
\rdg[wit={U1}]{\om}}\dd{}
\app{\lem[wit={ceteri}, alt={kvacid}]{kvaci\skp{d-a}}
  \rdg[wit={U2}]{kvacit}
  \rdg[wit={U1}]{\om}
}\app{\lem[wit={ceteri},alt={amṛtamayī}]{\skm{d-a}mṛtamayī}
  \rdg[wit={E}]{amṛtarūpamayī}
  \rdg[wit={B,L}]{amṛtamaī}
  \rdg[wit={U1}]{\om}}\dd{}
\app{\lem[wit={ceteri}]{svabhāvata}
  \rdg[wit={U1}]{\om}}
\app{\lem[wit={ceteri}]{eva}
  \rdg[wit={U1}]{\om}}
\app{\lem[wit={ceteri}]{bhavati}
  \rdg[wit={B,L}]{bhavataḥ}
  \rdg[wit={U1}]{\om}}\dd{}
%------------------------------
%tathaivātmā   manuṣyapakṣihariṇahastividyādharagandharvakinnaramahāpaṃḍitamahāmūrkha  rogyarogikrodhi---śāṃtarūpaḥ      svabhāvād eva bhavati/ \E
%tathaivātmā   manuṣyapakṣihariṇāhastividyādharagaṃdharvakinnaramahāpiṃḍitamahārmūkha  rogī-----krodhi---śāṃtarūpāḥ      svabhāvād eva bhavati \P
%tathaivātmā// manuṣyapakṣihariṇahastividyādharagaṃdharvakinnaramahāpiṃḍatamahāmūrkha  rogī-----krodhadhiśāṃtarūpaḥ      svabhāvād eva bhavatī/ \B
%tathaivātmā   manuṣyapakṣihariṇahastividyādharagaṃdharvakinnaramahāpaṃḍitamahāmūrkha  rogī-----krodhadhīśāṃtarūpāḥ      svabhāvād eva bhavatī/ \L
%tathātmā//    manuṣyapakṣihariṇahastīvidyādharagandharvakiṃnaramahāpaṃḍitamahāmūrva   rogīarogīkrodhī---śāntarūpa-------svabhāvād eva bhati/ \N1 %%%%%%%CRAZY SWITCH BETWEEN DAṆḌA AND COMMA
%tathātmā//    manuṣyapakṣihariṇahastīvidyādharagandharvakinnaramahāpaṃḍitamahāmūrva   rogīarogīkrodhī---śāṃtarūpa-------svabhāvād eva bhavati/ \D
%tathātmā//    manuṣyapakṣihariṇahastividyādharagandharvakinnaramahāpaṇḍitamahāmūrkha  rogīarogīkrodhī---śāṃtarūpa-------svabhāvād eva bhavati/ \N2
%                                     vidyādharagaṃdharvakinnaramahāpaṇḍitamahāmūrṣa   rogīarogīkrodhī---śāṃtarūpa       evaṃ svabhāvaṃ dharati  \U1
%tathaivātmā   manuṣyapakṣihariṇahastividyādharagaṃdharvakinnaramahāpaṃḍitamahāmūrkha  rogīarogīkrodhi---śāṃtarūpaḥ      svabhāvād eva bhavati// \U2 %%%410.jpg
%------------------------------
%Auf diese Weise nimmt auch das Selbst aufgrund seiner inhärenten Natur die Form eines Menschen, Vogels, einer Gazelle, eines Elefants, eines Vidyādharas, eines Gandharvas, Zentauren, eines großen Gelehrten oder großen Dummkopfes, eines Kranken oder Gesunden, eines Zornigen oder Friedlichen an.
%
%In the same way, the self also takes the form of a human, a bird, a gazelle, an elephant, a vidyādhara, a gandharva, a centaur, great scholar or a great fool, a sick or healthy, an angry or or peaceful person, by virtue of its inherent being.       
%------------------------------      
\note[type=source, labelb=146, lem={\textbf{Re}}]{PT\textsuperscript{qcr \cdot YSV} (Ed. p. 836): tathā ca devagandharvakinnarādyāḥ khagādayaḥ | sukhasampiṇḍito rogī tathaiva krodhaśāntadhīḥ |aśeṣarūpabalito nānābuddhirataḥ svayam | devatattvaṃ bhūtaśaktyā jīvasaṃjñā bhramātmikā | jñānayogī nirvikāro nistāpa eka īśvaraḥ | ātmaikamūrttimān bhūtvā nirvikalpo nirañjanaḥ | sukhī duḥkhī mohayukto 'nantacetāḥ svabhāvataḥ |}
\app{\lem[wit={Y}]{tathaivātmā}
  \rdg[wit={X}]{tathātmā}}
\app{\lem[wit={ceteri},alt={manuṣya°}]{manuṣya}
  \rdg[wit={U1}]{\om}
}\app{\lem[wit={ceteri},alt={°pakṣi°}]{pakṣi}
  \rdg[wit={U1}]{\om}
}\app{\lem[wit={ceteri},alt={°hariṇa°}]{hariṇa}
  \rdg[wit={P}]{°hariṇā°}
  \rdg[wit={U1}]{\om}
}\app{\lem[wit={D,N1},alt={°hastī°}]{hastī}
  \rdg[wit={ceteri}]{hasti}
  \rdg[wit={U1}]{\om}
}vidyādharagaṃdharvakinnaramahā\app{\lem[wit={ceteri},alt={°paṇḍita°}]{paṇḍita}
  \rdg[wit={B}]{piṃḍata}
}mahā\app{\lem[wit={ceteri},alt={°mūrkha°}]{mūrkha}
  \rdg[wit={P}]{°rmūkha°}
  \rdg[wit={D,N1}]{°mūrva°}
  \rdg[wit={U1}]{°mūrṣa°}
}\app{\lem[type=emendation, resp=egoscr]{rogyarogī}
  \rdg[wit={E}]{°rogyarogi}
  \rdg[wit={X,U2}]{°rogī arogī}
  \rdg[wit={B,L,P}]{°rogī}
}\app{\lem[wit={ceteri},alt={°krodhī°}]{krodhī}
  \rdg[wit={E,P}]{°krodhi°}
  \rdg[wit={B,L}]{°krodha°}
}\app{\lem[wit={ceteri},alt={°śānta°}]{śānta}
  \rdg[wit={B,L}]{°dhiśānta°}
}\app{\lem[wit={ceteri},alt={°rūpaḥ}]{rūpaḥ}
  \rdg[wit={P,L}]{°rūpāḥ}
  \rdg[wit={X}]{°rūpa}}
\app{\lem[wit={ceteri},alt={svabhāvād eva}]{svabhāvād-eva}
  \rdg[wit={U1}]{evaṃ svabhāvaṃ}}
\app{\lem[wit={ceteri}]{bhavati}
  \rdg[wit={B,L}]{bhavatī}
  \rdg[wit={N1}]{bhati}
  \rdg[wit={D}]{dharati}}\dd{}\textsuperscript{\begin{otherlanguage}{english}\coro{[\lowroman{10}]}\end{otherlanguage}}
%------------------------------      
%jñānayogādhikārarūparahito  jñāyate/  yathā plakṣasyotpattiḥ/ sthānam eva bhavati// \E
%jñānayogādhikārarūparahito  jñāyate   yathā phalasyotpattisthānam ekam eva bhavati \P  %%%7643.jpg        
%jñānayogādhikārarūparahito  jñāyate// yathā phalasyotpattisthānam ekam eva bhavatī// \B
%jñānayogādhikārarūparahito  jñāyate// yathā phalasyotpattisthānam ekam eva bhavati// \L
%jñānayogād vikārarūparahito jñāyate/  yathā phalasyotpattisthānam ekam eva bhavati/ \N1
%jñānayogādhikārarūparahito  jñāyate// yathā phalasyotpattisthānam ekaseva  bhavati// \D
%jñānayogadhikārarūparahito  jñāyate// yathā phalasyotpattisthānam eva kameva bhavati// \N2
%jñānayogāt vikārarūparahito jñāyate   yathā phalasyotpattisthāna  ekam eva ti \U1
%jñānayogādhikārarūparahito  jāyate//  yathā phalasyotpattisthānam ekam eva bhavati// \U2
%------------------------------
%Through Jñānayoga he realizes the emptiness of the mutability of form. Just as the place of origin of the fruit is%%only one.
%------------------------------
\app{\lem[wit={N1,U1}, alt={jñānayogād vikāra}]{jñānayogād-vikāra}
  \rdg[wit={ceteri}]{jñānayogadhikāra}
}rūparahito
\app{\lem[wit={ceteri}]{jñāyate}
  \rdg[wit={U2}]{jāyate}}/
yathā
\app{\lem[wit={ceteri}]{phalasyotpatti}
  \rdg[wit={E}]{plakṣasyotpattiḥ}
}\app{\lem[wit={ceteri},alt={°sthānam}]{sthāna\skp{m-e}}
  \rdg[wit={E}]{sthānam}
  \rdg[wit={U1}]{°sthāna}
}\app{\lem[wit={ceteri},alt={ekam}]{\skm{m-e}ka\skp{m-e}}
  \rdg[wit={D}]{ekas}
  \rdg[wit={N2}]{eva}
  \rdg[wit={E}]{\om}
}\app{\lem[wit={ceteri},alt={eva}]{\skm{m-e}va}
  \rdg[wit={N2}]{kam eva}}
\app{\lem[wit={ceteri}]{bhavati}
  \rdg[wit={B}]{bhavatī}
  \rdg[wit={U1}]{ti}}/
%------------------------------
%atha ca phalasya gatir bahudhā dṛśyate/ \E
%atha ca phalasya gati  bahudhā dṛśyate    \P
%atha ca phalasya gatir bahudhā dṛśyate// \B
%atha ca phalasya gatir bahudhā dṛśyate// \L
%atha ca phalasya gatir bahudhā dṛśyate/ \N1
%atha ca phalasya gatir bahudhā dṛśyate// \D
%atha ca phalasya gati  bahudhā dṛśyate/ \N2
%atra ca phalasya gati  bahudhā dṛśyate \U1
%atha ca phalasya gatir bahudhā dṛśyate// \U2
%------------------------------
%But the development of the fruit is seen manifold. 
%------------------------------
atha ca phalasya \app{\lem[wit={ceteri},alt={gatir}]{gati\skp{r-ba}}
  \rdg[wit={P,N2,U1}]{gati}
}\skm{r-ba}hudhā dṛśyate\dd{}
%------------------------------ %%%STEMMAPOINT śuklaṃ//śuṣkaṃ
% ekaṃ phalaṃ pṛthvīmadhye  patati/  śuklaṃ bhavati/   \E
% ekaṃ phalaṃ pṛthvīmadhye  patati   śuklaṃ bhavati    \P
% ekaṃ phalaṃ pṛthvīmadhye  patiśuklaṃ      bhavatī//  \B
% ekaṃ phalaṃ pṛthvīmadhye  patati   śuṣkaṃ bhavatī    \L
% ekaṃ phala--pṛthvīmadhye  patati/  śuklaṃ bhavati/   \N1 %%%p.7 recto letzte Zeile 
% ekaṃ phala--pṛthvīmadhye  patati// śuklaṃ bhavati//  \D
% eva  phala--pṛthvīmadhye  patati   śuklaṃ bhavati//  \N2
% ekaṃ phalaṃ pṛthivīmadhye patati   śuṣkaṃ bhavati    \U1
% ekaphalaṃ   pṛthvīmadhye  patati// śuṣkaṃ bhavati//  \U2
%------------------------------
%One fruit falls onto the ground, and becomes dry. 
%------------------------------
\app{\lem[wit={ceteri}]{ekaṃ}
  \rdg[wit={U2}]{eka°}
  \rdg[wit={N2}]{eva}}
\app{\lem[wit={ceteri}]{phalaṃ}
  \rdg[wit={D,N1,N2}]{phala°}}
\app{\lem[wit={ceteri},alt={pṛthvī°}]{pṛthvī}
  \rdg[wit={U1}]{pṛthivī°}
}madhye patati/
\app{\lem[wit={ceteri}]{śuklaṃ}
  \rdg[wit={L,U1,U2}]{śuṣkaṃ}}
\app{\lem[wit={ceteri}]{bhavati}
  \rdg[wit={B}]{bhavatī}}/\textsuperscript{\begin{otherlanguage}{english}\coro{[\lowroman{15}]}\end{otherlanguage}}
%------------------------------
% ekasya phalasya makaraṃdaṃ bhramaraḥ  pibati/  \E
% ekasya phalasya makaraṃdaṃ bhramaraḥ  pibaṃti  \P
% ekasya            karaṃdaṃ bhramaraṃ  pibatī/  \B
% ekasya          makaraṃdaṃ bhramaraṃ  pibati   \L
% ekasya phalasya makaraṃdabhramaraḥ    pibati/  \N1 %%%p.7 recto letzte Zeile 
% ekasya phalasya makaraṃdabhramaraḥ    pibati/  \D
% ekasya phalasya makaraṃdaṃ bhramara   pibati/  \N2
% ekasya phalasya makaraṃdaṃ bhramanaḥ  pibati   \U1
% ekasya phalasya makaraṃdaṃ bhramaraḥ  pibati// \U2
% ------------------------------
% Eine Biene trinkt den Saft der einen Frucht.
% A bee drinks the juice of the one fruit.     
%------------------------------
ekasya
\app{\lem[wit={ceteri}]{phalasya}
  \rdg[wit={P,L}]{\om}} 
\app{\lem[wit={E,L,P,N2,U1,U2}]{makarandaṃ}
  \rdg[wit={L,N1}]{makaraṃda°}
  \rdg[wit={B}]{karaṃdaṃ}}
\app{\lem[wit={ceteri}]{bhramaraḥ}
  \rdg[wit={B,L}]{bhramaraṃ}
  \rdg[wit={N2}]{bhramara}}
\app{\lem[wit={ceteri}]{pibati}
  \rdg[wit={P}]{pibaṃti}
  \rdg[wit={B}]{pibatī}}/
%------------------------------
% ekasya phalasya  mālāṃ kāminī tuṃgakucamaṃḍalopari dadhāti/ \E
% ekasya phalasya  mālāṃ kāminī tuṃgakucamaṃḍalopari dadhāti \P
% ekasya phalasya  mālāṃ kāminī tuṃgakucamaṃḍalopari dadhātī// \B
% ekasya phalasya  mālāṃ kāminī tuṃgakucamaṃḍalopari dadhāti// \L
% ekasya phalasya  mālāṃ kāminī tuṃgakucamaṃḍalopari dadhāvati/ \N1 %%%p.7 recto letzte Zeile 
% ekasya phalasya  mālāṃ kāmibī tuṃgakucamaṇḍalopari dadhāti// \D
% ekasya phalasyaṃ mālākāminī   tuṃgakucamaṇḍalopari dadhovati// \N2
% ekasya phalasya  mālāṃ kāmini tuṃ  kucamaṃḍalopari dadhāti \U1
% ekasya phalasya  mālāṃ kāminī tuṃgakucamaṃḍalopari dadhāti// \U2
%------------------------------
% of the one fruit Blütenkranz/Girlande die Verliebte (biene) führt ein unmittelbar über dem Kreis des Blütenstempels der wie eine Brust ist ein.  %tu.mga = hervorstehend 
% Die [nach Blumensaft] Verlangende [Biene] platziert sich auf dem Blütenkranz über dem emportstehenden kreisförmigen Blütenstempel.
%[Or] a woman places a garland [made of] the one fruit above her voluptuous bosom.   
%------------------------------
ekasya
\app{\lem[wit={ceteri}]{phalasya}
  \rdg[wit={N2}]{phalasyaṃ}}
\app{\lem[wit={ceteri}]{mālāṃ}
  \rdg[wit={N2}]{mālā°}}
\app{\lem[wit={ceteri}]{kāminī}
  \rdg[wit={D}]{kāmibī}}
\app{\lem[wit={ceteri},alt={tuṅga°}]{tuṅga}
  \rdg[wit={U1}]{tuṃ°}
}kucamaṃḍalopari
\app{\lem[wit={ceteri}]{dadhāti}
  \rdg[wit={N1}]{dadhāvati}
  \rdg[wit={N2}]{dadhovati}}/
%------------------------------ 
%ekaṃ phalaṃ mṛtamanuṣyopari   kṣipyate/  ayaṃ vastunaḥ svabhāvaḥ/  tathā eka  evātmā   svīyabhāvād evāṣṭau    bhogān  bhunakti/ \E
%ekaṃ phalaṃ mṛtamanuṣyopari   kṣipyate   ayaṃ vastunaḥ svabhāvaḥ   tathā eka  evātmā   svīyabhāvād evāṣṭau    bhogān  bhunakti \P
%ekaṃ phalaṃ mṛtamanuṣyopari   kṣapyate// ayaṃ vastunaḥ svabhāvaḥ/  tathā eka  evātmā   svabhāvād   evāṣṭau    bhogān  bhunakte// \B
%ekaṃ phalaṃ mṛtamanuṣyopari  kṣipyate//  ayaṃ vastunaḥ svabhāvaḥ   tathā eka  evātmā   svabhāvād   evāṣṭau    bhogān  bhunakte// \L
%ekaphalaṃ   mṛtamanuṣyopari   kṣipyate// ayaṃ vastunaḥ svabhāvaḥ/  tathā eka  evātmā   svīyabhāvād evāṣṭau    bhogān ābhunakti/ \N1
%ekaphalaṃ   mṛtamanuṣyopari  kṣipyate//  ayaṃ vastunaḥ svabhāvaḥ// tathā eka  evātmā   svīyabhāvād evāṣṭau    bhogān  bhunakti// \D
%ekaphalaṃ   mṛtamanuṣyopari   kṣipyate/  ayaṃ castunaḥ svabhāvaḥ/  tathā eka  evātmā    svīyabhāvād evāstau   bhogāt  bhunakti/ \N2
%ekaphalaṃ   mṛtamanuṣyopari   kṣipyate/  ayaṃ castunaḥ svabhāvaḥ/  tathā eka  evātmā    svīyabhāvād evāstau   bhogāt  bhunakti/ \U1 %%%276.jpg
%ekaṃ phalaṃ mṛtamanuṣyopari   kṣipyate// ayaṃ vastunaḥ svabhāvaḥ// tathā ekam eva ātmā svīyabhāvād evāṣṭabhogān    bhunakti// \U2
%------------------------------
%[Or] the one fruit is thrown onto a dead man. Dies ist das inhärente Wesen der Sache. So [ist es das inhärente Wesen der Sache] das [auch] eine Selbst aufgrund des eigenen Wesens die acht Genüsse genießt.  
%------------------------------
\note[type=testium, labelb=147, lem={\textbf{Re}}]{PT\textsuperscript{qcr \cdot YSV} (Ed. p. 837): strīpuṃrūpī mahān so hi parasparavimohitaḥ | amanaskaḥ svīyabhāvāt jñānayogī nirākulaḥ | srakcandanādivāmāsu svabhāvād bhogam icchukaḥ |}
\app{\lem[type=emendation, resp=egoscr,alt={ekaṃ phalam}]{ekaṃ phala\skp{m-a}}
  \rdg[wit={B,E,L,P}]{ekaṃ phalaṃ}
  \rdg[wit={X}]{eka°}}
\app{\lem[type=emendation, resp=egoscr,alt={amṛtam}]{\skm{m-a}mṛta\skp{m-a}}
  \rdg[wit={ceteri}]{mṛta°}
}\app{\lem[type=emendation, resp=egoscr,alt={anuṣṇopari}]{\skp{m-a}nuṣṇopari}
  \rdg[wit={ceteri}]{manuṣyopari}}
\app{\lem[wit={ceteri}]{kṣipyate}
  \rdg[wit={B}]{kṣapyate}}/
ayaṃ vastunaḥ svabhāvaḥ/
tathā \app{\lem[wit={ceteri}]{eka}
  \rdg[wit={U2}]{ekam}}
\app{\lem[wit={ceteri}]{evātmā}
  \rdg[wit={U2}]{eva ātmā}}
\app{\lem[wit={ceteri},alt={svīyabhāvād}]{svīyabhāvā\skp{d-e}}
  \rdg[wit={B,L}]{svabhāvād}
}\app{\lem[wit={ceteri},alt={evāṣṭau}]{\skm{d-e}vāṣṭau}
  \rdg[wit={N2,U1}]{evāstau}
  \rdg[wit={U2}]{evāṣṭa}}
\app{\lem[wit={ceteri}]{bhogān}
  \rdg[wit={N2,U1}]{bhogāt}}
\app{\lem[wit={ceteri}]{bhunakti}
  \rdg[wit={N1}]{ābhunakti}}/\textsuperscript{\begin{otherlanguage}{english}\coro{[\lowroman{20}]}\end{otherlanguage}}
%------------------------------
%ke te ṣṭau  bhogāḥ – suvāsaś ca   suvastrañ ca  suśayyā    sunitaṃbinī/       susthānañ cānnapānāni    aṣṭau bhogāś ca dhīmatām/      \E
%ke te ṣṭau  bhogauḥ  suvāsaś ca   suvāsaś   ca  suyyā      sunitāṃbinīḥ//     susthānāś cānpanānp------aṣṭau bhogāś ca dhīmatāṃ 1     \P %%%7643.jpg
%      aṣṭau bhogāḥ   suvāsac ca   suvasaś   ca  suśayyāḥ   sūnitaṃbinī/       susthānaś vānnapānāny----aṣṭau bhogāś cā sudhīmatām//1//\B
%      aṣṭau bhogāḥ   suvāsaś ca   suvāsaś   ca  suśayyāḥ   sūnitaṃbinī//      susthānāś cānnapānāny----aṣṭau bhogāś cā sudhīmatāṃ//1// \L
%ke te ṣṭau  bhogāḥ// suvāyaś ca/                suśayyā    sunitaṃbinī/       susthātāś cātmapanasyā----ṣṭau bhogāḥ    sudhipaṇa\N1
%ke te ṣṭau  bhogāḥ// suvāyaś ca//               suśayyā    sunittaṃbinī//     susthātāś cānmanasyā------ṣṭau bhogāḥ    sudhiṣaṇa \D
%ke te ṣṭau  bhogāḥ   suvāyaś ca                 suśayya    sunitaṃbinī/       susthānāś cānmanasyā------ṣṭau bhogāḥ    sudhiyane \N2
%ke te ṣṭe   bhogā –  suvāsaś ca                 suśayyā ca sunītavinīta       susthātāś cānnapānaḥ syādaṣṭau bhogāḥ   sudhiṣaṇāṃ\U1
%ke te aṣṭau bhogā // suvāsaś ca// suvaṃśaś ca// suśayyā//  sunitaṃbinī//      sudehaṃ// sukhasaṃtānaṃ// abhayādicāṣṭakaṃ//  \U2
%------------------------------
%What are the eight enjoyments? A nice perfume, good clothing, a good bed, a beautiful womna, a nice dwelling, food & drink. Those are the eight enjoyments of the wise. Clothes made from silk.
%------------------------------
\app{\lem[wit={ceteri}]{ke te}
  \rdg[wit={B,L}]{\om}}
\app{\lem[wit={ceteri}]{'ṣṭau}
  \rdg[wit={B,L}]{aṣṭau}
  \rdg[wit={U1}]{ṣṭe}}
\app{\lem[wit={ceteri}]{bhogāḥ}
  \rdg[wit={P}]{bhobauḥ}
  \rdg[wit={U1,U2}]{bhogā}}
\end{prose}
\begin{tlg}
\tl{\app{\lem[wit={ceteri},alt={suvāsaś ca}]{suvāsaś-ca}
  \rdg[wit={B}]{suvāsac ca}}
\app{\lem[wit={E},alt={suvastrañ ca}]{suvastrañ-ca}
  \rdg[wit={U2}]{suvaṃśaś ca}}
\app{\lem[wit={ceteri}]{suśayyā}
  \rdg[wit={U1}]{suśayyā ca}
  \rdg[wit={B,L}]{suśayyāḥ}
  \rdg[wit={P}]{suyyā}}
\app{\lem[wit={ceteri}]{sunitaṃbinī}
  \rdg[wit={P}]{sunitāṃbinīḥ}
  \rdg[wit={U1}]{sunītavinīta}}/}\\
\tl{\app{\lem[wit={E},alt={susthānañ ca}]{susthāna\skp{ñ-cā}}
  \rdg[wit={P,L,N2}]{susthānāś}
  \rdg[wit={D,N1,U1}]{susthātāś}
  \rdg[wit={U2}]{sudehaṃ}
}\app{\lem[wit={L},alt={°ānnapānāny}]{\skm{ñ-cā}nnapānā\skp{ny-a}}
  \rdg[wit={B}]{vānnapānāny}
  \rdg[wit={E}]{cānnapānāni}
  \rdg[wit={P}]{cānpanānp°}
  \rdg[wit={N1}]{cātmapanasyā°}
  \rdg[wit={D,N2}]{cānmanasyā°}
  \rdg[wit={U1}]{cānnapānaḥ syād°}
  \rdg[wit={U2}]{sukhasaṃtānaṃ}
}\app{\lem[wit={E,P},alt={aṣṭau bhogāś ca dhīmatām}]{\skm{ny-a}ṣṭau bhogāś-ca dhīmatām}
  \rdg[wit={B,L}]{aṣṭau bhogāś cā sudhīmatām}
  \rdg[wit={N1}]{ṣṭau bhogāḥ sudhipaṇa°}
  \rdg[wit={D}]{ṣṭau bhogāḥ sudhiṣaṇa°}
  \rdg[wit={U1}]{aṣṭau bhogāḥ sudhiṣaṇāṃ}
  \rdg[wit={N2}]{aṣṭau bhogāḥ}
  \rdg[wit={U2}]{abhayādicāṣṭakaṃ}}\dd{}\begin{otherlanguage}{english}\coro{\uproman{23}.\textsuperscript{\lowroman{22}}}\end{otherlanguage}\hskip-2pt\dd{}}
\end{tlg}
  \end{edition}
  \begin{translation}
    \ekddiv{type=trans}
    \begin{tlate}
      \ldots sometimes mottled, sometimes like various fruit, sometimes like flowers, sometimes like the nectar of immortality, [and that only] because of its inherent being.\textsuperscript{\coro{[\lowroman{10}]}} In this way, the self also takes the form of a human, a bird, a gazelle, an elephant, a Vidyādhara, a Gandharva, a centaur, a great scholar or a great fool, a sick or healthy, an angry or peaceful person, by virtue of its inherent nature. Through Jñānayoga he realizes the emptiness of the mutability of form. Just as the place of origin of the fruit is only one, but the fruit's actions and outcomes are seen as manifold.\\

      One fruit falls onto the ground and becomes dry:\textsuperscript{\coro{[\lowroman{15}]}} A bee drinks the fruit's juice; [or] a woman places a garland made of the fruit over her voluptuous bosom; [or] the fruit is thrown onto a dead person. This is the inherent being of the thing. Thus [in the same way], due to its being, the one self enjoys eight enjoyments."\textsuperscript{\coro{[\lowroman{20}]}}\footnote{The passage appears to describe a single object's multifaceted nature, using fruit as an example. The author suggests that even though the origin of the fruit is singular, the actions and outcomes that can arise from it are many and diverse. This can be seen in the various examples, such as the fruit falling onto the ground, a bee drinking the fruit's juice, a woman adorning herself with a garland made of the fruit, or the fruit being thrown onto a dead person.

The final statement about the self enjoying eight pleasures suggests that just as the fruit can have different outcomes and experiences, the self can also have different experiences and enjoyments. Overall, the passage emphasizes the multifaceted and diverse nature of things and experiences. Additionally, the passage suggests that the eight pleasures are perfectly natural to Rāmacandra and his readership.}
\\ \\
What are the eight pleasures?\footnote{I was not able to allocate the source of the \textit{aṣṭau bhogāḥ} yet. In the \textit{Mānasollāsa} of King Someśvara, one finds the mention of twenty royal \textit{upabhoga}s, which, however, includes all of the mentioned eight pleasures in much greater detail \parencite[5]{manasollasa}. This alludes to the royal lifestyle of Rāmacandra's audience.}  \hfill \break
\paragraph{\uproman{23}\textsuperscript{\lowroman{22}}.} A good perfume, fine clothing, a good bed, a beautiful women, a good dwelling (\textit{susthāna}) food and drink.\footnote{Suprisingly, the verse only gives seven enjoyments. What is lacking in comparison to the list given a little later is the horse.} Those are the eight enjoyments of the wise.\footnote{Right after the list presented on the next page, Rāmacandra teaches that the eight enjoyments cause suffering and attachment. However, the end of the verse with \textit{aṣṭau bhogāś ca dhīmatām}, ``the eight enjoyments of the wise/clever person'' suggests a rather positive connotation.}
    \end{tlate}
  \end{translation}
\end{alignment}
%%%%%%%%%%%%%%%%%%%%%%%%%%%%%%%%%%%%%%%%%%
%%%%%%%%%%%%%%%%%%%%%%%%%%%%%%%%%%%%%%%%%%
%%%%%%%%PAGEBREAK%%%%%%%PAGEBREAK%%%%%%%%%
%%%%%%%%%%%%%%%%%%%%%%%%%%%%%%%%%%%%%%%%%%
%%%%%%%%%%%%%%%%PAGEBREAK%%%%%%%%%%%%%%%%%
%%%%%%%%%%%%%%%%%%%%%%%%%%%%%%%%%%%%%%%%%%
%%%%%%%%PAGEBREAK%%%%%%%PAGEBREAK%%%%%%%%%
%%%%%%%%%%%%%%%%%%%%%%%%%%%%%%%%%%%%%%%%%%
%%%%%%%%%%%%%%%%%%%%%%%%%%%%%%%%%%%%%%%%%%
%%%%%%%%%%%%%%%%%%%%%%%%%%%%%%%%%%%%%%%%%%
%%%%%%%%%%%%%%%%%%%%%%%%%%%%%%%%%%%%%%%%%%
%%%%%%%%PAGEBREAK%%%%%%%PAGEBREAK%%%%%%%%%
%%%%%%%%%%%%%%%%%%%%%%%%%%%%%%%%%%%%%%%%%%
%%%%%%%%%%%%%%%%PAGEBREAK%%%%%%%%%%%%%%%%%
%%%%%%%%%%%%%%%%%%%%%%%%%%%%%%%%%%%%%%%%%%
%%%%%%%%PAGEBREAK%%%%%%%PAGEBREAK%%%%%%%%%
%%%%%%%%%%%%%%%%%%%%%%%%%%%%%%%%%%%%%%%%%%
%%%%%%%%%%%%%%%%%%%%%%%%%%%%%%%%%%%%%%%%%%
%%%%%%%%%%%%%%%%%%%%%%%%%%%%%%%%%%%%%%%%%%
%%%%%%%%%%%%%%%%%%%%%%%%%%%%%%%%%%%%%%%%%%
%%%%%%%%PAGEBREAK%%%%%%%PAGEBREAK%%%%%%%%%
%%%%%%%%%%%%%%%%%%%%%%%%%%%%%%%%%%%%%%%%%%
%%%%%%%%%%%%%%%%PAGEBREAK%%%%%%%%%%%%%%%%%
%%%%%%%%%%%%%%%%%%%%%%%%%%%%%%%%%%%%%%%%%%
%%%%%%%%PAGEBREAK%%%%%%%PAGEBREAK%%%%%%%%%
%%%%%%%%%%%%%%%%%%%%%%%%%%%%%%%%%%%%%%%%%%
%%%%%%%%%%%%%%%%%%%%%%%%%%%%%%%%%%%%%%%%%%
\begin{alignment}[
  texts=edition[class="edition"];
  translation[class="translation"],
  ]
  \begin{edition}
    \ekddiv{type=ed}
    \begin{prose}
%------------------------------
%padṛsūtramayāni vasrāṇi//  \E
%padasūtramayāni vastrāṇi?? \P %%%7643.jpg
%paṭasūtrāmayāni vasrāṇi//  \B
%paṭasūtrāmayāni vastrāṇi// \L
%padṛsūtrayāni   vasrāṇi    \N1
%padṛsūtrayāni   vasrāṇi    \D
%padṛsūtrayāni   vasrāṇi    \N2
%padṛsūtramayāni vasrāṇi    \U1
%paṭasūtramayāni vasrāṇi    \U2
%------------------------------
%Clothes made from silk,...
%------------------------------
%paṃcasaptā dṛālikā         yuktāni harmyāṇi teṣu vāsaḥ    ativipulā  mṛdutarasukhāsuśayyā/     \E
%paṃcasaptā dadhikā         yuktāni harmyāṇi teṣu cāsaḥ 2  ativipulā  mṛduttarachadavatīśayyā 2  \P
%paṃcasatyā dātikā          yuktāni harmyāṇi teṣu vāstu    ativipulā  mṛdutaralāśayyā//2//        \B
%paṃcasatyā dātikā          yuktāni harmyāṇi teṣu vāstu    ativipulā  mṛdutaralāśayyā//3//        \L
%paṃca vā sapta vā dṛālikā  yuktāni harmyāṇi/              ativapulā  mṛdu/uttaracchaṃdavatīśayyā/  \N1
%paṃca vā sapta vā dṛāṃlikā yuktāni harmyāṇi               ativapulā  mṛduuttarachaṃdavatīśayyā/     \D
%paṃca vā sapta vā tālikā---yuktāni harmyāni               ativipulā  mṛduuttarachaṃdavatīśayyā    \N2
%paṃca vā sapta vā dālikā---yuktāni harmyāṇi               ativipulāṃ mṛduuttarachadavatiśaiyyā     \U1
%--------------------------saudhāni harmyāṇi vāsāya kecit// aṣṭau bhogān āha// sugrahaṃ// suvastraṃ// suśayā sustrī//  \U2
%--------------------------------------------
%,a site of the palace in which there are mainsions endowned with five or seven rooms. A huge, very soft and lovely bed.  
%-------------------------------------------
      \noindent
      \note[type=source, labelb=147a, lem={\textbf{Re}}]{PT\textsuperscript{qcr \cdot YSV} (Ed. p. 837): ātmāvivekam āgamya calac cittaṃ mahākulam | viṣayāndhatamo dṛṣṭvā no vetti paramātmanaḥ | amāyātmā tattvātītaḥ satsandhānavivarjitaḥ | sukhī duḥkhī janma mṛtyuṃ yāti satyaṃ punaḥ punaḥ | vairāgyādidhanaṃ tyaktvā viṣavad duḥkhakṛddhiyaḥ | koṭisūryasamātmeti jñānayogād vimucyate |}
\app{\lem[type=emendation, resp=egoscr,alt={paṭṭa°}]{paṭṭa}
  \rdg[wit={B,L,U2}]{paṭa°}
  \rdg[wit={X,E}]{padṛ°}
  \rdg[wit={P}]{pada°}
}\app{\lem[wit={ceteri},alt={sūtra°}]{sūtra}
  \rdg[wit={B,L}]{sūtrā}
}\app{\lem[wit={ceteri}]{mayāni}
  \rdg[wit={D,N1,N2}]{yāni}}
\app{\lem[wit={P,L}]{vastrāṇi}
  \rdg[wit={ceteri}]{vasrāṇi}} 1\dd{}
\app{\lem[wit={X}]{paṃca vā sapta vā}
  \rdg[wit={E,P}]{paṃcasaptā}
  \rdg[wit={L,B}]{paṃcasatyā}}
\app{\lem[type=emendation, resp=egoscr]{śālikā}
  \rdg[wit={E,N1}]{dṛālikā}
  \rdg[wit={D}]{dṛāṃlikā}
  \rdg[wit={P}]{dadhikā}
  \rdg[wit={B,L}]{dātikā}
  \rdg[wit={N2}]{tālikā}
  \rdg[wit={U1}]{dālikā}
}\app{\lem[wit={ceteri}]{yuktāni}
  \rdg[wit={U2}]{saudhāni}}
harmyāṇi
\app{\lem[wit={L,B}]{teṣu vāstu}
  \rdg[wit={E}]{teṣu vāsaḥ}
  \rdg[wit={P}]{teṣu cāsaḥ}
  \rdg[wit={U2}]{vāsāya kecit}
  \rdg[wit={ceteri}]{\om}} 2\dd{}
\app{\lem[wit={ceteri}]{ativipulā}
  \rdg[wit={D,N1}]{ativapulā}
  \rdg[wit={U1}]{ativipulāṃ}
  \rdg[wit={U2}]{aṣṭau bhogān āha ||}}
\app{\lem[type=emendation, resp=egoscr,alt={mṛdūttara}]{mṛdūttara}
  \rdg[wit={B,E,L,P}]{mṛdutara°}
  \rdg[wit={X}]{mṛdu | uttara°}
  \rdg[wit={U2}]{sugrahaṃ ||}
}\app{\lem[wit={D,N1,N2},alt={°chandavatī°}]{chandavatī}
  \rdg[wit={P}]{°chadavatī°}
  \rdg[wit={U1}]{°chadavati°}
  \rdg[wit={U2}]{suvastraṃ ||}
}\app{\lem[wit={ceteri}]{śayyā}
  \rdg[wit={U2}]{suśayā sustrī}} 3\dd{}
%------------------------------
%padminī tārūṇyavatī  manoharā guṇavatī  tatropaviṣṭā kāṃtā/      \E
%padminī tārūṇyavatī  manoharā guṇavatī  tatopaviṣṭā  kāṃtā 4     \P
%padminī tārūnyavatī  manoharā guṇavatī//tatrāpavistā kāṃtā 4     \B
%padminī tārūnyavatī  manoharā guṇavatī//tatropavistā kāṃtā// 4// \L
%padmanī tārūṇyavatī  manoharā guṇavatī  tatropavistā//           \N1
%padminī tārūrāyavatī manoharā guṇavatī  tatropavistā//           \D
%padminī tārūnyavatī  manoharā guṇavatī  tatropavistā             \N2
%padminī tārūnyavati  manoharā guṇavati  tatropavistā             \U1
%                                                                 \U2
%--------------------------------------------
%[On which] there is situated [tatropaviṣṭā] a lotus-like [em. zu tāruṇyavatī] youthful, charming and virtuous wife.
%-------------------------------------------
\app{\lem[wit={ceteri}]{padminī}
  \rdg[wit={N1}]{padmanī}
  \rdg[wit={U2}]{\om}}
\app{\lem[type=emendation, resp=egoscr]{tāruṇyavatī}
  \rdg[wit={ceteri}]{tārūṇyavatī}
  \rdg[wit={N2}]{tārūrāyavatī}
  \rdg[wit={U2}]{\om}}
\app{\lem[wit={ceteri}]{manoharā guṇavatī}
  \rdg[wit={U2}]{\om}}
\app{\lem[wit={ceteri}]{tatropavistā}
   \rdg[wit={P}]{tatopavistā}
  \rdg[wit={B}]{tatrāpavistā}
  \rdg[wit={U2}]{\om}}
\app{\lem[wit={B,E,L,P}]{kāntā}
  \rdg[wit={ceteri}]{\om}} 4\dd{}
%------------------------------
%sādhu āśanam/      atimūlyañ ca/         manoramam annaṃ।       tathā vidhaṃ pānam/   \E
%sādhu āsanaṃ 5     atimūlo 'śvaḥ 6       manoramam annaṃ    7   tathā vidhaṃ pānaṃ 8  \P
%sādhu āsanaṃ 5     atimūlyo asvaṃ//6     manoramyam attaṃ //7   tathā vidhapānaṃ//8   \B
%sādhu āsanaṃ// 5// atimūlyo aśvaṃ//6//   manoramyam annaṃ //7// tathā vidhapānaṃ//8// \L
%sādhyāsanaṃ//      amūlyo svaś ca//      manoramam attaṃ        tathā vidhaṃ pānaṃ/   \N1
%sādhyāsanaṃ//      amūlyo svaś ca//      manoramam attaṃ        tathā vidhaṃ pānaṃ//  \D
%sādhyāsanaṃ        amūlyo svaś ca        manotamam annaṃ        tathā vidhapānaṃ//    \N2
%sādhyāsanaṃ        amolyo svaś ca        manoramam annaṃ        tathā vidhaṃ pānaṃ    \U1
%sādhu āsanaṃ//           suśvaḥ//        suṣṭu annaṃ//          tathā vidhayānaṃ//    \U2
%--------------------------------------------
%good throne/seat; atimūlyo (überaus wertvolles) 'śvaṃ (Pferd), manorama ( die Sinne erfreuendes) Essen, verschiedenes Trinken. 
%-------------------------------------------
\app{\lem[type=emendation, resp=egoscr]{sādhvāsanaṃ}
  \rdg[wit={E}]{sādhu āśanam}
  \rdg[wit={B,L,P,U2}]{sādhu āsanaṃ}
  \rdg[wit={D,N1,N2}]{sādhyāsanaṃ}} 5\dd{}
\app{\lem[type=emendation, resp=egoscr]{atimūlyo 'śvaḥ}
  \rdg[wit={E}]{atimūlyañ ca}
  \rdg[wit={P}]{atimūlo 'śvaḥ}
  \rdg[wit={B,L}]{atimūlyo asvaṃ}
  \rdg[wit={X}]{amūlyo svaś ca}
  \rdg[wit={U2}]{suśvaḥ}} 6\dd{}
\app{\lem[wit={ceteri},alt={manoramam annaṃ}]{manoramam-annaṃ}
  \rdg[wit={B}]{manoramyam attaṃ}
  \rdg[wit={L}]{manoramyam annaṃ}
  \rdg[wit={D,N1}]{manoramam attaṃ}
  \rdg[wit={U2}]{suṣṭu annaṃ}} 7\dd{}
tathā
\app{\lem[wit={ceteri}]{vidhaṃ pānaṃ}
  \rdg[wit={B,L,N2}]{vidhapānaṃ}
  \rdg[wit={U2}]{vidhayānaṃ}} 8\dd{}
%------------------------------
%ete   ṣṭau bhogāḥ   kathitāḥ/   eke  duḥkhaṃ   bhajante/  bhikṣāṃ  yācante// kiñca \E
%ete   ṣṭau bhogāḥ   kathitā 9   eke  duḥkha    bhajaṃte   bhikṣāṃ  yāṃcaṃte ca  \P
%ete   ṣṭau bhogāḥ//             eka  duḥkhā    bhajaṃte/  bhikṣā   yāṃcate ca//  \B
%ete   ṣṭau bhogāḥ//             eka  duḥkhā    bhajaṃte// bhikṣā   yāṃcate ca//  \L
%ete  aṣṭau bhogā    kathyate/   eke  duḥkhaṃ   bhajaṃte/  bhikṣyāṃ yācate ca/   \N1
%ete  aṣṭau bhogāḥ   kathyaṃte// ete  duḥkhaṃ   bhajaṃte/  bhikṣyāṃ yācaṃte ca// \D
%ete  aṣṭau ghogā    kathyate//  ete  duḥkhataṃ bhajate    bhikṣāṃ  yācate ca//  \N2
%rāte aṣṭau bhogāḥ   kathyate    ete  duḥkhaṃ   bhajate    bhikṣāṃ  pācate ca    \U1
%ete  ṣṭau  bhogāḥ// kathitāḥ//  ekaṃ duḥkhaṃ   bhajaṃte// bhikṣā   yācaṃte ca// \U2
%------------------------------
%The eight enjoyments are being described. They impart suffering, and [make one] begging for their sustenance.
%------------------------------
%\note[type=philcomm, labelb=148, lem={'ṣṭau bhogāḥ}]{The eight enjoyments are not attested in any of the sources.}
\app{\lem[wit={ceteri}]{ete}
  \rdg[wit={U1}]{rāte}}
\app{\lem[wit={Y}]{'ṣṭau}
  \rdg[wit={X}]{aṣṭau}}
\app{\lem[wit={ceteri}]{bhogāḥ}
  \rdg[wit={N1,N2}]{bhogā}
  \rdg[wit={U1}]{ghogā}}
\app{\lem[wit={E,U2}]{kathitāḥ}
  \rdg[wit={P}]{kathitā}
  \rdg[wit={N1,N2,U1}]{kathyate}
  \rdg[wit={D}]{kathyaṃte}
  \rdg[wit={B,L}]{\om}}/
\app{\lem[wit={D,N2,U1}]{ete}
  \rdg[wit={E,P,N1}]{eke}
  \rdg[wit={B,L}]{eka}
  \rdg[wit={U2}]{ekaṃ}}
\app{\lem[wit={D,E,N1,U1,U2}]{duḥkhaṃ}
  \rdg[wit={P}]{duḥkha}
  \rdg[wit={B,L}]{duḥkhā}
  \rdg[wit={N2}]{duḥkhataṃ}}
\app{\lem[wit={ceteri}]{bhajante}
  \rdg[wit={N2,U1}]{bhajate}}/
\app{\lem[wit={E,P,N2,U1}]{bhikṣāṃ}
  \rdg[wit={D,N1}]{bhikṣyāṃ}
  \rdg[wit={B,L,U2}]{bhikṣā}}
\app{\lem[wit={ceteri}]{yācante}
  \rdg[wit={P}]{yāṃcaṃte}
  \rdg[wit={B,L}]{yāṃcate}
  \rdg[wit={N2}]{yācate}
  \rdg[wit={U1}]{pācate}}
\app{\lem[wit={ceteri}]{ca}
  \rdg[wit={E}]{kiñca}}/\textsuperscript{\begin{otherlanguage}{english}\coro{[\lowroman{25}]}\end{otherlanguage}}
%------------------------------
%      yathā sūryasya tejaḥ   dugdhasya    ghṛtam   agner jvalanaṃ viṣān mūrchā   tilāttailam/    vṛkṣāc-chāyā/  phalāt parimalaḥ       kāṣṭhād agniḥ    arkarādibhyo   madhuro rasaḥ/ \E
%      yathā sūryasys tejaḥ   dugdhasya    ghṛtaḥ   agne dvāhaḥ    viṣān mūrchāti tilāttailaṃ     vṛkṣāt-chāyā   phalāsarimalaḥ         kāṣṭād  agniḥ    śarkvarādibhyo madhuro rasaḥ  \P
%      yathā sūryasye tejāḥ   dugdha-------ghṛtaḥ   agne dvāhaḥ//  viṣān mūrchā   tilāttailaṃ//   vṛkṣā--chāyā   phalāsarimalaḥ         kaṣṭād  agniḥ    śarkadībhyo    madhuro  \B
%      yathā sūryasya tejāḥ   dugdha-------ghṛtaḥ   agne dvāhaḥ//  viṣān mūrchā   tilātailaṃ//    vṛkṣā--chāyā   phalāt parimalaḥ       kaṣṭād  agniḥ    śarkadībhyo    madhuro  \L
%      yathā sūryasya tejaḥ/  dugdhasya    ghṛtaṃ/  agne dahiḥ??   viṣān mūrchā   tilāttailaṃ,    vṛkṣāc-chāyā/  phalāt parimalaḥ/      kāṣṭhād āgniḥ/   śarkkarādibhyo madhuro rasaḥ/ \N1
%      yathā sūryasya tejaḥ// dugdhasya    ghṛtaṃ// agne dadhiḥ    viṣān mūrchā   tilāttailaṃ//   vṛkṣā--chāyā// phalāt palātparimalaḥ//kāṣṭhād āgniḥ//  śarkarādibhyo  madhuro rasaḥ/ \D
%      yathā sūryasya tejaḥ   dusya        ghṛtaṃ   agne dadhi     viṣān mūrchā   tilatailaṃ      vṛkṣā--chāyā   phalāt parimalaḥ       kāṣṭhād āgniḥ    śarkarādibhyo  madhuro rasaḥ/ \N2
%      yathā sūryaśca tejaḥ   dugdhasy     ghṛttaṃ  agne dārhaṃ    viṣāt mūrchā   tilātailaṃ      vrakṣā-chāyā   phalāt parimalaḥ       kāṣṭhād āgniḥ    śarkarādibhyo  madhuro rasaḥ \U1
%      yathā sūryasya tejaḥ// dugdhasya    ghṛtaṃ// agne dāhiḥ//   viṣān mūrchā   tilātailaṃ//    vṛkṣā--chāyā// phalāt parimalaḥ//     kāṣṭād  agniḥ    śarkarādibhyo  madhuro rasaḥ// \U2
%------------------------------
%Gleichwie die Strahlen der Sonne, die Butter der Milch, das Brennen des Feuers, die Betäubung aufgrund von Gift, das Sesamöl aus dem Sesamkorn, der Schatten vom Baum, der Wohlgeruch von einer Frucht, das Feuer von einem Holzscheid, der Süße Saft [em. zu śārkara] a liquor prepared from Dhātakī with sugar] und so weiter,   
%------------------------------
%Like the rays of the sun, the butter of milk, the burning of fire, the stupor of poison, the sesame oil from the sesame seed, the shade from the tree, the sweet odor from a fruit, the fire from a scabbard, the sweet sap [em . to śārkara] a liquor prepared from Dhātakī with sugar] and so on,
%------------------------------
\note[type=source, labelb=149, lem={\textbf{Re}}]{PT\textsuperscript{qcr \cdot YSV} (Ed. p. 837): ravī tejo ghṛtaṃ dugdhe tile tailaṃ svabhāvataḥ | śaśam indau kule śākhaṃ kṣāre ca lavaṇaṃ yathā | tathā brahmaṇi saṃsāro hyakhaṇḍaparipūrvake |}
yathā
\app{\lem[wit={ceteri}]{sūryasya}
  \rdg[wit={U1}]{sūryaś ca}}
\app{\lem[wit={ceteri}]{tejaḥ}
  \rdg[wit={B,L}]{tejāḥ}}\dd{}
\app{\lem[wit={D,E,P,N1,U2}]{dugdhasya}
  \rdg[wit={B,L}]{dugdha°}
  \rdg[wit={N2}]{dusya}
  \rdg[wit={U1}]{dugdhasy}}
\app{\lem[wit={ceteri}]{ghṛtaṃ}
  \rdg[wit={B,L,P}]{ghṛtaḥ}}\dd{}
\app{\lem[wit={E}, alt={agner}]{agne\skp{r-dā}}
  \rdg[wit={ceteri}]{agne}
}\app{\lem[type=emendation, resp=egoscr, alt={dāhaḥ}]{\skm{r-dā}haḥ}
  \rdg[wit={B,L,P}]{dvāhaḥ}
  \rdg[wit={N1}]{dahiḥ}
  \rdg[wit={N2}]{dadhi}
  \rdg[wit={D}]{dadhiḥ}
  \rdg[wit={U1}]{dārhaṃ}
  \rdg[wit={U2}]{dāhiḥ}
  \rdg[wit={E}]{jvalanaṃ}}\dd{}
\app{\lem[wit={ceteri},alt={viṣān}]{viṣā\skp{n-mū}}
  \rdg[wit={U1}]{viṣāt}
}\skm{n-mū}rchā\dd{}
\app{\lem[wit={ceteri},alt={tilāt}]{tilā\skp{t-tai}}
  \rdg[wit={P}]{titilāt}
  \rdg[wit={N2}]{tila}
  \rdg[wit={U1}]{tilā}
}\skm{t-tai}laṃ\dd{}
\app{\lem[wit={E,N1}, alt={vṛkṣāt}]{vṛkṣā\skp{c-chā}}
  \rdg[wit={P}]{vṛkṣāt}
  \rdg[wit={B,D,L,N2,U2}]{vṛkṣā}
  \rdg[wit={U1}]{vrakṣā}
}\skm{c-chā}yā\dd{}
\app{\lem[wit={ceteri},alt={phalāt}]{phalā\skp{t-pa}}
  \rdg[wit={B,L}]{phalā}
}\app{\lem[wit={ceteri},alt={parimalaḥ}]{\skm{t-pa}rimalaḥ}
  \rdg[wit={B,L}]{sarimalaḥ}
  \rdg[wit={D}]{palāt parimalaḥ}}\dd{}i%\note[type=philcomm, labelb=150, lem={parimalaḥ}]{Clarification: Witness \getsiglum{D} reads \textit{phalāt palāt parimala}.}
\app{\lem[wit={ceteri}, alt={kāṣṭhād}]{kāṣṭhā\skp{d-a}}
  \rdg[wit={P,U2}]{kāṣṭād}
  \rdg[wit={B,L}]{kaṣṭād}
}\app{\lem[wit={ceteri}, alt={agniḥ}]{\skm{d-a}gniḥ}
  \rdg[wit={D,N1,N2,U1}]{āgniḥ}}\dd{}
\app{\lem[type=emendation, resp=egoscr]{śārkarādibhyo}
  \rdg[wit={E}]{arkarādibhyo}
  \rdg[wit={P}]{śarkvarādibhyo}
  \rdg[wit={L,B}]{śarkadībhyo}}
madhuro
\app{\lem[wit={ceteri}]{rasaḥ}
  \rdg[wit={B,L}]{\om}}\dd{}
%------------------------------
%himānībhyaḥ   śītam      ityādipadārthānāṃ svabhāvaḥ         tathā    saṃsāro'pi parameśvarasvarūpamadhye      tiṣṭhati/ \E
%himānībhyaḥ   śītaṃ      ityādipadārthasvabhāva        eva   tathā    saṃsāro'pi parameśvarasvarūpamadhye      tiṣṭhati \P
%sahīmānībhyaḥ śītaḥ/     ityādipadārthāsvabhāvataḥ// eva     tathā    saṃsāro pi paremesvara svarūpasya madhye tiṣṭhatī/ \B
%sahimānibhyaḥ śītaḥ//    ityādiphadārthāḥ svabhāvataḥ// eva  tathā    saṃsāro pi paremesvara svarūpasya madhye tiṣṭhati// \L
%himānibhyaḥ   śaityāṃ    ityādipadārthasvabhāva evā/         tathā    saṃsāro pi parameśvarasvarūpamadhye      tiṣṭhati// \N1
%himānibhyaḥ   śaityaṃ // ityādipadārthasvabhāva eva//        tathā    saṃsāro pi parameśvarasvarūpamadhye      tiṣṭhati// \D 
%himānitpa     śaityāś    atyādipadārtharthasvabhāva eva//    tathā    saṃsāro pi parameśvarasvarūpamadhye      tiṣṭhati \N2
%himānībhyaḥ   śaityaṃ    ityādipadārthasvabhāvaḥ ravaḥ?      tathā vā saṃsāro pi parameśvararūpamadhye         tiṣṭhati/ \U1
%himānībhyaḥ   śītyaṃ//   ityādipadārthāsvabhāva eva//        tathā    saṃsāro pi parameśvarasvarūpamadhye      tiṣṭhaṃti// \U2
%------------------------------
%die Kälte von Schneehaufen, und so weiter ist das inhärente Wesen der Dinge. IN gleicher Weise befindet sich auch der Weltengang im Zentrum der eigenen Gestalt von höchsten Gott.
%the cold of piles of snow, and so on is the inherent essence of things. In the same way, the course of the world is also in the center of the highest God's own form. 
%------------------------------
\app{\lem[wit={ceteri}]{himānībhyaḥ}
  \rdg[wit={B,L}]{sahimānibhyaḥ}
  \rdg[wit={N2}]{himānitpa}}
\app{\lem[wit={D,U1}]{śaityaṃ}
  \rdg[wit={N1}]{śaityāṃ}
  \rdg[wit={U2}]{śītyaṃ}
  \rdg[wit={N2}]{śaityāś}
  \rdg[wit={E,P}]{śītaṃ}
  \rdg[wit={B,L}]{śītaḥ}}\dd{}
\app{\lem[wit={D,N1,P}]{ityādipadārthasvabhāva}
  \rdg[wit={U2}]{ityādipadārthā°}
  \rdg[wit={B}]{ityādipadārthāsvabhāvataḥ}
  \rdg[wit={N2}]{atyādipadārtharthasvabhāva}
  \rdg[wit={U1}]{ityādisvabhāvaḥ}
  \rdg[wit={L}]{ityādiphadārthāḥ svabhāvataḥ}
  \rdg[wit={E}]{ityādipadārthānāṃ svabhāvaḥ}}
\app{\lem[wit={ceteri}]{eva}
  \rdg[wit={N1}]{evā}
  \rdg[wit={U1}]{ravaḥ}
  \rdg[wit={E}]{\om}}\dd{}
\app{\lem[wit={ceteri}]{tathā}
  \rdg[wit={U1}]{tathā vā}}
saṃsāro'pi
\app{\lem[wit={ceteri}]{parameśvarasvarūpamadhye}
  \rdg[wit={B,L}]{paremesvara svarūpasya madhye}
  \rdg[wit={U1}]{parameśvararūpamadhye}}
\app{\lem[wit={ceteri}]{tiṣṭhati}
  \rdg[wit={B}]{tiṣṭhatī}
  \rdg[wit={U2}]{tiṣṭhaṃti}}\dd{}
%------------------------------
%parameśvaro 'khaṇḍa--paripūrṇaḥ/  \E
%parameśvaro khaṃḍa---paripūrṇaś ca    \P
%parameśvaro khaṃḍa---paripūrṇaś ca// \B
%parameśvaro khaṃḍa---paripūrṇaś ca//  \L
%parameśvaro 'ṣaṃḍa---paripūrṇaś ca//  \N1
%parameśvaro  ṣaṃḍa---paripūrṇaś ca//  \D %%%S.9 verso
%parameśvaro yarāṇḍa--paripūrṇaś ca//  \N2
%parameśvaro khaṃḍaḥ  paripūrṇaś ca   \U1 %%%277.jpg
%parameśvaro 'khaṃḍa--paripūrṇaś ca//   \U2
%------------------------------
%Und der höchste Gott ist unteilbar und das All erfüllend.
%And the Most High God is indivisible and all-filling.
%------------------------------
parameśvaro\app{\lem[wit={ceteri}, alt={'khaṇḍa°}]{'khaṇda}
  \rdg[wit={D,N1}]{'ṣaṃḍa°}
  \rdg[wit={N2}]{yarānda°}
  \rdg[wit={U1}]{khaṃḍaḥ}
}\app{\lem[wit={ceteri},alt={°paripūrṇaś ca}]{paripūrṇaś\skp{-}ca}
  \rdg[wit={E}]{paripūrṇaḥ}}\dd{}\textsuperscript{\begin{otherlanguage}{english}\coro{[\lowroman{30}]}\end{otherlanguage}}\vfill
\end{prose}
  \end{edition}
  \begin{translation}
    \ekddiv{type=trans}
    \begin{tlate}
      \noindent
  1. Clothes made from silk;\footnote{Within the twenty \textit{upabhoga}s of the \textit{Mānasollāsa} we find the interesting topic of \textit{Vastropabhoga} or the ``enjoyment of garments''. Particularly in summer, the king is asked to wear silk or cotton clothes which are thin and charming (\citeauthor[1939:14]{manasollasa}).} 2. A site of the palace in which there are mainsions endowned with five or seven rooms.\footnote{The first \textit{adhyāya} of the third \textit{viṃśati} of the \textit{Mānasollāsa} discusses astrology for finding out auspicious moments while building new houses for princes. Here houses with one to four \textit{śālā}s are described (\citeauthor[1939:6-7]{manasollasa}).} 3. A huge, very soft and lovely bed;\footnote{This is found as \textit{Śayyābhoga} within the \textit{Mānasollāsa}. In this section, seven kinds of beds and eight kinds of bed-steads are described (\citeauthor[1939:21]{manasollasa}).} 4. [on which] there is seated a lotus-like youthful, charming and virtuous wife;\footnote{This is resembled as \textit{yosidupabhoga} within the \textit{Mānasollāsa}. In this chapter, King Someśvara describes the qualifications of women whom a king should marry. The two most important qualities he gives are beauty and full youth. Out of the four kinds of women: a) Padminī, b) Citriṇī, c) Śaṅkhinī, and d) Hastinī. He suggests that the latter two kinds are not worth enjoying (\parencite[21]{manasollasa}).} 5. An excellent throne;\footnote{The \textit{āsanopabhoga} or ``the enjoyment of seats'' within the \textit{Mānasollāsa} describe various kinds of royal seats (\citeauthor[1939:15]{manasollasa}).} 6. An exceptional valuable horse;\footnote{This is resembled as \textit{yānopabhoga} within the \textit{Mānasollāsa}. In this chapter, King Someśvara lists nine kinds of vehicles, including horses (\citeauthor[1939:24]{manasollasa}).} 7. Food that pleases the senses;\footnote{This is resembled as \textit{annabhoga} within the \textit{Mānasollāsa}. In this section, King Someśvara describes the names of various kinds of great food and the directions as to the preparations of various dishes (\citeauthor[1939:21]{manasollasa}).} 8. Various drinks.\footnote{This is resembled as \textit{pānīyabhoga} within the \textit{Mānasollāsa}. In this chapter everything related to drinking and drinks is described (\citeauthor[1939:23]{manasollasa}). Instead of following his template closely, Rāmacandra modifies the meaning of the passage. The original ideal of his sources text which emphasizes renunciation and detachment to wealth is weakened by the introduction of the eight pleasures. Instead of radically abandoning wealth, he just states that they cause suffering and craving, as we can \ldots}\hfill \break 
The eight enjoyments are described. They impart suffering, and [make one] begging for their sustenance.\textsuperscript{\coro{[\lowroman{25}]}}
Like the rays of the sun, the butter of milk, the burning of fire, the stupor of poison, the sesame oil from the sesame seed, the shade from the tree, the sweet odor from a fruit, the fire from a scabbard, the sweet sap of Śārkara\footnote{A liquor prepared from Dhātakī with sugar. Beleg?!} and so on, the cold of piles of snow, and so on is the inherent essence of things. In the same way, the course of the world is also in the center of the highest God's own form. And the highest God is indivisible and all-filling.\textsuperscript{\coro{[\lowroman{30}]}}
    \end{tlate}
  \end{translation}
\end{alignment}
\ekdpb*{}
%%%%%%%%%%%%%%%%%%%%%%%%%%%%%%%%%%%%%%%%%%
%%%%%%%%%%%%%%%%%%%%%%%%%%%%%%%%%%%%%%%%%%
%%%%%%%%PAGEBREAK%%%%%%%PAGEBREAK%%%%%%%%%
%%%%%%%%%%%%%%%%%%%%%%%%%%%%%%%%%%%%%%%%%%
%%%%%%%%%%%%%%%%PAGEBREAK%%%%%%%%%%%%%%%%%
%%%%%%%%%%%%%%%%%%%%%%%%%%%%%%%%%%%%%%%%%%
%%%%%%%%PAGEBREAK%%%%%%%PAGEBREAK%%%%%%%%%
%%%%%%%%%%%%%%%%%%%%%%%%%%%%%%%%%%%%%%%%%%
%%%%%%%%%%%%%%%%%%%%%%%%%%%%%%%%%%%%%%%%%%
%%%%%%%%%%%%%%%%%%%%%%%%%%%%%%%%%%%%%%%%%%
%%%%%%%%%%%%%%%%%%%%%%%%%%%%%%%%%%%%%%%%%%
%%%%%%%%PAGEBREAK%%%%%%%PAGEBREAK%%%%%%%%%
%%%%%%%%%%%%%%%%%%%%%%%%%%%%%%%%%%%%%%%%%%
%%%%%%%%%%%%%%%%PAGEBREAK%%%%%%%%%%%%%%%%%
%%%%%%%%%%%%%%%%%%%%%%%%%%%%%%%%%%%%%%%%%%
%%%%%%%%PAGEBREAK%%%%%%%PAGEBREAK%%%%%%%%%
%%%%%%%%%%%%%%%%%%%%%%%%%%%%%%%%%%%%%%%%%%
%%%%%%%%%%%%%%%%%%%%%%%%%%%%%%%%%%%%%%%%%%
%%%%%%%%%%%%%%%%%%%%%%%%%%%%%%%%%%%%%%%%%%
%%%%%%%%%%%%%%%%%%%%%%%%%%%%%%%%%%%%%%%%%%
%%%%%%%%PAGEBREAK%%%%%%%PAGEBREAK%%%%%%%%%
%%%%%%%%%%%%%%%%%%%%%%%%%%%%%%%%%%%%%%%%%%
%%%%%%%%%%%%%%%%PAGEBREAK%%%%%%%%%%%%%%%%%
%%%%%%%%%%%%%%%%%%%%%%%%%%%%%%%%%%%%%%%%%%
%%%%%%%%PAGEBREAK%%%%%%%PAGEBREAK%%%%%%%%%
%%%%%%%%%%%%%%%%%%%%%%%%%%%%%%%%%%%%%%%%%%
%%%%%%%%%%%%%%%%%%%%%%%%%%%%%%%%%%%%%%%%%%