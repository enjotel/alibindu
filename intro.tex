%Ultimatives Tool zur Datierung:
%https://www.cc.kyoto-su.ac.jp/~yanom/pancanga/
%skp = ignored in edition
%skm = ignored in xml
%%%---2-DO---%%%:
% - add xml ids for cladistics
% - produce diplomatic transcripts for saktumiva
% - read Sarvangayogapradipika, Maya Burger! 
% - maybe add second ciritical edition of yogasvarodaya?!
% - grep-search alle Verse!!!!
% - Mss spreadsheet
% - additions to U2: make footnotes for the bahir mātrā-s: explaining the inventions of female deities and tell that this is "schwer interpretierbar"
% - Consider changing Lakṣya to Lakṣa
%%%%%%%%%%%%%%%%%%%%%%%%%%%%%%%%%%%%%%%%%
% Don't forget
% Siddhasiddhantapaddhati Yogic Body descriptions are followed by Rāmacandra
% Quotes of the Yogasvarodaya in the Yoga Karṇikā
% Rāmacandra more a compiler than an author!!!
% Identify quotes of YTB in Haṭhasanketacandrikā -- done :D
%%%%%%%%%%%%%%%%%%%%%%%%%%%%%%%%%%%%%%%%%%%
%MSS notes
%
%--B: i and ī are not differenciated
%--P: no punctuation no daṇdas nothing
%--U1: dot . serves as daṇḍa 
%--\L and \U2 very similar
%--figure out for U2: // ajapājapaḥ sahasra // 6000 //gha 0 16 pa 0 40// \U2?!?!?!?!?!?
%%%%%%%%%%%%%%%%%%%%%%%%%%%%%%%%%%%%%%%%%%
%
% Einleitung Ideen 
% - sprachliche Simplizität
% - Potenzial als Anfängertext
% - Großartige Einführung in die Textkritik -> Synoptische Edition 
% - Gelegenheit Yogasvarodaya und Yogatattvabindu zu edieren 
% - Historische Evidenz entweder für das königliche Leben in einer Maṭha in der Nähe von Benares während der Muslimischen Herrschaft, oder sogar Lehrtext für die Bildung junger Prinzen  
% - eines der raren Beispiele der engen Verknüfung mehrerer Texte 
% - eines der raren Beispiele der Prosaisierung eines metrischen Textes 
% - Anwendung rezenter Technologie! 
% - How the text was construed -> intermingling of Ysv and SSP
% - Martin Straube: "jeder kleine Dorfhäuptling kann Rāja genannt werden". 
%%%%%%%%%%%%%%%%%%%%%%%%%%%%%%%%%%%%%%%%%%%
\documentclass[10pt]{memoir}
\setstocksize{220mm}{155mm} 	        
\settrimmedsize{220mm}{155mm}{*}	
\settypeblocksize{170mm}{116mm}{*}	
\setlrmargins{18mm}{*}{*}
\setulmargins{*}{*}{1.2}
%\setlength{\headheight}{5pt}%
\checkandfixthelayout[lines]
\linespread{1.16}
\flushbottom

%%% Hyphenation settings
\usepackage[htt]{hyphenat}
\hyphenation{he-lio-trope opos-sum}
\tracingparagraphs=1
%Hyphenation in Devanāgarī of the edition still missing? Probably this needs to be modified in babel-iast package? 

%%% babel
\usepackage[english]{babel}
\usepackage{babel-iast/babel-iast}

\babelfont[iast]{rm}[Renderer=Harfbuzz, Scale=1.3]{AdishilaSan}%AdishilaSan}
\babelfont[english]{rm}{Adobe Text Pro}

%%% more functionality
\PassOptionsToPackage{hyphens}{url}
\usepackage{hyperref}
\usepackage{pdflscape}
\usepackage{cleveref}
\usepackage{url}
\usepackage{cleveref}
\usepackage{microtype}
\usepackage{lineno}

%\usepackage{bigfoot}
%%% more functions
\usepackage[dvipsnames]{xcolor}
%\usepackage[para,perpage]{footmisc}

%%%für den Counter von Kapiteln und Sätzen! 
\newcommand{\uproman}[1]{\uppercase\expandafter{\romannumeral#1}}
\newcommand{\lowroman}[1]{\romannumeral#1\relax}

\makeindex
\newfontfamily\sanskritfont[Script=Devanagari,Mapping=RomDev,Scale=1.1]{Sanskrit2003}
\usepackage{pifont,fourier-orns,lettrine,psvectorian,paralist,enumitem,pdfpages,wrapfig,tabulary,lettrine,longtable}
\setlist[enumerate]{itemsep=0mm}
\usepackage[autostyle]{csquotes}
\usepackage[defaultlines=2,all]{nowidow}
\usepackage{ellipsis,adforn,booktabs,longtable,url,tikz}
\lineskiplimit=-3pt          

\makechapterstyle{IeT}{%
  \chapterstyle{default}
  \renewcommand*{\printchapternonum}{\centering}
  \renewcommand*{\clearforchapter}{\cleartorecto} 
  \aliaspagestyle{chapter}{empty}}
\chapterstyle{IeT}
\setsecnumdepth{none}  \openright  \nouppercaseheads
\settocdepth{subsubsection}

%%%% test better pagebreaks
%\def\fussy{%
%  \emergencystretch\z@
%  \tolerance 200%
%  \hfuzz .1\p@
%  \vfuzz\hfuzz}

%\interfootnotelinepenalty=10000\relax

%\usepackage[maxfloats=256]{morefloats}

%\maxdeadcycles=500

%raggedbottomsectiontrue
%%\checkandfixthelayout


%%%%%%%  biblatex
%\newcommand{\noun}[1]{\textsc{#1}}    %  philosophy-verbose
\usepackage[backend=biber, sorting=nyt, style=verbose]{biblatex} %%%%ORIGINAL TiE
\renewcommand*{\mkbibnamefamily}[1]{\textsc{#1}}


\DeclareFieldFormat{url}{%
  \mkbibacro{URL}\addcolon\space
  \href{#1}{\nolinkurl{\thefield{urlraw}}}}

\DeclareFieldFormat{citeurl}{%
  \href{#1}{\nolinkurl{\thefield{urlraw}}}} 


\DeclareFieldFormat{postnote}{#1}
\renewcommand{\postnotedelim}{, }
\addbibresource{bindu.bib}

%%% ekdosis
\usepackage[teiexport=tidy,parnotes=true]{ekdosis}% =tidy cleans up HTML and XML documents by fixing markup errors and upgrading legacy code to modern standards. parnotes=footnotes below or above critical apparatus

\SetLineation{lineation=page, modulo} %lineation=page sets thenumbering to start afresh at the top of each page. =modulo makes every fifth line numbered. {lineation=page} makes every line numbered! 

\renewcommand{\linenumberfont}{\selectlanguage{english}\footnotesize} %sets language of lines to English

\SetTEIxmlExport{autopar=false} %autopar=falseinstructs ekdosis to ignore blank lines in the.tex sourcefile as markers for paragraph boundaries. As a result, each paragraph of the edition must be found within an environment associated with the xml <p> element

\SetHooks{
  lemmastyle=\bfseries,
  %refnumstyle=\selectlanguage{english}\bfseries,
  refnumstyle=\selectlanguage{english}\color{blue}\bfseries,
  appheight=0.8\textheight,
}

\newif\ifinapparatus
\DeclareApparatus{source}[
%bhook=\inapparatustrue,
lang=english,
notelang=english,
% bhook=\selectlanguage{english},
bhook=\selectlanguage{english}\textbf{Sources:},%
%maxentries=4, 
%ehook=.]
%sep={] },
%nosep,
]

\newif\ifinapparatus
\DeclareApparatus{testium}[
%bhook=\inapparatustrue,
lang=english,
notelang=english,
% bhook=\selectlanguage{english},
bhook=\selectlanguage{english}\textbf{Testimonia:},
%maxentries=4, 
%ehook=.]
%nosep, 
]

% Declare \ifinapparatus and set \inapparatustrue at the beginning of
% the apparatus criticus block. Also set the language.  
\newif\ifinapparatus
  \DeclareApparatus{default}[
  %bhook=\inapparatustrue, 
  lang=english,
  %maxentries=33,
  %bhook=\selectlanguage{english},
  sep = {] },
  delim=\hskip 0.75em,
  rule=\rule{0.7in}{0.4pt},
]

\newif\ifinapparatus
\DeclareApparatus{philcomm}[
%bhook=\inapparatustrue,
lang=english,
notelang=english,
bhook=\selectlanguage{english}\textbf{Philological Commentary:},
%bhook=\selectlanguage{english},
sep={: },
]

\ekdsetup{
showpagebreaks,
spbmk = \textcolor{blue}{spb},
hpbmk = \textcolor{red}{hpb}
}

%\usepackage{fnpos}
%\makeFNmid
%\makeFNbottom
\usepackage[bottom]{footmisc}
%%%%%%%%%%%%%%%%%%%%%%%%%%%
\makeatletter
\def\blfootnote{\gdef\@thefnmark{}\@footnotetext}
\makeatother
%%%%%%%%%%%%%%%%%%%%%%%%%


% Macros and Definitions for the Print of Sigla
\def\acpc#1#2#3{{#1}\rlap{\textrm{\textsuperscript{#3}}}\textsubscript{\textrm{#2}}\space}
\def\sigl#1#2{{{#1}}\textsubscript{\textrm{#2}}}
\def\None{{\sigl{N}{1}}} \def\Noneac{\acpc{N}{1}{ac}\,} \def\Nonepc{\acpc{N}{1}{pc}\,}
\def\Ntwo{{\sigl{N}{2}}} \def\Noneac{\acpc{N}{2}{ac}\,} \def\Nonepc{\acpc{N}{2}{pc}\,}
\def\Done{{\sigl{D}{1}}} \def\Doneac{\acpc{D}{1}{ac}\,} \def\Donepc{\acpc{D}{1}{pc}\,}
\def\Dtwo{{\sigl{D}{2}}} \def\Dtwoac{\acpc{D}{2}{ac}\,} \def\Dtwopc{\acpc{D}{2}{pc}\,}
\def\Uone{{\sigl{U}{1}}} \def\Uoneac{\acpc{U}{1}{ac}\,} \def\Uonepc{\acpc{U}{1}{pc}\,}                 
\def\Utwo{{\sigl{U}{2}}} \def\Utwoac{\acpc{U}{2}{ac}\,} \def\Utwopc{\acpc{U}{2}{pc}\,}

%%%%%%%%%%%%%% Tattvabinduyoga - List of Witnesses   %%%%%%%%%%%%%%%%%%%
\DeclareWitness{ceteri}{\selectlanguage{english}cett.}{ceteri}[]   
\DeclareWitness{E}{\selectlanguage{english}E}{Printed Edition}[]    
\DeclareWitness{P}{\selectlanguage{english}P}{Pune BORI 664}[]  
\DeclareWitness{B}{\selectlanguage{english}B}{Bodleian 485}[]       
\DeclareWitness{N1}{\selectlanguage{english}N\textsubscript{1}}{NGMPP 38/31}[]
\DeclareWitness{N2}{\selectlanguage{english}N\textsubscript{2}}{NGMPP B 38/35}[]
\DeclareWitness{L}{\selectlanguage{english}L}{LALCHAND 5876}[]  
\DeclareWitness{D}{\selectlanguage{english}D}{IGNCA 30019}[] 
%\DeclareWitness{D2}{\selectlanguage{english}D\textsubscript{2}}{IGNCA 30020}[]  
\DeclareWitness{U1}{\selectlanguage{english}U\textsubscript{1}}{SORI 1574}[] 
\DeclareWitness{U2}{\selectlanguage{english}U\textsubscript{2}}{SORI 6082}[]
%%%%%%%%%%%%%% Tattvabinduyoga - Groups of Witnesses   %%%%%%%%%%%%%%%%%%%
\DeclareWitness{X}{\selectlanguage{english}\alpha}{Alpha Group: D,N1,N2,U1}[]
\DeclareWitness{Y}{\selectlanguage{english}\beta}{Beta Group: B,E,L,P,U2}[]
%%%%%%%%%%%%% Testimonia
\DeclareWitness{Ysv}{\selectlanguage{english}Ysv}{Yogasvarodaya}[] %%%add infos!  

%%%%%%%%%%%%%%%%%%%%%%%%%%%%%%%%%%%%%%%%%%%
% Macro for Editing Abbrevs.
\def\om{\textrm{\footnotesize \textit{om.}\ }} %prints om. for omitted in apparatus
\def\korr{\textrm{\footnotesize \textit{em.}\ }} %prints em. for emended in apparatus
\def\conj{\textrm{\footnotesize \textit{conj.}\ }} %prints conj. for conjectured in apparatus

% \supplied{text} EDITORIAL ADDITION -> Within \lem oder \rdg
% \surplus{text} EDITORIAL DELETION -> Within \lem oder \rdg
% \sic{text} CRUX
% \gap{text} LACUNAE -> [reason=??, unit=??, quantity=??, extent=??]


%%%%%%%%%%%%%%%%%%%%%%%%%%%%%%%%%%%%%%%%%%% All macros of this list can be used in 
% Macro for Editing Abbrevs.
\def\eyeskip{\textrm{{ab.\,oc. }}}
\def\aberratio{\textrm{{ab.\,oc. }}}
\def\ad{\textrm{{ad}}}
\def\add{\textrm{{add.\ }}}
\def\ann{\textrm{{ann.\ }}}
\def\ante{\textrm{{ante }}} 
\def\post{\textrm{{post }}}
%\def\ceteri{cett.\,}                   
\def\codd{\textrm{{codd.\ }}}

\def\coni{\textrm{{coni.\ }}}
\def\contin{\textrm{{contin.\ }}}
\def\corr{\textrm{{corr.\ }}}
\def\del{\textrm{{del.\ }}}
\def\dub{\textrm{{ dub.\ }}}

\def\expl{\textrm{{explic.\ }}} 
\def\explica t{\textrm{{explic.\ }}}
\def\fol{\textrm{{fol.\ }}}
\def\foll{\textrm{{foll.\ }}}
\def\gloss{\textrm{{glossa ad }}}
\def\ins{\textrm{{ins.\ }}}      
\def\inseruit{\textrm{{ins.\ }}} 
\def\im{{\kern-.7pt\lower-1ex\hbox{\textrm{\tiny{\emph{i.m.}}}\kern0pt}}} %\textrm{\scriptsize{i.m.\ }}}      
\def\inmargine{{\kern-.7pt\lower-.7ex\hbox{\textrm{\tiny{\emph{i.m.}}}\kern0pt}}}%\textrm{\scriptsize{i.m.\ }}}      
\def\intextu{{\kern-.7pt\lower-.95ex\hbox{\textrm{\tiny{\emph{i.t.}}}\kern0pt}}}%\textrm{\scriptsize{i.t.\ }}}           
\def\indist{\textrm{{indis.\ }}}  
\def\indis{\textrm{{indis.\ }}}
\def\iteravit{\textrm{{iter.\ }}} 
\def\iter{\textrm{{iter.\ }}}
\def\lectio{\textrm{{lect.\ }}}   
\def\lec{\textrm{{lect.\ }}}
\def\leginequit{\textrm{{l.n. }}} 
\def\legn{\textrm{{l.n. }}}
\def\illeg{\textrm{{l.n. }}}

\def\primman{\textrm{{pr.m.}}}
\def\prob{\textrm{{prob.}}}
\def\rep{\textrm{{repetitio }}}
\def\secundamanu{\textrm{\scriptsize{s.m.}}}            \def\secm{{\kern-.6pt\lower-.91ex\hbox{\textrm{\tiny{\emph{s.m.}}}\kern0pt}}}%   \textrm{\scriptsize{s.m.}}}
\def\sequentia{\textrm{{seq.\,inv.\ }}}  
\def\seqinv{\textrm{{seq.\,inv.\ }}}
\def\order{\textrm{{seq.\,inv.\ }}}
\def\supralineam{{\kern-.7pt\lower-.91ex\hbox{\textrm{\tiny{\emph{s.l.}}}\kern0pt}}} %\textrm{\scriptsize{s.l.}}}
\def\interlineam{{\kern-.7pt\lower-.91ex\hbox{\textrm{\tiny{\emph{s.l.}}}\kern0pt}}}   %\textrm{\scriptsize{s.l.}}}
\def\vl{\textrm{v.l.}}   \def\varlec{\textrm{v.l.}} \def\varialectio{\textrm{v.l.}}
\def\vide{\textrm{{cf.\ }}}
\def\cf{\textrm{{cf.\ }}} 
\def\videtur{\textrm{{vid.\,ut}}}
\def\crux{\textup{[\ldots]} }
\def\cruxx{\textup{[\ldots]}}
\def\unm{\textit{unm.}}
%%%%%%%%%%%%%%%%%%%%%%%%%%%%%%%%%%%%

% List of Scholars
\DeclareScholar{ego}{ego}[
forename=Nils Jacob,
surname=Liersch]

% Persons:14\DeclareScholar{ego}{ego}[15forename=Robert,16surname=Alessi]17% Useful shorthands:18\DeclareShorthand{codd}{codd.}{V,I,R,H}19\DeclareShorthand{edd}{edd.}{Lit,Erm,Sm}20\DeclareShorthand{egoscr}{\emph{scripsi}}{ego}

%Useful shorthands:
%\DeclareShorthand{codd}{codd.}{V,I,R,H}
%\DeclareShorthand{edd}{edd.}{Lit,Erm,Sm}
\DeclareShorthand{egoscr}{em.}{ego}
\DeclareShorthand{egoscrconj}{conj.}{ego}
\DeclareShorthand{egomute}{\unskip}{ego}

\usepackage{xparse}

\NewDocumentEnvironment{tlg}{O{}O{}}{\setlength{\leftskip}{0pt}\vspace{-1ex}\begin{quotation}}{\hfill #1\ \vspace{-1ex}\end{quotation}\vspace{-1ex}} %verse environment
%\NewDocumentEnvironment{tlg}{O{}O{}}{\begin{verse}}{॥#1\hskip-4pt ॥\\ \end{verse}}
\NewDocumentCommand{\tl}{m}{{\selectlanguage{iast} #1}}

\NewDocumentCommand{\extra}{m}{{\textcolor{gray}{#1}}} %command for additions to U2
\NewDocumentCommand{\crazy}{m}{{\textcolor{red}{#1}}} %totally corrupted passage
\NewDocumentCommand{\coro}{m}{{\textcolor{violet}{#1}}} %colour for sentence counter! 

\NewDocumentEnvironment{prose}{O{}}{\begin{otherlanguage}{iast}}{\end{otherlanguage}}
% \NewDocumentEnvironment{padd}{O{}}{\begin{otherlanguage}{iast}}{\end{otherlanguage}}
\NewDocumentEnvironment{tlate}{O{}}
%\NewDocumentEnvironment{tadd}{O{}}

%Define two commands: \skp ("sanskrit plus"), to be ignored by TeX in
%the edition text, but processed in the TEI output. Conversely, \skm
%("sanskrit minus") is to be processed in the edition text, but
%ignored if found in the apparatus criticus and in the TEI output:

\NewDocumentCommand{\skp}{m}{}
\TeXtoTEIPat{\skp {#1}}{#1}

%\NewDocumentCommand{\skpp}{m}{}
%\TeXtoTEIPat{\skpp {#1}}{#1}

\NewDocumentCommand{\skm}{m}{\unless\ifinapparatus#1-\fi}
\TeXtoTEIPat{\skm {#1}}{}

% \NewDocumentCommand{\dd}{}{/\hskip-4pt/}
\NewDocumentCommand{\dd}{}{\mbox{/\hskip-4pt/}}
\TeXtoTEIPat{\dd {}}{//}


%%% modify environments and commands
%%% TEI mapping
\TeXtoTEIPat{\begin {tlg}}{<lg>} %lg=(Group of verse (s)) contains one or more verses or lines of verse that together form a formal unit (e.g. stanza, chorus).
\TeXtoTEIPat{\end {tlg}}{</lg>}

\TeXtoTEIPat{\begin {prose}}{<p>}
\TeXtoTEIPat{\end {prose}}{</p>}

\TeXtoTEIPat{\begin {tlate}}{<p>}
\TeXtoTEIPat{\end {tlate}}{</p>}

\TeXtoTEIPat{\\}{}
\TeXtoTEIPat{\linebreak}{<br/>}
\TeXtoTEIPat{\noindent}{}
%\TeXtoTEI{tl}{l}
\TeXtoTEI{emph}{hi}
\TeXtoTEI{bigskip}{}
\TeXtoTEI{None}{N1}
\TeXtoTEI{Ntwo}{N2}
\TeXtoTEI{Done}{D1}
\TeXtoTEI{Dtwo}{D2}
\TeXtoTEI{Uone}{U1}
\TeXtoTEI{Utwo}{U2}
%\TeXtoTEIPat{/}{ |}
%\TeXtoTEI{//}{ ||}
\TeXtoTEIPat{\korr}{em. }
\TeXtoTEIPat{\conj}{conj.}
\TeXtoTEIPat{\om}{om.}
\TeXtoTEIPat{english}{}
\TeXtoTEIPat{\hskip}{}
\TeXtoTEIPat{\hskip-4pt}{}
\TeXtoTEIPat{\hskip-2pt}{}
\TeXtoTEIPat{-}{ }
\TeXtoTEIPat{4pt}{}
\TeXtoTEIPat{2pt}{}
\TeXtoTEIPat{\textcolor {#1}{#2}}{<hi rend="#1">#2</hi>} 

% Nullify \selectlanguage in TEI as it has been used in
% \DeclareWitness but should be ignored in TEI.
\TeXtoTEI{selectlanguage}{}



\FormatDiv{1}{\begin{center}\Large}{\end{center}}
\FormatDiv{2}{\begin{center}\small}{\end{center}}
\FormatDiv{3}{\bfseries}{.}
\title{Yogatattvabindu of Rāmacandra\\ A Critical Edition and Annotated Translation}
\date{\today}

\parindent=15pt
\begin{document}

% Zitiermöglichkeiten:
%\footcite[See][p.\,1]{goldstein01:_tibet_englis_diction_moder_tibet}
%\footnote{\cite{goldstein01:_tibet_englis_diction_moder_tibet}.}

\frontmatter
\thispagestyle{empty}
\begin{center}
  {\Large \emph{The Yogatattvabindu}}\\[3mm]
\end{center}



\newpage

\

\thispagestyle{empty}



\normalsize


\newpage


\begin{center}
\thispagestyle{empty}

\

\vskip 2mm

\begin{otherlanguage}{iast}
\LARGE \sanskritfont{Yogatattvabindu}
\end{otherlanguage}

\vskip .4cm

\Huge Yogatattvabindu \\[7mm]
\Large Critical Edition\\
with annotated Translation


\large

\vspace{3cm}

Von

Nils Jacob Liersch
\small
\vfill

\vfill

Indica et Tibetica Verlag \\ % $\cdot$ 
Marburg 2024

\vskip 6mm

\end{center}

\newpage
\newpage \ \thispagestyle{empty}
\small  \

\noindent

\
\vfill


\small
\noindent \textbf{Bibliographische Information Der Deutschen Bibliothek}

\noindent
Die Deutsche Bibliothek verzeichnet diese Publikation in der Deutschen Nationalbibliographie;
detaillierte bibliographische Informationen sind im Internet über http://dnb.ddb.de abrufbar.

\noindent
\textbf{Bibliographic information published by Die Deutschen Bibliothek}

\noindent
Die Deutsche Bibliothek lists this publication in the Deutsche Nationalbibliographie; detailed
bibliographic data is available in the Internet at http://dnb.ddb.de.  


\vskip 1cm

\noindent
\copyright\ Indica et Tibetica Verlag, Marburg 2024

\medskip

\noindent
Alle Rechte vorbehalten / All rights reserved

\medskip

\noindent
Ohne ausdrückliche Genehmigung des Verlages ist es nicht gestattet, das Werk oder einzelne Teile
daraus nachzudrucken, zu vervielfältigen oder auf Datenträger zu speichern.

\smallskip

\noindent
Apart from any fair dealing for the purpose of private study, research, criticism or review, no
part of this book may be reproduced or translated in any form, by print, photo form, microfilm, or
any other means without written permission. Enquiries should be made to the publishers.

\bigskip

\noindent
Satz: \ \ Nils Jacob Liersch \\
Herstellung: \ \ BoD – Books on Demand GmbH, Norderstedt  \\

\bigskip

\noindent
%\ISBN     

\normalsize

\newpage

%\maketitle
\clearpage
\tableofcontents
\addtocounter{page}{-1}
\thispagestyle{empty}
\clearpage


\mainmatter

\chapter{Introduction}
\cleardoublepage

\section{General remarks}
The \textit{Yogatattvabindu} is a premodern Sanskrit Yoga text on Rājayoga that was written in the first half of the seventeenth century\footnote{The dating of the text is discussed on p.\pageref{dating}.} in northern India.\footnote{The detailed discussion of the place of origin is found on p.\pageref{placeoforigin}.} The most salient feature of the work that makes it historically significant is its highly differentiated taxonomy of types of Yoga. In the \textit{Yogatattvabindu}'s introduction, most manuscripts name fifteen types of Yoga, presented as subtypes of Rājayoga. The text is a yogic compendium written in a mix of mainly prose and 41 verses in textbook-style, where its 58 topics topics are introduced in sections launched by recognizable phrases. Most sections deal with the subtypes of Rājayoga and their effects, but others also cover topics like yogic physiology and cosmogony.

The \textit{Yogatattvabindu} has not been discussed or considered in secondary literature on Yoga. The only exception is \citeauthor{birch2014} (2014: 415–416) who briefly described its list of fifteen Yogas in the context of the “fifteen medieval Yogas” and noted that a similar\footnote{My research suggests that list of fifteen Yogas in Nārāyaṇatīrtha’s \textit{Yogasiddhāntacandrikā} must be chronologically later than the ones found in the \textit{Yogatattvabindu} and its sources. As I will show in the discussion of the fifteen Yogas on p.\pageref{15yogas}, we have to assume that Nārāyaṇatīrtha saw the need to map the fifteen Yogas onto system of the \textit{Pātañjalayogaśāstra} due to their popularity among practitioners in his sphere of activity.} list occurs in Nārāyaṇatīrtha’s \textit{Yogasiddhāntacandrikā} (17th - 18th century), a commentary on the \textit{Pātañjalayogaśāstra} that integrates almost an identical taxonomy of yogas within the \textit{aṣṭāṅga} format. An incomplete account of the fifteen Yogas is found within the Sanskrit Yoga text \textit{Yogasvarodaya}, which is known only through quotations in the \textit{Prāṇatoṣinī} and \textit{Yogakarṇikā}.\footnote{Manuscripts under the name of \textit{Yogasvarodaya} seem to be lost. I was not able to allocate the manuscripts of the text in any manuscript catalogue at hand.} The \textit{Yogasvarodaya} provides a total of fifteen Yogas but names only eight of them in its introductory \textit{śloka}s. A complete account of the text is yet to be found and might be lost forever. The \textit{Yogasvarodaya} is the primary source and template for the compilation of the \textit{Yogatattvabindu}. Rāmacandra closely follows the content and structure by rewriting the \textit{Yogasvarodaya}’s \textit{śloka}s into prose. Due to the incomplete transmission of the \textit{Yogasvarodaya}, Rāmacandra’s \textit{Yogatattvabindu} is a natural and valuable starting point for an in-depth study of the taxonomy of the fifteen types of Yoga. The other source text that Rāmacandra used is the \textit{Siddhasiddhāntapaddhati} whose content he draws on, particularly in the last third of his composition. Another text that includes a similar taxonomy of twelve Yogas divided into three tetrads is Sundardās’s \textit{brāj bhāṣa} Yoga text named \textit{Sarvāṅgayogapradīpikā} which not just shares most of the types of Yogas but also many of the practices and contents found within the \textit{Yogatattvabindu} and \textit{Yogasvarodaya}.\footnote{For a comparative table of the complex Yoga taxonomies see table \ref{15yogastable} on p.\pageref{15yogastable}.}

These complex taxonomies that emerged during the 17th and 18th centuries crossed sectarian divides and were adapted to the specific needs of different authors and traditions. The \textit{Yogatattvabindu} thus encapsulates the diversity of Haṭha- and Rājayoga types and teachings after the \textit{Haṭhapradīpikā} (15th century) that were adopted by a broad spectrum of religious traditions and strata of Indian society. In the particular case of the \textit{Yogatattvabindu}, there are various statements throughout the text that reveal a strategy to detach Yoga from its renunciate connotations and to enforce the supremacy and universality of Rājayoga as a practice that can yield the highest benefits even for practitioners who enjoy worldly pleasures and an extravagant lifestyle. Textual evidence suggests the possibility that \textit{Yogatattvabindu} may be a unique example of a Rājayoga text that was composed for warrior aristocracy and members of an royal court. 

In addition, the analysis of the \textit{Yogatattvabindu} and the historical retracig of its teachings provides insight into a complex network of at least twenty texts,\footnote{This intertextual network which shares those specific teachings consists of the \textit{Netratantra}, \textit{Śāradatilakatantra}, \textit{Sarvadurgatipariśodhanatantra}, \textit{Ūrmikaulārṇavatantra}, \textit{Tantrāloka}, \textit{Manthanabhairavatantra}, \textit{Śārṅgadhārapaddhati}, \textit{Vivekamārtaṇḍa}, \textit{Śivayogapradīpikā}, (recensions of the \textit{Haṭhapradīpikā}), \textit{Amaraughaśāsana}, \textit{Yogasvarodaya}, \textit{Sarvāṅgayogapradīpikā}, \textit{Nityanāthapaddhati}, \textit{Siddhasiddhāntapaddhati}, \textit{Yogatattvabindu}, \textit{Yogacūḍāmaṇyupaniṣad}, \textit{Maṇḍalabrāhmaṇopaniṣat}, \textit{Haṭhatattvakaumudi} and \textit{Haṭhasaṃketacandrikā}.} all of which include one specific set of yoga theorems and practices with minor deviations - three to five \textit{cakra}s, sixteen \textit{ādhāra}s, two to five \textit{lakṣya}s, and five \textit{vyoma}s. This intertextual network spans at least an entire millennium. It begins in early śivaite Tantras such as the \textit{Netratantra} and ends in the large premodern Yoga compendiums like the \textit{Haṭhatattvakaumuḍī} and \textit{Haṭhasaṅketacandrikā}. The examination of this network provides insights into the history of the related yoga traditions and enables, for example, the reconstruction of the genesis of individual yoga categories mentioned in the fifteen Yogas, such as Lakṣyayoga, whose techniques were originally taught in early śivaite Tantras, but were only labeled as a separate type of yoga from the 17th century onwards.

One printed edition of the \textit{Yogatattvabindu} was published in 1905 with a Hindi translation and based on an unknown manuscript(s). This publication has the title ’\textit{Binduyoga}’ confirmed by the printed text’s colophon. However, as I discuss in the course of the introduction, the text was likely known as \textit{Yogatattvabindu}. The consulted manuscripts contain significant discrepancies, structural differences and variant readings between them and the printed edition. Furthermore, the manuscripts are scattered over the Indian subcontinent, which suggests that it was widely transmitted at some point. Lenghty passages of the \textit{Yogatattvabindu} are quoted without attribution in a text called \textit{Yogasaṃgraha} and Sundaradeva’s \textit{Haṭhasaṅketacandrikā}. A critical edition will undoubtedly improve on the published edition and shed further light on the transmission of this important work.

This book contains an introduction, critical edition and annotated translation of the \textit{Yogatattvabindu}. The introduction discusses provenance, authorship and the audience of the \textit{Yogatattvabindu}. A comprehensive discussion of the taxonomy of the fifteen Yogas based on the critical edition of the \textit{Yogatattvabindu}, together with a close examination of the above-mentioned related texts with similar taxonomies, aims to establish their position within the broader history of yoga and particularly elucidates the development of Haṭha- and Rājayoga traditions in the late medieval period. The remainder of the introduction contains an overview of the manuscript evidence and the editorial policies underlying the edition.

\section{Dating the \textit{Yogatattvabindu}}
\label{dating}
The oldest dated manuscript of the \textit{Yogatattvabindu} \getsiglum{N1}\footnote{For a description of the manuscript see  p.\pageref{n1description}.} was written in Nepal \textit{saṃvat} 837, which is 1716 CE. Since the text of this manuscript is missing a significant and lengthy passage (ca. 25\% of the entire text) and contains various corruptions, one can assume that some time had passed from the original composition for the transmission to deteriorate to this extent. Therefore, it is likely that the work was composed at least a few decades before the creation of this Nepalese manuscript, perhaps sometime in the 17th century. The discovery that Sundaradeva's \textit{Haṭhasaṅketacandrikā} quotes a lengthy passage of the \textit{Yogatattvabindu} without attribution confirms this suspicion. The passages quoted from the \textit{Yogatattvabindu} include the teachings on the sixteen \textit{ādhāra}s\footnote{\citetitle{hathasamketacandrikajodhpur} (ms. no. 2244, f. 95r l. 3 -- f. 96r l. 4).} and the teachings on Lakṣyayoga and its subtypes.\footnote{\citetitle{hathasamketacandrikajodhpur} (ms. no. 2244, f. 124r l. 7 -- f. 125r l. 3).} The dating of the \textit{Haṭhasaṅketacandrikā} just recently had to be revised due to the discovery that some first-hand notes surrounding the main text of the Ujjain \textit{Yogacintāmaṇi} were in all likelihood borrowed from Sundaradeva's \textit{Haṭhasaṅketacandrikā}.\footnote{Cf. \citeauthor{birch2024} (2024:52-54).} \citeauthor{birch2018proliferation} (2018) dated the Ujjain \textit{Yogacintāmaṇi} to 1659 CE.\footnote{Cf. \citeauthor{birch2018proliferation}, 2018: 50 [n. 111].} Thus, the \textit{terminus ante quem} for the compilation of the \textit{Haṭhasaṅketacandrikā} is 1659 CE which automatically makes it also the \textit{terminus ante quem} for the \textit{Yogatattvabindu} and the \textit{Yogasvarodaya}, due to the fact that Sundaradeva quoted from the \textit{Yogatattvabindu} and Rāmacandra quoted from and rewrote the contents of the \textit{Yogasvarodaya}. Thus, we can safely assume that the \textit{Yogatattvabindu} was written in the course of the first half of the 17th century or earlier. Because of that Rāmancandra's main source text \textit{Yogasvarodaya} must have been written even earlier.

\subsection{Implications for the dating of the \textit{Yogasvarodaya} and the \textit{Siddhasiddhāntapaddhati}}
Furthermore, \citeauthor{mallinsononline2013}\footnote{Cf. \fullcite{mallinsononline2013}.} estimated the age of the \textit{Siddhasiddhāntapaddhati} to circa 1700. Due to the above-mentioned new date of the \textit{Haṭhasaṅketacandrikā} and because Rāmacandra extensively quotes from \textit{Siddhasiddhāntapaddhati} the new terminus \textit{terminus ante quem} for the dating of the \textit{Siddhasiddhāntapaddhati} likewise must be set to 1659 CE. Thus, the \textit{Siddhasiddhāntapaddhati} was also likely composed during the first half of the 17th century or even ealier.

\chapter{The complex medieval yoga taxonomies}
\label{yogas_list}
\clearpage


\section{The rise of diversity: The increasing complexity of Yoga teaching systems in late medieval and pre-colonial India}

In diesem Kapitel soll es darum gehen, dass zwischen dem 17. und 18. Jh. in Indien parallel zu einer Populariserung des Yoga in breiten Schichten der Gesellschaft jenseits der asketischen Traditionen eine allgemeine Entwicklung zu beobachten ist, die sich in gesteigerter Komplexität äußert. In den damals zirkulierenden Texten kommt es zu einer Steiugerung der Anzahl der gelehrten Cakras, Āsanas, Kumbhakas, aber auch die Taxonomien der einzelnen Yogakategorien die gelehrt werden nehmen an Komplexität zu. 


\section{Comparative Analysis of the complex Yoga taxonomies}

The similarities between the Yoga taxonomies of Rāmacandra's \textit{Yogatattvabindu}, his source text, the \textit{Yogasvarodaya} as well as the taxonomies laid out by Nārāyaṇatīrtha in his \textit{Yogasiddhāntacandrikā} and Sundardās' \textit{Sarvāṅgayogadīpikā} which all emerged within the same time period have been initially observed and discussed briefly by \citeauthor{birch2014} (2014)\footnote{See \citeauthor{birch2014}, 2014: 415-416.} In the following chapter, the lists and their items are examined in a comparative analysis.

A complete comparative description of all Yoga categories used in the literature would go far beyond the scope of this work. However, with this presentation I hope to adequately cover our understanding of the concepts of different Yoga categories circulating in the literature of the 17th - 18th centuries that include these complex taxonomies.

The analysis will follow the structure of the individual Yogas outlined in the \textit{Yogatattvabindu}. Each Yoga will be described based on the explanations in the \textit{Yogatattvabindu}, and its content will be compared with the explanations of the corresponding Yoga in the texts with similar taxonomies. The comparison will broaden and clarify our understanding of the respective spectrum of meanings of the individual Yoga categories in the discursive field of the authors of the texts containing the taxonomies. This comparison results in the documentation of the discursive web of word usage of various Yoga categories between the 17th and 18th centuries CE, most probably mainly localised in central northern India.\footnote{The complex taxonmies evolved and circulated most likely in central northern India. For a detalled discussion see p.\pageref{location}.} Individual Yoga categories that do not appear in the list of the \textit{Yogatattvabindu} but are listed in the other texts with complex taxonomies will also be covered and outlined. In addition, Yoga categories that do not appear in any of the analysed lists but are nevertheless mentioned in the texts will also be covered so that this analysis attempts to approximate the overall picture of all Yoga categories used during the period under consideration as closely as possible. However, it is essential to emphasise that the comparison of Yoga categories focuses primarily on those texts that contain complex Yoga taxonomies and cannot claim to be exhaustive. Although the analysis and comparison of the Yoga categories can be extended to other Yoga texts, locations and time periods if necessary or valuable, the restriction to the complex Yoga taxonomies should be maintained to prevent this already complex endeavour going ad absurdum.\footnote{There are hundreds, if not thousand of Sanskrit and vernacular texts from different times and different regions of India, which operate with these categories.}      

\begin{table}[h]
    \centering
    \begin{tabularx}{\textwidth}{>{\raggedright\arraybackslash}p{0.05\textwidth}XXXX}
        \toprule
        No. & \textit{Yogatattvabindu} & \textit{Yogasvarodaya} & \textit{Yogasiddhāntacandrikā} & \textit{Sarvāṅgayogadīpikā} \\
        \midrule
        1. & \textit{kriyāyoga} & \textit{kriyāyoga} & \textit{kriyāyoga} & \textit{\textbf{bhaktiyoga}} \\
        2. & \textit{jñānayoga} & \textit{jñānayoga} & \textit{caryāyoga} & \textit{mantrayoga} \\
        3. & \textit{caryāyoga} & \textit{karmayoga} & \textit{karmayoga} & \textit{layayoga} \\
        4. & \textit{haṭhayoga} & \textit{haṭhayoga} & \textit{haṭhayoga} & \textit{carcāyoga} \\
        5. & \textit{karmayoga} & \textit{dhyānayoga} & \textit{mantrayoga} & \textit{\textbf{haṭhayoga}} \\
        6. & \textit{layayoga}  & \textit{mantrayoga} & \textit{jñānayoga} & \textit{rājayoga} \\
        7. & \textit{dhyānayoga} & \textit{urayoga}   & \textit{advaitayoga} & \textit{lakṣayoga} \\
        8. & \textit{mantrayoga} & \textit{vāsanāyoga} & \textit{lakṣyayoga} & \textit{aṣṭāṅgayoga} \\
        9. & \textit{lakṣyayoga} & -                   & \textit{brahmayoga} & \textit{\textbf{sāṃkhyayoga}} \\
        10. & \textit{vāsanāyoga} & -                   & \textit{śivayoga} & \textit{jñānayoga} \\
        11. & \textit{śivayoga} & -                    & \textit{siddhiyoga} & \textit{brahmayoga} \\
        12. & \textit{brahmayoga} & -                  & \textit{vāsanāyoga} & \textit{advaitayoga} \\
        13. & \textit{advaitayoga} & -                 & \textit{layayoga} & - \\
        14. & \textit{siddhayoga} & -                  & \textit{dhyānayoga} & - \\
        15. & \textit{rājayoga} & - [\textit{rājayoga}]& \textit{premabhaktiyoga} & - \\
        \bottomrule
    \end{tabularx}
    \caption{Complex Taxonomies of Yoga in Yoga Texts of the 17th - 18th Centuries}
    \label{tab:complextaxonomies}
\end{table}

\section{1. Kriyāyoga}

Kriyāyoga\footnote{See section II. on p.\pageref{kriyayogastart}-\pageref{kriyayogaend}.} is the first Yoga within the list of fifteen Yogas presented by Rāmacandra and his source text \textit{Yogasvarodaya}. Remarkably, Nārāyaṇatīrtha also positions Kriyāyoga at the first position within the list of fifteen Yogas in his \textit{Yogasiddhāntacandrikā}. Sundardās, on the other hand, omits Kriyāyoga within his taxonomy.

\subsection{The concept of Kriyāyoga in the \textit{Yogatattvabindu}}

Since Rāmacandra refers to all fifteen Yogas as variants of Rājayoga in his initial definition of Yoga, and no explicit hierarchy is recognisable from his formulations in the text, all variants of Rājayoga appear to have been regarded by him as equally effective. All Yogas aim towards the same goal: long-term durability of the body (\textit{bahutarakālaṃ śarīrasthitiḥ}). The positioning of Kriyāyoga does not initially provide any information about the efficiency or the assignment of differently talented practitioners to a particular type of Yoga, as was the case in the older fourfold taxonomies.\footnote{According to \citetitle{amaraugha2024}\textit{prabodha} 18-24, Mantrayoga is best suited for the weak, Layayoga for the average, Haṭhayoga for the talented and Rājayoga for the exceptionally talented practitioner. In \citetitle{datta2024} 14, one finds the statement that the lowest practitioner should perform mantra yoga, which is then also referred to as the lowest Yoga. \citetitle{mallinson2007} 12-28 expands this fourfold scheme of Yogas and practitioners with a temporal dimension. The weak practitioner needs twelve years to succeed with Mantrayoga, the average practitioner needs eight years with Laya, the able practitioner six years with Haṭha and the exceptional practitioner three years with Rājayoga} Implicit hierarchical aspects are nevertheless present - although all Yoga types are a type of Rājayoga, Rāmacandra nonetheless places Rājayoga in the final and topmost position of his taxonomy.
The only apparent reason why Rāmacandra specifies Kriyāyoga as the first Yoga seems to be that his primary source text, whose content structure he largely follows,\footnote{see the chapter on ``structural inconsistencies'' on p.\pageref{struktur},} specifies this type of Yoga as the first.

The passage on Kriyāyoga in the \textit{Yogatattvabindu} is relatively short. The four verses presented by Rāmacandra are quoted without attribution from the \textit{Yogasvarodaya}. A prose section repeats the content of the verses. By definition, Kriyāyoga in \textit{Yogatattvabindu} is ``liberation through [mental] action'' (\textit{kriyāmuktir ayaṃ yogaḥ}). In contrast to Rāmacandra's worldly definition of Rājayoga and its subcategories, here, liberation (\textit{mukti}) overrides this initial goal. In addition, the practitioner achieves ``success in one's own body'' (\textit{svapiṇḍe siddhidāyakaḥ}). The method of Kriyāyoga involves restraining any [mental] wave before an action. This restraint consists of reducing negative [mind-]waves and cultivating positive ones. Noticeably, the number of negative waves significantly exceeds the number of positive waves.

\begin{table}[h]
    \centering
    \begin{tabularx}{\textwidth}{XX}
        \toprule
        \textbf{Mental waves to be cultivated} & \textbf{Mental waves to be reduced} \\
        \midrule
        Patience (\textit{kṣamā}) & Envy (\textit{matsārya}) \\
        Discrimination (\textit{viveka}) & Selfishness(\textit{mamatā})\\
        Equanimity (\textit{vairāgya}) & Cheating (\textit{māyā})\\
        Peace (\textit{śānti}) & Violence (\textit{hiṃsā})\\
        Modesty (\textit{santoṣa}) & Intoxication (\textit{mada})\\
        Desirelessness (\textit{niṣpṛha}) & Pride (\textit{garvata})\\
        & Lust (\textit{kāma}) \\
        & Anger (\textit{krodha}) \\
        & Fear (\textit{bhaya})\\
        & Laziness (\textit{lajjā})\\
        & Greed (\textit{lobha})\\
        & Error (\textit{moha})\\
        & Impurity (\textit{aśuci})\\
        & Attachment and aversion (\textit{rāgadveśau}) \\
        & Disgust and laziness (\textit{ghṛṇālasya})\\
        & error (\textit{bhrānti})\\
        & Deceit (\textit{daṃbha})\\
        & Envy (repeatedly) (\textit{akṣama})\\
        & Confusion (\textit{bhrama})\\
        \bottomrule
    \end{tabularx}
    \caption{Mental waves to be cultivated and reduced in Rāmacandra's Kriyāyoga}
    \label{tab:waves}
\end{table}

The one who cultivates positive [mind-]waves and reduces the negative is called a \textit{kriyāyogī}. In the prose passage of the section, the term \textit{bahukriyāyogi} is used. The term is unprecedented in the rest of the yoga literature and presumably intends to express many reduced and cultivated waves.

\subsection{The concept of Kriyāyoga in the \textit{Yogasvarodaya}}
A closer examination of the Kriyāyoga section in the \textit{Yogasvarodaya} reveals Rāmancandra's reductionism since he excludes significant aspects of the original concept of the \textit{Yogasvarodaya}'s Kriyāyoga.

%YK 1.214-216

\begin{quote}
\textit{dhyānapūjādānayajñajapahomādikāḥ kriyāḥ} |\\
\textit{kriyāmuktimayo yogaḥ svapiṇḍe siddhidāyakaḥ}\footnote{svapiṇḍe siddhidāyakaḥ YTB] sapiṇḍisiddhidāyakaḥ YSv sapiṇḍisiddhidāyakaḥ YK} || 1 ||

(1) Actions are meditation, ritual veneration, donation, recitation, fire sacrifice, etc. 
The Yoga made of liberation through action[s] bestows success in one's own body. 

\textit{yat karomīti saṅkalpaṃ kāryārambhe manaḥ sadā} |\\
\textit{tat sāṅgācaraṇaṃ kurvan kriyāyogarato bhavet} || 2 ||

(2) ``Whatever I do'' at the beginning of an action, the mind always has an intention.  
Doing that [following] procedure with all its parts, one becomes established in Kriyāyoga.  

\textit{kṣamāvivekavairāgyaśāntisantoṣanispṛhāḥ} |\\
\textit{etad yuktiyuto yo'sau kriyāyogo nigadyate} || 3 ||

(3) Patience, discrimination, equanimity, peace, modesty, desirelessness:
The one endowed with these means is said to be a Kriyāyogī.

\textit{mātsaryaṃ mamatā māyā hiṃsā ca madagarvitā} |\\
\textit{kāmaḥ krodho bhayaṃ lajjā lobho mohas tathā'śuciḥ} || 4 ||

(4) Envy, selfishness, cheating, violence, intoxication and pride,
lust, anger, fear, laziness, greed, error, and impurity.

\textit{rāgadveṣau ghṛṇālasyaśrāntidambhakṣamābhramāḥ} |\\
\textit{yasyaitāni na vidyante kriyāyogī sa ucyate} || 5 ||

(5) Attachment and aversion, disgust and laziness, error, deceit, envy [and] confusion:
Whoever does not experience these is called a Kriyāyogī.

\textit{sa eva muktaḥ sa jñānī caṇḍināśena īśvaraḥ} |\\
\textit{kriyāmuktikaro yo'sau rājayogaḥ sa muktidaḥ} || 6 ||(om. YK)

(6) He alone, the wise one, the lord, through the destruction of impetuous [behaviour]
who performs the liberation through action[s] is liberated. This Rājayoga is the bestower of liberation.

\textit{yāvan mano layaṃ yāti kṛṣṇe svātmani cinmaye} | \\ 
\textit{bhaved iṣṭamanā mantrī japahomau samabhyaset} || 7 ||\footnote{7ab \approx \citetitle{rudrayamala1937} 38.58cd.}(om. YSv) 

(7) Until the mind enters absorption into Kṛṣṇa, in one's own self, into consciousness,
the mantra practitioner (\textit{mantrin}) should practise recitation and fire sacrifice with an aspiring mind. 

\textit{vidite paratattve tu samastair niyamair alam} |\\
\textit{tālavṛntena kiṃ kāryaṃ lavdhe malayamārute} || 8 ||\footnote{\approx \citetitle{kularnavatantra} 9.28 \& \citetitle{yuktabhavadeva} 1.80.} (om. YSv) 

(8) When the highest principle has been realised through all the \textit {niyama}s, as is proper,
Why should one wave the palm frond when the wind from the Himalayas has already reached?

\textit{tāvat karmmāṇi kurvanti yāvajjñānaṃ na vidyate} |\\ 
\textit{jñāne jāte pareśāni karmākarma na vidyate} || 9 ||(om. YSv) 

(9) As long as [regular?] actions are performed, so long realisation is unknown.
When knowledge ensues, oh, Supreme Goddess, neither action nor non-action is known.
\end{quote}

These verses\footnote{The numbering used here was introduced by me for practical reasons and does not correspond to the original numbering of the verses in the citations of the source texts. The \textit{Prāṇatoṣiṇī} does not number the verses at all. The verses can be found in the printed edition of the \textit{Prāṇatoṣiṇī} on p. 831. The verses here are in the \textit{Yogakarṇikā} with the numbering 1.209-216 and can be found in the edition on p. 17.} stem from the only two currently available sources of the \textit{Yogasvarodaya}, namely the quotations from the \textit{Prāṇatoṣiṇī}\footnote{A considerable part of the \textit{Yogasvarodaya} is quoted with source reference (\textit{yogasvarodaye}).} and the \textit{Yogakarṇikā}.\footnote{Normally the \textit{Yogakarṇikā} quotes its sources. This passage is one of the few exceptional cases in which the verses have been taken from the \textit{Yogasvarodaya} without citing the source. However, this passage ends after verse 1.216 with ``\textit{iti yogasaṅketāḥ |}''.} The quotations of both texts essentially correspond, but the last verses of the passage differ. It cannot be ruled out that the last three verses of the \textit{Yogakarṇikā} in particular come from a different source and were not present within the \textit{Yogasvarodaya}. However, their content is so closely interwoven with the preceding verses that this scenario can be considered unlikely.

The main difference to the Kriyāyoga that Rāmacandra has constructed from these verses is the definition of the actions (\textit{kriyāḥ}) mentioned immediately at the beginning of the verses, of which the actions (\textit{kriyā}s) of Kriyāyoga is then predominantly composed, namely of (1) meditation, (2) ritual worship of God, (3) offerings, (4) recitation and (5) fire sacrifice, etc. Furthermore, while Rāmacandra declares the elements mentioned in the table \ref{tab:waves} as waves (\textit{kallola}) of the mind which are either required to be cultivated or reduced before any action is executed, the same elements are conceptualised in the \textit{Yogasvarodaya} as the intentions (\textit{saṅkalpa}) preceding the previously defined actions (\textit{kriyā}s), which should be observed.

In the three verses concluding this section, which are only handed down in the \textit{Yogakarṇikā}, the practitioner is referred to as \textit{mantrin} and should perform recitation and fire offerings until entering absorption (\textit{laya}).

A possible historical link, particularly in front of the Vaiṣṇava background, is the model of Kriyāyoga as found in the \textit{Uddhavagīta}\footnote{See i.e., \citeauthor{uddhavagita2007} (2007).} which is a part of the famous \textit{Bhāgavatapurāṇa}\footnote{See i.e., \citeauthor{bhagavata} (1950).}. Here, in chapter XXII.1-55 Kṛṣṇa describes a Vaiṣṇava form of Kriyāyoga in response to a request by his disciple Uddhava. The practice entails a very complex and devotional ceremonial veneration of the deity through offerings such as flowers and food, accompanied by the recitation of prescribed mantras, meditation, and the ritual consecration of the deity, among other rites. According to the text, this type of Yoga is the most beneficial for women and the working class (22.4) and is considered a means for liberation from the fetters of Karma (22.5). The Kriyāyoga described here is presented to be in line with both the Vedas and the Tantras, considering enjoyment (\textit{bhukti}) and liberation (\textit{mukti}) and is promised to bestow perfection in both this life and the next, by the Lord's grace (22.49).  

Furthermore, this concept of Kriyāyoga in the \textit{Yogasvarodaya} might be linked to the \textit{kriyāpāda}\footnote{See e.g. \citeauthor{ganesan2016saiva} (2016) and \citetitle{mrgendragama}, Ed. pp. 1-205.} of the Śaiva \textit{āgama}s. The Śaiva \textit{āgama}s are collections of various tantric traditions, written in Sanskrit or Tamil, in which cosmology, epistemology, philosophical teachings, various practices such as meditation or Yoga, mantra recitation, worship of the gods, etc. are described. These texts\footnote{The fourfold division of \textit{pāda}s is only present in a limited number of Āgamas: \textit{Kiraṇa}, \textit{Suprabheda}, \textit{Mṛgendra} and \textit{Mataṅgaparameśvara} (as Upāgamas), see \citeauthor{brunner1994place} , 1993: 225-461 for an overview.} usually consist of four sections (\textit{pāda}s): The \textit{jñānapāda} (knowledge section), \textit{kriyāpāda} (action section), \textit{caryāpāda} (behaviour section) and the \textit{yogapāda} (yoga section).\footnote{The order or the \textit{pāda}s varies, but the \textit{yogapāda} is always the last.} It can be no coincidence that \textit{jñāna°}, \textit{kriyā°} and \textit{caryā°} were each integrated as a separate Yoga category within the taxonomy of the fifteen Yogas\footnote{see p.\pageref{intro}.}. The \textit{kriyāpāda} is the section of a Śaiva \textit{āgama} that describes rules and practices for the performance of various rituals such as the significant initiation (\textit{dīkṣa}), ceremonies and worship of the gods. Additionally, \textit{prāṇāyāma} techniques and meditations are often found as parts of these rituals. There are also explanations of the nature of \textit{mudrā}s, \textit{maṇḍala}s and \textit{mantra}s. Furthermore, various characteristics of different types of Śaiva initiates\footnote{These are \textit{samayin, putraka, sādhaka, ācārya,} and \textit{astrābhiṣeka}.} can be found here.\footnote{See \citeauthor{ganesan2016saiva} (2016) for a general overview of the four \textit{pāda}s. One of the few Śaiva \textit{āgama}s that has been edited and translated into a Western language (French) is the \citetitle{mrgendragama}. For this see \citeauthor{mrgendragama} (1962) \& \citeauthor{mrgendragamabrunner} (1985).} The \textit{kriyā}s mentioned at the beginning of the \textit{Yogasvarodaya} - meditation, ritual veneration, donation, recitation, fire sacrifice, etc. have hardly deniable parallels to the \textit{kriyāpāda}s of the Śaiva \textit{āgama}s and thus could have their reception-historical roots precisely there. The other part, however, which describes the cultivation or reduction of certain mental configurations preceding all actions (\textit{saṅkalpa}) or [mental] waves (\textit{kallola}), I have not yet been able to locate in the Śaiva \textit{āgama}s, but they seem to be a simplyfied rendering of the Pātañjalean model of Kriyāyoga that was passend on in hitherto unknown traditions that practiced this type of Kriyāyoga.

\subsection{The concept of Kriyāyoga in the \textit{Yogasiddhāntacandrikā}}

The Kriyāyoga in Nārāyaṇatīrtha's commentary on \textit{Pātañjalayogaśāstra} entitled \textit{Yogasiddhāntacandrikā} presents Kriyāyoga as the first of his fifteen Yogas, which he locates in Pātañjalayoga. The term Kriyāyoga occurs in \textit{Pātañjalayogaśāstra} 2.1. According to the introduction to this \textit{sūtra}, in the \textit{bhāṣya}-part of the \textit{Pātañjalayogaśāstra}, Kriyāyoga is the means by which someone with a distracted mind can also attain Yoga (\textit{vyutthitacitto 'pi yogayuktaḥ}). In \citetitle{yogasutra} 2.1, Kriyāyoga is defined as follows:
\begin{quote}  
  \textit{tapaḥsvādhyāyeśvarapraṇidhānāni kriyāyogaḥ} |
\end{quote}

Kriyāyoga, or ``yoga of action'', is the action oriented method of Yoga consisting of three elements. Namely, austerity (\textit{tapas}), which according to the \textit{bhāṣya} should be practised both mentally and physically, the repetition of \textit{mantra}s or the study of sacred literature (\textit{svadhyāya}) and devotion to God (\textit{īśvarapraṇidhāna}).
According to \citetitle{yogasutra} 2.2, these three elements of Kriyāyoga should lead the practitioner to attain \textit{samādhi} by reducing the so-called \textit{kleśa}s. This explanatory model is picked up by Nārāyaṇatīrtha.\footnote{\citeauthor{yogacandrika}, 2000:71.} The five \textit{kleśa}s consist of ignorance (\textit{avidyā}), self-centredness (\textit{asmitā}), attachment (\textit{rāga}), aversion (\textit{dveṣa}) and fear of death (\textit{abhiniveśa}). 
All three main components of Patañjali's Kriyāyoga are not mentioned in the \textit{Yogatattvabindu} and \textit{Yogasvarodaya}. Nevertheless, a practice similar to the reduction of the \textit{kleśa}s can also be found here. Although the specific fear of death (\textit{abhiniveśa}) is not mentioned, the more general term for fear (\textit{bhaya}) is cited.\footnote{The details of Nārāyaṇatīrtha's understanding of Kriyāyoga have already be discussed by \citeauthor{penna2004} (2004: 62-66) and will therefore not be covered here again.}
The Kriyāyoga in \textit{Yogatattvabindu} and \textit{Yogasvarodaya} could, therefore, be perhaps regarded as a degenerated or simplified variant of the Pātañjalean model, which restricts itself predominantly to the aspect of the reduction of negative waves of the mind, which is comparable to the reduction of \textit{kleśa}s and adds the aspect of cultivating positive mind waves to be mix. In both systems, Kriyāyoga is a means for liberation.\footnote{The Kriyāyoga of the \citetitle{yogasutra} will not be dealt with in detail here, as this has already been done in countless academic and informal publications. For the \textit{sūtra}s related to Kriyāyoga and Patañjali's autocommentary in Sanskrit with English translation, see \citeauthor{yogasutra} 1983: 113 et seqq. For a comprehensible and more accessible overview, see \citeauthor{bryant2009} 2009: 170 et seqq.}

\subsection{Kriyāyoga in the complex late-medieval Yoga taxonomies}

The analysis of Kriyāyoga within the taxonomies of fifteen yogas shows two distinct models. One is Nārāyaṇatīrtha's model, which draws directly on the Kriyāyoga of \textit{Pātañjalayogaśāstra}. Additional śaiva influences characterise the other model of Kriyāyoga that seems to have been locally prominent in the 17. - 18. century C.E. The precisely defined \textit{kriyā}s of the \textit{Yogasvarodaya} must be historically linked to the \textit{kriyāpāda}s of the Śaiva \textit{āgama}s, whereby the core practice of reducing and cultivating specific mental configurations before any action is loosely associated with the Kriyāyoga of the \textit{Pātañjalayogaśāstra}. The observation that the \textit{kriyā}-, \textit{caryā-}, and \textit{jñānayoga}s, are an allusion to the \textit{kriyā}-, \textit{caryā-}, \textit{jñāna-} and \textit{yogapāda}s of the Śaiva \textit{āgama}s, shows that Nārāyaṇatīrtha, as a proponent of the \textit{Pātañjalayoga}, was most likely not the originator of the fifteenfold taxonomy, but rather that the taxonomy of the fifteen Yogas originated from local discourses around the authors and had achieved such local popularity at the time that Nārāyaṇatīrtha forced the fifteenfold taxonomy into Patañjali's \textit{Yogaśāstra} in order to show that the Yogaśāstra \textit{par excellence} and all those varieties of Yogas that were discussed in his sphere are in truth already present in the ``classical'' system of Patañjali.

\subsection{Popularisation of Kriyāyoga in a global context}

The comparatively unique treatises on Kriyāyoga, which can only be found in the Yoga literature from the 17th-century onwards\footnote{The terminus \textit{ad quem} for the \textit{Yogasvarodaya} and \textit{Yogatattvabindu} is 1659 CE, see p.\pageref{dating} for the details.} in \textit{Yogasvarodaya} and Rāmacandra's \textit{Yogatattvabindu}, which deviate from the Pātañjala model, albeit not entirely, and, as shown, show clear influences of tantric origin, can be regarded as marginal phenomena for the time being. The briefly touched upon model of \textit{Uddhavagītā}, which describes a Kriyāyoga method for \textit{mukti} and \textit{bhukti} through ritual worship of god, is also comparatively rare in the literature. The overwhelming majority of the Sanskrit yoga texts written in the second millennium CE, as in the case of Nārāyaṇatīrtha's \textit{Yogasiddhāntacandrikā}, are based on the model of Kriyāyoga propagated in the \textit{Pātañjalayogaśāstra}. Accordingly, it was above all the publication of the \textit{Yogasūtra} in the West, beginning with the translation by Henry Thomas Colebrooke in 1805\footnote{See \parencite{colebrooke2014} for a detailed discussion,} which ensured that the concept of Kriyāyoga contained therein also dominated the understanding of the term in academic and informal discourse in the West for a long time. 

The Western discourse only changed with the global success and popularity of Paramahaṃsa Yogānanda (1893-1952) and the \textit{Self Realisation Fellowship} he founded in 1920, which, measured against the predecessor models forms of Kriyāyoga outlined above, spread an innovative Yoga practice under the generic term Kriyāyoga. The influence of Yogānanda and others significantly changed and expanded the range of meanings of the term Kriyāyoga. In addition to various books published by Yogānanda, it was above all, the book \citetitle{autobioyogi}, the autobiography of Yogānanda himself, published in 1946, which paved the way for Yogānanda's success. To this day, this work is considered a classic in popular Yoga literature, has been in print for over seventy years and has been translated into more than 50 languages.\footnote{Cf. \cite{yoganandawebsite}.} It also has a large global following to this day. Yogānanda, his books, his followers and the numerous books written by his followers have popularised this innovative and new form of Kriyāyoga beyond the Indian subcontinent. The term Kriyāyoga was allegedly already defined by Yogānanda's predecessors, namely Lahiḍi Mahāśaya (1828-1895) and Śrī Yukteśvar Giri (1855-1936), as the central generic term for the Yoga practice of this line of tradition.\footnote{Cf. \citeauthor{govindan2010} 2010:51-52} 

One of Yogānanda's contemporaries was Svāmī Śivānanda Sarasvatī (1887-1963), who similarly propagated a new form of Kriyāyoga. Although his Kriyāyoga was initially based mainly on the Pātañjalayoga model, it was expanded under the same umbrella term with Haṭhayoga practices and possibly influenced by Yogānanda's model. This expansion and integration of new practices under the umbrella term Kriyāyoga was continued excessively by his students, above all Svāmī Satyānanda Sarasvatī (1923-2009), the founder of the famous \textit{Bihar School of Yoga} (since 1962).

The resulting popularity of Kriyāyoga triggered a global wave and inspired others, who in turn developed similar but sometimes differently nuanced Kriyāyoga systems. One example is S.A.A. Ramaiah, who founded the \textit{Kriya Babaji Yoga Sangam} in 1952. In this case, too, there is a global following.\footnote{Cf. \cite{kriyababajiyoga}}.

It was the actors mentioned above, above all Yogānanda, who ensured the global popularisation of this new form of Kriyāyoga so that their concepts are at least as well known in recent public discourse, if not better known, than the Kriyāyoga of the \textit{Pātañjalayogaśāstra}.

These new forms of Kriyāyoga, which can only be traced from the beginning of the 19th century, are, as will be shown, a reservoir for innovative combinations and further developments of numerous practices already codified in Yoga texts in the medieval to pre-colonial period, which were integrated into seemingly coherent practice systems by actors such as Yogānanda, Śivānanda, Ramaiah, etc. The statements made by their traditions about the historicity of their Yoga practice utilise established narratives to lend this form of Kriyāyoga a tradition and historical legitimacy.\footnote{For example, the tracing back of the Yoga tradition to a legendary founding figure, the time of the master in the Himalayas, lost writings that suddenly reappear and legitimise the practice can already be found in a similar form in the lineages of T. Krishnamarcharya. See \citeauthor{singleton2013gurus}, 2013: 81-121.}


\subsection{The Kriyāyogas of the lineages of Paramahaṃsa Yogānanda, Svāmī Śivānanda Sarasvatī and Ramaiah}

So what constitutes these new forms of Kriyāyoga? To answer this question, recent publications on this topic were consulted.\footnote{This list is certainly not exhaustive. Nevertheless, I have consulted a wide range of these publications available to me. 1. For the Yogānanda model: \citeauthor{autobioyogi} (1949); \citeauthor{kriyayogalowenstein} (2021); \citeauthor{kriyayogasarasvati1981} (1981); \citeauthor{hariharananda1989} (1989); \citeauthor{kriyayogaupanishad1993} (1993) and \citeauthor{kriyayogasturgess2015} (2015). 2. For the Śivānanda model: \citeauthor{shivanandakriya1982} (1955) and \citeauthor{kriyayoganityananda2013} (2013). 3. And for the the Ramaiah model: \citeauthor{govindan2010} (2010).} The following is a brief outline of the main features of the Yogānanda, Śivānanda and Ramaiah models of Kriyāyoga without claiming to be exhaustive. To my knowledge, a comprehensive and complete historical study of Kriyāyoga has not yet been carried out and cannot be done within this framework. This attempt is an outline and should be understood as a first approach to the topic in order to differentiate between the models circulating in public discourse on the one hand and, on the other, to formulate a hypothesis on the transition from the older models to the newer models, as these are very close in time.  

\subsubsection{Definitions}

The publications consulted contain various creative etymologies and explanations of the term Kriyāyoga. \citeauthor{hariharananda1989}, a Kriyāyoga teacher authorised by Yogānanda \footnote{Cf. \citeauthor{hariharananda1989} 1989: 16.} himself explains: \begin{quote} 'Kriya Yoga' are Sanskrit words, a combination of two root words. One is Kriya and the other is yoga. In the word Kriya there are two syllables: kri and ya. Kri means to pursue your work in daily life and ya means to be ever aware of the invisible God who is abiding in you and is directing and accomplishing work through you. \ldots  The second word, 'yoga,' literally means union of the visible body with the invisible body. This union is always present in everyone. (\citeauthor{hariharananda1989} 1989: 83) \end{quote}
Another etymology of the term \textit{kriyā} can be found in \citeauthor{kriyayogalowenstein} (2021: 91): \begin{quote} \ldots kri meaning ``work'' and ya meaning ``soul'' or ``breath'' = The Work to be done with the Souls breath. \end{quote}
The most complex explanation of the term can be found in \citeauthor{kriyayoganityananda2013} (2013: 2-3), who also locates himself in the Yogānanda tradition: \begin{quote}
  The word \textit{kriyā} is composed of the letters \textit{k}, \textit{r}, \textit{i}, \textit{y}, and \textit{ā}. The letter -\textit{k} (or \textit{ka}), \textit{ka-kāra}, represents the Lord, \textit{Īśvara}. The Transcendental Lord, \textit{Parama Śiva}, when he manifests Himself in the suble world and makes Himself ready for creation He becomes \textit{Īśvara}. The letter-\textit{r} (or \textit{ra}), \textit{ra-kāra}, represents fire, light and manifestation. Creation is not seen by us with the ether and air elements since these are subtle elements. We are able to see manifestation from the fire element onwards. The letter -\textit{i}, \textit{i-kāra}, represents energy or \textit{śakti}. So \textit{kri} is the activating power of the Lord manifested in creation. The activating power is called \textit{prāṇa} or vital force. The letter -\textit{y} (or \textit{ya}), \textit{ya-kāra}, represents the air element and the letter -\textit{ā}, \textit{ā-kāra}, represents form. For the manifestations to take a form, \textit{ākāra}, the Lord acts with the air element. With the ether element there is no form. The air element or gaseous state is the first created form although we only see the forms from the fire element onwards. Through the action of air the whole universe is manifested. This is the action of the Life-force, \textit{prāṇakarma}, of the Lord. The word \textit{kriyā} normally means action, but this is the action of god. We are made with the same principle God is. Our identification with the physical body makes us separate from God and this is the state of ignorance. We have to eradicate this ignorance by the action of God, i.e., the action of the breath, \textit{prāṇakarma}. Our mind is the result of ignorance and is responsible for the wrong identification. Breath-practice, \textit{prāṇakarma}, absorbs the mind into the vital force. This action of God reverses the process and leads us from body to God. This is why it is so necessary to perform that action. That is our spiritual practice. Then that action, \textit{kriyā}, becomes yoga. \end{quote}
\citeauthor{kriyayogasarasvati1981} (1981: 699), an important proponent of the Śivānanda model, defines Kriyāyoga as follows: \begin{quote} The Sanskrit word \textit{kriya} means `action' or `movement'. \textit{Kriya Yoga} is so called because it is a system where one intentionally rotates one's attention along fixed pathways. This movement of awareness is done, however with control. Also kriya yoga is so called because one moves the body into specific mudras, bandhas and asanas according to a fixed scheme of practice. The word \textit{kriya} is often translated as meaning `practical'. This is indeed a good definition, for kriya yoga is indeed practical. It is concerned solely with practice, without the slightest philosophical speculation. The system is designed to bring results, not merely to talk about them. Sometimes the word \textit{kriya} is translated as `preliminary'. This too is a good definition, for kriya yoga is a preliminary practice that leads first to dharana and then eventually to the transcendental state of dhyana (meditation) and yoga (union). It is a technique which has been designed to lead to that state of being which is beyond all techniques. Finally, the word \textit{kriya} is used to describe each individual practice. Thus the process of kriya yoga consists of a number of kriyas each being done one after the other in a fixed sequence.\end{quote}
\citeauthor{govindan2010} (2010: 214), a student of Ramaiah offers a simple explanation of the term: \begin{quote} Kriyā is an activity performed with mindfulness.\end{quote}

As different as the concepts presented here may seem, they have in common that they are about consciously performed actions or practices that connect people with God or are intended to bring about a transcendent state, a state of yoga. In his definition, \citeauthor{kriyayoganityananda2013} already mentions the central action (\textit{kriyā}) that should lead to a connection with God, namely breathing practice (\textit{prāṇakarma}). In addition, \citeauthor{kriyayogasarasvati1981} also mentions other practices such as directing attention, \textit{mūdra}s, \textit{bandha}s and \textit{āsana}s.  

Further definitions can be found in the consulted texts. However, these are sufficient for the purposes here, as they illustrate the basic idea of the new models of Kriyāyoga on the one hand and show the fundamental diversity and openness of the model, which permeates all areas of these new forms of Kriyāyoga, on the other.  

\subsubsection{Histories of the new forms of Kriyāyoga from an emic perspective}

\citeauthor{kriyayoganityananda2013} (2013: 2-7), who places himself in the lineage of Yogānanda, explains that Kriyāyoga is an eternal tradition that stands at the beginning of human history. He explains that this is why many of the scriptures, such as the \textit{Śivasūtrā}, the \textit{Āgama}s and the writings of the Siddhas, teach the techniques and principles of Kriyāyoga in many different ways. Moreover, remnants of this primal Kriyāyoga can be found in almost all philosophies, be it Buddhism, Jainism, Sāṅkhya, Vaiśeṣika, Nyāya, Mīmāṃsā or Vedānta. 

\citeauthor{kriyayogasarasvati1981} (1981: 699), the founder of the \textit{Bihar school of Yoga}, explains that there is no history of Kriyāyoga and that its origins and development have been lost. Furthermore, the system of Kriyāyoga was so secret that there is not even a myth to explain its origin. Furthermore, he describes that parts of the Kriyāyoga taught by him are contained in the texts of Haṭhayoga, such as \textit{āsana}s, \textit{mudrā}s and \textit{bandha}s, but that these are not ``integrated together''. Furthermore, he speculates that Kriyāyoga must have been known in China, as he sees strong parallels to practices in \textit{Tai Chi Chuan}. Furthermore, he clearly distances himself from the Kriyāyoga of the \textit{Yogasūtra}, which has nothing to do with the Kriyāyoga of his book \citetitle{kriyayogasarasvati1981} and serves solely as a preparation for Rājayoga. However, the only definitive historical statement he can commit himself to is the following: \begin{quote} Of history, all we will say is that kriya yoga was passed on by Swami Sivananda of Rishikesh. \end{quote} Surprisingly, this same \citeauthor{shivanandakriya1982} of Rishikesh in his book \citetitle{shivanandakriya1982} (1955) explicitly traces the Kriyāyoga he taught back to \textit{Yogasūtra} 2.1. \citeauthor{shivanandakriya1982} (1982:168-182) uses the Kriyāyoga of the \textit{Yogasūtra} as the overarching framework of his teaching, which also integrates \textit{ṣatkarma} and breathing exercises from Haṭhayoga into it.

It is important to emphasise that \citeauthor{kriyayogasarasvati1981} recognises that the traditional lineage of Yogānanda also practises the same Kriyāyoga he teaches. However, he explicitly distances himself from their narrative: \begin{quote} Of course, there are various other groups of people in India who have practiced and taught kriya yoga. For example, Swami Yogananda, Yukteshwar Giri, Lahiri Mahasaya, Mahatma Gandhi and so forth practiced kriya yoga. In fact, a thriving organization still propagates it throughout the world. They also do now know the origin of kriya yoga, but they say that it was reintroduced by the great yogi Babaji as the ideal practice for sincere seekers of wisdom in the present Kali Yuga (Dark Age). \end{quote}

This narrative is by far the most widespread explanation of the origins of the new Kriyāyoga and is adopted not only in the tradition of Yogānanda, but also in the tradition of Ramaiah. In his book \textit{Kriya Yoga and the 18 Siddhas} (2010: 31-64), \citeauthor{govindan2010}, a disciple of Ramaiah, has compiled this narrative in detail, which I would now like to summarise in a nutshell.

Mahāvātara Babajī, who according to \citeauthor{govindan2010} is considered an incarnation of the Buddha, was born in 203 CE in Parangipetta in Tamil Nadu under the name Najaraj into a Brahmin family, joined a group of wandering Saṃnyāsins at a young age and studied the holy scriptures. His path soon led him to Śrī Laṅka in Katirkāma (now Kataragama), where he became a disciple of Siddha Boganathar and was initiated by him into various \textit{kriyā}s such as \textit{dhyāna}, \textit{āsana}, \textit{mantra} and \textit{bhaktiyoga}. Bhoganathar later sent Babajī to another teacher, namely Siddha Agastya in Courtallam in the Pothihai hills of Tamil Nadu, located in today's Tinneveley district. He learnt the particularly important \textit{kriyā} called \textit{kuṇḍalinīprāṇāyāma} from him. Agastya then sent Babajī to Badrinath in the Himalayas, where he practised for many months and finally attained \textit{samādhi}. After his enlightenment and attaining immortality at just 16, Babajī set himself the task of helping suffering humanity in its search for God-realisation. As an immortal, Babajī initiated great personalities such as Śaṅkarācārya (788-820) and Kabīr (1440-1518) into the techniques of Kriyāyoga over the centuries. Finally, in 1861, he initiated Lahiḍi Mahāśaya (1828-1895) into Kriyāyoga and gave him the task of passing it on to serious seekers. At this point, \citeauthor{govindan2010} quotes the autobiography of Yogānanda,\footnote{Cf. \citeauthor{autobioyogi}, 1949: 244 f.} which states that Babajī explained to Lahiḍi Mahāśaya that Kṛṣṇa had once passed on Kriyāyoga to Arjuna and that not only Patañjali knew it, but also Jesus Christ, who in turn had passed it on to John, Paul and other disciples. Among Lahiḍi Mahāśaya's 100 disciples was Śrī Yukteśvar (1855-1936), to whom Babajī is also said to have appeared three times. On one of these occasions, Babajī decided that he should send his disciple Yogānanda (1893-1952) to America to spread Kriyāyoga, which he did, gaining global fame and founding the \textit{Self Realisation Fellowship} in 1920, which is still very active today.   

\subsubsection{The practice of the new Kriyāyoga}

In the following, the practices of the new Kriyāyoga are presented in outline based on the publications mentioned and consulted above.\footnote{A comprehensive presentation and comparative analysis of the practices in the various traditions of the new Kriyāyoga would be too far-reaching for this chapter. The most detailed written practice instructions that I have consulted can be found for the Śivānanda/Satyānanda model in \citeauthor{kriyayogasarasvati1981}, (1981: 697-952) and for the Yogānanda model in \citeauthor{kriyayoganityananda2013}, (2013: 249-340).} The words of \citeauthor{hariharananda1989} (1989: 144) are surprisingly apt to give an essential first impression of this complex phenomenon: \begin{quote} Kriya Yoga is the essence and synthesis of all yoga techniques taught in the world.  \end{quote} 
\citeauthor{kriyayogasarasvati1981} (1981:703) explains that each Kriyā consists of a certain number of subordinate techniques. These always consist of a combination of the following six tools: \textit{āsana}, \textit{mudrā}, \textit{bandha}, \textit{mantra}, \textit{prāṇāyāma} and, as he calls it, `psychic passage awareness'. This last point includes a group of exercises mainly involving ``circulating awareness through the \textit{cakra}s in an ascending and descending way'' or similar. A single Kriyā is an exercise unit comprising individual exercises from the six categories mentioned. However, these are not arbitrary but are integrated in a specific, scientific way in order to induce the process of concentration (\textit{dhāraṇa}), meditation (\textit{dhyāna}) and meditative absorption (\textit{samādhi}). The main distinguishing feature from other yoga systems is the innovative and specific combination of the individual techniques into a practical and particularly effective sequence of exercises, referred to here as ``Kriyā''.

In every model the individual exercises are drawn from the vast body of Yoga literature but primarily from the exercises taught in the medieval to pre-colonial texts of the Haṭha- and Rājayoga genres. This always takes place against the background of tantric and medieval concepts of the yogic body, such as \textit{cakra}, \textit{nāḍī} and \textit{vāyu} systems. A common phenomenon in the new Kriyāyoga literature is scientific explanatory models that are used as a means of legitimisation. For example, certain \textit{nāḍī}s are located in schematic sketches of the brain\footnote{\citeauthor{kriyayoganityananda2013}, 2013: 215.}, or positive effects of Kriyāyoga practice are legitimised with evolutionary biology theories, such as the polyvagal theory\footnote{\citeauthor{kriyayogalowenstein}, 2021: 188.}

\citeauthor{govindan2010} (2010: 216-225) distinguishes a total of seven main categories of Kriyāyoga. The first category he mentions is \textit{Kriya Hatha Yoga}. According to him, this is the starting point for every student of Kriya Yoga. This includes eighteen basic relaxation postures (\textit{āsana}s), muscle blocks (\textit{bandha}s), certain gestures (\textit{mudrā}s) and the sun salutation (\textit{sūryanamaskāra}) defined by Babajī.

The second main category is what \citeauthor{govindan2010} calls \textit{Kriya Kundalini Pranayama}. According to him, this practice is the art and science of mastering the breath and is considered to be the most essential and effective tool in Babajī's Kriyāyoga. This is not only meant to awaken the \textit{kuṇḍaliṇī} but with regular practice, the student awakens all \textit{cakra}s and the associated levels of consciousness, which is supposed to ultimately lead to the breathless state of \textit{samādhi} and self-realisation.

The third main category is \textit{Kriya Dhyana Yoga}, which is intended to include meditation techniques that are not explained in detail but are supposed to awaken the mind's hidden faculties.

The fourth main category is \textit{Kriya Mantra Yoga}. This involves the recitation or murmuring (\textit{japa}) of mantras discovered by the Siddhas. The recitation of mantras must take place with faith, love and concentration.

\citeauthor{govindan2010} calls the fifth category \textit{Kriya Bhakti Yoga}, the yoga of love and devotion. In \citeauthor{govindan2010}'s words, this is the ``turbojet'' of self-realisation. This type of Kriyāyoga includes devotional love, chanting, ritual worship and pilgrimages to holy places.

Furthermore, \textit{Kriya Karma Yoga} is named as the sixth category. In this case he refers to \citetitle{kaushik1993} II.47 f. and thus defines this subtype as selfless service that is performed consciously. All actions are supposed to be performed without the expectation of receiving anything in return, free from anger, selfishness, greed and personal desires. Thus, the practitioner is meant to examine his motivation before every action and is always supposed to act without selfish motives.

The seventh and final category is \textit{Kriya Tantra Yoga}. According to this, the followers of Kriyāyoga, just like the Siddhas, lead a family life. This subtype of Kriyāyoga involves retaining the energy normally wasted during sexual activity and transporting it to the higher \textit{cakra}s. The partner is supposed to be loved as an embodiment of the divine.

A similar system is taught in \citeauthor{kriyayogalowenstein} (2021). This initially includes a total of twelve \textit{āsana}s and the five Tibetans, as well as typical \textit{prāṇāyāma} techniques, \textit{ujjāyi}, \textit{kapalabhāti}, various \textit{bandha} techniques such as \textit{uḍḍīyānabandha} or \textit{mahābandha}, various \textit{mūdrā} techniques such as \textit{mahāmudrā}, \textit{śāmbhavīmudrā}, \textit{yonimudrā}, or the so-called \textit{Kriya Breath}. \textit{Kriya Breath} is referred to as \textit{kevalakumbhaka}. In addition, classical gymnastic exercises are also added\footnote{\citeauthor{kriyayogalowenstein}, 2021: 118-124. Gymnastic exercises can also be found in \citeauthor{kriyayogasturgess2015}, 2015: 447-458.} In addition to the \textit{āsana}s of Haṭhayoga, \citeauthor{kriyayogalowenstein} also recommend \textit{Tai Chi}, \textit{Qigong}, physiotherapy or a personal trainer to stay fit. Now and then, a biblical quotation is used. For example, in the case of the \textit{Third Eye Gazing} practice, he quotes Matthew 6:22. Furthermore, \citeauthor{kriyayogalowenstein} emphasise the practice of \textit{Hong Sau} as an important element of the practice. For \citeauthor{kriyayoganityananda2013}, \textit{Hong Sau}, or in this case the indologically correct transliteration \textit{haṃsa}, is also referred to by him as \textit{Haṃsa Sādhanā},\footnote{The \textit{ajapājapa}, recitation of the non-recitation of the \textit{haṃsa} mantra.} ``the very foundation'' of Kriyāyoga.\\

As indicated at the beginning of this section, it is clear that the term Kriyāyoga has given rise to a kind of proliferation of different yoga techniques from earlier yoga traditions, which are integrated into innovative exercise systems and attempted to be historically legitimised in different ways. Depending on the lineage and the teacher, individual characteristics and different explanatory models exist.\footnote{In these books, one repeatedly comes across pseudo-scientific explanatory models and stumbles across parallels drawn here and there to other religions, such as Christianity and Buddhism, to emphasise the effectiveness and importance of certain practices and views. Particularly in the more recent publications, it can be seen that, depending on the author, typically individual expressions of the ideal type of postmodern spirituality and religiosity are expressed, which \citeauthor{bochinger2009} have labelled the ``spiritueller Wanderer'' (\citeauthor{bochinger2009} 2009: 33-49).}\\

One last particularly noteworthy and exemplary publication is \citetitle{kriyayogaupanishad1993} (1993) by \citeauthor{kriyayogaupanishad1993}. This book offers translations of ten well-known \textit{Yoga Upaniṣads} and one \textit{Kriya Yoga Upanishad}. The translator claims that the name of the author of this Sanskrit Yoga Upaniṣad was lost in the course of history. His book has no bibliography, nor are the sources of the translations mentioned. Further searches for a verifiable source text of the \textit{Kriya Yoga Upanishad} remain unsuccessful. The \textit{Kriya Yoga Upanishad} is neither to be found in the known publications and translations of the \textit{Yoga Upaniṣads},\footnote{Cf. \citetitle{yogaupaniṣaded} (1938),} nor in publications of previously unpublished Upaniṣads.\footnote{Cf. \citetitle{upanishads1938} (1938).}. Searching through various catalogues of Sanskrit manuscripts was also unsuccessful.\footnote{In \citetitle{kaivalyadamanuscripts2005} (2005: 50), two manuscripts with the title \textit{Kriyāyoga} (AGJ 665/1 and TSM 6716) are listed, which, unfortunately, I was unable to consult. Neither manuscript is dated. AGJ 665/1 is a Devanāgarī manuscript on paper, and TSM 6716 is a Telugu manuscript on palm leaf. The author of the latter is named Venkaṭayogin. I suspect these manuscripts are probably later works that were created in the 18th century at the earliest. For now, however, no definitive statement can be made on this. However, their consultation could shed further light on the historical development of Kriyāyoga.} It is also striking that the \textit{Kriya Yoga Upanishad} is not mentioned in any other publications on Kriyāyoga consulted. For the time being, therefore, the possibility must be considered that \citeauthor{kriyayogaupanishad1993} is not only the translator of the \textit{Kriya Yoga Upanishad} but also the secret author. Perhaps he wrote this supposedly ancient source text in order to legitimise his own Kriyāyoga doctrine.   

Goswami \citeauthor{kriyayogaupanishad1993} learnt Kriyāyoga from his teacher Shelly Trimmer, who, according to the official website of the \textit{Temple of Kriya Yoga}\footnote{\cite{goswamikriyananda}.} founded by \citeauthor{kriyayogaupanishad1993}, was a guru, yogi, kabbalist and direct disciple of Yogānanda. \citeauthor{kriyayogaupanishad1993} studied philosophy for four years at the University of Illinois and then embarked on a business career. Whether \citeauthor{kriyayogaupanishad1993} would have acquired the qualifications to translate a Sanskrit source text remains to be seen. Possibly, he was a gifted autodidact.

In the \textit{Kriya Yoga Upanishad}, the disciple Sanskriti asks the guru Dattatreya to teach him the teachings of Kriyāyoga. The latter agrees and explains Kriyāyoga in a total of ten chapters. The framework is formed by the eight-limbed Yoga system presented in 1.5, similar to the eight limbs of the Pātañjala scheme. The first chapter (1.6-25) presents the \textit{Ten Spiritual Restraints}. Dattatreya explains the \textit{Ten Spiritual Observances} in the second chapter (2.1-16). Chapter three, \textit{The Nine Postures} (3.1-13), deals with nine \textit{āsana}s with six sitting postures, one standing posture and one complex posture. The fourth chapter (4.1-63) discusses what \citeauthor{kriyayogaupanishad1993} calls \textit{Mystical Anatomy}. Here, six \textit{cakra}s named after the planets (i.e. the \textit{mūlādhāracakra} is called the ``Saturn mass-energy converter \textit{cakra}''), fourteen primary \textit{nāḍī}s and \textit{Kriya Kundalini}, which covers the `divine creative channel' with its mouth, are taught. The fifth chapter (5.1-14) is entitled \textit{Inner Purification} and contains simple \textit{prāṇāyāma} techniques such as \textit{sūryabhedana} and \textit{candrabhedana}. Chapter six (6.1-39), entitled \textit{Breath Control}, instructs another breathing exercise in combination with meditation on the three \textit{akṣara}s that constitute the sacred syllable \textit{auṃ}. During the inhalation (\textit{pūraka}), the yogi is supposed to meditate on \textit{a}, during the breathing posture on \textit{u} and during the exhalation on \textit{ṃ}. In addition, the breathing technique \textit{śītalī} (6.25) and a technique called \textit{yonimudrā} (6.33-34) are presented. Chapter seven (7.1-10) is about \textit{Withdrawal of the Senses}. The practitioner is instructed to let the breath move through the body in a specific order. The eighth chapter (8.1-9) is entitled \textit{Concentration}. Here, the yogin is meant to inhale and hold the breath at specific bodily locations (not the \textit{cakra}s), which are associated with the five elements and the syllables \textit{ya, ra, va, la} and \textit {ha}, as well as specific deities. The even shorter ninth chapter, \textit{Meditation} (9.1-6), basically only states that the practice of concentration leads to meditation after a while. The tenth chapter, \textit{Samadhi} (10.1-12), then describes the final state of Yoga, which is defined as the ``deep conscious trance in which the yogi experiences Absolute Wisdom''.

\subsection{Hypothesis on the transition from the late medieval models to the modern models of Kriyāyoga}

The \textit{Yogasvarodaya} and Rāmacandra's \textit{Yogatattvabindu} were written before 1659 CE. Nārāyaṇatīrtha must have lived between 1600 and 1690 CE., and because of that, his \textit{Yogasiddhāntacandrikā} was also written in this timeframe. Sant Sundardās, the author of the \textit{Sarvāṅgayogapradīpikā} lived from 1596 to 1689. Interestingly, Nārāyaṇatīrtha and Sundardās lived in Benares.\footnote{See \citeauthor{burger2014sarvangayogapradipika} (2014: 684) for dating and location of Sundardās and \citeauthor{penna2004} (2004: 24) for dating and location of Nārāyaṇatīrtha.} Thus, we can safely assume that the complex taxonomies of twelve-fifteen Yogas were part of the local discourse of 17th-century Benares. One might speculate that Rāmacandra might also have lived in these surroundings, but this remains uncertain. Lahiḍi Mahāśaya, the person to whom the new forms of Kriyāyoga seem to go back, lived about a century later, from 1828 to 1895 CE. Interestingly, Lahiḍi Mahāśaya is also said to have spent much of his life in Benares. It is, of course, utterly unclear whether Lahiḍi Mahāśaya ever read any of the works mentioned above. However, at least we know that he not only enjoyed an education in philosophy in Benares but also learnt English and Sanskrit.\footnote{\citeauthor{jones2008encyclopedia}, 2008: 255-56.} However, it is likely that the local discourse regarding the religious-spiritual offerings within Benares did not change abruptly. Lahiḍi Mahāśaya also lived as a family man and householder,\footnote{See \citeauthor{autobioyogi}, 1946: ???.} no sectarian affiliations are known so that the whole variety of religious-spiritual offerings of his time were open to him. He was able to combine them freely. As can be seen from the Yoga texts examined in this book, there was no lack of different Yoga categories in Benares between the 17th and 19th centuries CE. Although these were still labelled differently, they were freely combined in practice. Moreover, given the plethora of Yoga practices from different Yoga traditions and Yoga texts presented in the previous chapter and evident in the publications of the new Kriyāyoga consulted, it is not only credible but also plausible that this phenomenon already began with Lahiḍi Mahāśaya, as Yogānanda claims in his autobiography. However, why Lahiḍi Mahāśaya chose the category of Kriyāyoga as the generic term for his Yoga system cannot be answered conclusively. However, I would like to offer an educated guess.

I hypothesize that the term Kriyāyoga, as the generic term for his system of Yoga, was a strategic decision of Lahiḍi Mahāśaya. It is unlikely, and there is no clear evidence that Lahiḍi Mahāśaya knew the \textit{Yogasvarodaya}, \textit{Yogatattvabindu} and \textit{Yogasiddhāntacandrikā}. It is impossible to determine if there ever was any influence of these texts on Lahiḍi Mahāśaya and his new Kriyāyoga system. But if there was, only the fact that all three texts that mention Kriyāyoga as the very first item in their taxonomies could have influenced his decision to unite all possible Yogas and their techniques under the term Kriyāyoga. Another factor could have been that he was consciously or unconsciously driven by the emerging Yogasūtra hype in the West, which triggered a wave of enthusiasm in India. One wonders why he did not choose the term Rājayoga to integrate many systems as others have done before him. Maybe because the term Rājayoga was already used as a generic term for Pātañjalayoga by then.\footnote{See \citeauthor{birch2014}.} Perhaps the term Kriyāyoga had the advantage that it not only formed a link to the popular and hyped \textit{Yogasūtra}, but already provided a basic framework that was very open to interpretation with the three constitutional practices \textit{tapas}, \textit{svādhyāya} and \textit{īśvarapraṇidhāna}, but at the same time left the possibility for the integration of the variety of post-Pātañjalean physical and non-physical Yoga practices from the Tantras and texts of Haṭha- and Rājayoga through a literal interpretation of the compound prefix \textit{kriyā°} in the sense of ``action''. Whether his thoughts went in a similar direction must remain open. However, we must assume that the discursive  environment of Benares at his time certainly played its part in encouraging Lahiḍi Mahāśaya to integrate the various Yogas circulating in the local discourse of his time under one term.

\section{2. Jñānayoga}
\label{jnanayogaintro}

Jñānāyoga\footnote{See section \uproman{21} and \uproman{22} on p.\pageref{jnanayogastart}-\pageref{endsvabhava}.} is the second Yoga within Rāmacandra's list of fifteen Yogas as well as his source text, the \textit{Yogasvarodaya}. In Nārāyaṇatīrtha's list of fifteen Yogas within the \textit{Yogasiddhāntacandrikā}, Jñānayoga occupies the sixth place. Sundardās positions Jñānayoga at the tenth position in his list of twelve Yogas. Here, it is subsumed under the Sāṅkhyayoga category, the third and final tetrad of his list. 

\subsection{The concept of Jñānayoga in the \textit{Yogatattvabindu}}

Jñānayoga steht an zweiter Stelle in Rāmacandra's Taxonomie der fünfzehn Yogas, wird aber nicht als zweites Yoga beschrieben. Dies geschieht von Sektion \uproman{21}-\sektion{22} seines Textes.\footnote{Siehe Kapitel über strukturellen Probleme des \textit{Yogatattvabindu} auf p.\pageref{structuralissues}.} Das übergeordnete Ziel von Rāmacandra's Jñānayoga ist die bereits in der Einleitung (Sektion \uproman{1}) erwähnte long-term durability of the body (\textit{bahutarakālaṃ śarīrasthitiḥ}), die hier nochmal anders formuliert `From the execution of this [Jñānayoga], time does not bring about the destruction of the body' (\textit{tasya kāraṇāt kālaḥ śarīranāśaṃ na karoti}) ausgedrückt wird. Gleichzeitig verhilft Rāmacandra's Jñānayoga dem Übenden die `reality of Śāṃbhavī' (\textit{śāṃbhavīsattā}) zu erlangen.\footnote{Hiermit ist die höchste Realität und der Zustand des Rājayoga gemeint. Siehe S.\pageref{jnanayogatrans1} in der Edition für eine Diskussion des Begriffes.} Dieses Jñānayoga kann auf zweierlei Weise praktiziert werden kann. Die erste Methode (\uproman{21}.1) entsteht durch Anwendung des `non-dualistischen Denkens' (\textit{avikalpatayā yuktyā}) und die zweite Methode (\uproman{21}.2) entsteht durch die Realisierung, dass die gesamte Welt aus allem Wissen besteht (\ldots \textit{sarvajñānamayaṃ jagat} | \textit{ya evaṃ vetti bodhena} \ldots). Allerdings wird im Text primär auf die erste Methode eingegangen. Diese Methode besteht darin, die Welt als eine Einheit zu betrachten, die vom höchste Selbst (\textit{viśvātman}) erleuchtet ist. Nimmt man diese Einheit wahr, so befindet man sich in der `reality of Śāṃbhavī' (\textit{śāṃbhavīsattā}). Allerdings kann die `reality of Śāṃbhavī' nicht ohne Weiteres erkannt werden, da sie sich eben nicht als Einheit zeigt, sondern als eine zehnfache Vielheit (\uproman{21}.4ab). Er vergleicht dieses Verhältnis mit einem Samen (1) aus dem ein ganzer Baum mit seinen Einzelbestanteilen - den Wurzeln (2), Sprossen (3), Stamm (4), Ästen (5), Knospen (6), Zweigen (7), Pflanzensaft (8), Blüthen (9), und Früchten(10) - erwächst (\uproman{21}.4-\uproman{21}.5). Der Same steht für die Einheit. Der Baum mit seinen verschiedenen Teilen für die Vielheit. Die grundlegende Einheit der Welt ist, wie der Same, aus dem bereits ein ganzer komplexer Baum gewachsen ist, nicht mehr sichtbar und wird nicht wahrgenommen. Was jedoch Rāmacandra zufogle wahrgenommen wird, ist eine aus einer Vielheit von zehn Grundprinzipien (\textit{tattva}s) bestehende Welt: fünf [grobstoffliche] Elemente (\textit{pañcatattva}), denkender Verstand (\textit{manas}), Intellekt (\textit{buddhi}), Illusion (\textit{māya}), Individuation (\textit{ahaṃkāra}), und Modifikationen (\textit{vikriyā}).\footnote{Für die Diskussion des zehnfachen \textit{tattva}-Systems siehe S.\pageref{??} n.??? und S.\pageref{??} n. ??}. Mittles Jñānayoga soll die der Vielheilt zu Grunde liegende Einheit erkannt werden (\uproman{21}.7). Hierfür soll der Übende mittels der Sichtweise der Einheit (\textit{aikyena darśanam}) die Unterscheidung zwischen der sichtbaren, aus den zehn Grundprinzipien bestehenden Welt\footnote{Diese wird von Rāmacandra auch als \textit{saṃsāra} bezeichnet (\uproman{21} ll. 7-9.}, welche eine Einheit mit der reinen, ewigen, unwandelbaren und makellosen Realität bildet, die das höchste Selbst (\textit{viśvātma}) ist, und der unsichtbaren Realität des eigenen Selbst, aufheben. Durch Jñānayoga erkennt der Übende dann, dass das Selbst in Wirklichkeit eine Einheit bildet\footnote{Cf. \textit{Yogatattvabindu} \uproman{22} \pageref{svabhava1} l. 5: `Because of the power of Jñānayoga, there arises the conviction that the self is truly one (\textit{jñānayogaprabhāvād eka eva ātmā iti niścayo bhavati}).} und die sich verändernden Formen dessen materieller Erscheinung im Grunde leer\footnote{Cf. \textit{Yogatattvabindu} \uproman{22} p.\pageref{svabhava2} l.3: `Through Jñānayoga he realises the emptiness of the mutability of form' (\textit{jñānayogād vikārarūparahito jñāyate}).} sind.

\subsection{The concept of Jñānayoga in the \textit{Yogasvarodaya}}

Wenn wir von einer Korrekten Überlieferung des \textit{Yogasvarodaya} in der \textit{Prāṇatoṣiṇī} ausgehen, dann bezeichnet dieser Text sogar zwei unterschiedliche Arten von Jñānayoga. 

Das Jñānayoga der ersten Passage\footnote{Cf. \textit{Prāṇatoṣiṇī} Ed. p. 831 l. 4703 - p. 833 l. 4769.} beinhaltet eine Beschreibung wichtiger Bestandteile des yogischen Körpers. Insbesondere werden die drei primären Kanäle (\textit{nāḍī}s\footnote{Der linke Kanal wird hier als 


Die Zitate des \textit{Yogasvarodaya} in der \textit{Prāṇatoṣiṇī} (Ed. p. 831 l. 4703 - p. 833 l. 4769), sowie \textit{Prāṇatoṣiṇī} (Ed. p. 835 l. 4824 - p. 837 p. 4872) suggerieren, dass das \textit{Yogasvarodaya} zwei verschiedene Arten von Jñānayoga. Das Jñānayoga der ersten Passage  





yatnād dūraṃ parityajya jñānayogo bhavet sudhīḥ |
jñānasaṃyoga ekas tu ekas tu jñānayogavān |

Nachdem man durch Anstrengung die Distanz [dūram]\footnote{Die scheinbare Kluft zwischen der fragmentierten Welt der zehn \textit{tattva}s und ihrer zu Grunde liegenden inneren Einheit.} aufgegeben hat, entsteht Jñānayoga, oh Weiser! Die Vereinigung mit Wissen ist Einheit, fürwahr, Einheit für denjenigen, der sich dem Yoga des Wissens widmet.

ato hi jñānato 'bhinnaṃ jñeyaṃ jñānāt pṛthak pṛthak |
dūrīkṛtyaiva mā pṛthvībhedavākyena darśanāt |

Darum wahrlich, durch Wissen als ungeteilt zu wissen, aufgrund von Wissen [der] einzelnen [Bestandteile] nacheinander, also, durch nicht entfernt machen, aufgrund der Sichtweise mittels der Aufteilung der Welt, 

jñānayogī bhaved yena jñānayogas tu caikakaḥ |
evaṃ jñānān maheśāni ! kālajit śivatāṃ vrajet | 

würde er ein Yogin des Wissens sein, einzig durch solch ein Yoga des Wissens.
Auf diesem Weise oh große Herrin, möge der Besieger des Todes zum Śivazustand schreiten. 






%Notes:

%Chapter 15 - Trikāṇḍa-Yoga: Bhakti Surpasses
%Knowledge and Detachment
%(1) Śrī Uddhava said: 'The Vedic literature of Your Lordship, oh Lotus-eyed One,
%that pays attention to the injunctions concerning actions and prohibitions, deals with
%the good and bad sides of karma [akarma and vikarma]. (2) They also discuss the dif-
%ferences within the varṇāśrama system wherein the father may be of a higher [anulo-
%ma] or a lower [pratiloma] class than the mother, they are about heaven and hell and
%expound on the subjects of having possessions, one's age, place and time [see also 4.8:
%54 and *]. (3) How can human beings without Your prohibitive and regulatory words
%concerning final beatitude, tell the difference between virtue and vice [compare 11.19:
%40-45]? (4) The Vedic knowledge emanating from You offers the forefathers, the gods as
%72Uddhava Gītā
%also the human beings a superior eye upon the - not for everyone that evident - meaning
%of life, what would be the goal, and how we may achieve. (5) The difference between
%virtue and vice one can see with the help of Your Vedic knowledge and that insight does
%not arise by itself, but the Vedas also nullify such a difference and thus clearly confuse
%the issue....'
%(6) The Supreme Lord said: 'The three ways of yoga I described in My desire to
%grant human beings the perfection, are the path of philosophy [jñāna], the path of work
%[karma] and the path of devotion [bhakti]; no other means can be found [for one's
%emancipation. See also B.G. contents and trikāṇḍa].

\section{Caryāyoga - \uproman{3}}
\label{caryayogaintro}

Due to the absence of the term \textit{cāryayoga} in Rāmacandra's sources and the brevity of the section, it seems that he added his version of Caryāyoga to do justice to the list. Rāmacandra emphasizes the cultivation of detachment towards sin (\textit{pāpa}) and merit (\textit{puṇya}). Parallels can be identified with the concept of Caryāyoga as presented in the \citetitle{yogacandrika} (Ed. pp. 2, 52-53, 100-101, 150). Here, it appears that Caryāyoga is a discipline that aims to purify the mind. Nārāyaṇatīrtha introduces Caryāyoga in the context of Yogasūtra 1.33, Ed. p. 52 (\textit{maitrīkaruṇāmuditopekṣāṇāṃ sukhaduḥkhapuṇyāpuṇyaviṣayāṇāṃ bhāvanātaś cittaprasādanam}). According to Nārāyaṇatīrtha's commentary, the practice of it involves cultivating specific mental attitudes, such as \textit{maitrī} (loving-kindness), \textit{karuṇā} (compassion), \textit{muditā} (sympathetic joy), and \textit{upekṣā} (equanimity), towards different objects or situations, such as happiness, suffering, merit, and demerit. The practice of Caryāyoga is said to lead to eradicating mental impurities and attaining a calm and tranquil state of mind. Nārāyaṇatīrtha characterizes the practice as the renunciation of worldly attachments and desires and the performance of selfless actions or Karmayoga. Nārayaṇatīrtha states that Caryāyoga is the "primary discipline of detachment (\textit{vairāgya})," which suggests that it emphasizes the cultivation of detachment or dispassion towards worldly objects and desires as a means of achieving spiritual liberation. Within \citetitle{sarvangayoga} (2.40-51, Ed. pp. 96-98) Sundardās describes Cārcāyog as a type of Bhaktiyog which is \textit{bhakti} towards unmanifest consciousness (\textit{avyakta puruṣa}) in delightful devotion. The practice results in a beautiful inner being (50-51). He first describes the unmanifest consciousness (\textit{avyakta puruṣa}) as being formless and eternal and so on (40), as beginningless and endless, and so on (41). Next, Sundardās describes the various layers of creation emanating from \textit{oṃ} (42-45). He says the unmanifest consciousness illuminates every corner of existence (46), being the inner knower of all (47). Then, Sundardās expresses the importance of deep awe towards the infinite, divine, all-knowing and incomprehensible (48-49) unmanifest consciousness, which is the critical component of his Cārcāyog type of \textit{bhakti}.


\end{document}

