%Ultimatives Tool zur Datierung:
%https://www.cc.kyoto-su.ac.jp/~yanom/pancanga/
%skp = ignored in edition
%skm = ignored in xml
%%%---2-DO---%%%:
% - add xml ids for cladistics
% - produce diplomatic transcripts for saktumiva
% - read Sarvangayogapradipika, Maya Burger! 
% - maybe add second ciritical edition of yogasvarodaya?!
% - grep-search alle Verse!!!!
% - Mss spreadsheet
% - additions to U2: make footnotes for the bahir mātrā-s: explaining the inventions of female deities and tell that this is "schwer interpretierbar"
% - Consider changing Lakṣya to Lakṣa
%%%%%%%%%%%%%%%%%%%%%%%%%%%%%%%%%%%%%%%%%
% Don't forget
% Siddhasiddhantapaddhati Yogic Body descriptions are followed by Rāmacandra
% Quotes of the Yogasvarodaya in the Yoga Karṇikā
% Rāmacandra more a compiler than an author!!!
% Identify quotes of YTB in Haṭhasanketacandrikā -- done :D
%%%%%%%%%%%%%%%%%%%%%%%%%%%%%%%%%%%%%%%%%%%
%MSS notes
%
%--B: i and ī are not differenciated
%--P: no punctuation no daṇdas nothing
%--U1: dot . serves as daṇḍa 
%--\L and \U2 very similar
%--figure out for U2: // ajapājapaḥ sahasra // 6000 //gha 0 16 pa 0 40// \U2?!?!?!?!?!?
%%%%%%%%%%%%%%%%%%%%%%%%%%%%%%%%%%%%%%%%%%
%
% Einleitung Ideen 
% - sprachliche Simplizität
% - Potenzial als Anfängertext
% - Großartige Einführung in die Textkritik -> Synoptische Edition 
% - Gelegenheit Yogasvarodaya und Yogatattvabindu zu edieren 
% - Historische Evidenz entweder für das königliche Leben in einer Maṭha in der Nähe von Benares während der Muslimischen Herrschaft, oder sogar Lehrtext für die Bildung junger Prinzen  
% - eines der raren Beispiele der engen Verknüfung mehrerer Texte 
% - eines der raren Beispiele der Prosaisierung eines metrischen Textes 
% - Anwendung rezenter Technologie! 
% - How the text was construed -> intermingling of Ysv and SSP
% - Martin Straube: "jeder kleine Dorfhäuptling kann Rāja genannt werden". 
%%%%%%%%%%%%%%%%%%%%%%%%%%%%%%%%%%%%%%%%%%%
\input{preamble.tex}
\FormatDiv{1}{\begin{center}\Large}{\end{center}}
\FormatDiv{2}{\begin{center}\small}{\end{center}}
\FormatDiv{3}{\bfseries}{.}
\title{Yogatattvabindu of Rāmacandra\\ A Critical Edition and Annotated Translation}
\date{\today}

\parindent=15pt
\begin{document}

% Zitiermöglichkeiten:
%\footcite[See][p.\,1]{goldstein01:_tibet_englis_diction_moder_tibet}
%\footnote{\cite{goldstein01:_tibet_englis_diction_moder_tibet}.}

\frontmatter
\thispagestyle{empty}
\begin{center}
  {\Large \emph{The Yogatattvabindu}}\\[3mm]
\end{center}



\newpage

\

\thispagestyle{empty}



\normalsize


\newpage


\begin{center}
\thispagestyle{empty}

\

\vskip 2mm

\begin{otherlanguage}{iast}
\LARGE \sanskritfont{Yogatattvabindu}
\end{otherlanguage}

\vskip .4cm

\Huge Yogatattvabindu \\[7mm]
\Large Critical Edition\\
with annotated Translation


\large

\vspace{3cm}

Von

Nils Jacob Liersch
\small
\vfill

\vfill

Indica et Tibetica Verlag \\ % $\cdot$ 
Marburg 2024

\vskip 6mm

\end{center}

\newpage
\newpage \ \thispagestyle{empty}
\small  \

\noindent

\
\vfill


\small
\noindent \textbf{Bibliographische Information Der Deutschen Bibliothek}

\noindent
Die Deutsche Bibliothek verzeichnet diese Publikation in der Deutschen Nationalbibliographie;
detaillierte bibliographische Informationen sind im Internet über http://dnb.ddb.de abrufbar.

\noindent
\textbf{Bibliographic information published by Die Deutschen Bibliothek}

\noindent
Die Deutsche Bibliothek lists this publication in the Deutsche Nationalbibliographie; detailed
bibliographic data is available in the Internet at http://dnb.ddb.de.  


\vskip 1cm

\noindent
\copyright\ Indica et Tibetica Verlag, Marburg 2024

\medskip

\noindent
Alle Rechte vorbehalten / All rights reserved

\medskip

\noindent
Ohne ausdrückliche Genehmigung des Verlages ist es nicht gestattet, das Werk oder einzelne Teile
daraus nachzudrucken, zu vervielfältigen oder auf Datenträger zu speichern.

\smallskip

\noindent
Apart from any fair dealing for the purpose of private study, research, criticism or review, no
part of this book may be reproduced or translated in any form, by print, photo form, microfilm, or
any other means without written permission. Enquiries should be made to the publishers.

\bigskip

\noindent
Satz: \ \ Nils Jacob Liersch \\
Herstellung: \ \ BoD – Books on Demand GmbH, Norderstedt  \\

\bigskip

\noindent
%\ISBN     

\normalsize

\newpage

%\maketitle
\clearpage
\tableofcontents
\addtocounter{page}{-1}
\thispagestyle{empty}
\clearpage


\mainmatter

\chapter{Introduction}
\cleardoublepage

\section{General remarks}
The \textit{Yogatattvabindu} is a premodern Sanskrit Yoga text on Rājayoga that was written in the first half of the seventeenth century\footnote{The dating of the text is discussed on p.\pageref{dating}.} in northern India.\footnote{The detailed discussion of the place of origin is found on p.\pageref{placeoforigin}.} The most salient feature of the work that makes it historically significant is its highly differentiated taxonomy of types of Yoga. In the \textit{Yogatattvabindu}'s introduction, most manuscripts name fifteen types of Yoga, presented as subtypes of Rājayoga. The text is a yogic compendium written in a mix of mainly prose and 41 verses in textbook-style, where its 58 topics topics are introduced in sections launched by recognizable phrases. Most sections deal with the subtypes of Rājayoga and their effects, but others also cover topics like yogic physiology and cosmogony.

The \textit{Yogatattvabindu} has not been discussed or considered in secondary literature on Yoga. The only exception is \citeauthor{birch2014} (2014: 415–416) who briefly described its list of fifteen Yogas in the context of the “fifteen medieval Yogas” and noted that a similar\footnote{My research suggests that list of fifteen Yogas in Nārāyaṇatīrtha’s \textit{Yogasiddhāntacandrikā} must be chronologically later than the ones found in the \textit{Yogatattvabindu} and its sources. As I will show in the discussion of the fifteen Yogas on p.\pageref{15yogas}, we have to assume that Nārāyaṇatīrtha saw the need to map the fifteen Yogas onto system of the \textit{Pātañjalayogaśāstra} due to their popularity among practitioners in his sphere of activity.} list occurs in Nārāyaṇatīrtha’s \textit{Yogasiddhāntacandrikā} (17th - 18th century), a commentary on the \textit{Pātañjalayogaśāstra} that integrates almost an identical taxonomy of yogas within the \textit{aṣṭāṅga} format. An incomplete account of the fifteen Yogas is found within the Sanskrit Yoga text \textit{Yogasvarodaya}, which is known only through quotations in the \textit{Prāṇatoṣinī} and \textit{Yogakarṇikā}.\footnote{Manuscripts under the name of \textit{Yogasvarodaya} seem to be lost. I was not able to allocate the manuscripts of the text in any manuscript catalogue at hand.} The \textit{Yogasvarodaya} provides a total of fifteen Yogas but names only eight of them in its introductory \textit{śloka}s. A complete account of the text is yet to be found and might be lost forever. The \textit{Yogasvarodaya} is the primary source and template for the compilation of the \textit{Yogatattvabindu}. Rāmacandra closely follows the content and structure by rewriting the \textit{Yogasvarodaya}’s \textit{śloka}s into prose. Due to the incomplete transmission of the \textit{Yogasvarodaya}, Rāmacandra’s \textit{Yogatattvabindu} is a natural and valuable starting point for an in-depth study of the taxonomy of the fifteen types of Yoga. The other source text that Rāmacandra used is the \textit{Siddhasiddhāntapaddhati} whose content he draws on, particularly in the last third of his composition. Another text that includes a similar taxonomy of twelve Yogas divided into three tetrads is Sundardās’s \textit{brāj bhāṣa} Yoga text named \textit{Sarvāṅgayogapradīpikā} which not just shares most of the types of Yogas but also many of the practices and contents found within the \textit{Yogatattvabindu} and \textit{Yogasvarodaya}.\footnote{For a comparative table of the complex Yoga taxonomies see table \ref{15yogastable} on p.\pageref{15yogastable}.}

These complex taxonomies that emerged during the 17th and 18th centuries crossed sectarian divides and were adapted to the specific needs of different authors and traditions. The \textit{Yogatattvabindu} thus encapsulates the diversity of Haṭha- and Rājayoga types and teachings after the \textit{Haṭhapradīpikā} (15th century) that were adopted by a broad spectrum of religious traditions and strata of Indian society. In the particular case of the \textit{Yogatattvabindu}, there are various statements throughout the text that reveal a strategy to detach Yoga from its renunciate connotations and to enforce the supremacy and universality of Rājayoga as a practice that can yield the highest benefits even for practitioners who enjoy worldly pleasures and an extravagant lifestyle. Textual evidence suggests the possibility that \textit{Yogatattvabindu} may be a unique example of a Rājayoga text that was composed for warrior aristocracy and members of an royal court. 

In addition, the analysis of the \textit{Yogatattvabindu} and the historical retracig of its teachings provides insight into a complex network of at least twenty texts,\footnote{This intertextual network which shares those specific teachings consists of the \textit{Netratantra}, \textit{Śāradatilakatantra}, \textit{Sarvadurgatipariśodhanatantra}, \textit{Ūrmikaulārṇavatantra}, \textit{Tantrāloka}, \textit{Manthanabhairavatantra}, \textit{Śārṅgadhārapaddhati}, \textit{Vivekamārtaṇḍa}, \textit{Śivayogapradīpikā}, (recensions of the \textit{Haṭhapradīpikā}), \textit{Amaraughaśāsana}, \textit{Yogasvarodaya}, \textit{Sarvāṅgayogapradīpikā}, \textit{Nityanāthapaddhati}, \textit{Siddhasiddhāntapaddhati}, \textit{Yogatattvabindu}, \textit{Yogacūḍāmaṇyupaniṣad}, \textit{Maṇḍalabrāhmaṇopaniṣat}, \textit{Haṭhatattvakaumudi} and \textit{Haṭhasaṃketacandrikā}.} all of which include one specific set of yoga theorems and practices with minor deviations - three to five \textit{cakra}s, sixteen \textit{ādhāra}s, two to five \textit{lakṣya}s, and five \textit{vyoma}s. This intertextual network spans at least an entire millennium. It begins in early śivaite Tantras such as the \textit{Netratantra} and ends in the large premodern Yoga compendiums like the \textit{Haṭhatattvakaumuḍī} and \textit{Haṭhasaṅketacandrikā}. The examination of this network provides insights into the history of the related yoga traditions and enables, for example, the reconstruction of the genesis of individual yoga categories mentioned in the fifteen Yogas, such as Lakṣyayoga, whose techniques were originally taught in early śivaite Tantras, but were only labeled as a separate type of yoga from the 17th century onwards.

One printed edition of the \textit{Yogatattvabindu} was published in 1905 with a Hindi translation and based on an unknown manuscript(s). This publication has the title ’\textit{Binduyoga}’ confirmed by the printed text’s colophon. However, as I discuss in the course of the introduction, the text was likely known as \textit{Yogatattvabindu}. The consulted manuscripts contain significant discrepancies, structural differences and variant readings between them and the printed edition. Furthermore, the manuscripts are scattered over the Indian subcontinent, which suggests that it was widely transmitted at some point. Lenghty passages of the \textit{Yogatattvabindu} are quoted without attribution in a text called \textit{Yogasaṃgraha} and Sundaradeva’s \textit{Haṭhasaṅketacandrikā}. A critical edition will undoubtedly improve on the published edition and shed further light on the transmission of this important work.

This book contains an introduction, critical edition and annotated translation of the \textit{Yogatattvabindu}. The introduction discusses provenance, authorship and the audience of the \textit{Yogatattvabindu}. A comprehensive discussion of the taxonomy of the fifteen Yogas based on the critical edition of the \textit{Yogatattvabindu}, together with a close examination of the above-mentioned related texts with similar taxonomies, aims to establish their position within the broader history of yoga and particularly elucidates the development of Haṭha- and Rājayoga traditions in the late medieval period. The remainder of the introduction contains an overview of the manuscript evidence and the editorial policies underlying the edition.

\section{Dating the \textit{Yogatattvabindu}}
\label{dating}
The oldest dated manuscript of the \textit{Yogatattvabindu} \getsiglum{N1}\footnote{For a description of the manuscript see  p.\pageref{n1description}.} was written in Nepal \textit{saṃvat} 837, which is 1716 CE. Since the text of this manuscript is missing a significant and lengthy passage (ca. 25\% of the entire text) and contains various corruptions, one can assume that some time had passed from the original composition for the transmission to deteriorate to this extent. Therefore, it is likely that the work was composed at least a few decades before the creation of this Nepalese manuscript, perhaps sometime in the 17th century. The discovery that Sundaradeva's \textit{Haṭhasaṅketacandrikā} quotes a lengthy passage of the \textit{Yogatattvabindu} without attribution confirms this suspicion. The passages quoted from the \textit{Yogatattvabindu} include the teachings on the sixteen \textit{ādhāra}s\footnote{\citetitle{hathasamketacandrikajodhpur} (ms. no. 2244, f. 95r l. 3 -- f. 96r l. 4).} and the teachings on Lakṣyayoga and its subtypes.\footnote{\citetitle{hathasamketacandrikajodhpur} (ms. no. 2244, f. 124r l. 7 -- f. 125r l. 3).} The dating of the \textit{Haṭhasaṅketacandrikā} just recently had to be revised due to the discovery that some first-hand notes surrounding the main text of the Ujjain \textit{Yogacintāmaṇi} were in all likelihood borrowed from Sundaradeva's \textit{Haṭhasaṅketacandrikā}.\footnote{Cf. \citeauthor{birch2024} (2024:52-54).} \citeauthor{birch2018proliferation} (2018) dated the Ujjain \textit{Yogacintāmaṇi} to 1659 CE.\footnote{Cf. \citeauthor{birch2018proliferation} (2018: 50 [n. 111]).} Thus, the \textit{terminus ante quem} for the compilation of the \textit{Haṭhasaṅketacandrikā} is 1659 CE which automatically makes it also the \textit{terminus ante quem} for the \textit{Yogatattvabindu} and the \textit{Yogasvarodaya}, due to the fact that Sundaradeva quoted from the \textit{Yogatattvabindu} and Rāmacandra quoted from and rewrote the contents of the \textit{Yogasvarodaya}. Thus, we can safely assume that the \textit{Yogatattvabindu} was written in the course of the first half of the 17th century or earlier. Because of that Rāmancandra's main source text \textit{Yogasvarodaya} must have been written even earlier.

\subsection{Implications for the dating of the \textit{Yogasvarodaya} and the \textit{Siddhasiddhāntapaddhati}}
Furthermore, \citeauthor{mallinsononline2013}\footnote{Cf. \fullcite{mallinsononline2013}.} estimated the age of the \textit{Siddhasiddhāntapaddhati} to circa 1700. Due to the above-mentioned new date of the \textit{Haṭhasaṅketacandrikā} and because Rāmacandra extensively quotes from \textit{Siddhasiddhāntapaddhati} the new terminus \textit{terminus ante quem} for the dating of the \textit{Siddhasiddhāntapaddhati} likewise must be set to 1659 CE. Thus, the \textit{Siddhasiddhāntapaddhati} was also likely composed during the first half of the 17th century or even ealier.

\chapter{The complex medieval yoga taxonomies}
\label{yogas_list}
\clearpage



\section{The rise of diversity: The increasing complexity of Yoga teaching systems in late medieval and pre-colonial India}

In diesem Kapitel soll es darum gehen, dass zwischen dem 17. und 18. Jh. in Indien parallel zu einer Populariserung des Yoga in breiten Schichten der gesellschaft jenseits der asketischen Traditionen eine allgemeine Entwicklung zu beobachten ist, die sich in gesteigerter Komplexität äußert. In den damals zirkulierenden Texten kommt es zu einer Steiugerung der Anzahl der gelehrten Cakras, Āsanas, Kumbhakas, aber auch die Taxonomien der einzelnen Yogakategorien die gelehrt werden nehmen an Komplexität zu. 

\subsection{Ein kurzer Überblick der in texten zu Verwenung kommenden Taxonomien.}


\section{Comparative Analysis of the complex Yoga taxonomies}

The similarities between the Yoga taxonomies of Rāmacandra's \textit{Yogatattvabindu}, his source text, the \textit{Yogasvarodaya} as well as the taxonomies laid out by Nārāyaṇatīrtha in his \textit{Yogasiddhāntacandrikā} and Sundardās' \textit{Sarvāṅgayogadīpikā} which all emerged within the same time period have been initially observed and discussed briefly by \citeauthor{birch2014} (2014)\footnote{See \parencite[415-416]{birch2014}.} In the following chapter, the lists and their items are examined in a comparative analysis.

A complete comparative description of all yoga categories used in the literature would go far beyond the scope of this work. However, with this presentation I hope to adequately cover our understanding of the concepts of different yoga categories circulating in the literature of the 17th - 18th centuries.

The analysis will follow the structure of the individual yogas outlined in the \textit{Yogatattvabindu}. Each yoga will be described based on the explanations in the \textit{Yogatattvabindu}, and its content will be compared with the explanations of the corresponding yoga in the texts with similar taxonomies. The comparison will broaden and clarify our understanding of the respective spectrum of meanings of the individual yoga categories in the discursive field of the authors of the texts containing the taxonomies. This comparison results in the documentation of the discursive web of word usage of various yoga categories between the 17th and 18th centuries CE, most probably mainly localised in central northern India.\footnote{The complex taxonmies evolved and circulated most likely in central northern India. For a detalled discussion see p.\pageref{location}.} Individual yoga categories that do not appear in the list of the \textit{Yogatattvabindu} but are listed in the other texts with complex taxonomies will also be covered and outlined. In addition, yoga categories that do not appear in any of the analysed lists but are nevertheless mentioned in the texts will also be covered so that this analysis attempts to approximate the overall picture of all yoga categories used during the period under consideration as closely as possible. However, it is essential to emphasise that the comparison of yoga categories focuses primarily on those texts that contain complex yoga taxonomies and cannot claim to be exhaustive. Although the analysis and comparison of the yoga categories can be extended to other yoga texts, locations and time periods if necessary or valuable, the restriction to the complex yoga taxonomies should be maintained to prevent this already complex endeavour going ad absurdum.\footnote{There are hundreds, if not thousand of Sanskrit and vernacular texts from different times and different regions of India, which operate with these categories.}      

\begin{table}[h]
    \centering
    \begin{tabularx}{\textwidth}{>{\raggedright\arraybackslash}p{0.05\textwidth}XXXX}
        \toprule
        No. & \textit{Yogatattvabindu} & \textit{Yogasvarodaya} & \textit{Yogasiddhāntacandrikā} & \textit{Sarvāṅgayogadīpikā} \\
        \midrule
        1. & \textit{kriyāyoga} & \textit{kriyāyoga} & \textit{kriyāyoga} & \textit{\textbf{bhaktiyoga}} \\
        2. & \textit{jñānayoga} & \textit{jñānayoga} & \textit{caryāyoga} & \textit{mantrayoga} \\
        3. & \textit{caryāyoga} & \textit{karmayoga} & \textit{karmayoga} & \textit{layayoga} \\
        4. & \textit{haṭhayoga} & \textit{haṭhayoga} & \textit{haṭhayoga} & \textit{carcāyoga} \\
        5. & \textit{karmayoga} & \textit{dhyānayoga} & \textit{mantrayoga} & \textit{\textbf{haṭhayoga}} \\
        6. & \textit{layayoga}  & \textit{mantrayoga} & \textit{jñānayoga} & \textit{rājayoga} \\
        7. & \textit{dhyānayoga} & \textit{urayoga}   & \textit{advaitayoga} & \textit{lakṣayoga} \\
        8. & \textit{mantrayoga} & \textit{vāsanāyoga} & \textit{lakṣyayoga} & \textit{aṣṭāṅgayoga} \\
        9. & \textit{lakṣyayoga} & -                   & \textit{brahmayoga} & \textit{\textbf{sāṃkhyayoga}} \\
        10. & \textit{vāsanāyoga} & -                   & \textit{śivayoga} & \textit{jñānayoga} \\
        11. & \textit{śivayoga} & -                    & \textit{siddhiyoga} & \textit{brahmayoga} \\
        12. & \textit{brahmayoga} & -                  & \textit{vāsanāyoga} & \textit{advaitayoga} \\
        13. & \textit{advaitayoga} & -                 & \textit{layayoga} & - \\
        14. & \textit{siddhayoga} & -                  & \textit{dhyānayoga} & - \\
        15. & \textit{rājayoga} & - [\textit{rājayoga}]& \textit{premabhaktiyoga} & - \\
        \bottomrule
    \end{tabularx}
    \caption{Complex Taxonomies of Yoga in Yoga Texts of the 17th - 18th Centuries}
    \label{tab:complextaxonomies}
\end{table}

\section{Kriyāyoga - \uproman{1}}

Kriyāyoga\footnote{See section II. on p.\pageref{kriyayogastart}-\pageref{kriyayogaend}.} is the first Yoga within the list of fifteen Yogas presented by Rāmacandra and his source text \textit{Yogasvarodaya}. Remarkably, Nārāyaṇatīrtha also positions Kriyāyoga at the first position within the list of fifteen Yogas in his \textit{Yogasiddhāntacandrikā}. Sundardās, on the other hand, omits Kriyāyoga within his taxonomy.

\subsection{The concept of Kriyāyoga in the \textit{Yogatattvabindu}}

Since Rāmacandra refers to all fifteen Yogas as variants of Rājayoga in his initial definition of Yoga, and no explicit hierarchy is recognisable from his formulations in the text, all variants of Rājayoga appear to have been regarded by him as equally effective. All Yogas aim towards the same goal: long-term durability of the body (\textit{bahutarakālaṃ śarīrasthitiḥ}). The positioning of Kriyāyoga does not initially provide any information about the efficiency or the assignment of differently talented practitioners to a particular type of Yoga, as was the case in the older fourfold taxonomies.\footnote{According to \citetitle{amaraugha2024}\textit{prabodha} 18-24, Mantrayoga is best suited for the weak, Layayoga for the average, Haṭhayoga for the talented and Rājayoga for the exceptionally talented practitioner. In \citetitle{datta2024} 14, one finds the statement that the lowest practitioner should perform mantra yoga, which is then also referred to as the lowest Yoga. \citetitle{mallinson2007} 12-28 expands this fourfold scheme of Yogas and practitioners with a temporal dimension. The weak practitioner needs twelve years to succeed with Mantrayoga, the average practitioner needs eight years with Laya, the able practitioner six years with Haṭha and the exceptional practitioner three years with Rājayoga} Implicit hierarchical aspects are nevertheless present - although all Yoga types are a type of Rājayoga, Rāmacandra nonetheless places Rājayoga in the final and topmost position of his taxonomy.
The only apparent reason why Rāmacandra specifies Kriyāyoga as the first Yoga seems to be that his primary source text, whose content structure he largely follows,\footnote{see the chapter on ``structural inconsistencies'' on p.\pageref{struktur},} specifies this type of Yoga as the first.

The passage on Kriyāyoga in the \textit{Yogatattvabindu} is relatively short. The four verses presented by Rāmacandra are quoted without attribution from the \textit{Yogasvarodaya}. A prose section repeats the content of the verses. By definition, Kriyāyoga in \textit{Yogatattvabindu} is ``liberation through [mental] action'' (\textit{kriyāmuktir ayaṃ yogaḥ}). In contrast to Rāmacandra's worldly definition of Rājayoga and its subcategories, here, liberation (\textit{mukti}) overrides this initial goal. In addition, the practitioner achieves ``success in one's own body'' (\textit{svapiṇḍe siddhidāyakaḥ}). The method of Kriyāyoga involves restraining any [mental] wave before an action. This restraint consists of reducing negative [mind-]waves and cultivating positive ones. Noticeably, the number of negative waves significantly exceeds the number of positive waves.

\begin{table}[h]
    \centering
    \begin{tabularx}{\textwidth}{XX}
        \toprule
        \textbf{Mental waves to be cultivated} & \textbf{Mental waves to be reduced} \\
        \midrule
        Patience (\textit{kṣamā}) & Envy (\textit{matsārya}) \\
        Discrimination (\textit{viveka}) & Selfishness(\textit{mamatā})\\
        Equanimity (\textit{vairāgya}) & Cheating (\textit{māyā})\\
        Peace (\textit{śānti}) & Violence (\textit{hiṃsā})\\
        Modesty (\textit{santoṣa}) & Intoxication (\textit{mada})\\
        Desirelessness (\textit{niṣpṛha}) & Pride (\textit{garvata})\\
        & Lust (\textit{kāma}) \\
        & Anger (\textit{krodha}) \\
        & Fear (\textit{bhaya})\\
        & Laziness (\textit{lajjā})\\
        & Greed (\textit{lobha})\\
        & Error (\textit{moha})\\
        & Impurity (\textit{aśuci})\\
        & Attachment and aversion (\textit{rāgadveśau}) \\
        & Disgust and laziness (\textit{ghṛṇālasya})\\
        & error (\textit{bhrānti})\\
        & Deceit (\textit{daṃbha})\\
        & Envy (repeatedly) (\textit{akṣama})\\
        & Confusion (\textit{bhrama})\\
        \bottomrule
    \end{tabularx}
    \caption{Mental waves to be cultivated and reduced in Rāmacandra's Kriyāyoga}
    \label{tab:waves}
\end{table}

The one who cultivates positive [mind-]waves and reduces the negative is called a \textit{kriyāyogī}. In the prose passage of the section, the term \textit{bahukriyāyogi} is used. The term is unprecedented in the rest of the yoga literature and presumably intends to express many reduced and cultivated waves.
\newpage

\subsection{The concept of Kriyāyoga in the \textit{Yogasvarodaya}}
A closer examination of the Kriyāyoga section in the \textit{Yogasvarodaya} reveals Rāmancandra's reductionism since he excludes significant aspects of the original concept of the \textit{Yogasvarodaya}'s Kriyāyoga.

%YK 1.214-216

\begin{quote}
\textit{dhyānapūjādānayajñajapahomādikāḥ kriyāḥ} |\\
\textit{kriyāmuktimayo yogaḥ svapiṇḍe siddhidāyakaḥ}\footnote{svapiṇḍe siddhidāyakaḥ YTB] sapiṇḍisiddhidāyakaḥ YSv sapiṇḍisiddhidāyakaḥ YK} || 1 ||

(1) Actions are meditation, ritual veneration, donation, recitation, fire sacrifice, etc. 
The Yoga made of liberation through action[s] bestows success in one's own body. 

\textit{yat karomīti saṅkalpaṃ kāryārambhe manaḥ sadā} |\\
\textit{tat sāṅgācaraṇaṃ kurvan kriyāyogarato bhavet} || 2 ||

(2) ``Whatever I do'' at the beginning of an action, the mind always has intention.  
Doing that undertaking with all its parts, one becomes established in Kriyāyoga. 

\textit{kṣamāvivekavairāgyaśāntisantoṣanispṛhāḥ} |\\
\textit{etad yuktiyuto yo'sau kriyāyogo nigadyate} || 3 ||

(3) Patience, discrimination, equanimity, peace, modesty, desirelessness:
The one endowed with these means is said to be a Kriyāyogī.

\textit{mātsaryaṃ mamatā māyā hiṃsā ca madagarvitā} |\\
\textit{kāmaḥ krodho bhayaṃ lajjā lobho mohas tathā'śuciḥ} || 4 ||

(4) Envy, selfishness, cheating, violence, intoxication and pride,
lust, anger, fear, laziness, greed, error, and impurity.

\textit{rāgadveṣau ghṛṇālasyaśrāntidambhakṣamābhramāḥ} |\\
\textit{yasyaitāni na vidyante kriyāyogī sa ucyate} || 5 ||

(5) Attachment and aversion, disgust and laziness, error, deceit, envy [and] confusion:
Whoever does not experience these is called a Kriyāyogī.

\textit{sa eva muktaḥ sa jñānī caṇḍināśena īśvaraḥ} |\\
\textit{kriyāmuktikaro yo'sau rājayogaḥ sa muktidaḥ} || 6 ||(om. YK)

(6) He alone, the wise one, the lord, through the destruction of impetuous [behaviour]
who performs the liberation through action[s] is liberated. This Rājayoga is the giver of liberation.

\textit{yāvan mano layaṃ yāti kṛṣṇe svātmani cinmaye} | \\ 
\textit{bhaved iṣṭamanā mantrī japahomau samabhyaset} || 7 ||\footnote{7ab \approx \citetitle{rudrayamala1937} 38.58cd.}(om. YSv) 

(7) Until the mind enters absorption [and] would be in Kṛṣṇa, in one's own self, filled with consciousness,
the mantra practitioner (\textit{mantrin}) should practise recitation and fire sacrifice with an aspiring mind. 

\textit{vidite paratattve tu samastair niyamair alam} |\\
\textit{tālavṛntena kiṃ kāryaṃ lavdhe malayamārute} || 8 ||\footnote{\approx \citetitle{kularnavatantra} 9.28 \& \citetitle{yuktabhavadeva} 1.80.} (om. YSv) 

(8) When the highest principle has been realised through all the \textit {niyama}s, as is proper,
Why should one wave the palm frond when the wind from the Himalayas has already reached?

\textit{tāvat karmmāṇi kurvanti yāvajjñānaṃ na vidyate} |\\ 
\textit{jñāne jāte pareśāni karmākarma na vidyate} || 9 ||(om. YSv) 

(9) As long as [regular?] actions are performed, so long realisation is unknown.
When knowledge ensues, oh, Supreme Goddess, neither action nor non-action is known.
\end{quote}

These verses\footnote{The numbering used here was introduced by me for practical reasons and does not correspond to the original numbering of the verses in the citations of the source texts. The \textit{Prāṇatoṣiṇī} does not number the verses at all. The verses can be found in the printed edition of the \textit{Prāṇatoṣiṇī} on p. 831. The verses here are in the \textit{Yogakarṇikā} with the numbering 1.209-216 and can be found in the edition on p. 17.} stem from the only two currently available sources of the \textit{Yogasvarodaya}, namely the quotations from the \textit{Prāṇatoṣiṇī}\footnote{A considerable part of the \textit{Yogasvarodaya} is quoted with source reference (\textit{yogasvarodaye}).} and the \textit{Yogakarṇikā}.\footnote{Normally the \textit{Yogakarṇikā} quotes its sources. This passage is one of the few exceptional cases in which the verses have been taken from the \textit{Yogasvarodaya} without citing the source. However, this passage ends after verse 1.216 with ``\textit{iti yogasaṅketāḥ |}''.} The quotations of both texts essentially correspond, but the last verses of the passage differ. It cannot be ruled out that the last three verses of the \textit{Yogakarṇikā} in particular come from a different source and were not present within the \textit{Yogasvarodaya}. However, their content is so closely interwoven with the preceding verses that this scenario can be considered unlikely.

The main difference to the Kriyāyoga that Rāmacandra has constructed from these verses is the definition of the actions (\textit{kriyāḥ}) mentioned immediately at the beginning of the verses, of which the actions (\textit{kriyā}s) of Kriyāyoga is then predominantly composed, namely of (1) meditation, (2) ritual worship of God, (3) offerings, (4) recitation and (5) fire sacrifice, etc. Furthermore, while Rāmacandra declares the elements mentioned in the table \ref{tab:waves} as waves (\textit{kallola}) of the mind which are either required to be cultivated or reduced before any action is executed, the same elements are conceptualised in the \textit{Yogasvarodaya} as the intentions (\textit{saṅkalpa}) preceding the previously defined actions (\textit{kriyā}s), which should be observed.

In the three verses concluding this section, which are only handed down in the \textit{Yogakarṇikā}, the practitioner is referred to as \textit{mantrin} and should perform recitation and fire offerings until entering absorption (\textit{laya}).

Thus, this concept of Kriyāyoga in the \textit{Yogasvarodaya} is a clear reference to the \textit{kriyāpāda}\footnote{See e.g. \citeauthor{ganesan2016saiva} (2016) and \citetitle{mrgendragama} (Ed. pp. 1-205).} of the Śaiva \textit{āgama}s. The Śaiva \textit{āgama}s are collections of various tantric traditions, written in Sanskrit or Tamil, in which cosmology, epistemology, philosophical teachings, various practices such as meditation or yoga, mantra recitation, worship of the gods, etc. are described. These texts\footnote{The fourfold division of \textit{pāda}s is only present in a limited number of Āgamas: \textit{Kiraṇa}, \textit{Suprabheda}, \textit{Mṛgendra} and \textit{Mataṅgaparameśvara} (as Upāgamas), see \citeauthor{brunner1994place} (1993: 225-461) for an overview.} usually consist of four sections (\textit{pāda}s): The \textit{jñānapāda} (knowledge section), \textit{kriyāpāda} (action section), \textit{caryāpāda} (behaviour section) and the \textit{yogapāda} (yoga section).\footnote{The order or the \textit{pāda}s varies, but the \textit{yogapāda} is always the last.} It can be no coincidence that \textit{jñāna°}, \textit{kriyā°} and \textit{caryā°} were each integrated as a separate yoga category within the taxonomy of the fifteen yogas\footnote{see p.\pageref{intro}.}. The \textit{kriyāpāda} is the section of a Śaiva \textit{āgama} that describes rules and practices for the performance of various rituals such as the significant initiation (\textit{dīkṣa}), ceremonies and worship of the gods. Additionally, \textit{prāṇāyāma} techniques and meditations are often found as parts of these rituals. There are also explanations of the nature of \textit{mudrā}s, \textit{maṇḍala}s and \textit{mantra}s. Furthermore, various characteristics of different types of Śaiva initiates\footnote{These are \textit{samayin, putraka, sādhaka, ācārya,} and \textit{astrābhiṣeka}.} can be found here.\footnote{See \citeauthor{ganesan2016saiva} (2016) for a general overview of the four \textit{pāda}s. One of the few Śaiva \textit{āgama}s that has been edited and translated into a Western language (French) is the \citetitle{mrgendragama}. For this see \citeauthor{mrgendragama} (1962) \& \citeauthor{mrgendragamabrunner} (1985).} The \textit{kriyā}s mentioned at the beginning of the \textit{Yogasvarodaya} - meditation, ritual veneration, donation, recitation, fire sacrifice, etc. have hardly deniable parallels to the \textit{kriyāpāda}s of the Śaiva \textit{āgama}s and thus could have their reception-historical roots precisely there. The other part, however, which describes the cultivation or reduction of certain mental configurations preceding all actions (\textit{saṅkalpa}) or [mental] waves (\textit{kallola}), I have not yet been able to locate in the Śaiva \textit{āgama}s, but possibly they are a simplyfied rendering of the Pātañjalean model of Kriyāyoga that was passend on in hitherto unknown traditions that practiced this type of Kriyāyoga.

One other possible historical link which should not remain unmentioned might be the model of Kriyāyoga found in the \textit{Uddhavagīta}\footnote{See i.e., \citeauthor{uddhavagita2007} (2007).} which is a part of the famous \textit{Bhāgavatapurāṇa}\footnote{See i.e., \citeauthor{bhagavata} (1950).}. Here, in chapter XXII.1-55 Kṛṣṇa describes a Vaiṣṇava form of Kriyāyoga in response to a request by his disciple Uddhava. The practice entails a very complex and devotional ceremonial veneration of the deity through offerings such as flowers and food, accompanied by the recitation of prescribed mantras, meditation, and the ritual consecration of the deity, among other rites. According to the text, this type of Yoga is the most beneficial for women and the working class (22.4) and is considered a means for liberation from the fetters of Karma (22.5). The Kriyāyoga described here is presented to be in line with both the Vedas and the Tantras, considering enjoyment (\textit{bhukti}) and liberation (\textit{mukti} and is promised to bestow perfection in both this life and the next, by the Lord's grace (22.49).  

\subsection{The concept of Kriyāyoga in the \textit{Yogasiddhāntacandrikā}}

The Kriyāyoga in Nārāyaṇatīrtha's commentary on \textit{Pātañjalayogaśāstra} entitled \textit{Yogasiddhāntacandrikā} presents Kriyāyoga as the first of his fifteen Yogas, which he locates in Pātañjalayoga. The term Kriyāyoga occurs in \textit{Pātañjalayogaśāstra} 2.1. According to the introduction to this Sūtra, in the \textit{bhāṣya}-part of the\textit{Pātañjalayogaśāstra}, Kriyāyoga is the means by which someone with a distracted mind can also attain Yoga (\textit{vyutthitacitto 'pi yogayuktaḥ}). In the \textit{sūtra} itself, Kriyāyoga is defined as follows:
\begin{quote}  
  \textit{tapaḥsvādhyāyeśvarapraṇidhānāni kriyāyogaḥ} |\\
  \citetitle{yogasutra} 2.1  \\
\end{quote}

Kriyāyoga, or ``yoga through action'', consists of three elements. Namely, abstinence (\textit{tapas}), which according to \textit{bhāṣya} should be practised both mentally and physically, the repetition of \textit{mantra}s or the study of sacred literature (\textit{svadhyāya}) and devotion to God (\textit{īśvarapraṇidhāna}).
According to \citetitle{yogasutra} 2.2, these three elements of Kriyāyoga should lead the practitioner to attain \textit{samādhi} by reducing the so-called \textit{kleśa}s. This explanatory model is also used by Nārāyaṇatīrtha (\citeauthor{yogacandrika}, 2000:71). The five \textit{kleśa}s consist of ignorance (\textit{avidyā}), self-centredness (\textit{asmitā}), attachment (\textit{rāga}), aversion (\textit{dveṣa}) and fear of death (\textit{abhiniveśa}). 
All three main components of Patañjali's Kriyāyoga are not mentioned in the \textit{Yogatattvabindu} and \textit{Yogasvarodaya}. Nevertheless, a practice similar to the reduction of the \textit{kleśa}s can also be found here. Although the specific fear of death (\textit{abhiniveśa}) is not mentioned, the more general term for fear (\textit{bhaya}) is cited.\footnote{The details of Nārāyaṇatīrtha's understanding of Kriyāyoga have already be discussed by \citeauthor{penna2004} (2004): 62-66 and will therefore not be covered here again.}
The Kriyāyoga in \textit{Yogatattvabindu} and \textit{Yogasvarodaya} could, therefore, be perhaps regarded as a degenerated or simplified variant of the Pātañjalean model, which restricts itself predominantly to the aspect of the reduction of negative waves of the mind, which is comparable to the reduction of \textit{kleśa}s and adds the aspect of cultivating positive ``waves'' to be mix. In both systems, Kriyāyoga is a means for liberation.\footnote{The Kriyāyoga of the \citetitle{yogasutra} will not be dealt with in detail here, as this has already been done in countless academic and informal publications. For the \textit{sūtra}s related to Kriyāyoga and Patañjali's autocommentary in Sanskrit with English translation, see \citeauthor{yogasutra} (1983): 113 et seqq. For a comprehensible and more accessible overview, see \citeauthor{bryant2009} (2009): 170 et seqq.}

\subsection{Kriyāyoga in the complex late-medieval Yoga taxonomies}

The analysis of Kriyāyoga within the taxonomies of fifteen yogas shows two distinct models. One is Nārāyaṇatīrtha's model, which draws directly on the Kriyāyoga of \textit{Pātañjalayogaśāstra}. Additional śaiva influences characterise the other model of Kriyāyoga that seems to have been locally prominent in the 17. - 18. century C.E. The precisely defined \textit{kriyā}s of the \textit{Yogasvarodaya} must be historically linked to the \textit{kriyāpāda}s of the Śaiva \textit{āgama}s, whereby the core practice of reducing and cultivating specific mental configurations before any action is loosely associated with the Kriyāyoga of the \textit{Pātañjalayogaśāstra}. The observation that the \textit{kriyā}-, \textit{caryā-}, and \textit{jñānayoga}s, are an allusion to the \textit{kriyā}-, \textit{caryā-}, \textit{jñāna-} and \textit{yogapāda}s of the Śaiva \textit{āgama}s, shows that Nārāyaṇatīrtha, as a proponent of the \textit{Pātañjalayoga}, was most likely not the originator of the fifteenfold taxonomy, but rather that the taxonomy of the fifteen yogas originated from local discourses around the authors und had achieved such local popularity at the time that Nārāyaṇatīrtha forced the fifteenfold taxonomy into Patañjali's \textit{Yogaśāstra} in order to show that the Yogaśāstra \textit{par excellence} and all those varieties of Yogas that were discussed in his sphere are in truth already present in the ``classical'' system of Patañjali.

\subsection{Popularisierung des Kriyāyoga im globalen Kontext}

Die vergleichsweise einzigartigen Abhandlungen über Kriyāyoga, welche sich in der Yogaliteratur ab dem 17. Jh.\footnote{The terminus \textit{ad quem} for the \textit{Yogasvarodaya} is 1659 CE, see p.\pageref{dating} for the details.} nur im \textit{Yogasvarodaya} sowie Rāmacandra's \textit{Yogatattvabindu} zeigen, welche offenbar, wenn auch nicht gänzlich vom Pātañlaja-Modell abweichen, und, wie gezeigt, deutliche Einflüsse tantrischer Abstammung aufweisen, können bis auf Weiteres als Randphänomene betrachtet werden. Die überwältigende Mehrheit der im zweiten Jahrtausend n. u. Z. verfassten Sanskrit Yogatexte greift auf das im \textit{Pātañjalayogaśāstra} propagierte Modell des Kriyāyoga zurück. Dementsprechend war es auch vor allem die Erschließung des \textit{Yogasūtra}s im Westen, beginnend mit der Übersetzung von Henry Thomas Colebrooke im Jahr 1805\footnote{See \parencite{colebrooke2014} for a detailled discussion.} welche dafür sorgte, dass das darin enthaltene Konzept des Kriyāyoga auch im Westen lange Zeit das Verständnis des Begriffes im akademischen und informellen Diskurs dominierte. 

Dies änderte sich erst mit dem globalen Erfolg und der Popularität von Paramahaṃsa Yogānanda (1893-1952) und der im Jahre 1920 von ihm gegründeten \textit{Self Realization Fellowship}, welcher, gemessen an den weiter oben skizzierten historisch nachweisbaren Formen des Kriyāyoga, eine neuartige Yogapraxis unter dem Oberbegriff Kriyāyoga verbreitete. Hierdurch wurde das Bedeutungsspektrum des Begriffes Kriyāyoga signifikant verändert und erweitert. Neben diversen Büchern die Yogānanda veröffentlichte, war es vor allem das im Jahr 1946 erschienene Buch \citetitle{autobioyogi}, die Autobiographie von Yogānanda selbst, welches maßgeblich Yogānandas Erfolg den Weg ebnete. Bis zum heutigen Tag gilt dieses Werk als Klassiker in der populären Yogaliteratur, befindet sich seit über siebzig Jahren im Druck und wurde in mehr als 50 Sprachen (Stand Jan. 2019) übersetzt.\footnote{Beleg!} Außerdem existiert gegenwärtig eine globale Anhängerschaft. Yogānanda, seine Bücher, seine Anhänger und die zahlreichen Bücher seiner Anhänger machten diese innovative Form des Kriyāyoga jenseits des indischen Subkontinents bekannt. Der Ursprung dieser Art von Kriyāyoga wurde angeblich bereits von Yogānandas Vorgängern, namentlich Lahiri Mahasaya (1828-1895) und Śrī Yukteśvar Giri (1855-1936),\footnote{Cf. \citeauthor{govindan2010} 2010:51-52.} zum zentralen Oberbegriff der Yogapraxis dieser Tradtionslinie stilisiert. 

Einer von Yogānandas Zeitgenossen war Svāmī Śivānanda Sarasvatī (1987-1963), welcher in ähnlicher Weise eine neue Form des Kriyāyoga propagierte. Zwar basierte sein Kriyāyoga zunächst größtenteils auf dem Pātañjalayoga-Modell, wurde aber unter dem gleichen Oberbegriff mit Praktiken des Haṭhayoga erweitert und von Yogānandas Modell beeinflusst. Diese Erweiterung und Integration von neuen Praktiken unter den Oberbegriff Kriyāyoga setzte sich bei seinen Schülern, allen voran Svāmī Satyānanda Sarasvatī (1923-2009), dem Gründer der berühmten \textit{Bihar School of Yoga} (seit 1962), in exzessiver Weise fort.

Die hieraus resultierende Popularität des Kriyāyoga löste eine globale Welle aus und inspirierte weitere Akteure, die wiederum ähnliche, aber teils anders nuancierte Kriyāyoga-Systeme entwickelten. Hier ist beispielsweise S.A.A. Ramaiah zu nennen, der 1952 die \textit{Kriya Babaji Yoga Sangam} gründete. Auch in diesem Fall gibt es eine globale Anhängerschaft.\footnote{Cf. \cite{kriyababajiyoga}.}

Es waren die oben genannte Akteure, allen voran Yogānanda, welche für die globale Popularisierung dieser neuen Form des Kriyāyoga sorgten, sodass deren Konzepte im rezenten öffentlichen Diskurs mindestens genauso bekannt, wenn nicht sogar bekannter sind, wie das Kriyāyoga des \textit{Pātañjalayogaśāstra}.

Diese ab dem Beginn des 19. Jahrhunderts nachweislichen Formen des Kriyāyoga, sind, wie gezeigt werden wird, ein Sammelbecken für innovative Kombinationen und Weiterentwicklungen von zahlreichen, bereits in der mittelalterlichen bis vorkolonialen in Yogatexten kodifizierten Praktiken, die offenbar von Akteuren wie Yogānanda, Śivānanda, Ramaiah usw. je nach individueller Ausprägung, in kohärent anmutende Übungssysteme integriert worden sind. Die von den eigenen Tradtionen gemachten Angaben zur Historizität ihrer Yogapraxis bedienen sich erprobter Narrative, um dieser Form des Kriyāyoga eine Tradition und historische Legitimität zu verliehen.      

\subsection{Die Kriyāyogas der Traditionslinien von Paramahaṃsa Yogānanda, Svāmī Śivānanda Sarasvatī und Ramaiah}

Was macht nun diese neuen Formen des Kriyāyoga aus? Um diese Frage zu beanworten wurden rezente Publikationen zu diesem Thema konsultiert.\footnote{Diese Liste ist sicherlich nicht vollständig. Dennoch habe ich versucht und möglichst weites Spektrum dieser Art der Publikationen zu konsultieren. 1) Yogānanda-Modell: \citeauthor{autobioyogi} (1949); \citeauthor{kriyayogalowenstein} (2021); \citeauthor{kriyayogasarasvati1981} (1981); \citeauthor{hariharananda1989} (1989); \citeauthor{kriyayogaupanishad1993} (1993) and \citeauthor{kriyayogasturgess2015} (2015). 2) Śivānanda-Modell: \citeauthor{shivanandakriya1982} (1955) and \citeauthor{kriyayoganityananda2013} (2013). Ramaiah-Modell: \citeauthor{govindan2010} (2010).} Die folgende Betrachtung soll in Kürze die Grundzüge des Yogānanda-, Śivānanda- und Ramaiah-Modells des Kriyāyoga skizzieren, ohne dabei Anspruch auf Vollständigkeit zu erheben. Eine umfassende und vollständige historische Untersuchung von Kriyāyoga ist meiner Kenntnis nach bisher noch nicht geleistet worden und kann in diesem Rahmen auch nicht geleistet werden. Der hiesige Versuch ist eine Skizzierung und soll als Annährung des Themas verstanden werden, um einerseits die im öffentlichen Diskurs kursierenden Modelle zu differenzieren und andererseits eine Hypothese zum Übergang der älteren Modelle hin zu den neueren Modellen aufzustellen, da diese zeitlich sehr eng beieinander liegen.  

In den konsultierten Publikationen finden sich uneinheitliche Etymologien und Erklärungen des Begriffes Kriyāyoga.\citeauthor{hariharananda1989}, der von Yogānanda authorisierter\footnote{Cf. \citeauthor{hariharananda1989} 1989: 16.} Kriyāyoga-Lehrer exklärt: \begin{quote} 'Kriya Yoga' are Sanskrit words, a combination of two root words. One is Kriya and the other is yoga. In the word Kriya there are two syllables: kri and ya. Kri means to pursue your work in daily life and ya means to be ever aware of the invisible God who is abiding in you and is directing and accomplishing work through you. \ldots The second word, 'yoga,' literally means union of the visible body with the invisible body. This union is always present in everyone. (\citeauthor{hariharananda1989} 1989: 83) \end{quote}.
Die komplexeste Erklärung des Begriffes befindet sich bei \citeauthor{kriyayoganityananda2013} (2013), welcher sich ebenfalls in der Traditionslinie des Yogānanda verortet: \begin{quote}
  The word \textit{kriyā} is composed of the letters \textit{k}, \textit{r}, \textit{i}, \textit{y}, and \textit{ā}. The letter -\textit{k} (or \textit{ka}, \textit{ka-kāra}, represents the Lord, \textit{Īśvara}. The Transcendental Lord, \textit{Parama Śiva}, when he manifests Himself in the suble world and makes Himself ready for creation He becomes \textit{Īśvara}. The letter-\textit{r} (or \textit{ra}), \textit{ra-kāra}, represents fire, light and manifestation. Creation is not seen by us with the ether and air elements since these are subtle elements. We are able to see manifestation from the fire element onwards. The letter -\textit{i}, \textit{i-kāra}, represents energy or \textit{śakti}. So \textit{kri} is the activating power of the Lord manifested in creation. The activating power is called \textit{prāṇa} or vital force. The letter -\textit{y} (or \textit{ya}), \textit{ya-kāra}, represents the air element and the letter -\textit{ā}, \textit{ā-kāra}, represents form. For the manifestations to take a form, \textit{ākāra}, the Lord acts with the air element. With the ether element there is no form. The air element or gaseous state is the first created form although we only see the forms from the fire element onwards. Through the action of air the whole universe is manifested. This is the action of the Life-force, \textit{prāṇakarma}, of the Lord. The word \textit{kriyā} normally means action, but this is the action of god. We are made with the same principle God is. Our identification with the physical body makes us separate from God and this is the state of ignorance. We have to eradicate this ignorance by the action of God, i.e., the action of the breath, \textit{prāṇakarma}. Our mind is the result of ignorance and is responsible for the wrong identification. Breath-practice, \textit{prāṇakarma}, absorbs the mind into the vital force. This action of God reverses the process and leads us from body to God. This is why it is so necessary to perform that action. That is our spiritual practice. Then that action, \textit{kriyā}, becomes yoga. (\citeauthor{kriyayoganityananda2013} 2013: 2-3) \end{quote}
Satyānanda, ein wichtiger Akteur des Śivānanda-Modells definiert Kriyāyoga wiefolgt: \begin{quote} The Sanskrit word \textit{kriya} means `action' or `movement'. \textit{Kriya Yoga} is so called because it is a system where one intentionally rotates one's attention along fixed pathways. This movement of awareness is done, however with control. Also kriya yoga is so called because one moves the body into specific mudras, bandhas and asanas according to a fixed scheme of practice. The word \textit{kriya} is often translated as meaning `practical'. This is indeed a good definition, for kriya yoga is indeed practical. It is concerned solely with practice, without the slightest philosophical speculation. The system is designed to bring results, not merely to talk about them. Sometimes the word \textit{kriya} is translated as `preliminary'. This too is a good definition, for kriya yoga is a preliminary practice that leads first to dharana and then eventually to the transcendental state of dhyana (meditation) and yoga (union). It is a technique which has been designed to lead to that state of being which is beyond all techniques. Finally, the word \textit{kriya} is used to describe each individual practice. Thus the process of kriya yoga consists of a number of kriyas each being done one after the other in a fixed sequence. (\citeauthor{kriyayogasarasvati1981} 1981:699)\end{quote}
\citeauthor{govindan2010}, ein Schüler von Ramaiah bietet die simpelste Erläuterung des Begriffes: \begin{quote} Kriyā ist eine mit Achtsamkeit ausgeführte Aktivität. (\citeauthor{govindan2010} 2010: 214) \end{quote}

So unterschiedlich die hier wiedergegebenen Konzepte auch sind, so ist deren Schnittmenge jedoch, dass es hier um bewusst ausgeführte Handlungen bzw. Praktiken geht, die den Menschen mit Gott verbinden, bzw. einen Transzendenten Zustand hervorbringen sollen. \citeauthor{kriyayoganityananda2013} erwähnt schließlich bereis in seiner Definition die zentrale Handlung (\textit{kriyā}), welche zur Verbindung mit Gott führen, nämlich Atempraxis (\textit{prāṇakarma}). Darüber hinaus werden von \citeauthor{kriyayogasarasvati1981} auch weitere Praktiken wie Aufmerksamkeitslenkung, \textit{mūdra}s, \textit{bandha}s und \textit{āsana}s genannt.  

In den kolsultierten Texten finden sich weitere Definition, aber für die hiesigen Zwecke sind diese bereits ausreichend, denn sie veranschaulichen einerseits die Grundidee der neuen Modelle des Kriyāyoga, und zeigen andererseits eine grundlegende Vielfalt und Offenheit des Modells, welche alle Bereiche dieser Formen des Kriyāyoga durchdringt. 

Beispielsweise präsentiert uns \citeauthor{govindan2010} gleich sieben unterschiedliche Unterkategorien des Kriyāyoga im Yogānanda-Modell:




Yogānanda, seine Nachfolger, Anhänger und wohl auch dessen Vorgänger führen die Lehre des von ihnen propagierten Kriyāyoga auf eine legendäre Gründerfigur der Kriyāyoga-Traditionslinie namens Babaji zurück.  



Und wie ist es zu Stande gekommen?


Kirya Lineage graphics (S.9 Kiryayoga for Self Discovery)

I speculate that the term Kriyāyoga, as the generic term for his system of Yoga was likely a strategical decision of Lahiri Mahasaya. On the one hand he possibly wanted to profit from the newly gained Yogasūtra hype in the West, or at least was himself caught himself by this wave of excitement. On the other hand his involvement in the local discourse around Varanasi which in these days already was a melting pot of multiple traditions had possibly influenced him, too. As we know, the comlex taxonomies of with 15 Yogas and Kriyāyoga on top had an exalted position in the discourse and was not stragegically clever since it was not just regarded as one of the most accessable paths of Yoga, but a term that would allow him to profit from the popularity of the term, integrate Patañjali's Kriyāyoga and reinterprete the term \textit{kriyā} (action) to integrate the whole array of physical practices that were available in his days and invent new ones.  


\section{Jñānayoga - \uproman{2}}
\label{jnanayogaintro}

Jñānāyoga\footnote{See section \uproman{21} on p.\pageref{jnanayogastart}-\pageref{jnanayogaend}.} is the second Yoga within Rāmacandra's list of fifteen Yogas as well as his source text, the \textit{Yogasvarodaya}. In Nārāyaṇatīrtha's list of fifteen, Jñānayoga occupies the sixth place. Sundardās positions Jñānayoga at the tenth position in his list of twelve Yogas. Here, it is subsumed under the Sāṅkhyayoga category, the third and final tetrad of his list.







%Notes:

%Chapter 15 - Trikāṇḍa-Yoga: Bhakti Surpasses
%Knowledge and Detachment
%(1) Śrī Uddhava said: 'The Vedic literature of Your Lordship, oh Lotus-eyed One,
%that pays attention to the injunctions concerning actions and prohibitions, deals with
%the good and bad sides of karma [akarma and vikarma]. (2) They also discuss the dif-
%ferences within the varṇāśrama system wherein the father may be of a higher [anulo-
%ma] or a lower [pratiloma] class than the mother, they are about heaven and hell and
%expound on the subjects of having possessions, one's age, place and time [see also 4.8:
%54 and *]. (3) How can human beings without Your prohibitive and regulatory words
%concerning final beatitude, tell the difference between virtue and vice [compare 11.19:
%40-45]? (4) The Vedic knowledge emanating from You offers the forefathers, the gods as
%72Uddhava Gītā
%also the human beings a superior eye upon the - not for everyone that evident - meaning
%of life, what would be the goal, and how we may achieve. (5) The difference between
%virtue and vice one can see with the help of Your Vedic knowledge and that insight does
%not arise by itself, but the Vedas also nullify such a difference and thus clearly confuse
%the issue....'
%(6) The Supreme Lord said: 'The three ways of yoga I described in My desire to
%grant human beings the perfection, are the path of philosophy [jñāna], the path of work
%[karma] and the path of devotion [bhakti]; no other means can be found [for one's
%emancipation. See also B.G. contents and trikāṇḍa].

\section{Caryāyoga - \uproman{3}}
\label{caryayogaintro}

Due to the absence of the term \textit{cāryayoga} in Rāmacandra's sources and the brevity of the section, it seems that he added his version of Caryāyoga to do justice to the list. Rāmacandra emphasizes the cultivation of detachment towards sin (\textit{pāpa}) and merit (\textit{puṇya}). Parallels can be identified with the concept of Caryāyoga as presented in the \citetitle{yogacandrika} (Ed. pp. 2, 52-53, 100-101, 150). Here, it appears that Caryāyoga is a discipline that aims to purify the mind. Nārāyaṇatīrtha introduces Caryāyoga in the context of Yogasūtra 1.33, Ed. p. 52 (\textit{maitrīkaruṇāmuditopekṣāṇāṃ sukhaduḥkhapuṇyāpuṇyaviṣayāṇāṃ bhāvanātaś cittaprasādanam}). According to Nārāyaṇatīrtha's commentary, the practice of it involves cultivating specific mental attitudes, such as \textit{maitrī} (loving-kindness), \textit{karuṇā} (compassion), \textit{muditā} (sympathetic joy), and \textit{upekṣā} (equanimity), towards different objects or situations, such as happiness, suffering, merit, and demerit. The practice of Caryāyoga is said to lead to eradicating mental impurities and attaining a calm and tranquil state of mind. Nārāyaṇatīrtha characterizes the practice as the renunciation of worldly attachments and desires and the performance of selfless actions or Karmayoga. Nārayaṇatīrtha states that Caryāyoga is the "primary discipline of detachment (\textit{vairāgya})," which suggests that it emphasizes the cultivation of detachment or dispassion towards worldly objects and desires as a means of achieving spiritual liberation. Within \citetitle{sarvangayoga} (2.40-51, Ed. pp. 96-98) Sundardās describes Cārcāyog as a type of Bhaktiyog which is \textit{bhakti} towards unmanifest consciousness (\textit{avyakta puruṣa}) in delightful devotion. The practice results in a beautiful inner being (50-51). He first describes the unmanifest consciousness (\textit{avyakta puruṣa}) as being formless and eternal and so on (40), as beginningless and endless, and so on (41). Next, Sundardās describes the various layers of creation emanating from \textit{oṃ} (42-45). He says the unmanifest consciousness illuminates every corner of existence (46), being the inner knower of all (47). Then, Sundardās expresses the importance of deep awe towards the infinite, divine, all-knowing and incomprehensible (48-49) unmanifest consciousness, which is the critical component of his Cārcāyog type of \textit{bhakti}.


\end{document}

