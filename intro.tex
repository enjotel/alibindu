%Ultimatives Tool zur Datierung:
%https://www.cc.kyoto-su.ac.jp/~yanom/pancanga/
%skp = ignored in edition
%skm = ignored in xml
%%%---2-DO---%%%:
% - add xml ids for cladistics
% - produce diplomatic transcripts for saktumiva
% - read Sarvangayogapradipika, Maya Burger! 
% - maybe add second ciritical edition of yogasvarodaya?!
% - grep-search alle Verse!!!!
% - Mss spreadsheet
% - additions to U2: make footnotes for the bahir mātrā-s: explaining the inventions of female deities and tell that this is "schwer interpretierbar"
% - Consider changing Lakṣya to Lakṣa
%%%%%%%%%%%%%%%%%%%%%%%%%%%%%%%%%%%%%%%%%
% Don't forget
% Siddhasiddhantapaddhati Yogic Body descriptions are followed by Rāmacandra
% Quotes of the Yogasvarodaya in the Yoga Karṇikā
% Rāmacandra more a compiler than an author!!!
% Identify quotes of YTB in Haṭhasanketacandrikā -- done :D
%%%%%%%%%%%%%%%%%%%%%%%%%%%%%%%%%%%%%%%%%%%
%MSS notes
%
%--B: i and ī are not differenciated
%--P: no punctuation no daṇdas nothing
%--U1: dot . serves as daṇḍa 
%--\L and \U2 very similar
%--figure out for U2: // ajapājapaḥ sahasra // 6000 //gha 0 16 pa 0 40// \U2?!?!?!?!?!?
%%%%%%%%%%%%%%%%%%%%%%%%%%%%%%%%%%%%%%%%%%
%
% Einleitung Ideen 
% - sprachliche Simplizität
% - Potenzial als Anfängertext
% - Großartige Einführung in die Textkritik -> Synoptische Edition 
% - Gelegenheit Yogasvarodaya und Yogatattvabindu zu edieren 
% - Historische Evidenz entweder für das königliche Leben in einer Maṭha in der Nähe von Benares während der Muslimischen Herrschaft, oder sogar Lehrtext für die Bildung junger Prinzen  
% - eines der raren Beispiele der engen Verknüfung mehrerer Texte 
% - eines der raren Beispiele der Prosaisierung eines metrischen Textes 
% - Anwendung rezenter Technologie! 
% - How the text was construed -> intermingling of Ysv and SSP
% - Martin Straube: "jeder kleine Dorfhäuptling kann Rāja genannt werden". 
%%%%%%%%%%%%%%%%%%%%%%%%%%%%%%%%%%%%%%%%%%%
\documentclass[10pt]{memoir}
\setstocksize{220mm}{155mm} 	        
\settrimmedsize{220mm}{155mm}{*}	
\settypeblocksize{170mm}{116mm}{*}	
\setlrmargins{18mm}{*}{*}
\setulmargins{*}{*}{1.2}
%\setlength{\headheight}{5pt}%
\checkandfixthelayout[lines]
\linespread{1.16}
\flushbottom

%%% Hyphenation settings
\usepackage[htt]{hyphenat}
\hyphenation{he-lio-trope opos-sum}
\tracingparagraphs=1
%Hyphenation in Devanāgarī of the edition still missing? Probably this needs to be modified in babel-iast package? 

%%% babel
\usepackage[english]{babel}
\usepackage{babel-iast/babel-iast}

\babelfont[iast]{rm}[Renderer=Harfbuzz, Scale=1.3]{AdishilaSan}%AdishilaSan}
\babelfont[english]{rm}{Adobe Text Pro}

%%% more functionality
\PassOptionsToPackage{hyphens}{url}
\usepackage{hyperref}
\usepackage{pdflscape}
\usepackage{cleveref}
\usepackage{url}
\usepackage{cleveref}
\usepackage{microtype}
\usepackage{lineno}

%\usepackage{bigfoot}
%%% more functions
\usepackage[dvipsnames]{xcolor}
%\usepackage[para,perpage]{footmisc}

%%%für den Counter von Kapiteln und Sätzen! 
\newcommand{\uproman}[1]{\uppercase\expandafter{\romannumeral#1}}
\newcommand{\lowroman}[1]{\romannumeral#1\relax}

\makeindex
\newfontfamily\sanskritfont[Script=Devanagari,Mapping=RomDev,Scale=1.1]{Sanskrit2003}
\usepackage{pifont,fourier-orns,lettrine,psvectorian,paralist,enumitem,pdfpages,wrapfig,tabulary,lettrine,longtable}
\setlist[enumerate]{itemsep=0mm}
\usepackage[autostyle]{csquotes}
\usepackage[defaultlines=2,all]{nowidow}
\usepackage{ellipsis,adforn,booktabs,longtable,url,tikz}
\lineskiplimit=-3pt          

\makechapterstyle{IeT}{%
  \chapterstyle{default}
  \renewcommand*{\printchapternonum}{\centering}
  \renewcommand*{\clearforchapter}{\cleartorecto} 
  \aliaspagestyle{chapter}{empty}}
\chapterstyle{IeT}
\setsecnumdepth{none}  \openright  \nouppercaseheads
\settocdepth{subsubsection}

%%%% test better pagebreaks
%\def\fussy{%
%  \emergencystretch\z@
%  \tolerance 200%
%  \hfuzz .1\p@
%  \vfuzz\hfuzz}

%\interfootnotelinepenalty=10000\relax

%\usepackage[maxfloats=256]{morefloats}

%\maxdeadcycles=500

%raggedbottomsectiontrue
%%\checkandfixthelayout


%%%%%%%  biblatex
%\newcommand{\noun}[1]{\textsc{#1}}    %  philosophy-verbose
\usepackage[backend=biber, sorting=nyt, style=verbose]{biblatex} %%%%ORIGINAL TiE
\renewcommand*{\mkbibnamefamily}[1]{\textsc{#1}}


\DeclareFieldFormat{url}{%
  \mkbibacro{URL}\addcolon\space
  \href{#1}{\nolinkurl{\thefield{urlraw}}}}

\DeclareFieldFormat{citeurl}{%
  \href{#1}{\nolinkurl{\thefield{urlraw}}}} 


\DeclareFieldFormat{postnote}{#1}
\renewcommand{\postnotedelim}{, }
\addbibresource{bindu.bib}

%%% ekdosis
\usepackage[teiexport=tidy,parnotes=true]{ekdosis}% =tidy cleans up HTML and XML documents by fixing markup errors and upgrading legacy code to modern standards. parnotes=footnotes below or above critical apparatus

\SetLineation{lineation=page, modulo} %lineation=page sets thenumbering to start afresh at the top of each page. =modulo makes every fifth line numbered. {lineation=page} makes every line numbered! 

\renewcommand{\linenumberfont}{\selectlanguage{english}\footnotesize} %sets language of lines to English

\SetTEIxmlExport{autopar=false} %autopar=falseinstructs ekdosis to ignore blank lines in the.tex sourcefile as markers for paragraph boundaries. As a result, each paragraph of the edition must be found within an environment associated with the xml <p> element

\SetHooks{
  lemmastyle=\bfseries,
  %refnumstyle=\selectlanguage{english}\bfseries,
  refnumstyle=\selectlanguage{english}\color{blue}\bfseries,
  appheight=0.8\textheight,
}

\newif\ifinapparatus
\DeclareApparatus{source}[
%bhook=\inapparatustrue,
lang=english,
notelang=english,
% bhook=\selectlanguage{english},
bhook=\selectlanguage{english}\textbf{Sources:},%
%maxentries=4, 
%ehook=.]
%sep={] },
%nosep,
]

\newif\ifinapparatus
\DeclareApparatus{testium}[
%bhook=\inapparatustrue,
lang=english,
notelang=english,
% bhook=\selectlanguage{english},
bhook=\selectlanguage{english}\textbf{Testimonia:},
%maxentries=4, 
%ehook=.]
%nosep, 
]

% Declare \ifinapparatus and set \inapparatustrue at the beginning of
% the apparatus criticus block. Also set the language.  
\newif\ifinapparatus
  \DeclareApparatus{default}[
  %bhook=\inapparatustrue, 
  lang=english,
  %maxentries=33,
  %bhook=\selectlanguage{english},
  sep = {] },
  delim=\hskip 0.75em,
  rule=\rule{0.7in}{0.4pt},
]

\newif\ifinapparatus
\DeclareApparatus{philcomm}[
%bhook=\inapparatustrue,
lang=english,
notelang=english,
bhook=\selectlanguage{english}\textbf{Philological Commentary:},
%bhook=\selectlanguage{english},
sep={: },
]

\ekdsetup{
showpagebreaks,
spbmk = \textcolor{blue}{spb},
hpbmk = \textcolor{red}{hpb}
}

%\usepackage{fnpos}
%\makeFNmid
%\makeFNbottom
\usepackage[bottom]{footmisc}
%%%%%%%%%%%%%%%%%%%%%%%%%%%
\makeatletter
\def\blfootnote{\gdef\@thefnmark{}\@footnotetext}
\makeatother
%%%%%%%%%%%%%%%%%%%%%%%%%


% Macros and Definitions for the Print of Sigla
\def\acpc#1#2#3{{#1}\rlap{\textrm{\textsuperscript{#3}}}\textsubscript{\textrm{#2}}\space}
\def\sigl#1#2{{{#1}}\textsubscript{\textrm{#2}}}
\def\None{{\sigl{N}{1}}} \def\Noneac{\acpc{N}{1}{ac}\,} \def\Nonepc{\acpc{N}{1}{pc}\,}
\def\Ntwo{{\sigl{N}{2}}} \def\Noneac{\acpc{N}{2}{ac}\,} \def\Nonepc{\acpc{N}{2}{pc}\,}
\def\Done{{\sigl{D}{1}}} \def\Doneac{\acpc{D}{1}{ac}\,} \def\Donepc{\acpc{D}{1}{pc}\,}
\def\Dtwo{{\sigl{D}{2}}} \def\Dtwoac{\acpc{D}{2}{ac}\,} \def\Dtwopc{\acpc{D}{2}{pc}\,}
\def\Uone{{\sigl{U}{1}}} \def\Uoneac{\acpc{U}{1}{ac}\,} \def\Uonepc{\acpc{U}{1}{pc}\,}                 
\def\Utwo{{\sigl{U}{2}}} \def\Utwoac{\acpc{U}{2}{ac}\,} \def\Utwopc{\acpc{U}{2}{pc}\,}

%%%%%%%%%%%%%% Tattvabinduyoga - List of Witnesses   %%%%%%%%%%%%%%%%%%%
\DeclareWitness{ceteri}{\selectlanguage{english}cett.}{ceteri}[]   
\DeclareWitness{E}{\selectlanguage{english}E}{Printed Edition}[]    
\DeclareWitness{P}{\selectlanguage{english}P}{Pune BORI 664}[]  
\DeclareWitness{B}{\selectlanguage{english}B}{Bodleian 485}[]       
\DeclareWitness{N1}{\selectlanguage{english}N\textsubscript{1}}{NGMPP 38/31}[]
\DeclareWitness{N2}{\selectlanguage{english}N\textsubscript{2}}{NGMPP B 38/35}[]
\DeclareWitness{L}{\selectlanguage{english}L}{LALCHAND 5876}[]  
\DeclareWitness{D}{\selectlanguage{english}D}{IGNCA 30019}[] 
%\DeclareWitness{D2}{\selectlanguage{english}D\textsubscript{2}}{IGNCA 30020}[]  
\DeclareWitness{U1}{\selectlanguage{english}U\textsubscript{1}}{SORI 1574}[] 
\DeclareWitness{U2}{\selectlanguage{english}U\textsubscript{2}}{SORI 6082}[]
%%%%%%%%%%%%%% Tattvabinduyoga - Groups of Witnesses   %%%%%%%%%%%%%%%%%%%
\DeclareWitness{X}{\selectlanguage{english}\alpha}{Alpha Group: D,N1,N2,U1}[]
\DeclareWitness{Y}{\selectlanguage{english}\beta}{Beta Group: B,E,L,P,U2}[]
%%%%%%%%%%%%% Testimonia
\DeclareWitness{Ysv}{\selectlanguage{english}Ysv}{Yogasvarodaya}[] %%%add infos!  

%%%%%%%%%%%%%%%%%%%%%%%%%%%%%%%%%%%%%%%%%%%
% Macro for Editing Abbrevs.
\def\om{\textrm{\footnotesize \textit{om.}\ }} %prints om. for omitted in apparatus
\def\korr{\textrm{\footnotesize \textit{em.}\ }} %prints em. for emended in apparatus
\def\conj{\textrm{\footnotesize \textit{conj.}\ }} %prints conj. for conjectured in apparatus

% \supplied{text} EDITORIAL ADDITION -> Within \lem oder \rdg
% \surplus{text} EDITORIAL DELETION -> Within \lem oder \rdg
% \sic{text} CRUX
% \gap{text} LACUNAE -> [reason=??, unit=??, quantity=??, extent=??]


%%%%%%%%%%%%%%%%%%%%%%%%%%%%%%%%%%%%%%%%%%% All macros of this list can be used in 
% Macro for Editing Abbrevs.
\def\eyeskip{\textrm{{ab.\,oc. }}}
\def\aberratio{\textrm{{ab.\,oc. }}}
\def\ad{\textrm{{ad}}}
\def\add{\textrm{{add.\ }}}
\def\ann{\textrm{{ann.\ }}}
\def\ante{\textrm{{ante }}} 
\def\post{\textrm{{post }}}
%\def\ceteri{cett.\,}                   
\def\codd{\textrm{{codd.\ }}}

\def\coni{\textrm{{coni.\ }}}
\def\contin{\textrm{{contin.\ }}}
\def\corr{\textrm{{corr.\ }}}
\def\del{\textrm{{del.\ }}}
\def\dub{\textrm{{ dub.\ }}}

\def\expl{\textrm{{explic.\ }}} 
\def\explica t{\textrm{{explic.\ }}}
\def\fol{\textrm{{fol.\ }}}
\def\foll{\textrm{{foll.\ }}}
\def\gloss{\textrm{{glossa ad }}}
\def\ins{\textrm{{ins.\ }}}      
\def\inseruit{\textrm{{ins.\ }}} 
\def\im{{\kern-.7pt\lower-1ex\hbox{\textrm{\tiny{\emph{i.m.}}}\kern0pt}}} %\textrm{\scriptsize{i.m.\ }}}      
\def\inmargine{{\kern-.7pt\lower-.7ex\hbox{\textrm{\tiny{\emph{i.m.}}}\kern0pt}}}%\textrm{\scriptsize{i.m.\ }}}      
\def\intextu{{\kern-.7pt\lower-.95ex\hbox{\textrm{\tiny{\emph{i.t.}}}\kern0pt}}}%\textrm{\scriptsize{i.t.\ }}}           
\def\indist{\textrm{{indis.\ }}}  
\def\indis{\textrm{{indis.\ }}}
\def\iteravit{\textrm{{iter.\ }}} 
\def\iter{\textrm{{iter.\ }}}
\def\lectio{\textrm{{lect.\ }}}   
\def\lec{\textrm{{lect.\ }}}
\def\leginequit{\textrm{{l.n. }}} 
\def\legn{\textrm{{l.n. }}}
\def\illeg{\textrm{{l.n. }}}

\def\primman{\textrm{{pr.m.}}}
\def\prob{\textrm{{prob.}}}
\def\rep{\textrm{{repetitio }}}
\def\secundamanu{\textrm{\scriptsize{s.m.}}}            \def\secm{{\kern-.6pt\lower-.91ex\hbox{\textrm{\tiny{\emph{s.m.}}}\kern0pt}}}%   \textrm{\scriptsize{s.m.}}}
\def\sequentia{\textrm{{seq.\,inv.\ }}}  
\def\seqinv{\textrm{{seq.\,inv.\ }}}
\def\order{\textrm{{seq.\,inv.\ }}}
\def\supralineam{{\kern-.7pt\lower-.91ex\hbox{\textrm{\tiny{\emph{s.l.}}}\kern0pt}}} %\textrm{\scriptsize{s.l.}}}
\def\interlineam{{\kern-.7pt\lower-.91ex\hbox{\textrm{\tiny{\emph{s.l.}}}\kern0pt}}}   %\textrm{\scriptsize{s.l.}}}
\def\vl{\textrm{v.l.}}   \def\varlec{\textrm{v.l.}} \def\varialectio{\textrm{v.l.}}
\def\vide{\textrm{{cf.\ }}}
\def\cf{\textrm{{cf.\ }}} 
\def\videtur{\textrm{{vid.\,ut}}}
\def\crux{\textup{[\ldots]} }
\def\cruxx{\textup{[\ldots]}}
\def\unm{\textit{unm.}}
%%%%%%%%%%%%%%%%%%%%%%%%%%%%%%%%%%%%

% List of Scholars
\DeclareScholar{ego}{ego}[
forename=Nils Jacob,
surname=Liersch]

% Persons:14\DeclareScholar{ego}{ego}[15forename=Robert,16surname=Alessi]17% Useful shorthands:18\DeclareShorthand{codd}{codd.}{V,I,R,H}19\DeclareShorthand{edd}{edd.}{Lit,Erm,Sm}20\DeclareShorthand{egoscr}{\emph{scripsi}}{ego}

%Useful shorthands:
%\DeclareShorthand{codd}{codd.}{V,I,R,H}
%\DeclareShorthand{edd}{edd.}{Lit,Erm,Sm}
\DeclareShorthand{egoscr}{em.}{ego}
\DeclareShorthand{egoscrconj}{conj.}{ego}
\DeclareShorthand{egomute}{\unskip}{ego}

\usepackage{xparse}

\NewDocumentEnvironment{tlg}{O{}O{}}{\setlength{\leftskip}{0pt}\vspace{-1ex}\begin{quotation}}{\hfill #1\ \vspace{-1ex}\end{quotation}\vspace{-1ex}} %verse environment
%\NewDocumentEnvironment{tlg}{O{}O{}}{\begin{verse}}{॥#1\hskip-4pt ॥\\ \end{verse}}
\NewDocumentCommand{\tl}{m}{{\selectlanguage{iast} #1}}

\NewDocumentCommand{\extra}{m}{{\textcolor{gray}{#1}}} %command for additions to U2
\NewDocumentCommand{\crazy}{m}{{\textcolor{red}{#1}}} %totally corrupted passage
\NewDocumentCommand{\coro}{m}{{\textcolor{violet}{#1}}} %colour for sentence counter! 

\NewDocumentEnvironment{prose}{O{}}{\begin{otherlanguage}{iast}}{\end{otherlanguage}}
% \NewDocumentEnvironment{padd}{O{}}{\begin{otherlanguage}{iast}}{\end{otherlanguage}}
\NewDocumentEnvironment{tlate}{O{}}
%\NewDocumentEnvironment{tadd}{O{}}

%Define two commands: \skp ("sanskrit plus"), to be ignored by TeX in
%the edition text, but processed in the TEI output. Conversely, \skm
%("sanskrit minus") is to be processed in the edition text, but
%ignored if found in the apparatus criticus and in the TEI output:

\NewDocumentCommand{\skp}{m}{}
\TeXtoTEIPat{\skp {#1}}{#1}

%\NewDocumentCommand{\skpp}{m}{}
%\TeXtoTEIPat{\skpp {#1}}{#1}

\NewDocumentCommand{\skm}{m}{\unless\ifinapparatus#1-\fi}
\TeXtoTEIPat{\skm {#1}}{}

% \NewDocumentCommand{\dd}{}{/\hskip-4pt/}
\NewDocumentCommand{\dd}{}{\mbox{/\hskip-4pt/}}
\TeXtoTEIPat{\dd {}}{//}


%%% modify environments and commands
%%% TEI mapping
\TeXtoTEIPat{\begin {tlg}}{<lg>} %lg=(Group of verse (s)) contains one or more verses or lines of verse that together form a formal unit (e.g. stanza, chorus).
\TeXtoTEIPat{\end {tlg}}{</lg>}

\TeXtoTEIPat{\begin {prose}}{<p>}
\TeXtoTEIPat{\end {prose}}{</p>}

\TeXtoTEIPat{\begin {tlate}}{<p>}
\TeXtoTEIPat{\end {tlate}}{</p>}

\TeXtoTEIPat{\\}{}
\TeXtoTEIPat{\linebreak}{<br/>}
\TeXtoTEIPat{\noindent}{}
%\TeXtoTEI{tl}{l}
\TeXtoTEI{emph}{hi}
\TeXtoTEI{bigskip}{}
\TeXtoTEI{None}{N1}
\TeXtoTEI{Ntwo}{N2}
\TeXtoTEI{Done}{D1}
\TeXtoTEI{Dtwo}{D2}
\TeXtoTEI{Uone}{U1}
\TeXtoTEI{Utwo}{U2}
%\TeXtoTEIPat{/}{ |}
%\TeXtoTEI{//}{ ||}
\TeXtoTEIPat{\korr}{em. }
\TeXtoTEIPat{\conj}{conj.}
\TeXtoTEIPat{\om}{om.}
\TeXtoTEIPat{english}{}
\TeXtoTEIPat{\hskip}{}
\TeXtoTEIPat{\hskip-4pt}{}
\TeXtoTEIPat{\hskip-2pt}{}
\TeXtoTEIPat{-}{ }
\TeXtoTEIPat{4pt}{}
\TeXtoTEIPat{2pt}{}
\TeXtoTEIPat{\textcolor {#1}{#2}}{<hi rend="#1">#2</hi>} 

% Nullify \selectlanguage in TEI as it has been used in
% \DeclareWitness but should be ignored in TEI.
\TeXtoTEI{selectlanguage}{}



\FormatDiv{1}{\begin{center}\Large}{\end{center}}
\FormatDiv{2}{\begin{center}\small}{\end{center}}
\FormatDiv{3}{\bfseries}{.}
\title{Yogatattvabindu of Rāmacandra\\ A Critical Edition and Annotated Translation}
\date{\today}

\parindent=15pt
\begin{document}

% Zitiermöglichkeiten:
%\footcite[See][p.\,1]{goldstein01:_tibet_englis_diction_moder_tibet}
%\footnote{\cite{goldstein01:_tibet_englis_diction_moder_tibet}.}

\frontmatter
\thispagestyle{empty}
\begin{center}
  {\Large \emph{The Yogatattvabindu}}\\[3mm]
\end{center}



\newpage

\

\thispagestyle{empty}



\normalsize


\newpage


\begin{center}
\thispagestyle{empty}

\

\vskip 2mm

\begin{otherlanguage}{iast}
\LARGE \sanskritfont{Yogatattvabindu}
\end{otherlanguage}

\vskip .4cm

\Huge Yogatattvabindu \\[7mm]
\Large Critical Edition\\
with annotated Translation


\large

\vspace{3cm}

Von

Nils Jacob Liersch
\small
\vfill

\vfill

Indica et Tibetica Verlag \\ % $\cdot$ 
Marburg 2024

\vskip 6mm

\end{center}

\newpage
\newpage \ \thispagestyle{empty}
\small  \

\noindent

\
\vfill


\small
\noindent \textbf{Bibliographische Information Der Deutschen Bibliothek}

\noindent
Die Deutsche Bibliothek verzeichnet diese Publikation in der Deutschen Nationalbibliographie;
detaillierte bibliographische Informationen sind im Internet über http://dnb.ddb.de abrufbar.

\noindent
\textbf{Bibliographic information published by Die Deutschen Bibliothek}

\noindent
Die Deutsche Bibliothek lists this publication in the Deutsche Nationalbibliographie; detailed
bibliographic data is available in the Internet at http://dnb.ddb.de.  


\vskip 1cm

\noindent
\copyright\ Indica et Tibetica Verlag, Marburg 2024

\medskip

\noindent
Alle Rechte vorbehalten / All rights reserved

\medskip

\noindent
Ohne ausdrückliche Genehmigung des Verlages ist es nicht gestattet, das Werk oder einzelne Teile
daraus nachzudrucken, zu vervielfältigen oder auf Datenträger zu speichern.

\smallskip

\noindent
Apart from any fair dealing for the purpose of private study, research, criticism or review, no
part of this book may be reproduced or translated in any form, by print, photo form, microfilm, or
any other means without written permission. Enquiries should be made to the publishers.

\bigskip

\noindent
Satz: \ \ Nils Jacob Liersch \\
Herstellung: \ \ BoD – Books on Demand GmbH, Norderstedt  \\

\bigskip

\noindent
%\ISBN     

\normalsize

\newpage

%\maketitle
\clearpage
\tableofcontents
\addtocounter{page}{-1}
\thispagestyle{empty}
\clearpage


\mainmatter

\chapter{Introduction}
\cleardoublepage

\section{General remarks}
The \textit{Yogatattvabindu} is a premodern Sanskrit Yoga text on Rājayoga that was written in the first half of the seventeenth century\footnote{The dating of the text is discussed on p.\pageref{dating}.} in northern India.\footnote{The detailed discussion of the place of origin is found on p.\pageref{placeoforigin}.} The most salient feature of the work that makes it historically significant is its highly differentiated taxonomy of types of Yoga. In the \textit{Yogatattvabindu}'s introduction, most manuscripts name fifteen types of Yoga, presented as methods of Rājayoga. The text is a yogic compendium written in a mix of mainly prose and 41 verses in textbook-style, where its 58 topics are introduced in sections launched by recognizable phrases. Most sections deal with the methods of Rājayoga and their effects, but others also cover topics like yogic physiology, the importance of the guru, an innovative concept of the Āvadhūta, cosmogony and a \textit{yogarahasya}.  

The \textit{Yogatattvabindu} has basically not been discussed or considered in the secondary literature on Yoga. The only exception is \citeauthor{birch2014} (2014: 415–416) who briefly described its list of fifteen Yogas in the context of the ``fifteen medieval Yogas'' and noted that a similar taxonomy occurs in Nārāyaṇatīrtha’s \textit{Yogasiddhāntacandrikā} (17th century), a commentary on the \textit{Pātañjalayogaśāstra} that integrates fifteen medieval Yogas within its \textit{aṣṭāṅga} format. An incomplete account of the fifteen Yogas is found within the Sanskrit Yoga text \textit{Yogasvarodaya}, which is known only through quotations in the \textit{Prāṇatoṣinī}, \textit{Yogakarṇikā} and \emph{Śabdakalpadruma}.\footnote{Manuscripts under the name of \textit{Yogasvarodaya} seem to be lost. I was not able to allocate the manuscripts of the text in any manuscript catalogue at hand.} The \textit{Yogasvarodaya} provides a total of fifteen Yogas but names only eight of them in its introductory \textit{śloka}s. A complete account of the text is yet to be found and might be lost forever. The \textit{Yogasvarodaya} is the primary source and template for the compilation of the \textit{Yogatattvabindu}. Besides several authorial passages, Rāmacandra, in many instances, follows its content and structure by rewriting the \textit{Yogasvarodaya}’s \textit{śloka}s into prose or quoting them directly without attribution. Due to the incomplete transmission of the \textit{Yogasvarodaya}, Rāmacandra’s \textit{Yogatattvabindu} is a natural and valuable starting point for an unprecedented in-depth study of the complex early modern Yoga taxonomies, a phenomenon that can be narrowed down very precisely in terms of time and as I will show regarding its localisation. The other source text that Rāmacandra used is the \textit{Siddhasiddhāntapaddhati} whose content he draws on, particularly in the last third of his composition. Another text that includes a similar taxonomy of twelve Yogas divided into three tetrads is Sundardās’s \textit{brāj bhāṣa} Yoga text named \textit{Sarvāṅgayogapradīpikā} which not just shares most of the types of Yogas but also provides a different and valuable perspective on the addressed Yoga categories.\footnote{For a comparative table of the complex early modern Yoga taxonomies see table \ref{15yogastable} on p.\pageref{15yogastable}.}

These complex taxonomies that emerged during the 17th century crossed sectarian divides and were adapted to the specific needs of different authors and traditions. The \textit{Yogatattvabindu} thus encapsulates a large proportion of the diversity of Yoga types and teachings after the \textit{Haṭhapradīpikā} (15th century) that were adopted and practiced by a broad spectrum of religious traditions and strata of Indian society. In the particular case of the \textit{Yogatattvabindu}, there are various statements throughout the text that reveal a strategy to radically detach Yoga from its renunciate connotations and to enforce the supremacy and universality of Rājayoga as a practice that can yield the highest benefits even for practitioners who enjoy worldly pleasures and an extravagant lifestyle. Textual evidence suggests the possibility that \textit{Yogatattvabindu} may be a unique example of a Rājayoga text that was composed for warrior aristocracy and members of a royal court. 

%In addition, the analysis of the \textit{Yogatattvabindu} and the historical retracig of its teachings provides insight into a complex network of at least twenty texts,\footnote{This intertextual network which shares those specific teachings consists of the \textit{Netratantra}, \textit{Śāradatilakatantra}, \textit{Sarvadurgatipariśodhanatantra}, \textit{Ūrmikaulārṇavatantra}, \textit{Tantrāloka}, \textit{Manthanabhairavatantra}, \textit{Śārṅgadhārapaddhati}, \textit{Vivekamārtaṇḍa}, \textit{Śivayogapradīpikā}, (recensions of the \textit{Haṭhapradīpikā}), \textit{Amaraughaśāsana}, \textit{Yogasvarodaya}, \textit{Sarvāṅgayogapradīpikā}, \textit{Nityanāthapaddhati}, \textit{Siddhasiddhāntapaddhati}, \textit{Yogatattvabindu}, \textit{Yogacūḍāmaṇyupaniṣad}, \textit{Maṇḍalabrāhmaṇopaniṣat}, \textit{Haṭhatattvakaumudi} and \textit{Haṭhasaṃketacandrikā}.} all of which include one specific set of yoga theorems and practices with minor deviations - three to five \textit{cakra}s, sixteen \textit{ādhāra}s, two to five \textit{lakṣya}s, and five \textit{vyoma}s. This ancient Śaiva paradigm gave rise to an intertextual network that spans at least an entire millennium. It begins in early śivaite Tantras such as the \textit{Netratantra} and ends in the large late medieval Yoga compendiums like the \textit{Haṭhatattvakaumuḍī} and \textit{Haṭhasaṅketacandrikā}. The examination of this network provides insights into the history of the related yoga traditions and enables, for example, the reconstruction of the genesis of individual yoga categories mentioned in the fifteen Yogas, such as Lakṣyayoga, whose techniques were originally taught in early śivaite Tantras, but were only labeled as a separate type of yoga from the 16th century onwards.

One printed edition of the \textit{Yogatattvabindu} was published in 1905 with a Hindi translation and based on an unknown manuscript(s). This publication has the title ``\textit{Binduyoga}'' confirmed by the printed text’s colophon. However, as I discuss in the course of the introduction, the text was likely known as \textit{Yogatattvabindu}. The consulted manuscripts contain significant discrepancies, structural differences and variant readings between them and the printed edition. Furthermore, the manuscripts are scattered over the northern Indian subcontinent and Nepal, which suggests that it was widely transmitted at some point. Lenghty passages of the \textit{Yogatattvabindu} are quoted without attribution in a text called \textit{Yogasaṃgraha} and Sundaradeva’s \textit{Haṭhasaṅketacandrikā}. A critical edition will undoubtedly improve on the published edition and shed further light on the transmission of this important work.

This dissertation contains an introduction, critical edition and annotated translation of the \textit{Yogatattvabindu}. Besides an overview of the manuscript evidence and the editorial policies underlying the edition, the introduction discusses provenance, authorship and the audience of the \textit{Yogatattvabindu}. An unprecedented systematic and comprehensive comparative analysis of the complex early modern Yoga taxonomies based on the new critical edition of the \textit{Yogatattvabindu} and a novel and up-to-date examination of the texts mentioned above with similarly complex taxonomies will determine their position within the broader history of Yoga and improve our knowledge of the development of Yoga traditions in the early modern period.

\section{Dating the \textit{Yogatattvabindu}}
\label{dating}
The oldest dated manuscript of the \textit{Yogatattvabindu} \getsiglum{N1}\footnote{For a description of the manuscript see  p.\pageref{n1description}.} was written in Nepal \textit{saṃvat} 837, which is 1716 CE. Since the text of this manuscript is missing a significant and lengthy passage (ca. 25\% of the entire text) and contains various corruptions, one can assume that some time had passed from the original composition for the transmission to deteriorate to this extent. Therefore, it is likely that the work was composed at least a few decades before the creation of this Nepalese manuscript, perhaps sometime in the 17th century. The discovery that Sundaradeva's \textit{Haṭhasaṅketacandrikā} quotes a lengthy passage of the \textit{Yogatattvabindu} without attribution confirms this suspicion. The passages quoted from the \textit{Yogatattvabindu} include the teachings on the sixteen \textit{ādhāra}s\footnote{\citetitle{hathasamketacandrikajodhpur} (ms. no. 2244, f. 95r l. 3 -- f. 96r l. 4).} and the teachings on Lakṣyayoga and its subtypes.\footnote{\citetitle{hathasamketacandrikajodhpur} (ms. no. 2244, f. 124r l. 7 -- f. 125r l. 3).} The dating of the \textit{Haṭhasaṅketacandrikā} just recently had to be revised due to the discovery that some first-hand notes surrounding the main text of the Ujjain \textit{Yogacintāmaṇi} were in all likelihood borrowed from Sundaradeva's \textit{Haṭhasaṅketacandrikā}.\footnote{Cf. \citeauthor{birch2024} (2024:52-54).} \citeauthor{birch2018proliferation} (2018) dated the Ujjain \textit{Yogacintāmaṇi} to 1659 CE.\footnote{Cf. \citeauthor{birch2018proliferation}, 2018: 50 [n. 111].} Thus, the \textit{terminus ante quem} for the compilation of the \textit{Haṭhasaṅketacandrikā} is 1659 CE which automatically makes it also the \textit{terminus ante quem} for the \textit{Yogatattvabindu} and the \textit{Yogasvarodaya}, due to the fact that Sundaradeva quoted from the \textit{Yogatattvabindu} and Rāmacandra quoted from and rewrote the contents of the \textit{Yogasvarodaya}. Thus, we can safely assume that the \textit{Yogatattvabindu} was written in the course of the first half of the 17th century or earlier. Because of that Rāmancandra's main source text \textit{Yogasvarodaya} must have been written even earlier.

\subsection{Implications for the dating of the \textit{Yogasvarodaya} and the \textit{Siddhasiddhāntapaddhati}}
Furthermore, \citeauthor{mallinsononline2013}\footnote{Cf. \fullcite{mallinsononline2013}.} estimated the age of the \textit{Siddhasiddhāntapaddhati} to circa 1700. Due to the above-mentioned new date of the \textit{Haṭhasaṅketacandrikā} and because Rāmacandra extensively quotes from \textit{Siddhasiddhāntapaddhati} the new terminus \textit{terminus ante quem} for the dating of the \textit{Siddhasiddhāntapaddhati} likewise must be set to 1659 CE. Thus, the \textit{Siddhasiddhāntapaddhati} was also likely composed during the first half of the 17th century or even ealier.

\subsection{Rāmacandra Paramahaṃsa}

%War ein Daśanāmī Saṃnyāsī
%Yogasvarodaya zitier in Prāṇatoṣiṇī = Bengalen
%Yogakarṇikā wahrscheinlich Varanasī
%śabdakalpadruma in Kalkutta = zitiert Yogasvarodaya
%Rāmacandra zitier SSP= verbreitet in Nordindien 

\subsection{The audience of the \textit{Yogatattvabindu}}
\label{ytbaudience}

%\footnote{\citeauthor{mallinson2014} noted that the absence of \textit{ahiṃsa} and \textit{brahmacarya} favored the postulated \textit{kṣatriya} audience. \citeauthor{hanneder2006} noted that \textit{prāṇimātre samābuddhiḥ} is an important quality of a king. See \citeauthor{powell2023} 2023, pp. 194, 196 for comparative tables of \textit{yama}s and \textit{niyama}s.}e

%%%%Mit diesert Form des Yoga und der Beauftragung wollte man womöglich sicherstellen, dass der Nachwuchs nicht auf solch dumme Gedanken kommt, wie große Herrschersöhne vor ihm, beispielsweise der Buddha, der seine Thronfolge aufgab und sich dem Asketenpack anschloss.%%% 


%Seeing that the Yogavåsi‚†ha, which is a rewriting of the earlier
%Mok‚opåya, provides a doctrine of liberation for kings (Hanneder 2009:
%65), it is surprising that the term “råjayoga” did not appear in this text.
% 
%Dādūpanthı̄ Sundardās%
%
%The Rājayogabhāsya says that rājayoga is yoga fit for a king (p. 1: rājayogo rājña upayukto
% ̇
%yogas tathocyate) and Divākara, commenting on the Bodhasāra, says that rājayoga is so c%alled
%because kings can accomplish it even while remaining in their position (section 14, ver%se 1:
%rājayogo rājñām nrpānām svasthāne sthitvāpi sādhayitum śakyatvāt); see also Birch 2013:% 70,
% ̇  ̇  ̇  ̇
% ̇
% n. 269.%%
%
% (\citeauthor{mallinson2018vajrolimudra} 2018:195)%
%
 %In a similar fashion, in tantric traditions kings may be given special initiations that do not require them to carry out the time-consuming rituals and restrictive observances of other initiates, while still receiving the same rewards.43
%%
%
% Here lies the key to understanding vajrolı̄mudrā, and to understanding the
%history of Hatha Yoga as a whole. I have argued elsewhere that the physical
% ̇%
%ractices of Hatha Yoga developed within ascetic milieux, with records of some
% ̇
%perhaps going back as far as the time of the Buddha.44 The composition of the
%texts that make up the early Hatha corpus during the course of the eleventh to
% ̇
%fifteenth centuries CE brought these ascetic techniques, which had never pre-
%viously been codified, to a householder audience.%
%
%(\citeauthor{mallinson2018vajrolimudra} 2018:196)
%
%two early Hatha texts of the Siddha tradition (neither of which
% ̇
% teaches vajrolı̄), the Amrtasiddhi61 and Vivekamārtanda,62%
%%
%
%
%
%Like the asidhārāvrata, the hathayogic vajrolı̄mudrā most probably originated
% ̇
%in a celibate ascetic milieu. The yoga traditions associated with the early Hatha
% ̇
%texts were all celibate, even those that developed out of Kaula lineages which had
%practised ritual sex.68%
%
%(\citeauthor{mallinson2018vajrolimudra} 2018:201)
%%
%
%archetypal avadhūta yogi who can do what he wants.%
%
% See also the definition of rājyayoga in the th-century Gujarati Āgamaprakāśa as the% yoga of the Kaulas and
%the Yogaśikhopanis.ad’s definition of rājayoga as the union of rajas and retas, as note%d by B (:-).
%Cf. Hamsami
%t.t.hu’s designation of rājayoga as a śākta form of the rāsalīlā which involves sexual %rites (V
%.
%: )%
%
%e Dattātreyayogaśāstra follows its teachings
%on vajrolī by saying that it is the only way to bring about rājayoga (),
%and the Hatharatnāvalī
%(.) says that one becomes a rājayogī through
%.
%control of semen. e implication of the name rājayoga here is that one
%can live like a king, indulging oneself in sensory pleasures, yet still be a yogi,
%i.e. one need not renounce the world and become an ascetic.
%%
%
%on vajrolī by saying that it is the only way to bring about rājayoga (),
%and the Hatharatnāvalī
%(.) says that one becomes a rājayogī through
%.
%control of semen.


\chapter{The complex early modern Yoga taxonomies of the medieval Yogas}
\label{yogas_list}
\clearpage

\section{The texts of the complex Yoga taxonomies}

\subsection{Yogasiddhāntacandrikā}

Versucht alle 15 Yogas im Samādhipāda des Pātañjalayogaśāstra unterzubringen.
Siehe auch Powell 2023. 

%The Yogasiddhāntacandrikā is an ambitious work that attempts to unify a vast array of doctrines within the
%framework of the Yogasūtras of Patañjali. In particular, Nārāyaṇatīrtha attempts to synthesize fifteen
%different systems of yoga within the first chapter (pāda). All of these are believed to result in the state of
%Rājayoga, which Nārāyaṇatīrtha understands as synonymous with the nididhyāsana of Vedānta and
%the asaṃprajñātasamādhi of Pātañjalayoga.111


\section{Comparative analysis of the complex Yoga taxonomies}
\label{yogatax}

%%%%%%
%%%%%%hier adden das man sich von Rāmacandra gewünscht hätte, eine enzyklopädische Abhandlung der fünfzehn Yogas vorzufinden. Leider enttäuscht, strukturelle Probleme, bla bla ... aus diesem Grund "if you see a job its yours" 
%%%%%%
%%%%%%

The similarities between the Yoga taxonomies of Rāmacandra's \textit{Yogatattvabindu}, his source text, the \textit{Yogasvarodaya} as well as the taxonomies laid out by Nārāyaṇatīrtha in his \textit{Yogasiddhāntacandrikā} and Sundardās' \textit{Sarvāṅgayogadīpikā} which all emerged within the 17th century have been initially observed and discussed briefly by \citeauthor{birch2014} (2014).\footnote{See \citeauthor{birch2014}, 2014: 415-416.} I would like to call this specific literary phenomenon the ``complex early modern Yoga taxonomies of the medieval Yogas'' or simply ``complex Yoga taxonomies''. In the following chapter, the complex Yoga taxonomies and their single categories of Yoga are examined within a comparative analysis. The comparative analysis will follow the structure of the individual Yogas outlined in the \textit{Yogatattvabindu}. Each Yoga will initially be described based on the explanations in the \textit{Yogatattvabindu}, and its content will be compared with the explanations of the corresponding Yoga in the texts with similar taxonomies. Some Yogas only appear in the taxonomies of \textit{Yogatattvabindu} and \textit{Yogasvarodaya} but are not explicitly dealt with in the text. At these points, reference is made to this fact, and the analysis is continued based on the explanations of the other taxonomies that describe these Yogas.
The comparison will broaden and clarify our understanding of the respective spectrum of meanings of the individual Yoga categories in the discursive field of the authors of the texts containing the taxonomies. This comparison results in the documentation of the discursive web of word usage of various Yoga categories in the 17th century, most probably mainly localised in northeastern India. Individual Yoga categories that do not appear in the list of the \textit{Yogatattvabindu} but are listed in the other texts with complex taxonomies will also be covered and outlined. In addition, Yoga categories that do not appear in any of the analysed lists but are nevertheless mentioned in the texts will also be covered. Thus, this this comparative study will display the overall picture of all Yoga categories used during the period under consideration in an encyclopedic fashion and will serve scholars as a comprehensive reference. However, it is essential to emphasise that the comparison of Yoga categories is limited to those texts that contain the complex Yoga taxonomies. Although the analysis and comparison of the Yoga categories can be extended to other Yoga texts, locations, and periods if necessary or valuable, the restriction on the complex Yoga taxonomies is generally maintained to prevent this complex endeavour from going ad absurdum.\footnote{The historical tracing and analysis of developments in the reception history of the Yoga categories presented in the complex taxonomies generate valuable insights, as has been demonstrated by the example of the development of the late medieval Kriyāyoga into the modern forms of Kriyāyoga, beginning with the lineage of the world-famous Paramahaṃsa Yogānanda due to personal interest. See the chapter \textit{Excursus: Popularisation of a new Kriyāyoga in a global context} on p.\pageref{excursuss} et seqq. Unfortunately, it is beyond this work's scope to extend this analysis to the history of the reception of each Yoga category and term throughout the entire history of Yoga. A groundbreaking example for the history of Rājayoga is \citeauthor{birch2014} (2014), \citetitle{birch2014}. Even single yogic techniques can be extremely complex. For an outstanding article on the history of the haṭhayogic \textit{vajrolīmudrā} see for example \citeauthor{mallinson2018vajrolimudra} (2018), \citetitle{mallinson2018vajrolimudra}.} Ultimately, the comparative analysis of the texts, the authors and their multiple Yoga categories will help to formulate a new concise hypothesis as to why and under what circumstances the complex Yoga taxonomies emerged across traditions and largely independently of each other.

\begin{table}[H]
    \centering
    \begin{tabularx}{\textwidth}{>{\raggedright\arraybackslash}p{0.05\textwidth}XXXX}
        \toprule
        No. & \textit{Yogatattvabindu} & \textit{Yogasvarodaya} & \textit{Yogasiddhāntacandrikā} & \textit{Sarvāṅgayogadīpikā} \\
        \midrule
        1. & \textit{kriyāyoga} & \textit{kriyāyoga} & \textit{kriyāyoga} & \textit{\textbf{bhaktiyoga}} \\
        2. & \textit{jñānayoga} & \textit{jñānayoga} & \textit{caryāyoga} & \textit{mantrayoga} \\
        3. & \textit{caryāyoga} & \textit{karmayoga} & \textit{karmayoga} & \textit{layayoga} \\
        4. & \textit{haṭhayoga} & \textit{haṭhayoga} & \textit{haṭhayoga} & \textit{carcāyoga} \\
        5. & \textit{karmayoga} & \textit{dhyānayoga} & \textit{mantrayoga} & \textit{\textbf{haṭhayoga}} \\
        6. & \textit{layayoga}  & \textit{mantrayoga} & \textit{jñānayoga} & \textit{rājayoga} \\
        7. & \textit{dhyānayoga} & \textit{urayoga}   & \textit{advaitayoga} & \textit{lakṣayoga} \\
        8. & \textit{mantrayoga} & \textit{vāsanāyoga} & \textit{lakṣyayoga} & \textit{aṣṭāṅgayoga} \\
        9. & \textit{lakṣyayoga} & -                   & \textit{brahmayoga} & \textit{\textbf{sāṃkhyayoga}} \\
        10. & \textit{vāsanāyoga} & -                   & \textit{śivayoga} & \textit{jñānayoga} \\
        11. & \textit{śivayoga} & -                    & \textit{siddhiyoga} & \textit{brahmayoga} \\
        12. & \textit{brahmayoga} & -                  & \textit{vāsanāyoga} & \textit{advaitayoga} \\
        13. & \textit{advaitayoga} & -                 & \textit{layayoga} & - \\
        14. & \textit{siddhayoga} & -                  & \textit{dhyānayoga} & - \\
        15. & \textit{rājayoga} & - [\textit{rājayoga}]& \textit{premabhaktiyoga} & - \\
        16. & - & - & [\textit{rājayoga}] & - \\
        \bottomrule
    \end{tabularx}
    \caption{Complex Taxonomies of Yoga in Yoga Texts of the 17th - 18th Centuries}
    \label{tab:complextaxonomies}
\end{table}

\section{1. Kriyāyoga}

Kriyāyoga is the first method of Rājayoga within the list of fifteen Yogas presented by Rāmacandra and his source text \textit{Yogasvarodaya}. Remarkably, Nārāyaṇatīrtha also positions Kriyāyoga at the first position within the list of fifteen Yogas in his \textit{Yogasiddhāntacandrikā}. Sundardās, on the other hand, omits Kriyāyoga altogether. 

\subsection{Kriyāyoga in the \textit{Yogatattvabindu}}
\label{kriyaintro}
Since Rāmacandra refers to all fifteen Yogas as variants of Rājayoga in his initial definition of Yoga, and no explicit hierarchy is recognisable from his formulations in the text, all variants of Rājayoga appear to have been regarded by him as equally effective. All Yogas aim towards the same goal: long-term durability of the body (\textit{bahutarakālaṃ śarīrasthitiḥ}). The positioning of Kriyāyoga does not initially provide any information about the efficiency or the assignment of differently talented practitioners to a particular type of Yoga, as was the case in i.e. the widespread fourfold taxonomies.\footnote{According to \citetitle{amaraugha2024}\textit{prabodha} 18-24, Mantrayoga is best suited for the weak, Layayoga for the average, Haṭhayoga for the talented and Rājayoga for the exceptionally talented practitioner. In \citetitle{datta2024} 14, one finds the statement that the lowest practitioner should perform Mantrayoga, which is then also referred to as the lowest Yoga. \citetitle{mallinson2007} 12-28 expands this fourfold scheme of Yogas and practitioners with a temporal dimension. The weak practitioner needs twelve years to succeed with Mantrayoga, the average practitioner needs eight years with Laya, the able practitioner six years with Haṭha and the exceptional practitioner three years with Rājayoga.} Implicit hierarchical aspects are nevertheless present - although all Yoga types are a type of Rājayoga, Rāmacandra nonetheless places Rājayoga in the final and topmost position of his taxonomy.
The only apparent reason why Rāmacandra specifies Kriyāyoga as the first Yoga seems to be that his primary source text, whose content structure he largely follows, specifies this type of Yoga as the first.

The passage on Kriyāyoga in the \textit{Yogatattvabindu} is relatively short. The four verses presented by Rāmacandra are quoted without attribution from the \textit{Yogasvarodaya}. A prose section repeats the content of the verses. By definition, Kriyāyoga in the \textit{Yogatattvabindu} is ``liberation through [mental] action'' (\textit{kriyāmuktir ayaṃ yogaḥ}). In contrast to Rāmacandra's worldly definition of Rājayoga and its subcategories, here, liberation (\textit{mukti}) overrides this initial goal. In addition, the practitioner achieves ``success in one's own body'' (\textit{svapiṇḍe siddhidāyakaḥ}). The method of Kriyāyoga involves restraining any [mental] wave before an action. This restraint consists of reducing negative [mind-]waves and cultivating positive ones. Noticeably, the number of negative waves significantly exceeds the number of positive waves.

\begin{table}[H]
    \centering
    \begin{tabularx}{\textwidth}{XX}
        \toprule
        \textbf{Mental waves to be cultivated} & \textbf{Mental waves to be reduced} \\
        \midrule
        Patience (\textit{kṣamā}) & Envy (\textit{matsārya}) \\
        Discrimination (\textit{viveka}) & Selfishness(\textit{mamatā})\\
        Equanimity (\textit{vairāgya}) & Cheating (\textit{māyā})\\
        Peace (\textit{śānti}) & Violence (\textit{hiṃsā})\\
        Modesty (\textit{santoṣa}) & Intoxication (\textit{mada})\\
        Desirelessness (\textit{niṣpṛha}) & Pride (\textit{garvata})\\
        & Lust (\textit{kāma}) \\
        & Anger (\textit{krodha}) \\
        & Fear (\textit{bhaya})\\
        & Laziness (\textit{lajjā})\\
        & Greed (\textit{lobha})\\
        & Error (\textit{moha})\\
        & Impurity (\textit{aśuci})\\
        & Attachment and aversion (\textit{rāgadveśau}) \\
        & Disgust and laziness (\textit{ghṛṇālasya})\\
        & error (\textit{bhrānti})\\
        & Deceit (\textit{daṃbha})\\
        & Envy (repeatedly) (\textit{akṣama})\\
        & Confusion (\textit{bhrama})\\
        \bottomrule
    \end{tabularx}
    \caption{Mental waves to be cultivated and reduced in Rāmacandra's Kriyāyoga}
    \label{tab:waves}
\end{table}

The one who cultivates positive [mind-]waves and reduces the negative is called a \textit{kriyāyogī}. In the prose passage of the section, the term \textit{bahukriyāyogi} is used. The term is unprecedented in the rest of Yoga literature and presumably intends to express the great amount of reduced and cultivated [mind-]waves.\footnote{Cf. section \uproman{2} of the \textit{Yogatattvabindu} for its text on the subject Kriyāyoga.} 

\subsection{Kriyāyoga in the \textit{Yogasvarodaya}}
A closer examination of the Kriyāyoga section in the \textit{Yogasvarodaya} reveals Rāmancandra's reductionism since he excludes significant aspects of the original concept of the \textit{Yogasvarodaya}'s Kriyāyoga.

%YK 1.214-216

\begin{quote}
  \begin{ekdosis}
\textit{dhyānapūjādānayajñajapahomādikāḥ kriyāḥ} |\\
\textit{kriyāmuktimayo yogaḥ\app{\lem[wit={YTB}]{svapiṇḍe siddhidāyakaḥ}
  \rdg[wit={PT}]{sapiṇḍisiddhidāyakaḥ}
  \rdg[wit={YK}]{sapiṇḍisiddhidāyakaḥ}}} || 1 ||

(1) Actions are meditation, ritual veneration, donation, recitation, fire sacrifice, etc. 
The Yoga made of liberation through action[s] bestows success in one's own body. 

\textit{yat karomīti saṅkalpaṃ kāryārambhe manaḥ sadā} |\\
\textit{tat sāṅgācaraṇaṃ kurvan kriyāyogarato bhavet} || 2 ||

(2) ``Whatever I do'' at the beginning of an action, the mind always has an intention.  
Doing that [following] procedure with all its parts, one becomes established in Kriyāyoga.  

\textit{kṣamāvivekavairāgyaśāntisantoṣanispṛhāḥ} |\\
\textit{etad yuktiyuto yo'sau kriyāyogo nigadyate} || 3 ||

(3) Patience, discrimination, equanimity, peace, modesty, desirelessness:
The one endowed with these means is said to be a Kriyāyogī.

\textit{mātsaryaṃ mamatā māyā hiṃsā ca madagarvitā} |\\
\textit{kāmaḥ krodho bhayaṃ lajjā lobho mohas tathā 'śuciḥ} || 4 ||

(4) Envy, selfishness, cheating, violence, intoxication and pride,
lust, anger, fear, laziness, greed, error, and impurity.

\textit{rāgadveṣau ghṛṇālasyaśrāntidambhakṣamābhramāḥ} |\\
\textit{yasyaitāni na vidyante kriyāyogī sa ucyate} || 5 ||

(5) Attachment and aversion, disgust and laziness, error, deceit, envy [and] confusion:
Whoever does not experience these is called a Kriyāyogī.

\textit{sa eva muktaḥ sa jñānī caṇḍināśena īśvaraḥ} |\\
\textit{kriyāmuktikaro yo'sau rājayogaḥ sa muktidaḥ} || 6 || (om. YK)

(6) He alone, the wise one, the lord, through the destruction of impetuous [behaviour]
who performs the liberation through action[s] is liberated. This Rājayoga is the bestower of liberation.

\textit{yāvan mano layaṃ yāti kṛṣṇe svātmani cinmaye} | \\ 
\textit{bhaved iṣṭamanā mantrī japahomau samabhyaset} || 7 ||\footnote{7ab \approx \citetitle{rudrayamala1937} 38.58cd.} (om. YSv) 

(7) Until the mind enters absorption into Kṛṣṇa, in one's own self, into consciousness,
the mantra practitioner (\textit{mantrin}) should practise recitation and fire sacrifice with an aspiring mind. 

\textit{vidite paratattve tu samastair niyamair alam} |\\
\textit{tālavṛntena kiṃ kāryaṃ lavdhe malayamārute} || 8 ||\footnote{\approx \citetitle{kularnavatantra} 9.28 \& \citetitle{yuktabhavadeva} 1.80.} (om. YSv) 

(8) When the highest principle has been realised through all the \textit {niyama}s, as is proper,
why should one wave the palm frond when the wind from the Himalayas has already reached?

\textit{tāvat karmmāṇi kurvanti yāvajjñānaṃ na vidyate} |\\ 
\textit{jñāne jāte pareśāni karmākarma na vidyate} || 9 || (om. YSv) 

(9) As long as [regular?] actions are performed, so long realisation is unknown.
When knowledge ensues, oh, Supreme Goddess, neither action nor non-action is known.
\end{ekdosis}
\end{quote}


These verses\footnote{The numbering used here was introduced by me for practical reasons and does not correspond to the original numbering of the verses in the citations of the source texts. The \textit{Prāṇatoṣiṇī} does not number the verses at all. The verses can be found in the printed edition of the \textit{Prāṇatoṣiṇī} on p. 831. The verses here are in the \textit{Yogakarṇikā} with the numbering 1.209-216 and can be found in the edition on p. 17.} stem from the only two currently available sources of the \textit{Yogasvarodaya}, namely the quotations from the \textit{Prāṇatoṣiṇī}\footnote{A considerable part of the \textit{Yogasvarodaya} is quoted with source reference (\textit{yogasvarodaye}).} and the \textit{Yogakarṇikā}.\footnote{Normally the \textit{Yogakarṇikā} quotes its sources. This passage is one of the few exceptional cases in which the verses have been taken from the \textit{Yogasvarodaya} without citing the source. However, this passage ends after verse 1.216 with ``\textit{iti yogasaṅketāḥ} |''.} The quotations of both texts essentially correspond, but the last verses of the passage differ. It cannot be ruled out that the last three verses of the \textit{Yogakarṇikā} in particular come from a different source and were not present within the \textit{Yogasvarodaya}. However, their content is so closely interwoven with the preceding verses that this scenario can be considered unlikely.

The main difference to the Kriyāyoga that Rāmacandra has constructed from these verses is the definition of the actions (\textit{kriyāḥ}) mentioned immediately at the beginning of the verses, of which the actions (\textit{kriyā}s) of Kriyāyoga is then predominantly composed, namely of (1) meditation, (2) ritual worship of God, (3) offerings, (4) recitation and (5) fire sacrifice, etc. Furthermore, while Rāmacandra declares the elements mentioned in the table \ref{tab:waves} as waves (\textit{kallola}) of the mind which are either required to be cultivated or reduced before any action is executed, the same elements are conceptualised in the \textit{Yogasvarodaya} as the intentions (\textit{saṅkalpa}) preceding the previously defined actions (\textit{kriyā}s), which should be observed.

In the three verses concluding this section, which are only handed down in the \textit{Yogakarṇikā}, the practitioner is referred to as \textit{mantrin} and should perform recitation and fire offerings until entering absorption (\textit{laya}).

A possible historical link, particularly in front of the Vaiṣṇava background, is the model of Kriyāyoga as found in the \textit{Uddhavagīta}\footnote{See i.e., \citeauthor{uddhavagita2007} (2007).} which is a part of the famous \textit{Bhāgavatapurāṇa}\footnote{See i.e., \citeauthor{bhagavata} (1950).}. Here, in chapter XXII.1-55 Kṛṣṇa describes a Vaiṣṇava form of Kriyāyoga in response to a request by his disciple Uddhava. The practice entails a very complex and devotional ceremonial veneration of the deity through offerings such as flowers and food, accompanied by the recitation of prescribed mantras, meditation, and the ritual consecration of the deity, among other rites. According to the text, this type of Yoga is the most beneficial for women and the working class (22.4) and is considered a means for liberation from the fetters of Karma (22.5). The Kriyāyoga described here is presented to be in line with both the Vedas and the Tantras, considering enjoyment (\textit{bhukti}) and liberation (\textit{mukti}) and is promised to bestow perfection in both this life and the next, by the Lord's grace (22.49).  

Furthermore, this concept of Kriyāyoga in the \textit{Yogasvarodaya} might be linked to the \textit{kriyāpāda}\footnote{See e.g. \citeauthor{ganesan2016saiva} (2016) and \citetitle{mrgendragama}, Ed. pp. 1-205.} of the Śaiva \textit{āgama}s. The Śaiva \textit{āgama}s are collections of various tantric traditions, written in Sanskrit or Tamil, in which cosmology, epistemology, philosophical teachings, various practices such as meditation or Yoga, mantra recitation, worship of the gods, etc. are described. These texts\footnote{The fourfold division of \textit{pāda}s is only present in a limited number of Āgamas: \textit{Kiraṇa}, \textit{Suprabheda}, \textit{Mṛgendra} and \textit{Mataṅgaparameśvara} (as Upāgamas), see \citeauthor{brunner1994place} , 1993: 225-461 for an overview.} usually consist of four sections (\textit{pāda}s): The \textit{jñānapāda} (knowledge section), \textit{kriyāpāda} (action section), \textit{caryāpāda} (behaviour section) and the \textit{yogapāda} (yoga section).\footnote{The order or the \textit{pāda}s varies, but the \textit{yogapāda} is always the last.} It can be no coincidence that \textit{jñāna°}, \textit{kriyā°} and \textit{caryā°} were each integrated as a separate Yoga category within the taxonomy of the fifteen Yogas\footnote{see p.\pageref{intro}.}. The \textit{kriyāpāda} is the section of a Śaiva \textit{āgama} that describes rules and practices for the performance of various rituals such as the significant initiation (\textit{dīkṣa}), ceremonies and worship of the gods. Additionally, \textit{prāṇāyāma} techniques and meditations are often found as parts of these rituals. There are also explanations of the nature of \textit{mudrā}s, \textit{maṇḍala}s and \textit{mantra}s. Furthermore, various characteristics of different types of Śaiva initiates\footnote{These are \textit{samayin, putraka, sādhaka, ācārya,} and \textit{astrābhiṣeka}.} can be found here.\footnote{See \citeauthor{ganesan2016saiva} (2016) for a general overview of the four \textit{pāda}s. One of the few Śaiva \textit{āgama}s that has been edited and translated into a Western language (French) is the \citetitle{mrgendragama}. For this see \citeauthor{mrgendragama} (1962) \& \citeauthor{mrgendragamabrunner} (1985).} The \textit{kriyā}s mentioned at the beginning of the \textit{Yogasvarodaya} - meditation, ritual veneration, donation, recitation, fire sacrifice, etc. have hardly deniable parallels to the \textit{kriyāpāda}s of the Śaiva \textit{āgama}s and thus could have their reception-historical roots precisely there. The other part, however, which describes the cultivation or reduction of certain mental configurations preceding all actions (\textit{saṅkalpa}) or [mental] waves (\textit{kallola}), I have not yet been able to locate in the Śaiva \textit{āgama}s, but they seem to be a simplyfied rendering of the Pātañjalean model of Kriyāyoga that was passend on in hitherto unknown traditions that practiced this type of Kriyāyoga.

\subsection{Kriyāyoga in the \textit{Yogasiddhāntacandrikā}}

The Kriyāyoga in Nārāyaṇatīrtha's commentary on \textit{Pātañjalayogaśāstra} entitled \textit{Yogasiddhāntacandrikā} presents Kriyāyoga as the first of his fifteen Yogas, which he locates in Pātañjalayoga.\footnote{For an earlier brief discussion of Kriyāyoga in Nārāyaṇatīrtha's \textit{yogacandrika} see \citeauthor{penna2004}, 2004: 62-66.} The term Kriyāyoga occurs in \textit{Pātañjalayogaśāstra} 2.1. According to the introduction to this \textit{sūtra}, in the \textit{bhāṣya}-part of the \textit{Pātañjalayogaśāstra}, Kriyāyoga is the means by which someone with a distracted mind can also attain Yoga (\textit{vyutthitacitto 'pi yogayuktaḥ}). In \citetitle{yogasutra} 2.1, Kriyāyoga is defined as follows:
\begin{quote}  
  \textit{tapaḥsvādhyāyeśvarapraṇidhānāni kriyāyogaḥ} |
\end{quote}
\begin{quote}
The Yoga of action consists of auterity, the self-study and devotion to the supreme lord. 
\end{quote}

Kriyāyoga, or ``yoga of action'', is the action oriented method of Yoga consisting of three elements. Namely, austerity (\textit{tapas}), which according to the \textit{bhāṣya} should be practised both mentally and physically, the repetition of \textit{mantra}s or the study of sacred literature (\textit{svadhyāya}) and devotion to the supreme lord (\textit{īśvarapraṇidhāna}).
According to \citetitle{yogasutra} 2.2, these three elements of Kriyāyoga should lead the practitioner to attain \textit{samādhi} by reducing the so-called \textit{kleśa}s. This explanatory model is picked up by Nārāyaṇatīrtha.\footnote{\citeauthor{yogacandrika}, 2000:71.} The five \textit{kleśa}s consist of ignorance (\textit{avidyā}), self-centredness (\textit{asmitā}), attachment (\textit{rāga}), aversion (\textit{dveṣa}) and fear of death (\textit{abhiniveśa}). 
All three main components of Patañjali's Kriyāyoga are not mentioned in the \textit{Yogatattvabindu} and \textit{Yogasvarodaya}. Nevertheless, a practice similar to the reduction of the \textit{kleśa}s can also be found here. Although the specific fear of death (\textit{abhiniveśa}) is not mentioned, the more general term for fear (\textit{bhaya}) is cited.\footnote{The details of Nārāyaṇatīrtha's understanding of Kriyāyoga have already be discussed by \citeauthor{penna2004} (2004: 62-66) and will therefore not be covered here again.}
The Kriyāyoga in \textit{Yogatattvabindu} and \textit{Yogasvarodaya} could, therefore, be perhaps regarded as a degenerated or simplified variant of the Pātañjalean model, which restricts itself predominantly to the aspect of the reduction of negative waves of the mind, which is comparable to the reduction of \textit{kleśa}s and adds the aspect of cultivating positive mind waves to be mix. In both systems, Kriyāyoga is a means for liberation.\footnote{The Kriyāyoga of the \citetitle{yogasutra} will not be dealt with in detail here, as this has already been done in countless academic and informal publications. For the \textit{sūtra}s related to Kriyāyoga and Patañjali's autocommentary in Sanskrit with English translation, see \citeauthor{yogasutra} 1983: 113 et seqq. For a comprehensible and more accessible overview, see \citeauthor{bryant2009} 2009: 170 et seqq.}

\subsection{Kriyāyoga in the complex early modern Yoga taxonomies}

The comparative analysis of Kriyāyoga within the complex Yoga taxonomies shows two distinct models. One is Nārāyaṇatīrtha's model, which draws directly on the Kriyāyoga of \textit{Pātañjalayogaśāstra}. Additional Śaiva influences characterise the other model of Kriyāyoga that seems to have been locally prominent in the 17th century. The precisely defined \textit{kriyā}s of the \textit{Yogasvarodaya} must be historically linked to the \textit{kriyāpāda}s of the Śaiva \textit{āgama}s, whereby the core practice of reducing and cultivating specific mental configurations before any action is loosely associated with the Kriyāyoga of the \textit{Pātañjalayogaśāstra}. The observation that the \textit{kriyā}-, \textit{caryā-}, and \textit{jñānayoga}s, are an allusion to the \textit{kriyā}-, \textit{caryā-}, \textit{jñāna-} and \textit{yogapāda}s of the Śaiva \textit{āgama}s, shows that Nārāyaṇatīrtha, as a proponent of the \textit{Pātañjalayoga}, was most likely not the originator of the fifteenfold taxonomy, but rather that the taxonomy of the fifteen Yogas originated in local discourses around the authors and had achieved such local popularity at the time that Nārāyaṇatīrtha forced the fifteenfold taxonomy into Patañjali's \textit{Yogaśāstra} in order to show that the Yogaśāstra \textit{par excellence} and all those varieties of Yogas that were discussed in his sphere are in truth just single aspects of the superior ``classical'' system of Patañjali.

\subsection{Excursus: Popularisation of a new Kriyāyoga in a global context}
\label{excursus}
The comparatively unique treatises on Kriyāyoga, which can only be found in the Yoga literature of the 17th-century\footnote{The terminus \textit{ad quem} for the \textit{Yogasvarodaya} and \textit{Yogatattvabindu} is 1659 CE, see p.\pageref{dating} for the details.} in \textit{Yogasvarodaya} and Rāmacandra's \textit{Yogatattvabindu}, which deviate from the Pātañjala model, albeit not entirely, and, as shown, show clear influences of tantric origin, can be regarded as marginal phenomena for the time being. The briefly touched upon model of \textit{Uddhavagītā}, which describes a Kriyāyoga method for \textit{mukti} and \textit{bhukti} through ritual worship of god, is also comparatively rare in the literature. The overwhelming majority of the Sanskrit yoga texts written in the second millennium CE, as in the case of Nārāyaṇatīrtha's \textit{Yogasiddhāntacandrikā}, are based on the model of Kriyāyoga propagated in the \textit{Pātañjalayogaśāstra} and the commentary literature. Accordingly, it was above all the publication of the \textit{Yogasūtra} in the West, beginning with the translation by Henry Thomas Colebrooke in 1805\footnote{See \parencite{colebrooke2014} for a detailed discussion,} which ensured that the concept of Kriyāyoga contained therein also dominated the understanding of the term in academic and informal discourse in the West for a long time. 

The Western discourse only changed with the global success and popularity of Paramahaṃsa Yogānanda (1893-1952) and the \textit{Self Realisation Fellowship} he founded in 1920, which, measured against the predecessor models forms of Kriyāyoga outlined above, spread an innovative Yoga practice under the generic term Kriyāyoga. The influence of Yogānanda and others significantly changed and expanded the range of meanings of the term Kriyāyoga. In addition to various books published by Yogānanda, it was above all, the book \citetitle{autobioyogi}, the autobiography of Yogānanda himself, published in 1946, which paved the way for Yogānanda's success. To this day, this work is considered a classic in popular Yoga literature, has been in print for over seventy years and has been translated into more than 50 languages.\footnote{Cf. \cite{yoganandawebsite}.} It also has a large global following to this day. Yogānanda, his books, his followers and the numerous books written by his followers have popularised this innovative and new form of Kriyāyoga beyond the Indian subcontinent. The term Kriyāyoga was allegedly already defined by Yogānanda's predecessors, namely Lahiḍi Mahāśaya (1828-1895) and Śrī Yukteśvar Giri (1855-1936), as the central generic term for the Yoga practice of this specific lineage.\footnote{Cf. \citeauthor{govindan2010} 2010:51-52} 

One of Yogānanda's contemporaries was Svāmī Śivānanda Sarasvatī (1887-1963), who similarly propagated a new form of Kriyāyoga. Although his Kriyāyoga was initially based mainly on the Pātañjalayoga model, it was expanded under the same umbrella term with Haṭhayoga practices and possibly influenced by Yogānanda's model. This expansion and integration of new practices under the umbrella term Kriyāyoga was continued excessively by his students, above all Svāmī Satyānanda Sarasvatī (1923-2009), the founder of the famous \textit{Bihar School of Yoga} (since 1962).

The resulting popularity of Kriyāyoga triggered a global wave and inspired others, who in turn developed similar but sometimes differently nuanced Kriyāyoga systems. One example is S.A.A. Ramaiah, who founded the \textit{Kriya Babaji Yoga Sangam} in 1952. In this case, too, there is a global following.\footnote{Cf. \cite{kriyababajiyoga}}.

It was the actors mentioned above, above all Yogānanda, who ensured the global popularisation of this new form of Kriyāyoga so that their concepts are at least as well known in recent public discourse, if not better known, than the Kriyāyoga of the \textit{Pātañjalayogaśāstra}.

These new forms of Kriyāyoga, which can only be traced from the beginning of the 19th century, are, as will be shown, a reservoir for innovative combinations and further developments of numerous practices already codified in Yoga texts in the medieval to pre-colonial period, which were integrated into seemingly coherent practice systems by actors such as Yogānanda, Śivānanda, Ramaiah, etc. The statements made by their traditions about the historicity of their Yoga practice utilise established narratives to lend this form of Kriyāyoga a tradition and historical legitimacy.\footnote{For example, the tracing back of the Yoga tradition to a legendary founding figure, the time of the master in the Himalayas, lost writings that suddenly reappear and legitimise the practice can already be found in a similar form in the lineages of T. Krishnamarcharya. See \citeauthor{singleton2013gurus}, 2013: 81-121.}

\subsection{The Kriyāyogas of the lineages of Paramahaṃsa Yogānanda, Svāmī Śivānanda Sarasvatī and Ramaiah}

So what constitutes these new forms of Kriyāyoga? To answer this question, recent publications on this topic were consulted.\footnote{This list is certainly not exhaustive. Nevertheless, I have consulted a wide range of these publications available to me. 1. For the Yogānanda model: \citeauthor{autobioyogi} (1949); \citeauthor{kriyayogalowenstein} (2021); \citeauthor{kriyayogasarasvati1981} (1981); \citeauthor{hariharananda1989} (1989); \citeauthor{kriyayogaupanishad1993} (1993) and \citeauthor{kriyayogasturgess2015} (2015). 2. For the Śivānanda model: \citeauthor{shivanandakriya1982} (1955) and \citeauthor{kriyayoganityananda2013} (2013). 3. And for the the Ramaiah model: \citeauthor{govindan2010} (2010).} The following is a brief outline of the main features of the Yogānanda, Śivānanda and Ramaiah models of Kriyāyoga without claiming to be exhaustive. To my knowledge, a comprehensive and complete historical study of Kriyāyoga has not yet been carried out and cannot be done within this framework. This attempt is an outline and should be understood as a first approach to the topic in order to differentiate between the models circulating in public discourse on the one hand and, on the other, to formulate a hypothesis on the transition from the older models to the newer models, as these are very close in time.  

\subsubsection{Definitions}

The publications consulted contain various creative etymologies and explanations of the term Kriyāyoga. \citeauthor{hariharananda1989}, a Kriyāyoga teacher authorised by Yogānanda \footnote{Cf. \citeauthor{hariharananda1989} 1989: 16.} himself explains: \begin{quote} 'Kriya Yoga' are Sanskrit words, a combination of two root words. One is Kriya and the other is yoga. In the word Kriya there are two syllables: kri and ya. Kri means to pursue your work in daily life and ya means to be ever aware of the invisible God who is abiding in you and is directing and accomplishing work through you. \ldots  The second word, 'yoga,' literally means union of the visible body with the invisible body. This union is always present in everyone. (\citeauthor{hariharananda1989} 1989: 83) \end{quote}
Another etymology of the term \textit{kriyā} can be found in \citeauthor{kriyayogalowenstein} (2021: 91): \begin{quote} \ldots kri meaning ``work'' and ya meaning ``soul'' or ``breath'' = The Work to be done with the Souls breath. \end{quote}
The most complex explanation of the term can be found in \citeauthor{kriyayoganityananda2013} (2013: 2-3), who also situates himself in the Yogānanda tradition: \begin{quote}
  The word \textit{kriyā} is composed of the letters \textit{k}, \textit{r}, \textit{i}, \textit{y}, and \textit{ā}. The letter -\textit{k} (or \textit{ka}), \textit{ka-kāra}, represents the Lord, \textit{Īśvara}. The Transcendental Lord, \textit{Parama Śiva}, when he manifests Himself in the suble world and makes Himself ready for creation He becomes \textit{Īśvara}. The letter-\textit{r} (or \textit{ra}), \textit{ra-kāra}, represents fire, light and manifestation. Creation is not seen by us with the ether and air elements since these are subtle elements. We are able to see manifestation from the fire element onwards. The letter -\textit{i}, \textit{i-kāra}, represents energy or \textit{śakti}. So \textit{kri} is the activating power of the Lord manifested in creation. The activating power is called \textit{prāṇa} or vital force. The letter -\textit{y} (or \textit{ya}), \textit{ya-kāra}, represents the air element and the letter -\textit{ā}, \textit{ā-kāra}, represents form. For the manifestations to take a form, \textit{ākāra}, the Lord acts with the air element. With the ether element there is no form. The air element or gaseous state is the first created form although we only see the forms from the fire element onwards. Through the action of air the whole universe is manifested. This is the action of the Life-force, \textit{prāṇakarma}, of the Lord. The word \textit{kriyā} normally means action, but this is the action of god. We are made with the same principle God is. Our identification with the physical body makes us separate from God and this is the state of ignorance. We have to eradicate this ignorance by the action of God, i.e., the action of the breath, \textit{prāṇakarma}. Our mind is the result of ignorance and is responsible for the wrong identification. Breath-practice, \textit{prāṇakarma}, absorbs the mind into the vital force. This action of God reverses the process and leads us from body to God. This is why it is so necessary to perform that action. That is our spiritual practice. Then that action, \textit{kriyā}, becomes yoga. \end{quote}
\citeauthor{kriyayogasarasvati1981} (1981: 699), an important proponent of the Śivānanda model, defines Kriyāyoga as follows: \begin{quote} The Sanskrit word \textit{kriya} means `action' or `movement'. \textit{Kriya Yoga} is so called because it is a system where one intentionally rotates one's attention along fixed pathways. This movement of awareness is done, however with control. Also kriya yoga is so called because one moves the body into specific mudras, bandhas and asanas according to a fixed scheme of practice. The word \textit{kriya} is often translated as meaning `practical'. This is indeed a good definition, for kriya yoga is indeed practical. It is concerned solely with practice, without the slightest philosophical speculation. The system is designed to bring results, not merely to talk about them. Sometimes the word \textit{kriya} is translated as `preliminary'. This too is a good definition, for kriya yoga is a preliminary practice that leads first to dharana and then eventually to the transcendental state of dhyana (meditation) and yoga (union). It is a technique which has been designed to lead to that state of being which is beyond all techniques. Finally, the word \textit{kriya} is used to describe each individual practice. Thus the process of kriya yoga consists of a number of kriyas each being done one after the other in a fixed sequence.\end{quote}
\citeauthor{govindan2010} (2010: 214), a student of Ramaiah offers a simple explanation of the term: \begin{quote} Kriyā is an activity performed with mindfulness.\end{quote}

As different as the concepts presented here may seem, they have in common that they are about consciously performed actions or practices that connect people with God or are intended to bring about a transcendent state, a state of Yoga. In his definition, \citeauthor{kriyayoganityananda2013} already mentions the central action (\textit{kriyā}) that should lead to a connection with God, namely breathing practice (\textit{prāṇakarma}). In addition, \citeauthor{kriyayogasarasvati1981} also mentions other practices such as directing attention, \textit{mūdra}s, \textit{bandha}s and \textit{āsana}s.  

Further definitions can be found in the consulted texts. However, these are sufficient for the purposes here, as they illustrate the basic idea of the new models of Kriyāyoga on the one hand and show the fundamental diversity and openness of the model, which permeates all areas of these new forms of Kriyāyoga, on the other.  

\subsubsection{Histories of the new forms of Kriyāyoga from an emic perspective}

\citeauthor{kriyayoganityananda2013} (2013: 2-7), who places himself in the lineage of Yogānanda, explains that Kriyāyoga is an eternal tradition that stands at the beginning of human history. He explains that this is why many of the scriptures, such as the \textit{Śivasūtrā}, the \textit{Āgama}s and the writings of the Siddhas, teach the techniques and principles of Kriyāyoga in many different ways. Moreover, remnants of this primal Kriyāyoga can be found in almost all philosophies, be it Buddhism, Jainism, Sāṅkhya, Vaiśeṣika, Nyāya, Mīmāṃsā or Vedānta. 

\citeauthor{kriyayogasarasvati1981} (1981: 699), the founder of the \textit{Bihar school of Yoga}, explains that there is no history of Kriyāyoga and that its origins and development have been lost. Furthermore, the system of Kriyāyoga was so secret that there is not even a myth to explain its origin. Furthermore, he describes that parts of the Kriyāyoga taught by him are contained in the texts of Haṭhayoga, such as \textit{āsana}s, \textit{mudrā}s and \textit{bandha}s, but that these are not ``integrated together''. Furthermore, he speculates that Kriyāyoga must have been known in China, as he sees strong parallels to practices in \textit{Tai Chi Chuan}. Furthermore, he clearly distances himself from the Kriyāyoga of the \textit{Yogasūtra}, which has nothing to do with the Kriyāyoga of his book \citetitle{kriyayogasarasvati1981} and serves solely as a preparation for Rājayoga. However, the only definitive historical statement he can commit himself to is the following: \begin{quote} Of history, all we will say is that kriya yoga was passed on by Swami Sivananda of Rishikesh. \end{quote} Surprisingly, this same \citeauthor{shivanandakriya1982} of Rishikesh in his book \citetitle{shivanandakriya1982} (1955) explicitly traces the Kriyāyoga he taught back to \textit{Yogasūtra} 2.1. \citeauthor{shivanandakriya1982} (1982:168-182) uses the Kriyāyoga of the \textit{Yogasūtra} as the overarching framework of his teaching, which also integrates \textit{ṣatkarma} and breathing exercises from Haṭhayoga into it.

It is important to emphasise that \citeauthor{kriyayogasarasvati1981} recognises that the traditional lineage of Yogānanda also practises the same Kriyāyoga he teaches. However, he explicitly distances himself from their narrative: \begin{quote} Of course, there are various other groups of people in India who have practiced and taught kriya yoga. For example, Swami Yogananda, Yukteshwar Giri, Lahiri Mahasaya, Mahatma Gandhi and so forth practiced kriya yoga. In fact, a thriving organization still propagates it throughout the world. They also do now know the origin of kriya yoga, but they say that it was reintroduced by the great yogi Babaji as the ideal practice for sincere seekers of wisdom in the present Kali Yuga (Dark Age). \end{quote}

This narrative is by far the most widespread explanation of the origins of the new Kriyāyoga and is adopted not only in the tradition of Yogānanda, but also in the tradition of Ramaiah. In his book \textit{Kriya Yoga and the 18 Siddhas} (2010: 31-64), \citeauthor{govindan2010}, a disciple of Ramaiah, has compiled this narrative in detail, which I would now like to summarise in a nutshell.

Mahāvātara Babajī, who according to \citeauthor{govindan2010} is considered an incarnation of the Buddha, was born in 203 CE in Parangipetta in Tamil Nadu under the name Najaraj into a Brahmin family, joined a group of wandering Saṃnyāsins at a young age and studied the holy scriptures. His path soon led him to Śrī Laṅka in Katirkāma (now Kataragama), where he became a disciple of Siddha Boganathar and was initiated by him into various \textit{kriyā}s such as \textit{dhyāna}, \textit{āsana}, \textit{mantra} and \textit{bhaktiyoga}. Bhoganathar later sent Babajī to another teacher, namely Siddha Agastya in Courtallam in the Pothihai hills of Tamil Nadu, located in today's Tinneveley district. He learnt the particularly important \textit{kriyā} called \textit{kuṇḍalinīprāṇāyāma} from him. Agastya then sent Babajī to Badrinath in the Himalayas, where he practised for many months and finally attained \textit{samādhi}. After his enlightenment and attaining immortality at the tender age of 16, Babajī set himself the task of helping suffering humanity in its search for God-realisation. As an immortal, Babajī initiated great personalities such as Śaṅkarācārya (788-820) and Kabīr (1440-1518) into the techniques of Kriyāyoga over the centuries. Finally, in 1861, he initiated Lahiḍi Mahāśaya (1828-1895) into Kriyāyoga and gave him the task of passing it on to serious seekers. At this point, \citeauthor{govindan2010} quotes the autobiography of Yogānanda,\footnote{Cf. \citeauthor{autobioyogi}, 1949: 244 f.} which states that Babajī explained to Lahiḍi Mahāśaya that Kṛṣṇa had once passed on Kriyāyoga to Arjuna and that not only Patañjali knew it, but also Jesus Christ, who in turn had passed it on to John, Paul and other disciples. Among Lahiḍi Mahāśaya's 100 disciples was Śrī Yukteśvar (1855-1936), to whom Babajī is also said to have appeared three times. On one of these occasions, Babajī decided that he should send his disciple Yogānanda (1893-1952) to America to spread Kriyāyoga, which he did, gaining global fame and founding the \textit{Self Realisation Fellowship} in 1920, which is still very active today.

\subsubsection{The practice of the new Kriyāyoga}

In the following, the practices of the new Kriyāyoga are presented in outline based on the publications mentioned and consulted above.\footnote{A comprehensive presentation and comparative analysis of the practices in the various traditions of the new Kriyāyoga would be too far-reaching for this chapter. The most detailed written practice instructions that I have consulted can be found for the Śivānanda/Satyānanda model in \citeauthor{kriyayogasarasvati1981}, (1981: 697-952) and for the Yogānanda model in \citeauthor{kriyayoganityananda2013}, (2013: 249-340).} The words of \citeauthor{hariharananda1989} (1989: 144) are surprisingly apt to give an essential first impression of this complex phenomenon: \begin{quote} Kriya Yoga is the essence and synthesis of all yoga techniques taught in the world.  \end{quote} 
\citeauthor{kriyayogasarasvati1981} (1981:703) explains that each Kriyā consists of a certain number of subordinate techniques. These always consist of a combination of the following six tools: \textit{āsana}, \textit{mudrā}, \textit{bandha}, \textit{mantra}, \textit{prāṇāyāma} and, as he calls it, `psychic passage awareness'. This last point includes a group of exercises mainly involving ``circulating awareness through the \textit{cakra}s in an ascending and descending way'' or similar. A single Kriyā is an exercise unit comprising individual exercises from the six categories mentioned. However, these are not arbitrary but are integrated into a specific, and, as the protagonists of this tradition say ``scientific way'' in order to induce the process of concentration (\textit{dhāraṇa}), meditation (\textit{dhyāna}) and meditative absorption (\textit{samādhi}). The main distinguishing feature from other yoga systems is the innovative and specific combination of the individual techniques into a practical and particularly effective sequence of exercises, referred to here as ``Kriyā''.

In every model the individual exercises are drawn from the vast body of Yoga literature but primarily from the exercises taught in the medieval to pre-colonial texts of the Haṭha- and Rājayoga genres. This always takes place against the background of tantric and medieval concepts of the yogic body, such as \textit{cakra}, \textit{nāḍī} and \textit{vāyu} systems. A common phenomenon in the new Kriyāyoga literature is scientific explanatory models that are used as a means of legitimisation. For example, certain \textit{nāḍī}s are located in schematic sketches of the brain\footnote{\citeauthor{kriyayoganityananda2013}, 2013: 215.}, or positive effects of Kriyāyoga practice are legitimised with evolutionary biology theories, such as the polyvagal theory\footnote{\citeauthor{kriyayogalowenstein}, 2021: 188.}

\citeauthor{govindan2010} (2010: 216-225) distinguishes a total of seven main categories of Kriyāyoga. The first category he mentions is \textit{Kriya Hatha Yoga}. According to him, this is the starting point for every student of Kriya Yoga. This includes eighteen basic relaxation postures (\textit{āsana}s), muscle blocks (\textit{bandha}s), certain gestures (\textit{mudrā}s) and the sun salutation (\textit{sūryanamaskāra}) defined by Babajī.

The second main category is what \citeauthor{govindan2010} calls \textit{Kriya Kundalini Pranayama}. According to him, this practice is the art and science of mastering the breath and is considered to be the most essential and effective tool in Babajī's Kriyāyoga. This is not only meant to awaken the \textit{kuṇḍaliṇī} but with regular practice, the student awakens all \textit{cakra}s and the associated levels of consciousness, which is supposed to ultimately lead to the breathless state of \textit{samādhi} and self-realisation.

The third main category is \textit{Kriya Dhyana Yoga}, which is intended to include meditation techniques that are not explained in detail but are supposed to awaken the mind's hidden faculties.

The fourth main category is \textit{Kriya Mantra Yoga}. This involves the recitation or murmuring (\textit{japa}) of mantras discovered by the Siddhas. The recitation of mantras must take place with faith, love and concentration.

\citeauthor{govindan2010} calls the fifth category \textit{Kriya Bhakti Yoga}, the yoga of love and devotion. In \citeauthor{govindan2010}'s words, this is the ``turbojet'' of self-realisation. This type of Kriyāyoga includes devotional love, chanting, ritual worship and pilgrimages to holy places.

Furthermore, \textit{Kriya Karma Yoga} is named as the sixth category. In this case he refers to \citetitle{kaushik1993} II.47 f. and thus defines this subtype as selfless service that is performed consciously. All actions are supposed to be performed without the expectation of receiving anything in return, free from anger, selfishness, greed and personal desires. Thus, the practitioner is meant to examine his motivation before every action and is always supposed to act without selfish motives.

The seventh and final category is \textit{Kriya Tantra Yoga}. According to this, the followers of Kriyāyoga, just like the Siddhas, lead a family life. This subtype of Kriyāyoga involves retaining the energy normally wasted during sexual activity and transporting it to the higher \textit{cakra}s. The partner is supposed to be loved as an embodiment of the divine.

A similar system is taught in \citeauthor{kriyayogalowenstein} (2021). This initially includes a total of twelve \textit{āsana}s and the five Tibetans, as well as typical \textit{prāṇāyāma} techniques, \textit{ujjāyi}, \textit{kapalabhāti}, various \textit{bandha} techniques such as \textit{uḍḍīyānabandha} or \textit{mahābandha}, various \textit{mūdrā} techniques such as \textit{mahāmudrā}, \textit{śāmbhavīmudrā}, \textit{yonimudrā}, or the so-called \textit{Kriya Breath}. \textit{Kriya Breath} is referred to as \textit{kevalakumbhaka}. In addition, classical gymnastic exercises are also added\footnote{\citeauthor{kriyayogalowenstein}, 2021: 118-124. Gymnastic exercises can also be found in \citeauthor{kriyayogasturgess2015}, 2015: 447-458.} In addition to the \textit{āsana}s of Haṭhayoga, \citeauthor{kriyayogalowenstein} also recommend \textit{Tai Chi}, \textit{Qigong}, physiotherapy or a personal trainer to stay fit. Now and then, a biblical quotation is used. For example, in the case of the \textit{Third Eye Gazing} practice, he quotes Matthew 6:22. Furthermore, \citeauthor{kriyayogalowenstein} emphasise the practice of \textit{Hong Sau} as an important element of the practice. For \citeauthor{kriyayoganityananda2013}, \textit{Hong Sau}, or in this case the indologically correct transliteration \textit{haṃsa}, is also referred to by him as \textit{Haṃsa Sādhanā},\footnote{The \textit{ajapājapa}, recitation of the non-recitation of the \textit{haṃsa} mantra.} ``the very foundation'' of Kriyāyoga.\\

As indicated at the beginning of this section, it is clear that the term Kriyāyoga has given rise to a kind of proliferation of different Yoga techniques from earlier Yoga traditions, which are integrated into innovative exercise systems and attempted to be historically legitimised in different ways. Depending on the lineage and the teacher, individual characteristics and different explanatory models exist.\footnote{In these books, one repeatedly comes across pseudo-scientific explanatory models and stumbles across parallels drawn here and there to other religions, such as Christianity and Buddhism, to emphasise the effectiveness and importance of certain practices and views. Particularly in the more recent publications, it can be seen that, depending on the author, typically individual expressions of the ideal type of postmodern spirituality and religiosity are expressed, which \citeauthor{bochinger2009} have labelled the ``spiritueller Wanderer'' (\citeauthor{bochinger2009} 2009: 33-49).}\\

One last exemplary publication is \citetitle{kriyayogaupanishad1993} (1993) by \citeauthor{kriyayogaupanishad1993}. This book offers translations of ten well-known \textit{Yoga Upaniṣads} and one \textit{Kriya Yoga Upanishad}. The translator claims that the name of the author of this Sanskrit Yoga Upaniṣad was lost in the course of history. His book has no bibliography, nor are the sources of the translations mentioned. Further searches for a verifiable source text of the \textit{Kriya Yoga Upanishad} remain unsuccessful. The \textit{Kriya Yoga Upanishad} is neither to be found in the known publications and translations of the \textit{Yoga Upaniṣads},\footnote{Cf. \emph{Yoga Upaniṣads} (1938).} nor in publications of previously unpublished Upaniṣads.\footnote{Cf. \citetitle{upanishads1938} (1938).}. Searching through various catalogues of Sanskrit manuscripts was also unsuccessful.\footnote{In \citetitle{kaivalyadamanuscripts2005} (2005: 50), two manuscripts with the title \textit{Kriyāyoga} (AGJ 665/1 and TSM 6716) are listed, which, unfortunately, I was unable to consult. Neither manuscript is dated. AGJ 665/1 is a Devanāgarī manuscript on paper, and TSM 6716 is a Telugu manuscript on palm leaf. The author of the latter is named Venkaṭayogin. I suspect these manuscripts are probably later works that were created in the 18th century at the earliest. For now, however, no definitive statement can be made on this. However, their consultation could shed further light on the historical development of Kriyāyoga.} It is also striking that the \textit{Kriya Yoga Upanishad} is not mentioned in any other publications on Kriyāyoga consulted. For the time being, therefore, the possibility must be considered that \citeauthor{kriyayogaupanishad1993} is not only the translator of the \textit{Kriya Yoga Upanishad} but also the secret author. Perhaps he wrote this supposedly ancient source text in order to legitimise his own Kriyāyoga doctrine.   

Goswami \citeauthor{kriyayogaupanishad1993} learnt Kriyāyoga from his teacher Shelly Trimmer, who, according to the official website of the \textit{Temple of Kriya Yoga}\footnote{\cite{goswamikriyananda}.} founded by \citeauthor{kriyayogaupanishad1993}, was a guru, yogi, kabbalist and direct disciple of Yogānanda. \citeauthor{kriyayogaupanishad1993} studied philosophy for four years at the University of Illinois and then embarked on a business career. Whether \citeauthor{kriyayogaupanishad1993} would have acquired the qualifications to translate a Sanskrit source text remains to be seen. Possibly, he was a gifted autodidact.

In the \textit{Kriya Yoga Upanishad}, the disciple Sanskriti asks the guru Dattatreya to teach him the doctrine of Kriyāyoga. The latter agrees and explains Kriyāyoga in a total of ten chapters. The framework is formed by the eight-limbed Yoga system presented in 1.5, similar to the eight limbs of the Pātañjala scheme. The first chapter (1.6-25) presents the \textit{Ten Spiritual Restraints}. Dattatreya explains the \textit{Ten Spiritual Observances} in the second chapter (2.1-16). Chapter three, \textit{The Nine Postures} (3.1-13), deals with nine \textit{āsana}s with six sitting postures, one standing posture and one complex posture. The fourth chapter (4.1-63) discusses what \citeauthor{kriyayogaupanishad1993} calls \textit{Mystical Anatomy}. Here, six \textit{cakra}s named after the planets (i.e. the \textit{mūlādhāracakra} is called the ``Saturn mass-energy converter \textit{cakra}''), fourteen primary \textit{nāḍī}s and \textit{Kriya Kundalini}, which covers the `divine creative channel' with its mouth, are taught. The fifth chapter (5.1-14) is entitled \textit{Inner Purification} and contains \textit{prāṇāyāma} techniques such as \textit{sūryabhedana} and \textit{candrabhedana}. Chapter six (6.1-39), entitled \textit{Breath Control}, instructs another breathing exercise in combination with meditation on the three \textit{akṣara}s that constitute the sacred syllable \textit{auṃ}. During the inhalation (\textit{pūraka}), the yogi is supposed to meditate on \textit{a}, during the breath retention on \textit{u} and during the exhalation on \textit{ṃ}. In addition, the breathing technique \textit{śītalī} (6.25) and a technique called \textit{yonimudrā} (6.33-34) are presented. Chapter seven (7.1-10) is about \textit{Withdrawal of the Senses}. The practitioner is instructed to let the breath move through the body in a specific order. The eighth chapter (8.1-9) is entitled \textit{Concentration}. Here, the yogin is meant to inhale and hold the breath at specific bodily locations (not the \textit{cakra}s), which are associated with the five elements and the syllables \textit{ya, ra, va, la} and \textit {ha}, as well as specific deities. The even shorter ninth chapter, \textit{Meditation} (9.1-6), basically only states that the practice of concentration leads to meditation after a while. The tenth chapter, \textit{Samadhi} (10.1-12), then describes the final state of Yoga, which is defined as the ``deep conscious trance in which the yogi experiences Absolute Wisdom''.

\subsubsection{Hypothesis on the transition from the late medieval models to the modern models of Kriyāyoga}

The \textit{Yogasvarodaya} and Rāmacandra's \textit{Yogatattvabindu} were written before 1659 CE. Nārāyaṇatīrtha must have lived between 1600 and 1690 CE., and because of that, his \textit{Yogasiddhāntacandrikā} was also written in this timeframe. Sant Sundardās, the author of the \textit{Sarvāṅgayogapradīpikā} lived from 1596 to 1689. Interestingly, Nārāyaṇatīrtha and Sundardās lived in Benares.\footnote{See \citeauthor{burger2014sarvangayogapradipika} (2014: 684) for dating and location of Sundardās and \citeauthor{penna2004} (2004: 24) for dating and location of Nārāyaṇatīrtha.} Thus, we can safely assume that the complex taxonomies of twelve-fifteen Yogas were part of the local discourse of 17th-century Benares. One might speculate that Rāmacandra might also have lived in these surroundings, but this remains uncertain. Lahiḍi Mahāśaya, the person to whom the new forms of Kriyāyoga seem to go back, lived more than a century later, from 1828 to 1895 CE. Interestingly, Lahiḍi Mahāśaya is also said to have spent much of his life in Benares. It is, of course, utterly unclear whether Lahiḍi Mahāśaya ever read any of the works mentioned above. At least we know that he not only enjoyed an education in philosophy in Benares but also learnt English and Sanskrit.\footnote{\citeauthor{jones2008encyclopedia}, 2008 , pp. 255-256.} However, it is likely that the local discourse regarding the religious-spiritual offerings within Benares did not change abruptly. Lahiḍi Mahāśaya also lived as a family man and householder,\footnote{See \citeauthor{autobioyogi}, 1946: ???.} no sectarian affiliations are known so that the whole variety of religious-spiritual offerings of his time were open to him. He was able to combine them freely. As can be seen from the Yoga texts examined in this book, there was no lack of different Yoga categories in Benares between the 17th and 19th centuries CE. Although these were still labelled differently, they were without a doubt freely combined in practice. Moreover, given the plethora of Yoga practices from different Yoga traditions and Yoga texts presented in the previous chapter and evident in the publications of the new Kriyāyoga consulted, it is not only credible but also plausible that this phenomenon already began with Lahiḍi Mahāśaya, as Yogānanda claims in his autobiography. However, why Lahiḍi Mahāśaya chose the category of Kriyāyoga as the generic term for his Yoga system cannot be answered conclusively. However, I would like to offer an educated guess.

I hypothesize that the term Kriyāyoga, as the generic term for his system of Yoga, was a strategic decision of Lahiḍi Mahāśaya. It is unlikely, and there is no clear evidence that Lahiḍi Mahāśaya knew the \textit{Yogasvarodaya}, \textit{Yogatattvabindu} and \textit{Yogasiddhāntacandrikā}. It is impossible to determine if there ever was any influence of these texts on Lahiḍi Mahāśaya and his new Kriyāyoga system. But if there was, only the fact that all three texts that mention Kriyāyoga as the very first item in their taxonomies could have influenced his decision to unite all possible Yogas and their techniques under the term Kriyāyoga. Another factor could have been that he was consciously or unconsciously driven by the emerging Yogasūtra hype in the West, which triggered a wave of enthusiasm in India. One wonders why he did not choose the term Rājayoga to integrate many systems as others have done before him. Maybe because the term Rājayoga was already used as a generic term for Pātañjalayoga by then.\footnote{See \citeauthor{birch2014} (2014).} Perhaps, the term Kriyāyoga had the advantage that it not only formed a link to the popular and hyped \textit{Yogasūtra}, but also provided a basic framework that was open to interpretation due to the three constitutional practices \textit{tapas}, \textit{svādhyāya} and \textit{īśvarapraṇidhāna}. Thus, the term opened up the possibility to integrate the variety of post-Pātañjalean physical and non-physical Yoga practices from the Tantras and texts of Haṭha- and Rājayoga through a literal interpretation of the compound prefix \textit{kriyā°} in the sense of ``action''. This was likely a crucial aspect. As \citeauthor{birch2020} (2020: 471-472) demonstrated in his groundbreaking article ``\textit{Haṭhayoga’s Floruit on the Eve of Colonialism}'', the popularity of medieval Haṭhayoga practices reached an unprecedented peak across India during this period. What could be more logical than reducing the complex diversity of circulating Yogas to a simple, practice-oriented umbrella term? This apparently aligned with the \textit{Zeitgeist}. The formation of a new Hindu identity, which began in the 16th century, also culminated during the lifetime of Lahiḍi Mahāśaya. Therefore, it is not surprising that in creating his Kriyāyoga, he operated in line with the ``\textit{identidikatorischer Habitus}'' that Axel \citeauthor{michaels1998} (1998: 19-27) described as a characteristic mode of thinking in Hindu religion. Whether his thoughts consciously or unconsciously went in a similar direction must of cource remain open. However, we must assume that the discursive environment of Benares at his time certainly played its part in encouraging Lahiḍi Mahāśaya to integrate the various Yogas and basically all Yoga practices circulating in the local discourse of his time under this specific term.

\section{2. Jñānayoga}
\label{jnanayogaintro}

Jñānāyoga\footnote{see section \uproman{21} and \uproman{22} on p.\pageref{jnanayogastart}-\pageref{endsvabhava}} is the second method of Rājayoga in Rāmacandra's list of the fifteen yogas as well as in his source text, the \textit{Yogasvarodaya}. In Nārāyaṇatīrtha's list of the fifteen Yogas presented within the \textit{Yogasiddhāntacandrikā}, Jñānayoga takes sixth place. In the \textit{Sarvāṅgayogapradīpikā} Sundardās presents Jñānayoga as a form of Sāṃkhyayoga. It is the second among the four types of Sāṅkhyayoga together with Brahmayoga and Advaitayoga.  

\subsection{Jñānayoga in the \textit{Yogatattvabindu}}
\label{Jnanayogaintro2}
Jñānayoga occupies the second place in Rāmacandra's taxonomy of the fifteen Yogas but is not described according to this order in his text.\footnote{The description of Jñānayoga is preceded by Siddhakuṇḍalinīyoga and Mantrayoga (\uproman{3}-\uproman{12}), Lakṣyayoga (\uproman{13}-\uproman{15}), Rājayoga (\uproman{16}-\uproman{17}), Caryāyoga (\uproman{18}) and Haṭhayoga (\uproman{19}-\uproman{20}).} The description is given from section \uproman{21}-\uproman{22}. The overarching goal of Rāmacandra's Jñānayoga is the long-term durability of the body (\textit{bahutarakālaṃ śarīrasthitiḥ}) already mentioned in the introduction (section \uproman{1}), which is expressed here once again with other words: ``From the execution of this [Jñānayoga], time does not bring about the destruction of the body.'' (\textit{tasya kāraṇāt kālaḥ śarīranāśaṃ na karoti}). Simultaneously, Rāmacandra's Jñānayoga leads to the attainment of the ``reality of Śambhu'' (\textit{śāṃbhavīsattā}).\footnote{This refers to the highest reality and the state of Rājayoga. See p.\pageref{jnanayogatrans1} in the edition for a discussion of the term.} This Jñānayoga can be practised in two ways. The first method (\uproman{21}.1) arises through the application of ``non-dualistic thinking'' (\textit{avikalpatayā yuktyā}), and the second method (\uproman{21}.2) arises ``through the realisation that the entire world consists of all knowledge'' (\ldots \textit{sarvajñānamayaṃ jagat} | \textit{ya evaṃ vetti bodhena} \ldots). However, the text primarily deals with the first method. This method consists of viewing the world as a unity that is enlightened by the highest self (\textit{viśvātman}). If one perceives this unity, one finds oneself in the ``reality of Śaṃbhu''. However, this supreme reality cannot be recognised without further ado since it does not show itself as the desired unity but as a tenfold multiplicity (\uproman{21}.4ab). He compares this relationship to a seed from which a whole tree with its parts grows (\uproman{21}.4-\uproman{21}.5). The seed stands for the invisible unity of world and self. The tree, with its various parts, stands for the multiplicity of the visible world. The fundamental unity of the world is like the seed from which a whole tree has grown. It is no longer visible and is not perceived. However, what is perceived is a world consisting of a multiplicity. In the case of the seed, a tree with its branches, leaves, etc. In the case of the world ten basic principles (\textit{tattva}s): Five [gross] elements (\textit{pañcatattva}), thinking mind (\textit{manas}), intellect (\textit{buddhi}), illusion (\textit{māya}), individuation (\textit{ahaṃkāra}), and modifications (\textit{vikriyā}).\footnote{For a discussion of the tenfold \textit{tattva} system, see n. \cref{tentattvas} on p. \Cpageref{tentattvas}} Jñānayoga is supposed to produce the realisation of oneness (\uproman{21}.7). In order to realise this, the practitioner is supposed to apply the view of unity (\textit{aikyena darśanam}) to recognise the identity between the visible world of multiplicity\footnote{This is also referred to by Rāmacandra as \textit{saṃsāra} (\uproman{21} ll. 7-9).}, and the invisible self (\textit{viśvātma}). Through Jñānayoga, the practitioner then realises that the self is one with the world\footnote{Cf. \textit{Yogatattvabindu} \uproman{22} \pageref{svabhava1} l. 5: `Because of the power of Jñānayoga, there arises the conviction that the self is truly one (\textit{jñānayogaprabhāvād eka eva ātmā iti niścayo bhavati})} and the changing forms of the worlds material appearance are empty.\footnote{Cf. \textit{Yogatattvabindu} \uproman{22} p.\pageref{svabhava2} l.3: `Through Jñānayoga he realises the emptiness of the mutability of form.' (\textit{jñānayogād vikārarūparahito jñāyate} |)}

\subsection{Jñānayoga in the \textit{Yogasvarodaya}}
\label{svarodayajnana}
If we assume a correct transmission of the \textit{Yogasvarodaya} in the \textit{Prāṇatoṣiṇī}, then the text, in fact, describes two different types of Jñānayoga. 

The Jñānayoga of the first passage\footnote{Cf. \textit{Prāṇatoṣiṇī}, Ed. p. 831-833.} contains a description of the major components of the yogic body which the yogin is supposed to know. Gaining knowledge about the body is the aim of this Jñānayoga.\footnote{Cf. \textit{Prāṇatoṣiṇī} Ed. p. 831 (\textit{jñānayogam pravakṣyāmi tajjñānī śivatāṃ vrajet} | \textit{paṭhanāt smaraṇād vyānān maṇḍanāt brahmasādhakaḥ}) | \textit{tadbhedasyaikasandhānam aṣṭaiśvaryamayo bhavet} | \textit{tritīrthaṃ yatra nāḍī ca tripuṇyaṃ parameśvari} | \textit{svadehe yo na jānāti sa yogī nāmadhārakaḥ} | \textit{navacakraṃ kalādhāraṃ trilakṣaṃ vyomapañcakam} | \textit{svadehe yo na jānāti sa yogī nāmadhārakaḥ}).} In particular, the knowledge of the three primary channels (\textit{nāḍī}s)\footnote{The left lunar channel (\textit{iḍā}), the right solar channel (\textit{piṅgalā}) and the central channel (\textit{suṣūmnā}).}, as well as a system with a total of nine \textit{cakra}s is mandatory. These elements are described in detail. The introduction to this first form of Jñānayoga mentions other things the yogin should know, such as the three targets [for fixing the mind] (\textit{lakṣya}s),\footnote{In the sections on Lakṣyayoga in the \textit{Yogasvarodaya} and \textit{Yogatattvabindu} five targets (\textit{lakṣya}s) are described in total. This is one of many inconsistencies in the \textit{Yogasvarodaya} and the \textit{Yogattvabindu}.} sixteen containers [for holding mind and often breath in the context of this type of yogic practice] (\textit{ādhāra}s) and the five [meditative] spaces (\textit{vyoman}s) through which the yogin progresses on the path to the highest state of Yoga. 

This first form of Jñānayoga in the \textit{Yogasvarodaya}, like much of its content and its overall structure, is adopted by Rāmacandra in his \textit{Yogatattvabindu}. Surprisingly, he presents the first form of Jñānayoga under a different name for unknown reasons.\footnote{Perhaps, the designation \textit{jñānayoga} in this context is a result of textual corruption, as the second Jñānayoga presented later on in the text lives up to its name much better. However, without further textual evidence, this remains unproven.} Instead of Jñānayoga, Rāmacandra calls it Siddhakuṇḍaliniyoga and Mantrayoga. It is unclear why Rāmacandra made this change. Perhaps Rāmacandra did not want to teach two different forms of Jñānayoga, or he was convinced that Siddhakuṇḍaliniyoga and Mantrayoga were the more appropriate terms for this type of Yoga. There is also the possibility that Rāmacandra knew Nārāyaṇatīrtha's \emph{Yogasiddhāntacandrikā}, because he classifies Jñānayoga as a form of Mantrayoga, as will be shown in the next subsection. However, apart from similarities between the complex Yoga taxonomies, there are no other noticeable overlaps or even citations. A detailed discussion of Siddhakuṇḍalinīyoga and Mantrayoga in Rāmacandra's \textit{Yogatattvabindu} can be found on p.\pageref{siddhayogaintro}.

The second type of Jñānayoga of the \emph{Yogasvarodaya}\footnote{\textit{Prāṇatoṣiṇī}, Ed. p. 835-837.} is largely identical with Rāmacandra's Jñānayoga. Rāmacandra borrows most of the verses verbatim from the \textit{Yogasvarodaya}. There are minor details that Rāmcandra modifies, but they do not change the overall concept and aim of this type Jñānayoga. For this reason, it will not be repeated here. The passage is reproduced in its entirety in the first layer of the critical apparatus in section \uproman{21} on p. \pageref{jnanayogastart} of the critical edition of the \textit{Yogatattvabindu} and can be consulted there.   

\subsection{Jñānayoga in the \textit{Yogasiddhāntacandrikā}}
\label{jnanayogaintrocandrika}
Nārāyaṇatīrtha situates his Jñānayoga \footnote{For an earlier brief discussion of Jñānayoga in Nārāyaṇatīrtha's \textit{yogacandrika} see \citeauthor{penna2004}, 2004: 76.} in the context of \citetitle{yogasutra}'s \textit{sūtra} 1.28, which says:
\begin{quote} \textit{taj japas tadarthabhāvanam} || 28 || \end{quote}
\begin{quote} It's low-voice muttering; contemplation of its meaning. \end{quote}

This is the last \textit{sūtra} of an extensive section (1.23 - 1.28) in the \citetitle{yogasutra}\footnote{An entire monograph entitled \citetitle{harimoto2014} is dedicated to this section by \citeauthor{harimoto2014} (2014). It provides an edition, translation and detailed discussion of this critical passage in the \textit{Pātañjalayogaśāstravivaraṇa}.}, which is entirely dedicated to one of the means of attaining \textit{samādhi}, namely \textit{īśvarapraṇidhāna}, devotion to Īśvara, the Supreme Lord.

Īśvara is most aptly represented by the sacred syllable \textit{oṃ}. The above \textit{sūtra} instructs the quiet murmuring of this syllable while contemplating its meaning (\textit{tadarthabhāvanam}) as a practical method of \textit{īśvarapraṇidhāna} to attain the highest state of Yoga, which is called Rājayoga or \textit{asaṃprajñātasamādhi}.

In this context, Nārāyaṇatīrtha explains that in this \textit{sūtra}, the term \textit{japa} (``low-voice muttering'') refers to the practice of Mantrayoga. The term \textit{arthabhavana} (``contemplating its meaning'') refers to Jñānayoga as a form of practice that cultivates discriminating knowledge (see previous paragraph). Furthermore, Nārāyaṇatīrtha refers to Advaitayoga, also associated with this \textit{sūtra}, which is a form of Yoga characterised by the view of the non-differentiation of the individual self and the supreme self. The \textit{Yogasiddhāntacandrikā} (Ed. p. 46) reads:
\begin{quote}
  \textit{kiñca japa ity anena mantrayogaḥ arthabhāvanam ity anena vivekajñānā 'bhyāsarūpo jñānayogaḥ abhedabhāvarūpo 'dvaitayogaś ca saṃgṛhītaḥ} |
\end{quote}
\begin{quote}
Furthermore, by the term \textit{japa}, the practice of Mantrayoga is indicated; by \textit{arthabhavana}, the knowledge of discrimination, the form of practice [called] Jñānayoga, and Advaitayoga is the form of cultivating non-differentiation.
\end{quote}

Nārāyaṇatīrtha, thus, offers two alternatives about the specific performance of the contemplation. Either, while quietly murmuring the \textit{praṇava} syllable, which symbolises Īśvara and his qualities, the mind shall be focused on the distinction between consciousness (\textit{puruṣa}) and primordial nature (\textit{prakṛti}) including its effects (\textit{tatkārya}).\footnote{Cf. \textit{Yogasiddhāntacandrikā} (Ed. p. 45): \textit{tasya praṇavasya japaḥ vidhivad uccāraṇaṃ, tadarthasya praṇavārthasya acintyaiśvaryaśaktiyuktasya paramātmano bhāvanaṃ prakṛtitatkāryapuruṣebhyo vivekenānusaṃdhānam} \ldots ``The low-voice muttering of \textit{praṇava} [and] pronunciation according to the rules [along with] the contemplation of the meaning of that \textit{praṇava}, [being associated with] the Supreme Self endowed with inconceivable power and supremacy, is the fixation of the attention with discernment from the individual self and nature with its effects.''} This is Nārāyaṇatīrtha's Jñānayoga. Alternatively, one is supposed to reflect on the non-difference between the highest self (\textit{paramātman}) and the individual self (\textit{jīva}).\footnote{Ibid. (Ed. p. 45): \textit{athavā tadarthasya paramātmanaḥ pūrṇasya bhāvanaṃ jīvābhedena punaḥ punaś cetasi niveśanam} | ``Alternatively, its meaning is the repeated memorization in the mind of the non-distinction between the individual self and the total supreme self.''} This is Nārāyaṇatīrtha's Advaitayoga.

\subsection{Jñānayoga in the \textit{Sarvāṅgayogapradīpikā}}

The Jñānayoga of Dādūpanthı̄ Sundardās (SYP 4.13-24) is very similar to the Jñānayoga of Rāmacandras \textit{Yogatattvabindu} and the \textit{Yogasvarodaya}. Jñānayoga is the first subcategory of Sāṃkhyayoga.\footnote{Sundardās Sāṃkhyayoga is discussed on p.\pageref{samkhyayoga}.} Brahmayoga\footnote{Sundardās Brahmayoga is discussed on p.\pageref{sundarbrahma}.} and Advaitayoga\footnote{Sundardās Advaitayoga is discussed on p.\pageref{sundaradvaita}.} follow it. While Sundardās introduces Sāṃkhyayoga to teach how to distinguish the self (\textit{ātman}) from the not-self (\textit{anātman}) by differentiating twenty-four \textit{tattva}s of the world, Jñānayoga goes one step further and conveys the gnosis (\textit{jñāna}) that the world and the self nevertheless form an inseparable unity. As a result of this gnosis, Brahmayoga arises. Brahmayoga is a specific form of contemplation or state in which the yogin experiences himself as one with the Absolute and the entire universe within himself. Finally, this sequence culminates in Advaitayoga, by which the practitioner finally overcomes the state of duality and conceptualisation. Jñānayoga is the second step of the four-stage Sāṃkhyayoga.

This Jñānayoga emphasizes the recognition of the unity of the self and the universe.\footnote{See \citeauthor{burger2014sarvangayogapradipika} (2014: 702) for an earlier brief discussion of Sundardās's Jñānayoga in French.} According to Sundardās, the self is the cause, and the whole universe is the effect.\footnote{\citetitle{sarvangayoga} 4.13: \textit{jñāna yoga aba esaiṃ jānaiṃ} | \textit{kāraṇa aru kāraya pahicānaiṃ} | \textit{kāraṇa ātama āhi akhāṃḍā} | \textit{kāraya bhayau sakala brahmaṇḍā} || 13 || ``Now understand Jñānayoga. Recognize the cause and effect. The cause is the indivisible soul. The effect is the whole universe.''} To illustrate the relationship of cause and effect between self and universe, Sundardās presents the same metaphor of the seed and the tree as Rāmacandra in \uproman{21}.4-5.\footnote{\citetitle{sarvangayoga} 4.14: \textit{jyauṃ aṃkuru teṃ taru vistārā} | \textit{bahuta bhāṃti kari nikasī ḍārā} | \textit{śāṣā patra aura pharaphulā} | \textit{yauṃ ātamā viśva kau mūlā} || 14 || ``Just as the tree [grows] out of the seed, bringing forth countless branches, leaves, fruits and flowers, in the same way the self is the root of the universe.''} The rest of the section consists of different comparisons, which are supposed to illustrate the non-difference between the self and the whole or the universe.\footnote{For example \citetitle{sarvangayoga} 4.20: \textit{jyauṃ kuñcana ke bhūṣana nānā} | \textit{bhinna bhinna kari nāṃva baṣaṇā} | \textit{gāre sarba eka hi huvā} | \textit{yaiṃ ātamā biśva nahiṃ juvā} || 20 || ``Just like various ornaments made of gold, are worn with different names and forms. However, in essence, all become one in the melting pot. In the same way, the self is not separate from the universe.''} 

\subsection{Jñānayoga in the complex early modern Yoga taxonomies}

The comparative analysis of Jñānayoga within the intricate and multifaceted texts of the early modern Yoga taxonomies unveils four distinct models.  

The most pervasive model is the application of non-dualistic thinking, a profound concept that allows one to perceive the unity of the self and the world. This model, with a few nuanced variations, can be found in the \textit{Yogatattvabindu}, the \textit{Yogasvarodaya}, and the \textit{Sarvāṅgayogapradīpikā}. The most notable difference is that the former two texts classify Jñānayoga as a method of Rājayoga, whereas the \textit{Sarvāṅgayogapradīpikā} categorises Jñānayoga as a subtype of Sāṃkhyayoga. 
The model of Jñānayoga presented by Nārāyaṇatīrtha in his \textit{Yogasiddhāntacandrikā} is in stark contrast to the dominant model. Jñānayoga here is a form of Mantrayoga. During \textit{praṇavajapa}, the yogin should contemplate the distinction between consciousness or self (\textit{puruṣa}) and the primordial nature (\textit{prakṛti} and its effects (\textit{tatkārya}).    

The \textit{Yogatattvabindu} suggests an alternative model, which is not described further and involves contemplation aimed at realising that the world consists of all knowledge. The \textit{Yogasvarodaya} describes a further type of Jñānayoga. This consists of acquiring knowledge about the yogic body and the yogic paradigms (\textit{lakṣya}s, \textit{cakra}s and \textit{vyoma}s). Both methods are also subspecies of Rājayoga.

%Notes:

%Chapter 15 - Trikāṇḍa-Yoga: Bhakti Surpasses
%Knowledge and Detachment
%(1) Śrī Uddhava said: 'The Vedic literature of Your Lordship, oh Lotus-eyed One,
%that pays attention to the injunctions concerning actions and prohibitions, deals with
%the good and bad sides of karma [akarma and vikarma]. (2) They also discuss the dif-
%ferences within the varṇāśrama system wherein the father may be of a higher [anulo-
%ma] or a lower [pratiloma] class than the mother, they are about heaven and hell and
%expound on the subjects of having possessions, one's age, place and time [see also 4.8:
%54 and *]. (3) How can human beings without Your prohibitive and regulatory words
%concerning final beatitude, tell the difference between virtue and vice [compare 11.19:
%40-45]? (4) The Vedic knowledge emanating from You offers the forefathers, the gods as
%72Uddhava Gītā
%also the human beings a superior eye upon the - not for everyone that evident - meaning
%of life, what would be the goal, and how we may achieve. (5) The difference between
%virtue and vice one can see with the help of Your Vedic knowledge and that insight does
%not arise by itself, but the Vedas also nullify such a difference and thus clearly confuse
%the issue....'
%(6) The Supreme Lord said: 'The three ways of yoga I described in My desire to
%grant human beings the perfection, are the path of philosophy [jñāna], the path of work
%[karma] and the path of devotion [bhakti]; no other means can be found [for one's
%emancipation. See also B.G. contents and trikāṇḍa].

\section{3. Caryāyoga}
\label{caryayogaintro}

Caryāyoga occupies third place in Rāmcandra's list of the methods of Rājayoga. However, it is absent in the \textit{Yogasvarodaya}, mentioned as the second method in Nārāyaṇatīrtha's fifteen Yogas. It is absent in Sundardās \textit{Sarvāṅgayogapradīpikā}. However, Sundardās describes a Yoga with the almost homophonic name Carcāyoga. Carcāyoga is considered the fourth and final method of Bhaktiyoga after Mantrayoga and Layayoga. 

\subsection{Caryāyoga in the \textit{Yogatattvabindu}}

Rāmacandra keeps the section on Caryāyoga (section \uproman{18}) extremely short, with only eight prose sentences. After characterising the self as 'formless, permanent, immovable and indivisible', Rāmacandra lets the reader know that by stabilising the mind in such a self, the self does not come into contact with sin and merit. When the mind is absorbed into the formless [self], this is Cāryayoga. That is all that Rāmacandra has to say on this subject. The brevity of the passage and the fact that the testimony of the \textit{Yogasvarodaya} does not contain this type of Yoga, but Rāmacandra clearly constructs its description on the basis of a passage on Rājayoga of the \textit{Yogasvarodaya},\footnote{Cf. \textit{Yogatattvabindu} \uproman{18}, p. \pageref{caryayoga}} suggests that Rāmacandra did not understand Caryāyoga and merely wanted to do justice to his taxonomy mentioned at the beginning of his text.\footnote{One could argue that Rāmacandra may not have done so, since not all fifteen Yogas announced at the beginning are described in the course of his text anyway. I suspect that this may nevertheless have been his original intention but that Rāmacandra discarded this intention while writing his text, perhaps due to inconsistencies in his source text} It is puzzling why this particular Yoga with this particular description bears the name Caryāyoga. The apparent association of the first four Yogas in Rāmacandra's and \textit{Yogasvarodaya}'s list with the four \textit{pāda}s of the Śaiva Āgamas (\textit{kriyā}-, \textit{jñāna}-, \textit{caryā}- and \textit{yogapāda}) does not offer a convincing solution in this case, as \textit{caryā°} in this context has nothing to do with the original ritual discipline or day-to-day conduct of the śaivite practices, as would be the case in the \textit{caryāpada} of a Śaiva Āgamas. It seems, therefore, unlikely that any Yoga practitioners back then practised a Caryāyoga according to Rāmacandra's concept. 

\subsection{Caryāyoga in the \textit{Yogasvarodaya}}

The term Caryāyoga does not appear in the sources of the \textit{Yogasvarodaya}, namely the \textit{Prāṇatoṣinī} and \textit{Yogakarṇikā}. Thus, the term is absent from its Yoga taxonomy\footnote{\textit{Prāṇatoṣiṇī} ed. p. 831.} Although the verses postulate a total of fifteen Yogas, only eight are mentioned. Whether Caryāyoga is one of the seven unnamed ones is unclear. However, its presence in the taxonomies of the \textit{Yogatattvabindu}\footnote{\textit{Yogatattvabindu} I. ll. 1-4.} and the \textit{Yogasiddhāntacandrikā}\footnote{\textit{Yogasiddhāntacandrikā} Ed. p. 2.} would support this. For this reason, Caryāyoga was possibly a member of the \textit{Yogasvarodaya}'s fifteen-fold Yoga taxonomy. The original appearance and structure of the \textit{Yogasvarodaya} remains conjectural. While it almost appears that the entirety of the \textit{Yogasvarodaya} has been preserved in the \textit{Prāṇatoṣiṇī}, the \textit{Yogakarṇikā} includes several verses attributed to the \textit{Yogasvarodaya} not found in the \textit{Prāṇatoṣinī}.\footnote{It is striking that Rāmacandra's prosaisation is based almost exclusively on the verses of the \textit{Yogasvarodaya} quoted by the \textit{Prāṇatoṣiṇī}. Is it possible that this was the very recension that Rāmacandra used for his \emph{Yogatattvabindu}? Or, was he even the creator of this very recension found in the \textit{Prāṇatoṣiṇī}?} Hence, it is plausible that the text was more extensive and may have included a transmission of Caryāyoga.

The \textit{Yogakarṇikā} provides detailed descriptions of daily ritual conduct for the Yoga practitioner under the heading \textit{dinacaryā} (``daily [ritual] conduct'') in verses 1.23-61. It is notable that for a significant portion of the first chapter (1.1-168), the source(s) of the verses are not indicated, which is surprising given that the remainder of the first chapter and all other chapters of the text primarily consist of compilations of verses from other texts on typical yogic topics quoted with reference. Thus, throughout the \textit{Yogakarṇikā}, larger sections of the \textit{Yogasvarodaya} are repeatedly but not always quoted with reference. Is it possible that Nāth Aghorānanda, the compiler of the \textit{Yogakarṇikā}, also drew on verses from the \textit{Yogasvarodaya} here?

In the second part of the first chapter of the \textit{Yogakarṇikā} (verses 1.169-280), 37 verses (1.244-280) are quoted from the \textit{Yogasvarodaya} with reference, alongside at least four verses (1.210-213) of the \textit{Yogasvarodaya} without reference.\footnote{The verses lacking attribution were identified as originating from the \textit{Yogasvarodaya} due to their presence in the \textit{Prāṇatoṣiṇī}.}

The possibility of further verses from the \textit{Yogasvarodaya} within the first 168 verses of the \textit{Yogakarṇikā} cannot be definitively addressed without a close examination of manuscripts of the \textit{Yogasvarodaya} and \textit{Yogakarṇikā}. However, it remains one of the most plausible scenarios that the original Caryāyoga within the taxonomy of the fifteen Yogas of the \textit{Yogasvarodaya} resembles the content of the \textit{dinacaryā} section of the \textit{Yogakarṇikā}. This section delineates daily ritual ablutions, mantra recitation, visualisation, and meditation (1.23-36), as well as other ritual acts such as dressing, applying sectarian markings (\textit{tilaka}), including tying the hair into a knot (1.38), offerings, and the devotional performance of prostrations in front of one's own \textit{iṣṭadevatā} (1.39-61). As they are part of the daily Yoga practices, presenting them as a yogic discipline would seem natural.\footnote{As discussed in more detail on p. \pageref{sivayogaintro} the \textit{Śivayogapradīpikā} contains numerous similarities in content with the \textit{Yogatattvabindu}, the \textit{Yogasvarodaya} and the \textit{Siddhasiddhāntapaddhati}. With ten Yogas described in total, the \textit{Śivayogapradīpikā} even comes very close to the numbers of Yogas within the late medieval Yoga taxonomies. These parallels strongly suggest a close connection in terms of reception history. There may not be a direct connection, but all these texts likely drew on the same intertextual network when compiling their own texts. In his dissertation on the \textit{Śivayogapradīpikā}, \citeauthor{powell2023} (2023:115) presents excerpts from a translation of a Kannada commentary on the \textit{Śivayogapradīpikā} (\textit{ṭike}) by the commentator Basavārādhya. Basavārādhya precedes his commentary with the following praise of the author of the \textit{Śivayogapradīpikā}: \begin{quote} ``The \textit{ācārya} called Cennasadāśivayoginsadāśivayogīśvara, who was skilled in the \textit{jñāna}, \textit{kriyā}, \textit{caryā} and \textit{yoga} [\textit{pāda}s] of the Śivāgamas, which are the means of personal liberation, who had the intellect capable of grasping the Veda and Vedānta, who was not caught up in the confusion of the many Śāstras such as the Sāṅkhya and Pātañjala, who was accomplished in the eternal true yoga, who could visualise the many worlds such as \textit{bindu} and \textit{nāda} in the middle of his body (\textit{piṇḍa}), who was an expert in \textit{mantra}, whose mind was absorbed in \textit{laya}, who was devoted to \textit{haṭha}, who was worthy of worship in Rājayoga, who was an expert practitioner and who was knowledgeable in many branches of learning such as Tāraka and the teachings on Brahman (\textit{brahmopadeśa}), engaging in creating the Yogaśāstra called the \textit{Śivayogapradīpikā} in order to illuminate the inner soul of those desirous of liberation.'' \end{quote} This eulogy not only suggests the great variety of different Yoga teachings of the \textit{Śivayogapradīpikā}, it also confirms that authors like Cennasadāśivayogin were familiar with the Śaiva Āgamas in this intertextual network, which also influenced the \textit{Yogatattvabindu} and \textit{Yogasvarodaya}. On the one hand, this confirms my assumption that the first three Yogas in the taxonomy of the fifteen must have been derived from \textit{pāda}s of the Śaiva Āgamas, and on the other hand, that the original Caryāyoga was most likely a name for a Yoga that included day-to-day ritual conduct.}

\subsection{Caryāyoga in the \textit{Yogasiddhāntacandrikā}}

In his \citetitle{yogacandrika}\footnote{\citetitle{yogacandrika}, ed. pp. 2, 52-53, 100-101, 150.} Nārāyaṇatīrtha presents Caryāyoga\footnote{For an earlier brief discussion of Caryāyoga in Nārāyaṇatīrtha's \textit{yogacandrika} see \citeauthor{penna2004}, 2004: 66-67.} in the context of Yogasūtra 1.33 (\textit{Yogasiddhāntacandrikā}, Ed. p. 52):

\begin{quote}
  \textit{tasya cittasyāsūyādimalavato yogāsambhavāt tannirāsopāyaṃ caryāyogam āha}-\\
    \textit{maitrīkaruṇāmuditopekṣāṇāṃ sukhaduḥkhapuṇyāpuṇyaviṣayāṇāṃ bhāvanātaś cittaprasādanam} || 33 ||
  \end{quote}
\begin{quote}
  Due to impurities of the mind like jealousy, etc., preventing the attainment of Yoga, the method of removing them is Caryāyoga -\\
  Purity of the mind arises through the cultivation of friendliness, compassion, joy and equanimity in circumstances of happiness, suffering, virtue and vice. 
  \end{quote}
  
Caryāyoga is to cultivate kindness towards those in fortunate circumstances to prevent jealousy. Towards those who are in sorrowful circumstances, compassion is supposed to be cultivated to prevent ill-will. Towards those who act virtuously, one is supposed to cultivate joy to prevent aversion; and towards those who act unvirtuously, one is supposed to cultivate equanimity to prevent anger.\footnote{Cf. \textit{Yogasiddhāntacandrikā} (Ed. p. 52): \textit{tathā ca sukhiteṣu maitrīṃ sauhārdam īrṣyākāluṣyanivarttakaṃ, duḥkhiṣu karuṇāṃ dayāmasūyākāluṣyanivarttikāṃ, puṇyavṛttiṣu harṣaṃ dveṣanivarttakam, apuṇyaśabditapāpiṣu upekṣām amarṣakāluṣyanivarttikāṃ bhāvayet} |}        

 With this practice of Caryāyoga, which gradually purifies the mind, the sattvic nature of the mind is brought forth. This leads to a clear and serene mind.\footnote{Cf. \citetitle{yogacandrika} (Ed. pp. 52-53): \textit{tad evaṃ caryāyogena cittamalanirāsakena mukhyādiṣu yathākramamuktabhāvanārūpeṇa sāttviko dharmo jāyate} | \textit{tena ca śuklena dharmeṇa cittaṃ prasannaṃ bhavati} | \textit{prasāde ca sthitipadaṃ labhate} | \textit{etac ca puṣkalaṃ viraktasyaiva sambhavatīti mukhyacaryāyogo vairāgyameveti saṃkṣepaḥ} || 33 ||}

 Since the word \textit{caryā°} in this context refers to purposeful behaviour designed to give rise to the sattvic nature of the mind, the Caryāyoga of the \textit{Yogasiddhāntacandrikā} can be meaningfully translated as ``Yoga of [beneficial] behaviour''.  

\subsection{Carcāyoga in the \textit{Sarvāṅgayogapradīpikā}}
\label{carcasarvanga}
Within \citetitle{sarvangayoga} (2.40-51, Ed. pp. 96-98), Sundardās describes Cārcāyoga as one of the three subtypes of Bhaktiyoga which is \textit{bhakti} towards unmanifest consciousness (\textit{avyakta puruṣa}) in delightful devotion.\footnote{See \citeauthor{burger2014sarvangayogapradipika} (2014: 694-695) for an earlier brief discussion of Sundardās's Carcāyoga in French} He extensively describes the unmanifest consciousness (\textit{avyakta puruṣa}) as being formless and eternal and so on (40), as beginningless and endless, and so on (41). Next, Sundardās describes the various layers of creation emanating from \textit{oṃ} (42-45). He says the unmanifest consciousness illuminates every corner of existence (46), being the inner knower of all (47). Then, Sundardās expresses the importance of deep awe towards the infinite, divine, all-knowing and incomprehensible (48-49) unmanifest consciousness.

The entire passage on Carcāyoga is characterised by a discussion and description of the unmanifest consciousness (\textit{avyakta puruṣa}). This aspect is the core of this type of Yoga. Unlimited unmanifested consciousness can be put into limiting words only, and yet the practitioner is confronted with the question of how it is supposed to be defined and determined.\footnote{Cf. \citetitle{sarvangayoga} 2.41ab: \textit{avyakta puruṣa agama apārā} | \textit{kaisaiṃ kai kariye nirddhārā} |} And this is precisely the practice of Carcāyoga. The term \textit{carcā°} here refers to ``discussing'' or ``putting into words'' and emphasising individual details of unmanifest consciousness to generate deep reverence for the cultivation of Bhaktiyoga, the Yoga of devotional worship of \textit{avyakta puruṣa}. \textit{Sarvāṅgayogapradīpikā} 2.47 illustrates this:
\begin{quote}
  \textit{carcā karaiṃ kahāṃ laga svamī} | \textit{tum saba hī ke antarjāmī} | \\
  \textit{sṛṣṭi kahat kachu anta na āvai} | \textit{terā pāra kaiṃna dhaiṃ pāvai} || 47 ||
  \end{quote}
\begin{quote}
How to discuss, where to find you, O Lord? You are the inner knower of everything. There is no end to describing creation. Your limit cannot be reached by any means.
\end{quote}

Thus, it is clear that no direct conceptual connection exists between the Caryāyogas described above and Carcāyoga. A meaningful explanation for the conspicuous homophony of both terms cannot be offered for the time being.  

\subsection{Caryāyoga in the complex early modern Yoga taxonomies}

The comparative analysis of Caryāyoga within the intricate and multifaceted texts of early modern Yoga taxonomies reveals two distinct models. Additionally, the initial question regarding any connection between Caryāyogas and Carcāyoga was addressed, and a hypothesis was formulated on the original form of Caryāyoga.

In the \textit{Yogatattvabindu}, Caryāyoga is described as stabilizing the mind in the self. This rather banal description was likely an attempt to define Caryāyoga as mentioned in the initial list. It is plausible that Rāmacandra invented this description without any real understanding of Caryāyoga, as it seems to be derived from a description of Rājayoga in his source text. It appears highly unlikely that this form of Caryāyoga was ever practiced.

Caryāyoga is absent from the testimony of the \textit{Yogasvarodaya} and is not listed therein. However, the \textit{Yogakarṇikā}, which extensively quotes the \textit{Yogasvarodaya}, suggests that Caryāyoga was originally closely related to the \textit{caryāpāda}s of the Śaiva Āgamas, and thus consisted of daily ritual conduct as part of the yogic routine.

Furthermore, the comparison of Caryāyogas with Carcāyoga in Sundardās’s work showed that they are entirely unrelated. In this context, Carcāyoga represents the final method of Bhaktiyoga, which aims to articulate the unmanifest consciousness in order to generate the profound awe necessary for progress on the yogic path, as presented by Sundardās in his \textit{Sarvāṅgayogapradīpikā}.

\section{4. Haṭhayoga}
\label{hathayogaintro}

Haṭhayoga appears without exception in all complex late medieval yoga taxonomies. In the taxonomies with fifteen Yogas of the \textit{Yogatattvabindu}, the \textit{Yogasvarodaya} and the \textit{Yogasiddhāntacandrikā}, it occupies the fourth position. In the Yogataxonomy of Sundardā's \textit{Sarvāṅgayogapradīpikā}, it is the second main type of Yoga. Haṭhayoga is a category in itself and the superordinate category for the three subsequent Yogas described by Sundardās, namely Rāja-, Lakṣa- and Aṣṭāṅgayoga which are all considered to be methods of Haṭhayoga. 

\subsection{Haṭhayoga in the \textit{Yogatattvabindu} and \textit{Yogasvarodaya}}

Both texts consider Haṭhayoga as another method of Rājayoga. In section \uproman{19}-\uproman{20} of the \textit{Yogatattvabindu}, two categories of Haṭhayoga are distinguished. Both are based on the explanations of the \textit{Yogasvarodaya}, differ only slightly in formulation, and can, therefore, be considered together.\footnote{See \citetitle{ramatosana} (Ed. p. 835) and \citetitle{shabdakalpadruma} (Ed. p. 501). These passages contain quotations from the \textit{Yogasvarodaya} of both types of Haṭhayoga. See also \citetitle{yogakarnika} 12.23-26. Here, verses of the second category of Haṭhayoga are reproduced} Both passages in these two texts are characterized by their brevity. 

The first type of Haṭhayoga described teaches the control of the breath through exhalation (\textit{recaka}), inhalation (\textit{pūraka}) and breath retention (\textit{kumbhaka}) etc. With the term ``etc.'' (\textit{°ādi°}), the text probably refers to other known practices of \textit{Haṭhayoga}. In addition to other breathing exercises, this could also refer to the other known basic building blocks of Haṭhayoga, which have been associated with Haṭhayoga since Svātmarāma's \textit{Haṭhapradīpikā}: \textit{āsana}, \textit{mudrā} and \textit{nādānusandhāna}. At least \textit{āsana} is explicitly mentioned in the \textit{Yogasvarodaya}, but not in the \textit{Yogatattvabindu}.\footnote{Cf. \emph{Yogasvarodaya} (PT p. 835): \textit{kṛtvāsanaṃ pavanāśaṃ śarīre rogahārakam} |} Both texts mention the six actions that purify the body (\textit{ṣatkarma}) next. Then Rāmacandra states that when the full breath dwells within the solar channel (\textit{sūryanāḍi}), the mind becomes immobile. Through the immobility of the mind, bliss arises, and the mind is absorbed into emptiness (\textit{śūnya}). The resulting state leads to the delay of the time of death (\textit{kālaḥ samīpe nāgachati}). The naming of the sun channel is striking in this context. The \textit{Yogasvarodaya} is no concrete help here, as it merely speaks of an unspecified \textit{nāḍī},\footnote{Since the YSv mentions no specific \textit{nāḍī}, it is likely that it is the \textit{nāḍī} \textit{par excellance}, the \textit{suṣūmnā}} in which, triggered by the preceding practice, the fullness of breath is established.\footnote{Cf. \emph{Yogasvarodaya} (PT p. 835): \textit{etan nāḍyān tu deveśi vāyupūrṇaṃ pratiṣṭhitam} | \textit{tato mano niścalaṃ syāt tata ānanda eva hi} |} The majority of texts in the Haṭhayoga genre would certainly specify \textit{suṣūmnā}, the central channel, in the context of the ``immobility of the mind'', a central characteristic of the \textit{samādhi} state. They would not specify the right channel associated with the sun, called \textit{piṅgalā}. The occurrence of the Yoga state, or \textit{samādhi}, is generally associated with the entry of the breath into the central channel.\footnote{This is already evident, for example, in the oldest written testimony of the Haṭhyoga genre, the \textit{Amṛtasiddhi} 26.1-2: \textit{yo 'sau siddhimayo vāyur madhyamāpadaniścalaḥ} | \textit{tadānandamayaṃ cittam ekarūpaṃ nabhaḥsamam} || 26.1 || \textit{yadānandamayaṃ cittaṃ bāhyakleśāvivarjitam} | \textit{bhavaduḥkhāni saṃhṛtya samādhir jāyate tadā} || 26.2 || \citeauthor{asiddhi} translate: (1) ``When Breath is perfected and fixed in the place of the Goddess of the Centre, then consciousness has the nature of bliss, uniform like the sky.'' (2) ``When consciousness has the nature of bliss, free from external afflictions, then, having the sorrows of existence, Samādhi arises.'' This idea, which can be found in this genre from the 11th century at the latest, subsequently permeates the entire genre.} Either the term \textit{sūryanāḍi} is to be understood here as an unfortunate synonym,\footnote{In the sense of being ambiguous and overlapping with the \textit{piṅgalā} channel.} or the text is corrupt.\footnote{A conjecture of \textit{sūryanāḍī} to \textit{śūnyanāḍī} would be obvious. In \textit{Jyotsnā} 4.10, Brahmānanda understands ``the void'' (\textit{śūnya}) as the central channel. In \textit{Haṭhapradīpikā} 3.4, \textit{śūnyapādavī} is a synonym of \textit{suṣumnā}.} Another possibility would be to assume a practice associated with the \textit{piṅgalā} channel. The term \textit{sūryanāḍī} is found in the \textit{Siddhasiddhāntapaddhati}, a text that also served as a model for Rāmacandra.\footnote{Cf. \textit{Siddhasiddhāntapaddhati} 2.5: \textit{pañcamaṃ kaṇṭhacakraṃ caturaṅgulaṃ tatra vāme iḍā candranāḍī dakṣiṇe piṅgalā sūryanāḍī tanmadhye suṣumnāṃ dhyāyet saivānāhatakalā anāhatasiddhir bhavati} |}

The second type of Haṭhayoga in \textit{Yogatattvabindu} instructs the yogin to contemplate a non-specific form (\textit{kiṃcidrūpā}) in the colours white, yellow, blue and red equal to the radiance of ten million suns in one's own body from head to toe (\textit{cintyate}). This is supposed to burn away all diseases of the body and prolong life. In the \textit{Yogasvarodaya}, there is no mention of an unspecific form. Instead, these colours and the sun's radiance are meant to be contemplated in the area of the tip of the nose.\footnote{Cf. \emph{Yogasvarodaya} (PT p. 835): \textit{ākāśe nāsikāgre tu sūryakoṭisamaṃ smaret} | \textit{śvetaṃ raktaṃ tathā pītaṃ kṛṣṇam ity ādirūpataḥ} |} Rāmacandra and the \textit{Yogasvarodaya} describe the second type of Haṭhayoga so briefly and vaguely that the reader is denied a clearer picture. It should be noted at this point that the formulation is very reminiscent of Bāhyalakṣya's explanations in section \uproman{23}\footnote{Cf. p. \pageref{bahya}}. Interestingly, in Sundardā's \citetitle{sarvangayoga}, Lakṣ(y)ayoga is a subcategory, i.e. a partial practice, of Haṭhayoga. Is this hinting the source for this differentiation? Further parallels to practices of other texts of Haṭhayoga involving coloured or non-coloured light exist but are still conceptually too distant to convincingly assign Rāmacandra's second type,\footnote{see p.\pageref{bahyatrans} for the parallel passages} and thus remain enigmatic for the time being.

\subsection{Haṭhayoga in the \textit{Yogasiddhāntacandrikā}}

In the \textit{Yogasiddhāntacandrikā}, the discussion and description of Nārāyaṇatīrthas Haṭhayoga is spread over several \textit{sūtra}s of the first two chapters, the \textit{samādhipāda} (1.34) and the \textit{sādhanapāda} (2.46-52). The commentary by Nārāyaṇatīrtha is particularly extensive and detailed here.\footnote{For an earlier, short discussion of Haṭhyoga in Nārāyaṇatīrtha's \textit{yogacandrika} see \citeauthor{penna2004}, 2004: 76.}

Initially, Nārāyaṇatīrtha locates Haṭhayoga in the context of \textit{sūtra} 1.34. This \textit{sūtra} is one of several options (1.32-40) that can be applied to overcome the distractions described in \textit{sūtra}s 1.30-31, which hinder the attainment of the final state of yoga (\textit{asaṃprajnātasamādhi}, \textit{nirbījasamādhi}, or \textit{kaivalya}):\footnote{This final state of yoga is called \textit{rājayoga} by Nārāyaṇatīrtha.} 
\begin{quote} \textit{pracchardanavidhāraṇābhyāṃ vā prāṇasya} || 34 || \end{quote}
\begin{quote} Or, through exhaling and restraining of the breath. \end{quote}

This method thus serves to establish a clear mind. This is referred to by Nārāyaṇatīrtha as Haṭhayoga. In his commentary, Nārāyaṇatīrtha explains that the term \textit{pracchardana} means the slow outward emptying of the breath of the abdomen through one of the two nostrils in measured quantities.\footnote{\citetitle{yogacandrika} 1.34 (Ed. p. 53): \textit{kauṣṭhyasya vāyoḥ pracchardanam, ekataranāsāpuṭena mātrāpramāṇena śanaiḥ śanair bāhar niḥsāraṇam} |} The term \textit{vidhārana} is the external continuous breath-holding of exhaled air.\footnote{Ibid. 1.34 (Ed. p. 53): \textit{vidhāraṇaṃ recitasya vāyor bahir eva sthāpanaṃ kumbhakaṃ} |} Furthermore, Nārāyaṇatīrtha specifies this method of breath retention as \textit{recitakumbhaka}. It is the first of a total of seven breath retentions (\textit{saptakumbhaka}) and is considered particularly praiseworthy, as hardly any rules need to be observed for this type. However, this group of seven \textit{kumbhaka}s - \textit{recita}, \textit{pūrita}, \textit{śānta}, \textit{pratyāhāra}, \textit{uttara}, \textit{ādhāra}, and \textit{sama} - is specified later on in the second chapter, in the context of the fourth limb of \textit{aṣṭāṅgayoga}, known as \textit{prāṇāyāma} (2.49-53). The seven \textit{kumbhaka}s are discussed alongside seven out of the eight \textit{kumbhaka}s of the \textit{Haṭhapradīpikā}.\footnote{Ibid. 1.34 (Ed. p. 53): \textit{tathā cātra pūrakavarjanād recitapūritaśāntapratyāhārottarādhārasamabhedena saptakumbhakeṣu madhye recitakumbhako 'yaṃ prathamābhyāse 'nekaniyamānapekṣatayā praśastaḥ} | \textit{sarvam etad agre prāṇāyāmaprakaraṇe sphuṭī bhaviṣyati} |}

According to Nārāyaṇatīrtha, the mastery of the breath and the mastery of the mind are intrinsically linked. At the same time, \textit{prāṇāyāma} has the power to eradicate all sins, which enables the mind to concentrate and stabilize on a meditative focal point or goal (\textit{lakṣya}).\footnote{\citetitle{yogacandrika} 1.34 (Ed. p. 53): \textit{tad etābhyāṃ prāṇajaye cittajayas tayor avinābhāvāt prāṇāyāmasya sarvapāpanāśakatvāt pāpanivṛttyā ca cittam ekatra lakṣye sthiraṃ bhavati} |}

Finally, Nārāyaṇatīrtha authenticates the linking of \textit{prāṇāyāma} and Haṭhayoga (\textit{prāṇāyāmasya haṭhayogatvam uktaṃ smṛtau}) with the famous verse of \citetitle{yogabija} (148cd-149ab), in which the syllable ``\textit{ha}'' is linked to the sun and the syllable ``\textit{ṭha}'' to the moon. Thus, \textit{haṭha} is understood as the union of sun and moon. \footnote{Ibid. 1.34 (ed. p. 53): \textit{hakāreṇa tu sūryo 'sau ṭhakāreṇendur ucyate} | \textit{sūryācandramasor aikyaṃ haṭha ity abhidhīyate} || The context suggests here, that Nārāyaṇatīrtha associates the sun and moon with the \textit{piṅgalānāḍī} (representing the sun) and \textit{iḍānāḍī} (representing the moon). Their union would then be the inhalation through these channels with a subsequent breath retention.}\\

The next section of the \textit{Yogasiddhāntacandrikā}, which discusses aspects of Haṭhayoga, is only found in the context of the third limb of the \textit{aṣṭāṅgayoga}, which is described beginning with \textit{sūtra} 2.46.

\begin{quote} \textit{itaḥ paraṃ sakalarogādinivṛttidvārā haṭhayogasyopāyam āsanam āha- \\
sthirasukham āsanam} || 46 || \end{quote}
\begin{quote} From here on, postures, being the means of Haṭhayoga, are said to be the gateways to preventing all diseases etc. \\
A comfortable and steady position.
\end{quote}

Nārāyaṇatīrtha then presents various \textit{āsana}s. Of a total of 84 \textit{āsana}s, he describes 38 in detail. \citeauthor{birch2018proliferation} (2018) observed\footnote{Cf. \citeauthor{birch2018proliferation} 2018, p. 105, fn. 9. } that Nārāyaṇatīrtha's descriptions of the \textit{āsana}s were borrowed from earlier yoga texts, such as the \textit{Haṭhapradīpikā} (which Nārāyanatı̄rtha refers to as \textit{Yogapradīpa}), the \citetitle{vasisthasamhita2017} and the \textit{Dharmaputrikā}.\footnote{A list of the 38 of 84 \textit{āsana}s can be found in \citetitle{yogacandrika} 2.46 (Ed. p. 107-108): \textit{tac ca padma-siddha-bhadra-vīra-svastika-siṃha-daṇḍa-sopāśraya-paryaṅka-mayūra-kukkuṭa-uttānakukkuṭa-paścimatāna-matsyendrapīṭha-cakra-gomukha-karma-dhanu-mṛgasvastika-arddhacandra-añjalika-pīṭha-vajra-mukta-candra-arddhaprasāritaśava-kapāla-guruḍa-arddhāsana-kamala-krauñcaniṣadana-hastiniṣadana-uṣṭraniṣadanakapiniṣadana-yogāsana-yonyāsana-samasthāna-ādibhedena caturāśītiprakāram} | \textit{eteṣāṃ lakṣaṇāni yogapradīpādāv uktāni} | The detailed descriptions of the 38 \textit{āsana}s can be found immediately following on p. 108-114.}\footnote{\citeauthor{penna2004} (2004: 207-209) has briefly discussed the \textit{āsana}s of the \textit{Yogasiddhāntacandrikā}.}

In 2.47-48, Nārāyaṇatīrtha provides additional details on the execution of the Yoga postures, which will not be elaborated upon here.\footnote{A detailled sketch of the \textit{prāṇāyāma}-system of Nārāyaṇatīrtha's \textit{Yogasiddhāntacandrikā} can be found in \citeauthor{penna2004} (2004: 209-18).} Far more important for the determination of Nārāyaṇatīrtha's Haṭhayoga is 2.49-51. In addition to a detailed discussion of the three basic elements of \textit{prāṇāyāma} - exhalation (\textit{recaka}), inhalation (\textit{pūraka}) and breath retention (\textit{kumbhaka}) as well as their specifics in the commentary to 2.49-50, Nārāyaṇatīrtha then discusses \textit{kevalakumbhaka}, the fourth aspect of \textit{prāṇāyāma}, the overarching goal and ultimate result of breath retention.\footnote{Cf. \citetitle{yogacandrika} 1.34 (Ed. p. 116): \textit{asya ca lakṣaṇaṃ yājñavalkya āha- recakaṃ pūrakaṃ tyaktvā yat sukhaṃ vāyudhāraṇam} | \textit{prāṇāyāmo 'yam ity uktaḥ sa vai kevalakumbhakaḥ} || ``Yājñavalkya declares its characteristic as follows - Having abandoned inhalation and exhalation, that comfortable restraint of breath is breath-control. This indeed is indeed taught as `isolated retention'.''}\footnote{See \citetitle{hathapradipika2024} 2.72-80 for the \textit{locus classicus} of all descriptions of \textit{kevalakumbhaka}.} 

This \textit{kevalakumbhaka} is achieved in a lengthy process with gradually more subtle advances through the practice of ordinary \textit{kumbhaka}, which is specified as \textit{sahitakumbhaka}.\footnote{This \textit{kumbhaka} is "accompanied" (\textit{sahita}) because, unlike \textit{kevalakumbhaka}, it is still accompanied by inhalation and exhalation. Cf. \citetitle{hathapradipika2024} 2.73.} Only when the bodily channels have been purified through practice, and the movements of exhalation and inhalation have entirely ceased does \textit{kevalakumbhaka} arise. An appropriate translation is ``isolated breath retention'', as it is isolated from the inhalation and exhalation.\footnote{Cf. \citetitle{yogacandrika} 2.51: \textit{evambhūta ubhayoḥ śvāsapraśvāsayor gativicchedaś caturthaḥ prāṇāyāma ity arthaḥ} | \textit{etena sahitakumbhakābhyāsa evāsyā 'sādhāraṇam} | \textit{yadā nāḍīviśuddhiḥ syād yoginastattvadarśinaḥ} | \textit{tadā vidhvastadoṣasya bhavet kevalasambhavaḥ} ||}

The yogin who masters \textit{kevalakumbhaka} can hold the breath for an indefinite period.\footnote{Cf. \citetitle{hathapradipika2024} 2.76.} Nārāyaṇatīrtha then quotes seven of the eight \textit{kumbhaka}s\footnote{\citetitle{yogacandrika} 2.51, ed. p. 118-121. The seven \textit{kumbhaka}s mentioned by Nārāyaṇatīrtha are: 1. \textit{sūryabhedana}; 2. \textit{ujjāyī}; 3. \textit{sītkā(ra)}; 4. \textit{śītalī}; 5. \textit{brahmarī}; 6.\textit{mūrchā}; and 7. \textit{bhastrikā}.} of \citetitle{hathapradipika2024} (except \textit{plāvanī}, cf. \citetitle{hathapradipika2024} 2.71).\footnote{Cf. \citetitle{hathapradipika2024} 2.48-71.} Then the other seven \textit{kumbhaka}s already mentioned in the commentary to 1.54 are explained in more detail.\footnote{\textit{Yogasiddhāntacandrikā} 2.51, p. 121: \textit{kumbhaḥ saptavidho jñeyo recitādiprabhedataḥ} | \textit{recitaṃ pūratiḥ śāntaḥ pratyāhārottaro'dharaḥ} || \textit{samaśceti vinirdiṣṭaḥ kumbhakaḥ saptabhedataḥ} \textit{iti eteṣāṃ lakṣaṇāni cāha-} \textit{recitasya bahistambho vāyo recitakumbhakaḥ} \\textit{pūrakeṇa vinā samyag yogo 'yaṃ sukhado nṛṇām} || 1 || \textit{pūritasyodare rodhaḥ paścādrecakasaṃyutaḥ} | \textit{nāḍīśuddhikaraḥ samyak proktaḥ pūritakumbhakaḥ} || 2 || \textit{kāyasyāntarbahir vyāptir yā sa syāc chāntakumbhakaḥ} || 3 || \textit{sthānayorantare rodhaḥ pratyāhārākhyakumbhakaḥ} || 4 || \textit{āpūrayet kramādūrdhvam ūrdhvarodho hṛdādiṣu} || 5 || \textit{uttaraḥ kumbhakaḥ sa syādadho 'dho mūrddhato 'dharaḥ} || 6 || \textit{recanāpūraṇe tyaktvā manasā maruto dhṛtiḥ} | \textit{yā nābhyādpradeśeṣu samaḥ kumbhaḥ prakīrttitaḥ} || 7 ||} The commentary to 2.50 then quotes further explanations from various texts, such as \textit{Yogabhāskara}, \textit{Nandipurāṇa} and \textit{Mārkaṇḍeyapurāṇa} on the subject of \textit{prāṇāyāma}. In addition, the four stages (\textit{avasthā}) of yoga practice - \textit{ārambha}, \textit{ghāṭa}, \textit{paricaya} and \textit{niṣpatti} are introduced,\footnote{See \citetitle{asiddhi} \textit{viveka} 19,21,29 and 31 for the oldest account of the four stages. Also cf. \citetitle{hathapradipika2024} 4.16-25.} etc.\footnote{For example, the yogic dietary guidelines and the dwelling of the yogin based on the explanations of the first chapter of \citetitle{hathapradipika2024}.}

The Haṭhayoga of Nārāyaṇatīrtha thus consists primarily of two of the four main classical categories of Haṭhayoga according to the \citetitle{hathapradipika2024}\footnote{Cf. \citetitle{hathapradipika2024} 1.56.} - \textit{āsana} and \textit{kumbhaka}, which are located in Pātañjalayoga. The third main category of Haṭhayoga after the \citetitle{hathapradipika2024}, namely \textit{mudrā}, is also found in the \textit{Yogasiddhāntacandrikā}. However, surprisingly, the \textit{mudrā}s, together with the \textit{ṣatkarma}s, are only taught in the context of Karmayoga. Surprisingly, because \textit{mudrā} and \textit{ṣaṭkarma} are the elements of Haṭhayoga that form the main distinguishing feature from other Yoga systems. Nārāyaṇatīrtha is not unaware of this. At the end of his section on Karmayoga, he mentions them belonging to Haṭhayoga, but nonetheless decides to present them in the context of Karmayoga. These will, therefore, only be dealt with in the corresponding sub-chapter of this work. The fourth main category of the \citetitle{hathapradipika2024}, \textit{nādānusandhāna}, is not found in the \textit{Yogasiddhāntacandrikā}. Concerning his concept of Haṭhayoga, Nārāyaṇatīrtha makes a significant point at the end of his commentary on \textit{sūtra} 2.28. There, he informs us that the results of Haṭhayoga are limited to bodily perfection. Therefore, they do not directly pertain to Rājayoga.\footnote{\emph{Yogasiddhāntacandrikā} (Ed. p. 98): \textit{etac ca sarvaṃ yogāṅgānuṣṭhānāditi sūtre sūtritamapi haṭhayogāṅgatvena deha siddhamātraphalatvena sākṣādrājayogā 'naṅgatvāt kaṇṭharaveṇa sūtrakṛtā noktam iti mantavyam iti saṃkṣepaḥ} || 28 ||}

\subsection{Haṭhayoga in the \textit{Sarvāṅgayogapradīpikā}}

In the \textit{Sarvāṅgayogapradīpikā} (3.1-52), Haṭhayoga is both an individual category (3.1-12) and a superordinate category. In the following, Haṭhayoga is primarily discussed as the individual category. As a superordinate category, it subsumes three other Yogas, namely Rājayoga (3.13-24), Lakṣayoga (3.25-36) and Aṣṭāṅgayoga (3.37-52). These subcategories will be only briefly characterised in this chapter. They are then discussed in detail in the respective chapter according to the order of the list of the fifteen Yogas of the \textit{Yogatattvabindu}.\footnote{A French description of Haṭhayoga in the \textit{Sarvāṅgayogapradīpikā} can be found in \citeauthor{burger2014sarvangayogapradipika} 2014, pp. 701-709.}

Sundardās initially locates Haṭhayoga within the Āditnātha tradition and specifies the union of sun and moon as its definition.\footnote{\citetitle{sarvangayoga} 3.1: \textit{abahi hahūṃ haṭhayoga sunāī} | \textit{ādinātha ke bandaiṃ pāī} | \textit{ravi śaśi doū eka milāvai} | \textit{yāhī teṃ haṭhayoga kahāvai} || 1 ||}

This is followed by describing the ideal environment for Yoga practice, short practice instructions and dietary rules (3.2-8). These are very reminiscent of the explanations in the first chapter of the \textit{Haṭhapradipikā}.\footnote{See \citetitle{hathapradipika2024} 1.57-60.} The chapter concludes with the naming of the six actions (\textit{ṣaṭkarma}s). Due to the lack of details in his descriptions, it is hardly comprehensible to perform the practices without a teacher or other instructive texts. Sundardās could not have conceived his chapter on Haṭhayoga as an instruction manual. Instead, his primary aim must have been to characterise it and integrate Haṭhayoga into the overall context of his successive sequence of Yogas.  

The ideal environment for Haṭhayoga is in a well-governed country where justice prevails. Here, the yogin is supposed to build a hut (\textit{maṭhikā}) with a small door and no holes. The yogin shall smear the hut with cow dung for this purpose. A small well is dug into the ground next to the hut.\footnote{Ibid. 3.2-3ab: \textit{prathama sudharma deśa kahuṃ tākai} | \textit{bhalau rājya kachu deṣala na jākai} | \textit{tāhāṃ jāī kai maṭhikā karī} | \textit{alpa dvāra aru chidra su bharaī} || 2 || \textit{lipta karai cahūṃ ora sugandhā} | \textit{kūpa sahita maṭha ihīṃ bidhi baṃdhā} |}\footnote{Cf. \citetitle{hathapradipika2024} 1.12-13.}

The yogin is supposed to sit in the hut, devote himself to Haṭhayoga and regulate the breath.\footnote{\textit{Sarvāṅgayogapradīpikā} 3.3cd: \textit{tāmahiṃ paiṭhi karai abhyāsā | gutu gami haṭha kari jātai svāsā} || 3 ||} Accordingly, for Sundardās, as in all texts with complex Yoga taxonomies without exception, breath cultivation is the central element of Haṭhayoga. In the following, he specifies the practice of Yoga postures (\textit{āsana}).\footnote{\textit{Sarvāṅgayogapradīpikā} 3.5ab: \textit{haṭhi kari āsana sādhaiṃ bhāī} \textit{hatha kari nidrā tajatau jāīī} |} Furthermore, Sundardās recommends ritual washing and god worship in the morning.\footnote{Ibid. 3.7b: \textit{prāta sanāna upāsana koī} | What this might have looked like is described in great detail within the first chapter of the \textit{Yogakarṇikā}.} The diet is supposed to be regulated.\footnote{Ibid. 3.5c: \textit{haṭha hī kari āhāra ghaṭāvai} |} For Sundardās, this means avoiding hot, spicy and sour foods. Specifically mustard, sesame, alcohol, meat, green vegetables, ginger and garlic, shall be avoided, too.\footnote{Ibid. 3.6: \textit{haṭha kari tīkṣaṇa kaṭuka sutyāgai} | \textit{sarasoṃ tila mada māṃsa na māṃgai} | \textit{harita śāka kabahū nahiṃ ṣaī} | \textit{hiṃgu lasanu saba deśa bahāī} || 6 ||} A diet of rice, milk,\footnote{Ibid. 3.7c: \textit{gohūṃ śāli su karai ahārā} |} ghee, honey and gourd vegetables is recommenced. Furthermore, clear water is supposed to be ingested.\footnote{Ibid. 3.8ab: \textit{ṣīra ṣāṃḍa ghṛta madhi puni sāṃnī} \textit{sūṃṭhi paṭola nirmala ati pāṃnī} |} When the haṭhayogin eats in this way, his body is freed from disease.\footnote{Ibid. 3.8cd: \textit{yahu bhojana su karai haṭha yogī} \textit{dina dina kāyā hoī nirogī} || 8 ||}

Verses 3.9-11 mention the six actions (\textit{ṣaṭkarma}s) - \textit{dhauti}, \textit{basti}, \textit{netī}, \textit{trāṭaka}, \textit{naulī} and \textit{kapālabhātī}. They are supposed to to purify the channels,\footnote{Ibid. 3.9b: \textit{nāḍī śuddha hoṃhi mala ṭalai} |} and lead to success.\footnote{Ibid. 3.10c: \textit{ye ṣaṭa karma siddhi ke dātā} |} In the last verse of this section, we learn that the power of Haṭhayoga leads to bliss.\footnote{Ibid. 3.12a: \textit{yā haṭha yoga prabhāva teṃ, pragaṭa hoī ānanda} |}

As already mentioned at the beginning, Sundardās also subsumes Rājayoga (3.13-24), Lakṣayoga (3.25-36) and Aṣṭāṅgayoga (3.37-52) under the superordinate category Haṭhayoga. Sundardā's Rājayoga practice is that what is commonly known as \textit{vajrolīmudrā}.\footnote{The verses do not specify the term, but the practice is identical.} Lakṣ(y)ayoga, a practice found in all complex late medieval taxonomies, is the fixation of the gaze (\textit{dṛṣṭi}) on differently located focal points or objects inside or outside the body. In the context of Aṣṭāṅgayoga, the generally known eight limbs are then discussed individually. Similar to Nārāyaṇatīrtha, characteristic practices of Haṭhayoga such as \textit{āsana}s, \textit{kumbhaka}s, \textit{mudrā}s and \textit{bandha}s are assigned to the individual limbs. A detailed comparative discussion of the subcategories takes place in the following chapters.

\subsection{Haṭhayoga in the complex early modern Yoga taxonomies}

The comparative analysis of Haṭhayoga within the complex early modern Yoga taxonomies revealed several interesting nuances across the texts. The practices attributed to Haṭhayoga are essentially identical. The major differences are based on the categorical attributions and categorisations in the texts' respective superordinate systemic approaches.

\emph{Yogatattabindu} and \textit{Yogasvarodaya} present a remarkable categorisation of Haṭhayoga into two main categories. The first category names \textit{prāṇāyāma} and the \textit{ṣaṭkarma}s as characteristic practices. The second category mentions contemplation on coloured light as a characteristic practice. Both texts understand Haṭhayoga as a method of Rājayoga.

In the \textit{Yogasiddhāntacandrikā}, Haṭhayoga is primarily defined via \textit{prāṇāyāma} and \textit{āsana}. Nārāyaṇatīrtha, however, subordinates the \textit{ṣaṭkarma}s and \textit{mudrā}s to Karmayoga. For him, Haṭhayoga is merely a means to physical perfection but cannot lead directly to Rājayoga.

For Sundardās, Rājayoga is, in turn, subordinate to Haṭhayoga, whereby he does not understand Rājayoga as \textit{samādhi}, but as a synonym for \textit{vajrolīmudrā}. For him, Haṭhayoga also consists primarily of \textit{prāṇāyāma}, \textit{āsana}s and the \textit{ṣaṭkarma}s. However, the \textit{mudrā}s and \textit{bandha}s can then be found in the subcategory of \textit{Haṭhayoga}, the \textit{Aṣṭāṅgayoga}. Sundardās does not regard all twelve Yogas as alternatives but as interrelated limbs that lead to the final state of Yoga, which he calls Advaitayoga. In his three main categories, 1. Bhaktiyoga, 2. Haṭhayoga and 3. Sāṃkhyayoga, he sees Haṭhayoga as the central practical component of his path to the final Yoga state. At the same time, Bhaktiyoga covers the devotional and Sāṃkhyayoga, the philosophical aspect of his twelve-limbed Yoga path. 


\section{5. Karmayoga}
\label{karmayogaintro}

In formal discourse, the term Karmayoga is particularly known from the \citetitle{kaushik1993}\footnote{Cf. for example \citetitle{kaushik1993} 2.47-49, 3.1-7, \& 4.20. Here, Karmayoga is a path (\textit{marga}) to liberation (\textit{mokṣa}) through action (\textit{karma}) without attachment to one's deeds.}. In the four complex late medieval taxonomies of Yogas, it appears in fifth place in the \textit{Yogatattvabindu} and third place in the \textit{Yogasvarodaya} and \textit{Yogasiddhāntacandrikā}. The \textit{Sarvāṅgayogapradīpikā} does not mention Karmayoga at all.  

\subsection{Karmayoga in the \textit{Yogatattvabindu} and \textit{Yogasvarodaya}}

In both texts, the term Karmayoga is not mentioned, despite its inclusion in the taxonomies. This absence surprises the reader, as the structure of the text, beginning with the list of fifteen Yogas and then treating individual Yogas, raises the expectation that all the subtypes of Yoga mentioned in the list will be treated. It is particularly noteworthy that Kriyāyoga, as the first entry in the list, is also treated first, and the following sections of the text largely follow the order of the list, reinforcing this expectation. However, this expected structure becomes less and less clear as the text progresses. This results in two possible explanations. Either the list merely served to illustrate the diversity of the different categories of Yoga, and it was never the authors' intention to cover all the Yogas, or the transmission of the text has fallen victim to corruption.

The analysis of the texts showed that Rāmacandra based at least the first half and also large parts of the second half of the text on the \textit{Yogasvarodaya}.\footnote{In the second half of his text, Rāmacandra also frequently uses content and verses from the \citetitle{ssplonavla} and almost without exception follows the structure as given by the quotations from the \textit{Yogasvarodaya} in the \textit{Prāṇatoṣinī}.}
However, we also know that the transmission of the \textit{Prāṇatoṣinī} is by no means complete. Many of the verses of the \textit{Yogasvarodaya} found in the \textit{Prāṇatoṣinī} can also be found in the \textit{Yogakarṇikā}. In addition, the \textit{Yogakarṇikā} contains a non-negligible number of verses that are not found in the \textit{Prāṇatoṣinī} but are nevertheless attributed to the \textit{Yogasvarodaya}.\footnote{Surprisingly, the contents of the verses of the \textit{Yogasvarodaya} cannot be traced in the \textit{Yogakarṇikā} either. Does this mean that \textit{Yogatattvabindu} used the quotations from \textit{Prāṇatoṣinī} as a template? This is impossible, as the \textit{Prāṇatoṣinī} dates from the 19th century. There were probably several recensions of the \textit{Yogasvarodaya}.} This means that the transmission of the \textit{Yogasvarodaya} based only on the verses of the \textit{Prāṇatoṣinī} and the \textit{Yogakarṇikā} cannot possibly be complete, and the original text may also have described the other fifteen Yogas not mentioned in the quotations. The structural analysis of both texts in the context of Karmayoga reveals a strong indication of corruption in the tradition.
This reference is in section \uproman{41}. Like the previous sections, starting with \uproman{32}, this section deals with the microcosmic equivalents of the macrocosm in the yogic body. In particular, it deals with the listing of various contents of the yogic body, such as twenty-seven stars, twelve signs of the zodiac, nine planets, the fluctuation of the Ūrmi, which sets the body in motion, countless deities inhabiting the pores of the arms, celestial ascetics (\textit{divyatapasvin}s) residing in the pores of the back, etc.
Then, the topic changes abruptly. In both the \textit{Yogatattvabindu} and the \textit{Yogasvarodaya}, there is suddenly a passage that describes \textit{mukti} through \textit{karma}, without a corresponding preceding introduction. Rāmacandra, apparently, as so often, prosaises the contents of \textit{Yogasvarodaya}. Therefore, the text's structural problem originates in the \textit{Yogasvarodaya}. The change in content is so abrupt that one or more folios of the copy of an archetype on which the surviving text was ultimately based may have been lost. This section of the text, which concludes the \uproman{41} section, could well be part of an original description of Karmayoga due to the abrupt change of subject. \\

The \textit{Yogasvarodaya} (PT, Ed. p. 843-44) reads:

\begin{quote}
  \textit{samagradarśanān muktaḥ svargabhogañ ca matsukham} | \\
  \textit{tad etac cintayā yāti rogaśokavivarjjitaḥ} ||\\
  \textit{yat karmā karmaṇā śaṅkā manomadhye bhaved bahiḥ\footnote{\textit{bahiḥ} em.] \textit{vahiḥ} YSv (PT).}} |\\
  \textit{tat karmākaraṇaṃ}\footnote{\textit{karmākaraṇaṃ} em.] \textit{karmakaraṇaṃ} YSv (PT).} \textit{muktir ity āha bhagavān śivaḥ} ||
\end{quote}
\begin{quote}
  As a result of complete vision\footnote{It seems very unlikely that this \textit{samagradarśanāt} refers back to the previously mentioned microcosmic contents of the macrocosm. Especially given the following statements about \textit{karma}. What it refers to is unclear.} one is liberated from heavenly pleasures and happiness. Through contemplating that, one reaches freedom from sorrow and disease. Whatever action creates concern within the mind by [considering] the action, externally, the non-execution of that [very] action brings about liberation. Thus says the exalted Śiva.
\end{quote}

The modified prosaisation of this passage in the \textit{Yogatattvabindu} (Section \uproman{41}, Ed. p. \pageref{ascetics}) reads:

\begin{quote}
  \textit{puruṣasya nṛtyadarśanāt} || \textit{gītaśravaṇāt} || \textit{vallabhavastuno darśaṇāt} || \textit{ya ānanda utpadyate saḥ svargalokaḥ kathyate} | \textit{rogapīḍito durjanebhyaḥ puruṣasya yad duḥkhaṃ utpadyate} | \textit{tad bahutaraṃ narakaṃ kathyate} | \textit{atha ca yatkarmakaraṇāt sarveṣāṃ lokānāṃ svamanasi ca śubhaṃ na bharete tat karma bandhanam ity ucyate} | \textit{atha ca yatkarmakaraṇān manomadhye śaṅkā na bhavati tatkarma muktikāraṇam} | 
\end{quote}
\begin{quote}
Whatever bliss arises as a result of witnessing dance, listening to songs, [and] viewing beloved objects, that [bliss] is called heaven. The suffering which arises for a person afflicted by disease or by evil persons is considered the greatest hell. Moreover, as a result of performing actions that do not bring about happiness in all worlds and one’s mind, it is said that this [very] action is binding. Furthermore, from whatever action within the mind, concern does not arise; that action becomes the cause of liberation.
  \end{quote}

It is probably not possible to extrapolate the complete concept from this hypothetical remnant of Karmayoga. However, it is clear that even though it is not specified as Karmayoga, a path to liberation through specific actions (\textit{karma}s) is laid out here. In the \textit{Yogasvarodaya}, all actions are not supposed to cause worry. In the \textit{Yogatattvabindu}, it is the cultivation of all actions that make one happy and the renunciation of actions that lead to sorrow. At the same time, this passage is another reference to Rāmacandra's wealthy and pleasure-oriented audience. There is also a radical contrast to the ``classical'' Karmayoga of the \citetitle{kaushik1993}. The focus is no longer on the non-attachment towards the action but on actions that bring about happiness.

\subsection{Karmayoga in the \textit{Yogasiddhāntacandrikā}}

Nārāyaṇatīrtha situates his Karmayoga\footnote{See \citeauthor{penna2004} 2004, pp. 67-20 for an earlier discussion of Karmayoga in the \textit{Yogasiddhāntacandrikā}.} in the context of his commentary on \textit{sūtra} 2.28:\footnote{Cf. \textit{Yogasiddhāntacandrikā}, ed. pp. 92-98.}

\begin{quote}
  \textit{yogāṅgānuṣṭhānād aśuddhikṣaye jñānadīptir āvivekakhyāteḥ} || 28 ||
\end{quote}
\begin{quote}
As a result of the practice of the limbs of Yoga upon the destruction of impurities, the lamp of knowledge up to the realisation of discrimination arises. 
\end{quote}

This \textit{sūtra} introduces a description of the eight well-known limbs of Pātañjalayoga. Nārāyaṇatīrtha explains that the practice of the eight limbs leads to the realisation of the overarching goal of Yoga, the discriminating knowledge of \textit{puruṣa} and \textit{prakṛti}, thereby removing ignorance (\textit{vidyā}) and manifesting liberation. He then presents Karmayoga as an alternative to attaining the lamp of knowledge:\footnote{This differentiation inevitably awakens the association with the differentiation of the eight-fold yoga according to Yajñavalkya and the Haṭhayoga with \textit{mudrā}s etc. of Kapila already stated in \citetitle{datta2024} in verse 29}

\begin{quote}
\textit{athavā yogāṅgānāṃ dhautīvastītyādiṣaṭkarmaṇāṃ mahāmudrādīnāṃ ca anuṣṭhānād dṛḍhābhyāsāj jñānadīptiḥ} | \textit{jñāyate 'neneti jñānaṃ karaṇavargaḥ} | \textit{tasya dīptiḥ rogādyanabhighātena tejasvitā dṛḍhatā ca, āvivekakhyāteḥ vivekakhyātiparyantaṃ bhavatīty arthaḥ} | \textit{rogādinā jñānasya kuṇṭhabhāvas tu prasiddha eva} | \textit{sa caiteṣv aṅgeṣv anuṣṭhiteṣu rogapratibandhān na bhavatīty arthaḥ} | \textit{tathā ca karaṇadārḍhyadvārā samādhidārḍhyārthārthakarmayogo 'pi prathamato 'nuṣṭheyo rogabhīruṇeti bhāvaḥ} | \textit{sa ca karmayogaḥ ṣaṭkarmarūpo mudrārūpaś ceti dvividho nirūpita ākare yathā} | 
\end{quote}
\begin{quote}
Alternatively, as a result of executing consistent practice of the limbs of yoga, [particularly] of the six actions like Dhautī, Vastī etc. and the great seal etc., the lamp of knowledge arises. By this [word] ``\textit{jñāna} (knowledge)'', the group of sense organs is understood. Its ``\textit{dīpti} (lamp)'' becomes brilliant and robust without damage through diseases, etc. The meaning of [the word] ``\textit{āvivekakhyāteḥ} (up to the realisation of discrimination)'' extends as far as the realisation of discrimination. Through diseases, etc., the state of the inefficiency of the sense organs (\textit{jñāna}) is thus established. Furthermore, the meaning of ``after having practised these limbs'' is [that] there are no obstacles from diseases. And thus, Karmayoga is the means for acquiring resilience of the sense organs for the steadfastness of \textit{samādhi}, which shall be practised first so that one does not become afraid of disease. And that Karmayoga, having the nature of the six actions and having the nature of the seals is discussed twofold accordingly.
\end{quote}

Next, Nārāyaṇatīrtha simply lists the \textit{ṣatkarma}s and nine \textit{mudrā}s: 
\begin{quote}
\textit{dhāutī vastī tathā neti trāṭakaṃ naulikaṃ tathā} |
\textit{kapālabhātī caitāni ṣaṭ karmāṇi pracakṣate} ||
\textit{karmaṣaṭkam idaṃ gopyaṃ dehaśodhanakārakam iti} |
\textit{mahāmudrā mahābandho mahāvedhaś ca khecarī} ||
\textit{śakticālo mūlabandha uḍḍīyānaṃ tataḥ param} |
\textit{jālandharābhidho yogo viparītakṛtis tatheti} ||
\textit{lakṣaṇāni ca tatraivoktāni} |
\end{quote}
\begin{quote}
  Dhautī, Vastī, as well as Neti, Trāṭaka and Nauli,
  and also Kapālabhāti - these six actions are being told.
  This hexade of action is to be kept secret as it produces the purification of the body.
  The great seal, the great lock, the great piercing and Khecarī,
  the stimulation of the goddess, the root lock, Uḍḍīyāṇa [and] thereafter
  [that] Yoga [practice which is] known as Jālandhara as well as the act of inversion.
  The characteristics are described there [in the following]. 
  \end{quote}

After that, Nārāyaṇatirtha presents verses containing instructive descriptions of every practice borrowed from earlier Yoga texts.\footnote{The section on the \textit{ṣaṭkarma}s is based on \textit{Haṭhapradipikā} 2.24-26, whereas the descriptions of the \textit{mudrā}s are primarily taken from the \textit{Yogacintāmanī} (Ed. p. 132 ff).} Even though Nārāyaṇatīrtha situates the \textit{ṣaṭkarma}s and \textit{mudrā}s within his Karmayoga, at the very end of the section on Karmayoga he notes that they are part of the practice of Haṭhayoga.\footnote{Cf. \textit{Yogasiddhāntacadrikā} (Ed. p. 98): \textit{etac ca sarvaṃ yogāṅgānuṣṭhānāditi sūtre sūtritam api haṭhayogāṅgatvena deha siddhamātraphalatvena sākṣādrājayogā 'naṅgatvāt kaṇṭharaveṇa sūtrakṛtā noktam iti mantavyam iti saṃkṣepaḥ} || 28 ||}

\section{6. Layayoga}
\label{layayogaintro}

Layayoga occupies fifth place in the taxonomy of the \textit{Yogatattvabindu}'s methods of Rājayoga but is not listed in the verses on the fifteen Yogas of the \textit{Yogasvarodaya}. Ultimately, however, the description of Layayoga is missing in both texts. In the taxonomy of the \textit{Yogasiddhāntcandrikā}, Layayoga is in thirteenth place. In Sundardā's \textit{Sarvāṅgayogapradīpikā} Layayoga is presented as a method of Bhaktiyoga.

\subsection{Layayoga in the \textit{Yogasiddhāntacandrikā}}

Nārāyaṇatīrtha places his discussion of Layayoga\footnote{For an earlier discussion see \citeauthor{penna2004} 2004, pp. 85-89.} in the context of his commentary of \textit{sūtra} 1.41:\footnote{\textit{Yogasiddhāntacandrikā} Ed. p. 64.} 

\begin{quote}
\textit{samprajñātasya viṣayaṃ pradarśayan na samprajñātāpararyāyaṃ layayogam āha}-\\
\textit{kṣīṇavṛtter abhijātasyeva maṇer grahītṛgrahaṇagrāhyeṣu tatsthatadañjanatā samāpattiḥ} || 41 ||
\end{quote}
\begin{quote}
Pointing out the object of [the] \textit{saṃprajñāta}[-type of \textit{samādhi}], it is said that Layayoga is for nothing other than [the] \textit{saṃprajñāta}[-type of \textit{samādhi}] - 

\textit{Samāpatti}, the state of complete absorption of the mind when it is devoid of its mental fluctuations, happens when the mind becomes like a transparent jewel that takes the form of the object placed before it, whether it is the knower, the instrument of knowing or that which is to be known.
\end{quote}

After the previous \textit{sūtra}s introduced various objects that can support the mind in meditation, this \textit{sūtra} now continues the analysis of different stages within the state of meditation, regardless of its object.\footnote{This analysis already began in \textit{Pātañjalayogaśāstra} I.17.} When the \textit{vṛtti}s of the mind fade, the mind becomes more and more like a crystal (\textit{maṇi}). Just as a crystal takes on the colouring (\textit{añjanatā}) of any object placed in front of it, the clear mind focusing on any object also takes on the colouring of that very object. \footnote{\textit{Yogasiddhāntacandrikā} 1.34 (Ed. p. 64): \textit{uparāgeṇa tadākāratāyāṃ dṛṣṭāntam āha- abhijātasyeva maṇer iti} | \textit{nirmalasya sphaṭikāder yathā japākusumādy uparāgeṇa raktādyākāratā tathety arthaḥ} |} With regard to the objects that serve absorption, the \textit{sūtra} specifies here the hierarchical sequence of the knower (\textit{grahītṛ}), the instrument of knowledge (\textit{grahaṇa}) and that what is to be known (\textit{grahyā}). For Nārāyaṇatīrtha, the knower is \textit{puruṣa}. The instrument of knowledge is the sense organs, and what is to be known is the object that can be grasped by the mind.\footnote{Ibid. 1.34 (Ed. p. 64): \textit{kṣīṇavṛtter iti} | \textit{abhyāsavairāgyābhyām apagamavṛttyantarasya cittasya grahītṛgrahaṇagrāhyeṣu, grahītā puruṣaḥ sthūlasūkṣmabhedena, grahaṇaṃ gṛhyate 'rtho 'nenetīndriyam, evaṃ grāhyaṃ ca grahītṛgrahaṇagrāhyāni} |} Depending on which object the mind focuses on, it takes on its colour and nature. The term \textit{samāpatti} refers to the complete identification of the mind with the object of meditation. Nārāyaṇatīrtha (ed. p. 64) then equates the term \textit{samāpatti} with \textit{laya}:

\begin{quote}
\textit{teṣu yā tatsthatadañjanatā tatsthena uparāgeṇa tadañjanatā tanmayatā samyak tadākāratā samāpattiḥ samyagāpattir layaḥ samprajñātalakṣaṇo yogo bhavatīty arthaḥ} |
\end{quote}
\begin{quote}
In those [objects] which are ``coloured by that which resides there'', by colouring, that [state of] colouration, being absorbed in it, thoroughly being in the state of that form, is absorption (\textit{samāpatti}), the total entering into [that] state is Laya, being a Yoga characterized as \textit{samprajñāta}. This is the meaning.
\end{quote}

For Nārāyaṇatīrtha, Layayoga is therefore a synonym for the state of \textit{samāpatti} and is attributed to the \textit{samprajñāta} form of \textit{samādhi}, in which the consciousness is still focussed on one of the aforementioned objects. \textit{Samprajñātasamādhi} is also known as `\textit{samādhi} with discrimination', as the meditator retains awareness of the distinction between the meditator, the meditation object and the process of meditation itself. It is therefore a \textit{samādhi} in which there is still a minimal remainder of \textit{vṛtti}s, in contrast to the final \textit{asaṃprajñāta} form of \textit{samādhi} in which the last \textit{vṛtti} also expires and final liberation and \textit{kaivalya} occur.\footnote{See \citetitle{yogasutra} 1.17-22 for more detailed explanations of the \textit{samprajñāta} and \textit{asaṃprajñāta} forms of \textit{samādhi}.}

\subsection{Layayoga in the \textit{Sarvāṅgayogapradīpikā}}
\label{layaintrosarvanga}
For Sundardās, Layayoga (2.28-39) is a subcategory of Bhaktiyoga,\footnote{A description of Layayoga in French can be found in \citeauthor{burger2014sarvangayogapradipika} 2014, pp. 693-94.}\footnote{?????Reference to Bhaktiyoga chapter!} and recognises it as a method for the liberation from the cycle of birth and death.\footnote{Cf. \textit{Sarvāṅgayogapradīpikā} 2.28c: \textit{laya binu janma marana nahīṃ chūṭai} |} Sundardās emphasises that Layayoga is an incomparable method and therefore attaches great importance to it among the Yoga methods he presents.\footnote{Cf. ibid. 2.29a: \textit{laya samāna nahīṃ aura upāī} |} Layayoga dispels all illusion,\footnote{Cf. Ibid. 2.29c: \textit{āvāgamana sakala bhrama bhāgai} || 29 ||} makes one attain the highest state,\footnote{Cf. ibid. 2.30d: \textit{parama sthāna samāvai soī} || 30 ||} dispels anger and difficulties,\footnote{Cf. ibid. 2.32cd: \textit{esī laya jo koī lāvai} | \textit{jonī saṃkaṭa bahuri na āvai} || 32 ||} and makes one equal to Brahman.\footnote{Cf. Ibid. 2.31a: \textit{yaha laya yoga anupa hai karai brahma samāna} |} The main emphasis of the practice is the continuous absorption of the mind into a specific goal, which he defines as Rāma\footnote{Cf. Ibid. 2.29b: \textit{jo jana rahai rāma laya lāī} |} or Hari.\footnote{Cf. Ibid. 2.38ab: \textit{sa saṃprakāra hari sauṃ lavai} | \textit{koī videha parama pada pāvai} |} This absorption is supposed to be continued throughout day and night.\footnote{Cf. ibid. 2.29c: \textit{niśi vāsara esaiṃ lai lāgai} |} To illustrate how exactly this practice is to be carried out, he draws various comparisons. For example, \textit{Sarvāṅgayogapradīpikā} reads 2.35: 

\begin{quote}
\textit{jaisaiṃ gāu jaṃgala kauṃ dhāvai} | \textit{pānī pivai ghāsa cari āvai} |\\ 
\textit{citta rahai bacharā kai pāsā} | \textit{aisī laya lāvai haridāsā} || 2.35 ||
\end{quote}
\begin{quote}
  Just as a cow walks towards the forest, drinks water, and grazes, but its mind remains near the calf, in such a way, Haridāsā practices Laya.
\end{quote}

Another example is \textit{Sarvāṅgayogapradīpikā} 2.35:

\begin{quote}
\textit{jyauṃ jananī gṛha kāja karāī} | \textit{putra piṃghrau pauḍhata bhāī} |\\
\textit{ura apnai taiṃ kṣaṇ na na bisārai} | \textit{aisī laya jana kauṃ nistārai} || 36 ||
\end{quote}
\begin{quote}
Just as a mother does the housework while her son plays or crawls nearby and never for a moment forgets him in her heart, Laya liberates the person who practices it.
\end{quote}

These comparisons illustrate Sundardā's concept of Layayoga. Layayoga is the continuous absorption or centring of the mind on Rāma or Hari while performing the necessary daily activities. The examples of the cow and the mother emphasise that this is supposed to be done in a way that resembles the tireless love and attention of a mother towards her child.

\section{7. Dhyānayoga}
\label{dhyānayogaintro}

Rāmacandra positions Dhyānayoga at the seventh place in his taxonomy of fifteen methods of Rājayoga. In the \textit{Yogasvarodaya}, Dhyānayoga is to be found at the fifth position. In both cases, Dhyānayoga as a single subcategory of Rājayoga is not discussed explicitly in the remainder of the text. In the \textit{Yogasiddhāntacandrikā}, it is in the fourteenth position. Sundardās, in his taxonomy of the three Yoga tetrads presented in the \textit{Sarvāṅgayogapradīpikā}, does not list Dhyānayoga at all.

Thus, the only explicit description of Dhyānayoga within the texts of the complex Yoga taxonomies occurs only in the \textit{Yogasiddhāntacandrikā}. However, this description parallels various contents of the \textit{Yogatattvabindu} and \textit{Yogasvarodaya}.   

\subsection{Dhyānayoga in the \textit{Yogasiddhāntacandrikā}}

Nārāyaṇatīrtha situates Dhyānayoga in the context of his comparatively extensive commentary on \textit{sūtra} 1.39:\footnote{Cf. \emph{Yogasiddhāntacandrikā} ed. p. 56-63.}

\begin{quote}
\textit{dhyānayogam āha} - 
\textit{yathā 'bhimatadhyānād vā} || 39 ||
\end{quote}
\begin{quote}
[With regard to] Dhyānayoga, it is said: 
 Or, as a result of meditation on what one favours.
\end{quote}

Below, Nārāyaṇtīrtha's commentary offers two alternative explanations of Dhyānayoga. The first explanation is presented briefly and reads as follows: 

\begin{quote}
  \textit{yatheti} | \textit{kim bahunā, harirāmādirūpaṃ parameśvaraṃ bāhyaṃ candrasūryādijyotir vā yad eveṣṭaṃ tad eva dhyāyet} | \textit{tasmād api dhyānāl labdhasthitikasya cittasya sādhanāntaraṃ vināpi kevale paramātmani sthitau yogyatā bhavatīty arthaḥ} | \textit{ayam eva dhyānayoga ukto yogagrantheṣu} |\\
  
  \textit{vinā deśādibandhena vṛttir yā 'bhimate sthirā} |\\
  \textit{dhyānayogo bhaved eva cittacāñcalyanāśakaḥ} ||\\
    \textit{ity ādinā} | 
\end{quote}

\begin{quote}
  [Regarding the term] ``yathā''.  Why [say] more? One should meditate on the supreme lord in the form of Hari, Rāma, etc., or on an external light such as the moon, sun, etc. [or] just to what is favored. Because of that, as a result of meditation alone, the stability of the mind is attained without the need for any other means, enabling one to reside in the supreme self. This is the meaning. This very Dhyānayoga is taught in the texts of Yoga; [for example] in quotations such as: \\

  Without being confined by place, etc., the fluctuations of the mind become stable in the preferred [object]. In fact, Dhyānayoga is the destroyer of the fickleness of the mind.\footnote{I am yet to identify the source of this \textit{śloka}.}\\
  
\end{quote}

The first model refers to the meditation of primarily to certain external objects in general, which leads to the reduction of fluctuations in the mind. 
The second model, on the other hand, is described in the following sentences and then explained in detail:

\begin{quote}
\textit{yad vā yathābhimatānāṃ tīrthadevalokavarṇatattvādīnāṃ yathābhimateṣu svadehādiṣu dhyānād bhāvanāviśeṣān manasaḥ sthitir bhavatīty arthaḥ} | \textit{tatra yady api brahmavido brahmamayatvādinā sarvam eva tīrthaṃ pratilomakūpaṃ ca tīrthāni bhavantīti tathāpi yuñjānena cittaśuddhy arthaṃ prathamatas tīrthādikam avaśyaṃ bhāvanīyam} |
\end{quote}
\begin{quote}
 Alternatively, that stability of the mind arises from a specific application of meditation onto favoured [objects] like, for example, sacred sites, deities, worlds, letters, principles, etc., with regard to favoured locations within one's own body. In that case, it is stated, although the knowers of Brahman assert that because of the pervasiveness of Brahman, everything indeed is a sacred place, and even the pores of the skin become places of pilgrimage. Nevertheless, the yogin (\textit{yuñjāna}) who is aiming at the purification of the mind, must inevitably contemplate sacred places, etc. in the beginning [of pracitce].   
  \end{quote}

Nārāyaṇatīrtha differentiates an alternative that is aimed particularly at beginners in meditation practice. Nārāyaṇatīrtha devotes the rest of his commentary on \textit{sūtra} 1.39 to this type of meditation, which is aimed at objects located inside the body. He first specifies \textit{tīrthabhāvanā},\footnote{Cf. \emph{Yogasiddhāntacandrikā} Ed. p. 57-59} the meditation on sacred places, in which the practitioner is supposed to meditate on various sacred places of India in different body parts. He then specifies \textit{devabhāvanā},\footnote{Cf. Ibid. Ed. p. 59.} the meditation of different deities, which are located in body parts, and \textit{lokabhāvanā},\footnote{Cf. Ibid. Ed. p. 59.} the meditation on the worlds in the body and \textit{varṇabhāvanā},\footnote{Cf. Ibid. Ed. p. 59.} the meditation on letters in the body, each placed in one of six \textit{cakra}s.\footnote{Cf. Ibid. Ed. p. 59-61}. Then \textit{tattvabhāvana}, the meditation on the principles, is described.\footnote{Cf. Ibid. Ed. p. 61-63} The commentary concludes by discussing manipulating air currents through the nostrils for beneficial results, such as in heat or cold exposure, intercourse, travelling, etc. A useful summary of the details of this part of Nārāyaṇatīrtha's commentary has already been provided by \citeauthor{penna2004} (2004: 91-97) and does not need to be repeated here.

\subsection{Dhyāna in the \textit{Yogatattvabindu} and \textit{Yogasvarodaya}}

Dhyānayoga is mentioned in the taxonomies of both texts\footnote{The list of mentions of \textit{dhyāna} is based on the sections of the \textit{Yogatattvabindu}. The corresponding passages of the \textit{Yogasvarodaya} can be taken from the critical apparatus of the present edition of the text.} but is does treated as an individual topic. However, various \textit{dhyāna} practices can be found throughout the text. The first mention of \textit{dhyāna} occurs in the context of nine \textit{cakra}s in the sections \uproman{4}-\uproman{12}. Rāmacandra and the unknown author of the \textit{Yogasvarodaya} instructs \textit{dhyāna} on the respective \textit{cakra}, or a \textit{mūrti} located in the respective \textit{cakra}. The scribe-author of manuscript \getsiglum{U2} even adds more precise instructions on the duration of the meditations on the respective \textit{cakra}s. However, as we discover in section \uproman{3}, this meditation practice is attributed to Siddhakuṇḍalinīyoga or Mantrayoga and not to Dhyānayoga. This is surprising since earlier sources which include a ninefold system of \textit{cakra}s - the \emph{śārṅgadharapaddhati}, the \emph{Vivekamārtaṇḍa} and \textit{Śivayogapradīpikā} all teach these nine bodily locations within their sections on \textit{dhyāna}. We also encounter the term \textit{dhyāna} in the description of \textit{adholakṣya} in section \uproman{15}, in the second subtype of Haṭhayoga in section \uproman{20}, in the description of \textit{bāhylākṣya} in section \uproman{23}, as well as within \textit{antaralakṣya} in section \uproman{24}. Another mention can be detected within the list and the eight limbs of \textit{aṣṭāṅgayoga} in section \uproman{31}. Here, Rāmacandra states that \textit{dhyāna} will not be discussed in this context, as this has happened many times before.\footnote{\textit{dhyānaṃ ca bahutaraṃ prāg uktaṃ tenātra cocyate} |} In \uproman{32}-\uproman{41} the identity of the external universe with the body is taught. Various contents, such as the fourteen worlds, mountains and rivers, etc., are located in the body, similar to the \textit{Yogasiddhāntacandrikā}. However, Rāmacandra does not specify a concrete reason for listing these physical equivalents of the external universe in the body. The same is true for the parallel passages of \textit{Yogasvarodaya} and \textit{Siddhasiddhāntapaddhati}. In section \uproman{48}, in the context of the divisions of the lotus in the heart, meditation on this heart lotus is precribed. This meditation is supposed to lead to the illumination of the self and enhance vitality. Therefore, I conclude that although Dhyānayoga is not provided with its own section in either text, it is at least implicitly present in both texts and the generic term of meditation (\textit{dhyāna}) is nevertheless a central theme. 

\section{8. Mantrayoga}
\label{dhyānayogaintro}

Mantrayoga occupies the eighth position in the taxonomy of the Rājayoga methods within the\textit{Yogatattvabindu}. It occupies the sixth position in the \textit{Yogasvarodaya} and the fifth position in the \textit{Yogasiddhāntacandrikā}. Among the Yogas of Sundardās's \textit{Sarvāṅgayogapradīpikā} Mantrayoga is considered to be one of the four methods of Bhaktiyoga.

\subsection{Mantrayoga in the \textit{Yogatattvabindu} and \textit{Yogasvarodaya}}
\label{bindumantra}
Apart from the mention of Mantrayoga in the first verses of the quotations of the \textit{Yogasvarodaya} in the \textit{Prāṇatoṣinī}\footnote{Cf. \emph{Prāṇatoṣinī} ed. p. 831 quoted with reference \textit{yogasvarodaye}.} the quotations we have at hand contain no dedicated description of Mantrayoga. However, in the context of the \emph{Yogasvarodaya}'s description of Aṣṭāṅgayoga\footnote{Cf. \emph{Yogasvarodaya} (PT p. 841.)} a practice involving \textit{mantra}s is mentioned in passing. The unknown author distinguishes two types of \textit{dhyāna} - one is said to be gross and the other subtle. The gross type is associated with \textit{mantra}s, while the subtle type is devoid of  \textit{mantra}s. The available testimonies of the \emph{Yogasvarodaya} do not provide further details.  

In the \textit{Yogatattvabindu}, however, the term Mantrayoga appears again in section \uproman{3}:

\begin{quote}
  \textit{idānīṃ rājayogasya bhedāḥ kathyante} | \textit{ke te} | \textit{ekaḥ siddhakuṇḍalinīyogaḥ mantrayogaḥ amū rājayogau kathyete} |
\end{quote}
\begin{quote}
  Now, varieties of Rājayoga are described. Which are these? One is Siddhakuṇḍalinīyoga and one is Mantrayoga. These two Rājayogas are described [in the following].
\end{quote}

This is followed by an explanation of the three primary channels of the yogic body: Iḍā, Piṅgalā and Suṣumnā. The section concludes with the assertion that the practitioner becomes omniscient once knowledge about the central channel is generated. In the following sections (\uproman{4}-\uproman{12}), a system consisting of a total of nine \textit{cakra}s is then described.

This passage is problematic from a text-critical perspective. Rāmancandra is very much orientated towards his textual source, the \textit{Yogasvarodaya}, in terms of structure and content, particularly in the first half of his text and mainly in the second half. However, the \textit{Yogasvarodaya} specifies \textit{jñānayoga} instead of \textit{siddhakuṇḍalinīyoga mantrayogaḥ}. As usual, the remainder of the section is very similar in content to the \textit{Yogasvarodaya}. However, the manuscripts offer no alternatives for the conspicuous passage, so the text must be accepted for now. Another reason is the seemingly strange sentence construction, which is ultimately unsurprising if one knows the rest of the text and can be accepted. Right after the term \textit{mantrayogaḥ}, the reader would have wished for a \textit{ca} (``and''). Only the manuscript \getsiglum{L} omits the term \textit{mantrayogaḥ} but preserves the following dual forms, so this is not a solution either.    

The first \textit{cakra} named \textit{mūlacakra} is provided with the following introduction:

\begin{quote}
  \textit{idānīṃ suṣumṇāyāḥ jñānotpattāv upāyāḥ kathyante} | \textit{ādau caturdalaṃ mūlacakraṃ vartate} | 
\end{quote}
\begin{quote}
  Now, the means for the genesis of knowledge of the central channel is described. At the beginning [of the central channel] exists the four-petalled root-cakra.
  \end{quote}

  On the basis of this description, it can only be assumed that the sections \uproman{4}-\uproman{12} describing the nine \textit{cakra}s are assigned by Rāmacandra to Siddhakuṇḍalinīyoga and Mantrayoga. However, almost all manuscripts, with the exception of the \getsiglum{U2} manuscript, do not allow any conclusions to be drawn in this context about a practice that could be described as Mantrayoga.

  However, the manuscript \getsiglum{U2} contains detailed additional passages that solve the problem and supplement a practice that can be described as Mantrayoga. For each \textit{cakra}, all manuscripts instruct \textit{dhyāna} on the respective \textit{cakra}. Manuscript \getsiglum{U2}, in addition to various additional details, always contains an indication of the duration of the meditation, which is measured in \textit{ajapājapa}s (``The recitations of the non-recited.'').\footnote{The \textit{cakra}s additionally receive the same time indication measured in \textit{ghaṭi}s, \textit{pala}s and \textit{akṣara}s. Instructions for the duration of the practice of meditation are in most of the additions of U\textsubscript{2} for each \textit{cakra}, except the seventh \textit{cakra} at the palate and the ninth \textit{cakra} named \textit{mahāśūnyacakra}. For example manuscript \getsiglum{U2} instructs a total of 600 \textit{ajapājapa}s as the duration of meditation onto the \textit{mūlacakra}. This refers to the duration of the voiceless uttering of the natural \textit{mantra} of the breath: \textit{so 'haṃ} (``he is I'') - \textit{haṃ sa} (``I am him''). As in many other Yoga texts, the total amount of \textit{ajapājapa} per day is declared to be 21600 (cf. section \uproman{11}. on p.\pageref{cakra8}, l.7). If 21600 \textit{ajapājapa} equals 24 hours, then 600 \textit{ajapājapa} would equal 40 minutes. In the additions of U\textsubscript{2}, one finds the same numbers of \textit{ajapājapa} as in the instructions for meditation onto the seven \textit{cakra}-system of Jayatarāma (cf. \citeauthor[2006: 163]{jogpradipyaka} and \citetitle{jogpradipyaka} 889-912.). The redactor of the text as found in U\textsubscript{2} applied the system of the durations for seven \textit{cakra}s to the ninefold \textit{cakra} system of Rāmacandra. Next, the duration that was mentioned before as 600 \textit{ajapājapa} is repeated in another scheme by stating ``\textit{ghaṭi} 1 \textit{palāni} 40''. One \textit{ghaṭi} equals 1/60 of a day (cf. \citeauthor[1966: 114]{sircar1966}), which is 24 minutes. One \textit{pala} equals 1/60 of a \textit{ghaṭi}, which is 24 seconds (cf. \citeauthor[1858: 4]{petersburger4}). The \textit{Amanaska} in 1.35 (cf. \citeauthor[2013: 231]{birch2013}) uses the same concept. For a more detailed tracing of the usage of the system in yogic and tantric literature, see \citeauthor[2013: 265, endnote 46]{birch2013}. In our case, the 24 minutes of the one \textit{ghaṭi} plus the 16 minutes (40x24 seconds) of 40 \textit{pala}s once more sums up to 40 minutes for the instructed duration of meditation onto the first \textit{cakra}. Other systems are less specific. \citetitle{kumbhaka} 208, i.e. states that ``Six winkings are one \textit{prāṇa}, six \textit{prāṇa}s make up one \textit{pala}. Sixty \textit{pala}s equal the time-period of a \textit{ghaṭikā}.'' (\textit{ṣaṇṇimeṣo bhavat prāṇaḥ ṣaḍbhiḥ prāṇaiḥ palaṃ smṛtaṃ} | \textit{palaiḥ ṣaṣṭibhir eva syād ghaṭikākālasammitā} ||).
According to \citeauthor{birch2013} (2013) the time unit \textit{akṣara} appears in Bhāskara's \citetitle{siddharoma} (17cd − 18ab of the \textit{Kālamānādhyāya} in the \textit{Madhyamādhikāra}): (\textit{gurvakṣaraiḥ khendumitair asus taiḥ | ṣaḍbhiḥ palaṃ tair ghaṭikā khaṣaḍbhiḥ || syād vā ghaṭīṣaṣṭir ahaḥ kharāmair māso dinaistair dvikubhiś ca varṣam |}) Translation by \citeauthor[2013:265, n. 46]{birch2013}: ``A breath is ten long syllables, and a Pala is six breaths, sixty Palas is one Ghaṭikā, sixty Ghaṭikās is a day, thirty days is a month, and twelve months is a year.'' If one assumes an \textit{akṣara} to be 1/10 of a breath and 21600 breaths per day, one hour would have 900 breaths, one minute would equal 16 breaths, one breath would equal 4 seconds, and one \textit{akṣara} would be 0,4 seconds or 400 milliseconds.\label{ghatinote}} Finally, the additional material in section \uproman{11} of manuscript \getsiglum{U2} makes it clear that the so-called \textit{ajapā mantra} or \textit{haṃsa mantra} must be meant here:\footnote{Probably first taught in the Yoga literature in \citetitle{vivekamartandaolda} 28-30}.
    
  \begin{quote}
    \textit{sakāreṇa bahir yāti hakāreṇa viśet punaḥ} |\\  
    \textit{haṃsaḥ so 'haṃ tato mantraṃ jīvo japati sarvadā} ||
    \end{quote}
  \begin{quote}
With the sound ``sa'', he exhales. With the sound ``ha'', he inhales again: ``I am he, he is I''. Because of that, the embodied soul constantly utters the Mantra.
\end{quote}

The \textit{ajapā mantra} (``unmuttered mantra'') consists of the two syllables \textit{haṃ} and \textit{saḥ} according to the phonological association with the sound of inhalation and exhalation. Because all living beings inhale and exhale, they recite the \textit{ajapā mantra} continuously day and night. At the same time, \textit{haṃsa}, best translated as "swan" or "goose" in English, is a famous and ancient metaphor for the soul travelling through the wheel of Brahman or Saṃsāra.\footnote{See \citetitle{hauschild1927} 1.6 and 3.18.} Sometimes this mantra is also specified as \textit{ajapā gāyatrī}.\footnote{The \textit{ajapā} can be seen as a yogic appropriation of the Vedic \textit{gāyatrīmantra} (\citetitle{rootsofyoga2017} 2017, 134).}  

Manuscript \getsiglum{U2} explains that the total daily number of all silent recitations of the \textit{haṃsa mantra} is 21600.\footnote{The number of total breaths is based on the assumption of an average breath duration of four seconds. Each day has 86400 seconds. If one divides this total number by four, one gets the 21600 breaths of the \textit{ajapā mantra}. \citeauthor{birch2013} (2013, 265, n. 46) argues that this assumption comes from \citetitle{svacchandatantra} 7.54-55. In addition to the \getsiglum{U2} manuscript of \textit{Yogatattvabindu}, this yogic axiom is widely used in Sanskrit Yoga literature. See for example \textit{Amaraughaprabodha} 58, Hemacandra's \citetitle{hemacandras} 5.232, \textit{Vivekamārtaṇḍa} 46, \textit{Gheraṇḍasaṃhitā} 5.79, \textit{Dhyānabindūpaniṣad} 62ab-63ab or \citetitle{jogpradipyaka} 913.} The association of the term Mantrayoga with the practice of \textit{haṃsa mantra} is widespread in Sanskrit Yoga literature.\footnote{See e.g. \citetitle{yogabija} 147; \citetitle{shivayogapradipika} 2.26-27 and 2. 29-32 (\citeauthor{powell2023} 2023, p. 205), explains that here, however ``mantra is reframed and interiorised within a \textit{prāṇāyāma} environment, specifically in the form of the \textit{ajapā}, the "unuttered" mantra''); \emph{yogacintamani} (Ed. p. 12); \citetitle{hathatattvakaumudi} 55.28; and \citetitle{yogasikhopanisad} 132.}
%%maybe incorporate ŚYP : (2.34) Having made so ’ham one’s personal mantra—in which the two syllables are expressed as one's self and the Supreme—[the yogin] should take away the two consonants and refashion it as the divine mantra oṃ. Having joined it with the nasal sound (anusvāra), it is the best of all mantras. He who leads it to the brahmanāḍī (i.e. suṣumṇā) is full of bliss, [even if] deprived of the experience of Kuṇḍalinī. He attains release from [all] karma. 

From a text-critical perspective, there is ambivalent evidence regarding the authenticity of the passages under discussion. All manuscripts mention Mantrayoga in the above passage. We must, therefore, assume that Mantrayoga was originally and perhaps even deliberately specified here by Rāmacandra, even if, or precisely because, he reads the source text differently. The fact that only the manuscript \getsiglum{U2} explicitly teaches a Mantrayoga must make one suspicious. This manuscript only contains additional material in the sections \uproman{4}-\uproman{12}. The most likely scenario is that the scribe of the manuscript \getsiglum{U2} made these additions to provide the missing explanations on Mantrayoga.\footnote{The connection between Siddhakuṇḍalinīyoga and Mantrayoga established in \getsiglum{U2} is found in a similar form in \citetitle{sarada} 25.37ab: ``The \textit{kuṇḍalī} Śakti abides in the \textit{haṃsaḥ} [and] supports the [individual] Self.'' (\textit{bibharti kuṇḍalī śaktir ātmānaṃ haṃsaṃ āśritā} |), see \citeauthor[2011: pp. 218, 228]{saradatilakafull}.} Manuscript \getsiglum{U2} belongs to the \beta group of manuscripts, which often contains poorer readings in a large part of the text than the \alpha group with the oldest manuscript \getsiglum{N1}. This also makes the other scenario seem far less likely at first, namely that \getsiglum{U2}, despite its later dating, transmits a more original text than all other textual witnesses. However, the oldest manuscript \getsiglum{N1} has immense gaps, at least in the last third of the text. On the other hand, manuscript \getsiglum{U2} is complete here, together with some candidates of the \beta-group. Furthermore, only manuscript \getsiglum{U2} preserves the correct variant of the sentence \begin{quote} \textit{bhuktimuktidā śivarūpiṇī suṣumṇānāḍī pravartate} | \textit{asyā jñānotpattau satyāṃ puruṣaḥ sarvajño bhavati} | \end{quote} in section \uproman{3}. Therefore, the additions of \getsiglum{U2} were printed in greyscale in the edition and not relegated to a footnote.

\subsection{Mantrayoga in the \textit{Yogasiddhāntacandrikā}}
\label{mantrayogaintrocandrika}
Nārāyaṇatīrtha locates Mantrayoga, like Jñānayoga before it, in the context of \textit{sūtra} 1.28. This \textit{sūtra} and the corresponding commentary by Nārāyatīrtha have already been discussed in the chapter on Jñānayoga in the \textit{Yogasiddhāntacandrikā} (p.\pageref{jnanayogaintrocandrika} et seqq.) and therefore need not be repeated here.\footnote{For another discussion of Mantrayoga in the \textit{Yogasiddhāntacandrikā} see \citeauthor{penna2004} 2004, pp. 71-76.} Mantrayoga in the \textit{Yogasiddhāntacandrikā} is \textit{japa} (``low-voice muttering'') of \textit{praṇava} (``sacred syllable \textit{auṃ}''), which can be performed in two alternative ways, as Jñānayoga\footnote{I discuss the concept of Jñānayoga in the \textit{Yogasiddhāntacandrikā} on p. \pageref{jnanayogaintrocandrika}.} or Advaitayoga.\footnote{The concept of Advaitayoga in the \textit{Yogasiddhāntacandrikā} I discuss on p.\pageref{advaitayogaintrocandrika}.}      
%\begin{quote}
%\textit{taj japas tadarthabhāvanam} || 28 ||
%\end{quote}
%\begin{quote}
%Its low-voice muttering; contemplation of its meaning. || 28 ||
%\end{quote}
%Dieses \textit{sūtra} gehört zu den \textit{sūtra}s des \textit{Pātañjalayogaśāstra}, welches verschiedene Methoden erläutern \textit{samādhi} zu erlangen. \textit{Sūtra} 1.28 gehört zu einer Gruppe von \textit{sūtra}s (1.23-1.28), welche als ``hingebungsvolle Verehrung des höchsten Gottes'' (\textit{īśvarapraṇidhāna}) bezeichnet wird. Während 1.24-26 das Konzept von \textit{īśvara} erläutern, erklärt 1.27 das dessen Bezeichnung (\textit{vācaka}) die Silbe \textit{auṃ} ist, die hier mit \textit{praṇava} bezeichnet wird. 
%Nārāyaṇatīrtha erklärt zu 1.28, dass sich \textit{taj japas} (``its low-voice muttering'') auf die stille Rezitation von \textit{praṇava} bezieht. Gleichzeitig solle dessen Bedeutung, nämlich \textit{īśvara} kontempliert werden. Hierfür richtet der Übende dieser Kontemplation richtet seinen Geist auf das höchste Selbst, also \textit{īśvara}, welches mit unbegreiflich oberherrlicher Macht versehen ist. Diese Kontemplation führe zur unterscheidenen Erkenntnis zwischen der Urnatur (\textit{prakṛti}, dessen Effekten\footnote{Für eine Übersicht siehe \citeauthor{bryant2009}, 2009, pp. xlvii-liii.} und dem Selbst (\textit{puruṣa}.\footnote{\textit{Yogasiddhāntacandrikā} (Ed. p. 45): \textit{taj japa iti} | \textit{tasya praṇavasya japaḥ vidhivaduccāraṇaṃ, tadarthasya praṇavārthasya acintyaiśvaryaśaktiyuktasya paramātmano bhāvanaṃ prakṛtitatkāryapuruṣebhyo vivekenānusaṃdhānam} |} Insbesondere soll der Übende das eigene Selbst mit dem höchsten Brahman\footnote{Hier explizit als das höchste Selbst (\textit{puruṣa}) gedacht.} identifizieren, so wie unteilbare Natur von einer Qualität von dessen Substanz. Bei der kontemplativen Rezitation von \textit{praṇava} wrid eben dieses in drei Teile geteilt, denen je eine Bedeutung zugewiesen wird. Das \textit{a} (\textit{akāra} steht für das eigene Selbst, das \textit{m} (\textit{makāra}) für Brahman (das höchste Selbst) und \textit{u} (\textit{ukāra}) für Zweifelslosigkeit (\textit{avicikitsita}).\footnote{Ibid. (Ed. p. 45): \textit{tam etam ātmānaṃ ūm̐mati brahmaṇaikīkṛtya brahma vātmanomityekīkṛtya} | \textit{iti śruteḥ} | \textit{vastutas tu} - \textit{akāreṇa mamātmānamanviṣya makāreṇa brahmaṇā 'nusaṃdadhyāt ukāreṇā 'vicikitsitaḥ} | ity ādiśrutes tadarthasya jīvaparamātmanor abhedasyaitanmate dravyaguṇayor ivāvinābhāvarūpasya bhāvena cintanamityarthaḥ | japapūrvakaṃ bhāvanaṃ karttavyam |}                                                    
%
%Am Ende des Kommentares zu 1.28 baut Nārayaṇatīrtha dann Assoziationen mit insgesamt dreien seiner fünfzehn Yogas auf, unter anderem auch das hier im Fokus stehende Mantrayoga:
% \begin{quote}
%\textit{kiñ ca japa ity anena mantrayogaḥ arthabhāvanam ity anena vivekajñānā 'bhyāsarūpo jñānayogaḥ abhedabhāvarūpo 'dvaitayogaś ca saṃgṛhītaḥ} |
%\end{quote}
%\begin{quote}
%Furthermore, by \textit{japa} (``low-voice muttering''), Mantrayoga is implied; by \textit{arthabhāvanam} (``contemplation of its meaning,'') Jñānayoga is implied, which is characterized by the practice of discriminative knowledge; and [additionally] Advaitayoga is implied, which has the nature of non-difference.
%\end{quote}

\subsection{Mantrayoga in the \textit{Sarvāṅgayogapradīpikā}}
\label{mantrayogaintrosarva}

Sundardās introduces his Mantrayoga (2.16-27) with the question of how the formless and featureless highest reality can be named.\footnote{\textit{Sarvāṅgayogapradīpikā} 2.16cd: \textit{jākai kachū rūpa nahiṃ reṣā} \textit{kauna prakāra jāī so deṣā} || 16 ||} For without giving it a name, one cannot refer to it.\footnote{Ibid. 2.17b: \textit{nāma binā nahiṃ lagai piyārā} |} A personal surrender, a devotion to the highest reality, is the basic prerequisite for Bhaktiyoga, the superordinate category of Sundardā's Mantrayoga. The best, or verbatim the crown of all names for the highest reality, is \textit{rāma}.\footnote{Ibid. 2.19cd: \textit{rāma mantra sabakai siramaurā} \textit{tāhi na koī pūjata aurā} || 19 ||} After verses of praise of the \textit{rāma mantra} Sundardās explains that the \textit{rāma mantra} has to be learnt from the Guru. At the beginning of Mantrayoga practice, one is supposed to recite the \textit{rāma mantra} with the tongue, i.e. audibly.\footnote{Ibid. 2.23cd: \textit{prathama ..vana suni guru kai pāsā} \textit{puni so rasanā karat abhyāsā} || 23 ||} In the course of the practice, the \textit{rāma mantra} is then supposed to be recited mentally, constantly, day and night, in order to unite the practitioner with the omnipresent highest reality:

\begin{quote}
\textit{..pīchai hiradai maiṃ dhārai} | \textit{jihvā rahita maṃtra uccārai} |\\ 
\textit{niśa dina mana tāsauṃ raha lāgau} | \textit{kabahūṃ naiṃka na ṭūṭai dhāgau} || 24 ||\\
\textit{puni tahāṃ pragaṭa hoī raṃkārā} | \textit{āpuhi āpu akhaṇḍita dhārā} |\\
\textit{tana mana bisari jāī tahāṃ soī} | \textit{romahi roma rāma dhuni hoī} || 25 ||\\
\end{quote}
\begin{quote}
(24) Afterwards, retain it [the mantra] in the heart; recite the mantra without the tongue.
Night and day, let your mind stay attached to it; may the thread never break.\\
(25) Then there, the omnipresent one manifests; oneself becomes an unbroken stream.
Body and mind forgotten there, in that state; in every hair, the sound of Rāma resonates.
\end{quote}

Thus, Mantrayoga in \textit{Sarvāṅgayogapradīpikā} is a form of Bhaktiyoga that seeks union with the highest reality in the form of devotional recitation of the \textit{rāma mantra}. 

\section{9. Lakṣyayoga}
\label{laksyayogaintro}

Lakṣyayoga is one of the most voluminous and most important topics\footnote{In the \textit{Śivayogapradīpikā} 1.8, the one who has attained the realisation of Brahman using the (in this case) three \textit{lakṣya}s is called a knower of Rājayoga. In this text, the practice of \textit{lakṣya}s is the primary characteristic practice of Rājayoga. In addition, being free from mental fluctuation through gnosis is specified as the second characteristic practice. (\textit{triṣu laṣyeṣu yo brahmasākṣātkāraṃ gamiṣyati} | \textit{jñāne vātha manovṛttirahito rājayogavit} || 1.8 ||} in the \textit{Yogatattvabindu}.\footnote{Cf. \textit{Yogatattvabindu} sections \uproman{13} (overview of the five \textit{lakṣya}s), \uproman{14} (\textit{adholakṣya}), \uproman{15} (\textit{ūrdhvalakṣya}), \uproman{23} (\textit{bāhyalakṣya}), \uproman{24} (\textit{antaralakṣya}) and \uproman{27} (\textit{madhyalakṣya}) of the \textit{Yogatattvabindu} deal exclusively with the types of Lakṣyayoga.} The concept of this type of Yoga has a complex history of reception, and its origins as a category of specific Yoga techniques can be traced far back into early Tantric texts.\footnote{The yoga practice of \textit{lakṣya}s derives from an ancient Śaiva paradigm. The exact roots of this paradigm are difficult to reconstruct precisely. In many cases, the \textit{lakṣya}s are taught together with a system of six to nine \textit{cakra}s, sixteen \textit{ādhāra}s and five \textit{vyoma}s, \textit{ākāśa}s or \textit{kha}s. In most texts that take up this paradigm, there is a variant of a verse also contained in the \textit{Yogatattvabindu}, which lists the elements just mentioned as essential components of Yoga. See \textit{Yogatattvabindu} section \uproman{28}.1 for the verse and its variants in other contemporary and earlier texts. Perhaps the oldest datable textual evidence for the practice of yogic \textit{lakṣya}s can be found in \textit{Netratantra} 7.1-2, which was composed between 700-850 CE, cf. \citeauthor{sanderson2004} 2004, p. 243. However, here, the \textit{lakṣya}s are only listed and not further explained, so we can assume that this practice is probably older than the \textit{Netratantra} itself. Kṣemarāja, in his \textit{Netroddyota} commentary, further elaborates on the three \textit{lakṣya}s. He briefly states: \textit{trīṇy antarbahirubhayarūpāṇi lakṣyāṇi lakṣaṇīyāni yatra} | \textit{nirāvaraṇarūpatvāt ``khamanantaṃ tu janmākhyaṃ''} \textit{Netratantra} (7.27). `The three foci, internal, external or both, are to be attained, and because they are unobstructed, ``The endless void is called the birth''. Furthermore, the \textit{lakṣya}s are no longer mentioned directly in the text. However, the \textit{Netratantra} in 8.39-44 seems to refer to the techniques of the \textit{lakṣya}s. At this passage of the text, the yogin has already reached \textit{samādhi}. In this state, he is instructed not to direct his meditation towards various foci anymore. The descriptions of the foci negated here sound very similar to the descriptions of the three to five \textit{lakṣya}s of the late medieval texts of the complex Yoga taxonomies. For example, \textit{Netratantra} 8.42 explains: \textit{nāntaḥ śarīrasaṃsthāne na bāhye bhāvayet kvacit} | \textit{nākāśe bandhayel lakṣyaṃ nādho dṛṣṭiṃ niveśayet} || 42 ||. `One should not contemplate any place of the body inside or outside. One should not fix one's attention towards the sky (open space), nor should one direct one's gaze downwards.' Instead, the yogin should abandon everything and focus the mind on the supreme alone and in isolation". Cf. \textit{Netratantra} 8.44cd.

The \textit{Mālinīviyajottaratantra} (12.9) and other linked Tantras (e.g. \textit{Kiraṇatantra} 2.22-23 and \textit{Dīkṣottara} 2.2-3.) also contain a system of \textit{lakṣya}s. In the \textit{Mālinīviyajottaratantra}, there are six \textit{lakṣya}s. These six \textit{lakṣya}s are labelled as follows: 1. emptiness (\textit{vyoman}), 2. body (\textit{vigraha}), 3. drop (\textit{bindu}), 4. phoneme (\textit{arṇa}), 5. world (\textit{bhuvana}) and 6. resonance (\textit{dhvani}). According to \citeauthor{vasudeva2004} (2004: 255), \textit{lakṣyabheda} in \textit{Mālinīviyajottaratantra} denotes `the ultimate destination upon which the Yogin must fix his attention’. These \textit{lakṣya}s are `different manifestations through which Śiva can be approached’. He further states: `To the Yogin engaged in the conquest of realities the \textit{lakṣya}s serve as teleological magnets drawing him towards the sought after rewards’. Despite the same basic concept, the \textit{lakṣya}s of the \textit{Mālinīviyajottaratantra} appear very different at first glance. On closer inspection, however, there are striking parallels with the \textit{lakṣya} systems found in the late medieval texts treated in this chapter. For example, the first \textit{lakṣya} of the \textit{Mālinīviyajottaratantra} 12.10abc is described as follows: \textit{bāhyabhyantarabhedena samuccayakṛtena ca} \textit{trividhaṃ kīrtitaṃ vyoma}. `The void is said to be threefold by the division of external, internal and that arising from accumulation’. \citeauthor{vasudeva2004} (2004: 263) maintains that this elliptical definition can only be explained on the basis of the teachings on the voids of other Śaiva Tantras but notes that none of the systems he consulted show complete congruence with the position of the \textit{Mālinīviyajottaratantra}. Nevertheless, he cites, for example, the passages from \textit{Dīkṣottara} 3.10c-11 and \textit{Svaccandatantra} 4.289 that are particularly interesting for our context, in which an upper emptiness (\textit{ūrdhvaśūnya}), a lower emptiness (\textit{adhaḥśūnya}) and a middle emptiness (\textit{madhyaśūnya}) are distinguished.
  
Taken together, the basic features of the late medieval differentiation of the five \textit{lakṣya}s into \textit{ūrdhva}-, \textit{adho}-, \textit{bāhya}-, \textit{antara}-, and \textit{madhyalakṣya} can already be discerned here. The \textit{lakṣya}s of the \textit{Mālinīviyajottaratantra} are discussed in detail in \citeauthor{vasudeva2004} (2004: 253-293). This rough overview illustrates that different systems of yogic \textit{lakṣya} practices have been circulating in the Śaiva Tantras for a very long time. Over the centuries, the techniques were passed on, copied and reused in the yoga traditions of Haṭha- and Rājayoga. In addition to the four texts analysed in this chapter, different forms of \textit{lakṣya} practice can also be found, for example, in \textit{Vivekamārtaṇḍa}, \textit{Śivayogapradīpikā}, (recensions of the \textit{Haṭhapradīpikā}), \textit{Yogasvarodaya}, \textit{Nityanāthapaddhati}, \textit{Siddhasiddhāntapaddhati}, \textit{Yogacūḍāmaṇyupaniṣad}, \textit{Maṇḍalabrāhmaṇopaniṣat}, \textit{Haṭhatattvakaumudi} and \textit{Haṭhasaṃketacandrikā}.} However, it was not labelled as an independent Yoga category until the texts of the complex late medieval Yoga taxonomies emerged. In the fifteen-fold Yoga taxonomy of \textit{Yogatattvabindu}, Lakṣyayoga is listed as the ninth method of Rājayoga. The \textit{Yogasvarodaya} does not mention Lakṣyayoga in its introductory verses. The \textit{Yogasvarodaya} dedicates two verses to listing the fifteen Yogas. Although the verses announce fifteen Yogas, only eight Yogas are specified, probably for metrical reasons. Lakṣyayoga is not among the eight Yogas mentioned but is dealt with in detail throughout the text. In the \textit{Yogasiddhāntacandrikā}, Lakṣyayoga is the eighth Yoga method Nārāyaṇatīrtha mentions.\footnote{For an earlier discussion of \textit{Lakṣyayoga} in the \textit{Yogasiddhāntacandrikā}, see \citeauthor{penna2004} 2004, pp. 77-78.} Within the \textit{Sarvāṅgayogapradīpikā} Sundardās presents Lakṣayoga\footnote{The terms vary in the literature. The most common term is \textit{lakṣya}, but \textit{lakṣa} or \textit{lakṣana} were also commonly specified.} as one of the four methods of Haṭhayoga alongside Rāja- and Aṣṭāṅgayoga..\footnote{See \citeauthor{burger2014sarvangayogapradipika} 2014, pp. 697-98 for another discussion of Lakṣayoga in the \textit{Sarvāṅgayogapradīpikā} in French.} In contrast to the Yoga categories discussed so far, Lakṣyayoga is conceptually largely congruent within the late medieval texts of the complex Yoga taxonomies and differs only in a few details.

\subsection{Lakṣyayoga in the \textit{Yogatattvabindu}, \textit{Yogasvarodaya} and \textit{Sarvāṅgayogapradīpikā}}

The three texts present Lakṣyayoga as a simple Yoga method right at the beginning of their respective discourses. The descriptions of the texts are very similar. A separate analysis of them separately, as in the previous chapters, would be redundant. The word \textit{lakṣya} means ‘goal’. In the practice of Lakṣyayoga, it refers to goals on which the gaze (\textit{dṛṣṭi}) and the mind are directed, i.e. a ‘focus’ for stabilising the mind on which one constantly meditates. The three texts distinguish five categories from one another, depending on the place to be focussed. The following order\footnote{The order in the \textit{Sarvāṅgayogapradīpikā} is not identical, but as follows: 1. \textit{adho lakṣa}, 2. \textit{ūrddha lakṣa}, 3. \textit{madhya lakṣa}, 4. \textit{bāhya lakṣa} and 5. \textit{aṃtar lakṣa}.} is given in the \textit{Yogatattvabindu} and \textit{Yogasvarodaya}: 1. the upper focus (\textit{ūrdhvalakṣya}), 2. the lower focus (\textit{adholakṣya}), 3. the outer focus (\textit{bāhyalakṣya}), 4. the middle focus (\textit{madhyalakṣya}) and 5. the inner focus (\textit{antar(a)lakṣya}).\footnote{Only in \textit{Yogatattvabindu} is this \textit{lakṣya} is designated as \textit{antaralakṣya}. In all other texts, including the \textit{Haṭhasaṃketacandrikā}, which quotes the \textit{Yogatattvabindu}, the term \textit{antarlakṣya} is used.}\footnote{In the \textit{Yogatattvabindu} section \uproman{13}, in the \textit{Yogasvarodaya} (PT) ed. p. 833-34 and \textit{Sarvāṅgayogapradīpikā} 3.25-36.} Meditation on particular foci produces specific results.

\subsubsection{Ūrdhvalakṣya}
The upper focus (\textit{ūrdhvalakṣya})\footnote{\emph{Yogatattvabindu} \uproman{15}, \emph{Yogasvarodaya} PT p. 834 and \emph{Yogakarṇikā} 2.5.} refers to the fixation of the gaze (\textit{dṛṣṭi}) and the mind (\textit{manas}) on the centre of the sky, or the zenith (\textit{ākāśamadhye}). This results in the unity of the gaze with the splendour of the Supreme God (\textit{parameśvara}). In addition, an object arises in the sky within the practitioner’s scope of vision, an object that was previously unseen.\footnote{Cf. \textit{Yogatattvabindu} \uproman{14} (Ed. p. \pageref{urdhvalaksya}): \textit{etasya lakṣyasya dṛḍhīkaraṇāt parameśvarasya tejasā saha dṛṣṭairkyaṃ bhavati} | \textit{atha cākāśamadhye yaḥ kaścid adṛṣṭaḥ padārtho bhavati} | \textit{sa sādhakasya dṛṣṭigocare bhavati} |} The latter effect is cryptic. The source text, the \textit{Yogasvarodaya}, also does not contribute to clarity in this case, as there is no parallel passage. The \textit{Haṭhasaṃketacandrikā}\footnote{\textit{Haṭhasaṃketacandrikā} 2244 fol. 124v ll. 1-2.} quotes this passage literally, without further explanation. The only clue I found is in the description of \textit{ūrddha lakṣa} in \textit{Sarvāṅgayogapradīpikā} 3.27. The technique described here is identical. Here, the practitioner shall focus the gaze on the sky day and night. Sundardās explains the effect resulting from the practice in similar terms.\footnote{\textit{Sarvāṅgayogapradīpikā} 3.27: \textit{ūrddha lakṣa karai ihīṃ bhāṃtī} | \textit{duṣṭy ākāśa rahai dina rātī} | \textit{bibidh prakāra hoi ujiyārā} | \textit{gopi padāratha dīsahiṃ sārā} || 27 ||} In 3.27cd Sundardās states: `Various kinds of splendour manifest, the essence of the Gopīs’ object of consideration becomes visible’. Due to the striking similarity of the formulations and the fact that Sundardās must have been a contemporary of Rāmacandra, a correlation is probable. Sundardās was a disciple of Dādu Dayāl (1544-1603) and a member of the school named after him, and therefore a Vaiṣṇava, so the phrase ‘the essence of the object of the Gopīs’ consideration’ is probably the essence of Krṣṇa. Gopīs are paradigmatic figures of devotion (\textit{bhakti}) to Kṛṣṇa.\footnote{See e.g. \citetitle{bhagavata} 10.29.} Undoubtedly, the object of contemplation of the Gopīs must be Kṛṣṇa. Since Kṛṣṇa is considered the eighth \textit{avātara} of Viṣṇu, the essence or being of Kṛṣṇa is probably Viṣṇu, who is sometimes called \textit{puruṣottama} or \textit{parameśvara}. Whether the \textit{adṛṣṭaḥ padārthaḥ} of Rāmacandra is the same as the \textit{gopi padāratha} is uncertain, but the parallels to the wording of the \textit{sarvāṅgayogapradīpikā} are striking. Rāmacandra does not seem to favour any sectarian affiliation, and despite the clear Śaiva orientation of the main source text of his compilation, he is remarkably neutral in his formulations. Here, once more, he maintains his neutrality.\\\ 

\subsubsection{Adholakṣya}
The lower focus (\textit{adholakṣya}) of Rāmacandra is the stabilisation of the gaze (\textit{dṛṣṭi}) at a distance of twelve fingers' breadth from the tip of the nose or on the tip of the nose itself. The technique stabilises the \textit{dṛṣṭi}, the breath and prolongs life.\footnote{Cf. \emph{Yogasvarodaya} (PT): \textit{nāsikopari deveśi dvādaśāṅgulamānataḥ} \textit{dṛṣṭiḥ sthirā} (\textit{dṛṣṭisthiran} YK 2.5) \textit{tu karttavyā} (\textit{karttavyam} YK 2.5) \textit{adholakṣam idaṃ bhaja} (\textit{bhajet} YK 2. 5) | \textit{athavā} (\textit{tathā ca} YK 2.5) \textit{nāsikāgre tu sthirā dṛṣṭir iyaṃ bhavet} (\textit{śṛṇu} YK 2. 5) \textit{sthirā dṛṣṭiś cirāyuḥ syāt tathāsau} (\textit{yasya bhavet sthirā dṛṣṭiś cirāyuḥ} YK 2. 6) \textit{sthiradṛṣṭimān} |}\footnote{Rāmacandra, in contrast to \textit{Yogasvarodaya}, notes himself at this point that both options are taught as techniques of external focus (\textit{bāhyalakṣya}). The difference for Rāmacandra appears to be not only the designation but, above all, the subsequent focussing on \textit{śūnya}.} Afterwards, the practitioner is supposed to focus inwardly and outwardly on emptiness (\textit{śūnya}), which leads to freedom from the fear of death (\textit{maraṇatrāsa}).\footnote{Rāmacandra reduces and massively changes his source text. See edition \uproman{15} Ed. p. \pageref{adholaksya}. Rāmacandra's \textit{adholakṣya} on \textit{śūnya} is attributed to \textit{antarlakṣya} in the \textit{Yogasvarodaya}. For a translation of the passage, see the subchapter on \textit{antar(a)lakṣya} on p.\pageref{antarsvayotrans}.} 
\label{samketaadd1}
Sundaradeva, in his \citetitle{hathasamketacandrikachennai},\footnote{The collation of the passages of the \citetitle{hathasamketacandrikachennai} I based on ORI B 220 (f.239 r l.8 - f. 240r l.13), GOML R 3239 (f. 258 l.14 - f. 259 l.10) and HSC 2244 (HSC 2244 f. 124r ll. 5-9 - f. 125r ll. 1-2).} quotes the \textit{Yogatattvabindu} without attribution. He adds the following alternative techniques to his description of \textit{adholakṣya}: 
%\begin{quote}
%  \textit{athavā dṛṣṭir netrayor dvayor netrādhobhāgayor akṣikūṭayos tad adhogallayor ūbhayor upari sthirā kartavyā} | \textit{ekānte vijane dīpam āvarake saṃsthāpya ciraṃ gatvāvalokya stheyaṃ} | \textit{ghaṭīmātraṃ vā ghaṭikārdhaṃ vā tato dīpam ācchādya bhūmau sarvatrāvalokane sarvaṃ śvetanīlapītasphuliṅgakaṇāṃ 'te maṇḍalākāriṇiś ceta jyotiścakrāṇi pañcaṣaṭ vā dṛśyante} | \textit{tataś cāndhakāre dṛśyate} | \textit{dīptamatsarvaṃ svaśarīraṃ dṛśyate bhāsate sarvo 'pi sapradeśo dīptimān sphuṭo dṛśyate} | \textit{etad ārḍye jyotirmayacakrāṃte parameśvarasya tejomūrtir dṛśyate} | \textit{puṃsaḥ paramānandotpattir jāyate} | \textit{svadehavismṛtiś ca saṃbhavati} |
%\end{quote}
\begin{itquote}
  \begin{ekdosis}
    \note[type=witnesses, labelb=_intro1b, labele=_intro1e, nosep]{J = Jodhpur MS. No. 2244; C = Chennai GOML Ms. No. R 3239; C\textsubscript{pc} = Ibid. \textit{post correctionem}; M = Mysore ORI Ms. No. B 220.}
\linelabel{_intro1b}
% ----------------------
% athavā dṛṣṭir netrayor dvayo  netrā 'dhobhāgayor akṣikūṭayos tad adhogallayo rūpayor upari sthirā kartavyā (||) \J
% athavā dṛṣṭi  netrayor dvayor netrādhobhāgayor   akṣikūṭayos tad adhogallayo rūpa     pari sthirā kartavyā  ||  \M
% athavā dṛṣṭi  netrayor dvayor netrādhobhāgayor   akṣikūṭayos tad adhogallayo rūpayor upari sthirā kartavyā      \C
% athavā dṛṣṭi  netrayor dvayor netrādhobhāgayor   akṣikūṭayos tad adhogallayo ūbhayor upari sthirā kartavyā      \Cpc
% ----------------------
  athavā
  \app{\lem[wit={J}]{dṛṣṭir}
    \rdg[wit={C,Cpc,M}]{dṛṣṭi}} netrayor
  \app{\lem[wit={C,Cpc,M}]{dvayor}
    \rdg[wit={J}]{dvayo}}
  \app{\lem[wit={C,Cpc,M}]{netrādhobhāgayor}
    \rdg[wit={J}]{netrā 'dhobhāgayor}}
  akṣikūṭayos tad adhogallayo
  \app{\lem[wit={Cpc}]{ūbhayor}
    \rdg[wit={C,J}]{rūpayor}
    \rdg[wit={M}]{rūpa}}
  \app{\lem[wit={C,Cpc,J}]{upari}
    \rdg[wit={M}]{pari}}
 sthirā kartavyā \normalpipe  
% ----------------------  
% ekāṃte vijane dīpam āvarake saṃsthāpya ciraṃ gatvāvalokyastheyaṃ (||)  \J
% ekāṃte vijane dīpam āvake   saṃsthāpya ciraṃ gatvāvālokyastheyaṃ  ||  \M
% ekānte vijane dīpam āvake   saṃsthāpya ciraṃ gatvāvalokyastheyaṃ --       \C
% ----------------------
 ekānte vijane dīpam
 \app{\lem[wit={J}]{āvarake}
   \rdg[wit={C,Cpc, M}]{āvake}}
saṃsthāpya ciraṃ gatvāvalokyastheyaṃ \normalpipe  
% ---------------------- 
% ghaṭīmātra  vā ghaṭikārdhaṃ vā tato dīpam ācchādya bhūmau sarvatrāvalokane sarvaṃ śvetanīlapīta--sphuliṃgakaṇāṃ 'te maṃḍalākāriṇiś ceta jyotiśrakrāṇi paṃcaṣaṭ vā dṛśyaṃte (||) \J
% ghaṭīmātraṃ vā ghaṭikārdhaṃ vā tato dīpam ācchādya bhūmau sarvatrāvalokane sarvaṃ śvetanīlayoṃta-sphuliṃgakaṇāṃ te  maṃḍalākāriṇiś ceti jyotiśrakrāṇi paṃcaṣaṭ vā dṛśyaṃte  ||  \M
% ghaṭīmātraṃ vā ghaṭikārdhaṃ vā tato dīpam ācchādya bhūmau sarvatrāvalokane sarvaṃ śvetanīlayomta-sphuliṃgakaṇān te  maṇḍalākāriṇiś ceti jyotiścakrāṇi paṃcaṣad vā dṛśyaṃte ||   \C
% ----------------------
\app{\lem[wit={C,Cpc,M}]{ghaṭīmātraṃ}
  \rdg[wit={J}]{ghaṭīmātra}}
vā ghaṭikārdhaṃ vā tato dīpam ācchādya bhūmau sarvatrāvalokane sarvaṃ
śvetanīla\app{\lem[wit={J},alt={°pīta°}]{pīta}
  \rdg[wit={M}]{yoṃta}
  \rdg[wit={C,Cpc}]{yomta}}
sphuliṅgakaṇāṃ 'te maṇḍalākāriṇiś
\app{\lem[wit={C,Cpc,M}]{ceti}
  \rdg[wit={J}]{ceta}} 
 jyotiścakrāṇi pañcaṣad vā dṛśyante \normalpipe  
% ----------------------
% tataś cāṃdhakāre dṛśyate (||) \J
% tataś vāṃdhakāre dṛśyate || \M
% tataś cāndhakāre dṛśyate --- \C
% ----------------------
 tataś
 \app{\lem[wit={C,Cpc,J}]{cāṃdhakāre}
   \rdg[wit={M}]{vāṃdhakāre}}
 dṛśyate \normalpipe  
% ----------------------
% dīptamatsarvaṃ svaśarīraṃ dṛśyate bhāsate sarvo pi sapradeśo dīptimān sphuṭo dṛśyate (||) \J
% dīptamatsarvaṃ svaśarīraṃ dṛśyate bhāsate sarvo pi sapradeśo dīptimān sphuṭo dṛśyate || \M
% dīptimatsarvaṃ svaśarīraṃ dṛśyate bhāsate sarvo pi sapradeśo dīptimān sphuṭo dṛśyate -- \C
% ----------------------
 dīptimatsarvaṃ svaśarīraṃ dṛśyate bhāsate sarvo 'pi sapradeśo dīptimān sphuṭo dṛśyate \normalpipe  
% ----------------------
% etadārḍye jyotirmayacakrāṃte parameśvarasya tejomūrtir dṛśyate || \J
% etadārḍye jyotirmayacakrāṃte parameśvarasya tejomūrtir dṛśyate || \M
% ekadārḍye jyotirmayacakrāṃte parameśvarasya tejomūrtir dṛśyate \C
% ----------------------
 ekadārḍye jyotirmayacakrāṃte parameśvarasya tejomūrtir dṛśyate \normalpipe  
% ---------------------- 
% puṃsaḥ paramānaṃdotpattir jāyate (||) \J
% puṃsaḥ paramānaṃdotpattir jāyate|| \ M
% puṃsaḥ paramānandotpattir jāyate -- \C
% ----------------------
puṃsaḥ paramānandotpattir jāyate \normalpipe  
% ----------------------
% svadehavismṛtiś ca saṃbhavati (||) \J
% svadehavismṛtiś ca saṃbhavati || \M
% svadehavismṛtiś ca saṃbhavati -- \C
% ----------------------
svadehavismṛtiś ca
\app{\lem[wit={C,Cpc,M}]{saṃbhavati}
  \rdg[wit={J}]{saṃbhavati | athavā svanetrayor vartmanīr dakṣahastamadhyamātarjanībhyām akṣikū dehavismṛtiś ca saṃbhavati |}} \normalpipe   
% ----------------------
% athavā svanetrayor vartmanīr dakṣahastamadhyamātarjanībhyām akṣikū dehavismṛtiś ca saṃbhavati (|)
% \om
% \om
% ----------------------
\linelabel{_intro1e}
\end{ekdosis}
\end{itquote}
\begin{quote}
 Alternatively, the gaze should be fixed without wavering on both lower parts of the corners of the two eyes, below the cheekbones. In a lonely place without people, a lamp shall be placed in the darkness and observed for a long time. After one \textit{ghaṭikā} (24 minutes) or half a \textit{ghaṭikā} (12 minutes) [already], cover the lamp and then gaze all around on the ground; one may see all white, blue, and yellow sparkles forming circular patterns, and perhaps even fifty-six such circles of light become visible. As a consequence, one can see in the dark. One’s own body is seen illuminated. Also, the entire place lights up [and] is seen brightly and clearly. In this phase, within the circle of light, the luminous form of the supreme lord is seen. The generation of supreme bliss arises for the person. Forgetting of one’s own body occurs.
\end{quote} 
%\begin{quote}
%  \textit{athavā svanetrayor vartmanīr dakṣahastamadhyamātarjanībhyām akṣikūṭayor adhaḥ kṛtvā akṣivartmani dṛḍhaṃ cālanī ye ghaṭikārdhaṃ cā ghaṭīmātraṃ tata evaṃ kṛte sādhyakasyāgre suśvetajyotiḥ prākāśaḥ prāg bhavatīti} |
%\end{quote}
\begin{itquote}
  \begin{ekdosis}
    \note[type=witnesses, labelb=_intro2b, labele=_intro2e, nosep]{J = Jodhpur MS. No. 2244; C = Chennai GOML Ms. No. R 3239; C\textsubscript{pc} = Ibid. \textit{post correctionem}; M = Mysore ORI Ms. No. B 220.}
\linelabel{_intro2b}
% ----------------------
% athavā  svanetrayor  vartamanīr dakṣahastamadhyamatarjanibhyāṃ akṣikūtvā                    akṣivanmanī dṛdhaṃ cālanī ye ghaṭikārdhaṃ vā ghaṭīmātraṃ tatta evaṃ kṛte sādhyakasyāgre suśvetajyotiḥ prākāśaḥ prāgvad bhavatīti  \J
% athavā  svanetrayor  vartmanā   dakṣahastamadhyamātarjanībhyām ākṣikoṭayor     adhaḥ kṛtvā akṣivartmanī dṛḍhaṃ cālanī ye ghaṭikārdhaṃ vā ghaṭīmātraṃ tata  evaṃ kṛte sādhyakasyāgre suśvītajyotiḥ prākāśaḥ prāg    bhavatīti \M
% athavā  svanetrayor  vartmanā   dakṣahastamadhyamātarjanībhyām akṣikūṭakūṭayor adhaḥ kṛtvā akṣivartmani dṛḍhaṃ cālanī ye ghaṭikārdhaṃ cā ghaṭīmātraṃ tata  evaṃ kṛte sādhyakasyāgre suśvetajyotiḥ prākāśaḥ prāg    bhavatīti \C
% ----------------------
athavā svanetrayor
\app{\lem[wit={J}]{vartamanīr}
  \rdg[wit={C,Cpc,M}]{vartmanā}}
dakṣahastamadhyamātarjanībhyām
\app{\lem[type=emendation, resp=egoscr]{akṣikuṭayor}
  \rdg[wit={M}]{ākṣikoṭayor}
  \rdg[wit={C,Cpc}]{akṣikūṭakūṭayor}
  \rdg[wit={J}]{akṣikūtvā}}
\app{\lem[wit={C,Cpc,M}]{adhaḥ kṛtvā}
  \rdg[wit={J}]{\om}}
\app{\lem[wit={C,Cpc,M}]{akṣivartmanī}
  \rdg[wit={J}]{akṣivanmanī}} 
dṛḍhaṃ cālanī ye ghaṭikārdhaṃ vā ghaṭīmātraṃ tata  evaṃ kṛte sādhyakasyāgre suśvītajyotiḥ prākāśaḥ
\app{\lem[wit={C,Cpc,M}]{prāg}
  \rdg[wit={J}]{prāgvad}}
bhavatīti \normalpipe
\linelabel{_intro2e}
\end{ekdosis}
\end{itquote}

\begin{quote}
Alternatively, having placed the thumb and index finger of the right hand below the edge of the eye socket at the eyelids of the own eyes, and steadily causing to move [the fingers] at the eyelids, either for a half \textit{ghaṭikā} (12 minutes) or for a \textit{ghaṭikā} (24 minutes), as a result of having done this, very highly bright white light becomes visible in front of the practitioner.
\end{quote}

Sundardā's \textit{adho lakṣa} is the simple focusing of the gaze on the tip of the nose, which leads to the stabilisation of breath and mind.\footnote{\textit{Sarvāṅgayogapradīpikā} 2.26: \textit{prathamahīṃ adho lakṣa kauṃ jānaiṃ} | \textit{nāśā agra dṛṣṭi sthira ānaiṃ} | \textit{yātoṃ mana pavanā thira hoī} | \textit{adho lakṣa jo sādhai koī} || 26 ||}\\

\subsubsection{Bāhyalakṣya}
The external focus (\textit{bāhyalaksya})\footnote{\textit{Yogatattvabindu} \uproman{23}; \textit{Yogasvarodaya} (PT Ed. p.837).} is the fixation of the gaze (\textit{dṛṣṭi}) on one of the five gross elements at different distances from the tip of the nose or, in one case, directly on the tip of the nose. The texts present the foci as alternatives. The presentation of the three texts follows the same pattern in every case. They list a specific location, followed by an element (in most cases) and a characteristic, such as an associated colour. A table is the best way to illustrate the spread of the various techniques across the texts.
\newpage 
\begin{landscape}
\footnotesize  
\begin{longtable}{|m{3cm}|m{1cm}|m{3cm}|m{1.5cm}|m{1.7cm}|m{1.5cm}|m{1.5cm}|}
    \caption{Foci of Bāhyalakṣya}\\
    \hline
    \textbf{Location} & \textbf{Element} & \textbf{Characteristic} & \textbf{\textit{Yogatattvabindu}} & \textbf{{Yogasvarodaya}} & \textbf{\textit{Haṭhasaṃketacadrikā}} & \textbf{\textit{Sarvāṅgayogapradīpikā}} \\ 
    \hline
    \endfirsthead
    
    \multicolumn{6}{c}%
    {{\tablename\ \thetable{} -- continued from previous page}} \\
    \hline
    \textbf{Distance} & \textbf{Location} & \textbf{Characteristic} & \textbf{\textit{Yogatattvabindu}} & \textbf{\textit{Yogasvarodaya}} & \textbf{\textit{Haṭhasaṃketacadrikā}} & \textbf{\textit{Sarvāṅgayogapradīpikā}}\\ 
    \hline
    \endhead
    
    \hline
    \multicolumn{6}{|r|}{{Continued on next page}} \\ \hline
    \endfoot
    
    \hline \hline
    \endlastfoot

    Four finger breadths from the nose & Space & Appearing blue, full of splendour & X & X (Element missing) & X (Element = Wind; Characteristic= In the shape of smoke)\footnote{Possibly the text is corrupt and merged the first and second focus.} & X\\ 
    \hline
    Six finger breadths from the nose & Wind & In the shape of smoke & X & X & - & X\\ 
    \hline
    Eight finger breadths from the nose & Fire & Very red & X & X & X & X \\ 
    \hline
    Ten finger breadths from the nose & Water & White, fickle & X & -  & - & X \\ 
    \hline
    Twelve finger breadths from the nose & Earth & Yellow-coloured & X & - & - & X \\ 
    \hline
    At the tip of the nose & Space & Full of fire, shining like ten million suns & X & - & - & - \\ 
    \hline
    Above the space-element & Space & Connected to the sun without the sun (thousand rays) & X & - & - & - \\ 
    \hline
    Seventeen-finger wide distance above the head & Light & Mass of light & X & X & - & - \\ 
    \hline
    In front of the gaze & Earth & Appearing in the colour of molten gold & X & X & - & - \\
    \hline
\end{longtable}
 \end{landscape}
\normalsize

 The table shows that the \textit{Yogatattvabindu} contains the greatest variety of foci of the \textit{bāhyalakṣya} category. Sundaradeva does not adopt all the foci in his \textit{Yogasaṃketacandrikā}. However, the text appears rather corrupt, as the text mixes up the first two foci. The \textit{Yogasvarodaya} only contains five of the nine foci in the table. Rāmacandra has added further foci based on the explanations of Bahirlakṣya in the \textit{Siddhasiddhāntapaddhati} 2.28 (ed. 38-40).\footnote{The \textit{Siddhasiddhāntapaddhati} teaches only three instead of five Lakṣyas: \textit{antarlakṣya} (2.26-27); \textit{bahiryalakṣya} (2.28); and \textit{madhyalakṣya} (2.29).} Sundardās describes the first five foci for the five elements in a perfectly analogous fashion.\footnote{Cf. \textit{Sarvāṅgayogapradīpikā} 2.29-31.} In the last verse of his explanation of \textit{bāhya lakṣa}, he explains that there are many more \textit{bāhya lakṣa}s, but they must be revealed by the Guru.\footnote{Cf. Ibid. 2.32: \textit{bāhya lakṣa aur bahuterī} \textit{so jānaṃ jo pāvai serī} | \textit{sataguru kṛpā karai jau kabahī} | \textit{dei batāi chinak maiṃ sabahī} || 32 ||}
The effects attributed to the practice of \textit{bāhyalakṣya} are similar throughout the texts. Regardless of the variant practised, the practice promises rejuvenation, improved health, but moreover an improved social life\footnote{\textit{Yogatattvabindu} \uproman{23}: \textit{samagrāḥ śatravaḥ svapne ‘pi mitratām ayānti} |} and a longer life span etc. 

\subsubsection{Antar(a)lakṣya}
\label{antaralaksya}
The inner focus (\textit{antar(a)lakṣya}) is a special case, as there are noticeable deviations between Rāmacandra’s \textit{Yogatattvabindu} and the \textit{Yogasvarodaya}. Although Rāmacandra continues to follow the \textit{Yogasvarodaya} in terms of structure and content for the description of his \textit{antar(a)lakṣya}, the passages in the \textit{Yogasvarodaya} are not explicitly attributed to \textit{antaralakṣya}, but are evidently assigned to the preceding \textit{bāhyalakṣya}.\footnote{Cf. \textit{Yogatattvabindu} \uproman{24} and \textit{Yogasvarodaya} (PT Ed. pp. 837-38).} In addition, Rāmacandra simultaneously uses the \textit{Siddhasiddhāntapaddhati} (2.26-27) as a template for this passage, which attributes largely similar practices to the category of \textit{antar(a)lakṣya}. In the \textit{Yogasvarodaya}, there is a separate description of \textit{antarlakṣya}, the core practice of which was already integrated by Rāmacandra in the context of his \textit{adholakṣya}.\footnote{This is the meditation on emptiness (\textit{śūnya}). Cf. \textit{Yogatattvabindu} \uproman{15} and \textit{Yogasvarodaya} (PT Ed. p. 834).} 
The concept of the \textit{antar lakṣa} of Sundardās is essentially identical.

In the \uproman{24} section of the \textit{Yogatattvabindu}, Rāmacandra specifies a total of three alternative \textit{antar(a)lakṣya}s. 
As part of the explanations of the first \textit{antar(a)lakṣya}, Rāmacandra first presents a description of the central channel in the yogic body, which is labelled here as \textit{brahmanāḍī}. It originates from the spine (\textit{brahmadaṇḍa}) and passes through the spine from bottom to top. The central channel extends from the root bulb (\textit{mūlakanda}) to the opening of Brahman (\textit{brahmarandhra}) at the top of the head. It is shaped like the stem of a lotus flower and shines like ten million suns. The practice of \textit{antar(a)lakṣya} consists of meditating on it, which allows the practitioner to acquire supernatural abilities. Just the first of the three techniques appears in the context of \textit{antar lakṣa} in the \textit{Sarvāṅgayogapradīpikā} of Sundardās, albeit in less detail. According to Sundardās, one is supposed to meditate on the central channel known as Brahmanāḍī, which leads to the eight supernatural faculties.\footnote{Cf. \textit{Sarvāṅgayogapradīpikā} 3.33: \textit{aṃtar lakṣa ju sunahuṃ prakāśā} | \textit{brahma nāḍikā karahu abhyāsā} | \textit{aṣṭa siddhi nava niddhi jahāṃlauṃ} | \textit{ṭarahiṃ na kabahūṃ jivai jahāṃ lauṃ} || 33 ||}.
Rāmacandra’s second technique for the practice of \textit{antaralakṣya} is a meditation on a bright light above the forehead, preventing certain diseases.
The third alternative for the practice of \textit{antaralakṣya} is meditation on the very fine red light in the centre between the eyebrows, which causes the yogin to be loved by everyone in the royal court and ensures that no one can take their eyes off him. \footnote{All three techniques of \textit{antar(a)lakṣya} are also specified in the \textit{Yogasvarodaya} (PT Ed. p. 837-28), but still in the context of \textit{bāhyalakṣya}: \textit{mūlakandotthatalato brahmanāḍīsamudbhavā} | \textit{śvetavarṇā brahmarandhraparyantam eva tiṣṭhati} | \textit{eṣā tu brahmarandhrākhyā tanmadhye varttate parā} | \textit{padmatantusamākārā koṭisūryataḍitprabhā} | \textit{calaty ūrddhaṃ mahāmūrttir asya dhyānād bhavec chivaḥ} | \textit{aṇimādy aṣṭasiddhis tu samagreṇa prasīdati} | \textit{lalāṭopari vā dhyātvā candraṃ vā jyotir īśvaram} | \textit{nāśayet kuṣṭharogādīn mahāyuṣmān śivaḥ paraḥ25↩} | \textit{bhruvor madhye’ thavā dhyātvā arkantu teja īśvaram} | \textit{sthiradṛṣṭau rājapūjyo jīvanmuktaḥ śivo yathā} | \textit{ātmānam ātmarūpaṃ hi dhyātvā yo niṣkriyo bhavet} | \textit{nirāśīryatattvo ‘yaṃ itaro na nṛpasthitiḥ} |}  

The \textit{antar(a)lakṣya} of the \textit{Yogasvarodaya},\footnote{\textit{Yogasvarodaya} (PT Ed. p. 824) and \textit{Yogakarṇikā} 2.8-13.} the \textit{Yogatattvabindu}, \textit{Sarvāṅgayogapradīpikā}, and \textit{Siddhasiddhāntapaddhati} differs greatly from the models in \textit{Yogatattvabindu}, \textit{Sarvāṅgayogapradīpikā}, and \textit{Siddhasiddhāntapaddhati}. It is exclusively about meditation on emptiness (\textit{śūnya}): 

\label{antarsvayotrans}
  \begin{quote}
    \begin{ekdosis}
      \note[type=witnesses, labelb=_intro3b, nosep]{PT= \textit{Prāṇatoṣiṇī} quotes \textit{Yogavarodaya} with reference \textit{yogasvarodaye}. YK= \textit{Yogakarṇikā} quotes \textit{Yogavarodaya} with reference \textit{yogasvarodaye}.}
      \linelabel{_intro3b}
    \textit{
  antarlakṣaṃ śṛṇu \app{\lem[wit={PT}, alt={subhru°}]{subhru}
    \rdg[wit={YK}]{śukra°}}digvidigādivarjitam} |\\
\textit{
\app{\lem[wit={YK}]{bāhyabhyantara ākāśaṃ vādhāmantraṃ paraṃ mataṃ}
  \rdg[wit={PT}]{\om}}} ||
  \end{ekdosis}
\end{quote}
\begin{quote}
Listen to the internal focus, oh lovely-browed [Goddess], being devoid of the major and minor directions, etc. The internal and external space is the magical formula against pain, the supreme view.
\end{quote}
\begin{quote}
  \begin{ekdosis}
\textit{calajjāgratsuṣupteṣu bhojaneṣu ca sarvadā} |\\
\textit{sarvāvasthāsu deveśi cittaṃ śūnye niyojayet} ||
  \end{ekdosis}
\end{quote}
\begin{quote}
While walking, waking, sleeping and eating at all times
[and] in all states, oh Goddess, the mind shall be focussed onto emptiness.  
\end{quote}
\begin{quote}
  \begin{ekdosis}
    \textit{karttā kārayitā \app{\lem[wit={YK}]{śūnyaṃ}
        \rdg[wit={PT}]{śunyaḥ}}mūrtimān śūnya īśvaraḥ} |\\
    \textit{harṣaśokaghaṭastho ’yaṃ janmamṛtyū labhet svayam} ||
      \end{ekdosis}
\end{quote}
\begin{quote}
The actor and he who causes to act are void; the form-bearer in the void is the supreme lord.
Situated in a vessel of joy and sorrow, he himself experiences both birth and death. 
\end{quote}
\begin{quote}
  \begin{ekdosis}
\textit{\app{\lem[wit={YK}]{ghaṭasthāṃ}
    \rdg[wit={PT}]{ghaṭasthā}}
\app{\lem[wit={YK}]{cintayen}
  \rdg[wit={PT}]{cintyayor}}
\app{\lem[wit={YK}]{mūrttimitaś}
  \rdg[wit={PT}]{mūrtir hata°}}cintāsvarūpadhṛk} |\\
\textit{viṣayaṃ viṣavad \app{\lem[wit={YK}]{dṛṣṭvā}
    \rdg[wit={PT}]{duṣṭaṃ}} tyaktvā jñātvā tu mārutam} ||
      \end{ekdosis}
    \end{quote}
    \begin{quote}
He shall contemplate [himself as] being situated in a vessel, established as form [and] carrying the nature of thought. 
Having abandoned sense objects as defective like poison, having realized them as consisting of the Maruts, \ldots 
\end{quote}
\begin{quote}
  \begin{ekdosis}
\textit{saṃjñāśūnyamanā bhūtvā puṇyapāpair na lipyate} |\\
\textit{bāhyam ābhyantaraṃ
\app{\lem[wit={PT}]{khaṃ}
  \rdg[wit={YK}]{\om}}
\app{\lem[type=emendation, resp=egoscr]{yad}
  \rdg[wit={YK}]{yad hi}
  \rdg[wit={PT}]{hi}} antarlakṣam iti smṛtam} ||
     \end{ekdosis}
    \end{quote}
\begin{quote}
\ldots having become aware of the emptiness of conception, he is not tainted by merits or sin.
That which is the inner and outer space is taught as the internal focus.
\end{quote}
\begin{quote}
  \begin{ekdosis}
\textit{etad dhyānāt sadā kiñcid duḥkhaṃ na syāc chivo bhavet} |\\
\textit{śūnyan tu saccidānandaṃ niḥśabdaṃ brahmaśabditam} |\\
\textit{saśabdaṃ jñeyam
\app{\lem[wit={PT}]{ākāśam}
  \rdg[wit={YK}]{ākāśa}}iti bhedadvayan tv iha} ||
   \end{ekdosis}
 \end{quote}
    \begin{quote}
Because of this meditation, any kind of suffering will no longer arise [and] one would become Śiva.
Emptiness is being-consciousness-bliss, [and] called the soundless Brahman;
space [on the other hand] is to be understood as with sound. Indeed, this is the twofold distinction in this world.  
\end{quote}

\subsubsection{Madhyalakṣya}

The concept of the central focus (\textit{madhyalakṣya}) is very similar in all three texts. In the \textit{Yogatattvabindu}\footnote{see \textit{Yogatattvabindu} \uproman{27}, Ed. p. \pageref{madhyalaksya}.}, a light is visualised by the mind. The light is supposed to be the size of one's own body. Like a room on fire, this body shall be envisioned as filled with light. The light shall be white, yellow, red, grey or blue. The envisioned light is compared to the light of the sun, lightning or a crescent moon. \textit{Madhyalakṣya} leads to the burning of the impurities of the mind. It also produces the sattvic quality of the mind. The practitioner becomes blissful. Rāmacandra remains very close to his original text regarding the choice of terminology and the content. Thus, there is no significant conceptual difference in comparison with the \textit{madhyalakṣya} of the \textit{Yogasvarodaya}.\footnote{Cf. \textit{Yogasvarodaya} (Ed. p. 839): \textit{idānīṃ madhyalakṣantu kathyate siddhikārakam} | \textit{śvetaṃ raktaṃ tathā pītaṃ dhūmrākārantu nīlabham} | \textit{agnijvālāsamānābhā vidyutpuñjasamaprabhā} | \textit{ādityamaṇḍalākāramathavā candramaṇḍalam} | \textit{jvaladākāśatulyaṃ vā bhāvayed rūpamātmanaḥ} | \textit{etaj jyotirmayaṃ dehaṃ manomadhye tu lakṣayet} | \textit{eteṣāñ ca kṛte lakṣe nānāduḥkhaṃ praṇaśyati} | \textit{manas astu malo yāti mahānando bhavet tataḥ} |} Sundardā's descriptions in the \textit{Sarvāṅgayogapradīpikā} are shorter, but equally similar. The mind is supposed to dwell in its centre and focus on the form of the body. The practice brings about the sattvic quality of the mind. However, Sundardās does not specify any visualisation of a light.\footnote{Cf. \textit{Sarvāṅgayogapradīpikā} 3.28: \textit{madhya lakṣa mana madhya bicārai} | \textit{vapu pramāna koi rūpa nihārai} |\textit{yāte sātvik upajai āī} | \textit{madhya lakṣa jo sādhai bhāīī} ||)}

\subsection{Lakṣyayoga in the \textit{Yogasiddhāntacandrikā}}
\label{laksyayogaintrocandrika}

Nārāyaṇatīrtha neither divides Lakṣyayoga into five,\footnote{As in the \textit{Yogatattvabindu}, the \textit{Yogasvarodaya} or in the \textit{Sarvāṅgayogapradīpikā}. } nor in three subcategories.\footnote{As in the \textit{Siddhasiddhāntapaddhati} or the \citetitle{shivayogapradipika}.} His explanations are of a more general nature. He locates Lakṣyayoga within the framework of his commentary on \textit{Yogasūtra} 1.35.

\begin{quote}
  \textit{lakṣyayogasvarūpam upāyāntaram āha}-\\
  \textit{viṣayavatī vā pravṛttir utpannā manasaḥ sthitinibandhinī} || 35 ||
\end{quote}
\begin{quote}
It is said [there is] another method having the nature of Lakṣyayoga - \\
Alternatively, activity directed to a sense object, which is generated, causes the stopping of the mind.  
\end{quote}

Nārāyaṇatīrtha explains:

\begin{quote}
  \textit{viṣayavatīti} | \textit{nāsāgrādau cittasya saṃyamarūpāl lakṣyayogād divyagandhādisākṣātkāro bhavati} | \textit{seyaṃ viṣayavatī pravṛttir viśvāsam utpādya parameśvarādāv atisūkṣme manasaḥ sthitiṃ sampādayatīty arthaḥ} | \textit{tathā ca śāstrīyānubhavaviṣaye jāte śraddhayā yogino dhyānādau sthirā bhavatīty ayaṃ lakṣyayogaḥ} |\\

  \textit{yā hi nāsādideśeṣu dṛṣṭiḥ puṃsāṃ sthirā bhavet} |\\
  \textit{sa lakṣyayoga ākhyāto yoge śraddhākaraḥ paraḥ} ||\\
  
\textit{iti smṛter iti} || 35 ||
\end{quote}
\begin{quote}
  [Regarding the term] ``\textit{viṣayavatī}''. As a result of Lakṣyayoga, which has the nature of concentration of the mind (\textit{saṃyama}) on the tip of the nose, etc., a direct perception of divine fragrances and other objects occurs. This activity being directed to sense objects, having produced confidence, causes to generate fixedness of the mind in [something] very subtle, in [something like] the supreme Lord, etc. Such is the meaning. 
  And thus, stability in meditation, etc., arises for the yogin after the sense object from the experience of scripture has been produced with confidence. This is Lakṣyayoga.\\
  
  For indeed, when the gaze of the person becomes steady at places like the tip of the nose, etc., that is called Lakṣyayoga, which in Yoga, is considered the supreme faith-inspiring [practice].\\

  Thus, it is remembered.
  \end{quote}

  Nārāyaṇatīrtha is referring to the \textit{bhāṣya} part of the \textit{Pātañjalayogaśāstra} concerning \textit{sūtra} 1.35.\footnote{\citetitle{yogasutraed} (ed. p. 80): \textit{nāsikāgre dhārayato ‘sya yā divyagandhasaṃvit sā gandhapravṛttiḥ} | \textit{jihvāgre rasasaṃvit} | \textit{tāluni rūpasaṃvit} | \textit{jihvāmadhye sparśasaṃvit} | \textit{jihvāmūle śabdasaṃvid ity etā vṛttaya utpannāś cittaṃ sthitau nibadhnanti}, \textit{saṃśayaṃ vidhamanti}, \textit{samādhiprajñāyāṃ ca dvārībhavantīti} | \textit{etena candrādityagrahamaṇipradīparaśmyādiṣu pravṛttir utpannā viṣayavaty eva veditavyā yady api hi tattacchāstrānumānācāryopadeśair avagatam arthatattvaṃ sadbhūtam eva bhavati} | \textit{eteṣāṃ yathābhūtārthapratipādanasāmarthyāt}, \textit{tathāpi yāvad ekadeśo ‘pi kaścin na svakaraṇasaṃvedyo bhavati tāvat sarvaṃ parokṣam ivāpavargādiṣu sūkṣmeṣv artheṣu na dṛṃ buddhim utpādayati} | \textit{tasmāc chāstrānumānācācāryopadeśopodbalanārtham evāvaśyaṃ kaścid arthaviśeṣaḥ pratyakṣīkartavyaḥ} | \textit{tatra tadupadiṣṭārthaikadeśapratyakṣatve sati sarvaṃ sūkṣmaviṣayam api āpavargāc chraddhīyate} | \textit{etadartham evedaṃ cittaparikarma nirdiśyate} | \textit{aniyatāsu vṛttiṣu tadviṣayāyāṃ vaśīkārasaṃjñāyām upajātāyāṃ samarthaṃ syāt tasya tasyārthasya pratyakṣīkaraṇāyeti} | \textit{tathā ca sati śraddhāvīryasmṛtisamādhayo ‘syāpratibandhena bhaviṣyantīti} |} In the \textit{bhāṣya} part, various foci for meditation and specific effects that arise through concentration on the respective point are listed. Concentration on the tip of the nose creates absolute odour perception. Concentration on the tip of the tongue leads to absolute perception of flavour. Concentration on the palate leads to absolute perception of form. Concentration on the centre of the tongue leads to absolute perception of touch. Concentration on the root of the tongue leads to absolute perception of sound. In addition, the \textit{bhāṣya} lists the moon, sun, planets, jewels and lamps as sensory objects for focussing the mind. The resulting heightened perceptions stabilise the mind, remove doubt and are a gateway to \textit{samādhi}. Furthermore, the \textit{bhāṣya} explains that although the true nature of reality can be revealed through scriptures, inferences or instructions from teachers, these must be experienced personally, through one's own senses, so that the experience is not second-hand. Otherwise doubts occur for the practitioner. However, if these heightened perceptions referred to in this \textit{sūtra} are experienced personally, then faith, trust or confidence (\textit{śraddhā}) in the statements of the scriptures etc., the entire yogic endeavour and especially the possibility of the desired liberation is strengthened.

\section{10. Vāsanāyoga}
\label{vasanayogaintro}

Vāsanāyoga is in tenth position of the methods of Rājayoga presented at the beginning of \textit{Yogatattvabindu}. In the \textit{Yogasvarodaya}, it is in position eight. However, neither text contains a specific description of Vāsanāyoga. However, the term \textit{vāsanā} appears in several places in the texts. In the \textit{Yogasiddhāntacandrikā}, Vāsanayoga is at position twelve.\footnote{For an earlier discussion of Vāsanāyoga in the \textit{Yogasiddhāntacandrikā} see \citeauthor{penna2004} 2004, pp. 82-85.} The \textit{Sarvāṅgayogapradīpikā} does not list Vāsanayoga. The term \textit{vāsanāyoga} is scarce in the entire yoga literature and only appears in the context of late medieval yoga taxonomies. It is not found at all in the early and medieval yoga texts. The compound \textit{vāsanāyoga} appears in a few places in tantric literature but never as an independent yoga category.  

The term \textit{vāsanā} is a technical term frequently used in Indian philosophy, especially in the context of the concept of \textit{karma}. It plays a significant role in Yoga and Advaita Vedānta. Furthermore, this term is important in Buddhist philosophy. The concept of the term \textit{vāsanā} can be characterised as follows in the Yoga philosophy of Pātañjalayoga and Advaita Vedānta, which is congruent with the context of the texts discussed here. \textit{Vāsanā} denotes a certain type of karmic imprint. In the commentary literature of the \textit{Pātañjalayogaśāstra}, the term and concept of \textit{vāsanā} is closely linked to the term and concept of \textit{saṃskāra}. Both terms are often even used synonymously. However, a nuanced understanding can be expressed as follows: A \textit{saṃskāra} is a mental imprint that is left in the mind (\textit{citta}) by every action (\textit{karma}). \textit{Saṃskāra}s trigger thoughts, memories and further actions (\textit{karma}). \textit{Vāsanā}, on the other hand, refers primarily to cumulative inherent imprints (\textit{saṃskāra}s) that exert a subconscious influence on the person's personality and actions, a behavioural tendency caused by past actions. \textit{Vāsanā}s are also those \textit{saṃskāra}s that exert an influence on later rebirths or control the configuration of rebirth.\footnote{Cf. \citeauthor{bryant2009} 2009, p. 418}. Every action performed by a subject leaves an imprint or trace in the \textit{karma} storage (\textit{karmāśaya}) of the mind (\textit{citta}).

Because the mind in Pātañjalayoga is the main component of the transmigrating subtle body (\textit{sūkṣmaśarīra}), the configuration of the karma storage in the mind will determine the nature of future rebirth.\footnote{Cf. \textit{Pātañjalayogaśāstra} 4.7-11.} Literally, \textit{vāsanā} even means ``scent'' or, in this context, ``scent trail''. Metaphorically speaking, the actions leave behind a certain scent. This scent permeates the person and will continue to be felt in future actions for a long time because the accumulation of these habitual tendencies predisposes the person to certain future patterns of thought and behaviour. Thus, I think ``mental residues'' is a suitable translation. These patterns of thought and behaviour can be activated at any time, for example, triggered by sensory stimuli. In the context of a meditative Yoga practice aimed at achieving the state called \textit{samādhi} using concentration, a state characterised by a temporary standstill of mental activity, the \textit{saṃskāra}s and \textit{vāsanā}s in the yogin's mind, when activated by sensory stimuli, would repeatedly lead to newly arising mental activity and thus to distraction from this desired goal.

If these are active, most are considered a hindrance to the ultimate goal of Yoga practice and are either to be reduced or at least rendered inactive or latent. If the yogin is free from activated \textit{saṃskāra}s and \textit{vāsanā}s through Yoga practice, he can not only reach the \textit{samādhi} state, but he will also no longer be reborn. Thus he is freed from the cycle of rebirth (\textit{saṃsāra}). It is important to emphasise that there are other highly positive \textit{saṃskāra}s and \textit{vāsanā}s that favour the practice of Yoga, such as the habit of regular Yoga practice (\textit{yogābhyāsa}) itself or good eating habits. However, all positive \textit{saṃskāra}s and \textit{vāsanā}s must be rendered inactive, for the final state of Yoga of \textit{Pātañjalayogaśāstra}, the \textit{asaṃprajñātasamādhi}. \footnote{See \textit{Pātañjalayogaśāstra} 1.18, 1.50-51 and \citeauthor{bryant2009} 2009, p. 70-72 (1.18) and p. 164-68 (1.50-51) for a summary of the classical commentaries}.  

Thus, when we read about a Vāsanāyoga, we naturally expect a Yoga that aims at reducing the \textit{vāsanā}s in order to achieve mental stillness and thereby \textit{mokṣa}.

\subsection{The term \textit{vāsanā} in \textit{Yogatattvabindu} and \textit{Yogasvarodaya}}

Similar to the case of Dhyānayoga, which both texts do not introduce as a separate category, but the concept of \textit{dhyāna} can nevertheless be extrapolated, conclusions can also be drawn about the useage and concept of the term \textit{vāsana} despite the absence of a dedicated description of Vāsanayoga.

In \textit{Yogatattvabindu}, the term plays a role in the interpretation (\textit{nirukti}) of the word \textit{avadhūta}. This word interpretation is explained in \uproman{44}.3 and \uproman{44}.4:\footnote{Although most of the verses and passages in \textit{Yogatattvabindu} \uproman{44} are taken from \textit{Siddhasiddhāntapaddhati}, there is no correspondence to the verses \uproman{44}.3-4 in this case. These verses may be authorial. The \textit{Yogasvarodaya} does not thematise the \textit{avadhūta} at all.}

\begin{quote}
  \textit{ātmā hy akāro vijñeyo vakāro bhavavāsana} |
  \textit{dhūta tatkaṃpanaṃ proktaṃ so 'vadhūta udāhṛtaḥ} || \uproman{44}.3 ||
\end{quote}

\begin{quote}
  The letter \textit{a} is to be known as the self, and the letter \textit{va} as the impressions of [mundane] existence; \textit{dhūta} (`has shaken off') is said to be the special weapon; he is called an Avadhūta.
\end{quote}

\begin{quote}
  \textit{akārārtho jīvabhūto vakārārtho 'tha vāsanā} |
  \textit{etad dvayaṃ yaḥ jānati so 'vadhūta udāhṛtaḥ} || \uproman{44}.4 ||
    \end{quote}

    \begin{quote}
      The meaning of the letter \textit{a} is the being of the embodied soul, and the meaning of the letter \textit{va} is then impressions. He who knows this couple is declared to be an Avadhūta.
    \end{quote}
    
    Accordingly, an Avadhūta is characterised by not only knowing the embodied soul (\textit{jīva}) and the \textit{vāsana}s (``mental residues'') produced by action (\textit{karma}), but the Avādhūta is an embodied soul (\textit{jīva}) who has already shaken off all \textit{vāsanā}s and, as the following verses \uproman{44} 5-10 let us know, has become a yogin (\textit{siddhayogin}) perfected by means of Yoga. \\
    
    In addition, the term \textit{vāsanā} appears again in the context of \textit{Yogatattvabindu} section \uproman{52}. This section is part of a thematic sequence of sections that differentiate metaphysical concepts of cosmogony. The sections on cosmogeny begin with section \uproman{48}: ``Now, through the accomplishment of yoga, such knowledge arises.''.\footnote{\textit{Yogatattvabindu} \uproman{48}: \textit{idānīṃ yogasiddhar anantaraṃ etādṛśaṃ jñānaṃ utpadyate}.} From here Rāmacandra unfolds a cosmogony based on the descriptions of the \textit{Yogasvarodaya} and \textit{Siddhasiddāntapaddhati}. However, he mixes, reduces and reorganises the contents of his source texts.

Creation itself begins even before the Creator existed. He is composed of \textit{kula} (Śakti) and \textit{akula} (Śiva). That which existed before the Creator is called the unmanifest (\textit{avyakta}), nameless (\textit{anāmā}) supreme reality (\textit{paraṃ tattvaṃ}). In the sections \uproman{48} - \uproman{56}, the cosmogony unfolds in pentads, giving rise to five qualities each. In section \uproman{52}, Rāmacandra introduces the next pentad, which he does not name for unknown reasons. However, it is based on the explanations of the pentad on \textit{vyaktaśakti} of \textit{Siddhasiddhāntapaddhati}.\footnote{Cf. \textit{Siddhasiddhāntapaddhati} 1.54.} This pentad consists of will (\textit{icchā}), activity (\textit{kriyā}), illusion (\textit{māyā}), primordial nature (\textit{prakṛti}) and speech (\textit{vācā}). Each pentad has five properties. The will (\textit{icchā}) consists of the five properties - intense passion (\textit{unmāda}), mental imprints (\textit{vāsanā}), desire (\textit{vāñchā}), mental state (\textit{caitta}) and behaviour (\textit{ceṣṭā}). This pentad can also be found in the \textit{Yogasvarodaya}.\footnote{\textit{Yogasvarodaya} (PT Ed. p. 847).} None of the texts provides additional information on these five qualities. \\

  The last mention of \textit{vāsanā} is in section \uproman{57}. This section is one of the most extended sections of the entire text and is therefore considered particularly important for the entire Yoga system of Rāmacandra. It bears the title ``Majesty of Yoga'' (\textit{yogasya māhātmyaṃ}) and vehemently emphasises the indispensability of a teacher (\textit{guru}) for the attainment of the reality of yoga (\textit{yogatattva}). However, this should not be just any teacher, but a true teacher (\textit{sadguru}):
  \begin{quote}
    \textit{vikalpa etādṛśo yathā samudramadhye mahttarakallolāḍambaraḥ prapañcacāsanā etādṛśī yathodakamadhye mahattaraṅgāḥ} | \textit{tādṛśāt saṃsārārṇavād yo nāvā paraṃ pāraṃ prāpayati} | \textit{sa sadguruḥ kathyate} |
    \end{quote}
  \begin{quote}
    The changing thought is like the roar of waves within the ocean. The manifold mental imprints are like the ripples in the water. He who causes to navigate the boat from such an ocean of \textit{saṃsāra} to the other shore is called a true teacher.
  \end{quote}

Overall, within the tradition of the \textit{Yogasvarodaya} available to us, the term \textit{vāsanā} only appears in the context of cosmogony, and Vāsanāyoga is not present. In all three contexts in which \textit{vāsana} is mentioned in the \textit{Yogatattvabindu} - \textit{avadhūta}, cosmogony and the importance of the teacher for Yoga practice - it is not possible to speak of a Vāsanāyoga.

\subsection{Vāsanāyoga in the \textit{Yogasiddhāntacandrikā}}
\label{laksyayogaintrocandrika}  

The \textit{Yogasiddhāntacandrikā} is the only text amongst the texts of the complex late medieval taxonomies that contains a dedicated description of a Vāsanāyoga.

Nārāyaṇatīrtha locates Vāsanayoga in the framework of his commentary on \textit{Yogasūtra} 1.37 and 1.38\footnote{Cf. \textit{Yogasiddhāntacandrikā} ed. p. 55-56} and distinguishes two different methods of Vāsanāyoga. Let us first look at the first:

\begin{quote}
\textit{avāntaravāsanāyogam āha}-
\textit{vītarāgaviṣayaṃ vā cittam} || 37 ||
\end{quote}
\begin{quote}
With regard to [the two different methods of] Vāsanāyoga, it is said: \\
Or, [the mind becomes stable when directed], on a mind without the desire for sense objects. 
\end{quote}

This \textit{sūtra} states another way of attaining \textit{samādhi}. Here, the method for stabilising the mind is a meditation on the mind (\textit{citta}) of someone whose mind is already free from craving for sense objects, for example, on the mind of a person known to have already attained this state. This person can be one's own realised teacher, but it can also be a famous Yoga master of the past. In particular, the mind of the chosen person should be free of \textit{vāsanā}s. Nārāyaṇatīrtha explains:

\begin{quote}
  \textit{vīteti} | \textit{vītarāgaṃ nirvāsanaṃ yat sanakādīnāṃ cittaṃ tadviṣayaṃ tadvibhāvanaparaṃ kuryāt} | \textit{nirvāsanavāsitam antaḥkaraṇaṃ kuryād iti yāvat} | \textit{anenātra yogino mumukṣālābhena vāsanāyogo darśitaḥ} |
\end{quote}
\begin{quote}
[Regarding the term] \textit{vīta} [``without'']. On a mind without desire, without sublime impressions, which is like that of Sanaka and others, he shall be entirely devoted to that reflection [which has] that [type of mind] as its object. To be precise, the mind shall be free from subliminal impressions. In this case, Vāsanayoga revealed [itself] through the attainment of the yogi's strong desire for liberation. 
\end{quote}

The most important characteristic of the chosen mind is freedom from \textit{vāsanā}s. When the right mind has been chosen as the object of meditation, this manifests itself for the practitioner initially, particularly through an increased desire for liberation. In the further course of the commentary to 1.37, Nārāyaṇatīrtha goes on to explain that Vāsanayoga primarily leads to an increase in the sattvic quality of mind. This increase of Sattva, in turn, increases the efficiency of all other practised Yoga methods.\footnote{Cf. \textit{Yogasiddhāntacandrikā} (Ed. p. 56) regarding \textit{sūtra} 1.37: \textit{uktañ ca smṛtau} - \textit{sattvāvalambanaṃ yat tad bījaṃ cittaviśodhane} | \textit{bhavet sa vāsanāyogo yogāntaravivarddhakaḥ} || \textit{iti} || "It is said in the Smṛti: That which supports the sattvic constitution is the primary cause for the purification of the mind, this is the Vāsanāyoga which enhances the other Yogas". I was unfortunately unable to identify the source of this verse}. The clue of this practice is that by meditating on a mind that is free of \textit{vāsanā}s, one's own \textit{vāsanā}s are also automatically extinguished through this method.\footnote{Cf. Ibid: \textit{tejaḥpratibandhajalaśaityavad iti vinaiva sādhanāntaraṃ yogino mokṣasukhaniṣṭhāsambhavāt} | \textit{ayaṃ śubho vāsanāyogo viruddhavāsanānivarttaka iti} || 37 || ``As without that which is `like cold water combined with heat’ is the yogi's inner practice, [for] this auspicious Vāsanayoga is that which removes the blocking sublime impressions, as a result of that the state of happiness and liberation arises for the yogi.''}\\

Let us now turn towards the second method of Vāsanayoga. Nārāyaṇatīrtha introduces this method as follows:
\begin{quote}
\textit{vāsanāyogasyāvāntaraṃ bhedam āha}-\\
\textit{svapnanidrājñānālambanaṃ vā} || 38 ||
\end{quote}
\begin{quote}
With regards to the [other] distinction of Vāsanayoga, he says:\\ 
Or, [onto] the support of knowledge from dreams and sleep. 
\end{quote}

Nārāyaṇatīrtha explains in this regard that during sleep in dreams, some people have a vision of the favoured form of the divine, and others experience happiness through sleep. If this is the case, one can use these experiences as objects of meditation. This method works well because these experiences are based on previous very sattvic \textit{vāsanā}s. Meditating on them, therefore, also increases the sattvic quality in the waking state and thus leads to liberation.\footnote{Cf. Ibid.: \textit{svapne bhagavato yadrūpaṃ priyam ārādhayann eva prabuddha, evaṃ nidrādau yatsukham anubhūyate tad avalambanaṃ tad vibhāvanaparaṃ cittaṃ kuryāt} | \textit{pūrvavāsanāprāptasattvapradhānam evāntaḥkaraṇaṃ kuryād iti yāvat} || 38 || ``With regard to a dream, worshipping the divine in the favoured form, similarly, when one is awake, the mind should make the happiness experienced during sleep, etc., the support; that is what should be contemplated. To put it plainly: The mind should indeed cultivate the predominance of purity obtained from previous impressions.''}\\

Thus, the first method of Vāsanayoga stands in stark contrast to the second method of Vāsanayoga. The first method of Vāsanayoga reduces \textit{vāsanā}s by focusing the practitioner's mind on another mind that has already dissolved its \textit{vāsanā}s. The second method specifically uses very positive \textit{vāsanā}s to cultivate the sattvic quality, which can also be a path to \textit{mokṣa}.   

\section{11. Śivayoga}
\label{sivayogaintro}

Rāmacandra positioniert Śivayoga an der elften Stelle seiner Taxonomie der fünfzehn Methoden des Rājayoga, widmet Śivayoga aber weder eine Sektion, noch fällt der Begriff im Laufe des Textes erneut. Die beiden Verse, welche in der \textit{Yogasvarodaya} die Gesamtzahl der fünfzehn Yogas erwähnen, listen nur acht davon auf. Śivayoga ist in dieser unvollständigen Liste nicht vorhanden und wird auch in der \textit{Yogasvarodaya} nicht als eigenständiges Thema eingeführt. Das Śivayoga auch im \textit{Yogasvarodaya} zu den fehlenden sieben Yogas gehören dürfte ist sehr wahrscheinlich. Einerseits ist das \textit{Yogasvarodaya} eindeutig ein Yogatext, der aus einem Śaiva Milieu entstammt. Andererseits nennen alle anderen Texte die fünfzehn Yogas behandeln auch Śivayoga. Das \textit{Yogatattvabindu} übernimmt zwar einen Großteil der Lehrinhalte des \textit{Yogasvarodaya}, verschleiert jedoch fast alle Spuren religiöser Affiliation, die in seinem Quelltext vorlagen. Wenn Rāmacandra von einem Gott spricht, dann verwendet er ausschließlich den neutralen Begriff \textit{īśvara}. In der \textit{Sarvāṅgayogapradīpikā} wird Śivayoga nicht erwähnt. Ein Śivayoga wäre im Milieu des Vaiṣṇava \textit{bhakti} eines Sants wie Sundardās auch nicht unbedingt zu erwarten gewesen.\footnote{Cf. \citeauthor{horstmann2023shrine} 2023, p. 7.} Die einzige dezidierte Beschreibung eines Śivayoga im Rahmen der Texte der komplexen Taxonomien findet sich erneut ausschließlich in Nārāyaṇatīrthas \textit{Yogasiddhāntacandrikā}.\footnote{See \citeauthor{penna2004} 2004, pp. 80-82 for an earlier discussion of Śivayoga in the \textit{Yogasiddhāntacandrikā}.}

\subsection{Śivayoga in the \textit{Yogasvarodaya} and \textit{Yogatattvabindu}?}

Das \textit{Yogasvarodaya} und das \textit{Yogatattvabindu} widmen Śivayoga wird als Unterkategorie des Rājayoga keine eigene Sektion, wie dies bei anderen in der Taxonomie der fünfzehn Yogas aufgelisteten Unterkategorien des Rājayoga der Fall war. Die Frage, warum Śivayoga überhaupt aufgelistet, dann aber nicht beschrieben wird, wirft eine weitere Frage auf. Nämlich was überhaupt in solch einer Beschreibung von Śivayoga als Methode des Rājayoga zu erwarten gewesen wäre. Der Vergleich der Lehrinhalte beider Texte mit denen der \textit{Śivayogapradīpikā},\footnote{Erst vor kurzem wurde eine kritische Edition im Rahmen einer umfangreichen Dissertatiosstudie von \citeauthor{powell2023} (2023) abgeschlossen. An dieser Stelle möchte ich Dr. Seth \citeauthor{powell2023} danken, dass er mir noch der Veröffentlichung seiner Dissertation, seine Arbeit zur Einsicht zur Verfügung stellte.} der erste Text überhaupt, welcher Śivayoga als einzigartiges System des Yoga in Beziehung zu anderen Yogasystemen postuliert,\footnote{A textual history of the Sanskrit compound \textit{śivayoga} is presented by \textit{powell2023} 2023, pp. 48-57.} zeigt frappante inhaltliche Parallelen. Außerdem wirft der Vergleich die nächste Frage auf, nämlich ob nicht auch das gesamte im \textit{Yogasvarodaya} and \textit{Yogatattvabindu} präsentierte Yogasystem auch als Śivayoga begriffen werden könnte, bzw. ob es denn überhaupt einen Unterschied gibt, der groß genug wäre, Śivayoga nach dessen Nennung in der Taxonomie nochmal getrennt zu beschreiben, denn bereits Cennasadāśivayogin, der Autor der \textit{Śivayogapradīpikā} setzt Śivayoga und Rājayoga in Vers 1.13 gleich:
\begin{quote}
In reality, there is no difference between Śivayoga and Rājayoga. Yet for those who worship Śiva [a difference] is thus declared, in order to increase wisdom.\footnote{Translated by \citeauthor{powell2023} 2023, p. 315.}\footnote{\textit{Śivayogapradīpikā} 1.13: \textit{na bhedaḥ śivayogasya rājayogasya tattvataḥ} | \textit{śivārcināṃ evam ukto buddeḥ pravṛddhaye} || 13 ||} 
\end{quote}
Eine ähnliche Aussage findet sich ebenfalls in der \citetitle{yogasarasangraha}. Hier werden Rājayoga, Śivayoga, \textit{samādhi} und andere Bezeichnungen für den höchsten soteriologischen Zustand gleichgesetzt.\footnote{\citetitle{yogasarasangraha} p. 60: \textit{rājayogaḥ samādhiś conmanī ca manonmanī} | \textit{śivayogo layastatvaṃ śūnyāśūnyaṃ nirañjanam} || \textit{amanaskaṃ yathā caitannirālambaṃ nirañjanam} | \textit{jīvanmuktiś ca sahajam ity adir hy ekavācakam} ||}.

Das \textit{Yogasvarodaya} ist ein Text des Rājayoga Genres, der einem Śaiva Milieu entsprungen ist. So heißt es im Text beipsielsweise, dass der Yogin als Kenner des ersten Typus des Jñānayoga den Rang eines Śiva genannten Erlösten erlangt,\footnote{ \textit{Yogasvarodaya} (PT Ed. p. 831): \textit{jñānayogaṃ pravakṣyāmi tajjñānī śivatāṃ vrajet} |}, dass der Yogin durch die Praxis von Haṭhayoga dem Śiva gleich wird,\footnote{Ibid. (PT Ed. p. 835): \textit{śivatulyo mahātmāsau haṭhayogaprasādataḥ} |} oder dass der Yogin als Ergebnis der Praxis des \textit{madhyalakṣya} einer ist, der in der Welt lustwandelt wie Śiva, ohne Sünde oder Verdienst,\footnote{Ibid. (PT Ed. p. 839): \textit{śivavad vihared viśve pāpapuṇyavivarjitaḥ} |} Darüber hinaus wird im Abschnitt über \textit{yogamāhātmya} ein wahrer Lehrer (\textit{sadguru}) mit Śiva gleichgesetzt.\footnote{Ibid. (PT Ed. p. 848): \textit{nānāvikalpavibhrāntināśañca kurute tu yaḥ} | \textit{sadguruḥ sa tu vijñeyo na tu vairaprakalpakaḥ} | \textit{ata eva maheśāni sadguruḥ śiva āditaḥ} |} Es finden sich weitere Erwähnungen von Śiva in der \textit{Yogasvarodaya}. Rāmacandra hingegen bedient sich zwar großzügig bei der \textit{Yogasvarodaya} für die Kompilation seines Textes, blendet die śivaitischen Begriffe seiner Vorlage jedoch weitestgehend aus, um religiöse Neutralität zu wahren.\footnote{Nur sehr wenige Passagen des \textit{Yogatattvabindu} verraten die śivaitische Abstammung der Inhalte: In Sektion \uproman{3} wird der zentrale Kanal als \textit{śivarūpiṇī} (``Śiva-gestaltig'' bzw. ``in Form des Wohlwollens'') bezeichnet. In Sektion \uproman{21}.3 wird der höchste soteriologische Zustand, der durch Jñānayoga hervorgebracht werden kann als \textit{śāmbhavīsattā} (``die zu Śiva gehörige Realität'') bezeichnet und in Sektion \uproman{48}.1 tauchen Śakti und Śiva als \textit{kula} und \textit{akula} in Rāmacandras Ausführungen zur Kosmogonie auf. Darüber hinaus stammen viele der von Rāmacandra präsentierten Yogapraktiken und Konzepte aus älteren Śaiva Yoga Systemen.}   
Die inhaltlichen Parallelen unserer Texte mit der \textit{Śivayogapradīpikā} sind frappant, sodass es im Hinblick auf die Fragestellung dieses Unterkapitels sinnvoll ist an dieser Stelle die Grundzüge dieser Ähnlichkeit darzustellen. Die \textit{Śivayogapradīpikā} von Cennasadāśivayogin wird von \citeauthor{powell2023} auf circa 1400 – 1450 n. u. Z. datiert.\footnote{\citeauthor{powell2023} 2023, p. 157.} Somit befinden wir uns rund zweihundert Jahre vor der Abfassung des \textit{Yogatattvabindu} und des \textit{Yogasvarodaya}. 
Im Gegensatz zu der fünfzehnfachen Yogataxonomie unserer Texte greift Cennasadāśivayogin auf das oftmals in der mittelalterlichen Yogaliteratur verwendete Modell von Mantra-, Laya-, Haṭha- und Rājayoga zurück, welche als Unterkategorien von Śivayoga betrachtet werden.\footnote{\textit{Śivayogapradīpikā} 1.3-4: \textit{śivatattvavidāṃ śreṣṭha vakṣyāmi śṛṇu te 'dhūna} | \textit{śivayogaṃ paraṃ guhyam api tvadbhaktigauravāt} || 3 || \textit{mantro layo haṭho rājayogaś ceti caturvidham} | \textit{tam āhuḥ pūrvamunayaḥ siddhāḥ śaṃbhuprabodhitāḥ} || 4 ||} Wie bereits im obigen Zitat von \textit{Śivayogapradīpikā} 1.13 erwähnt, setzt Cennasadāśivayogin Śivayoga mit Rājayoga gleich, wobei er Rājayoga wiederum in drei Unterkategorien aufteilt, nämlich Sāṅkhyayoga, Tārakayoga und Amanaska Rājayoga.\footnote{Ibid. \textit{Śivayogapradīpikā} 1.10-11: \textit{so 'pi tridhā bhavet sāṅkhyas tārakaś cāmanā iti} | \textit{pañcaviṃśatitattvānāṃ jñānaṃ tat sāṅkhyaṃ ucyate} || 10 || \textit{bahirmudrāparijñānād yogas tāraka ucyate} | \textit{antarmudrāparijñānād amanaska itīritaḥ} || 11 ||} Cennasadāśivayogin bezeichnet sein Sāṅkhyayoga abschließend auch als Jñānayoga.\footnote{Idid. 4.31.} Um seinen Text und dessen Lehren zu strukturieren verwendet Cennasadāśivayogin die acht Glieder des Aṣṭāṅgayoga.\footnote{Ibid. 2.4-5: \textit{śivayogaḥ sādhakānāṃ sādhyas tatsādhanaṃ haṭhaḥ} | \textit{tasmād ādau prayoktavyaṃ haṭhayogam imam śṛṇu} || 4 || \textit{aṅgāny aṣṭau haṭhasyāpi bāhyāny abhyantarāṇi ca} | \textit{yamādihir ato 'ṣṭāṅgair devapūjāṃ samācaret} || 5 ||} Dabei handelt es sich nicht um das Standard-Modell des achtgliedrigen Yoga des \textit{Pātañjalayogaśāstra}, sondern ein spezifisches Modell einer Gruppe von Texten, welche \textit{dhyāna} und \textit{dhāraṇa} vertauschen. Dieses Phänomen findet sich ansonsten nur in \textit{ṣaḍaṅga} oder \textit{pañcāṅga} Yogasystemen.\footnote{See table 10: \textit{Yogāṅgas with Dhyāna before Dhāraṇa} in \citeauthor{powell2023} 2023, p. 166 for an overview.} \citeauthor{powell2023} (2023: 168) erklärt, diese Vertauschung von \textit{dhyāna} und \textit{dhāraṇa} in einem achtgliedrigen System nur in der \textit{Śivayogapradīpikā} gefunden zu haben. Erst die kritische Edition des \textit{Yogatattvabindu}, insbesondere die Inspektion der ältesten Handschriften konnte zeigen, dass auch andere Texte mit achtgliedrigen Systemen diese Reihenfolge konservieren.\footnote{Siehe Sektion \uproman{31} in der kritischen Edition des \textit{Yogatattvabindu} auf p.\pageref{ashtanga}.} Darüber hinaus findet sich diese vertauschte Reihenfolge auch in der Überlieferung der eng mit der \textit{Śivayogapradīpikā} und dem \textit{Yogatattvabindu} verknüpften \textit{Siddhasiddhāntapaddhati} in den Handschriften J\textsubscript{1} und J\textsubscript{2}.\footnote{Siehe krititsche Edition der \citetitle{ssplonavla} von \citeauthor{ssplonavla} (2016) zu Sektion 2.32 (Ed. p. 45).} Die Überlieferung der \textit{Yogasvarodaya} erscheint an dieser Stelle wirr, denn sie benennt zwar ein achtgliedriges Yoga, nennt im Vers der die Glieder auflistet allerdings nur \textit{dhāraṇa}, erläutert im Verlauf des Abschnittes allerdings \textit{dhyāna} und belässt \textit{dhāraṇa} unerklärt. Nichtsdestotrotz belegt diese spezifische Phänomen zusammengenommen die enge rezeptionsgeschichtliche Verknüfung der vier involvierten Texte. Darüber hinaus listet die \textit{Śivayogapradīpikā} alle im Text benannten Yogas zwar nicht in er Taxonomie auf, ingesamt ergibt sich aber bereits hier eine ähnliche Vielfalt an Yogakategorien, wie in den anderen Texten mit komplexen Yogataxonomien.\footnote{Die \textit{Śivayogapradīpikā} benennt zehn Yogakategorien. Das gesamte System ist ein System des 1. Śivayoga, welches in ein System des 2. Aṣṭāṅgayoga eingebettet ist. Hierin werden 3. Mantrayoga, 4. Layayoga, 5. Haṭhayoga und 6. Rājayoga verortet. Letzteres teilt sich wiederum auf in 7. Sāṅkhyayoga = 8. Jñānayoga, 9. Tārakayoga und 10. Amanaska Rājayoga.}

Im Kontext des vierten Gliedes \textit{prāṇāyāma} differenziert Cennasadāśivayogin drei Arten des \textit{prāṇāyāma}: 1. natürlich (\textit{prākṛta}), 2. modifiziert (\textit{vaikṛta}) und 3. \textit{kevalakumbhaka}, welches sich von selbst entfaltet, mit oder ohne die Praxis der beiden erstgenannten Varianten.\footnote{Cf. \textit{Śivayogapradīpikā} 2.22: \textit{prāṇāyāmas tridhā proktaḥ prākṛto vaikṛtas tathā} | \textit{dvābhyāṃ vinā jṛmbhate 'sau kevalaḥ kumbhakaḥ svayam} || 22 ||} Bei der ersten Variante\footnote{Ibid. 2.29-34} handelt es sich tatsächlich um das \textit{ajapā mantra}, welches möglicherweise auch von Rāmacandra in Sektion \uproman{3} angedeutet, bzw. im Rahmen von der Handschrift \getsiglum{U2} dann dezidiert im Kontext der Meditationen (\textit{dhyāna}s) über die neun \textit{cakra}s instruiert wird. Das Mantrayoga der \textit{Śivayogapradīpikā} wird allerdings dem \textit{prāṇāyāma} untergeordnet.\footnote{Siehe hierzu \citeauthor{powell2023} 2023, p. 205.} Die zweite Variante des des \textit{prāṇāyāma} ist deckungsgleich mit der in \textit{Yogatattvabindu} Sektion \uproman{31}.\footnote{Ibid. 22.4: \textit{āgamoktavidhānena recapūrasvabhāvataḥ} | \textit{yadi prāṇanirodhaḥ syād vaikṛtaḥ sa udītritaḥ} || 24 ||} Im dritten Kapitel der \textit{Śivayogapradīpikā}, welches dem fünfen der acht Glieder \textit{dhyāna} gewidmet ist, finden wir dann eine ausführliche Beschreibung, der auch im \textit{Yogatattvabindu} und \textit{Yogasvarodaya} so zentralen Themen der neun \textit{cakra}s\footnote{Ibid. 3.7-16.} und der sechszehn \textit{ādhāra}s\footnote{Ibid. 3.17-32}. Die Beschreibungen der individuellen Elemente beider Themen sind größtenteils kongruent.  
Neben diversen Ähnlichkeiten gibt es auch signifikante Unterschiede zwischen den Texten. Beispielsweise beinhalten beide Texte Varianten des Jñānayoga (\textit{Śivayogapradīpikā} 4.31 bezeichnet Sāṃkhyayoga als Jñānayoga). Die \textit{Śivayogapradīpikā} lehrt ein System mit insgesamt fünfundzwanzig \textit{tattva}s plus \textit{puruṣa}.\footnote{Siehe \textit{Śivayogapradīpikā} 4.19-31. Außerdem wird System der \textit{tattva}s der \textit{Śivayogapradīpikā} asuführlich von \citeauthor{powell2023} 2023, pp. 239-42 analysiert.} \textit{Yogasvarodaya} und \textit{Yogatattvabindu} lehren ein simpleres System mit nur zehn \textit{tattva}s.\footnote{Cf. \textit{Yogatattvabindu} \uproman{31}.6 und \textit{Yogasvarodaya} (PT Ed. p. 836).} Während Cennasadāśivayogi zunächst eine große Seele (\textit{mahātman}) als eine Seele definiert, die weiß, dass das wahre Selbst (\textit{ātman}) ontologisch von den Evoluten der \textit{prakṛti} zu differenzieren ist,\footnote{\textit{Śivayogapradīpikā} 4.28: \textit{dehatrayaṃ prathitaṣoḍaśadhāvikārān liṅgāni saptadaśadhā navadhā padārthān} | \textit{ātmānām aṣṭavidhayā prakṛtisvabhāvaṃ jñātvā tad anya iti jīvati yo mahātmā} || 28 ||} verkündet er unmittelbar darauf jedoch die Nondualität von \textit{ātman} und \textit{brahman} im Sinne des Advaitavedānta bzw. der \textit{bhedābheda} Schulen des Vedānta.\footnote{Ibid. 4.29-30: \textit{satyaṃ jñānam anantaṃ yad brahmeti vadati śrutiḥ} | \textit{muktānandasvarūpaṃ ca nanu tat tvam asi sthiram} || 29 || \textit{naitad ahaṃ naidrad ahaṃ ceti yad anyaṃ vibhāvayātmānam} | \textit{so 'haṃ iti so 'ham iti nanu bhāvaya sarvaṃ tvam ātmānam || 30 ||}} \textit{Yogasvarodaya} und \textit{Yogatattvabindu} hingegen lehren einen radikle Non-dualität, die radikale Einheit von Allseele, Individualseele und Schöpfung,\footnote{Siehe \textit{Yogatattvabindu} Sektion \uproman{21}.7 und \textit{Yogasvarodaya} (PT Ed. p. 836).} was eher an Formen des Śuddhādvaita erinnert.\footnote{Siehe \citeauthor{glasenapp1949philosophie} 1985, pp. 270–72.}
 Im Rahmen des Tārakayoga im vierten Kapitel der \textit{Śivayogapradīpikā}\footnote{Ibid. 4.32-52.} werden die drei \textit{lakṣya}s \textit{antar}-, \textit{bāhya}- und \textit{madhyalakṣya} gelehrt, wohingegen in \textit{Yogasvarodaya} und \textit{Yogatattvabindu} fünf \textit{lakṣya}s gelehrt werden.
Es existieren weitere Unterschiede, aber der wahrscheinlich zentralste Unterscheid ist, dass alle Lehren in Cennasadāśivayogins \textit{Śivayogapradīpikā} in den rituellen und devotionalen Rahmen der Vīraśaivas eingebettet sind.\footnote{\citeauthor{powell2023} 2023, p. 8.} So definiert Cennasadāśivayogin Śivayoga in Vers 1.15 als:
\begin{quote}
  Śivayoga is five-fold, indeed: gnosis (\textit{jñāna}) comprised of Śiva, devotion (\textit{bhakti}) to Śiva,
meditation (\textit{dhyāna}) comprised of Śiva, Śaiva religious observance (\textit{vrata}), and worship of Śiva
(\textit{arcā}).\footnote{\textit{Śivayogapradīpikā} 1.15: \textit{jñānaṃ śivamayaṃ bhaktiḥ śaivī dhyānaṃ śivātmakam} | \textit{śaivavrataṃ śivārceti śivayogo hi pañcadhā} || 15 || Translation by \citeauthor{powell2023} 2023, p. 315.} 
  \end{quote}
  Trotz der klaren Śaiva Affiliation des \textit{Yogasvarodaya} lassen sich diese Elemente dort nirgends finden. Gleiches gilt für das \textit{Yogatattvabindu}. Selbst das achtgliedrige (\textit{aṣṭāṅga}) Schema wird in diesem Text als rituelle Verehrung von Śiva (\textit{śivapūja}) betrachtet\footnote{Cf. Ibid. 2.1-5.} und \citeauthor{powell2023} (2023) schlussfolgert, dass es eben diese hingebungsvolle und rituelle Ausrichtung ist, die das Yogasystem der \textit{Śivayogapradīpikā} zum Śivayoga macht.\\

  Kann man dieser vergleichenden Betrachtung sagen, dass die Yogasysteme der \textit{Yogasvarodaya} and \textit{Yogatattvabindu} implizit Śivayoga lehren? Diese Frage lässt sich, nicht ganz eindeutig beantworten. Es ist Fakt, dass auf der Ebene der Lehrinhalte alle drei Texte zahlreiche Gemeinsamkeiten aufweisen. Inhaltlich gesehen könnte diese Frage tendenziell positiv beantwortet werden. Die stark śivaitische Ausrichtung,\footnote{Das Wort \textit{śiva} wird in der \textit{Śivayogapradīpikā} insgesamt neunungsiebzig Mal erwähnt.} wie sie in der \textit{Śivayogapradīpikā} zu erkennen ist, ist jedoch in der \textit{Yogasvarodaya} und dem \textit{Yogatattvabindu} weitestgehend abwesend und beide Texte ordnen den Śivayoga faktisch dem Rājayoga unter. Der Grad der śivaitische Ausrichtung im \textit{Yogasvarodaya} ist mit zehn Erwähnungen des Wortes \textit{śiva} eher mäßig und im \textit{Yogatattvabindu} fast vollständig erloschen. Aus diesem Blickwinkel heraus muss die Fragestellung dieses Unterkapitels eindeutig negativ beantwortet werden. Nichstdestotrotz, wäre vor dem hier präsentierten Hintergund die mysteriöse Präsenz der Kategorie Śivayoga in den fünfzehnfachen Taxonomien, welche den Śivayoga als Unterkategorie des Rājayoga nennt, und zu unserem Leidwesen nicht explizit erläutert, leicht zu erklären. Śivayoga und Rājayoga wären gemäß der eingangs genannten Auffassung von Cennasadāśivayogin inhaltlich weitestgehend Deckungsgleich und somit als Synonyme zu betrachten. Der Fakt, dass beide Systeme auch weitestgehend die gleichen Praktiken lehren, würde die Abwesenheit einer gesonderten Widmung einer einzelnen Sektion, welche explizit Śivayoga erläutert völlig überflüssig machen. Es scheint als habe Rāmacandra die Auffassung Cennasadāśivayogin geteilt.        

  Außerdem lassen die frappanten inhaltlichen Ähnlichkeiten, wie etwa die spezielle Reihenfolge der acht Glieder der Aṣṭāṅgayogas, keinen anderen Schluss zu, als dass die \textit{Śivayogapradīpikā} und das \textit{Yogasvarodaya} und hierüber auch das \textit{Yogatattvabindu}, welches verwobenerweise auch auf die \textit{Siddhasiddhāntapaddhati} zurückgreift, einem Text, welcher der \textit{Śivayogapradīpikā} wiederum extrem Nahe steht\footnote{For a discussion of the relationship between the \textit{Śivayogapradīpikā} and \textit{Siddhasiddhāntapaddhati} see \citeauthor{powell2023} 2023, pp. 147-52.} aus dem gleichen intertextuellen Netzwerk entsprungen sind. 

\subsection{Śivayoga in the \textit{Yogasiddhāntacandrikā}}
\label{sivayogacandri}
Nārāyaṇatīrtha situiert Śivayoga, zusammen mit Brahmayoga\footnote{Die Diskussion von Brahmayoga findet im nachfolgenden Kapitel auf p.\pageref{brahmayogaintro} statt.} in seinem Kommentar zu \textit{sūtra} 1.36:\footnote{See \citeauthor{penna2004} 2004, pp. 80-82 for another discussion of Śivayoga in the \textit{Yogasiddhāntacandrikā}.}

\begin{quote}
\textit{brahmayogaṃ śivayogañ cāha}-\\
\textit{viśokā vā jyotiṣmatī} || 36 ||
\end{quote}
\begin{quote}
It is said about Brahmayoga and Śivayoga: \\
Or, [steadiness of the mind is gained when it is directed onto that which is] without sorrow [and] luminous.  
\end{quote}

Die Methode des Śivayoga besteht laut Nārāyaṇatīrtha darin, den Blick innerlich und äußerlich auf das Selbst in der Form von Licht in der Mitte der Augenbrauen etc. zu fixieren. Das Ergebis dieser Zurückhaltung des Geistes (\textit{saṃyama}) ist ohne Kummer (\textit{viśokā}).\footnote{\textit{Yogasiddhāntacandrikā} (Ed. p. 55): \textit{athavā bhrūmadhyādau jyotīrūpe pratyagātmani bahirdṛṣṭibandhena manasaḥ saṃyamād viśokā} |} Daraufhin beschreibt er, dass Śivayoga der Grund für Stabilität des Geistes sei. Weiterhin sei es die Gnosis, durch das luminous [Meditations-]Objekt des Zeugen (\textit{sākṣin}) und frei von den Qualen, welche durch Praktiken des Haṭhayoga etc, ausgelöst werden. Er bezeichnet Śivayoga dann als Śāmbhavīmudrā.\footnote{Ibid.: \textit{haṭhayogādāvivāyāsakṛtakleśarahitā jyotiṣmatī sākṣiviṣayāsaṃvin manasaḥ sthairyahetur iti śivayogaḥ} | \textit{ayam eva śāmbhavī mudrety ucyate} |}

Daraufhin zitiert Nārāyaṇatīrtha \textit{Amanaska}\footnote{Cf. \textit{Amanaska} 2.10 which reads \textit{antarlakṣyaṃ} instead.} ohne Referenz:
\begin{quote}
\textit{antarlakṣyā bahirdṛṣṭir nimeṣonmeṣavarjitā} |
\textit{eṣā hi śāmbhavī mudrā sarvatantreṣu gopitā} ||
\end{quote}
\begin{quote}
 The focus is internal, the gaze external, unblinking: this is the \emph{śāmbhavī mudrā} concealed in all the Tantras.\footnote{This is the translation of our critical Edition of the \textit{Haṭhapradīpikā} (2024), which also quotes this verse in 4.6.} 
  \end{quote}

Unmittelbar darauf erklärt Nārāyaṇatīrtha, dass Śāmbhavīmudrā auch mittels Yogāsana-, Cāñcarī-, Bhūcarī-, Khecarī-, Agaucarī- [and] Nirvāṇamudrā accomplished werden kann, wobei die Instruktionen für diese über einen Lehrer in Erfahrung gebracht werden müssen.\footnote{Ibid.: \textit{sā ca yogāsanacāñcarībhūcarīkhecarya'gaucarīnirvāṇamudrābhiḥ siddhyati} | \textit{prakāras gurumukhād avagantayaḥ} |}.    

Nārāyaṇatīrthas Assoziation von Śambhavīmudrā und Śivayoga ist aufschlussreich, denn einerseits ist Śambhavīmudrā zentrale Praxis des Rājayoga des \textit{Amanaska}\footnote{Cf. \textit{Amanaska} 2.2-10.}, andererseits lehrt auch Cennasadāśivayogin Śambhavīmudrā\footnote{Cf. \textit{śivayogapradīpikā} 5.3.} als Teil seines Śivayogasystems. Hierdurch wird eine konzeptuelle Brücke zwischen Rāja- und Śivayoga aufgebaut.  

\section{12. Brahmayoga}
\label{brahmayogaintro}

In der Taxonomie der fünfzehn Methoden des Rājayoga platziert Rāmacandra Brahmayoga auf Position zwölf. Abgesehen von dieser Nennung fehlt von Brahmayoga jede Spur. Möglicherweise ist Brahmayoga im \emph{Yogatattvabindu}, ähnlich wie im Falle des Śivayoga als Synonym für Rājayoga zu betrachten und wird aus diesem Grund nicht separat diskutiert.\footnote{In den einführenden Vers des \emph{Yogasvarodaya} (Ed. p. 831) heißt es zu den fünfzehn Methoden des Rājayoga: ``By [means of] these fifteen [yogas], this [person] who is resting in Brahman shines [like a king].'' (\textit{rājaty etad brahmaśīva ebhiś ca pañcadaśadhā} ||).} Im \textit{Yogasvarodaya} ist der Begriff Brahmayoga, zumindest in der uns vorliegenden Überlieferung vollständig abwesend. Die beiden Verse, welche im \textit{Yogasvarodaya} die Gesamtzahl der fünfzehn Yogas erwähnen, listen nur acht davon auf. Brahmayoga ist in dieser unvollständigen Liste nicht vorhanden und wird auch in der \textit{Yogasvarodaya} nicht als eigenständiges Thema eingeführt. Das Brahmayoga auch im \textit{Yogasvarodaya} zu den fehlenden sieben Yogas gehören dürfte ist jedoch relativ wahrscheinlich, da diese Yogakategorie in allen anderen komplexen Yogataxonomien genannt wird. So positioniert Nārāyaṇatīrtha Brahmayoga in der \textit{Yogasiddhāntaycandrikā} auf Position neun. In der \textit{Sarvāṅgayogapradīpikā} subsummiert Sundardās Brahmayoga unter der Oberkategorie des Sāṃkhyayoga zusammen mit Jñānayoga und Advaitayoga. In diesem Rahmen ist es das elfte und somit vorletzte von Sundardās beschriebene Yoga. Sowohl Nārāyaṇatīrthas als auch Sundardās erläutern Brahmayoga im Detail. 

\subsection{Brahmayoga in the \textit{Yogasiddhāntacandrikā}}

Wie bereits hinsichtlich Śivayoga zuvor, verortet Nārāyaṇatīrtha Brahmayoga im Kontext seines Kommentares zu \textit{Yogasūtra} 1.36.\footnote{Siehe p. \pageref{sivayogacandri} für die Übersetzung dieses \textit{sūtra}s.}\footnote{See \citeauthor{penna2004} 2004, pp. 89-80 for another discussion of Brahmyoga in the \textit{Yogasiddhāntacandrikā}.} Auch in diesem Fall handelt es sich mit Brahmayoga um eine Methode den Geist auf ein leuchtendes (\textit{jyotṣmatī}) Meditationsobjekt zu richten, welches frei von Kummer (\textit{viśokā}) ist. Dieses Meditationsobjekt ist Brahman in Form des \textit{nāda} (innere Resonanz) und befindet sich im achtblättrigen Lotus des Herzens. Die Vereinigung des Geistes mit \textit{nāda} ist frei von Kummer (\textit{viśokā}). Dies liegt laut Nārāyaṇatīrtha daran, weil Brahmayoga frei von Elend ist, das durch die Anstrengung vielfältiger Methoden [des Yoga] verursacht wird. Nārāyaṇatīrtha bezeichnet diese Methode ebenfalls als leuchtend, weil es ein Licht als Objekt hat. Dieses Licht sei die Gnosis durch das Objekt, das aus Bewusstsein und Glückseligkeit besteht und im \textit{nāda} enthalten ist. Gelingt es dem Übenden den Geist mit dem \textit{nāda} im Lotus des Herzen zu vereinigen wird der Geist zum Stillstand gebracht.\footnote{\textit{Yogasiddhāntacandrikā} (Ed. p. 54): \textit{viśoketi}| \textit{aṣṭadalādau nādākhye brahmaṇi manasaḥ saṃyogād viśokā bahutarasādhanādyāyāsakṛtaduḥkhaśūnyā jyotiṣmatī jyotirviṣayā nādagatacidānandaviṣayāsaṃvin manasaḥ sthitihetur ity arthaḥ} |} 

Die Praxis selbst ist beinhaltet eine detailreiche Meditation, welche Visualisierung, \textit{prāṇāyāma} und die drei Buchstaben A-U-M, welche den Klang des Mantras \textit{oṃ} bilden, beinhaltet:

\begin{quote}
\textit{tathā hy ayam atra kramaḥ} |\\ \textit{hṛdayādho 'dhomukhamaṣṭadalaṃ kamalaṃ recakeṇordhvamukhaṃ vibhāvya}, \textit{tatra sūryamaṇḍalaṃ dvādaśakalātmakaṃ jāgaritasthānam akāraṃ},
\textit{tadupari candramaṇḍalaṃ ṣoḍaśakalātmakaṃ svapnasthānam ukāraṃ}, \textit{tadupari vahnimaṇḍalaṃ daśakalātmakaṃ suṣuptisthānam makāraṃ}, \textit{tadupari nādākhyaṃ turīyaṃ brahma vibhāvayed iti brahmayogaḥ} |
\end{quote}
\begin{quote}
Thus, indeed this is the respective sequence: \\
In the lower [part of] the heart there is an eight-petalled lotus facing downward, by means of \textit{recaka}[-\textit{kumbhaka}]\footnote{Dies meint wahrscheinlich nicht einfach nur eine Ausatmung (\textit{recaka}), sondern Atemanhaltung (\textit{kumbhaka}) nach erfolgter Ausatmung (\textit{recaka}) in der Atemleere.} it should be made upward facing, there, one should contemplate the orb of the sun, consisting of twelve digits, the wakeful state [and] the letter A; above that the orb of the moon, consisting of sixteen digits, the dreamful state [and] and the letter U; above that the orb of fire, consisting of ten digits, the deep sleep state [and] the letter M; above that, that which is known as Nāda, the fourth state, the Brahman. This is Brahmayoga. 
\end{quote}

Die Beschreibung der Praxis ist nicht eindeutig nachzuvollziehen. Nārāyaṇatīrthas lässt offen, ob der Übende des Brahmayoga alle einzelnen Schritte der Visualierung während genau eines \textit{recaka}s oder je ein oder gar mehrere \textit{recaka}s pro Einzelschritt der Visualiserung ausführen soll. Die Praxis könnte auch so aufgefasst werden, dass bereits ein einzelner \textit{recaka} genügt um den achtblättrigen Lotus nach oben zeigen zu lassen und die Meditationsschritte dann ohne weitere Atemtechniken ausgeführt werden. Genauso wäre es möglich zu verstehen, das mehrere \textit{recaka}s geübt werden bis der achtblättige nach oben zeigt und dann die Meditationsschritte ohne weitere Atemtechniken geübt werden.
Die Ausübung der Meditation ist leichter nachzuvollziehen. Die drei Schritte sollen offenbar unmittelbar aufeinanderfolgend ausgeführt werden, um mental ein lang gezogenes \textit{oṃ} zu rezitieren, dessen auslautendes M (\textit{makāra}) in den \textit{nāda} (interne Resonanz) überleitet, welcher mit Brahman und dem vierten Zustand (\textit{turīya}) assoziiert wird.

Im Anschluss spezifiziert Nārāyaṇatīrtha diesen \textit{nāda} mittels eines Zitates, welches er aus der \textit{Gītāsāra} entnimmt:\footnote{=\textit{Uttaragīta 41cd-42} and \textit{Haṭhapradīpikā} 4.49.}

\begin{quote}
\textit{taduktaṃ gītasāre} -
\textit{anāhatasya śabdasya tasya śabdasya yo dhvaniḥ} |
\textit{dhvanerantargataṃ jyotir jyotirantargataṃ manaḥ} ||
\textit{tanmano vimalaṃ yāti tadviṣṇoḥ paramaṃ padam} |
\end{quote}
\begin{quote}
The tone of that sound is that of the unstruck sound. A light is inside the tone [and] the mind is inside the light. That mind dissolves. That is the supreme state of Viṣṇu.\footnote{The translation is taken from our new Edition of the \textit{Haṭhapradīpikā} (2024).}
  \end{quote}

  Etwas überraschend zitiert Nārāyaṇatīrtha unmittelbar darauf die \textit{Haṃsopaniṣad}, welche jedoch nicht die Rezitation des Mantra \textit{oṃ}, sondern die Rezitation von \textit{haṃsa}, also das \textit{ajapa} Mantra? beschreibt.\footnote{\textit{Yogasiddhāntacandrikā} (Ed. pp. 54-55): \textit{haṃsopaniṣadi coktaḥ} – \textit{haṃsānusaṃdhānaphalabhūto 'nekavidhaḥ saphalaḥ} |} Dieser Unterschied scheint für den Punkt den Nārayaṇatīrtha machen möchte, keine Rolle zu spielen. Die Konzentration auf den \textit{nāda} führt den Übenden dann durch eine Sequenz von insgesamt zehn verschiedenen Klängen, welche der Übende während dessen Kontemplation wahrnehmen kann:
\begin{quote}
  \textit{asyaiva japakoṭyā nādam anubhāvayati yas tasya daśavidha upajāyate} | 
  \textit{ciṇīti prathamaḥ, ciṇiciṇīti dvitīyaḥ ghaṇṭānādastṛtīyaḥ, śaṅkhanādaścaturthaḥ, pañcamastantrīnādaḥ, ṣaṣṭhastalanādaḥ, saptamo veṇunādaḥ, aṣṭamo bherīnādo, navamo mṛdaṅganādo, daśamo meghanādaḥ} | \textit{navamaṃ pariatyajya daśamam eva 'bhyaset} |
\end{quote}
\begin{quote}
  Thus, caused by practicing 10 million repititions (\textit{japa}) of that sound, then types of that [sound] arise: \\
  The first sound is \textit{ciṇi}, the second \textit{ciṇciṇi},\footnote{Vielleicht sind diese Begriffe onomatopoetisch gemeint. Der Klang erinnert an das Zwitschern eines Vogels oder das zirpen einer Grille.} the third the sound of a bell, the fourth the sound of a conch, the fifth the sound of strings (\textit{tantrī}), the sixth the sound of clasping, the seventh the sound of a flute, the eighth the sound of the \textit{bherī}-drum, the ninth the sound of the \textit{mṛdaṅga}-drum, and tenth the sound of a cloud. Having given up the ninth, he shall practice the tenth only. 
\end{quote}

Wenn der Geist hierauf fixiert wird dann kommt Nārāyaṇatīrtha zufolge, der Geist in den Zustand der Absorption über und die mentale Aktivität schwindet. Sünde und Verdienst werden verbrannt. By the nature of pure energy (\textit{maśakti}) Sadāśiva is caused to be revealed as all-encompassing peace of mind. \footnote{\textit{Yogasiddhāntacandrikā} (Ed. p. 55): \textit{tasmān manovilīne manasi gate saṃkalpavikalpe dagdhapuṇyapāpe sadāśivo maśaktyātmanā sarvatrā 'vasthitaḥ śāntaḥ prakāśayati} | \textit{ity ādinā} |}

\subsection{Brahmayoga in the \textit{Sarvāṅgayogapradīpikā}}
\label{sundarbrahma}
Beim Brahmayoga des Sundardās in seiner \textit{Sarvāṅgayogapradīpikā} (4.25-35)\footnote{Siehe \citeauthor{burger2014sarvangayogapradipika} 2014, p. 703-704 für eine frühere Diskussion von Brahmayoga in der \textit{Sarvāṅgayogapradīpikā} auf französisch.} handelt es sich um eine Form der Kontemplation,\footnote{\textit{Sarvāṅgayogapradīpikā} 4.25c: \textit{brahmayoga kā kaṭhina bicārā} |}, die jedoch als schwierig beschrieben wird.\footnote{Ibid. 4.26a: \textit{brahmayoga ati dūrlabha kahiye} |.} Ohne Erfahrung, kann man ihr Ende nicht erreichen.\footnote{Ibid. 4.25d: \textit{anubhava vinā na pāvai pārā} || 25 ||} Sundardās beschreibt, dass nur eine selbstlose Person Brahmayoga erlangt, wer jedoch den Sinnesgelüsten hingibt, der wandert ziellos umher\footnote{Ibid. 4.26bd: \textit{paracā hoī tabahiṃ tau lahiye} | \textit{brahmayoga pāvai niḥkāmī} | bhramata su phirai indriyārāmī || 26 ||}

Es heißt in \textit{Sarvāṅgayogapradīpikā} 4.27:
\begin{quote}
\textit{brahmayoga soī bhala pāvai} | \textit{pahile sakala sādhi kari āvai} |\\
\textit{brahmayoga saba upara soī} | \textit{brahmayoga bina mukti na hoī} || 27 ||
\end{quote}
\begin{quote}
That person truly attains Brahmayoga who first masters all practices and then comes to it.  
Brahmayoga is supreme above all, [and] without Brahmayoga, there is no liberation.
\end{quote}

Mit Brahmayoga, scheint Sundardās zunächst einen Zustand zu beschreiben, da dieser erlangt werden muss. Als eigenständige Praxis ist Brahmayoga eine fortgeschrittene Form des Yoga, denn um diese auszuüben, müssen, wie Sundardās erklärt, alle Übungen gemeistert worden sein. Damit meint er wohl eine über einen längeren Zeitraum kultivierte Yogapraxis, bestehend aus den zuvor von ihm beschriebenen Yogas, die den Übenden für Brahmayoga qualifiziert. Man muss so weit auf dem yogischen Weg fortgeschritten sein, dass, wie weiter oben erwähnt, Selbstloskeit eingetreten ist und sich nicht mehr den Sinnesgelüsten hingeben darf. In den Versen 4.29-35 beschreibt Sundardās dann das was entweder als eine mystische Form der Kontemplation, oder als eine Innenschau einer mystischen Einheitserfahrung beschrieben werden kann. Dies geschiet in Form einer Verbalisierung aus der Ich-Perspektive, die anhand zweier jener Verse demonstriert werden soll: 

In \textit{Sarvāṅgayogapradīpikā} 4.29 schreibt Sundardās:
\begin{quote}
\textit{saba saṃsāra āpa maiṃ deṣai} | \textit{pūraṇa āpu jagata mahiṃ peṣai} |\\
\textit{āpuhi karatā āpuhi haratā} | \textit{āpuhi dātā āpuhi bharatā} || 29 || 
\end{quote}
\begin{quote}
  All of existence reveals itself within me, I pervade the entire universe. 
  I am the creator, I am the destroyer. I am the giver, I am the sustainer.
\end{quote}
\begin{quote}
\textit{ahaṃ abhedya achedya aleṣā} | \textit{ahaṃ agādha su akala adeṣā} | \\
\textit{ahaṃ sadodita sadā prakāśā} | \textit{sakṣī ahaṃ sarva mahiṃ vāsā} || 33 ||
\end{quote}
\begin{quote}
  I am inseparable, I am unassailable, without stain. I am unfathomable, supremely timeless, and without direction.
  I am eternally arisen, always luminous. I am the witness, dwelling in all the universe.
\end{quote}

Im letzten Vers wird Brahmayoga sogar mit dem Brahman selbst gleichgesetzt:
\begin{quote}
\textit{ahaṃ parama ānandamaya ahaṃ jyoti nija soī} |\\
\textit{brahmayoga brahmahi bhayā dubidhyā rahī na koī} || 36 ||
\end{quote}
\begin{quote}
I am supremely filled with bliss, I am the self-luminous light. Brahmayoga is Brahman itself, fear and doubt do not remain anymore.
\end{quote}

\section{13. Advaitayoga}
\label{advaitayogaintro}

Im \textit{Yogasvarodaya} ist der Begriff Advaitayoga, zumindest in der uns vorliegenden Überlieferung vollständig abwesend. Die beiden Verse, welche im \textit{Yogasvarodaya} die Gesamtzahl der fünfzehn Yogas erwähnen, nennen lediglich acht davon. Brahmayoga ist in dieser unvollständigen Liste nicht vorhanden und wird auch in der \textit{Yogasvarodaya} nicht als eigenständiges Thema eingeführt. Da Brahmayoga in allen anderen komplexen Yogataxonomien vorhanden ist, kann davon ausgegangen werden, dass auch diese Liste ein Advaitayoga impliziert.  
Advaitayoga ist im \textit{Yogatattvabindu} die dreizehnte Methode des Rājayoga. Darüber hinaus fällt der Begriff \textit{advaitayoga} im gesamten Text nicht mehr, und wird folglich nicht als eigenständiges Thema behandelt. Ähnlich wie im Fall des Śivayoga und Brahmayoga könnte Advaitayoga jedoch implizit im Text vorhanden sein, sodass eine separate Beschreibung Rāmacandra redundant erschienen sein könnte. Tatsächlich wird beispielsweise im Kontext von Sektion \uproman{21} explizit auf die Anwendung des non-dualistischen Denkens zur Erlangung von Jñānayoga hingewiesen.\footnote{\textit{Yogatattvabindu} \uproman{21}.1: \textit{ekam eva jagat paśyed viśvātmā suvibhāsvaram} | \textit{avikalpatayā yuktyā jñānayogaṃ samācaret} || ``He shall see the world as only one, illumined by the supreme self. By the method of non-dualistic thinking, he shall accomplish Jñānayoga.''} Rāmacandra sagt außerdem kurz darauf, dass derjenige, der sich stets der Nicht-Dualität widmet immer die Wirklichkeit des Śaṃbhu erlangt.\footnote{Ibid. \uproman{21.3ab}: \textit{prāpnoti śāmbhavīṃ sattāṃ sadādvaita parāyaṇaḥ} |} Einerseits involviert Jñānayoga bei Rāmancandra die Anwendung des non-dualistischen Denkens, andererseits verortet Nārāyaṇatīrtha sowohl Jñānayoga als auch Advaitayoga in seiner \textit{Yogasiddhāntacandrikā} im Kontext seines Kommentares von \textit{Yogasūtra} 1.28. Beide Methoden basieren für Nārāyaṇatīrtha auf dem Murmeln (\textit{japa}) des Mantras \textit{oṃ} bzw. \textit{praṇava}. Sie unterscheiden sich nur hinsichtlich ihrer Methode der begleitenden Kontemplation. Jñānayoga ist die dazugehörige Kontemplation, welche den Fokus auf den Unterschied zwischen Bewusstsein (\textit{puruṣa}), Urnatur \textit{prakṛti} und ihre Effekte (\textit{tatkārya}), richtet. Advaitayoga hingegen, ist eine alternative Kontemplation, welche den Fokus auf den Nichtunterschied zwischen höchstem Selbst (\textit{paramātman}) und individuellem Selbst (\textit{jīva}) richtet.\footnote{\emph{Yogasiddhāntacandrikā} (Ed. p. 46): \textit{kiñ ca, japa ityanena mantrayogaḥ, arthabhāvanamityanena vivekajñānā 'bhyāsarūpo jñānayogaḥ, abhedabhāvarūpo 'dvaitayogaś ca saṃgṛhītaḥ} |} Nicht ganz unähnlich ist es in Sundardās \textit{Sarvāṅgayogapradīpikā} in der Jñānayoga und Advaitayoga in der gleichen Tetrade zusammen mit Brahmayoga angesiedelt sind. Alle drei Yogas sind Formen des Sāṃkhyayoga. Sundardās Advaitayoga wird allerdings als der finale non-duale Yogazustand dargestellt und nicht mehr als eine Methode, die angewendet werden kann, um diesen Zustand zu erreichen. Sollte Rāmacandra eine ähnliche Perspektive vertreten haben wäre es durchaus plausibel, warum er dem Advaitayoga im \textit{Yogatattvabindu} keine eigene Sektion gewidmet hat, auch wenn man im \textit{Yogasvarodaya} vergeblich nach Advaitayoga sucht. Somit ist es allein die \textit{Yogasiddhāntacandrikā}, welche im Rahmen der frühneuzeitlichen Texte mit komplexen Taxonomien eine explizite Methode des Advaitayoga beinhaltet. Da Advaitayoga in der \textit{Yogasiddhāntacandrikā} bereits im Kontext der Analyse von Jnānayoga auf p.\pageref{jnanayogaintrocandrika} abgedeckt wurde, muss dies an dieser Stelle nicht wiederholt werden. An dieser Stelle steht somit lediglich die Bestimmung des Advaitayogas in der \textit{Sarvāṅgayogapradīpikā} aus.

\subsection{Advaitayoga in the \textit{Sarvāṅgayogapradīpikā}}
\label{sundaradvaita}
Sundardās Beschreibung von Advaitayoga (4.37-50)\footnote{Siehe \citeauthor{burger2014sarvangayogapradipika} 2014, p. 703-704 für eine frühere Diskussion von Advaitayoga in der \textit{Sarvāṅgayogapradīpikā} auf französisch.} folgt unmittelbar auf seine Beschreibung von Brahmayoga. Hierbei handelt es sich nicht, wie bereits erwähnt, um eine Praxis, sondern vielmehr um den finalen Yogazustand, dessen Beschreibung, bereits in den Versen 4.30-36 eingesetzt hat. War die mystische Erfahrung, welche in den Brahmayoga-Versen beschrieben wurde und sich als unendlich und absolute Einheitserfahrung in Worte fassen lässt, noch im Bereich des Begreifbaren, so lässt Sundardās den Leser durch seine Formulierungen mit in die endgültige Auflösung des Zustandes der Non-dualität eintauchen, den finalen Yogazustand seiner Ausführungen. Das Advaitayoga ist also das unmittelbare Resultat der vorangehenden Kontemplation des Brahmayoga. Mittels zahlreicher Negationen versucht Sundardās dem Leser zu zeigen, was jenseits jeder Form der Beschreibung oder des Begreifens liegt. Dies lässt sich anhand einiger Beispiele gut veranschaulichen:

\begin{quote}
\textit{aba advaita sunahuṃ ju prakāsā} | \textit{nāhaṃ nā tvaṃ nāṃ yahu bhāsā} |
\textit{nahiṃ prapaṃca tahāṃ nahīṃ pasārā} | \textit{na tahāṃ sṛṣṭi na sirajanahārā} || 37 ||
\end{quote}
\begin{quote}
Now listen to the realisation of non-duality: there is no ``I'', no ``you'' and nothing that arises. There is no mundane illusion, no spaciousness, no creation and no creator.
\end{quote}
\begin{quote}
\textit{na tahāṃ prakṛti puruṣa nahiṃ icchā} | \textit{na tahāṃ kāla karma nahiṃ vaṃchā} |
\textit{na tahāṃ śūnya aśūnya na mūlā} | \textit{na tahāṃ sukṣma nahīṃ sathūla} || 38 ||
\end{quote}
\begin{quote}
There, neither primordial nature nor consciousness exists, there is no desire. There, neither time nor activity nor aspirations exist. There, neither void nor non-void is the root. There, neither subtle nor gross matter exist.
\end{quote}
\begin{quote}
\textit{na tahāṃ bhāva nahīṃ tahāṃ bhaktī} | \textit{na tahāṃ mokṣa nahīṃ tahāṃ muktī} |
\textit{na tahāṃ jāpya nahīṃ tahāṃ jāpī} | \textit{na tahāṃ mantra nahīṃ laya thāpī} || 46 || 
\end{quote}
\begin{quote}
There, neither existance nor devotion exists. There, neither liberation nor salvation exists. There, neither the recitation nor the one who recites exists. There, neither Mantra nor absorption established exists.
\end{quote}

Es folgen diverse weitere Verneinungen, die auch spezifische Yogapraktiken vernehren:
\begin{quote}
\textit{na tahāṃ sādhaka siddha samādhī} | \textit{na tahāṃ yoga na yuktyārādhī} | 
\textit{na tahāṃ mudrā baṃdhana lāgai} | \textit{na tahāṃ kuṇḍalinī nahīṃ jāgai} || 47 ||
\end{quote}
\begin{quote}
There, neither the practitioner nor the accomplished dwelling in \textit{samādhi} exists. There, neither Yoga nor the means of worship exists. There, neither seals nor locks apply. There, the Kuṇḍalinī does not awaken. 
\end{quote}

Abschließend heißt es:

\begin{quote}
\textit{jñe jñātā nahiṃ jñāna tahaṃ dhye dhyātā nahiṃ dhyāna} |
\textit{kahanahāra sundara nahīṃ yaha advaita baṣāna} || 50 || 
\end{quote}
\begin{quote}
There, neither the knower, the known, nor knowledge exists. There, neither the meditator, the meditated upon, nor meditation exists. Sundar says, there is no speaker; this is the abode of non-duality.
\end{quote}

Strukturell, ist Advaitayoga zusammen mit Jñānayoga und Brahmayoga innerhalb der Oberkategorie Sāṅkhyayoga angesiedelt. Sundardās zeichnet ein Bild der Progression durch diese vier Yogas. Sāṅkhyayoga lehrt zunächst den Unterschied zwischen Selbst und Nicht-Selbst, die Lehre des Dualismus zwischen Bewusstsein und Materie anhand der Perspektive des klassichen Sāṅkhyasystems. Das Ziel von Sāṅkhyayoga ist es diese Dualität, als den Unterschied zwischen dem was Selbst ist und dem was Nicht-Selbst ist, zu erkennen. Darauf folgt Jñānayoga, dessen Anschauung die Perspektive grundliegend ändert. Es kommt zu einer Verschiebung von der Dualität zu einer Identifikation. Das Ziel von Jñānayoga ist es die Nicht-Verschiedenheit von Selbst (\textit{ātman}), Körper und der Welt zu erkennen. Erst nachdem der Übende diese fundamnetale Einheit erkannt hat, kann er mittels Brahmayoga die gesamte Welt in sich selbst erkennen. Schlussendlich wird im daraus resultierenden Advaitayoga die der Zustand der Dualität und der Begrifflichkeiten überwunden und alle Gegensätze lösen sich auf. Der Übende ist von der Welt losgelöst. Er ist allen existierenden Phänomenen gegenüber gleichmütig, ohne deren Existenz zu verneinen. Alle Yogas, die Sundardās im Rahmen der zwölf Yogas beschreibt, zielen letztendlich auf diesen non-dualen Zustand ab. Der Zustand des Advaitayoga, in welchem die Dualität überwunden ist, existieren keine beschränkenden Konzepte mehr und der Übende befindet sich im Zustand der finalen Befreiung.  
\newpage
\section{14. Siddhayoga}
\label{siddhayogaintro}

Siddhayoga is the fourteenth method of Rājayoga in Rāmacandra's \textit{Yogatattvabindu}. The text itself describes two types of Siddhayoga. In the \textit{Yogasvarodaya} it is entirely absent. It does neither appear within it's list, nor within the rest of the text. Nārāyaṇatīrtha describes a Siddhiyoga which occupies position number eleven in his \textit{Yogasiddhāntacandrikā}. Sundardās does not include a Siddhayoga or a Siddhiyoga in his \textit{Sarvāṅgayogapradīpikā}.       

\subsection{Siddhakuṇḍalinīyoga and Siddhayoga in the \textit{Yogatattvabindu}} 

In \textit{Yogatattvabindu} Sektion \uproman{3} wird ein Yoga beschrieben, welches als Siddhakuṇḍalinīyoga (``Das Kuṇḍaliṇīyoga der Siddhas'') bezeichnet wird.\footnote{Siddhas, often called masters of yogic and tantric practices, are highly renowned figures who cannot be confined to a single religious tradition or order. These accomplished practitioners appear in medieval Sanskrit and Tibetan texts associated with Haṭhayoga, Śaiva Tantra and Vajrayāna Buddhism, spanning the Indian subcontinent and the Himalayan regions. For example, the \emph{Haṭhapradīpikā} (1.4-9) is an early fifteenth-century text that provides a famous list of Siddhas. Svātmārāma, the author, refers to a lineage beginning with Ādinātha and Matsyendranātha. However, he lists twenty-nine great adepts (\textit{mahāsiddha}s) who are described as ``used the power of Haṭhayoga to smash the rod of death and [so] are roaming the worlds''. Although Nātha figures such as Gorakṣa and Cauraṅgī are included, the list is not exclusive to the Nātha order. It is not a traditional lineage or order of succession. Many of the personalities listed, such as Manthānabhairava, Kākacaṇḍīśvara and Pūjyapāda, are associated with the alchemical traditions of the Rasāyana Siddhas. Figures such as Virūpākṣa are revered in both the Śaiva and Buddhist traditions. Therefore, Siddhas embody the ideals of Tantra and Haṭhayoga and illustrate the different sectarian roots of these practices. Cf. \citeauthor{powell2023} 2023, pp. 35-36.} Das Vorhandensein des zweiten Gliedes des Kompositums \textit{°kuṇḍalinī°} ist nur schwer zu erklären, da \textit{kuṇḍalinī} weder in den zu diesem Yoga zugehörigen Sektionen, noch im Rest des Textes erwähnt wird. Siddhakuṇḍalinīyoga wird außerdem unmittel zusammen mit Mantrayoga genannt.\footnote{Der Aspekt des Mantrayoga und die sich aus dem dem Begriff in diesem Kontext ergebende Problematik wurde bereits ausführlich im dazugehörigen Kapitel auf p.\pageref{bindumantra} diskutiert.} Im Yogasvarodaya wird die entsprechende Passage in der Überlieferung des \textit{Prāṇatoṣiṇī} (Ed. pp. 831-23) als Jñānayoga gekennzeichnet. Inhaltlich sind beide Passagen jedoch größtenteils identisch. Rāmcandra scheint nur dem Namen ausgetauscht zu haben. Bevor wir uns der Frage widmen, warum dieses Yoga den Namen Siddhakuṇḍalinīyoga, soll zunächst dessen Praxis charakterisiert werden. 

Diese Sektion rund um Siddhakuṇḍalinīyoga beschreibt die Namen und den Verlauf der drei Hauptkanäle des yogischen Körpers: Iḍā (linker Kanal), Piṅgalā (rechter Kanal) and Suṣumnā (zentraler Kanal). Rāmacandra hebt die Wichtigkeit des zentralen Kanals hervor indem er erklärt, dass der zentrale Kanal Genuss und Befreiung gewährt (\textit{bhuktimuktipradā}). Daraufhin erklärt Rāmacandra dass der Übende Allwissenheit erlangt, sobald das Wissen über den zentralen Kanal entsteht. Dies leitet dann in die nachfolgenden Sektionen \uproman{4}-\uproman{12} über in denen ein System bestehend aus insgesamt neun \textit{cakra}s beschrieben wird.\footnote{Die Rezeptionsgeschichte und Genese des neunfachen \textit{cakra} System wurde überzeugend von \citeauthor{powell2023} dargelegt und muss daher hier nicht wiederholt werden. Cf. \citeauthor{powell2023} 2023, pp. 215-218.} DieOPräsentation der \textit{cakra}s wird mit folgendem Statement eingefürt: ``Now, the means for the genesis of knowledge of the central channel are described.''\footnote{\emph{Yogatattvabindu} Sektion \uproman{4}: \textit{idāṇīṃ suṣumṇāyāḥ jñānotpattāv upāyāḥ kathyante} |} Über ein jedes \textit{cakra} soll meditiert werden, woraufhin extravagante Resultate entstehen:
\newpage 
\footnotesize
\begin{longtable}{|>{\raggedright\arraybackslash}p{1.75cm}|>{\raggedright\arraybackslash}p{1.75cm}|>{\raggedright\arraybackslash}p{3.5cm}|>{\raggedright\arraybackslash}p{3.5cm}|}
\caption{The nine \textit{cakra}s of Siddhakuṇḍalinīyoga} \label{cakra_table} \\
\hline
\textbf{Name} & \textbf{Location} & \textbf{Focus of Meditation} & \textbf{Result of the Meditation} \\
\hline
\endfirsthead
\caption[]{(continued)} \\
\hline
\textbf{Name} & \textbf{Location} & \textbf{Focus of Meditation} & \textbf{Result of the Meditation} \\
\hline
\endhead
\hline
\multicolumn{4}{|r|}{Continued on next page} \\
\hline
\endfoot
\hline
\endlastfoot
1. \textit{mūlacakram} & At the beginning of the central channel. & In its middle is kāmapīṭha in the shape of a triangle. In the middle of this seat (pīṭha) exists a single form in the shape of a flame of fire. & Any literature, [such as] śāstras, poetry, drama, etc., appears in the person’s mind without learning. \\
\hline
2. \textit{svādhiṣṭhānacakram} [divine seat of \textit{uḍḍīyāṇa}] & Penis & In its middle exist an extremely red light. & The adept becomes very handsome. \\
\hline
3. \textit{nābhisthāne padmam} & Navel & In its middle exists a \textit{cakra} with five angles. In the middle of it is a single form. It is not possible to describe its splendour. & The body of the person becomes durable. \\
\hline
4. \textit{hṛdayamadhe kamalam} [\textit{anāhatacakra}] & Heart & In its middle exists an eight-petalled lotus facing downwards. Within the eight-petalled lotus [which is within the twelve petalled lotus] is a central receptacle (\textit{karṇikā}) in the form of a \textit{liṅga}. Within the bud is a single thumb-sized figurine (\textit{puttalikā}), the embodied soul (\textit{jīva}). & The women of the inhabitants of the world [which are] Humans, Gandharvas, Kinnaras, Guhyakas, Vidyādharas, in the heavenly world, underworld, and open space become obedient to the will of the practicing person. \\
\hline
5. \textit{kaṇṭhasthāne kamalam} & Throat & In its middle exists the one consciousness (\textit{puruṣa} shining like a thousand moons. & All diseases which are [otherwise] not possible to be controlled vanish. The person lives up to 1001 years. \\
\hline
6. \textit{ājñācakram} & Middle of the eyebrows. & In its middle exists a certain object in the form of a blazing fire without parts. & The body of the person becomes non-aging and immortal. \\
\hline
7. \textit{cakram tālumadhye} & In the middle of the palate. & In its middle exists a unique red central receptacle named ``the little bell'' (\textit{ghāṇṭikā}). In its centre is a site. In the middle of that exists the hidden digit of the moon, which is oozing a stream of nectar. & As a result of meditation on this digit, death does not reach him. As a result of uninterrupted meditation, the stream (\textit{dhārā}) of nectar flows. \\
\hline
8. \textit{aṣṭamacakra brahmarandhrasthāne} [divine seat \textit{jālandhārapītha}] & aperture of Brahman (fontanelle on the head) & In middle of it, there is a streak looking like the form of smoke and fire, and in such a way, the unique image of the person exists. & Direct perception of both the coming and going of the soul in space. Affliction from the earth-element does not arise [anymore] even if one is within the earth. One constantly sees everything direct [and] one becomes separate [from matter]. The span of life increases greatly. \\
\hline
9. \textit{mahāśūnyacakram} and \textit{mahāsiddhacakram} [divine seat of \textit{pūrṇagiri}] & somewhere above the previous \textit{cakra} & (\textbf{A}) In the middle is a single upward-facing extremely red thousand-petalled lotus. In centre of this lotus exists one central receptacle in the shape of a triangle. In the middle of that central receptacle exists the seventeenth digit. (\textbf{B}) Above that is the place of infinite supreme bliss. There exists the upper power (\textit{ūrdhvaśakti}) as a unique digit. & (\textbf{A}) Suffering does not arise in the mind of the practitioner. (\textbf{B}) Whatever the person wants arises. Even though [one is] enjoying royal pleasures, amusing oneself amongst women and watching musical performances, the digit of the person grows daily like the digit of the moon in the bright half of the month. His body is not affected by merit and sin. As a result of uninterrupted meditation [onto this \textit{digit}], the ability to illuminate one's own nature arises. He sees remote objects as if they were near. \\
\end{longtable}
\normalsize 

Warum wird das Yoga mittels der Meditation über die \textit{cakra}s von Rāmacandra als Siddhakuṇḍalinīyoga bezeichnet? Eine sehr einfache Erklärung wäre die Verderbnis eines frühen Archetypen des \textit{Yogatattvabindu} von dem alle überlebenden Textzeugen abstammen. Hätte Rāmacandra dieses Yoga einfach als Siddhayoga bezeichnet, wäre die Passage völlig unproblematisch. 

Die Sektionen \uproman{3}-\uproman{12} des \emph{Yogatattvabindu} sind weitestgehend eine Prosaisierung der \textit{Yogasvarodaya}. Im Gegensatz zum \emph{Yogatattvabindu} fällt hier jedoch einmal der Name \textit{kuṇḍalī} und zwar im Kontext des vierten \textit{cakra}s im Herzen.\footnote{\emph{Yogasvarodaya} (PT Ed. p. 832): \textit{prāṇavāyoḥ sthalañcāsya liṅgākāran tu karṇikā} | \textit{kālikākhyā karṇikeyaṃ asyā madhye tu kuṇḍalī} |.} Es ist rätselhaft, warum Rāmacandra in seiner Prosaisierung dieser Passage ausgerechnet den Begriff \textit{kuṇḍalī} nicht übernimmt. Daher könnte eine weitere durchaus realistische Erklärung einfach mangelnde Sorgfalt beider Niederschrift des Textes sein. Denn die Passage hat weiterhin deutlich Einflüsse der \textit{Siddhasiddhāntapaddhati}.\footnote{Dies zeigt sich beispielsweise an der Inklusion des Konzeptes der \textit{ūrdvhaśakti} im Rahmen des neunten \textit{cakra}s in Sektion \uproman{12}.} Und es ist auffällig, dass auch die \textit{Siddhasiddhāntapaddhati} im Kontext des dritten \textit{cakra}s am Nabel die \textit{kuṇḍalinī} verortet, ein Konzept das ebenfalls nicht von Rāmacandra übernommen wird.\footnote{Cf. \citetitle{ssplonavla} 2.3: \textit{tṛtīyaṃ nābhicakraṃ pañcāvartaṃ sarpavat kuṇḍalākāram} | \textit{tanmadhye kuṇḍalinīṃ śaktiṃ bālārkakoṭisannibhāṃ dhyāyet} | \textit{sā madhyā śaktiḥ sarvasiddhidā bhavati} || 2.3 ||} Darüber hinaus ist \textit{kuṇḍalinī} insgesamt ein wichtiges zentrales Element in der Metpahysik der Nāths.\footnote{Cf. for example \textit{Siddhasiddhāntapaddhati} 1.7, 1.12, 1.14, 2.3, 4.21.} Die Niederschrift der \textit{Siddhasiddhāntapaddhati} markiert laut \citeauthor{mallinsonnath} (2011:20) den Moment als die Nāth Sampradāya eine solide sektarische Identität zu schaffen. Außerdem führt sich die heutige Nāth Sampradāya selbst auf die sog. ``neun Nāths'' zurück, eine Liste von Siddhas mit Namen, die sich sehr an frühen Listen von Siddhas orientiert.\footnote{Cf. \citeauthor{mallinsonnath} 2011, p. 5.} Eine berühmte mit den Siddhas der Nāths in Verbindung gebrachte Yogatechnik ist die Erweckung der \textit{kuṇḍalinī}, welche, als Resultat ihrer Erweckung, dann durch eine bestimmte Anzahl von \textit{cakra}s den zentralen Kanal nach oben aufsteigt.\footnote{Cf. \textit{Siddhasiddhāntapaddhati} 6.86: \textit{śaktyākuñcanam agnidīptikaraṇaṃ tv ādhārasaṃpīḍaṇāt sthānāt kuṇḍalinīprabodhanam ataḥ kṛtvā tato mūrdhani} || \textit{nītvā pūrṇagiriṃ nipātanam adhaḥ kurvanti tasyāś ca ye khaṇḍajñānaratās te nijapadaṃ teṣāṃ hi dūraṃ padam} || Cf. also \citetitle{yogatarangini} 1.48-49: \textit{kiṃ ca, yena dvāreṇa gantavyaṃ brahmasthānam anāmayam} | \textit{mukhenācchādya taddvāraṃ prasuptā parameśvarī} || 48 || \textit{yena dvāreṇa yena mārgeṇa kṛtvā anāmayaṃ jananamaraṇādiduḥkhacintārahitaṃ brahmasthānam akhaṇḍānandapadaṃ gantavyaṃ taddvāraṃ mukhenācchādya prasuptā parameśvarī kuṇḍalī śaktiḥ} ||48|| \textit{prabuddhā vahniyogena manasā marutā saha} | \textit{sūcīvad guṇam ādāya vrajaty ūrdhvaṃ suṣumnayā} || 49 || \textit{vahniyogena prāṇapreritānalaśikhāsambandhena kṛtvā prabuddhā tyaktanidrā satī manasā marutā prāṇena ca saha yuktā suṣumnāyāvadhyanāḍyā kṛtvā ūrdhvaṃ sahasradalābhimukhaṃ vrajati} | \textit{dṛṣṭāntam āha—sūcīvad iti yathā sūcī svasaṃktaṃ guṇam ādāya ūrdhvaṃ paṭasya prati tantvantarālaṃ vrajati tadvad iyam api svakalpitaṣaṭcakraṃ tad adhitiṣṭhati tat tad evatādi sakalaprapañcaṃ saṃhṛtya vrajati} || 49 ||} Der Begriff \textit{siddhakuṇḍalinīyoga} kann daher einzig vor diesem Hintergrund mit einer Beschreibung eines \textit{cakra}-Systems verwendet worden sein. Diese Assoziation ist somit völlig nachvollziehbar. Nicht nachvollziehbar bleibt der Umstand, warum Rāmacandra \textit{kuṇḍalinī} in seinen Ausführungen nicht mehr erwähnt und seine Praxis völlig ohne diese auskommt und sich die Praxis seines Siddhakuṇḍalinīyoga auf Meditationen über die einzelnen \textit{cakra}s beschränkt. \\

In Sektion \uproman{44} findet sich die zweite, und diesmal eindeutige Erwähnung von Siddhayoga inklusive einer Beschreibung der Eigenschafen die einen Siddhayogin ausmachen. Diese Passage basiert diesmal nicht auf den Ausführungen der \textit{Yogasvarodaya}, sondern die meisten der hier wiedergegebenen Verse entstammen der \textit{Siddhasiddhāntapaddhati} mit teils deutlichen redaktionellen Änderungen. Einige Verse dieser Passgae stammen wahrscheinlich sogar aus Rāmacandras eigener Hand. Die gesamte Sektion dreht sich um die Charakteristik einer Avadhūta-Person (\textit{avadhūtapuruṣa}).\footnote{Eine rezente Diskussion der rezeptionsgeschichtlichen Entwicklung des Begriffes \textit{avadhūta} findet sich bei \fullcite{pudi2023}.} In den letzten drei Versen dieser Passage des Textes wird der Avadhūta mit einem Siddhayogin, im Sinne eines ``Yogin, der den Yoga vollendet hat'' gleichgesetzt. Mittels Siddhayoga wird man zum Siddhayogin:

\begin{quote}
  \textit{viśvātītaṃ tayā viśvam ekam eva virājate} |
  \textit{saṃyogena sadā yasya siddhayogī sa gadyate} || \uproman{44}.8 ||
\end{quote}
\begin{quote}
He is called a Siddhayogin for whom always, by means of Yoga, the
universe as such shines forth as one by means of transcending the universe.
\end{quote}
\begin{quote}
  \textit{sarvāsāṃ nijavṛttīnāṃ vismṛtiṃ bhajet tu yaḥ} |
  \textit{sa bhavet siddhasiddhānte siddhayogī sa gadyate} || \uproman{44}.9 ||
\end{quote}
\begin{quote}
He who obtains oblivion from all inherent fluctuations [of the mind], he is called a Siddhayogin according to the doctrine of the Siddhas.
\end{quote}
  \begin{quote}
  \textit{udāsīnaḥ sadā śānto mahānandamayo 'pi ca} |
  \textit{yo bhavet siddhayogena siddhayogī sa kathyate} || \uproman{44}.10 ||
\end{quote}
\begin{quote}
One who is always indifferent, peaceful and immersed in great bliss
by means of Siddhayoga\footnote{Interestingly, the term \textit{siddhayogena} is not attested in the \textit{Siddhasiddhāntapaddhati}.} is said to be a Siddhayogin.
\end{quote}

Thus, a Siddhayogin has realized the unity within the Universe, has stilled his mind, and is always indifferent, peaceful and immersed in great bliss. He has attained all this by means of Siddhayoga.

Unfortunately, there are no clear instructions or explicit descriptive statements which would define the practice of Siddhayoga. Nonetheless, it is possible to derive them from the previous statements.

Verse \uproman{44}.2 sagt beispielsweise, dass des Avadhūtas Almosen\footnote{Ursprünglich galt der Avadhūta als antinomischer Asket, der sich von allen gesellschaftlichen Banden löste und das tut was er will. Im Lauf der Jahrhunderte wird er in den Worten von \citeauthor{pudi2023} (2023) ``sanitized'' und Salonfähig. Der Avadhūta wurde in das brahmanische \textit{āśrama}-System integriert, seine unkonventionellen Züge und unorthodoxen Praktiken wurden gezähmt, und der Avadhūta wurde dadurch, so \citeauthor{pudi2023}, zu einem legitimen und schließlich sogar zur höchsten Klasse des \textit{saṃnyāsa āśrama} erhoben.} ``difference and non-difference'' (\textit{bhedābheda}) seien. Eine ganz ähnliche Perspektive findet sich bereits der Jñānayoga-Sektion \uproman{21}. Zwar fällt wer Begriff \textit{bhedābheda} in dieser Sektion nicht, jedoch können wir ableiten, dass zur Praxis des Siddhayoga die Kultivierung dieser spezifischen philosophischen Perspektive gehören sollte.

In Vers\uproman{44}.3 findet sich eine Wortdeutung (\textit{nirukti}) des Begriffes \textit{avadhūta}, welche aus Rāmacandras eigener Hand stammen könnte. Der Buchstabe ``\textit{a}'' stehe für das Selbst (\textit{ātman}), der Buchstabe ``\textit{va}'' stehe für die mentalen Rückstände (\textit{vāsanā}s) und \textit{dhūta} ``shaking off'' sei seine Spezialwaffe. Somit ist ein Avadhūta/Siddhayogin jemand, der sich einer Yogapraxis widmet, welche dafür sorgt die \textit{vāsanā}s zu reduzieren. Dies ist eine weiterer Hinweis auf das Konzept, welche hinter dem Begriff Siddhayoga stehen dürfte.

Darüber hinaus lassen sich keine weiteren Aussagen treffen, die ein Siddhayoga anhand dieser Passage weiter spezifizieren könnten. Zusammenfassend lässt sich ableiten, dass Siddhayoga aus einer Methode der Reduktion der \textit{vāsanā}s bestehen dürfte, sowie eine spezifische Form der Philosophie einstudiert wird, vermutlich die Philosophie der Doktrin der Siddhas (\textit{siddhasiddhānta}). Diese nimmt hier eine universalistische Ausprägung an, heißt es doch in \uproman{44}.5, dass der Avadhūta sich am Ort des universellen Geistes sich befindet (\textit{nirākārapade sthitaḥ}) und sich alle philosophischen Ansichten in seiner eigenen essenziellen Natur offenbaren (\textit{sarveṣāṃ darśanānāṃ ca svasvarūpaṃ prakāśate}). 

\subsection{Siddhiyoga in the \textit{Yogasiddhāntacandrikā}}

Siddhiyoga in der \textit{Yogasiddhāntacandrikā} bezeichnet ein fortgeschrittenes Stadium der Yogapraxis auf dem Weg zu \textit{samādhi}. Dieses Stadium setzt ein, wenn ein hoher Grad der Meisterschaft über die Stabilität des Geistes erreicht wurde. Ausgehend von dieser Meisterschaft können diverse übernatürliche Fähigkeiten durch bestimmte Übungen, wie Askese (\textit{tapas}) oder Meditationsübungen (\textit{saṃyama}s) erlangt werden, die bereits in zweiten Kapitel (\textit{sādhanapāda}), aber vor allem im dritten Kapitel (\textit{vibhūtipāda} des \textit{Pātañjalayogaśāstra}, dargestellt werden. Dieses Stadium ist laut Nārāyaṇatīrtha förderlich für die Erlangung von \textit{samādhi}.\footnote{Die Beschreibung des Siddhiyoga in der \textit{Yogasiddhāntacandrikā} von \citeauthor{penna2004} (2004, pp. 84-85) stellt Siddhiyoga einzig als einen Zustand dar, welcher \textit{samādhi} begünstigt, lässt aber den praktischen Aspekt des Siddhiyoga, die Erlangung von bestimmten überntürlichen Fähigkeiten (\textit{siddhi}s völlig außer acht.}    

Wie bereits alle anderen Yogas verortet Nārāyaṇatīrtha auch Siddhiyoga im ersten Kapitel, nämlich im Rahmen seines Kommentares zu \textit{sūtra} 1.40:

\begin{quote}
  \textit{cittasthitijayasya jñāpakaṃ siddhiyogaṃ samādhy anukūlam āha} -\\
\textit{paramāṇuparamamahattvānto 'sya vaśīkāraḥ} || 40 ||
\end{quote}
\begin{quote}
It is said that Siddhiyoga indicates mastery of the stability of the mind, which is conducive to \textit{samādhi} -

His (the yogins) mastery extends from the smallest particle of matter up the greatest extend. \end{quote}

Nārāyaṇatīrtha erklärt, dass die Kontrolle des Yogis die volle, ungehinderte Kontrolle über alles umfasst. Diese Beherrschung ist förderlich für \textit{samādhi}, weil sie Leiden wie Hunger und Durst unterbindet. Sie zeigt an, dass der Geist beherrscht vom Yogin beherrscht wird.\footnote{Cf. \textit{Yogasiddhāntacandrikā} (Ed. p. 106): \textit{parameti} | \textit{asya sthiracittasya yoginaḥ paramamahattvāntaḥ paramamahattvaṃ yeṣāṃ viyatpuruṣādīnāṃ tatparyanto vaśīkāraḥ apratighātaḥ kenāpy apratibandhyatā} | \textit{saiva kṣutpipāsādi- pīḍāpratibandhadvārā samādhisādhikā cittajayasya ca jñāpiketi bhāvaḥ} |}

Zum Ende seines Kommentares zu \textit{sūtra} 1.40 erklärt Nārāyaṇatīrtha, dass dieses fortgeschrittene Stadium der Yogapraxis, mittels verschiedener Methoden erreicht werden kann und zur höchstmöglichen Stärke (\textit{dṛḍhatā parā}) führt.

\begin{quote}
  \textit{ayam eva siddhiyogaḥ}
  \textit{prāṇaspandanirodhādyair upāyair dṛḍhatā parā} | 
  \textit{siddhiyogo bhaved atra yogaḥ siddhikaraḥ paraḥ} ||
  \textit{ityādinā} || 40 ||
\end{quote}
\begin{quote}
By means of such as the cessation of the motion of the breath, supreme strength arises. Here,Siddhiyoga means the supreme Yoga that brings about accomplishment. Thus it is said.
\end{quote}

Als nächstes verknüpft Nārāyaṇatīrtha Siddhiyoga mit spezifischen übernatürlichen Fähigkeiten (\textit{siddhi}s) wie er in seinem Kommentar zu \textit{sūtra} 2.35 illustriert. Siddhiyoga wird hier am Beispiel des Resultates der Praxis von nicht-Gewalt (\textit{ahiṃsā}) veranschautlicht.\footnote{Die aus der Praxis von \textit{ahiṃsā} resultierende übernatürliche Fähigkeit ist die Erzeugung einer Sphäre der Gewaltlosigkeit. Dem Yogin, dessen \textit{ahiṃs} perfektioniert ist, kann keine Gewalt mehr wiederfahren. Dies ist übrigens die erste übernatürliche Fähigkeit die sich für den Yogin der Pātañjalayoga praktiziert, einstellt.} Sobald für den Yogin durch die Praxis von \textit{pratipakṣabhāvana}\footnote{The intentional cultivation of opposing thoughts, e.g. when one feels sorrow or anxiety and then concentrates on positive or pleasurable thoughts.} Stabilität in der nicht-Gewalt erreicht wurde, erzeugt er eine Sphäre der Gewaltlosigkeit, in der keine Feindschaft existiert. Diese Sphäre der nicht-Gewalt neutralisiert sogar natürliche Feindschaften, wie z.B. die von Mungo und Schlange, wie uns Nārāyaṇatīrtha wissen lässt.\footnote{Cf. \textit{Yogasiddhāntacandrikā} (Ed. p. 104): \textit{evam ahiṃsāparasya yoginas tatphalabhūtaṃ siddhiyogaṃ darśayati}-\textit{ahiṃsāpratiṣṭhāyāṃ tatsannidhau vairatyāgaḥ} || 35 || \textit{ahiṃseti} | \textit{uktapratipakṣabhāvanenāhiṃsāsthairye sati tatsannidhau ahiṃsāṃ bhāvayataḥ samīpe vairatyāgaḥ} | \textit{sahajavirodhinaḥ mahinakulādīnām api nirmatsaratayā 'vasthitir bhavatīty arthaḥ} || 35 ||}

Schließlich begegnen wir Siddhiyoga in Nārāyaṇatīrthas Kommentar zu \textit{sūtra} 2.43, in dem Siddhiyoga explizit mit einem breiteren Spektrum der Erlangung von übernatürlichen Fähigkeiten (\textit{siddhis}) verknüpft wird. Insbesondere ist hier Askese (\textit{tapas}) der entscheidende Katalysator für den Erfolg in Siddhiyoga. Erst hier wird der volle Umfang des Konzeptes von Siddhiyoga verständlich indem an dieser Stelle explizit die Erklärungen von \textit{sūtra} 1.40 aufgreift und erweitert:

\begin{quote}
\textit{tapaḥ sādhyāṃ siddhim āha}- \\
\textit{kāyendriyasiddhair aśuddhikṣayāt tapasaḥ} || 43 ||
\end{quote}
\begin{quote}
  Asceticism is said to cultivate perfection:\\
  Perfection of the senses and the body manifests as a result of asceticism on account of the removal of impurities. || 43 ||
\end{quote}
\textit{kāyeti} | \textit{tapasaḥ tapo 'bhyasād, aśuddhikṣayād yatheṣṭagatyādipratibandhaka pāpamalāder nāśāt, kāyendriyasiddhiḥ kāyendriyāṇām alpatvamahattvadūrārthadarśitvādisāmarthyarūpā siddhir bhavatīty arthaḥ} | \textit{kāyasyātilāghavena dūradeśagamanādikaṃ dharmaviśeṣāyattaṃ mahattvenānyair abādhyatvādi ca bhavati} | \textit{indriyāṇāṃ dūrārthasūkṣmārthavyavahitānekārthagrāhitā bhavatīti yāvat} | \textit{etena paramāṇuparamamahattvānto 'sya vaśīkāra iti sūtreṇokta- siddhiyogasyātrāntarbhāvo jñātavyaḥ} |
\begin{quote}
[Regarding the term] \textit{kāya} (``body''). As a result of the practice, the heat of asceticism (\textit{tapas}) arises (\textit{tapas}).\footnote{The heat \textit{tapas} bakes the body and destroys impurities that are mentioned in the following.} As a result of the destruction of impurities and other obstructions such as sin and filth, which hinder free movement and the like, perfection of the senses and the body manifests, meaning the ability of the body and senses to become small, large, see distant objects, etc. Through the extreme lightness of the body, there is the ability to travel to distant places, etc. and through other capacities dependent on special qualities, unobstructedness, etc., arises. To be precise, the [ability] of the senses to perceive distant, subtle, covered, and multiple objects arises. This is indicated by the \textit{sūtra} ``his control extends from the smallest atom to the greatest magnitude.'' - It should be understood that Siddhiyoga is included here. 
\end{quote}

\section{15. Rājayoga}
\label{rajayogaintro}

Rājayoga nimmt in Rāmacandras \emph{Yogatattvabindu} die fünfzehnte und somit die finale wie auch höchste Position seiner Taxonomie ein. Im \textit{Yogasvarodaya} nimmt Rājayoga eine ähnliche übergeordnete Rolle ein. In beiden Texten ist Rājayoga zunächst ein Yogaweg mit fünfzehn unterschiedlichen Methoden. Die Methoden sind einzelne Yogas mit unterschiedlichen Bezeichnungen, die jedoch alle zu Rājayoga als Zustand führen. Rājayoga somit als gleichzeitig ein Sammelbegriff für unterschiedliche Methoden und bezeichnet auch den höchsten Erlösungszustand. Anders verhält es sich in Nārāyaṇatirtha's \textit{Sarvāṅgayogapradīpikā}. Hier ist Rājayoga ein Synonym für \textit{samādhi}, im Sinne des finalen Zustandes des Pātañjalayoga. Rājayoga kann in Nārāyaṇatīrthas Text durch fünfzehn Methoden erreicht werden. In Sundardās \emph{Sarvāṅgayogapradīpikā} gilt Rājayoga zusammen mit Lakṣayoga und Aṣṭāṅgayoga als eine Unterkategorie des Haṭhayoga und bezeichnet hier vornhemlich eine Technik, die weitläufig unter dem Begriff \textit{vajrolīmudrā} bekannt ist.

\subsection{Rājayoga in the \emph{Yogatattvabindu}}

Rājayoga ist das übergeordnete Thema des \emph{Yogatattvabindu}. Rāmacandras Text zielt darauf ab die Methode des Rājayoga niederzuschreiben. Dies wird unmittelbar am Anfang der Abhandlung klargestellt. \footnote{\emph{Yogatattvabindu} section \uproman{1}: \textit{śrī ganeśāya namaḥ} || \textit{atha rājayogaprakāra likhyate} |} Rāmacandra's Rājayoga hat insgesamt fünfzehn Varianten, die aufgelistet werden: Kriyāyoga, Jñānayoga, Caryāyoga, Haṭhayoga, Karmayoga, Layayoga, Dhyānayoga, Mantrayoga, Lakṣyayoga, Vāsanāyoga, Śivayoga, Brahmayoga, Advaitayoga, Siddhayoga and Rājayoga itself. Von diesen Varianten werden jedoch nur Kriyāyoga, Jñānayoga, Caryāyoga, Haṭhayoga, Mantrayoga, Lakṣyayoga und Siddha[kuṇḍalinī]yoga explizit als Methoden mit einer eigenen Sektion eingeführt. Dhyānayoga, Vāsanayoga, Karmayoga und Advaitayoga sind zumindest implizit vorhanden. Śivayoga und Brahmayoga werden eingangs als Methode genannt, werden im Text dann aber nicht nochmal aufgegriffen. Wir können jedoch davon ausgehen, dass Śivayoga\footnote{Das Śivayoga der \textit{Śivayogapradīpikā} steht dem Gesamtinhalt des \textit{Yogatattvabindu} so nahe, dass eine synonyme Verwendung als Möglichkeit in Betracht gezogen werden muss.} und Brahmayoga\footnote{Beispielsweise schreibt Divākāra im ersten Vers des \textit{Bodhasāra}: \textit{rājayogo rājñāṃ nṛpāṇāṃ svasthāne sthitvāpi vādhayituṃ śakyatvāt tatsambandhī yogo jīvabrahmaiyaviṣayakajñānalakṣaṇo} \ldots ``Rājayoga is the Yoga of kings, because rulers can accomplish it even when [they] remain in their position (that is, as kings). In this connection, its [main] characteristic is knowledge concerning the union of the individual self with Brahman'' (translation by \citeauthor{birch2014} 2014, p. 430 n. 51). Das \emph{Yogatattvabindu} nimmt die gleiche Position ein. Rājayoga ist auch hier Yoga für Könige. Gleichzeitig ist die Hauptcharakteristik des \emph{Bodhasāra} die Vereinigung von \textit{jīva} und Brahman. Daher ist es möglich, dass Rāmacandra diese Auffassung teilte und Brahmayoga als Synonym vom Rājayoga betrachtete.} einfach als Synonyme für Rājayoga betrachtet wurden, sodass Rāmacandra nicht die Notwendigkeit sah, diese als eigenständige Kategorien einzuführen. Die Nennung von Rājayoga in der Liste für Methoden des Rājayoga erscheint redundant, wurde jedoch möglicherweise ans Ende der Liste gestellt um dessen Superiorität im Sinne des ``Königs der Yogas''\footnote{Ganz im Sinne des \emph{Amanaska} 2.3cd: \textit{rājatvāt yogānāṃ rājayoga iti smṛtā}.} auszudrücken, denn eine separate Yogamethode mit dem Namen Rājayoga suchen wir im Text vergeblich. Allerdings werden noch weitere Yogakategorien genannt, welche nicht in den eingangs erwähnten fünfzehn Methoden für Rājayoga aufgelistet wurden. Dies sind Aṣṭāṅgayoga,\footnote{Eine Diskussion von Aṣṭāṅgayoga findet sich auf S.\pageref{ashtangayoga}.} Satyayoga,\footnote{Eine Diskussion von Satyayoga findet sich auf S.\pageref{satyayoga}.} und Sahajayoga.\footnote{Eine Diskussion von Sahajayoga findet sich auf S.\pageref{sahajayoga}.} Die Gründe für deren Abwesenheit in der eingangs erwähnten Liste eindeutig nachvollziehbar, zumal Sahajayoga im letzten Satz des \textit{Yogatattvabindu} sogar als ``universal ruler among the [methods] of Rājayoga''\footnote{Cf. \emph{Yogatattvabindu} \uproman{58}: \textit{rājayogamadhye iti cakravartī nāmakathanaṃ} |} bezeichnet wird, was dessen extrem hohe, aber dennoch dem Rājayoga zugeordnete Stellung suggeriert.

Eine Besonderheit des \emph{Yogatattvabindu} ist, dass direkt in der Einleitung genannte Resultat von Rājayoga. Hier wird von ``long-term durability of the body''\footnote{\emph{Yogatattvabindu} section \uproman{1}: \ldots \textit{bahutarakālaṃ śarīrasthitir bhavati} |} gesprochen, welche, so betont Rāmacandra ganz ausdrücklich, unter den besonderen Umständen entsteht, nämlich ``even if the practitioner is enjoying manifold royal pleasures and even when there is manifold royal entertainment and spectacle.''\footnote{Ibid. section \uproman{1}: \ldots \textit{yena rājayogenānekarājyabhogasamaya eva anekapārthivavinodaprekṣaṇasamaya eva} \ldots} Der Name Rājayoga impliziert hier, dass die Übende Person der mittels der von Rāmacandra genannten Methoden des Rājayoga wie ein König leben kann und trotz exzessiven Formen des weltlichen Genusses, die positiven Effekte des Yogas erfahren, ohne der Welt zu entsagen und Asket werden zu müssen. Zahlreiche Passagen des Textes suggerieren, dass sich das Rājayoga des \textit{Yogatattvabindu} tatsächlich direkt an Könige, insbesondere angehende Könige, also junge Prinzen (\textit{kumāra}s) gerichtet haben könnte. Aufgrund des Umfangs und der Bedeutsamkeit dieses Themas, wird dies an anderer Stelle dieser Arbeit behandelt.\footnote{See p.\pageref{ytbaudience}.} Es ist jedoch wichtig zu betonen, dass der Begriff Rājayoga in diesem Text ebenfalls stets, neben den anderen genannten Konnotationen, die Bedeutung ``Yoga für Könige'' trägt.  

Darüber hinaus werden folgende Effekte bzw. Anzeichen der Rājayoga Methoden in den Sektionen \uproman{16},\footnote{\emph{Yogatattvabindu} Sektion \uproman{16}: \textit{idānīṃ rājayogayuktasya puruṣasya yac charīracihnaṃ tat kathyate |}} \uproman{17},\footnote{Ibid. Sektion \uproman{17}: \textit{anyad rājayogasya cihnaṃ kathyate} |} \uproman{42}\footnote{Ibid. Sektion \uproman{42}: \textit{idānīṃ rājayogāc charīre etādṛśāni cihnāni bhavanti} |} explizit\footnote{Indirekt haben auch die einzelnen Methoden des Rājayoga eigene Effekte.} thematisiert. Die folgende Tabelle listet diese Effekte entsprechend der Sektionen auf:
\newpage 
\begin{table}[H]
\scriptsize 
\centering
\begin{tabular}{|p{3cm}|p{4cm}|p{4cm}|}
\hline
\textbf{Section \uproman{16}}  & \textbf{Section \uproman{17}} & \textbf{Section \uproman{42}} \\
\hline
\begin{itemize}
    \item He is rich at all times.
    \item He dwells distant from the world. He dwells in the world, having permeated it.
    \item Neither birth nor death exists for him.
    \item Happiness does not exist.
    \item Suffering does not exist.
    \item Descent does not exist.
    \item Moral conduct does not exist.
    \item Abode does not exist.
    \item In the mind of this perfected one, a light appears immediately before him, which is the connection with God.
    \item Neither does he have a caste, nor does he have any sign.
    \item He is without parts, immaculate and uncharacterized.
    \item Whatever wish for the most excellent fruit, affectionate woman, etc. arises, he obtains that very enjoyment. His mind truly does not suffer attachment in this situation.
\end{itemize} & 
\begin{itemize}
    \item Even when there is the attainment of a kingdom etc., the perception of a reward does not arise.
    \item Even in loss, suffering does not arise within the mind and neither does desire arise.
    \item Even when whatever object has been obtained, aversion towards any object does not arise; and concerning this object, affection of the mind does not arise.
    \item The mind is equal towards a person who has expertise in sacred scriptures, a friend or an enemy.
    \item An indifferent view arises.
    \item When for him who freely moves across the entire world being furnished with enjoyment and happiness, the pride of the ability to do these things does not arise within the mind; and one does not proclaim the ability to do these things among all his followers—this is also said to be of Rājayoga.
    \item Whether one has new clothes made of silk, or old, worn clothes with holes, whether one is smeared with sandalwood and musk, or smeared with mud—when delight and grief do not reside within the mind, it is that which is Rājayoga.
    \item When the mind is neither bored nor overwhelmed situated in a city, a forest, an uninhabited village, or a village full of people, also this is Rājayoga.
\end{itemize} & 
\begin{itemize}
    \item The eradication of all diseases occurs.
    \item He has a vision of the entire earth.
    \item Knowledge of the principles (\textit{tattva}s) arises.
    \item He understands all languages.
    \item The body becomes as strong as a diamond.
    \item Even with the bite of a snake, death does not occur.
    \item Hunger, thirst, drowsiness, and heat do not trouble the person.
    \item Perfection of speech arises.
    \item Fatigue does not occur in the body.
    \item The person assumes the nature of the wind.
    \item He sees the entire earth with a glance.
    \item The eight supernatural powers beginning with ``becoming infinitely small'' etc. (\textit{aṇimādi}) arise.
    \item The nine treasures (\textit{navanidhi}) approach nearby.\footnote{\emph{Yogatattvabindu} Section \uproman{42}.1: \textit{mahāpadmaś ca padmaś ca śaṅkho makarakacchapau} | \textit{mukundakundanīlaś ca kharvaś ca nidhayo nava} ||}
    \item Within the ten cardinal points in space, the power over death and rebirth arises.
    \item Wherever there is a desire to go in the world, one goes there.
    \item Ignorance disappears everywhere.
    \item One sees the Supreme Lord nearby.
    \item There is the capability of accomplishing tasks and removing obstacles.
\end{itemize} \\
\hline
\end{tabular}
\caption{Effects of Rājayoga}
\end{table}
\newpage 
\subsection{Rājayoga in the \textit{Yogasvarodaya}}

Genau wie im \emph{Yogatattvabindu} hat Rājayoga im \emph{Yogasvarodaya} fünfzehn Varianten. Von diesen fünfzehn, werden jedoch nur acht Yogas genannt: Kriyāyoga, Jñānayoga, Karmayoga, Haṭhayoga, Dhyānayoga, Mantrayoga, Urayoga\footnote{The term Urayoga is possibly a corruption of the text. Jason Birch suggested emending to \textit{lakṣyayoga}, as Lakṣyayoga plays a central role in the course of the text. Karen O'Brien-Kop suggested \textit{ūha°} as a possible reading for \textit{ura°} - a term derived from the older meditation framework of Sāṃkhya, which emphasises \textit{ūha} (reflection), \textit{śabda} (speech) and \textit{adhyayana} (study). Oberhammer, for example, discusses this term in his analysis of the \emph{Yuktidīpikā} (commentary on the \emph{Sāṃkhyakārikā} from the 7th century BC). Unfortunately, the term is not found a second time in the surviving material of the \emph{Yogasvarodaya}. In view of the mention of Sāṃkhyayoga in Sundardā's \textit{Sarvāṅgayogapradīpkā}, this possibility cannot be ruled out. Unfortunately, the surviving material of the \textit{Yogasvarodaya} does not support this idea. Sven Sellmer suggested that it may not be a mistake, but an abbreviated form of \textit{uraga°}. Uragayoga translated as ``Snakeyoga'' and could be a synonym for Kuṇḍalinīyoga. However, I could not find this word attested anywhere else.} und Vāsanāyoga. Die anderen Varianten bleibenwerden, vermutlich aus metrischen Gründen, nicht genannt. In diesem Fall wird Rājayoga erneut als fünfzehnfache Methode, als auch als Zustand betrachtet. Alle fünfzehn Methode führen dazu, das die übende Person in Brahman verweilt. Der Begriff impliziert hier gleichzeitig einerseits die höchste bzw. übergeordnete Form des Yoga des zu sein, und gleichzeitig den höchsten Yogazustand. Rājayoga führt zu einem langen leben und zur Erlangung der acht übernatürlichen Fähigkeiten. Im Vergleich zum \emph{Yogatattvabindu}, das Rājayoga als ein Yoga darstellt, welches trotz der königlicher Sinnesfreuden ausgeübt werden kann, heißt es in dieser Einleitung einzig, dass die übende Person würdig von Königen verehrt zu werden.

\begin{quote}
\textit{atha rājayogaḥ} || \textit{yogasvarodaye} |\\
\textit{īśvara uvāca} |\\
\textit{rājayogaṃ pravakṣyāmi śṛṇu sarvatra siddhidam} |\\
\textit{guhyād guhyataraṃ devi nānādharmaṃ parātparam} ||\\
\textit{rājayogena deveśi nṛpapūjyo bhaven naraḥ} |\\
\textit{rājayogī cirāyuś ca aṣṭaiśvaryamayo bhavet} ||\\
\textit{pañcadaśaprakāro'yaṃ rājayogaḥ} ||\\
\textit{kriyāyogo jñānayogaḥ karmayogo haṭhas tathā} |\\
\textit{dhyānayogo mantrayoga urayogaśca vāsanā} |\\
\textit{rājaty etad brahmaśīva ebhiś ca pañcadaśadhā} ||\\
\end{quote}

\begin{quote}
Now Rājayoga. [As described] in the Yogasvarodaya.
God said:
``I will teach Rājayoga, listen! In every case it bestows completion.
[It is] more secret than secret, oh goddess, [its] nature is manifold, [and it is] higher than the highest. 
By means of Rājayoga, oh goddess, a man becomes [worthy] of being worshipped by kings.
The Rājayogin may have a long life and he may be equipped with the eight [supernatural] powers.
This Rājayoga has fifteen varieties: Kriyāyoga, Jñānayoga, Karmayoga, Haṭhayoga,
Dhyānayoga, Mantrayoga, Urayoga and Vāsanāyoga.
By [means of] these fifteen [Yogas], that [person] who is resting in Brahman shines [like a king].''
\end{quote}

Hinsichtlich der dem Rājayoga zugeschriebenen Effekten existieren im \emph{Yogasvarodaya} im Vergleich zum \emph{Yogatattvabindu} kleine nennenwerten Unterschiede.  

\subsection{Rājayoga in the \emph{Yogasiddhāntacandrikā}}

In his introduction to the first \textit{sūtra}, Nārāyaṇatīrtha takes Rājayoga as a synonym of \textit{samādhi} (``meditative absorption'') and \textit{nididhyāsana} (``profound meditation''). Later on, he equates Rājayoga more specifically with \textit{asaṃprajñātasamādhi} and \textit{nirbījasamādhi}.\footnote{\textit{Yogasiddhāntacandrikā} 1.20 (Ed. p. 25): \textit{tataḥ paravairāgyādasamprajñāta itareṣāṃ pūrvavilakṣaṇānāṃ manuṣyāṇāṃ mumukṣūṇāṃ bhavatīty arthaḥ} | \textit{ayam eva ca rājayoga ity ucyate} | \textit{tad uktaṃ smṛtau - samādhis tatra nirbījo rājayogaḥ prakīrttitaḥ} | \textit{dīpavad rājate yasmād ātmā saccinmayaḥ prabhuḥ} ||} Thus, the Rājayoga of the \emph{Yogasiddhāntacandrikā} designates the final state, the goal of the Pātañjalayoga system,\footnote{This has previously been noted within the distinguished article by Jason \citeauthor{birch2014} (2014:414-415) on the reception history of Yoga named \citetitle{birch2014}. Here, \citeauthor{birch2014} states that the earliest occurrence of the term ``\textit{rājayoga}'' in a commentary on the Yogasūtra may be Vijñānabhikṣu’s \textit{Yogasārasaṃgraha} (16th century). Here, too, Vijñānabhikṣu understood Rājayoga as \textit{samādhi}.} and not a method to achieve the state. He provides fifteen different Yogas to reach this state. All Yoga methods are, in turn, embedded within the eight limbs of Pātañjalayoga. In his commentary, Nārāyaṇatīrtha situates and explains all of them within the \textit{samādhipāda} of the Pātañjalayogaśāstra. This can best be understood from Nārāyaṇatīrtha's own words: 

\begin{quote}
\textit{brahmavid āpnoti param} | \textit{brahmavid brahmaiva bhavati} | \textit{tam eva viditvā 'timṛtyum eti nānyaḥ panthā vidyate 'yanāya} | \textit{tarati śokam ātmavit} | \textit{ity ādiśrutisiddhaparamapuruṣārtha sādhanatānandātmasākṣātkārasādhanatayā śravaṇamanananididhyāsanādīni, ātmā vā 're draṣṭavyaḥ śrotavyo mantavyo nididhyāsitavyaḥ} | \textit{ity ādinā 'mnātāni} | \textit{tatra nididhyāsanaṃ pradhānam} | \textit{tatsahakṛtād eva manaso 'laukikā 'bādhitātmagocarapramāsambhavāt, sarvavijñānādirūpaphalasaṃvādāc ca} | \textit{nididhyāsanañcaika tānatādirūpo rājayogāparaparyāyaḥ samādhiḥ} | \textit{tatsādhanaṃ tu kriyāyogaḥ, caryāyogaḥ, karmayogo, haṭhayogo, mantrayogo, jñānayogaḥ, advaitayogo, lakṣyayogo, brahmayogaḥ, śivayogaḥ, siddhiyogo, vāsanāyogo, layayogo, dhyānayogaḥ, premabhaktiyogaś ca} | \textit{tad etat sarvaṃ sāmānyaviśeṣabhāvenāṣṭāṅgayogena kavalīkṛtam iti manasi nidhāya sāṣṭāṅgaṃ saphalaṃ yogaṃ samādhisādhanavibhūtikaivalyārthakaiś caturbhiḥ pādair vyutpādayiṣyan prekṣāvat pravṛttaye viṣayaprayojanādhikārisambandhān darśayan prathamaṃ śāstrasyārambhaṃ pratijānīte bhagavān patañjaliḥ} | \textit{atha yogānuśāsanam} || 1 ||
  
\end{quote}
\begin{quote}
  The knower of Brahman attains the supreme. The knower of Brahman becomes Brahman itself. Having realized that alone, one transcends death; no other path is known. The knower of the self traverses sorrow. This is established in the scriptures as the supreme purpose of human life, and the means to realize the blissful nature of the self is hearing, reflection, profound meditation, etc. Oh, the self, indeed, must be seen, heard, reflected upon, and profoundly meditated upon. Among these, profound meditation is the most essential part. Only because of that, the extraordinary perceptions of the mind arise - as a result of the occurrence of the right idea of the dwelling place of the unobstructed self and, as a result, the information of the fruits of the first appearance of all-encompassing insight. Profound meditation, characterized by being humbly inclined towards unity, is another synonym for Rājayoga or \textit{samādhi}. The means to this include Kriyāyoga, Cāryāyoga, Karmayoga, Haṭhayoga, Mantrayoga, Jñānayoga, Advaitayoga, Lakṣyayoga, Brahmayoga, Śivayoga, Siddhiyoga, Vāsanāyoga, Layayoga, Dhyānayoga, and Premabhaktiyoga. All these are encompassed generally and specifically through the eight-limbed Yoga. Keeping this in mind, the Yoga with eight limbs which is fruitful, through the chapters regarding the subjects \textit{samādhi} (``meditative absorption''), \textit{sādhana} (``practice''), \textit{vibhūti} (\textit{supernatural powers}) and \textit{kaivalya} (\textit{isolation}) teaching the connections regarding the objective of the topic for its application in a comprehensible manner the venerable Patañjali revealing the most excellent beginning of his treatise states: Now, the teaching of Yoga begins.
  \end{quote}

\subsection{Rājayoga in the \emph{Sarvāṅgayogapradīpikā}}

Das Rājayoga des Sundardās (3.13-24) wird zusammen mit Lakṣyayoga und Aṣṭāṅgayoga in der Tetrade des Haṭhayoga subsummiert.\footnote{Für eine frühere Beschreibung auf französisch siehe \citeauthor{burger2014sarvangayogapradipika} 2014, p. 696-697.} Im Gegensatz zu Rāmacandra, der Haṭhayoga als eine Form des Rājayoga darstellt, wird Rājayoga von Sundardās als eine Form des Haṭhayoga begriffen. Bei dieser Form handelt es sich ausschließlich um das, was im Allgemeinen unter dem Namen \textit{vajrolīmudrā} bekannt ist.\footnote{In einem herausragenden und bahnbrechenden Artikel namens \citetitle{mallinson2018vajrolimudra} (2018) hat \citeauthor{mallinson2018vajrolimudra} anhand von textlichen, ethnographischen, erfahrungsbezogenen und anatomischen Daten, die Geschichte, die Methode und den Zweck des \textit{vajrolīmudrā} bestimmt.}

Obwohl im Lichte der modernen Auffassung von Rājayoga als Meditation,\footnote{Cf. \citeauthor{demichelishistory2004} 2004, p. 178-180.} oder der in mittelalterlichen Yogatexten (12. - 15. Jh. n. u. Z.) verbreiteten Verwendung als Synonym für \textit{samādhi}\footnote{Cf. \citeauthor{birch2014} 2014, p. 401} die Assoziation von Rājayoga und \textit{vajrolī} überraschend erscheinen mag, galt \textit{vajrolīmudrā} bereits in den frühen \textit{haṭha}-Texten, wie dem \emph{Dattātreyayogaśāstra}\footnote{Cf. \emph{Dattātreyayogaśāstra} 183-184.} als geeignete Methode um Rājayoga zu erreichen. Śrīnivāsayogī geht in der \emph{Haṭharatnāvalī} geht sogar einen Schritt weiter. Für ihn kann man einzig durch die Praxis von \textit{vajrolī} ein \textit{rājayogī} werden.\footnote{\emph{Haṭharatnāvalī} 2.104.} \citeauthor{mallinson2018vajrolimudra} (2018: 205) zufolge, wurde \textit{Vajrolīmudrā} ursprünglich von alten asketischen Traditionen als eine Technik zur Zurückhaltung bzw. Konservierung von Samen (\textit{bindudhārana}) genutzt, indem die von beiden Geschlechtern beim Akt emittierten Flüssigkeiten durch die Harnröhre resorbiert werden, sodass das für diese Traditionen so wichtige Zölibat unter allen Umständen aufrecht erhalten werden konnte.\citeauthor{mallinson2018vajrolimudra} konnte anhand von Texten die ab dem zweiten Jahrtausend n. u. Z. verfasst wurden zeigen, dass diese Praktiken einerseits erstmals einem Publikum jenseits ihrer asketischen Urheber zugänglich gemacht wurden, sodass auch Haushältern ermöglicht wurde von den Vorteilen einer Yogapraxis profitieren ohne auf die Freuden des Geschlechtsverkehrs verzichten zu müssen. Gleichzeitig wurde diese Technik von tantrischen Śaiva Traditionen adaptiert. Diese synthetisierten das ursprünglich rein physiologische Konzept von \textit{vajrolīmudrā} mit zwei eigenen Konzepten, nämlich dem Konzept von Sexualflüssigkeiten als ultimative Darbringing in Ritualen, sowie dessen Verinnerlichung als Visualisierung der vereinigten Sexualflüssigkeiten, die den zentralen Kanal nach oben geleitet werden. Hierdurch wurde \textit{vajrolīmudrā} dann nicht mehr nur als eine rein physiologische Methode zur Aufrechterhaltung zölibatärer Bemühungen betrachtet, sondern durch die Neusynthese mit den tantrischen Vorstellungen zu einer Methode, die zu einem göttlichen Körper,\footnote{\emph{Śivasaṃhitā} 4.87}, zur Erlangung aller übernatürliche Fähigkeiten(\textit{siddhi}s),\footnote{\emph{Dattātreyayogaśāstra} 175} oder der Erweckung von \textit{kuṇḍalinī} führen kann.\footnote{Cf. \citetitle{hatharatnavali} 2.82}

Nur vor diesem Hintergrund kann das Rājayoga von Dādūpanthī Sundardās richtig begriffen werden, dessen Praxisapekt aus nur zwei Versen abgeleitet werden muss.

\begin{quote}
\textit{rājayoga kīnā śiva rāī} | \textit{gaurā saṃga anaṃga na jāī} |
\textit{ghṛta nahiṃ ḍharai agni ke pāsā} | \textit{rājayoga kā baḍā tamāsā} || 14 ||
\end{quote}
\begin{quote}
Śiva performed Rājayoga with Gaurī (Pravatī), without being overcome by the god of love (\textit{anaṅga})\footnote{Anaṅga is another name for Kāma (lit. ``desire''), the god of love.} Just as clarified butter cannot stay near fire, Rājayoga is a great challange.   
\end{quote}
\begin{quote}
\textit{nāḍīcakra bheda jau pāvai} | \textit{tau caḍhi biṃda apūṭhau āvai} | 
\textit{karanī kaṭhina āhi ati bhārī} | \textit{baśabarttanī hoi jau nārī} || 15 || 
\end{quote}
\begin{quote}
  Having pierced the network of channels (\textit{nāḍīcakra}), then the rising semen arrives unbroken.
  The practice is hard and very difficult, even when the women is under control. 
\end{quote}

Der Name \textit{vajrolīmudrā} wird zwar nicht genannt, aber die in diesen Versen als Rājayoga bezeicheten Praxis ist praktisch mit den weiter oben beschriebenen mittelalterlichen Modellen von \textit{vajrolīmudrā} identisch.\footnote{Den selben Schluss ziehen auch \citeauthor{burger2014sarvangayogapradipika} 2014, p. 696 und \citeauthor{mallinson2018vajrolimudra} 2018, p. 195.} Rājayoga besteht aus einer Praxis, welche Geschlechtsverkehr zwischen Mann und Frau involviert, die dazu führt nicht vom Gott der Liebe, der Personfikation des Verlangens, übermannt zu werden.\footnote{\citeauthor{mallinson2018vajrolimudra} (2018) erwähnt \citetitle{hbp} (fol. 28r, ll. 6–9) in der beschrieben ist, dass der Yogi mit sechszehn Frauen pro Tag Sex haben kann, ohne dabei das Zölibat aufzugeben und ohne dabei der Leidenschaft zu verfallen sobald die Praxis von \textit{vajrolī} gut etabliert ist.} Der Yogi wird dazu angehalten den Samen aufsteigen zu lassen nachdem dieser das Netzwerk der Kanäle (\textit{nāḍīcakra}) durchbrochen hat. Das ganze wird als äußerst schwierig beschrieben, selbst wenn die Frau mitspielt.

Der erste Vers beschreibt, das Śiva mit Parvatī Rājayoga praktizierte. Aufgrund nachfolgenden Anspielungen auf \textit{vajrolī} impliziert dies die Ausübung des Geschlechtsaktes. Und trotz der körperlichen Vereinigung mit Parvatī, welche Schönheit und Leidenschaft symbolisiert, wurde Śiva nicht von dem Gott der Liebe (\textit{anaṅga}), welcher für Verlangen und Leidenschaft steht, übermannt. Zur Illustration der Schwierigkeit des Rājayoga verwendet Sundardās das Bild der geklärten Butter, die nicht neben dem Feuer bleiben kann, ohne zu schmelzen und zu brennen. Dieses Bild symbolisiert wie etwas sehr reines wie Ghee, der Präsenz von etwas, dass es konsumieren oder zerstören kann, im Regelfall nicht standhält. Dieses sehr Reine, die geklärte Butter, steht für das yogische Zölibat, das Feuer für die Quelle des Verlangens, nämlich die Frau, hier Parvatī. Das Zölibat eines Yogi ist in dieser Situation stark bedroht gebrochen zu werden, da es äußerst schwierig ist sich dem Trieb nicht hinzugeben. Ganz anders Śiva, der den Geschlechtsakt mit Parvatī genießen kann, ohne von ihren Flammen aufgezehrt zu werden und das ganz ohne sein Zölibat aufzugeben. Denn ihm gelingt es durch Rājayoga mit dem Samen\footnote{Es bleibt allerdings unklar, ob es sich in diesem Fall um den eigenen Samen oder eine Mixtur aus dem männlichen und weiblichen Samen handelt, wie z.B. in der \citetitle{yogasikhopanisad} 1.137cd: \textit{rajaso retaso yogād rājayoga iti smṛtā} | Rājayoga wird hier als die Vereinigung von female generative fluid or menstrual blood (\textit{rajas}) und Samen (\textit{retas}) definiert.} das Netzwerk der Kanäle zu durchbrechen und diesen dann in sich aufsteigen zu lassen. Der Vergleich veranschaulicht den hohen Grad der Selbstbeherrschung und die Schwierigkeit, welche nötig ist um diese Form des Rājayoga zu praktizieren, denn für jeden Menschen ist es natürlich, dass Verlangen in der Gegenwart von attraktiven Stimuli entsteht. Dementsprechend heißt es im letzten Vers dieses Kapitels:

\begin{quote}
  \textit{rājayoga cinha ye jānaiṃ biralā koi} |
  \textit{triyā saṃga mati kījiyahu jo aisā nahīṃ hoi} || 24 ||
\end{quote}
\begin{quote}
Those who truly understand the characteristic of Rājayoga are rare indeed; he who does not should shun the company of women. 
\end{quote}
Die anderen Verse beschreiben denjenigen, der Rājayoga gemeistert hat. Die positiven Effekte des Rājayoga sind weitreichend. Die Wiederstandsfähigkeit des Übenden wird ins unermessliche gesteigert. Weder Hunger noch Durst, Schlaf oder Faulheit, Kälte und Hitze oder Alter können ihm etwas anhaben (3.19). Feuer kann ihn nicht verbrennen, noch Wasser ihn ertränken, er altert nicht und wird unsterblich mit einem Körper hart wie ein Diamant (3.20). Er geht wohin er will, nichts in der Welt kann ihn aufhalten, er kann im Himmel mit den Göttern oder in der Hülle mit den Dämonen verweilen, wenn er es so will (3.21) usw. Der hiesige Rājayogi erinnert stark an den archtypischen Avadhūta, der tun kann was er will.

Besonders interessant ist die Aussage, dass er befreit ist (\textit{muktā}) und dennoch die acht Genüsse genießt, unberührt von Sünde und Verdienst.\footnote{Cf. \emph{Sarvāṅgayogapradīpikā} 3.17: \textit{dīsai saṃga pūni muktā} | \textit{aṣṭa prakāra bhoga kau bhuktā} | \textit{pāpa punya kachu parasai nāṃhīṃ} | \textit{jaisaiṃ kamala rahai jala māṃhīṃ} || 16 ||} Die acht Genüsse konnte ich sonst nur im \textit{Yogatattvabindu} (Sektion \uproman{22} identifizieren. Diese werden dort aufgelistet und beinhalten: 1. Seidene Kleidung, 2. Ville mit fünf oder sieben Räumen, 3. Ein großes Bett mit einer weichen und Decke, 4. eine Frau, die zur Padminī-Gruppe der Frauen gehört,\footnote{See n. \cref{padmini} on p. \Cpageref{padmini}.} 5. einen bequemen Sitz, 6. ein außergewöhnlich wertvolles Pferd, 7. appetitanregendes Essen und 8. verschiedene Getränke. Offenbar wird der ursprüngliche Gedanke des \textit{vajrolī}, nämlich trotz des Genusses von Geschlechtsverkehr das Zölibat zu brechen, hier nochmal erweitert. Durch das Rājayoga des Sundardās muss der Yogi offenbar auf überhaupt nichts mehr verzichten.

Dies zeigt den Charakter dieser umfassenden synkretistische Gleichsetzung von \textit{vajrolī} und Rājayoga, die Sundardās hiermit vornimmt. Er reduziert die Praxis des Rājayoga auf eine der elementaren Praktiken des Haṭhayoga, nämlich \textit{vajolī}, eine Praxis, die es dem Yogi erlaubt, die Genüsse der Welt zu genießen, ohne mit den Konsequenzen konfrontiert zu sein, die sich auf die übergeordneten Ziele des Yoga auswirken würden. Dies öffnet Sundardās die Tür, die allgemeinen Charakteristika der Resultate des Rājayoga anderer Traditionen uneingeschränkt auf sein eigenes Modell zu übertragen. Allein aufgrund der Einordnung des Rājayoga als Unterkategorie des Haṭhayoga zeigt, dass Sundardās Rājayoga sicherlich nicht wie in \textit{Amanaska} als König aller Yogas verstanden hat,\footnote{\textit{Amanska} 2.3cd: \textit{rājatvāt yogānāṃ rājayoga iti smṛtaḥ} |} vielmehr impliziert der Begriff Rājayoga hier, dass der Übende wie ein König leben und sich den damit einhergehenden Sinnesfreuden hingeben kann, und trotzdem ein Yogi bleibt, und nicht besitzlos als weltendsagender und nach Erlösung strebender Asket umherwandern muss. 

%
%starker Kontrast zu Nārāyaṇatīrtha:
%At the end of his commentary on \textit{sūtra} 2.28 Nārāyaṇatīrtha informs us, that the results of Haṭhayoga are limited to bodily perfection, and therfore it does not directly pertain to Rājayoga.\footnote{\emph{Yogasiddhāntacandrikā}% (Ed. p. 98): \textit{etac ca sarvaṃ yogāṅgānuṣṭhānāditi sūtre sūtritamapi haṭhayogāṅgatvena deha siddhamātraphalatvena sākṣādrājayogā 'naṅgatvāt kaṇṭharaveṇa sūtrakṛtā noktam iti mantavyam iti saṃkṣepaḥ} || 28 ||}
%


%%%%%%%%%%%%%%%%%%%%%%%%%%%%
%%%%%%%%%%%%%%%%%%%%%%%%%%%%%%
%%%%%%%%%%%%%%%%%%%%%%%%%%%%%%
%%%%%%%%%%%%%%%%%%%%%%%%%%%%%%
%LESE MARION RASTELLI FÜR CARYĀYOGA!!!
%LESE MARION RASTELLI FÜR CARYĀYOGA!!!
%LESE MARION RASTELLI FÜR CARYĀYOGA!!!
%LESE MARION RASTELLI FÜR CARYĀYOGA!!!
%Yoga in the Daily Routine of the Pa¯ ñcara¯ trins*
%1. Introduction
%In the Vais˙n˙ ava tradition of Pa¯ñcara¯tra, yoga and yogic techniques – which may
%assume many different forms aswe will see in this chapter – are utilised in various
%contexts and for mani
%LESE MARION RASTELLI FÜR CARYĀYOGA!!!
%LESE MARION RASTELLI FÜR CARYĀYOGA!!!
%LESE MARION RASTELLI FÜR CARYĀYOGA!!!
%LESE MARION RASTELLI FÜR CARYĀYOGA!!!
%%%%%%%%%%%%%%%%%%%%%%%%%%%%
%%%%%%%%%%%%%%%%%%%%%%%%%%%%%%
%%%%%%%%%%%%%%%%%%%%%%%%%%%%%%
%%%%%%%%%%%%%%%%%%%%%%%%%%%%%%
\section{Other Yogas}

Bis zu diesem Punkt wurden diejenigen Yogas der Reihenfolge nach beschrieben und miteinander verglichen, die sich in der Yogataxonomie des \textit{Yogatattvabindu} finden. Hierdurch wurden bereits die meisten aller in den komplexen mittelalterlichen Yogataxonimien vorkommneden Yogakategorien abgedeckt. Darüber hinaus tauchen in diseen Texten derweil noch weitere Yogakategorien auf. Diese sollen im folgenen behandelt werden.   

\subsection{Bhaktiyoga}

Formen des Bhaktiyoga spielen im \textit{Yogatattavabindu} und \textit{Yogasvarodaya} keinerlei Rolle. Dies sieht Nārāyaṇatīrtha offenbar völlig anders. Für ihn steht an fünfzehnter und somit höchster Stelle der Yogataxonomie der \textit{Yogasiddhāntacandrikā}, die Yogamethode, die er Premabhaktiyoga nennt. Diese Form des Yoga ist für ihn ein unumgängliches Element zur Erlangung von Rājayoga bzw. \textit{samādhi}.\footnote{Eine frühere Beschreibung von Premabhaktiyoga der \textit{Yogasiddhāntacandrikā} findet sich in \citeauthor{penna2004} 2004, pp. 97-102. Im Gegensatz zu \citeauthor{penna2004} erwähnt Nārayaṇatīrtha die vier Arten des \textit{prāṇidhāna} nicht aus rein informellen Absichten. Vielmehr veranschaulicht Nārāyaṇatīrtha damit die Überlegenheit seines Konzeptes des Premabhaktiyoga, welches alle vier Methoden umfasst.} Sundardās hingegen, nennt Bhaktiyoga in seiner Abhandlung aller Glieder des Yoga, \textit{Sarvāṅgayogapradipkā} den Bhaktiyoga an aller erster Stelle.\footnote{Eine frühere Diskussion des Bhaktiyoga der \textit{Sarvāṅgayogapradipkā} in französischer Sprache findet sich bei \citeauthor{burger2014sarvangayogapradipika} 2014, pp. 690-692.} Bhaktiyoga ist gleichzeitig eine einzelne Kategorie und Oberbegriff der ersten von insgesamt drei Tetraten und subsummiert die drei bereits behandelten Methoden Mantrayoga\footnote{See p.\pageref{mantrayogaintrosarva}.}, Layayoga\footnote{See p.\pageref{layaintrosarvanga}.} und Carcāyoga\footnote{See p.\pageref{carcasarvanga}.}.    

\subsubsection{Premabhaktiyoga in the \emph{Yogasiddhāntacanrikā}}
\label{premabhaktiyoga}

Nārāyaṇatīrtha führt Premabhaktiyoga in seinem Kommentar zu \textit{sūtra} 1.32 ein:
\begin{quote}
  \textit{tatpratiṣedhārtham ekatattvābhyāsaḥ} || 32 ||
\end{quote}
\begin{quote}
For the purpose of their elimination, the practice of concentrating on a single principle [should be performed].
\end{quote}

This \textit{sūtra} refers back to the disturbances (\textit{vikṣepa}s) mentioned in \textit{sūtra} 1.30 which lists the disturbances to the stilling of the fluctuations of the mind (\textit{cittavṛttinirodha}). These disturbances are disease (\textit{vyādhi}), incompetence (\textit{sthyāna}), doubt (\textit{saṃśaya}), carelessness (\textit{pramāda}), sloth (\textit{ālasya}), lack of detachment (\textit{avirati}), erroneous conception (\textit{bhrāntidarśana}), not obtaining a base for concentration (\textit{alabdhabhūmikatva}) and instability (\textit{anavasthitatva}).

Zur Beseitigung der neun Störungen ist laut Nārāyaṇatīrtha die hingebungsvolle Verehrung Īśvaras (\textit{īśvarapraṇidhāna}) unumgänglich. Nārāyaṇatīrtha erklärt, dass bereits eine minimale bzw. unvollständige Ausübung von \textit{praṇidhāna} großartige Resultate erzielen kann. Das bloße Aussprechen des Namens von \textit{īśvara} zerstöre die angehäuften Sünden. Durch deren Zerstörung und durch Glauben etc. bewirkt es weiterhin das volle Maß hingebungsvoller Verehrung und in der Folge alle gewünschten Ergebnisse.\footnote{Cf. \textit{Yogasiddhāntacandrikā} 1.32 (Ed. pp. 49-50): \textit{yathā 'gnikaṇo 'tisvalpo 'pi tṛṇarāśaiṃ jvālayam̐s tenaiva varddhitaḥ pūrṇaḥ sarvāṇi sūcitāni kāryāṇi janayati} | \textit{tathā bhagavato yathākathañ cinnāmoccāraṇādirūpam api praṇidhānam ajāmilāder iva pāparāśiṃ nāśayat tena nāśenaivādhikaṃ sampādyamānaṃ śraddhādinā pūrṇaṃ bhajanīya icchāsahakṛtaṃ sarvābhilaṣitaṃ sādhayate} | \textit{tasmāt praṇidhānam evāvaśyakam} |}

Die hingebungsvolle Verehrung (\textit{praṇidhāna}) kann auf vier Arten ausgeübt werden: die höchste (\textit{paramamukhya}) [Art], die vorzüglichste (\textit{mukhya}) [Art], die zur vorzüglichsten [Art] gehörige (\textit{mukhyajātīya}) [Art] und die [Art, die] zur vorzüglichsten [Art] befähigt (\textit{mukhyakalpa}).\footnote{Cf. Ibid. 1.32 (Ed. p. 50): \textit{tac ca caturvidham paramamukhyaṃ, mukhyaṃ mukhyajātīyaṃ, mukhyakalpañ ceti} |} Alle vier Methoden dienen der sukzessiven Fixierung des Geistes auf \textit{īśvara} und basieren auf den Ausführungen der \textit{Bhagavadgītā} \uproman{12}.8-11. 

Die erste Art (\textit{paramamukhya}) der hingebungsvollen Verehrung ist das liebende ununterbroche Fixieren des Geistes und des Intellektes auf \textit{īśvara}. Diese Form von \textit{praṇidhāna} wird hier mit der Hingabe und Liebe der Gopīs an Kṛṣṇa verglichen. Der Geist der Gopī schmilzt beim Hören der Vielzahl von Gottes Eigenschaften und nimmt, wie geschmolzenes Kupfer, das in eine Form gegossen wurde, fest seine Gestalt an.\footnote{Cf. Ibid. 1.32 (Ed. p. 50): \textit{tatrādyaṃ gopīnām iva tadguṇagaṇaśravaṇādinā drutacetaso drutatām rasyeva dṛḍhatadākāratā tadviṣayakavṛttipravāharūpaṃ prema mayy eva mana ādhatsva mayiṃ buddhiṃ niveśaya} | \textit{ity ādinoktam} | \textit{anena premabhaktiyogo darśitaḥ} | \textit{sa ca parameśvaracaraṇāravindaviṣayakaikāntikātyantikapremapravāho 'navacchinna ity arthaḥ} |} 

Die zweite Art (\textit{mukhya}) ist auch die \textit{nididhyāsana} genannte Praxis. Wenn die erste Art noch nicht möglich ist, soll zunächst auf diese Form zurückgegriffen werden. Sie zeichnet sich dadurch aus, den nach außen gerichteten und abschweifenden Geist durch wiederholte Übung immer wieder zurückzuziehen und ihn auf das erhabene Selbst im Innern zu konzentrieren. \footnote{Cf. Ibid. 1.32 (Ed. p. 50): \textit{dvitīyaṃ tadasāmarthye bahiḥpravṛttisvabhāvasya manasaḥ pratyāhāreṇa punaḥ punarbhagavatyātmani niveśanarūpo 'bhyāso nididhyāsanākhyaḥ}| To illistrate this explanation Nārāyaṇatīrtha quotes \textit{Bhagavadgītā} \uproman{12}.9: \textit{atha cittaṃ samādhātuṃ na śaknoṣi mayi sthiram} | \textit{abhyāsayogena tato mām icchāptuṃ dhanañjaya} || \textit{ity ādy uktam} |} Diese Variante ist für diejenigen Übenden vorgesehen, welche nicht im Stande sind, die für die erste Methode erforderliche mentale Konstanz durchgehend aufrecht zu erhalten. 

Für diejenigen, die ihren Geist noch nicht durch Liebe oder durch konstante Wiederholung auf den höchsten Gott fixieren können, wird die dritte Art (\textit{mukhyajātīya}) genannt. Hiermit ist insbesondere hingebungsvoller Dienst für den Gott, wie z.B. Rezitation von Gottes Namen, Fasten etc. gemeint. Dieser Dienst und alles Handeln, egal ob gut oder schlecht, soll ohne Anhanftung an die Resultate dem höchsten Gott gewidmet werden.\footnote{Cf. Ibid. 1.32 (Ed. p. 50): \textit{tṛtīyan tu tadasāmarthye 'pi svabhāvata eva kṛtānām api karmaṇāṃ phalecchāṃ tyaktvā parameśvare paramagurāvarpaṇam sādhu vā 'sādhu vā karma yadyadācaritaṃ mayā} | \textit{tatsarvaṃ tvayi saṃnyastaṃ tvatprayuktaḥ karomy aham} || \textit{iti saṃkalpaviśeṣarūpam} |}

Die vierte Art (\textit{mukhyakalpa}) wiederum für diejenigen, welche noch nicht im Stande sind die dritte Art, den hingebungsvollen Gottesdienst, auszuüben. Der Übende wird hier dazu angehalten den Früchten aller Taten zu entsagen und im Selbst zu ruhen.\footnote{Cf. Ibid. 1.32 (Ed. p. 50): \textit{athaitad apy aśakto 'si kartṛṃ madyogam āśritaḥ} | \textit{sarvakarmaphalatyāgaṃ tataḥ kuru yatātmavān} || \textit{ity ādinoktaṃ bhagavadgītādvādaśādhyāye} |} 

Im Kontext der wiederholten Praxis [der Konzentration] auf ein einziges Prinzip zur Eliminierung der neun Störungen von \textit{sūtra} 1.32 ist für Nārāyaṇatīrtha das eine Prinzip (\textit{ekatattva}) der höchste Gott und die Praxis (\textit{abhyāsa}) Premabhaktiyoga. Denn Premabhaktiyoga sei das Resultat aller weiter oben genannten Methoden. Desweiteren vereine Premabhaktiyoga laut Nārāyaṇatīrtha beide Resultate, nämlich einerseits die Entfernung der Störungen und andererseits das Empfangen von Gottes Gnade.\footnote{Cf. Ibid. 1.32 (Ed. pp. 50-51): \textit{atra ca praṇidhānaśabdenoktam} | \textit{tatra nididhyāsanaṃ samānaviṣayatayā sākṣātkāra janakatvasādhanaṃ karmādi yogebhyaścāntaraṅgamityabhipretya, arthabhāvanaśabdena pūrvamādṛtaṃ yady api tathāpy adṛṣṭadvārā kṛpātiśayaphalakādantarāyābhāvaphalakāc ca }| \textit{tasmāt paramaṃ mukhyaṃ bhaktiyogaṃ sarvopāyaphalabhūtam ayam ata eva tadubhayaphalakaṃ premākhyam abhyāsam āha- tatpratiṣedhārthamekatattvābhyāsaḥ} || 32 ||  \textit{tatpratiṣedheti} | \textit{teṣāṃ vikṣepāṇāṃ pratiṣedhārtham anāyāsena nāśārthaṃ ekasmim̐s tattve mukhyabhagati, abhyāsaḥ gopīnām iva tadguṇagaṇaśravaṇādinā dravībhūtasya cetaso mūṣānikṣiptadrutatām rasyeva dṛḍhatadākāratārūpaprema abhyāsayogayuktena cetasā nānyagāminā} | \textit{ity ādinā bhagavatsūcitaḥ kārya ityarthaḥ} | \textit{yad balād anāyāsena sampannāyāṃ jīvanmuktau vikṣepāḥ praśamam upayānti} | \textit{na vāsudevabhaktānām aśubhaṃ vidyate kvacit} | \textit{ity ādismṛteḥ} |}

Der im Rahmen des Premabhhaktiyoga erzeugte durchgängige Strom der Geistesfluktuationen (\textit{vṛttipravāha}) kann entweder mit Qualitäten (\textit{saguṇa}) und mit Unterscheidungen (\textit{savikalpa}) z.B. auf eine spezifische Form Gottes, Nārāyaṇatīrtha nennt hier Vāsudeva, oder ohne Qualitäten (\textit{nirguṇa}) und ohne unterscheidungen (\textit{nirvikalpa}) auf die ``unteilbare Realität, frei von inneren, äußeren und gegensätzlichen Unterscheidungen und nicht mit irgendwelchen Attributen überlagert, die die wahre, untrennbare Essenz ist'' (\textit{ekaṃ sajātīvijātīyasvagatabhedarahitaṃ tattvamanāropitam akhaṇḍārtha}) ausgeübt werden.\footnote{Cf. Ibid. 1.32 (Ed. pp. 51-52): \textit{athavā ekaṃ sajātīvijātīyasvagatabhedarahitaṃ tattvamanāropitam akhaṇḍārtha iti yāvat} | \textit{tasya abhyāsaḥ tad ekam ajaram amṛtam abhayam iti vṛttipravāhaḥ kārya ity arthaḥ} |\textit{atrātmavyatirekeṇa dvitīyaṃ yo na paśyati} | \textit{ātmarāmaḥ sa yogīndro brahmībhūto bhaved iha} || \textit{ātmakrīḍasya satataṃ sadātmamithunasya ca} | \textit{ātmany eva sutṛptasya yogasiddhir adūrataḥ} || \textit{abhiyogāt sadābhyāsāt tatraiva ca viniścayāt} | \textit{punaḥ punaranirvedāt siddhyed yogo na cānyathā} || \textit{iti skandokteḥ} | \textit{tasmāt saguṇātmavadākārākāradhārāvāhivṛttipravāhaḥ savikalpo nirguṇagocaro dhārāvāhiko nirvikalpako vā kāryo 'nāyāsena mokṣam icchateti yāvat} || 32 ||}

Die besondere Bedeutung des Premabhaktiyoga hebt Nārayaṇatīrthas schließlich in seinem Kommentar zu \textit{sūtra} 2.45 hervor. Hier ist es \textit{īśvarapraṇidhāna} in Form von \textit{premabhakti}, welche dem Yogin die Gnade Gottes zur Vollendung von \textit{samādhi} führt.\footnote{Cf. Ibid. 1.32 (Ed. p. 107): \textit{īśvarapraṇidhānasādhyasiddhim āha- samādhisiddharīśvarapraṇidhānāt} || 45 || \textit{samādhīti} | \textit{īśvarapraṇidhānaṃ pūrvaṃ vyākhyātam} | \textit{samādher uktalakṣaṇasya siddhir apratibandhenānāyāsena tatkṛpayā gurvādidvārā ca dṛḍhā prāptir bhavatīty arthaḥ} | \textit{etac ca phalaṃ premabhakteḥ svataḥ puruṣārtharūpāyā nāntarīyakaṃ yathā phalārthino vṛkṣādisānnidhyec chāyāṃ gandhādir ity anyatra vistaraḥ} || 45 ||}  

\subsubsection{Bhaktiyoga in the \emph{Sarvāṅgayogapradīpikā}}

Sundardās beschreibt die Bhaktiyoga in den Versen 2.1-15. In den Versen 2.2-7 wird das zum Bhaktiyoga notwendige Verhalten erläutert: Hierfür sollen die Sinne kontrolliert werden. Egal wo man sich befindet, soll man diesen nicht anhaften.\footnote{Cf. \emph{Sarvāṅgayogapradīpikā} 2.2cd: \textit{jitendriya aru rahai udāsī} | \textit{athavā gṛha athavā bana vāsī} || 2 ||} Ohne der Illusion (\textit{māyā}) und Täuschung anheimzufallen, soll man allem gegenüber gleichmütig sein. Gold und Frauen soll man verbannen und sich sich nicht vom Verlangen überwältigen lassen.\footnote{Cf. Ibid. 2.3cd: \textit{māyā moha karai nahiṃ kāhū} | \textit{rahai sabani sauṃ beparavāhū} | \textit{kanaka kāminī chāḍai saṃgā} | \textit{āśā tṛṣnā karai na aṃgā} || 3 ||} Darüber hinaus sollen beispielsweise gutes Verhalten, Zufriedenheit, Vergbung, Geduld und Mitgefühl kultiviert werden. \footnote{Cf. Ibid. 2.4ab: \textit{śīla santoṣa kṣamā ura ghārai} | \textit{dhīraja sahita dayā pratipārai} |} Weiterhin soll man alle Wesen als gleich betrachten, den König wie auch das Insekt,\footnote{Cf. Ibid. 2.5d: \textit{kīrī kuṃjara sama kari jānaiṃ} || 5 ||} und die Worte des wahren Gurus im Herzen behalten, etc. Desweiteren heißt es in den Worten Sundardās in Vers 2.7:
\begin{quote}
\textit{sāra grahai kūkasa saba nāṣai} | \textit{ramitā rāma iṣta sira rāṣai} | 
\textit{āṃna deva kī karai na sevā} | \textit{pūjai eka niraṃjana devā} || 7 ||
\end{quote}
\begin{quote}
One should seize the essence and abandon all impurities, keeping the beloved Rama at the forefront of the mind. One should not serve any other gods, but worship only the one pure and formless deity.
\end{quote}

Obwohl er hier das Göttliche als Rāma bezeichnet, vertritt Sundardās den Nirguṇa-Typus des Bhaktiyoga.\footnote{In \textit{Sarvāṅgayogapradīpikā} 2.15 bezeichet Sundardās selbst diese Form des Bhakti als eigenschaftslos: \textit{yaha so bhakti aliṃganī} |} Für Sundardās ist Rāma ist nichts anderes als eine Bezeichnung des unnmanifest consciousness (\textit{avyakta puruṣa}).  

Die in 2.7d genannte kultische Gottesverehrung (\textit{pūjā}), wird hier von Sundardās als Methapher und Vergleich für die von ihm beschriebenen Form des Bhaktiyoga in den Versen 2.9-11 aufgegriffen. Die externe \textit{pūjā} wird im Bhaktiyoga im Inneren ausgeführt.

Seine Disziplin ist die rituelle Waschung, und er bringt die Blumen der Liebe und Hingabe dar.\footnote{Cf. Ibid. 2.9cd: \textit{saṃjama udaka sanāna karāvai} | \textit{prema prītī ke puṣpa caḍhāvai} || 9 || }. Die Lampe (\textit{āratī}) für den Gottesdienst das Wissen und die Glocke (\textit{ghaṇṭā}) der unstruck sound (\textit{anāhada śabda}) den er kontempliert, etc. Er gibt seinen gesamten Körper und Geist hin, wird demütig und fällt zu den Füßen des Göttlichen.\footnote{Cf. Ibid. 2.11: \textit{jñāna dīpa āratī utārai} | \textit{ghaṇṭā anahada śabda vacārai} | \textit{tana mana sakala samarpana karaī} | \textit{dīna hoī puni pāyani paraī} || 11 ||}

Die abschließenden Verse veranschaulichen, die tiefe emotionale Hingabe. Niemals gibt man die Haltung des Dieners auf und die Liebe wächst von Tag zu Tag.\footnote{Cf. Ibid. 2.12cd: \textit{sevaka bhāva kadai nahiṃ caurai} | \textit{dina dina prīti adhika hī jorai} || 12 ||} Diese innere Haltung des Dienens wird mit der Haltung der treuen Ehefrau gegenüber ihrem Ehemann verglichen.\footnote{Cf. Ibid. 2.13ab:\textit{jyauṃ pratibratā rahai pati pāsā} | \textit{aisaiṃ svāmī kī ḍhiṃga dāsā} |} Sie dient immer ununterbrochen. Sundardās nennt diese Form der Hingabe ``unerschütterlich'' (\textit{bhakti ananya}).\footnote{Cf. Ibid. 2.14cd: \textit{sadā aṣaṇḍita sevā lāvai} | \textit{soī bhakti ananya kahāvai} || 14 ||} 

\subsection{Aṣṭāṅgayoga}
\label{ashtangayogacomplex}

Aṣṭāṅgayoga wird ausnahmslos in allen komplexen mittelalterlichen Taxonomien von den Autoren implementiert. Dies geschieht jedoch auf völlig unterschiedliche Weise. Rāmacandra nennt Aṣṭāṅgayoga im Rahmen seiner vollständigen Aufzählung der fünfzehn Methoden für Rājayoga nicht, führt Aṣṭāṅgayoga jedoch mit einer eigenen Sektion im Verlauf seines Textes ein. Der Autor der \textit{Yogasvarodaya} präsentiert eine unvollständige Aufzählung der fünfzehn Yogas und nennt dort nur acht der insgesamt fünfzehn Yogas beim Namen. Aṣṭāṅgayoga wird hier ebenfalls nicht beim Namen genannt. Er führt aber, wie auch Rāmacandra, Aṣṭāṅgayoga im Laufe des Textes ein. Im Gegensatz zum \textit{Yogatattvabindu} liegt in siesem Fall jedoch der Schluss nahe, dass Aṣṭāṅgayoga im \textit{Yogasvarodaya} als Bestandteil der fünfzehn Methoden des Rājayoga betrachtet worden ist. Rāmacandra scheint dies bei der Kompilation seines Textes, der zu großen Teilen auf dem \textit{Yogasvarodaya} basiert, nicht ausreichend berücksichtigt zu haben, als er seine Variante der fünfzehn Yogas festlegte, die er im Verlauf des Textes ohnehin nicht sonderlich konsistent und systematisch beschreibt. Nichtsdestotrotz ist auch das Aṣṭāṅgayoga des \textit{Yogatattvabindu} ein Teil des Rājayoga.

Nārāyaṇatīrtha hingegen nennt Aṣṭāṅgayoga nicht in seinen fünfzehn Methoden zur Erlangung von Rājayoga. Streng genommen ist Aṣṭāṅgayoga keine der von Nārāyaṇatīrtha gennanten Methoden des Rājayoga. Nichtsdestotrotz verortet er seine fünfzehn Yogas innerhalb des \textit{Pātañjalayogaśāstra}, dem ``locus classicus'' des Aṣṭāṅgayoga, sodass der Aṣṭāṅgayoga in der \textit{Yogasiddhāntacandrikā} den fünfzehn Yogas zumindest teilweise ihren Rahmen gibt.\footnote{Einerseits aufgrund der Abwesenheit innerhalb von Nārāyaṇatīrthas fünfzehn Yogas, andererseits weil das Aṣṭāṅgayoga des \textit{Pātañjalayogaśāstra} bereits sehr häufig in der Sekundärliteratur behandelt worden ist, wird hier von einer erneuten Diskussion abgesehen. Die acht Glieder werden von Nārāyaṇatīrtha in seinem Kommentar zu \textit{sūtra} 2.29 wiefolgt kommentiert: \textit{yamaniyamāsanaprāṇāyāmapratyāhāradhāraṇādhyānasamādhayo 'ṣṭāvaṅgāni} || 29 || \textit{yameti} | \textit{tatra yamāḥ svata eva saphalatvādatyāvaśyakāḥ} | \textit{sarvamumukṣujanasevyā ādau, paścāt tatsāpekṣā niyamāḥ} | \textit{etad ubhayādhīna cittasthairyasāpekṣāṇyāsanāni} । \textit{tatsāpekṣaḥ prāṇāyāmaḥ} | \textit{tatsāpekṣaḥ pratyāhāraḥ} | \textit{tatsāpekṣā dhāraṇā} |\textit{tatsāpekṣaṃ dhyānam} | \textit{tatsāpekṣaḥ samādhiḥ savikalpa ityaṣṭau sākṣāt paramparayā vā nirvikalpasya samādher aṅgānīty arthaḥ} | \textit{aṣṭāṅgamaithune kriyāniṣpatteraṅgino 'ṅgatvavat samādhiḥ savikalpa ity aṣṭau sākṣāt paramparayā vā nirvikalpasya samādher aṅgānīty arthaḥ} | \textit{aṣṭāṅgamaithune kriyāniṣpatter aṅgino 'ṅgatvavat samādher aṅgino 'ṅgatvaṃ vā bodhyam} | \textit{tena na ko'pi doṣaḥ} || 29 || Kurz zusammengefasst sagt Nārāyaṇatīrtha hier, dass die Glieder, beginnend mit den \textit{yama}s aufeinander aufbauen und aufeinander aufbauend in \textit{samādhi} münden, welches er eingangs auch als Rājayoga bezeichnete.}

Ganz anders verhält es sich in Sundardās \textit{Sarvāṅgayogapradīpikā}, der Aṣṭāṅgayoga als die letzte der vier Methoden des Haṭhayoga präsentiert.  

\subsubsection{Aṣṭāṅgayoga in the \textit{Yogatattvabindu} and \textit{Yogasvarodaya}}

Da sich Rāmacandra zur Kompilation seiner Aṣṭāṅgayoga Sektion größtenteils an seiner Vorlage dem \textit{Yogasvarodaya} orientiert,\footnote{Außerdem können in dieser Passage Einflüsse der \emph{Siddhasiddhāntapaddhati} 2.32-38 identifiziert weden.} seine Präsentation der acht Glieder jedoch an einigen interessanten Stellen abweicht, ist es erkenntnisreich, beide vergleichend zu betrachten.\footnote{Siehe \emph{Yogatattvabindu} Sektion \uproman{31} und \textit{Yogasvarodaya} (PT Ed. p. 841).}

Beide Texte beginnen mit einer Aufzählung der Glieder des Aṣṭāṅgayoga. Hier fällt auf, dass im \emph{Yogasvarodaya} nur sieben der acht Glieder aufgelistet werden und das \textit{dhyāna} fehlt.\footnote{Cf. \textit{Yogasvarodaya} (PT p. 841: \textit{idānīṃ yogamaṣṭāṅgaṃ śṛṇu lakṣaṇasaṃyutam} | \textit{yamaś ca niyamaś caiva cāsanaṃ prāṇasaṃyamaḥ} | \textit{pratyāhāro dhāraṇā ca samādhiś ca viśeṣataḥ} |} Störenderweise findet sich dann in den nachfolgenden Versen eine Beschreibung von \textit{dhyāna}, nicht aber von \textit{dhāraṇa}. Das \textit{Yogatattvabindu} listet alle acht Glieder auf, allerdings weicht Rāmacandra hier vom Pātañjalayoga-Modell ab und präsentiert \textit{dhyāna} vor \textit{dhāraṇa}.\footnote{See p.\Cpageref{padmini} n.\cref{padmini} for a discussion and further references of the reversed order of the limbs of Aṣṭāṅgayoga.} Rāmacandra sieht jedoch entsprechend seiner Vorlage von einer nachfolgenden Erläuterung von \textit{dhāraṇā} ab.

Beide Texte präsentieren ausdifferenzierte Listen von insgesamt sechs \textit{yama}s und \textit{niyama}s. Dabei lassen sich erste Abweichungen verzeichnen, die im \textit{Yogatattvabindu} auf den Einfluss der \emph{Siddhasiddhāntapaddhati}, aber auch auf Rāmacandras eigene Vorstellungen zurückzuführen sind.\footnote{The \textit{yama}s of \textit{Siddhasiddhāntapaddhati} 2.32 are: calmness (\textit{upaśamaḥ}), Conquest of all senses (\textit{sarvendriyajayaḥ}) and conquenst of food, sleep, cold, wind, and heat (\textit{āhāranidrāśītavātātapajayaḥ}). The \textit{niyama}s of the \textit{Siddhasiddhāntapaddhati} 2.33 are: living in solitude (\textit{ekāntavāsa}), detachment (\textit{niḥsaṃgata}), indifference (\textit{udāsīnyaṃ}), contentment with what is obtained (\textit{yathāprāptisaṃtuṣṭiḥ}), aversion (\textit{vairasyaṃ}) and dedication to the feet of the guru (\textit{gurucaraṇāvarūḍhatvam}).} Die Gemeinsamkeiten und unterschiede können den folgenden beiden Tabellen entnommen werden.

\footnotesize
\begin{table}[H]
    \footnotesize
\centering
\begin{tabular}{|m{5cm}|m{5cm}|}
\hline
\textbf{Die \textit{yama}s des \textit{Yogatattvabindu}} & \textbf{Die \textit{yama}s des \textit{Yogasvarodaya}} \\
\hline
\begin{itemize}
\item peace (\textit{śāntiḥ})
\item conquer of the six senses (\textit{ṣaṇṇāṃ indiyānāṃ jayaḥ})
\item little food (\textit{āhāraḥ svalpaḥ})
\item conquer of sleep (\textit{nidrājayaḥ})
\item conquer of cold (\textit{śaityajayaḥ})
\item conquer of heat (\textit{uṣṇajayaḥ})
\end{itemize}
&
\begin{itemize}
\item peace (\textit{śāntiḥ})
\item contentment (\textit{santoṣaḥ})
\item appropriate diet (\textit{āhāraḥ})
\item minimal sleep (\textit{nidrālpā})
\item control of the mind (\textit{manaso damaḥ})
\item an empty mental faculty (\textit{śūnyāntaḥkaraṇam})
\end{itemize}
\\
\hline
\end{tabular}
\caption{Die \textit{yama}s des \textit{Yogatattvabindu} und \textit{Yogasvarodaya}}
\normalsize
\end{table}

\begin{table}[H]
  \footnotesize
\centering
\begin{tabular}{|m{5cm}|m{5cm}|}
\hline
\textbf{Die \textit{niyama}s des \textit{Yogatattvabindu}} & \textbf{Die \textit{niyama}s des \textit{Yogasvarodaya}} \\
\hline
\begin{itemize}
\item restraining the mind from fickleness and establishing steadiness (\textit{khalu manaḥ cāpalabhāvān nivārya sthairye sthāpyate})
\item seeking solitude (\textit{ekānte sevanam})
\item equanimity towards all living beings (\textit{prāṇimātre samābuddhiḥ})
\item indifference
\item one shall not desire any object (\textit{udāsīnyaṃ kasyāpi vastuni icchā na kartavyā})
\item contentment with whatever is obtained (\textit{yathā lābhasantoṣaḥ})
\item never forgetting the name of the Supreme Lord (\textit{parameśvaranāma na vismaraṇīyam})
\item not indulging in self-pity (\textit{manomadhye dainyaṃ na karttavyam})
\end{itemize}
&
\begin{itemize}
\item discarding fickleness (\textit{cāpalyan tu dūre tyaktvā})
\item establishing steadiness of mind (\textit{manaḥsthairyaṃ vidhāya})
\item constantly uniting the mind with the breath (\textit{ekatra melanaṃ nityaṃ prāṇamātreṇa sā matiḥ})
\item always maintaining a detached attitude (\textit{sadodāsīnabhāva})
\item renouncing all desires (\textit{sarvatrecchāvivarjana})
\item being satisfied with whatever comes (\textit{yathālābhena santuṣṭaḥ})
\item keeping the mind fixed on the Supreme Lord (\textit{parameśvaramānasaḥ})
\item giving up pride and oblations (\textit{mānadānaparityāga})
\end{itemize}
\\
\hline
\end{tabular}
\caption{Comparison of \textit{niyama}s from Yogatattvabindu and Yogasvarodaya}
\normalsize
\end{table}
\normalsize 
Die Besprechung der Körperhaltungen (\textit{āsana}s) fällt in beiden Texten sehr kurz aus. Das \emph{Yogasvarodaya} behandelt das Thema \textit{āsana} in einem Halbvers. Hier heißt es, dass es so viele Haltungen wie Lebewesen gibt.\footnote{Cf. \textit{Yogasvarodaya} (PT p. 841): \textit{āsanāni ca tāvanti yāvanto jīvajantavaḥ} | Der Verfasser dieses Verses spielt auf die zahlreichen Tiernamen vieler Yogapositionen an. Siehe z.B. \citeauthor{encyclopediaasana} (2006).} Rāmacandra erklärt, dass die Merkmlae der Körperhaltung wurde in vielen Abhandlungen diskutiert wird. Aus diesem Grund behandelt er dieses Thema nicht.\footnote{Cf. \textit{Yogatattvabindu} section \uproman{31}: \textit{āsanasya lakṣaṇaṃ bahūgrantheṣu nirūpitam asti} | \textit{tenātra na nirūpyate} |} 

Das \textit{Yogasvarodaya} erklärt, dass Atembeherrschung (\textit{prāṇāyāma}) von dreierlei Art und hat mannigfaltigen Formen. Es heißt, dass Jünglinge nicht fähig seien, \textit{āsana} und \textit{prāṇāyama} zu üben, doch durch die große Macht des spirituellen Verdienstes (\textit{mahāpuṇyaprabhāva}) sei die große Seele (\textit{mahātmā}) dazu fähig.\footnote{Cf. \textit{Yogasvarodaya} (PT p. 841): \textit{prāṇāyāmas tridhā ceti bahudhā prathamaṃ śṛṇu} | \textit{āsane prāṇasaṃyāme na śaktāḥ sukumārakāḥ} | \textit{mahāpuṇyaprabhāveṇa śakyate tu mahātmanā} |} Der Verfasser erklärt hier nur die erste Art des \textit{prāṇāyama}. Hierbei handelt es sich um eine Standardform Form der Wechselatmung, welche von einer Visualisierung begleitet wird. Der \textit{mahātmā} atmet, durch den linken Kanal (\textit{iḍā}), also das linke Nasenloch ein, und meditiert dabei über dessen mondgleiche Erscheinung. Dann hält er den Atem (\textit{kumbhaka}) für so lange er kann an. Nachdem der Geist von großer Pracht erfüllt worden ist und der Körper voller Atemluft ist, und die Angst vor der Macht des starken Zitterns eintritt, soll er die Luft dann sehr langsam durch den rechten Kanal (\textit{piṅgalā}), also das rechte Nasenloch, ausatmen und dabei über dessen sonnenartiges Wesen meditieren. Dies soll er wiederholen wie eine Motte, die immer wieder ins Feuer fliegt, was schließich zur Reinigung des Körpers führt.\footnote{Cf. \textit{Yogasvarodaya} (PT p. 841): \textit{iḍāṃ śaśiprabhāṃ dhyātvā mandendunā tu pūrayet} | \textit{pūrayitvā yathāśakti dhyānayogī tu kumbhayet} | \textit{mahājyotirmano bhūtvā vāyuḥpūrṇakalevaraḥ} | \textit{śaktitrāsan tu santrāsya recayed vāyum arhitaḥ} | \textit{piṅgalām arkavarṇān tu tyajed dhyātvā śanaiḥ śanaiḥ} | \textit{ayaṃ pataṅgaḥ kāṣṭhāgnipratyāsena punaḥ punaḥ} | \textit{kṛtvā kalevaraṃ śuddhaṃ kuryād yatnair mahātmanā} |} 

Rāmacandra hat zum Thema \textit{prāṇāyāma} nur zu sagen, dass es für Jünglinge (\textit{sukūmāra}) nicht geeignet ist. Daher nennt erwähnt er es nur, erläutert es jedoch nicht.\footnote{Cf. \textit{Yogatattvabindu} section \uproman{31}:: \textit{prāṇāyāmas tu sukumāreṇa sādhituṃ na śakyate} | \textit{atas tasya nāmamātraṃ kathyate} |} Nimmt man Rāmacandra beim Wort, könnte diese Formulierung ein klarer Hinweis für die Spezifizierung seines intendierten Publikums sein. Warum sonst, sollte er in seinem gesamten Text nicht eine einzige Atemtechnik beschreiben?\footnote{Das intendierte Publikum von Rāmacandras \textit{Yogatattvabinu} wird auf S.\pageref{ytbaudience} im Detail diskutiert.}

Beide Texte beschreiben im Anschluss das Zurückziehen der Sinne (\textit{pratyāhāra}) in Kürze. Im \textit{Yogasvarodaya} bedeutet \textit{pratyāhāra} den Geist von der zyklischen Existenz (\textit{saṃsāra}) und von den Obliegenheiten der Sinnensobjekte zurückzuhalten. Nachdem der Yogi so die Zustände und Umwandlungen des Geistes abgelegt hat besteht er nur noch aus Leere. Ganz ähnlich heißt es im \textit{Yogatattvabindu}, dass der Yogi den Geist von der zyklischen Existenz abwenden und im Selbst verweilensoll. Die Geist entstehenden Veränderungen werden zurückgehalten.

Ähnlich knapp wird in beiden Texten die Beschreibung der Meditation \textit{dhyāna} gehalten. Im \textit{Yogasvarodaya} werden zwei Arten der Meditation genannt, eine Grobe die aus \textit{mantra} besteht (\textit{mantramaya}) und eine Feine ohne \textit{mantra}. Weitere Details bleiben dem Leser verwehrt. Im \textit{Yogatattvabindu} heißt es lediglich, dass Meditation bereits viele Male zurvor gelehrt wurde\footnote{Wahrscheinlich ist hier Rāmacandra's eigener Text gemeint.} und deshalb an dieser Stelle nicht diskutiert wird.\footnote{Cf. \textit{Yogatattvabindu} section \uproman{31}: \textit{dhyānaṃ ca bahutaraṃ prāg uktaṃ tenātra nocyate} |}

\textit{Samādhi} ist im \textit{Yogasvarodaya} der Zustand des motionless Intellektes (\textit{buddhi}), frei von Ein- und Ausatmung. Rāmacandra äußert sich im \textit{Yogatattvabindu} hierzu überhaupt nicht. Dies ist auf den ersten Blick sehr merkwürdig, da der Leser durchaus ein Kommentar zum höchsten Yogazustand erwartet. Wenn Rāmacandra hier jedoch tatsächlich Jünglinge (\textit{sukūmāra}s) addressiert, ist es völlig einleuchtet, warum er diesen hier nicht empfiehlt den Atem zum verlöschen zu bringen.  

\subsubsection{Aṣṭāṅgayoga in the \textit{Sarvāṅgayogapradīpikā}}

Aṣṭāṅgayoga (3.37-52) ist für Sundardās die letzte Methode seines vier Methoden umspannenden Haṭhayoga. Sie ergänzt die vorangehenden drei Beschreibungen von Haṭhayoga (environment for Yoga practice, dietary rules and \textit{ṣaṭkarma}s), Lakṣayoga (\textit{foci for meditation}), and Rājayoga (\textit{vajrolīmudrā} zur Wahrung des Zölibats) und gibt der gesamten \textit{haṭha} Praxis einen Rahmen. Erst durch die Einführung von Aṣṭāṅgayoga wird Sundardās System komplett. Es bildet sozusagen den Mittelbau. Während die erste Oberkategorie namens Bhaktiyoga den devotionalen Aspekt seines Yogasystems abdeckt, deckt Haṭhayoga den Körper betonten und praxisorientierten Teil des Yogas ab. Die finale Oberkategorie namens Sāṃkhyayoga bildet den Schlusstein des Yogasystems, indem es vor allem die philosophischen Hintergründe (\textit{Sāṃkhyayoga selbst und vor allem Jñānayoga}) und höhere Formen der Kontemplation im Endstadium des Yogaweges nach Sundardās (\textit{Brahmayoga und Advaitayoga} abdeckt. Es ist genau hier, im Kontext des Aṣṭāṅgayoga, in dem die viele charakteristischen Praktiken des Haṭhayoga, nämlich \textit{āsana}s, \textit{kumbhaka}s, \textit{mudrā}s und \textit{bandha}s eingeführt werden. Die Reihenfolge der acht Glieder entspricht dem Pātañjalayoga-Modell. 

Als erstes erwähnt Sundardās jedoch die beiden ersten Glieder, observances \textit{yama} und restrictions \textit{niyama} welche je zehn unterschiedliche Aspekte haben, die Sundardās sich aufzulisten erspart.\footnote{Cf. \textit{Sarvāṅgayogapradīpikā} 3.37cd: \textit{prathamahiṃ yama aru niyama bicārai} | \textit{palari ṭeka daśa daśahiṃ prakārai} || 37 ||}

Das zweite Glied, die Praxis von Körperhaltungen (\textit{āsana}), soll regelmäßig ausgeführt werden, damit der Körper gereinigt wird. Für Sundardās sind die beiden wichigsten \textit{āsana}s die vollendete Haltung (\textit{siddhāsana}) und die Lotushaltung (\textit{padmāsana}).\footnote{Cf. Ibid. 3.38: \textit{bahuryau karai su āsana sabahī} | \textit{nirma śarīra hoi puni tabahī} | \textit{tāmahiṃ sārabhūta dvai sādhai} | \textit{siddhāsana padmāsana baṃdhai} || 38 ||} 

Im Kontext der Atembeherrschung (\textit{prāṇāyāma}) macht Sundardās eingangs deutlich, dass diese Übungen vom Guru gelernt werden müssen. Er beschreibt dann jedoch zunächst die Grundform der Wechselatmung. Dies bedeutet, Einatmung durch das linke Nasenloch (\textit{iḍā nāḍī}), gefolgt von einer Haltephase und dann einer Ausatmung durch das rechte Nasenloch (\textit{piṅgalā nāḍī}). In den Haltephasen soll das Mantra im Geist zwölf mal rezitiert werden. Zum Mantra erfahren wir an dieser Stelle keine weiteren Details. Rückblickend könnte es sich dabei jedoch um das im Kontext von Sundardās Mantrayoga (2.16-27) genannte \textit{rāma mantra} handeln. Die Haltephasen sollen mit der Zeit verdoppelt und verdreifacht werden und heißen ensprechend obere (\textit{uttama}), mittlere (\textit{madhyama}) und untere (\textit{kaiṣṭa}) [Stufe].\footnote{Cf. Ibid. 3.39-40: \textit{prāṇāyāma karai bibhi aisī} | \textit{sataguru saṃdhi batāvai jaisī} | \textit{iḍā nāḍi kati pūrai bāī} | \textit{recaka karai piṃgalā jāī} || 39 || \textit{pūri piṃgalā iḍā nikārai} | \textit{dvādaśa vāra mantra bidhi dhārai} |
\textit{dviguṇa triguṇa kari prāṇāyāmaṃ} | \textit{uttama madhyama kaniṣṭa nāmaṃ} || 40 ||} Darüber hinaus nennt Sundardās die acht Atemanhaltungen (\textit{kumbhaka}s) und fünferlei Siegel (\textit{mudrā}s), sowie die drei Verschlüsse (\textit{bandha}s. Weiter Differenzierungen erhält der Yogin vom Guru.\footnote{Cf. Ibid. 3.41: \textit{kuṃbhaka aṣṭa bhāṃti ke jānaiṃ} | \textit{mudrā paṃca prakāra su ṭhānaiṃ} | \textit{baṃdha tīni nīkī bidhi lāvai} | \textit{aura bheda sadaguru taiṃ pāvai} || 41 ||} Detailliertere Beschreibungen dieser Praktiken nennt Sundardās nicht. 

Hinsichtlich des fünften Gliedes, dem zurückziehen der Sinne (\textit{pratyāhāra}) hält sich Sundardās sehr kurz. Er definiert \textit{pratyāhāra} als das bändigen des Geistes, sodass man niemals nach dem ``Geschmack der Sinnesobjekte'' (\textit{biṣai svāda}) verlangt. Dann verwendet er das Bild einer Scildkröte, die ihre Glieder in ihren Panzer zurückzieht, um den Vorgang von \textit{pratyāhāra} zu veranschaulichen.\footnote{Cf. Ibid. 3.42: \textit{pratyāhāra pakari mana rāṣai} | \textit{biṣai svāda kabahūṃ nahiṃ cāṣai} | \textit{jaisaiṃ kurama sakucai aṃgā} | \textit{esaiṃ indrī rāṣai saṃgā} || 42 ||}

Konzentration (\textit{dhāraṇā}), das sechste Glied von Sundardās ist die Konzentration auf eines der fünf Elemente, Erde (\textit{pṛthvi}), Wasser (\textit{apa}), Feuer (\textit{teja}), Luft (\textit{vāyu}) und Äther (\textit{ākāśa}) und deren assoziierten Gottheiten für jeweils fünf \textit{ghaṭikā}s.\footnote{One \textit{ghaṭikā} equals 1/60 of a day (cf. \citeauthor[1966: 114]{sircar1966}). 1/60 of a day corresponds to 24 minutes. Five \textit{ghaṭikā}s equals excatly two hours.}\footnote{Cf. Ibid. 3.43: \textit{paṃca dhāraṇā tatva prakāśā} | \textit{pṛthi apa teja vāyu ākāśā} | \textit{akṣara sahita devatani dhyāvai} | \textit{paṃca paṃca ghaṭikā laya lāvai} || 43 ||}

Sundardās teilt Meditation (\textit{dhyāna}), das siebte Glied seines Aṣṭāṅgayoga in zwei Kategorien auf. Einerseits die Meditation mit Qualitäten (\textit{saguṇa}) und andererseits die Meditation ohne Qualitäten (\textit{nirguṇa}). Erstere Kategorie bezieht sich auf die Meditation über eines von sechs \textit{cakra}s. Letztere Kategorie auf die Meditation über das formlose Selbst.\footnote{Cf. Ibid. 3.44: \textit{dhyāna su āhi ubhai ja prakāra} | \textit{eka saguṇa ika nirguna sārā} | \textit{saguna su kahiye cakra sthānaṃ} | \textit{nirguṇa rūpa ātamā dhyānaṃ} || 44 ||}  

Die verse 3.45-48 beschreiben dann das Standardsystem der sechs \textit{cakra}s: \textit{ādhāra}, \textit{svādhiṣṭāna}, \textit{maṇipūra}, \textit{anāhata}, \textit{viśuddha} und \textit{ājñā}. Abschließend erklärt Sundardās, dass nur durch die Meditation über die sechs \textit{cakra}s die Realisierung des Formlosen (\textit{nirguṇa}), also die zweite Form bzw. Stufe der Meditation erreicht werden kann.\footnote{Cf. Ibid. 3.48: \textit{iti ṣaṭa cakra dhyāna jau tānai} | \textit{tabahiṃ jāī nirguṇa pahacānai} | \textit{gaganākāra dhyāya saba ṭhairā} | \textit{prabhā marīcī jala nahiṃ aurā} || 48 |}|  

Aus der \textit{nirguṇa}-Stufe der Meditation entsteht dann das achte Glied namens meditative Absorption (\textit{samādhi}). Wenn die Fluktuationen des Geistes und der Sinne absorbiert sind, verschmilzen im letzten Stadium des Aṣṭāṅgayoga das individuelle Selbst (\textit{jīvātma}) und das höchste Selbst (\textit{paramātmā}) wie Salz im Wasser.\footnote{Cf. Ibid. 3.49-50: \textit{aba samādhi aisī bidhī karaī} | \textit{jaisaiṃ laiṃna nīra mahiṃ garaī} | \textit{mana indrī kī vṛtya samāvai} | \textit{tākau nāma samādhi kahāvai} || 49 || \textit{jīvātma paramātma doī} | \textit{sama rasa kari jaba ekai hoī} | \textit{bisarai āpa kachu nahiṃ jānai} | \textit{tākau nāma samadhi vaṣānai} || 50 ||} Diesen Zustand charakterisiert Sundardās abschließend wiefolgt:

\begin{quote}
\textit{kāla na ṣāi śastra nahiṃ lāgai} | \textit{yaṃtra maṃtra tā deṣata bhāgai} | 
\textit{śīta uṣna kabahūṃ nahiṃ hoī} | \textit{parama sāmādhi kahāvai soī }|| 51 || 
\end{quote}
\begin{quote}
Time cannot affect it and no weapon can violate it. It is beyond the effect of \textit{yantra}s and \textit{mantra}s. It is not affected by cold or heat; this is called the supreme \textit{samādhi}.
\end{quote}


\subsection{Sāṃkhyayoga}
\label{samkhyayoga}

In den komplexen frühneuzeitlichen Yogataxonomien findet sich der Begriff Sāṃkhyayoga (4.1-12) nur in der \textit{Sarvāṅgayogapradīpikā}.\footnote{Sāṃkhyayoga findet sich ebenfalls in \emph{Śivayogapradīpikā} 4.19-31. Hier gehört es zum Rājayoga innerhalb der Beschreibung von \textit{samādhi} und ist dementsprechend eingebettet im Grundgerüst eines Aṣṭāṅgayoga.} Nach Bhaktiyoga und Haṭhayoga, ist Sāṃkhyayoga die dritte und finale Hauptkategorie im Yogasystem des Sundardās. Sāṃkhyayoga bildet den Ausgangspunkt für die drei nachfolgenden und dem Sāṃkhyayoga zugeordneten Yogas, nämlich Jñānayoga,\footnote{Eine Diskussion von Jñānayoga in der \emph{Sarvāṅgayogapradīpikā} findet sich auf p.\pageref{jnanayogaintrocandrika}.} Brahmayoga\footnote{Eine Diskussion von Brahmayoga in der \emph{Sarvāṅgayogapradīpikā} findet sich auf p.\pageref{sundarbrahma}.} und Advaitayoga.\footnote{Eine Diskussion von Advaitayoga in der \emph{Sarvāṅgayogapradīpikā} findet sich auf p.\pageref{sundaradvaita}.} Bevor der Yogin durch Jñānayoga die Einheit mit der Welt erkennt, sich im Rahmen von Brahmayoga als Einheit mit dem Universum erfährt und er im Endstadium die Dualität überwindet, dient Sāṃkhyayoga vor allem dazu ein Bewusstsein für die final zu überwindende Dualität zu schaffen, indem es das Selbst vom Nicht-Selbst differenziert.\footnote{Cf. \emph{Sarvāṅgayogapradīpikā} 4.1: \textit{aba sāṃkhya su yoga hi suni lehū} | \textit{pīchai hamako doṣa na dehū} | \textit{ātama ana ātamā bicārā} | \textit{yāhī teṃ saṃkhya su nirddhārā} || 1 ||} Im Vergleich zum klassischen Sāṅkhya wird das Bewusstsein (\textit{puruṣa}) hier als Selbst (\textit{ātama}) bezeichnet und die Urnatur (\textit{prakṛti}) als Nicht-Selbst (\textit{anātama}): 

\begin{quote}
\textit{ātama śuddha su nitya prakāśā} | \textit{ana ātamā deha kā nāśā} |
\textit{ātama sukṣma vyāpaka mūlā} | \textit{ana ātamā so paṃca sthūlā} || 2 || 
\end{quote}
\begin{quote}
The self is pure, eternal and illuminating. The not-self relates to the destructible body. The self is subtle, omnipresent and the fundamental cause, while the non-self is composed of the five gross elements.
\end{quote}
\begin{quote}
\textit{pṛthi apu teja vāyu aru gaganā} | \textit{ye paṃcauṃ ātama saṃlagnā} | 
\textit{paṃcani maiṃ mila aura bikārā} | \textit{tini yaha kiyā prapaṃca pasārā} || 3 ||
\end{quote}
\begin{quote}
Earth, water, fire, air and ether - these five are attached to the self. In these five elements, other transformations occur, and these three [self, non-self and transformations] have created the proliferation of the universe.
\end{quote}

Das Nicht-Selbst besteht aus den fünf grobstofflichen Elementen Erde (\textit{pṛthi}), Wasser (\textit{apu}), Feuer (\textit{teja}), Luft (\textit{vāyu}) und Äther (\textit{gāganā}), den fünf feinstofflichen Elemeten Klang (\textit{śabda}), Berührung (\textit{saparśa}), Form (\textit{rūpa}), Geschmack (\textit{rasa}) und Geruch (\textit{gandhā}), den fünf Erkenntnissinnen (\textit{jñānendriya}s), nämlich Hören (\textit{śrotra}), Berühren (\textit{tvak}), Sehen (\textit{cakṣu}), Schmecken (\textit{jihvā}) und Riechen (\textit{ghrāṇa}), den fünf Handlungssinne namens Sprechen (\textit{vākya}), Greifen (\textit{pāṇi}), Bewegen (\textit{pāda}), Ausscheiden (\textit{pāyu}) und Fortpflanzen (\textit{upastha}), sowie dem inneren Organ (\textit{ataḥkaraṇa}) bestehend aus Verstand (\textit{mana}), Intellekt (\textit{buddhi}), Geist (\textit{citta}) und Ego (\textit{ahaṃkārā}).\footnote{Cf. Ibid. 4.4-6: \textit{śabda saparśa rūpa rasa gaṃdhā} | \textit{tanmātṛkā paṃca tana baṃdhā} | \textit{śrotra tvak cakṣu jihvā ghrāṇaṃ} | \textit{jñāna su indriya kiyau baṣāṇaṃ} || 4 || \textit{vākya hi pāṇi pāda aru pāyuḥ} | \textit{upastha sahita paṃca samaj~nāyuḥ} | \textit{karma su indriya ina kau nāmā} | \textit{tatpara apanai apanai kāmā} || 5 || \textit{mana uru buddhi citta ahaṃkārā} | \textit{catuṣṭa antahakaraṇa vicārā} | \textit{tina kai lakṣaṇa bhinnai bhinnā} | \textit{mahāpuruṣa samuj~nāye cinhā} || 6 ||} Dies sind insgesamt vierundzwanzig \textit{tattva}s. 

Sundardās geht dann näher auf das innere Organ ein. Der denkende Verstand (\textit{mana}) ist dadurch charakterisiert Gedanken und Zweifel zu erschaffen. Der Intellekte (\textit{buddhi}) versteht und bemerkt was gut ist oder schlecht. Der Geist (\textit{citta}) erzeugt die Aufmerksamkeit. Das Ego (\textit{ahaṃkārā}) das Ich-Bewusstsein und den Stolz.\footnote{Cf. Ibid. 4.7-8ab: \textit{saṃkalpai aru bikalapa karai} | \textit{mana so lakṣaṇa esau dharai} | \textit{buddhi su lakṣaṇa bodhahiṃ jāṃnī} | \textit{kīkai burau leī pahicānī} || 7 || \textit{caitana lakṣaṇa citta anūpā} | \textit{ahaṃkāra abhimāna svarūpā} |}

Schließlich differenziert Sundardās noch den feinstofflichen transmigrierenden Körper (\textit{liṅga śarīra}), welcher aus den fünf feinstofflichen Elementen und dem inneren Orgam besteht, also insgesamt aus neun \textit{tattva}s. Und der grobstoffliche Körper, der dem Verfall ausgesetzt ist, besteht aus den grobstofflichen Elementen, sowie den Erkenntnissinnen und den Handlungssinnen.\footnote{Cf. Ibid. 4.8cd: \textit{nau tatvani kau liṃga śarīrā} | \textit{paṃdraha tatva sthūla gaṃbhīrā} || 8 ||}

Sundardās erklärt abschließend, dass diese vierungzwanzig Element alle zusammen wirken und das die Seele (\textit{jīva}) die dahinterstehende Kraft ist. Sundardās nennt sie hier auch Feldwisser (\textit{kṣetrajña}) oder ewig segensreich (\textit{nirantara śīvā}). Sie durchdringt allse und ist omnipräsent. Es scheint als wäre sie mit allem, aber letztlich ist sie ungebunden. Als der Zeuge, ist sie von allem anderen, also den vierungzwanzig \textit{tattva}s des Nicht-Selbst zu unterscheiden. Sowohl das Selbst als auch das Nicht-Selbst sind ewig und nicht dem Alter und dem Tod unterworfen. Der grobstoffliche Körper (\textit{deha}) ist jedoch vergänglich. \footnote{Cf. Ibid. 4.9-12: \textit{ye caubīsa tatva baṃdhānaṃ} | \textit{bhinna-bhinna karikiyau vaṣānaṃ} | \textit{saba kau preraka kahiye jīvā} | \textit{so kṣetajña nirantara śīvā} || 9 || \textit{sakala viyāpaka aru sarvagā} | \textit{dīsai saṃgī āhi asaṃgā} | \textit{sākṣī rūpa sabani teṃ nyārā} | \textit{tāhi kachū nahiṃ lipai bikārā} || 10 || \textit{yaha ātama ana ātama niranā} | \textit{sagaj~ai takauṃ jarā na maraṇā} | \textit{sāṃkhya su mata yāhī sauṃ kahiye} | \textit{sataguru binā kahauṃ kyauṃ lahiye} || 11 || \textit{sāṃkhya yoga so yaha kahyau, bhinna hi hbinna prakāra} | \textit{ātama nitya svarūpa hai, deha anitya vicāra} || 12 ||}  

\subsection{Satyayoga}
\label{satyayoga}

The term Satyayoga appears in the \emph{Yogatattvabindu} in \uproman{44}.7 in the section on \textit{avadhūtapuruṣasya lakṣaṇam}. Rāmacandra adopted the verse from the \emph{Siddhasiddhāntapaddhati} and changed it editorially. All manuscripts of the \emph{Yogatattvabindu} read \textit{satyayogabhāk} in the fourth \textit{pāda} of the verse. Here, the source text reads \textit{siddhayogiraṭ}. As discussed in the edition on p. \Cpageref{satyayoganote} n. \cref{satyayoganote}, Rāmacandra might have used the term as a synonym for Siddhayoga\footnote{The siddhayoga of \textit{Yogatattvabindu} is discussed on p. \pageref{siddhayogaintro}.} or is is typographical error of Siddhayoga. By definition of this verse,\footnote{Assuming I have reconstructed it correctly.}, Satyayoga would be a Yoga in which the practitioner dedicates himself to the union of Śakti, here equated with expansion (\textit{prasāra}), and Śiva, equated with contraction (\textit{saṃkoca}).

\subsection{Sahajayoga}
\label{sahajayoga}

Rāmacandra definiert Sahajayoga ganz am Ende seines Textes in Sektion \uproman{58}. Diese Sektion wird als ``secret teaching of the scriptures of Yoga in all of the scriptures'' präsentiert.
Dieses Geheimnis richet sich explizit an Könige.\footnote{Cf.\emph{Yogatattvabindu} \uproman{58}: \textit{yasya rājño manomadhye kapaṭaṃ nāsti} | \textit{yasmin dṛṣṭe deśakasya trāso na bhavati} | \textit{yasya manaḥ śuddhaṃ bhavati} | \textit{yasya pṛthivyāṃ kīrtir bhavati} | \textit{yasya manomadhye satpuruṣavacanaviśvāso bhavati} | \textit{yo rājā sadānandapūrṇo bhavati} | \textit{yasya pārśve pratyakṣam anekaṁ manohārivastūni bhavanti} | \textit{etādṛśasya rājño 'gre yogarahasyaṃ kartavyaṃ} |} Kurz gesagt verkündet Rāmacandra an dieser Stelle, dass dieses Geheimnis des Yoga nur einem König offenbart werden soll, dessen Geist frei von Täuschung ist, der keine Angst vor dem Lehrer hat, der Vertrauen in edle Worte hat, stets von Glück erfüllt ist und von bezaubernden Dingen umgeben ist, usw.

Dann erklärt Rāmacandra wem dieses Geheimnis explizit nicht offenbart wird, nämlich nicht jenen, die andere beschuldigen, die kritisieren, die schlecht handeln, die nicht die Wahrheit sagen, die kein Mitgefühl zeigen und Freude am Streit haben.\footnote{Cf. Ibid. \uproman{58}: \textit{yaḥ paranindā rato bhavati} | \textit{dūrācāro bhavati} | \textit{bhrātumitrasya ca yogyaṃ vastu na dadāti} | \textit{yo satyaṃ na vati} | \textit{yo yogināṃ manomadhye nindāṃ karoti} | \textit{yasya manomadhye dayā na bhavati} | \textit{yaḥ kalahapriyo bhavati} | \textit{svakāryakaraṇe sāvadhāno bhavati} | \textit{guroḥ kāryakaraṇe 'nādito bhavati} | \textit{etādṛśasyāgre na yogaḥ kriyate na paṭhyate} |}

Dann verkündet Rāmacandra die geheime Lehre. Die geheime Lehre enthält die Beschreibung der Person, welche die höchste Realität. Diese Person ist befreit von Existenz und nicht Existenz (\textit{bhāvābhāvavinirmuktaḥ}.\footnote{Diese Aussage impliziert eine Charakterisierung der höchsten Realiätt (\textit{tattva}), die in \emph{Amanaska} 2.62 als frei von der Dualität der Existenz und der nicht Existenz beschrieben wird: \textit{bhāvābhāvadvayātītaṃ svapnajāgaraṇātigam} | \textit{mṛtyujīvananirmuktaṃ tattvaṃ tattvavido viduḥ} || 62 || ``The knowers of the highest reality know that the highest reality is beyond the duality of existence and non-existence, passes beyond [both] sleep and waking and is free from dying and living.'' (Translated by \citeauthor{birch2013}: 318).} und obwohl diese Person ständig den weltlichen Genüssen ausgesetzt ist, ist sie frei von allen Anhanftungen. Diese Person ist ein Yogi, der aus dauerhafter Glückseeligkeit gemacht ist (\textit{sadānandamayo yogī}) und praktiziert konstanten Gleichmut gegenüber Glück und Leid. Die Person hat das unteilbare höchste Selbst erkannt und führt Handlungen ohne persönliche Wünsche oder Anhaftungen aus.\footnote{Cf. Ibid. \uproman{58}.1-8.} 

 Während eine gewöhnliche Person, um diesen Zustand zu erreichen, zunächst den Blick stabilisierend muss (\textit{dṛṣṭiḥ sthirā kartavyā}), die Sitzposition stabilisierend muss (\textit{āsanaṃ dṛḍhaṃ kartavyaṃ}) und den Atem stabilisierend muss (\textit{pavanaḥ sthiraḥ kartavyaḥ}), muss eine vollendete Person dieser Disziplin nicht mehr folge leisten (\textit{etādṛśaḥ kaścin niyamaḥ siddhasya noktaḥ}).

\begin{quote}
  \textit{manaḥpavanābhyāṁ yadā sahajānandaḥ svasvarūpeṇa prakāśyate} | \textit{sa sahajayogaḥ kathyate} | \textit{rājayogamadhye iti cakravartī nāmakathanaṁ} |
\end{quote}

\begin{quote}
  When by means of mind and breath the natural bliss appears through ones own true nature, it is called Sahajayoga (``natural Yoga''). Among [the methods] of Rajayoga, it is referred to by the name of ``Universal Ruler''.
\end{quote}

Dies ist die Kernbotschaft des Geheimnisses des Yoga aller Schriften. Sie erinnert an das \textit{rājaguhyam} des Mokṣopaya, eine Lehre, durch die Herrscher einen Zustand frei von Leiden erreichen konnten.\footnote{Cf. \emph{Mokṣopaya} 2.11.10-17.} Diese Worte beenden Rāmacandras \emph{Yogatattvabindu}. Sahajayoga wird als \textit{cakravartī} (``Universal Ruler'') bezeichnet. Dies zeigt einerseits, dass Sahajayoga zwar immernoch zur der Kategorie des Rājayoga gehört, Rājayoga an dieser Stelle vornehmlich als ``Yoga für Könige'' verstanden werden muss, denn innerhalb des ``Yogas für Könige'' gilt Sahajayoga als der unangefochtene Oberherrscher. Für einen König, der herrschen und die Annehmlichkeiten, welche diese Position mit sich bringt genießt, ist dies der angestrebte Zustand, das Nonplusultra. Als \textit{kṣatriya} kann er mittels Sahajayoga den soteriologischen Erlösungszustand aufrechterhalten und ohne eine kontinuierliche Praxis aufrechtzuerhalten den eigenen Obliegenheiten seiner Kaste nachgehen. Die Obliegenheiten als Herrscher beinhalten teilweise ``grausame'' Handlungen, wie z.B. die Durchsetzung von Gesetzen im schlimmsten Fall durch Krieg. Außerdem verhilft Sajahayoga dem König, obwohl er ein ``Genießer der Erde'' ist, ganz ohne Entbehrungen, wie es z.B. für Asketen der Fall wäre, die soteriologische Vollendung zu erlangen.\footnote{Siehe hierzu auch \citeauthor{hanneder2006}, p. 121.} Der Begriff \textit {sahaja°} bedeutet in diesem Kontext somit vor allem ``seine ursprügliche Beschaffenheit bewahrend'' und ``sich nicht weiter verändernd''.\footnote{Cf. \citetitle{petersburger7} 1858, p. 99.}

\section{Conclusion}

Der Vergleich der mittelalterlichen komplexen Yogataxoniomien der vier Texte \textit{Yogatattvabindu}, \textit{Yogasvarodaya}, \textit{Yogasiddhāntacandrikā} und \textit{Sarvāṅgayogapradīpikā} zeigt eine erstaunliche Vielfalt von zusammengenommen dreirundzwanzig verschiedenen Yogakategorien, die je nach Text nicht nur unterschiedlichen kontextualisiert und strukturiert wurden, sondern sich bei überschneidenen Yogakategorien in den meisten Fällen sogar nochmal deutlich unterscheiden.

\begin{enumerate}
\item Kriyāyoga
\item Jñānayoga
\item Caryāyoga
\item Carcāyoga
\item Haṭhayoga
\item Karmayoga
\item Layayoga
\item Dhyānayoga
\item Mantrayoga
\item Lakṣyayoga
\item Vāsanāyoga
\item Śivayoga
\item Brahmayoga
\item Advaitayoga
\item Siddhayoga
\item Siddhakuṇḍalinīyoga 
\item Siddhiyoga
\item Aṣṭāṅgayoga
\item Bhaktiyoga
\item Premabhaktiyoga
\item Sāṃkhyayoga
\item Satyayoga
\item Sahajayoga 
\end{enumerate}

Der Grad der Unterschiedlichkeit der Taxonomien und die teils stark voneinander abweichenden Interpretationen der Yogakategorien untereinander zeigt, dass die Überlieferung kein rein auf Texte beschränktes Phänomen mit einer linearen Rezeptionsgeschichte gewesen sein kann, sondern dass die komplexen Yoga Taxonomien Teil eines traditionsübergreifenden oralen Diskurses gewesen sein dürften. Hierfür spricht, dass die Autoren der Texte, welche die komplexen frühneuzeitlichen Yogataxonomien beinhalten aus völlig unterschiedlichen religiösen Traditionen stammen.

Während Rāmacandra Paramahaṃsa, der Autor des \emph{Yogatattvabindu} als Daśanāmī Saṃnyāsī iniitiert war, der als Advaita Vedāntin trotz der Śaiva Wurzeln seiner \textit{sampradāya} einen religiösen Universalismus propagierte, entsprang der Autor des \textit{Yogasvarodaya} zweifelsohne einem Śaiva-Milieu. Nārāyaṇatīrtha war ein berühmter \textit{saṃnyāsa} Intellektueller, ein erfolgreicher Schriftsteller geboren als Brāhmaṇa, Anhänger des Kṛṣṇa und Vertreter der Vidyāraṇya Schule,\footnote{Cf. \citeauthor{endo1993}, p. 41.}, und Sant Sundardās galt als einer der hochgebildetsten Dādūpanthīs überhaupt, in seinen Werken propagiert er das Sant Glaubenssystem als Vertreter des Vaiṣṇava \textit{bhakta}.\footnote{Cf. \citeauthor{horstmann2023shrine} pp. 84-87.}

Es ist bemerkenswert, dass zwei der Autoren, nämlich Sundardās und Nārāyaṇatīrtha einen großen Teil ihres Lebens in Benares verbrachten. Sundardās lebte zwischen der ersten Dekade des siebzehnten Jh. - c. 1625 CE in Benares und wurde hier in den dominanten Wissenssystemen der damaligen Zeit, unter anderem in der Ästhetik und der Tradition der Kunstdichtung (\textit{kāvya}) ausgebildet.\footnote{Cf. Ibid. p. 86.} \citeauthor{endo1993} (1993: 56) grenzt die Blütezeit von Nārāyaṇatīrtha überzeugend auf 1600-1690 CE ein \citeauthor{endo1993}. Es ist allgemeiner Konsens, dass Nārāyaṇatīrtha ebenfalls einen Großteil seines Lebens in Benares verbrachte, allerdings kann der genaue Zeitraum von keinem mir bekannten Gelehrten genauer eingegrenzt werden.\footnote{\citeauthor{penna2004}, p. 24.} Möglicherweise lebten beide Autoren gleichzeitig in Benares.

Die wenigen Anhaltspunkte, die es für eine Lokalisierung Rāmacandra's und das \textit{Yogatattvabindu} gibt, grenzen den Ort der Niederschrift weitestgehend auf den nördlichen Teil Indiens ein. Da das \textit{Yogasvarodaya} nur aus Zitaten bekannt ist, nämlich im \textit{Prāṇatoṣinī} verfasst in der Nähe von Kalkutta\footnote{Cf. \citeauthor{shastri1905} 1905.}, der \emph{Yogakarṇikā}, deren Ursprung unbekannt ist,\footnote{Die einzige erhältliche Druckausgabe der \emph{Yogakarṇikā} von \citeauthor{yogakarnika} (2004) von basiert allerdings auf einem vermutlich aus Benares stammenden Manuskript, cf. \citeauthor{yogakarnika} 2004, p. \lowroman{6}.} und dem \citetitle{shabdakalpadruma}, welches von Radhakanta Deva (1784-1867) ebenfalls in Kalkutta verfasst wurde, kann provisorisch abgeleitet werden, dass basierend auf den Werken, welche das \textit{Yogasvarodaya} zitieren, das Hauptzirkulationsgebiet Nordostindien eingegrenzt ist. Da Rāmacandra für die Kompilation des \textit{Yogatattvabindu}s großzügig aus dem \textit{Yogasvarodaya} schöpft, und seine fünfzehn Yogas offenbar Teil eines oralen und literarischen Diskurses waren, der in Benares zusammenläuft, wäre es durchaus plausibel, dass auch Rāmacandra und/oder der Autor des \textit{Yogasvarodaya} ebenfalls Teil dieses oralen Diskurses waren, der rund um Benares sein Zentrum zu haben scheint.\footnote{Darüber hinaus wird diese Lokalisierung auch durch Rāmacandra's Liste selbst befürwortet. Sein Quelltext, das \emph{Yogasvarodaya} listet nur acht von fünfzehn genannten Yogas in seinen einleitenden Versen auf und bespricht, zumindest in der uns vorliegenden Überlieferung auch nicht jedes einzelne Yoga, sodass Rāmacandra die Liste nicht einfach eins zu eins übernehmen konnte. Die Wahl der von ihm ergänzten Yogas ähnelt stark den Yogataxonomien von Nārāyaṇatīrtha und Sundardās, sodass wir davon ausgehen können, dass auch Rāmacandra teil des bereits angesprochenen zeitlich und räumlich stark eingegrenzten oralen Diskurses rund um die komplexen Yogataxonomien gewesen sein dürfte.} Basierend auf der mir vorliegenden Evidenz, erscheint mir diese Lokalisierung am wahrscheinlichsten, auch wenn diese Hypothese bis auf Weiteres als tendenziell spekulativ zu gelten hat. Fest steht allerdings, dass das \textit{Yogatattvabindu} und das \textit{Yogasvarodaya} beide, wie gezeigt worden ist,\footnote{Die Datierung des \textit{Yogatattvabindu} und des \textit{Yogasvarodaya} findet sich auf p.\pageref{datierung}.} vor 1659 CE geschrieben worden sein müssen. Somit entstanden alle komplexen frühneuzeitlichen Yogataxonomien in einem sehr eng eingegrenzten Zeitraum zwischen 1600-1690, und die Hälfte, vielleicht sogar alle, im diskursiven Umfeld von Benares. Dieser Umstand ist ein wichtiges Indiz für Rückschlüsse darüber, aus welchem Grund sich ausgerechnet im diesem Zeitraum und ausgerechnet in diesem eingegrenzten Gebiet die komplexen Yogataxonomien entwickelten.

Sicherlich existieren zunächst von Autor zu Autor und Text zu Text individuelle Gründe für die Auseinandersetzung bzw. die Kodifizierung der komplexen Yogataxonomien. Rāmacandra's Text diente allem Anschein nach der Ausbildung junger Prinzen, also potenzieller Könige und die Taxonomie erfüllt hier einerseits den Zweck einer Enzyklopädie, gleichzeitig soll dem Publikum in Form einer taxonomischen Hierarchie die Superiorität des Rājayoga und seiner Methoden vermittelt werden. Die Integration der fünfzehnfachen Yogataxonomie des \textit{Yogasvarodaya} ist zweifelsohne eine Fortsetzung eines älteren Śaiva Strategems, welches darin bestand die Yogas aus verschiedenen Traditionen in ein hierarchisches Schema zu integrieren, in dem eine Śaiva-Interpretation von \textit{samādhi} und die Befreiung im Leben (\textit{jīvanmukti} in den Vordergrund gestellt wurden.\footnote{The Śaiva subordination of Yogas into a Rājayoga model began after the 10th century CE was frist described by \citeauthor{birch2019}.} Es handelt sich somit grundsätzliches um eine Erweiteruzng der älteren vierfachen Taxonomie, welche Mantra-, Laya- und Haṭhayoga in hierarchischer Abfolge dem Rājayoga unterordnet, bzw. diesem zuordnet. Im Falle der \textit{Yogasiddhāntacandrikā} vermutet \citeauthor{endo1998} (1998: 34-35), dass Nārāyaṇatīrtha die in seinem Umfeld immer populärer werdenden mittelalterlichen Yogas im Sinne von Paul \citeauthor{hacker1979}'s ``Inklusivismus'' (1979). \citeauthor{endo1998} sieht darin den Versuch Nārāyaṇatīrtha's, die als unterlegen betrachteten fremden und wahrscheinlich auch populären Formen des mittelalterlichen Yoga der eigenen Form des Yoga, hier dem Pātañjalayoga Modell nicht nur gleichzusetzen, sondern dieses Yogas eben auch dem Pātañjalayoga unterzuordnen. Sundardās \textit{Sarvāṅgayogapradīpikā} hat einen weniger instruktiven, sondern eher informellen bzw. enzyklopädischen Charakter, welcher gleichermaßen großen Wert auf Systematik und Ästhetik legt, verfolgt aber gleichermaßen eine klare Agenda, diese ist weitaus weniger hierarchisch anmutend als diejenigen Taxonomien, welche Rājayoga an die Spitze der Taxonomien stellen. Vielmehr sucht Sundardās den roten Faden und stellt einen übergeordneten Sinnzusammenhang zwischen den von ihm besungenen Yogakategorien her. Er schafft eine harmonische Ordnung in Form einer sukzessiven und logischen Abfolge, beginnend mit Bhaktiyoga mit klarer Vaiṣṇava Färbung, über Haṭhayoga, dass diverse Formen der Körper-orientierten Yogas umfasst. Sein System mündet in den philosophisch orientierten Yogas beginnend mit Sāṃkhyayoga, welches über die philosophische Betrachtung der Bestandteile der Welt ein Bewusstsein für die Dualität schafft, dessen Differenzierungen in der Einheitserfahrung des Brahmayoga wieder aufgehoben werden und schlussendlich in dem von ihm als \textit{summum bonum} betrachteten Yogazustand namens Advaitayoga münden. Wie bereits \citeauthor{burger2014sarvangayogapradipika} (2004: 706) beobachtene, erinnern seine drei Yoga Tetraden stark an die Dreiteilung der Yogas der \emph{Bhagavadgītā} mit Bhakti, Karma und Jñāna. 

Diese Vielfalt, die sich in den komplexen Taxonomien wiederspiegelt, suggeriert einen regelrechten traditionsübergreifenden Yogaboom im Umfeld der betrachteten Autoren, eine bis dato unübertroffene Welle der Populariät, vor allem der mittelalterlichen und körperorientierten Yogaformen wie Haṭhayoga mit denen sich diverse Traditionen, aber auch die Bildungselite offenbar verstärkt auseinandersetzen. Für unser Fallbeispiel gilt, dass in dessen diskursiv klar abgegrenzten Rahmen, diese Popularität zu diesem Zeitpunkt viele Gesellschaftsschichten durchdrang. Dem Yoga und dessen Wirksamkeit wurde eine soteriologisch hohe Bedeutung begemeissen, sodass Yoga bereits im 17. Jh. auch im Curriculum in Kreisen der herrschenden Klassen nicht fehlen durfte. Desweiteren zeigt sich, dass die Bedeutungen der vielfältigen Yogakategorien sehr fluide waren und in lebhaften und dynamischen Austauschprozessen diskursiv ausgehandelt wurden. Es ist daher nicht verwunderlich, dass sich dass Phänomen der frühneuzeitlichen komplexen Yogataxonomien auf einem ganz bestimmten Nährboden entwickelte. Dieser Nährboden war offenbar das in Norostindien gelegene Benares des 17. Jh.

Benares war schon immer ein Zentrum des Wissens. Viele Jahrhunderte lang zog die Stadt Gelehrte aus nah und fern an. Mit dem Beginn der muslimischen Herrschaft in Indien verließen jedoch viele der bedeutenden Gelehrten, die in Benares lehrten, die Stadt aus Angst vor religiöser Verfolgung. Bereits im 16. Jahrhundert erlebte Benares allerdings eine kulturelle Wieberbelebung, vor allem ausgelöst durch die offene Religionspolitik des Moghulkaisers Akbar und seinen unmittelbaren Nachfolgern.\footnote{Im Jahr 1556, im Alter von 13 Jahren, übernahm Akbar den Thron des Mogulreichs, das von seinem Vater nur teilweise zurückerobert worden war und nach dessen Tod in einer fast aussichtslosen Situation sofort wieder zusammenbrach, cf. \citeauthor{stietencron1989} 1989, p. 53. Nach einer Phase der militärischen Konsolidierung seines Reiches herrschte eine zerstrittene, von sozialen und religiösen Spannungen geprägte Situation in Nordindien. Vor allem Hindus wurden gedemütigt und ausgebeutet. Rajasthan, Gujarat und Zentralindien gehörten nicht zu seinem Reich. Einsetzend mit seiner Heirat der Tochter des Rajputenfürsten Rājā Bihārī Mal von Amber leitete Akbar ab 1562 eine Politik der Befriedung ein. Weitere Gemahlinnen aus den Fürstenhäusern Rajasthans filgten. Alle hielten ihre religiösen Gebräuche bei. Er erließ Dekrete, die den Hindus erlaubte wieder Tempel zu bauen und religiöse Gebräuche in der Öffentlichkeit auszuführen. Insgesamt zeichnet sich Akbars Herrschaft durch eine sehr offene Religionspolitik aus. Beispielsweise ließ er erstmals Hindus wie Todar Mal und Mān Singh in hohe politische Positionen aufsteigem, cf. Ibid., pp. 70. Akbar förderte den interreligiösen Dialog und erschuf sogar einen logenartigen freidenkersichen Orden, den Dīn-i-Ilāhī, den ``göttlichen Glauben'', der geleitet vom Versuch das Beste aus allen Religionen zusammentragen und alles rational nicht Überzeugende abzustoßen und eine gemeinsame Wahrheit zu finden, cf. Ibid., pp. 62. So wurden zu dieser Zeit unter anderem in Benares zahlreiche zuvor zerstörte Hindutempel wieder aufgebaut, cf. Ibid., pp. 58-59. Spätere Vasallen der Moghulkaiser, insbesondere die späteren Kachwaha Rajputen Herrscher von Amber, vor allem unter Rāja Man Singh I. bauten zahlreiche Tempel und ghats in der Stadt, cf. \citeauthor{hooja2006} 2006, pp. 493-495. Die offene Religionspolitik wurde auch von seinem unmittelbaren Nachfolger Shāh Jahāngīr (1605–1627) fortgesetzt, cf. \citeauthor{jahangir1999} 1999. Erst im Laufe der Herrschaft von Shāh Jahān, der von 1628-1658 das Moghulreich regierte, wurde die allgemeine Stimmung seitens der Regierung ab dem Jahr 1632 wieder Hindu-unfreundlicher. Shāh Jahān ordnete ein Gesetz zur Zerstörung im Bau befindlicher Hindu-Tempel an, da seine islamischen Theologen den Bau und die Renovierung von Tempeln fremder Religionen verhindern wollten. Es bleibt jedoch unklar, wie strikt dieses Gesetz durchgesetzt wurden und inwieweit sich dieser religionspolitische Bruch nach ein dreiviertel Jahrhundert religiösen Dialoges und Tolerenz auf die Bevölkerung übertrug. Die Auswirkungen waren höchstens marginal. Während Shāh Jahāns Herrschaft sind sieben Fälle von Tempelzerstörungen dokumentiert.\footnote{Cf. \citeauthor{eaton2001}.} Inwieweit sich seine Herrschaft auf Benares ausgewirkt hat ist unklar, vermutlich jedoch gering. Schließlich war beispielsweise der damalaige Rāja von Amber Jai Singh I. der von 1627–1667 regierte ein wichtiger militärischer Verbündeter und sein Vater hatte Man Singh I. hatte große Summen in den Bau hinduistischer Tempel in Benares investiert. Erst unter Aurangzeb (1658-1707) gab es dezidiertes Wiederaufflammen der Anti-Hindu-Politik.} Dies war ein wichtiger Katalysator für die einsetzende Ausbildung einer immer ausgepräteren ``hinduistischen'' Identität, die in der Folge im Laufe des sechzehnten bis siebzehnten Jahrhunterts Gestalt annahm.\footnote{\citeauthor{clark2006}, p. 188.} Yoga spielte bei der Ausbildung dieser neuen Identität eine wichtige Rolle. Unter der Schirmherrschaft der Moghul-Kaiser wurde Benares erneut zu einem Schmelztigel des intellektuellen und religiösen Austauschs und viele Gelehrte siedelten sich wieder an, wie wir ebenfalls am Beispiel von Sundardās oder Nārāyaṇatīrtha sehen konnten. Parallel dazu kommt es, wie \citeauthor{birch2020} (2020: 471-472) in seinem bahnbrechenden Artikel \citetitle{birch2020} zeigte, zu einem allgemeinen Erblühen der Literatur über Haṭhayoga sowohl in Nord- als auch in Südindien. Die Literatur über Haṭhayoga wurde ab dem sechszehnten Jahrhundert immer diversifizierter. Autoren verschiedener Traditionen, insbesondere gelehrte Brahmanen, versuchten Haṭhayoga zu erweitern und andere Yogas sowie unterschiedliche Religionen zu integrieren. Der Beginn dieser von \citeauthor{birch2020} sichtbar gemachten Entwicklung zeichnet sich hier deutlich ab.

Die komplexen frühneuzeitlichen Yogataxonomien der mittelalterlichen Yogas sind ein Ergebnis des Zusammentreffens verschiedener yogischer Traditionen und der damit verbundenen Religionen in einem Schmelztiegel intellektuellen und religiösen Austauschs zu Beginn der Blütephase einer neuen, diversifizierteren Welle der Yogaliteratur, die sich insbesondere über Knotenpunkte wie Benares verbreitete. Darüber hinaus sind sie ein literarisches Zeugnis dieses Prozesses und ein Spiegel der diskursiven Aushandlungsprozesse und der Neuverortung der Autoren aus unterschiedlichen Traditionen angesichts neuer yogischer Impulse, die auf sie einwirkten.


\end{document}

