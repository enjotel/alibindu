%Ultimatives Tool zur Datierung:
%https://www.cc.kyoto-su.ac.jp/~yanom/pancanga/
%skp = ignored in edition
%skm = ignored in xml
%%%---2-DO---%%%:
% - add xml ids for cladistics
% - produce diplomatic transcripts for saktumiva
% - read Sarvangayogapradipika, Maya Burger! 
% - maybe add second ciritical edition of yogasvarodaya?!
% - grep-search alle Verse!!!!
% - Mss spreadsheet
% - additions to U2: make footnotes for the bahir mātrā-s: explaining the inventions of female deities and tell that this is "schwer interpretierbar"
% - Consider changing Lakṣya to Lakṣa
%%%%%%%%%%%%%%%%%%%%%%%%%%%%%%%%%%%%%%%%%
% Don't forget
% Siddhasiddhantapaddhati Yogic Body descriptions are followed by Rāmacandra
% Quotes of the Yogasvarodaya in the Yoga Karṇikā
% Rāmacandra more a compiler than an author!!!
% Identify quotes of YTB in Haṭhasanketacandrikā -- done :D
%%%%%%%%%%%%%%%%%%%%%%%%%%%%%%%%%%%%%%%%%%%
%MSS notes
%
%--B: i and ī are not differenciated
%--P: no punctuation no daṇdas nothing
%--U1: dot . serves as daṇḍa 
%--\L and \U2 very similar
%--figure out for U2: // ajapājapaḥ sahasra // 6000 //gha 0 16 pa 0 40// \U2?!?!?!?!?!?
%%%%%%%%%%%%%%%%%%%%%%%%%%%%%%%%%%%%%%%%%%
%
% Einleitung Ideen 
% - sprachliche Simplizität
% - Potenzial als Anfängertext
% - Großartige Einführung in die Textkritik -> Synoptische Edition 
% - Gelegenheit Yogasvarodaya und Yogatattvabindu zu edieren 
% - Historische Evidenz entweder für das königliche Leben in einer Maṭha in der Nähe von Benares während der Muslimischen Herrschaft, oder sogar Lehrtext für die Bildung junger Prinzen  
% - eines der raren Beispiele der engen Verknüfung mehrerer Texte 
% - eines der raren Beispiele der Prosaisierung eines metrischen Textes 
% - Anwendung rezenter Technologie! 
% - How the text was construed -> intermingling of Ysv and SSP
% - Martin Straube: "jeder kleine Dorfhäuptling kann Rāja genannt werden". 
%%%%%%%%%%%%%%%%%%%%%%%%%%%%%%%%%%%%%%%%%%%
\documentclass[10pt]{memoir}
\setstocksize{220mm}{155mm} 	        
\settrimmedsize{220mm}{155mm}{*}	
\settypeblocksize{170mm}{116mm}{*}	
\setlrmargins{18mm}{*}{*}
\setulmargins{*}{*}{1.2}
%\setlength{\headheight}{5pt}%
\checkandfixthelayout[lines]
\linespread{1.16}
\flushbottom

%%% Hyphenation settings
\usepackage[htt]{hyphenat}
\hyphenation{he-lio-trope opos-sum}
\tracingparagraphs=1
%Hyphenation in Devanāgarī of the edition still missing? Probably this needs to be modified in babel-iast package? 

%%% babel
\usepackage[english]{babel}
\usepackage{babel-iast/babel-iast}

\babelfont[iast]{rm}[Renderer=Harfbuzz, Scale=1.3]{AdishilaSan}%AdishilaSan}
\babelfont[english]{rm}{Adobe Text Pro}

%%% more functionality
\PassOptionsToPackage{hyphens}{url}
\usepackage{hyperref}
\usepackage{pdflscape}
\usepackage{cleveref}
\usepackage{url}
\usepackage{cleveref}
\usepackage{microtype}
\usepackage{lineno}

%\usepackage{bigfoot}
%%% more functions
\usepackage[dvipsnames]{xcolor}
%\usepackage[para,perpage]{footmisc}

%%%für den Counter von Kapiteln und Sätzen! 
\newcommand{\uproman}[1]{\uppercase\expandafter{\romannumeral#1}}
\newcommand{\lowroman}[1]{\romannumeral#1\relax}

\makeindex
\newfontfamily\sanskritfont[Script=Devanagari,Mapping=RomDev,Scale=1.1]{Sanskrit2003}
\usepackage{pifont,fourier-orns,lettrine,psvectorian,paralist,enumitem,pdfpages,wrapfig,tabulary,lettrine,longtable}
\setlist[enumerate]{itemsep=0mm}
\usepackage[autostyle]{csquotes}
\usepackage[defaultlines=2,all]{nowidow}
\usepackage{ellipsis,adforn,booktabs,longtable,url,tikz}
\lineskiplimit=-3pt          

\makechapterstyle{IeT}{%
  \chapterstyle{default}
  \renewcommand*{\printchapternonum}{\centering}
  \renewcommand*{\clearforchapter}{\cleartorecto} 
  \aliaspagestyle{chapter}{empty}}
\chapterstyle{IeT}
\setsecnumdepth{none}  \openright  \nouppercaseheads
\settocdepth{subsubsection}

%%%% test better pagebreaks
%\def\fussy{%
%  \emergencystretch\z@
%  \tolerance 200%
%  \hfuzz .1\p@
%  \vfuzz\hfuzz}

%\interfootnotelinepenalty=10000\relax

%\usepackage[maxfloats=256]{morefloats}

%\maxdeadcycles=500

%raggedbottomsectiontrue
%%\checkandfixthelayout


%%%%%%%  biblatex
%\newcommand{\noun}[1]{\textsc{#1}}    %  philosophy-verbose
\usepackage[backend=biber, sorting=nyt, style=verbose]{biblatex} %%%%ORIGINAL TiE
\renewcommand*{\mkbibnamefamily}[1]{\textsc{#1}}


\DeclareFieldFormat{url}{%
  \mkbibacro{URL}\addcolon\space
  \href{#1}{\nolinkurl{\thefield{urlraw}}}}

\DeclareFieldFormat{citeurl}{%
  \href{#1}{\nolinkurl{\thefield{urlraw}}}} 


\DeclareFieldFormat{postnote}{#1}
\renewcommand{\postnotedelim}{, }
\addbibresource{bindu.bib}

%%% ekdosis
\usepackage[teiexport=tidy,parnotes=true]{ekdosis}% =tidy cleans up HTML and XML documents by fixing markup errors and upgrading legacy code to modern standards. parnotes=footnotes below or above critical apparatus

\SetLineation{lineation=page, modulo} %lineation=page sets thenumbering to start afresh at the top of each page. =modulo makes every fifth line numbered. {lineation=page} makes every line numbered! 

\renewcommand{\linenumberfont}{\selectlanguage{english}\footnotesize} %sets language of lines to English

\SetTEIxmlExport{autopar=false} %autopar=falseinstructs ekdosis to ignore blank lines in the.tex sourcefile as markers for paragraph boundaries. As a result, each paragraph of the edition must be found within an environment associated with the xml <p> element

\SetHooks{
  lemmastyle=\bfseries,
  %refnumstyle=\selectlanguage{english}\bfseries,
  refnumstyle=\selectlanguage{english}\color{blue}\bfseries,
  appheight=0.8\textheight,
}

\newif\ifinapparatus
\DeclareApparatus{source}[
%bhook=\inapparatustrue,
lang=english,
notelang=english,
% bhook=\selectlanguage{english},
bhook=\selectlanguage{english}\textbf{Sources:},%
%maxentries=4, 
%ehook=.]
%sep={] },
%nosep,
]

\newif\ifinapparatus
\DeclareApparatus{testium}[
%bhook=\inapparatustrue,
lang=english,
notelang=english,
% bhook=\selectlanguage{english},
bhook=\selectlanguage{english}\textbf{Testimonia:},
%maxentries=4, 
%ehook=.]
%nosep, 
]

% Declare \ifinapparatus and set \inapparatustrue at the beginning of
% the apparatus criticus block. Also set the language.  
\newif\ifinapparatus
  \DeclareApparatus{default}[
  %bhook=\inapparatustrue, 
  lang=english,
  %maxentries=33,
  %bhook=\selectlanguage{english},
  sep = {] },
  delim=\hskip 0.75em,
  rule=\rule{0.7in}{0.4pt},
]

\newif\ifinapparatus
\DeclareApparatus{philcomm}[
%bhook=\inapparatustrue,
lang=english,
notelang=english,
bhook=\selectlanguage{english}\textbf{Philological Commentary:},
%bhook=\selectlanguage{english},
sep={: },
]

\ekdsetup{
showpagebreaks,
spbmk = \textcolor{blue}{spb},
hpbmk = \textcolor{red}{hpb}
}

%\usepackage{fnpos}
%\makeFNmid
%\makeFNbottom
\usepackage[bottom]{footmisc}
%%%%%%%%%%%%%%%%%%%%%%%%%%%
\makeatletter
\def\blfootnote{\gdef\@thefnmark{}\@footnotetext}
\makeatother
%%%%%%%%%%%%%%%%%%%%%%%%%


% Macros and Definitions for the Print of Sigla
\def\acpc#1#2#3{{#1}\rlap{\textrm{\textsuperscript{#3}}}\textsubscript{\textrm{#2}}\space}
\def\sigl#1#2{{{#1}}\textsubscript{\textrm{#2}}}
\def\None{{\sigl{N}{1}}} \def\Noneac{\acpc{N}{1}{ac}\,} \def\Nonepc{\acpc{N}{1}{pc}\,}
\def\Ntwo{{\sigl{N}{2}}} \def\Noneac{\acpc{N}{2}{ac}\,} \def\Nonepc{\acpc{N}{2}{pc}\,}
\def\Done{{\sigl{D}{1}}} \def\Doneac{\acpc{D}{1}{ac}\,} \def\Donepc{\acpc{D}{1}{pc}\,}
\def\Dtwo{{\sigl{D}{2}}} \def\Dtwoac{\acpc{D}{2}{ac}\,} \def\Dtwopc{\acpc{D}{2}{pc}\,}
\def\Uone{{\sigl{U}{1}}} \def\Uoneac{\acpc{U}{1}{ac}\,} \def\Uonepc{\acpc{U}{1}{pc}\,}                 
\def\Utwo{{\sigl{U}{2}}} \def\Utwoac{\acpc{U}{2}{ac}\,} \def\Utwopc{\acpc{U}{2}{pc}\,}

%%%%%%%%%%%%%% Tattvabinduyoga - List of Witnesses   %%%%%%%%%%%%%%%%%%%
\DeclareWitness{ceteri}{\selectlanguage{english}cett.}{ceteri}[]   
\DeclareWitness{E}{\selectlanguage{english}E}{Printed Edition}[]    
\DeclareWitness{P}{\selectlanguage{english}P}{Pune BORI 664}[]  
\DeclareWitness{B}{\selectlanguage{english}B}{Bodleian 485}[]       
\DeclareWitness{N1}{\selectlanguage{english}N\textsubscript{1}}{NGMPP 38/31}[]
\DeclareWitness{N2}{\selectlanguage{english}N\textsubscript{2}}{NGMPP B 38/35}[]
\DeclareWitness{L}{\selectlanguage{english}L}{LALCHAND 5876}[]  
\DeclareWitness{D}{\selectlanguage{english}D}{IGNCA 30019}[] 
%\DeclareWitness{D2}{\selectlanguage{english}D\textsubscript{2}}{IGNCA 30020}[]  
\DeclareWitness{U1}{\selectlanguage{english}U\textsubscript{1}}{SORI 1574}[] 
\DeclareWitness{U2}{\selectlanguage{english}U\textsubscript{2}}{SORI 6082}[]
%%%%%%%%%%%%%% Tattvabinduyoga - Groups of Witnesses   %%%%%%%%%%%%%%%%%%%
\DeclareWitness{X}{\selectlanguage{english}\alpha}{Alpha Group: D,N1,N2,U1}[]
\DeclareWitness{Y}{\selectlanguage{english}\beta}{Beta Group: B,E,L,P,U2}[]
%%%%%%%%%%%%% Testimonia
\DeclareWitness{Ysv}{\selectlanguage{english}Ysv}{Yogasvarodaya}[] %%%add infos!  

%%%%%%%%%%%%%%%%%%%%%%%%%%%%%%%%%%%%%%%%%%%
% Macro for Editing Abbrevs.
\def\om{\textrm{\footnotesize \textit{om.}\ }} %prints om. for omitted in apparatus
\def\korr{\textrm{\footnotesize \textit{em.}\ }} %prints em. for emended in apparatus
\def\conj{\textrm{\footnotesize \textit{conj.}\ }} %prints conj. for conjectured in apparatus

% \supplied{text} EDITORIAL ADDITION -> Within \lem oder \rdg
% \surplus{text} EDITORIAL DELETION -> Within \lem oder \rdg
% \sic{text} CRUX
% \gap{text} LACUNAE -> [reason=??, unit=??, quantity=??, extent=??]


%%%%%%%%%%%%%%%%%%%%%%%%%%%%%%%%%%%%%%%%%%% All macros of this list can be used in 
% Macro for Editing Abbrevs.
\def\eyeskip{\textrm{{ab.\,oc. }}}
\def\aberratio{\textrm{{ab.\,oc. }}}
\def\ad{\textrm{{ad}}}
\def\add{\textrm{{add.\ }}}
\def\ann{\textrm{{ann.\ }}}
\def\ante{\textrm{{ante }}} 
\def\post{\textrm{{post }}}
%\def\ceteri{cett.\,}                   
\def\codd{\textrm{{codd.\ }}}

\def\coni{\textrm{{coni.\ }}}
\def\contin{\textrm{{contin.\ }}}
\def\corr{\textrm{{corr.\ }}}
\def\del{\textrm{{del.\ }}}
\def\dub{\textrm{{ dub.\ }}}

\def\expl{\textrm{{explic.\ }}} 
\def\explica t{\textrm{{explic.\ }}}
\def\fol{\textrm{{fol.\ }}}
\def\foll{\textrm{{foll.\ }}}
\def\gloss{\textrm{{glossa ad }}}
\def\ins{\textrm{{ins.\ }}}      
\def\inseruit{\textrm{{ins.\ }}} 
\def\im{{\kern-.7pt\lower-1ex\hbox{\textrm{\tiny{\emph{i.m.}}}\kern0pt}}} %\textrm{\scriptsize{i.m.\ }}}      
\def\inmargine{{\kern-.7pt\lower-.7ex\hbox{\textrm{\tiny{\emph{i.m.}}}\kern0pt}}}%\textrm{\scriptsize{i.m.\ }}}      
\def\intextu{{\kern-.7pt\lower-.95ex\hbox{\textrm{\tiny{\emph{i.t.}}}\kern0pt}}}%\textrm{\scriptsize{i.t.\ }}}           
\def\indist{\textrm{{indis.\ }}}  
\def\indis{\textrm{{indis.\ }}}
\def\iteravit{\textrm{{iter.\ }}} 
\def\iter{\textrm{{iter.\ }}}
\def\lectio{\textrm{{lect.\ }}}   
\def\lec{\textrm{{lect.\ }}}
\def\leginequit{\textrm{{l.n. }}} 
\def\legn{\textrm{{l.n. }}}
\def\illeg{\textrm{{l.n. }}}

\def\primman{\textrm{{pr.m.}}}
\def\prob{\textrm{{prob.}}}
\def\rep{\textrm{{repetitio }}}
\def\secundamanu{\textrm{\scriptsize{s.m.}}}            \def\secm{{\kern-.6pt\lower-.91ex\hbox{\textrm{\tiny{\emph{s.m.}}}\kern0pt}}}%   \textrm{\scriptsize{s.m.}}}
\def\sequentia{\textrm{{seq.\,inv.\ }}}  
\def\seqinv{\textrm{{seq.\,inv.\ }}}
\def\order{\textrm{{seq.\,inv.\ }}}
\def\supralineam{{\kern-.7pt\lower-.91ex\hbox{\textrm{\tiny{\emph{s.l.}}}\kern0pt}}} %\textrm{\scriptsize{s.l.}}}
\def\interlineam{{\kern-.7pt\lower-.91ex\hbox{\textrm{\tiny{\emph{s.l.}}}\kern0pt}}}   %\textrm{\scriptsize{s.l.}}}
\def\vl{\textrm{v.l.}}   \def\varlec{\textrm{v.l.}} \def\varialectio{\textrm{v.l.}}
\def\vide{\textrm{{cf.\ }}}
\def\cf{\textrm{{cf.\ }}} 
\def\videtur{\textrm{{vid.\,ut}}}
\def\crux{\textup{[\ldots]} }
\def\cruxx{\textup{[\ldots]}}
\def\unm{\textit{unm.}}
%%%%%%%%%%%%%%%%%%%%%%%%%%%%%%%%%%%%

% List of Scholars
\DeclareScholar{ego}{ego}[
forename=Nils Jacob,
surname=Liersch]

% Persons:14\DeclareScholar{ego}{ego}[15forename=Robert,16surname=Alessi]17% Useful shorthands:18\DeclareShorthand{codd}{codd.}{V,I,R,H}19\DeclareShorthand{edd}{edd.}{Lit,Erm,Sm}20\DeclareShorthand{egoscr}{\emph{scripsi}}{ego}

%Useful shorthands:
%\DeclareShorthand{codd}{codd.}{V,I,R,H}
%\DeclareShorthand{edd}{edd.}{Lit,Erm,Sm}
\DeclareShorthand{egoscr}{em.}{ego}
\DeclareShorthand{egoscrconj}{conj.}{ego}
\DeclareShorthand{egomute}{\unskip}{ego}

\usepackage{xparse}

\NewDocumentEnvironment{tlg}{O{}O{}}{\setlength{\leftskip}{0pt}\vspace{-1ex}\begin{quotation}}{\hfill #1\ \vspace{-1ex}\end{quotation}\vspace{-1ex}} %verse environment
%\NewDocumentEnvironment{tlg}{O{}O{}}{\begin{verse}}{॥#1\hskip-4pt ॥\\ \end{verse}}
\NewDocumentCommand{\tl}{m}{{\selectlanguage{iast} #1}}

\NewDocumentCommand{\extra}{m}{{\textcolor{gray}{#1}}} %command for additions to U2
\NewDocumentCommand{\crazy}{m}{{\textcolor{red}{#1}}} %totally corrupted passage
\NewDocumentCommand{\coro}{m}{{\textcolor{violet}{#1}}} %colour for sentence counter! 

\NewDocumentEnvironment{prose}{O{}}{\begin{otherlanguage}{iast}}{\end{otherlanguage}}
% \NewDocumentEnvironment{padd}{O{}}{\begin{otherlanguage}{iast}}{\end{otherlanguage}}
\NewDocumentEnvironment{tlate}{O{}}
%\NewDocumentEnvironment{tadd}{O{}}

%Define two commands: \skp ("sanskrit plus"), to be ignored by TeX in
%the edition text, but processed in the TEI output. Conversely, \skm
%("sanskrit minus") is to be processed in the edition text, but
%ignored if found in the apparatus criticus and in the TEI output:

\NewDocumentCommand{\skp}{m}{}
\TeXtoTEIPat{\skp {#1}}{#1}

%\NewDocumentCommand{\skpp}{m}{}
%\TeXtoTEIPat{\skpp {#1}}{#1}

\NewDocumentCommand{\skm}{m}{\unless\ifinapparatus#1-\fi}
\TeXtoTEIPat{\skm {#1}}{}

% \NewDocumentCommand{\dd}{}{/\hskip-4pt/}
\NewDocumentCommand{\dd}{}{\mbox{/\hskip-4pt/}}
\TeXtoTEIPat{\dd {}}{//}


%%% modify environments and commands
%%% TEI mapping
\TeXtoTEIPat{\begin {tlg}}{<lg>} %lg=(Group of verse (s)) contains one or more verses or lines of verse that together form a formal unit (e.g. stanza, chorus).
\TeXtoTEIPat{\end {tlg}}{</lg>}

\TeXtoTEIPat{\begin {prose}}{<p>}
\TeXtoTEIPat{\end {prose}}{</p>}

\TeXtoTEIPat{\begin {tlate}}{<p>}
\TeXtoTEIPat{\end {tlate}}{</p>}

\TeXtoTEIPat{\\}{}
\TeXtoTEIPat{\linebreak}{<br/>}
\TeXtoTEIPat{\noindent}{}
%\TeXtoTEI{tl}{l}
\TeXtoTEI{emph}{hi}
\TeXtoTEI{bigskip}{}
\TeXtoTEI{None}{N1}
\TeXtoTEI{Ntwo}{N2}
\TeXtoTEI{Done}{D1}
\TeXtoTEI{Dtwo}{D2}
\TeXtoTEI{Uone}{U1}
\TeXtoTEI{Utwo}{U2}
%\TeXtoTEIPat{/}{ |}
%\TeXtoTEI{//}{ ||}
\TeXtoTEIPat{\korr}{em. }
\TeXtoTEIPat{\conj}{conj.}
\TeXtoTEIPat{\om}{om.}
\TeXtoTEIPat{english}{}
\TeXtoTEIPat{\hskip}{}
\TeXtoTEIPat{\hskip-4pt}{}
\TeXtoTEIPat{\hskip-2pt}{}
\TeXtoTEIPat{-}{ }
\TeXtoTEIPat{4pt}{}
\TeXtoTEIPat{2pt}{}
\TeXtoTEIPat{\textcolor {#1}{#2}}{<hi rend="#1">#2</hi>} 

% Nullify \selectlanguage in TEI as it has been used in
% \DeclareWitness but should be ignored in TEI.
\TeXtoTEI{selectlanguage}{}



\FormatDiv{1}{\begin{center}\Large}{\end{center}}
\FormatDiv{2}{\begin{center}\small}{\end{center}}
\FormatDiv{3}{\bfseries}{.}
\title{Yogatattvabindu of Rāmacandra\\ A Critical Edition and Annotated Translation}
\date{\today}

\parindent=15pt
\begin{document}

% Zitiermöglichkeiten:
%\footcite[See][p.\,1]{goldstein01:_tibet_englis_diction_moder_tibet}
%\footnote{\cite{goldstein01:_tibet_englis_diction_moder_tibet}.}

\frontmatter
\thispagestyle{empty}
\begin{center}
  {\Large \emph{The Yogatattvabindu}}\\[3mm]
\end{center}



\newpage

\

\thispagestyle{empty}



\normalsize


\newpage


\begin{center}
\thispagestyle{empty}

\

\vskip 2mm

\begin{otherlanguage}{iast}
\LARGE \sanskritfont{Yogatattvabindu}
\end{otherlanguage}

\vskip .4cm

\Huge Yogatattvabindu \\[7mm]
\Large Critical Edition\\
with annotated Translation


\large

\vspace{3cm}

Von

Nils Jacob Liersch
\small
\vfill

\vfill

Indica et Tibetica Verlag \\ % $\cdot$ 
Marburg 2024

\vskip 6mm

\end{center}

\newpage
\newpage \ \thispagestyle{empty}
\small  \

\noindent

\
\vfill


\small
\noindent \textbf{Bibliographische Information Der Deutschen Bibliothek}

\noindent
Die Deutsche Bibliothek verzeichnet diese Publikation in der Deutschen Nationalbibliographie;
detaillierte bibliographische Informationen sind im Internet über http://dnb.ddb.de abrufbar.

\noindent
\textbf{Bibliographic information published by Die Deutschen Bibliothek}

\noindent
Die Deutsche Bibliothek lists this publication in the Deutsche Nationalbibliographie; detailed
bibliographic data is available in the Internet at http://dnb.ddb.de.  


\vskip 1cm

\noindent
\copyright\ Indica et Tibetica Verlag, Marburg 2024

\medskip

\noindent
Alle Rechte vorbehalten / All rights reserved

\medskip

\noindent
Ohne ausdrückliche Genehmigung des Verlages ist es nicht gestattet, das Werk oder einzelne Teile
daraus nachzudrucken, zu vervielfältigen oder auf Datenträger zu speichern.

\smallskip

\noindent
Apart from any fair dealing for the purpose of private study, research, criticism or review, no
part of this book may be reproduced or translated in any form, by print, photo form, microfilm, or
any other means without written permission. Enquiries should be made to the publishers.

\bigskip

\noindent
Satz: \ \ Nils Jacob Liersch \\
Herstellung: \ \ BoD – Books on Demand GmbH, Norderstedt  \\

\bigskip

\noindent
%\ISBN     

\normalsize

\newpage

%\maketitle
\clearpage
\tableofcontents
\addtocounter{page}{-1}
\thispagestyle{empty}
\clearpage


\mainmatter

\chapter{Introduction}
\cleardoublepage

\section{General remarks}
The \textit{Yogatattvabindu} is a premodern Sanskrit Yoga text on Rājayoga that was written in the first half of the seventeenth century\footnote{The dating of the text is discussed on p.\pageref{dating}.} in northern India.\footnote{The detailed discussion of the place of origin is found on p.\pageref{placeoforigin}.} The most salient feature of the work that makes it historically significant is its highly differentiated taxonomy of types of Yoga. In the \textit{Yogatattvabindu}'s introduction, most manuscripts name fifteen types of Yoga, presented as subtypes of Rājayoga. The text is a yogic compendium written in a mix of mainly prose and 41 verses in textbook-style, where its 58 topics topics are introduced in sections launched by recognizable phrases. Most sections deal with the subtypes of Rājayoga and their effects, but others also cover topics like yogic physiology and cosmogony.

The \textit{Yogatattvabindu} has not been discussed or considered in secondary literature on Yoga. The only exception is \citeauthor{birch2014} (2014: 415–416) who briefly described its list of fifteen Yogas in the context of the `fifteen medieval Yogas' and noted that a similar\footnote{My research suggests that list of fifteen Yogas in Nārāyaṇatīrtha’s \textit{Yogasiddhāntacandrikā} must be chronologically later than the ones found in the \textit{Yogatattvabindu} and its sources. As I will show in the discussion of the fifteen Yogas on p.\pageref{15yogas}, we have to assume that Nārāyaṇatīrtha saw the need to map the fifteen Yogas onto system of the \textit{Pātañjalayogaśāstra} due to their popularity among practitioners in his sphere of activity.} list occurs in Nārāyaṇatīrtha’s \textit{Yogasiddhāntacandrikā} (17th - 18th century), a commentary on the \textit{Pātañjalayogaśāstra} that integrates almost an identical taxonomy of yogas within the \textit{aṣṭāṅga} format. An incomplete account of the fifteen Yogas is found within the Sanskrit Yoga text \textit{Yogasvarodaya}, which is known only through quotations in the \textit{Prāṇatoṣinī} and \textit{Yogakarṇikā}.\footnote{Manuscripts under the name of \textit{Yogasvarodaya} seem to be lost. I was not able to allocate the manuscripts of the text in any manuscript catalogue at hand.} The \textit{Yogasvarodaya} provides a total of fifteen Yogas but names only eight of them in its introductory \textit{śloka}s. A complete account of the text is yet to be found and might be lost forever. The \textit{Yogasvarodaya} is the primary source and template for the compilation of the \textit{Yogatattvabindu}. Rāmacandra closely follows the content and structure by rewriting the \textit{Yogasvarodaya}’s \textit{śloka}s into prose. Due to the incomplete transmission of the \textit{Yogasvarodaya}, Rāmacandra’s \textit{Yogatattvabindu} is a natural and valuable starting point for an in-depth study of the taxonomy of the fifteen types of Yoga. The other source text that Rāmacandra used is the \textit{Siddhasiddhāntapaddhati} whose content he draws on, particularly in the last third of his composition. Another text that includes a similar taxonomy of twelve Yogas divided into three tetrads is Sundardās’s \textit{brāj bhāṣa} Yoga text named \textit{Sarvāṅgayogapradīpikā} which not just shares most of the types of Yogas but also many of the practices and contents found within the \textit{Yogatattvabindu} and \textit{Yogasvarodaya}.\footnote{For a comparative table of the complex Yoga taxonomies see table \ref{15yogastable} on p.\pageref{15yogastable}.}

These complex taxonomies that emerged during the 16th and 17th centuries crossed sectarian divides and were adapted to the specific needs of different authors and traditions. The \textit{Yogatattvabindu} thus encapsulates the diversity of Haṭha- and Rājayoga types and teachings after the \textit{Haṭhapradīpikā} (15th century) that were adopted by a broad spectrum of religious traditions and strata of Indian society. In the particular case of the \textit{Yogatattvabindu}, there are various statements throughout the text that reveal a strategy to detach Yoga from its renunciate connotations and to enforce the supremacy and universality of Rājayoga as a practice that can yield the highest benefits even for practitioners who enjoy worldly pleasures and an extravagant lifestyle. Textual evidence suggests the possibility that \textit{Yogatattvabindu} may be a unique example of a Rājayoga text that was composed for warrior aristocracy and members of an royal court. 

%In addition, the analysis of the \textit{Yogatattvabindu} and the historical retracig of its teachings provides insight into a complex network of at least twenty texts,\footnote{This intertextual network which shares those specific teachings consists of the \textit{Netratantra}, \textit{Śāradatilakatantra}, \textit{Sarvadurgatipariśodhanatantra}, \textit{Ūrmikaulārṇavatantra}, \textit{Tantrāloka}, \textit{Manthanabhairavatantra}, \textit{Śārṅgadhārapaddhati}, \textit{Vivekamārtaṇḍa}, \textit{Śivayogapradīpikā}, (recensions of the \textit{Haṭhapradīpikā}), \textit{Amaraughaśāsana}, \textit{Yogasvarodaya}, \textit{Sarvāṅgayogapradīpikā}, \textit{Nityanāthapaddhati}, \textit{Siddhasiddhāntapaddhati}, \textit{Yogatattvabindu}, \textit{Yogacūḍāmaṇyupaniṣad}, \textit{Maṇḍalabrāhmaṇopaniṣat}, \textit{Haṭhatattvakaumudi} and \textit{Haṭhasaṃketacandrikā}.} all of which include one specific set of yoga theorems and practices with minor deviations - three to five \textit{cakra}s, sixteen \textit{ādhāra}s, two to five \textit{lakṣya}s, and five \textit{vyoma}s. This ancient Śaiva paradigm gave rise to an intertextual network that spans at least an entire millennium. It begins in early śivaite Tantras such as the \textit{Netratantra} and ends in the large late medieval Yoga compendiums like the \textit{Haṭhatattvakaumuḍī} and \textit{Haṭhasaṅketacandrikā}. The examination of this network provides insights into the history of the related yoga traditions and enables, for example, the reconstruction of the genesis of individual yoga categories mentioned in the fifteen Yogas, such as Lakṣyayoga, whose techniques were originally taught in early śivaite Tantras, but were only labeled as a separate type of yoga from the 16th century onwards.

One printed edition of the \textit{Yogatattvabindu} was published in 1905 with a Hindi translation and based on an unknown manuscript(s). This publication has the title ’\textit{Binduyoga}’ confirmed by the printed text’s colophon. However, as I discuss in the course of the introduction, the text was likely known as \textit{Yogatattvabindu}. The consulted manuscripts contain significant discrepancies, structural differences and variant readings between them and the printed edition. Furthermore, the manuscripts are scattered over the Indian subcontinent, which suggests that it was widely transmitted at some point. Lenghty passages of the \textit{Yogatattvabindu} are quoted without attribution in a text called \textit{Yogasaṃgraha} and Sundaradeva’s \textit{Haṭhasaṅketacandrikā}. A critical edition will undoubtedly improve on the published edition and shed further light on the transmission of this important work.

This book contains an introduction, critical edition and annotated translation of the \textit{Yogatattvabindu}. The introduction discusses provenance, authorship and the audience of the \textit{Yogatattvabindu}. A comprehensive discussion of the taxonomy of the fifteen Yogas based on the critical edition of the \textit{Yogatattvabindu}, together with a close examination of the above-mentioned related texts with similar taxonomies, aims to establish their position within the broader history of yoga and particularly elucidates the development of Haṭha- and Rājayoga traditions in the late medieval period. The remainder of the introduction contains an overview of the manuscript evidence and the editorial policies underlying the edition.

\section{Dating the \textit{Yogatattvabindu}}
\label{dating}
The oldest dated manuscript of the \textit{Yogatattvabindu} \getsiglum{N1}\footnote{For a description of the manuscript see  p.\pageref{n1description}.} was written in Nepal \textit{saṃvat} 837, which is 1716 CE. Since the text of this manuscript is missing a significant and lengthy passage (ca. 25\% of the entire text) and contains various corruptions, one can assume that some time had passed from the original composition for the transmission to deteriorate to this extent. Therefore, it is likely that the work was composed at least a few decades before the creation of this Nepalese manuscript, perhaps sometime in the 17th century. The discovery that Sundaradeva's \textit{Haṭhasaṅketacandrikā} quotes a lengthy passage of the \textit{Yogatattvabindu} without attribution confirms this suspicion. The passages quoted from the \textit{Yogatattvabindu} include the teachings on the sixteen \textit{ādhāra}s\footnote{\citetitle{hathasamketacandrikajodhpur} (ms. no. 2244, f. 95r l. 3 -- f. 96r l. 4).} and the teachings on Lakṣyayoga and its subtypes.\footnote{\citetitle{hathasamketacandrikajodhpur} (ms. no. 2244, f. 124r l. 7 -- f. 125r l. 3).} The dating of the \textit{Haṭhasaṅketacandrikā} just recently had to be revised due to the discovery that some first-hand notes surrounding the main text of the Ujjain \textit{Yogacintāmaṇi} were in all likelihood borrowed from Sundaradeva's \textit{Haṭhasaṅketacandrikā}.\footnote{Cf. \citeauthor{birch2024} (2024:52-54).} \citeauthor{birch2018proliferation} (2018) dated the Ujjain \textit{Yogacintāmaṇi} to 1659 CE.\footnote{Cf. \citeauthor{birch2018proliferation}, 2018: 50 [n. 111].} Thus, the \textit{terminus ante quem} for the compilation of the \textit{Haṭhasaṅketacandrikā} is 1659 CE which automatically makes it also the \textit{terminus ante quem} for the \textit{Yogatattvabindu} and the \textit{Yogasvarodaya}, due to the fact that Sundaradeva quoted from the \textit{Yogatattvabindu} and Rāmacandra quoted from and rewrote the contents of the \textit{Yogasvarodaya}. Thus, we can safely assume that the \textit{Yogatattvabindu} was written in the course of the first half of the 17th century or earlier. Because of that Rāmancandra's main source text \textit{Yogasvarodaya} must have been written even earlier.

\subsection{Implications for the dating of the \textit{Yogasvarodaya} and the \textit{Siddhasiddhāntapaddhati}}
Furthermore, \citeauthor{mallinsononline2013}\footnote{Cf. \fullcite{mallinsononline2013}.} estimated the age of the \textit{Siddhasiddhāntapaddhati} to circa 1700. Due to the above-mentioned new date of the \textit{Haṭhasaṅketacandrikā} and because Rāmacandra extensively quotes from \textit{Siddhasiddhāntapaddhati} the new terminus \textit{terminus ante quem} for the dating of the \textit{Siddhasiddhāntapaddhati} likewise must be set to 1659 CE. Thus, the \textit{Siddhasiddhāntapaddhati} was also likely composed during the first half of the 17th century or even ealier.


\chapter{The complex late-medieval yoga taxonomies}
\label{yogas_list}
\clearpage


\section{The rise of diversity: The increasing complexity of Yoga teaching systems in late medieval and pre-colonial India}

In diesem Kapitel soll es darum gehen, dass zwischen dem 17. und 18. Jh. in Indien parallel zu einer Populariserung des Yoga in breiten Schichten der Gesellschaft jenseits der asketischen Traditionen eine allgemeine Entwicklung zu beobachten ist, die sich in gesteigerter Komplexität äußert. In den damals zirkulierenden Texten kommt es zu einer Steiugerung der Anzahl der gelehrten Cakras, Āsanas, Kumbhakas, aber auch die Taxonomien der einzelnen Yogakategorien die gelehrt werden nehmen an Komplexität zu. 

\section{The texts of the complex yoga taxonomies}

\subsection{Yogasiddhāntacandrikā}

Versucht alle 15 Yogas im Samādhipāda des Pātañjalayogaśāstra unterzubringen.
Siehe auch Powell 2023. 

%The Yogasiddhāntacandrikā is an ambitious work that attempts to unify a vast array of doctrines within the
%framework of the Yogasūtras of Patañjali. In particular, Nārāyaṇatīrtha attempts to synthesize fifteen
%different systems of yoga within the first chapter (pāda). All of these are believed to result in the state of
%Rājayoga, which Nārāyaṇatīrtha understands as synonymous with the nididhyāsana of Vedānta and
%the asaṃprajñātasamādhi of Pātañjalayoga.111


\section{Comparative analysis of the complex Yoga taxonomies}

The similarities between the Yoga taxonomies of Rāmacandra's \textit{Yogatattvabindu}, his source text, the \textit{Yogasvarodaya} as well as the taxonomies laid out by Nārāyaṇatīrtha in his \textit{Yogasiddhāntacandrikā} and Sundardās' \textit{Sarvāṅgayogadīpikā} which all emerged within the same time period (16th - 17th centuries) have been initially observed and discussed briefly by \citeauthor{birch2014} (2014)\footnote{See \citeauthor{birch2014}, 2014: 415-416.} In the following chapter, the complex taxonomies and their single categories of Yoga are examined within a comparative analysis.

The comparative analysis will follow the structure of the individual Yogas outlined in the \textit{Yogatattvabindu}. Each Yoga will initially be described based on the explanations in the \textit{Yogatattvabindu}, and its content will be compared with the explanations of the corresponding Yoga in the texts with similar taxonomies. The comparison will broaden and clarify our understanding of the respective spectrum of meanings of the individual Yoga categories in the discursive field of the authors of the texts containing the taxonomies. This comparison results in the documentation of the discursive web of word usage of various Yoga categories between the 16th and 17th centuries CE, most probably mainly localised in central northern India.\footnote{The complex taxonmies evolved and circulated most likely in central northern India. For a detalled discussion see p.\pageref{location}.} Individual Yoga categories that do not appear in the list of the \textit{Yogatattvabindu} but are listed in the other texts with complex taxonomies will also be covered and outlined. In addition, Yoga categories that do not appear in any of the analysed lists but are nevertheless mentioned in the texts will also be covered so that this analysis attempts to approximate the overall picture of all Yoga categories used during the period under consideration as closely as possible. However, it is essential to emphasise that the comparison of Yoga categories focuses primarily on those texts that contain complex Yoga taxonomies and cannot claim to be exhaustive. Although the analysis and comparison of the Yoga categories can be extended to other Yoga texts, locations and time periods if necessary or valuable, the restriction to the complex Yoga taxonomies should be maintained to prevent this already complex endeavour going ad absurdum.\footnote{The historical tracing and analysis of developments in the reception history of the Yoga categories presented in the complex taxonomies generates valuable insights, as has been demonstrated by the example of the development of the late medieval Kriyāyoga into the modern forms of Kriyāyoga, beginning with the lineage of the world-famous Paramahaṃsa Yogānanda. See the chapter \textit{Excursus: Popularisation of a new Kriyāyoga in a global context} on p.\pageref{excursuss} et seqq. Unfortunately, it is not possible to analyse the developments in the history of reception with regard to all other Yoga categories, in particular the transition to the modern and global context, within the scope of this work, but in my view it would offer a promising starting point for further research contributions that have not yet been realised}.      

\begin{table}[h]
    \centering
    \begin{tabularx}{\textwidth}{>{\raggedright\arraybackslash}p{0.05\textwidth}XXXX}
        \toprule
        No. & \textit{Yogatattvabindu} & \textit{Yogasvarodaya} & \textit{Yogasiddhāntacandrikā} & \textit{Sarvāṅgayogadīpikā} \\
        \midrule
        1. & \textit{kriyāyoga} & \textit{kriyāyoga} & \textit{kriyāyoga} & \textit{\textbf{bhaktiyoga}} \\
        2. & \textit{jñānayoga} & \textit{jñānayoga} & \textit{caryāyoga} & \textit{mantrayoga} \\
        3. & \textit{caryāyoga} & \textit{karmayoga} & \textit{karmayoga} & \textit{layayoga} \\
        4. & \textit{haṭhayoga} & \textit{haṭhayoga} & \textit{haṭhayoga} & \textit{carcāyoga} \\
        5. & \textit{karmayoga} & \textit{dhyānayoga} & \textit{mantrayoga} & \textit{\textbf{haṭhayoga}} \\
        6. & \textit{layayoga}  & \textit{mantrayoga} & \textit{jñānayoga} & \textit{rājayoga} \\
        7. & \textit{dhyānayoga} & \textit{urayoga}   & \textit{advaitayoga} & \textit{lakṣayoga} \\
        8. & \textit{mantrayoga} & \textit{vāsanāyoga} & \textit{lakṣyayoga} & \textit{aṣṭāṅgayoga} \\
        9. & \textit{lakṣyayoga} & -                   & \textit{brahmayoga} & \textit{\textbf{sāṃkhyayoga}} \\
        10. & \textit{vāsanāyoga} & -                   & \textit{śivayoga} & \textit{jñānayoga} \\
        11. & \textit{śivayoga} & -                    & \textit{siddhiyoga} & \textit{brahmayoga} \\
        12. & \textit{brahmayoga} & -                  & \textit{vāsanāyoga} & \textit{advaitayoga} \\
        13. & \textit{advaitayoga} & -                 & \textit{layayoga} & - \\
        14. & \textit{siddhayoga} & -                  & \textit{dhyānayoga} & - \\
        15. & \textit{rājayoga} & - [\textit{rājayoga}]& \textit{premabhaktiyoga} & - \\
        \bottomrule
    \end{tabularx}
    \caption{Complex Taxonomies of Yoga in Yoga Texts of the 17th - 18th Centuries}
    \label{tab:complextaxonomies}
\end{table}

\section{1. Kriyāyoga}

Kriyāyoga is the first Yoga within the list of fifteen Yogas presented by Rāmacandra and his source text \textit{Yogasvarodaya}. Remarkably, Nārāyaṇatīrtha also positions Kriyāyoga at the first position within the list of fifteen Yogas in his \textit{Yogasiddhāntacandrikā}. Sundardās, on the other hand, omits Kriyāyoga within his taxonomy.

\subsection{Kriyāyoga in the \textit{Yogatattvabindu}}

Since Rāmacandra refers to all fifteen Yogas as variants of Rājayoga in his initial definition of Yoga, and no explicit hierarchy is recognisable from his formulations in the text, all variants of Rājayoga appear to have been regarded by him as equally effective. All Yogas aim towards the same goal: long-term durability of the body (\textit{bahutarakālaṃ śarīrasthitiḥ}). The positioning of Kriyāyoga does not initially provide any information about the efficiency or the assignment of differently talented practitioners to a particular type of Yoga, as was the case in i.e. the widespread fourfold taxonomies.\footnote{According to \citetitle{amaraugha2024}\textit{prabodha} 18-24, Mantrayoga is best suited for the weak, Layayoga for the average, Haṭhayoga for the talented and Rājayoga for the exceptionally talented practitioner. In \citetitle{datta2024} 14, one finds the statement that the lowest practitioner should perform mantra yoga, which is then also referred to as the lowest Yoga. \citetitle{mallinson2007} 12-28 expands this fourfold scheme of Yogas and practitioners with a temporal dimension. The weak practitioner needs twelve years to succeed with Mantrayoga, the average practitioner needs eight years with Laya, the able practitioner six years with Haṭha and the exceptional practitioner three years with Rājayoga} Implicit hierarchical aspects are nevertheless present - although all Yoga types are a type of Rājayoga, Rāmacandra nonetheless places Rājayoga in the final and topmost position of his taxonomy.
The only apparent reason why Rāmacandra specifies Kriyāyoga as the first Yoga seems to be that his primary source text, whose content structure he largely follows,\footnote{see the chapter on ``structural inconsistencies'' on p.\pageref{struktur},} specifies this type of Yoga as the first.

The passage on Kriyāyoga in the \textit{Yogatattvabindu} is relatively short. The four verses presented by Rāmacandra are quoted without attribution from the \textit{Yogasvarodaya}. A prose section repeats the content of the verses. By definition, Kriyāyoga in \textit{Yogatattvabindu} is ``liberation through [mental] action'' (\textit{kriyāmuktir ayaṃ yogaḥ}). In contrast to Rāmacandra's worldly definition of Rājayoga and its subcategories, here, liberation (\textit{mukti}) overrides this initial goal. In addition, the practitioner achieves ``success in one's own body'' (\textit{svapiṇḍe siddhidāyakaḥ}). The method of Kriyāyoga involves restraining any [mental] wave before an action. This restraint consists of reducing negative [mind-]waves and cultivating positive ones. Noticeably, the number of negative waves significantly exceeds the number of positive waves.

\begin{table}[h]
    \centering
    \begin{tabularx}{\textwidth}{XX}
        \toprule
        \textbf{Mental waves to be cultivated} & \textbf{Mental waves to be reduced} \\
        \midrule
        Patience (\textit{kṣamā}) & Envy (\textit{matsārya}) \\
        Discrimination (\textit{viveka}) & Selfishness(\textit{mamatā})\\
        Equanimity (\textit{vairāgya}) & Cheating (\textit{māyā})\\
        Peace (\textit{śānti}) & Violence (\textit{hiṃsā})\\
        Modesty (\textit{santoṣa}) & Intoxication (\textit{mada})\\
        Desirelessness (\textit{niṣpṛha}) & Pride (\textit{garvata})\\
        & Lust (\textit{kāma}) \\
        & Anger (\textit{krodha}) \\
        & Fear (\textit{bhaya})\\
        & Laziness (\textit{lajjā})\\
        & Greed (\textit{lobha})\\
        & Error (\textit{moha})\\
        & Impurity (\textit{aśuci})\\
        & Attachment and aversion (\textit{rāgadveśau}) \\
        & Disgust and laziness (\textit{ghṛṇālasya})\\
        & error (\textit{bhrānti})\\
        & Deceit (\textit{daṃbha})\\
        & Envy (repeatedly) (\textit{akṣama})\\
        & Confusion (\textit{bhrama})\\
        \bottomrule
    \end{tabularx}
    \caption{Mental waves to be cultivated and reduced in Rāmacandra's Kriyāyoga}
    \label{tab:waves}
\end{table}

The one who cultivates positive [mind-]waves and reduces the negative is called a \textit{kriyāyogī}. In the prose passage of the section, the term \textit{bahukriyāyogi} is used. The term is unprecedented in the rest of the yoga literature and presumably intends to express many reduced and cultivated waves.\footnote{Cf. section \uproman{2} of the \textit{Yogatattvabindu} for its text on the subject Kriyāyoga.} 

\subsection{Kriyāyoga in the \textit{Yogasvarodaya}}
A closer examination of the Kriyāyoga section in the \textit{Yogasvarodaya} reveals Rāmancandra's reductionism since he excludes significant aspects of the original concept of the \textit{Yogasvarodaya}'s Kriyāyoga.

%YK 1.214-216

\begin{quote}
\textit{dhyānapūjādānayajñajapahomādikāḥ kriyāḥ} |\\
\textit{kriyāmuktimayo yogaḥ svapiṇḍe siddhidāyakaḥ}\footnote{svapiṇḍe siddhidāyakaḥ YTB] sapiṇḍisiddhidāyakaḥ YSv sapiṇḍisiddhidāyakaḥ YK} || 1 ||

(1) Actions are meditation, ritual veneration, donation, recitation, fire sacrifice, etc. 
The Yoga made of liberation through action[s] bestows success in one's own body. 

\textit{yat karomīti saṅkalpaṃ kāryārambhe manaḥ sadā} |\\
\textit{tat sāṅgācaraṇaṃ kurvan kriyāyogarato bhavet} || 2 ||

(2) ``Whatever I do'' at the beginning of an action, the mind always has an intention.  
Doing that [following] procedure with all its parts, one becomes established in Kriyāyoga.  

\textit{kṣamāvivekavairāgyaśāntisantoṣanispṛhāḥ} |\\
\textit{etad yuktiyuto yo'sau kriyāyogo nigadyate} || 3 ||

(3) Patience, discrimination, equanimity, peace, modesty, desirelessness:
The one endowed with these means is said to be a Kriyāyogī.

\textit{mātsaryaṃ mamatā māyā hiṃsā ca madagarvitā} |\\
\textit{kāmaḥ krodho bhayaṃ lajjā lobho mohas tathā'śuciḥ} || 4 ||

(4) Envy, selfishness, cheating, violence, intoxication and pride,
lust, anger, fear, laziness, greed, error, and impurity.

\textit{rāgadveṣau ghṛṇālasyaśrāntidambhakṣamābhramāḥ} |\\
\textit{yasyaitāni na vidyante kriyāyogī sa ucyate} || 5 ||

(5) Attachment and aversion, disgust and laziness, error, deceit, envy [and] confusion:
Whoever does not experience these is called a Kriyāyogī.

\textit{sa eva muktaḥ sa jñānī caṇḍināśena īśvaraḥ} |\\
\textit{kriyāmuktikaro yo'sau rājayogaḥ sa muktidaḥ} || 6 ||(om. YK)

(6) He alone, the wise one, the lord, through the destruction of impetuous [behaviour]
who performs the liberation through action[s] is liberated. This Rājayoga is the bestower of liberation.

\textit{yāvan mano layaṃ yāti kṛṣṇe svātmani cinmaye} | \\ 
\textit{bhaved iṣṭamanā mantrī japahomau samabhyaset} || 7 ||\footnote{7ab \approx \citetitle{rudrayamala1937} 38.58cd.}(om. YSv) 

(7) Until the mind enters absorption into Kṛṣṇa, in one's own self, into consciousness,
the mantra practitioner (\textit{mantrin}) should practise recitation and fire sacrifice with an aspiring mind. 

\textit{vidite paratattve tu samastair niyamair alam} |\\
\textit{tālavṛntena kiṃ kāryaṃ lavdhe malayamārute} || 8 ||\footnote{\approx \citetitle{kularnavatantra} 9.28 \& \citetitle{yuktabhavadeva} 1.80.} (om. YSv) 

(8) When the highest principle has been realised through all the \textit {niyama}s, as is proper,
why should one wave the palm frond when the wind from the Himalayas has already reached?

\textit{tāvat karmmāṇi kurvanti yāvajjñānaṃ na vidyate} |\\ 
\textit{jñāne jāte pareśāni karmākarma na vidyate} || 9 ||(om. YSv) 

(9) As long as [regular?] actions are performed, so long realisation is unknown.
When knowledge ensues, oh, Supreme Goddess, neither action nor non-action is known.
\end{quote}

These verses\footnote{The numbering used here was introduced by me for practical reasons and does not correspond to the original numbering of the verses in the citations of the source texts. The \textit{Prāṇatoṣiṇī} does not number the verses at all. The verses can be found in the printed edition of the \textit{Prāṇatoṣiṇī} on p. 831. The verses here are in the \textit{Yogakarṇikā} with the numbering 1.209-216 and can be found in the edition on p. 17.} stem from the only two currently available sources of the \textit{Yogasvarodaya}, namely the quotations from the \textit{Prāṇatoṣiṇī}\footnote{A considerable part of the \textit{Yogasvarodaya} is quoted with source reference (\textit{yogasvarodaye}).} and the \textit{Yogakarṇikā}.\footnote{Normally the \textit{Yogakarṇikā} quotes its sources. This passage is one of the few exceptional cases in which the verses have been taken from the \textit{Yogasvarodaya} without citing the source. However, this passage ends after verse 1.216 with ``\textit{iti yogasaṅketāḥ |}''.} The quotations of both texts essentially correspond, but the last verses of the passage differ. It cannot be ruled out that the last three verses of the \textit{Yogakarṇikā} in particular come from a different source and were not present within the \textit{Yogasvarodaya}. However, their content is so closely interwoven with the preceding verses that this scenario can be considered unlikely.

The main difference to the Kriyāyoga that Rāmacandra has constructed from these verses is the definition of the actions (\textit{kriyāḥ}) mentioned immediately at the beginning of the verses, of which the actions (\textit{kriyā}s) of Kriyāyoga is then predominantly composed, namely of (1) meditation, (2) ritual worship of God, (3) offerings, (4) recitation and (5) fire sacrifice, etc. Furthermore, while Rāmacandra declares the elements mentioned in the table \ref{tab:waves} as waves (\textit{kallola}) of the mind which are either required to be cultivated or reduced before any action is executed, the same elements are conceptualised in the \textit{Yogasvarodaya} as the intentions (\textit{saṅkalpa}) preceding the previously defined actions (\textit{kriyā}s), which should be observed.

In the three verses concluding this section, which are only handed down in the \textit{Yogakarṇikā}, the practitioner is referred to as \textit{mantrin} and should perform recitation and fire offerings until entering absorption (\textit{laya}).

A possible historical link, particularly in front of the Vaiṣṇava background, is the model of Kriyāyoga as found in the \textit{Uddhavagīta}\footnote{See i.e., \citeauthor{uddhavagita2007} (2007).} which is a part of the famous \textit{Bhāgavatapurāṇa}\footnote{See i.e., \citeauthor{bhagavata} (1950).}. Here, in chapter XXII.1-55 Kṛṣṇa describes a Vaiṣṇava form of Kriyāyoga in response to a request by his disciple Uddhava. The practice entails a very complex and devotional ceremonial veneration of the deity through offerings such as flowers and food, accompanied by the recitation of prescribed mantras, meditation, and the ritual consecration of the deity, among other rites. According to the text, this type of Yoga is the most beneficial for women and the working class (22.4) and is considered a means for liberation from the fetters of Karma (22.5). The Kriyāyoga described here is presented to be in line with both the Vedas and the Tantras, considering enjoyment (\textit{bhukti}) and liberation (\textit{mukti}) and is promised to bestow perfection in both this life and the next, by the Lord's grace (22.49).  

Furthermore, this concept of Kriyāyoga in the \textit{Yogasvarodaya} might be linked to the \textit{kriyāpāda}\footnote{See e.g. \citeauthor{ganesan2016saiva} (2016) and \citetitle{mrgendragama}, Ed. pp. 1-205.} of the Śaiva \textit{āgama}s. The Śaiva \textit{āgama}s are collections of various tantric traditions, written in Sanskrit or Tamil, in which cosmology, epistemology, philosophical teachings, various practices such as meditation or Yoga, mantra recitation, worship of the gods, etc. are described. These texts\footnote{The fourfold division of \textit{pāda}s is only present in a limited number of Āgamas: \textit{Kiraṇa}, \textit{Suprabheda}, \textit{Mṛgendra} and \textit{Mataṅgaparameśvara} (as Upāgamas), see \citeauthor{brunner1994place} , 1993: 225-461 for an overview.} usually consist of four sections (\textit{pāda}s): The \textit{jñānapāda} (knowledge section), \textit{kriyāpāda} (action section), \textit{caryāpāda} (behaviour section) and the \textit{yogapāda} (yoga section).\footnote{The order or the \textit{pāda}s varies, but the \textit{yogapāda} is always the last.} It can be no coincidence that \textit{jñāna°}, \textit{kriyā°} and \textit{caryā°} were each integrated as a separate Yoga category within the taxonomy of the fifteen Yogas\footnote{see p.\pageref{intro}.}. The \textit{kriyāpāda} is the section of a Śaiva \textit{āgama} that describes rules and practices for the performance of various rituals such as the significant initiation (\textit{dīkṣa}), ceremonies and worship of the gods. Additionally, \textit{prāṇāyāma} techniques and meditations are often found as parts of these rituals. There are also explanations of the nature of \textit{mudrā}s, \textit{maṇḍala}s and \textit{mantra}s. Furthermore, various characteristics of different types of Śaiva initiates\footnote{These are \textit{samayin, putraka, sādhaka, ācārya,} and \textit{astrābhiṣeka}.} can be found here.\footnote{See \citeauthor{ganesan2016saiva} (2016) for a general overview of the four \textit{pāda}s. One of the few Śaiva \textit{āgama}s that has been edited and translated into a Western language (French) is the \citetitle{mrgendragama}. For this see \citeauthor{mrgendragama} (1962) \& \citeauthor{mrgendragamabrunner} (1985).} The \textit{kriyā}s mentioned at the beginning of the \textit{Yogasvarodaya} - meditation, ritual veneration, donation, recitation, fire sacrifice, etc. have hardly deniable parallels to the \textit{kriyāpāda}s of the Śaiva \textit{āgama}s and thus could have their reception-historical roots precisely there. The other part, however, which describes the cultivation or reduction of certain mental configurations preceding all actions (\textit{saṅkalpa}) or [mental] waves (\textit{kallola}), I have not yet been able to locate in the Śaiva \textit{āgama}s, but they seem to be a simplyfied rendering of the Pātañjalean model of Kriyāyoga that was passend on in hitherto unknown traditions that practiced this type of Kriyāyoga.

\subsection{Kriyāyoga in the \textit{Yogasiddhāntacandrikā}}

The Kriyāyoga in Nārāyaṇatīrtha's commentary on \textit{Pātañjalayogaśāstra} entitled \textit{Yogasiddhāntacandrikā} presents Kriyāyoga as the first of his fifteen Yogas, which he locates in Pātañjalayoga.\footnote{For an earlier brief discussion of Kriyāyoga in Nārāyaṇatīrtha's \textit{yogacandrika} see \citeauthor{penna2004}, 2004: 62-66.} The term Kriyāyoga occurs in \textit{Pātañjalayogaśāstra} 2.1. According to the introduction to this \textit{sūtra}, in the \textit{bhāṣya}-part of the \textit{Pātañjalayogaśāstra}, Kriyāyoga is the means by which someone with a distracted mind can also attain Yoga (\textit{vyutthitacitto 'pi yogayuktaḥ}). In \citetitle{yogasutra} 2.1, Kriyāyoga is defined as follows:
\begin{quote}  
  \textit{tapaḥsvādhyāyeśvarapraṇidhānāni kriyāyogaḥ} |
\end{quote}
\begin{quote}
The Yoga of action consists of auterity, the self-study and devotion to the supreme lord. 
\end{quote}

Kriyāyoga, or ``yoga of action'', is the action oriented method of Yoga consisting of three elements. Namely, austerity (\textit{tapas}), which according to the \textit{bhāṣya} should be practised both mentally and physically, the repetition of \textit{mantra}s or the study of sacred literature (\textit{svadhyāya}) and devotion to the supreme lord (\textit{īśvarapraṇidhāna}).
According to \citetitle{yogasutra} 2.2, these three elements of Kriyāyoga should lead the practitioner to attain \textit{samādhi} by reducing the so-called \textit{kleśa}s. This explanatory model is picked up by Nārāyaṇatīrtha.\footnote{\citeauthor{yogacandrika}, 2000:71.} The five \textit{kleśa}s consist of ignorance (\textit{avidyā}), self-centredness (\textit{asmitā}), attachment (\textit{rāga}), aversion (\textit{dveṣa}) and fear of death (\textit{abhiniveśa}). 
All three main components of Patañjali's Kriyāyoga are not mentioned in the \textit{Yogatattvabindu} and \textit{Yogasvarodaya}. Nevertheless, a practice similar to the reduction of the \textit{kleśa}s can also be found here. Although the specific fear of death (\textit{abhiniveśa}) is not mentioned, the more general term for fear (\textit{bhaya}) is cited.\footnote{The details of Nārāyaṇatīrtha's understanding of Kriyāyoga have already be discussed by \citeauthor{penna2004} (2004: 62-66) and will therefore not be covered here again.}
The Kriyāyoga in \textit{Yogatattvabindu} and \textit{Yogasvarodaya} could, therefore, be perhaps regarded as a degenerated or simplified variant of the Pātañjalean model, which restricts itself predominantly to the aspect of the reduction of negative waves of the mind, which is comparable to the reduction of \textit{kleśa}s and adds the aspect of cultivating positive mind waves to be mix. In both systems, Kriyāyoga is a means for liberation.\footnote{The Kriyāyoga of the \citetitle{yogasutra} will not be dealt with in detail here, as this has already been done in countless academic and informal publications. For the \textit{sūtra}s related to Kriyāyoga and Patañjali's autocommentary in Sanskrit with English translation, see \citeauthor{yogasutra} 1983: 113 et seqq. For a comprehensible and more accessible overview, see \citeauthor{bryant2009} 2009: 170 et seqq.}

\subsection{Kriyāyoga in the complex late-medieval Yoga taxonomies}

The analysis of Kriyāyoga within the taxonomies of fifteen yogas shows two distinct models. One is Nārāyaṇatīrtha's model, which draws directly on the Kriyāyoga of \textit{Pātañjalayogaśāstra}. Additional śaiva influences characterise the other model of Kriyāyoga that seems to have been locally prominent in the 17. - 18. century C.E. The precisely defined \textit{kriyā}s of the \textit{Yogasvarodaya} must be historically linked to the \textit{kriyāpāda}s of the Śaiva \textit{āgama}s, whereby the core practice of reducing and cultivating specific mental configurations before any action is loosely associated with the Kriyāyoga of the \textit{Pātañjalayogaśāstra}. The observation that the \textit{kriyā}-, \textit{caryā-}, and \textit{jñānayoga}s, are an allusion to the \textit{kriyā}-, \textit{caryā-}, \textit{jñāna-} and \textit{yogapāda}s of the Śaiva \textit{āgama}s, shows that Nārāyaṇatīrtha, as a proponent of the \textit{Pātañjalayoga}, was most likely not the originator of the fifteenfold taxonomy, but rather that the taxonomy of the fifteen Yogas originated from local discourses around the authors and had achieved such local popularity at the time that Nārāyaṇatīrtha forced the fifteenfold taxonomy into Patañjali's \textit{Yogaśāstra} in order to show that the Yogaśāstra \textit{par excellence} and all those varieties of Yogas that were discussed in his sphere are in truth already present in the ``classical'' system of Patañjali.

\section{Excursus: Popularisation of a new Kriyāyoga in a global context}
\label{excursus}
The comparatively unique treatises on Kriyāyoga, which can only be found in the Yoga literature from the 17th-century onwards\footnote{The terminus \textit{ad quem} for the \textit{Yogasvarodaya} and \textit{Yogatattvabindu} is 1659 CE, see p.\pageref{dating} for the details.} in \textit{Yogasvarodaya} and Rāmacandra's \textit{Yogatattvabindu}, which deviate from the Pātañjala model, albeit not entirely, and, as shown, show clear influences of tantric origin, can be regarded as marginal phenomena for the time being. The briefly touched upon model of \textit{Uddhavagītā}, which describes a Kriyāyoga method for \textit{mukti} and \textit{bhukti} through ritual worship of god, is also comparatively rare in the literature. The overwhelming majority of the Sanskrit yoga texts written in the second millennium CE, as in the case of Nārāyaṇatīrtha's \textit{Yogasiddhāntacandrikā}, are based on the model of Kriyāyoga propagated in the \textit{Pātañjalayogaśāstra}. Accordingly, it was above all the publication of the \textit{Yogasūtra} in the West, beginning with the translation by Henry Thomas Colebrooke in 1805\footnote{See \parencite{colebrooke2014} for a detailed discussion,} which ensured that the concept of Kriyāyoga contained therein also dominated the understanding of the term in academic and informal discourse in the West for a long time. 

The Western discourse only changed with the global success and popularity of Paramahaṃsa Yogānanda (1893-1952) and the \textit{Self Realisation Fellowship} he founded in 1920, which, measured against the predecessor models forms of Kriyāyoga outlined above, spread an innovative Yoga practice under the generic term Kriyāyoga. The influence of Yogānanda and others significantly changed and expanded the range of meanings of the term Kriyāyoga. In addition to various books published by Yogānanda, it was above all, the book \citetitle{autobioyogi}, the autobiography of Yogānanda himself, published in 1946, which paved the way for Yogānanda's success. To this day, this work is considered a classic in popular Yoga literature, has been in print for over seventy years and has been translated into more than 50 languages.\footnote{Cf. \cite{yoganandawebsite}.} It also has a large global following to this day. Yogānanda, his books, his followers and the numerous books written by his followers have popularised this innovative and new form of Kriyāyoga beyond the Indian subcontinent. The term Kriyāyoga was allegedly already defined by Yogānanda's predecessors, namely Lahiḍi Mahāśaya (1828-1895) and Śrī Yukteśvar Giri (1855-1936), as the central generic term for the Yoga practice of this line of tradition.\footnote{Cf. \citeauthor{govindan2010} 2010:51-52} 

One of Yogānanda's contemporaries was Svāmī Śivānanda Sarasvatī (1887-1963), who similarly propagated a new form of Kriyāyoga. Although his Kriyāyoga was initially based mainly on the Pātañjalayoga model, it was expanded under the same umbrella term with Haṭhayoga practices and possibly influenced by Yogānanda's model. This expansion and integration of new practices under the umbrella term Kriyāyoga was continued excessively by his students, above all Svāmī Satyānanda Sarasvatī (1923-2009), the founder of the famous \textit{Bihar School of Yoga} (since 1962).

The resulting popularity of Kriyāyoga triggered a global wave and inspired others, who in turn developed similar but sometimes differently nuanced Kriyāyoga systems. One example is S.A.A. Ramaiah, who founded the \textit{Kriya Babaji Yoga Sangam} in 1952. In this case, too, there is a global following.\footnote{Cf. \cite{kriyababajiyoga}}.

It was the actors mentioned above, above all Yogānanda, who ensured the global popularisation of this new form of Kriyāyoga so that their concepts are at least as well known in recent public discourse, if not better known, than the Kriyāyoga of the \textit{Pātañjalayogaśāstra}.

These new forms of Kriyāyoga, which can only be traced from the beginning of the 19th century, are, as will be shown, a reservoir for innovative combinations and further developments of numerous practices already codified in Yoga texts in the medieval to pre-colonial period, which were integrated into seemingly coherent practice systems by actors such as Yogānanda, Śivānanda, Ramaiah, etc. The statements made by their traditions about the historicity of their Yoga practice utilise established narratives to lend this form of Kriyāyoga a tradition and historical legitimacy.\footnote{For example, the tracing back of the Yoga tradition to a legendary founding figure, the time of the master in the Himalayas, lost writings that suddenly reappear and legitimise the practice can already be found in a similar form in the lineages of T. Krishnamarcharya. See \citeauthor{singleton2013gurus}, 2013: 81-121.}

\subsection{The Kriyāyogas of the lineages of Paramahaṃsa Yogānanda, Svāmī Śivānanda Sarasvatī and Ramaiah}

So what constitutes these new forms of Kriyāyoga? To answer this question, recent publications on this topic were consulted.\footnote{This list is certainly not exhaustive. Nevertheless, I have consulted a wide range of these publications available to me. 1. For the Yogānanda model: \citeauthor{autobioyogi} (1949); \citeauthor{kriyayogalowenstein} (2021); \citeauthor{kriyayogasarasvati1981} (1981); \citeauthor{hariharananda1989} (1989); \citeauthor{kriyayogaupanishad1993} (1993) and \citeauthor{kriyayogasturgess2015} (2015). 2. For the Śivānanda model: \citeauthor{shivanandakriya1982} (1955) and \citeauthor{kriyayoganityananda2013} (2013). 3. And for the the Ramaiah model: \citeauthor{govindan2010} (2010).} The following is a brief outline of the main features of the Yogānanda, Śivānanda and Ramaiah models of Kriyāyoga without claiming to be exhaustive. To my knowledge, a comprehensive and complete historical study of Kriyāyoga has not yet been carried out and cannot be done within this framework. This attempt is an outline and should be understood as a first approach to the topic in order to differentiate between the models circulating in public discourse on the one hand and, on the other, to formulate a hypothesis on the transition from the older models to the newer models, as these are very close in time.  

\subsubsection{Definitions}

The publications consulted contain various creative etymologies and explanations of the term Kriyāyoga. \citeauthor{hariharananda1989}, a Kriyāyoga teacher authorised by Yogānanda \footnote{Cf. \citeauthor{hariharananda1989} 1989: 16.} himself explains: \begin{quote} 'Kriya Yoga' are Sanskrit words, a combination of two root words. One is Kriya and the other is yoga. In the word Kriya there are two syllables: kri and ya. Kri means to pursue your work in daily life and ya means to be ever aware of the invisible God who is abiding in you and is directing and accomplishing work through you. \ldots  The second word, 'yoga,' literally means union of the visible body with the invisible body. This union is always present in everyone. (\citeauthor{hariharananda1989} 1989: 83) \end{quote}
Another etymology of the term \textit{kriyā} can be found in \citeauthor{kriyayogalowenstein} (2021: 91): \begin{quote} \ldots kri meaning ``work'' and ya meaning ``soul'' or ``breath'' = The Work to be done with the Souls breath. \end{quote}
The most complex explanation of the term can be found in \citeauthor{kriyayoganityananda2013} (2013: 2-3), who also locates himself in the Yogānanda tradition: \begin{quote}
  The word \textit{kriyā} is composed of the letters \textit{k}, \textit{r}, \textit{i}, \textit{y}, and \textit{ā}. The letter -\textit{k} (or \textit{ka}), \textit{ka-kāra}, represents the Lord, \textit{Īśvara}. The Transcendental Lord, \textit{Parama Śiva}, when he manifests Himself in the suble world and makes Himself ready for creation He becomes \textit{Īśvara}. The letter-\textit{r} (or \textit{ra}), \textit{ra-kāra}, represents fire, light and manifestation. Creation is not seen by us with the ether and air elements since these are subtle elements. We are able to see manifestation from the fire element onwards. The letter -\textit{i}, \textit{i-kāra}, represents energy or \textit{śakti}. So \textit{kri} is the activating power of the Lord manifested in creation. The activating power is called \textit{prāṇa} or vital force. The letter -\textit{y} (or \textit{ya}), \textit{ya-kāra}, represents the air element and the letter -\textit{ā}, \textit{ā-kāra}, represents form. For the manifestations to take a form, \textit{ākāra}, the Lord acts with the air element. With the ether element there is no form. The air element or gaseous state is the first created form although we only see the forms from the fire element onwards. Through the action of air the whole universe is manifested. This is the action of the Life-force, \textit{prāṇakarma}, of the Lord. The word \textit{kriyā} normally means action, but this is the action of god. We are made with the same principle God is. Our identification with the physical body makes us separate from God and this is the state of ignorance. We have to eradicate this ignorance by the action of God, i.e., the action of the breath, \textit{prāṇakarma}. Our mind is the result of ignorance and is responsible for the wrong identification. Breath-practice, \textit{prāṇakarma}, absorbs the mind into the vital force. This action of God reverses the process and leads us from body to God. This is why it is so necessary to perform that action. That is our spiritual practice. Then that action, \textit{kriyā}, becomes yoga. \end{quote}
\citeauthor{kriyayogasarasvati1981} (1981: 699), an important proponent of the Śivānanda model, defines Kriyāyoga as follows: \begin{quote} The Sanskrit word \textit{kriya} means `action' or `movement'. \textit{Kriya Yoga} is so called because it is a system where one intentionally rotates one's attention along fixed pathways. This movement of awareness is done, however with control. Also kriya yoga is so called because one moves the body into specific mudras, bandhas and asanas according to a fixed scheme of practice. The word \textit{kriya} is often translated as meaning `practical'. This is indeed a good definition, for kriya yoga is indeed practical. It is concerned solely with practice, without the slightest philosophical speculation. The system is designed to bring results, not merely to talk about them. Sometimes the word \textit{kriya} is translated as `preliminary'. This too is a good definition, for kriya yoga is a preliminary practice that leads first to dharana and then eventually to the transcendental state of dhyana (meditation) and yoga (union). It is a technique which has been designed to lead to that state of being which is beyond all techniques. Finally, the word \textit{kriya} is used to describe each individual practice. Thus the process of kriya yoga consists of a number of kriyas each being done one after the other in a fixed sequence.\end{quote}
\citeauthor{govindan2010} (2010: 214), a student of Ramaiah offers a simple explanation of the term: \begin{quote} Kriyā is an activity performed with mindfulness.\end{quote}

As different as the concepts presented here may seem, they have in common that they are about consciously performed actions or practices that connect people with God or are intended to bring about a transcendent state, a state of yoga. In his definition, \citeauthor{kriyayoganityananda2013} already mentions the central action (\textit{kriyā}) that should lead to a connection with God, namely breathing practice (\textit{prāṇakarma}). In addition, \citeauthor{kriyayogasarasvati1981} also mentions other practices such as directing attention, \textit{mūdra}s, \textit{bandha}s and \textit{āsana}s.  

Further definitions can be found in the consulted texts. However, these are sufficient for the purposes here, as they illustrate the basic idea of the new models of Kriyāyoga on the one hand and show the fundamental diversity and openness of the model, which permeates all areas of these new forms of Kriyāyoga, on the other.  

\subsubsection{Histories of the new forms of Kriyāyoga from an emic perspective}

\citeauthor{kriyayoganityananda2013} (2013: 2-7), who places himself in the lineage of Yogānanda, explains that Kriyāyoga is an eternal tradition that stands at the beginning of human history. He explains that this is why many of the scriptures, such as the \textit{Śivasūtrā}, the \textit{Āgama}s and the writings of the Siddhas, teach the techniques and principles of Kriyāyoga in many different ways. Moreover, remnants of this primal Kriyāyoga can be found in almost all philosophies, be it Buddhism, Jainism, Sāṅkhya, Vaiśeṣika, Nyāya, Mīmāṃsā or Vedānta. 

\citeauthor{kriyayogasarasvati1981} (1981: 699), the founder of the \textit{Bihar school of Yoga}, explains that there is no history of Kriyāyoga and that its origins and development have been lost. Furthermore, the system of Kriyāyoga was so secret that there is not even a myth to explain its origin. Furthermore, he describes that parts of the Kriyāyoga taught by him are contained in the texts of Haṭhayoga, such as \textit{āsana}s, \textit{mudrā}s and \textit{bandha}s, but that these are not ``integrated together''. Furthermore, he speculates that Kriyāyoga must have been known in China, as he sees strong parallels to practices in \textit{Tai Chi Chuan}. Furthermore, he clearly distances himself from the Kriyāyoga of the \textit{Yogasūtra}, which has nothing to do with the Kriyāyoga of his book \citetitle{kriyayogasarasvati1981} and serves solely as a preparation for Rājayoga. However, the only definitive historical statement he can commit himself to is the following: \begin{quote} Of history, all we will say is that kriya yoga was passed on by Swami Sivananda of Rishikesh. \end{quote} Surprisingly, this same \citeauthor{shivanandakriya1982} of Rishikesh in his book \citetitle{shivanandakriya1982} (1955) explicitly traces the Kriyāyoga he taught back to \textit{Yogasūtra} 2.1. \citeauthor{shivanandakriya1982} (1982:168-182) uses the Kriyāyoga of the \textit{Yogasūtra} as the overarching framework of his teaching, which also integrates \textit{ṣatkarma} and breathing exercises from Haṭhayoga into it.

It is important to emphasise that \citeauthor{kriyayogasarasvati1981} recognises that the traditional lineage of Yogānanda also practises the same Kriyāyoga he teaches. However, he explicitly distances himself from their narrative: \begin{quote} Of course, there are various other groups of people in India who have practiced and taught kriya yoga. For example, Swami Yogananda, Yukteshwar Giri, Lahiri Mahasaya, Mahatma Gandhi and so forth practiced kriya yoga. In fact, a thriving organization still propagates it throughout the world. They also do now know the origin of kriya yoga, but they say that it was reintroduced by the great yogi Babaji as the ideal practice for sincere seekers of wisdom in the present Kali Yuga (Dark Age). \end{quote}

This narrative is by far the most widespread explanation of the origins of the new Kriyāyoga and is adopted not only in the tradition of Yogānanda, but also in the tradition of Ramaiah. In his book \textit{Kriya Yoga and the 18 Siddhas} (2010: 31-64), \citeauthor{govindan2010}, a disciple of Ramaiah, has compiled this narrative in detail, which I would now like to summarise in a nutshell.

Mahāvātara Babajī, who according to \citeauthor{govindan2010} is considered an incarnation of the Buddha, was born in 203 CE in Parangipetta in Tamil Nadu under the name Najaraj into a Brahmin family, joined a group of wandering Saṃnyāsins at a young age and studied the holy scriptures. His path soon led him to Śrī Laṅka in Katirkāma (now Kataragama), where he became a disciple of Siddha Boganathar and was initiated by him into various \textit{kriyā}s such as \textit{dhyāna}, \textit{āsana}, \textit{mantra} and \textit{bhaktiyoga}. Bhoganathar later sent Babajī to another teacher, namely Siddha Agastya in Courtallam in the Pothihai hills of Tamil Nadu, located in today's Tinneveley district. He learnt the particularly important \textit{kriyā} called \textit{kuṇḍalinīprāṇāyāma} from him. Agastya then sent Babajī to Badrinath in the Himalayas, where he practised for many months and finally attained \textit{samādhi}. After his enlightenment and attaining immortality at just 16, Babajī set himself the task of helping suffering humanity in its search for God-realisation. As an immortal, Babajī initiated great personalities such as Śaṅkarācārya (788-820) and Kabīr (1440-1518) into the techniques of Kriyāyoga over the centuries. Finally, in 1861, he initiated Lahiḍi Mahāśaya (1828-1895) into Kriyāyoga and gave him the task of passing it on to serious seekers. At this point, \citeauthor{govindan2010} quotes the autobiography of Yogānanda,\footnote{Cf. \citeauthor{autobioyogi}, 1949: 244 f.} which states that Babajī explained to Lahiḍi Mahāśaya that Kṛṣṇa had once passed on Kriyāyoga to Arjuna and that not only Patañjali knew it, but also Jesus Christ, who in turn had passed it on to John, Paul and other disciples. Among Lahiḍi Mahāśaya's 100 disciples was Śrī Yukteśvar (1855-1936), to whom Babajī is also said to have appeared three times. On one of these occasions, Babajī decided that he should send his disciple Yogānanda (1893-1952) to America to spread Kriyāyoga, which he did, gaining global fame and founding the \textit{Self Realisation Fellowship} in 1920, which is still very active today.   
\subsubsection{The practice of the new Kriyāyoga}

In the following, the practices of the new Kriyāyoga are presented in outline based on the publications mentioned and consulted above.\footnote{A comprehensive presentation and comparative analysis of the practices in the various traditions of the new Kriyāyoga would be too far-reaching for this chapter. The most detailed written practice instructions that I have consulted can be found for the Śivānanda/Satyānanda model in \citeauthor{kriyayogasarasvati1981}, (1981: 697-952) and for the Yogānanda model in \citeauthor{kriyayoganityananda2013}, (2013: 249-340).} The words of \citeauthor{hariharananda1989} (1989: 144) are surprisingly apt to give an essential first impression of this complex phenomenon: \begin{quote} Kriya Yoga is the essence and synthesis of all yoga techniques taught in the world.  \end{quote} 
\citeauthor{kriyayogasarasvati1981} (1981:703) explains that each Kriyā consists of a certain number of subordinate techniques. These always consist of a combination of the following six tools: \textit{āsana}, \textit{mudrā}, \textit{bandha}, \textit{mantra}, \textit{prāṇāyāma} and, as he calls it, `psychic passage awareness'. This last point includes a group of exercises mainly involving ``circulating awareness through the \textit{cakra}s in an ascending and descending way'' or similar. A single Kriyā is an exercise unit comprising individual exercises from the six categories mentioned. However, these are not arbitrary but are integrated in a specific, scientific way in order to induce the process of concentration (\textit{dhāraṇa}), meditation (\textit{dhyāna}) and meditative absorption (\textit{samādhi}). The main distinguishing feature from other yoga systems is the innovative and specific combination of the individual techniques into a practical and particularly effective sequence of exercises, referred to here as ``Kriyā''.

In every model the individual exercises are drawn from the vast body of Yoga literature but primarily from the exercises taught in the medieval to pre-colonial texts of the Haṭha- and Rājayoga genres. This always takes place against the background of tantric and medieval concepts of the yogic body, such as \textit{cakra}, \textit{nāḍī} and \textit{vāyu} systems. A common phenomenon in the new Kriyāyoga literature is scientific explanatory models that are used as a means of legitimisation. For example, certain \textit{nāḍī}s are located in schematic sketches of the brain\footnote{\citeauthor{kriyayoganityananda2013}, 2013: 215.}, or positive effects of Kriyāyoga practice are legitimised with evolutionary biology theories, such as the polyvagal theory\footnote{\citeauthor{kriyayogalowenstein}, 2021: 188.}

\citeauthor{govindan2010} (2010: 216-225) distinguishes a total of seven main categories of Kriyāyoga. The first category he mentions is \textit{Kriya Hatha Yoga}. According to him, this is the starting point for every student of Kriya Yoga. This includes eighteen basic relaxation postures (\textit{āsana}s), muscle blocks (\textit{bandha}s), certain gestures (\textit{mudrā}s) and the sun salutation (\textit{sūryanamaskāra}) defined by Babajī.

The second main category is what \citeauthor{govindan2010} calls \textit{Kriya Kundalini Pranayama}. According to him, this practice is the art and science of mastering the breath and is considered to be the most essential and effective tool in Babajī's Kriyāyoga. This is not only meant to awaken the \textit{kuṇḍaliṇī} but with regular practice, the student awakens all \textit{cakra}s and the associated levels of consciousness, which is supposed to ultimately lead to the breathless state of \textit{samādhi} and self-realisation.

The third main category is \textit{Kriya Dhyana Yoga}, which is intended to include meditation techniques that are not explained in detail but are supposed to awaken the mind's hidden faculties.

The fourth main category is \textit{Kriya Mantra Yoga}. This involves the recitation or murmuring (\textit{japa}) of mantras discovered by the Siddhas. The recitation of mantras must take place with faith, love and concentration.

\citeauthor{govindan2010} calls the fifth category \textit{Kriya Bhakti Yoga}, the yoga of love and devotion. In \citeauthor{govindan2010}'s words, this is the ``turbojet'' of self-realisation. This type of Kriyāyoga includes devotionallove, chanting, ritual worship and pilgrimages to holy places.

Furthermore, \textit{Kriya Karma Yoga} is named as the sixth category. In this case he refers to \citetitle{kaushik1993} II.47 f. and thus defines this subtype as selfless service that is performed consciously. All actions are supposed to be performed without the expectation of receiving anything in return, free from anger, selfishness, greed and personal desires. Thus, the practitioner is meant to examine his motivation before every action and is always supposed to act without selfish motives.

The seventh and final category is \textit{Kriya Tantra Yoga}. According to this, the followers of Kriyāyoga, just like the Siddhas, lead a family life. This subtype of Kriyāyoga involves retaining the energy normally wasted during sexual activity and transporting it to the higher \textit{cakra}s. The partner is supposed to be loved as an embodiment of the divine.

A similar system is taught in \citeauthor{kriyayogalowenstein} (2021). This initially includes a total of twelve \textit{āsana}s and the five Tibetans, as well as typical \textit{prāṇāyāma} techniques, \textit{ujjāyi}, \textit{kapalabhāti}, various \textit{bandha} techniques such as \textit{uḍḍīyānabandha} or \textit{mahābandha}, various \textit{mūdrā} techniques such as \textit{mahāmudrā}, \textit{śāmbhavīmudrā}, \textit{yonimudrā}, or the so-called \textit{Kriya Breath}. \textit{Kriya Breath} is referred to as \textit{kevalakumbhaka}. In addition, classical gymnastic exercises are also added\footnote{\citeauthor{kriyayogalowenstein}, 2021: 118-124. Gymnastic exercises can also be found in \citeauthor{kriyayogasturgess2015}, 2015: 447-458.} In addition to the \textit{āsana}s of Haṭhayoga, \citeauthor{kriyayogalowenstein} also recommend \textit{Tai Chi}, \textit{Qigong}, physiotherapy or a personal trainer to stay fit. Now and then, a biblical quotation is used. For example, in the case of the \textit{Third Eye Gazing} practice, he quotes Matthew 6:22. Furthermore, \citeauthor{kriyayogalowenstein} emphasise the practice of \textit{Hong Sau} as an important element of the practice. For \citeauthor{kriyayoganityananda2013}, \textit{Hong Sau}, or in this case the indologically correct transliteration \textit{haṃsa}, is also referred to by him as \textit{Haṃsa Sādhanā},\footnote{The \textit{ajapājapa}, recitation of the non-recitation of the \textit{haṃsa} mantra.} ``the very foundation'' of Kriyāyoga.\\

As indicated at the beginning of this section, it is clear that the term Kriyāyoga has given rise to a kind of proliferation of different yoga techniques from earlier yoga traditions, which are integrated into innovative exercise systems and attempted to be historically legitimised in different ways. Depending on the lineage and the teacher, individual characteristics and different explanatory models exist.\footnote{In these books, one repeatedly comes across pseudo-scientific explanatory models and stumbles across parallels drawn here and there to other religions, such as Christianity and Buddhism, to emphasise the effectiveness and importance of certain practices and views. Particularly in the more recent publications, it can be seen that, depending on the author, typically individual expressions of the ideal type of postmodern spirituality and religiosity are expressed, which \citeauthor{bochinger2009} have labelled the ``spiritueller Wanderer'' (\citeauthor{bochinger2009} 2009: 33-49).}\\

One last exemplary publication is \citetitle{kriyayogaupanishad1993} (1993) by \citeauthor{kriyayogaupanishad1993}. This book offers translations of ten well-known \textit{Yoga Upaniṣads} and one \textit{Kriya Yoga Upanishad}. The translator claims that the name of the author of this Sanskrit Yoga Upaniṣad was lost in the course of history. His book has no bibliography, nor are the sources of the translations mentioned. Further searches for a verifiable source text of the \textit{Kriya Yoga Upanishad} remain unsuccessful. The \textit{Kriya Yoga Upanishad} is neither to be found in the known publications and translations of the \textit{Yoga Upaniṣads},\footnote{Cf. \citetitle{yogaupaniṣaded} (1938),} nor in publications of previously unpublished Upaniṣads.\footnote{Cf. \citetitle{upanishads1938} (1938).}. Searching through various catalogues of Sanskrit manuscripts was also unsuccessful.\footnote{In \citetitle{kaivalyadamanuscripts2005} (2005: 50), two manuscripts with the title \textit{Kriyāyoga} (AGJ 665/1 and TSM 6716) are listed, which, unfortunately, I was unable to consult. Neither manuscript is dated. AGJ 665/1 is a Devanāgarī manuscript on paper, and TSM 6716 is a Telugu manuscript on palm leaf. The author of the latter is named Venkaṭayogin. I suspect these manuscripts are probably later works that were created in the 18th century at the earliest. For now, however, no definitive statement can be made on this. However, their consultation could shed further light on the historical development of Kriyāyoga.} It is also striking that the \textit{Kriya Yoga Upanishad} is not mentioned in any other publications on Kriyāyoga consulted. For the time being, therefore, the possibility must be considered that \citeauthor{kriyayogaupanishad1993} is not only the translator of the \textit{Kriya Yoga Upanishad} but also the secret author. Perhaps he wrote this supposedly ancient source text in order to legitimise his own Kriyāyoga doctrine.   

Goswami \citeauthor{kriyayogaupanishad1993} learnt Kriyāyoga from his teacher Shelly Trimmer, who, according to the official website of the \textit{Temple of Kriya Yoga}\footnote{\cite{goswamikriyananda}.} founded by \citeauthor{kriyayogaupanishad1993}, was a guru, yogi, kabbalist and direct disciple of Yogānanda. \citeauthor{kriyayogaupanishad1993} studied philosophy for four years at the University of Illinois and then embarked on a business career. Whether \citeauthor{kriyayogaupanishad1993} would have acquired the qualifications to translate a Sanskrit source text remains to be seen. Possibly, he was a gifted autodidact.

In the \textit{Kriya Yoga Upanishad}, the disciple Sanskriti asks the guru Dattatreya to teach him the doctrine of Kriyāyoga. The latter agrees and explains Kriyāyoga in a total of ten chapters. The framework is formed by the eight-limbed Yoga system presented in 1.5, similar to the eight limbs of the Pātañjala scheme. The first chapter (1.6-25) presents the \textit{Ten Spiritual Restraints}. Dattatreya explains the \textit{Ten Spiritual Observances} in the second chapter (2.1-16). Chapter three, \textit{The Nine Postures} (3.1-13), deals with nine \textit{āsana}s with six sitting postures, one standing posture and one complex posture. The fourth chapter (4.1-63) discusses what \citeauthor{kriyayogaupanishad1993} calls \textit{Mystical Anatomy}. Here, six \textit{cakra}s named after the planets (i.e. the \textit{mūlādhāracakra} is called the ``Saturn mass-energy converter \textit{cakra}''), fourteen primary \textit{nāḍī}s and \textit{Kriya Kundalini}, which covers the `divine creative channel' with its mouth, are taught. The fifth chapter (5.1-14) is entitled \textit{Inner Purification} and contains simple \textit{prāṇāyāma} techniques such as \textit{sūryabhedana} and \textit{candrabhedana}. Chapter six (6.1-39), entitled \textit{Breath Control}, instructs another breathing exercise in combination with meditation on the three \textit{akṣara}s that constitute the sacred syllable \textit{auṃ}. During the inhalation (\textit{pūraka}), the yogi is supposed to meditate on \textit{a}, during the breathing posture on \textit{u} and during the exhalation on \textit{ṃ}. In addition, the breathing technique \textit{śītalī} (6.25) and a technique called \textit{yonimudrā} (6.33-34) are presented. Chapter seven (7.1-10) is about \textit{Withdrawal of the Senses}. The practitioner is instructed to let the breath move through the body in a specific order. The eighth chapter (8.1-9) is entitled \textit{Concentration}. Here, the yogin is meant to inhale and hold the breath at specific bodily locations (not the \textit{cakra}s), which are associated with the five elements and the syllables \textit{ya, ra, va, la} and \textit {ha}, as well as specific deities. The even shorter ninth chapter, \textit{Meditation} (9.1-6), basically only states that the practice of concentration leads to meditation after a while. The tenth chapter, \textit{Samadhi} (10.1-12), then describes the final state of Yoga, which is defined as the ``deep conscious trance in which the yogi experiences Absolute Wisdom''.

\subsubsection{Hypothesis on the transition from the late medieval models to the modern models of Kriyāyoga}

The \textit{Yogasvarodaya} and Rāmacandra's \textit{Yogatattvabindu} were written before 1659 CE. Nārāyaṇatīrtha must have lived between 1600 and 1690 CE., and because of that, his \textit{Yogasiddhāntacandrikā} was also written in this timeframe. Sant Sundardās, the author of the \textit{Sarvāṅgayogapradīpikā} lived from 1596 to 1689. Interestingly, Nārāyaṇatīrtha and Sundardās lived in Benares.\footnote{See \citeauthor{burger2014sarvangayogapradipika} (2014: 684) for dating and location of Sundardās and \citeauthor{penna2004} (2004: 24) for dating and location of Nārāyaṇatīrtha.} Thus, we can safely assume that the complex taxonomies of twelve-fifteen Yogas were part of the local discourse of 17th-century Benares. One might speculate that Rāmacandra might also have lived in these surroundings, but this remains uncertain. Lahiḍi Mahāśaya, the person to whom the new forms of Kriyāyoga seem to go back, lived about a century later, from 1828 to 1895 CE. Interestingly, Lahiḍi Mahāśaya is also said to have spent much of his life in Benares. It is, of course, utterly unclear whether Lahiḍi Mahāśaya ever read any of the works mentioned above. At least we know that he not only enjoyed an education in philosophy in Benares but also learnt English and Sanskrit.\footnote{\citeauthor{jones2008encyclopedia}, 2008: 255-56.} However, it is likely that the local discourse regarding the religious-spiritual offerings within Benares did not change abruptly. Lahiḍi Mahāśaya also lived as a family man and householder,\footnote{See \citeauthor{autobioyogi}, 1946: ???.} no sectarian affiliations are known so that the whole variety of religious-spiritual offerings of his time were open to him. He was able to combine them freely. As can be seen from the Yoga texts examined in this book, there was no lack of different Yoga categories in Benares between the 17th and 19th centuries CE. Although these were still labelled differently, they were without a doubt freely combined in practice. Moreover, given the plethora of Yoga practices from different Yoga traditions and Yoga texts presented in the previous chapter and evident in the publications of the new Kriyāyoga consulted, it is not only credible but also plausible that this phenomenon already began with Lahiḍi Mahāśaya, as Yogānanda claims in his autobiography. However, why Lahiḍi Mahāśaya chose the category of Kriyāyoga as the generic term for his Yoga system cannot be answered conclusively. However, I would like to offer an educated guess.

I hypothesize that the term Kriyāyoga, as the generic term for his system of Yoga, was a strategic decision of Lahiḍi Mahāśaya. It is unlikely, and there is no clear evidence that Lahiḍi Mahāśaya knew the \textit{Yogasvarodaya}, \textit{Yogatattvabindu} and \textit{Yogasiddhāntacandrikā}. It is impossible to determine if there ever was any influence of these texts on Lahiḍi Mahāśaya and his new Kriyāyoga system. But if there was, only the fact that all three texts that mention Kriyāyoga as the very first item in their taxonomies could have influenced his decision to unite all possible Yogas and their techniques under the term Kriyāyoga. Another factor could have been that he was consciously or unconsciously driven by the emerging Yogasūtra hype in the West, which triggered a wave of enthusiasm in India. One wonders why he did not choose the term Rājayoga to integrate many systems as others have done before him. Maybe because the term Rājayoga was already used as a generic term for Pātañjalayoga by then.\footnote{See \citeauthor{birch2014}.} Perhaps, the term Kriyāyoga had the advantage that it not only formed a link to the popular and hyped \textit{Yogasūtra}, but also provided a basic framework that was open to interpretation due to the three constitutional practices \textit{tapas}, \textit{svādhyāya} and \textit{īśvarapraṇidhāna}. Thus, the term opened up the possibility to integrate the variety of post-Pātañjalean physical and non-physical Yoga practices from the Tantras and texts of Haṭha- and Rājayoga through a literal interpretation of the compound prefix \textit{kriyā°} in the sense of ``action''. Whether his thoughts went in a similar direction must remain open. However, we must assume that the discursive environment of Benares at his time certainly played its part in encouraging Lahiḍi Mahāśaya to integrate the various Yogas circulating in the local discourse of his time under this specific term.

\section{2. Jñānayoga}
\label{jnanayogaintro}

Jñānāyoga\footnote{see section \uproman{21} and \uproman{22} on p.\pageref{jnanayogastart}-\pageref{endsvabhava}} is the second yoga in Rāmacandra's list of the fifteen yogas as well as in his source text, the \textit{Yogasvarodaya}. In Nārāyaṇatīrtha's list of the fifteen yogas in the \textit{Yogasiddhāntacandrikā}, Jñānayoga takes sixth place. Sundardās positions Jñānayoga in tenth place in his list of twelve yogas in his \textit{Sarvāṅgayogapradīpikā}. Here, it is subsumed within his fourth tetrad of Yogas together with Brahmayoga and Advaitayoga under the main category Sāṅkhyayoga.  

\subsection{Jñānayoga in the \textit{Yogatattvabindu}}
\label{Jnanayogaintro2}
Jñānayoga occupies the second place in Rāmacandra's taxonomy of the fifteen Yogas but is not described as the second yoga in his text.\footnote{The description of Jñānayoga is preceded by Siddhakuṇḍalinīyoga and Mantrayoga (\uproman{3}-\uproman{12}), Lakṣyayoga (\uproman{13}-\uproman{15}), Rājayoga (\uproman{16}-\uproman{17}), Caryāyoga (\uproman{18}) and Haṭhayoga (\uproman{19}-\uproman{20}). See chapter ????? on structural problems of \textit{Yogatattvabindu} on p.\pageref{structuralissues}.} The description is given from section \uproman{21}-\uproman{22}. The overarching goal of Rāmacandra's Jñānayoga is the long-term durability of the body (\textit{bahutarakālaṃ śarīrasthitiḥ}) already mentioned in the introduction (section \uproman{1}), which is expressed here once again with other words: `From the execution of this [Jñānayoga], time does not bring about the destruction of the body' (\textit{tasya kāraṇāt kālaḥ śarīranāśaṃ na karoti}). Simultaneously, Rāmacandra's Jñānayoga leads to the attainment of the `reality of Śambhu' (\textit{śāṃbhavīsattā}).\footnote{This refers to the highest reality and the state of Rājayoga. See p.\pageref{jnanayogatrans1} in the edition for a discussion of the term.} This Jñānayoga can be practised in two ways. The first method (\uproman{21}.1) arises through the application of 'non-dualistic thinking' (\textit{avikalpatayā yuktyā}), and the second method (\uproman{21}.2) arises through the realisation that the entire world consists of all knowledge (\ldots \textit{sarvajñānamayaṃ jagat} | \textit{ya evaṃ vetti bodhena} \ldots). However, the text primarily deals with the first method. This method consists of viewing the world as a unity that is enlightened by the highest self (\textit{viśvātman}). If one perceives this unity, one finds oneself in the 'reality of Śaṃbhu'. However, this supreme reality cannot be recognised without further ado since it does not show itself as the desired unity but as a tenfold multiplicity (\uproman{21}.4ab). He compares this relationship to a seed from which a whole tree with its parts grows (\uproman{21}.4-\uproman{21}.5). The seed stands for the invisible unity of world and self. The tree, with its various parts, stands for the multiplicity of the visible world. The fundamental unity of the world is like the seed from which a whole tree has grown. It is no longer visible and is not perceived. However, what is perceived is a world consisting of a multiplicity. In the case of the seed, a tree with its branches, leaves, etc. In the case of the world ten basic principles (\textit{tattva}s): Five [gross] elements (\textit{pañcatattva}), thinking mind (\textit{manas}), intellect (\textit{buddhi}), illusion (\textit{māya}), individuation (\textit{ahaṃkāra}), and modifications (\textit{vikriyā}). \footnote{For a discussion of the tenfold \textit{tattva} system, see S.\pageref{??} n.??? and S.\pageref{??} n. ??}. Jñānayoga is supposed to produce the realisation of oneness (\uproman{21}.7). In order to realise this, the practitioner is supposed to apply the view of unity (\textit{aikyena darśanam}) to recognise the identity between the visible world of multiplicity\footnote{This is also referred to by Rāmacandra as \textit{saṃsāra} (\uproman{21} ll. 7-9).}, and the invisible self (\textit{viśvātma}). Through Jñānayoga, the practitioner then realises that the self is one with the world\footnote{Cf. \textit{Yogatattvabindu} \uproman{22} \pageref{svabhava1} l. 5: `Because of the power of Jñānayoga, there arises the conviction that the self is truly one (\textit{jñānayogaprabhāvād eka eva ātmā iti niścayo bhavati})} and the changing forms of the worlds material appearance are empty.\footnote{Cf. \textit{Yogatattvabindu} \uproman{22} p.\pageref{svabhava2} l.3: `Through Jñānayoga he realises the emptiness of the mutability of form.' (\textit{jñānayogād vikārarūparahito jñāyate} |)}

\subsection{Jñānayoga in the \textit{Yogasvarodaya}}
\label{svarodayajnana}
If we assume a correct transmission of the \textit{Yogasvarodaya} in the \textit{Prāṇatoṣiṇī}, then the text, in fact, describes two different types of Jñānayoga. 

The Jñānayoga of the first passage\footnote{Cf. \textit{Prāṇatoṣiṇī}, Ed. p. 831-833.} contains a description of the major components of the yogic body which the Yogi is supposed to know. Gaining knowledge about the body is the aim of this Jñānayoga.\footnote{Cf. \textit{Prāṇatoṣiṇī} Ed. p. 831 (\textit{jñānayogam pravakṣyāmi tajjñānī śivatāṃ vrajet} | \textit{paṭhanāt smaraṇād vyānān maṇḍanāt brahmasādhakaḥ}) | \textit{tadbhedasyaikasandhānam aṣṭaiśvaryamayo bhavet} | \textit{tritīrthaṃ yatra nāḍī ca tripuṇyaṃ parameśvari} | \textit{svadehe yo na jānāti sa yogī nāmadhārakaḥ} | \textit{navacakraṃ kalādhāraṃ trilakṣaṃ vyomapañcakam} | \textit{svadehe yo na jānāti sa yogī nāmadhārakaḥ}).} In particular, the three primary channels (\textit{nāḍī}s)\footnote{The left lunar channel (\textit{iḍā}), the right solar channel (\textit{piṅgalā}) and the central channel (\textit{suṣūmnā}).}, as well as a system with a total of nine \textit{cakra}s are supposed to be known. They are described in detail. The introduction to this first form of Jñānayoga mentions other things the Yogi should know, such as the three targets [for fixing the mind] (\textit{lakṣya}s),\footnote{In the sections on Lakṣyayoga in the \textit{Yogasvarodaya} and \textit{Yogatattvabindu} five targets (\textit{lakṣya}s) are described in total. This is one of many inconsistencies in the \textit{Yogasvarodaya} and the \textit{Yogattvabindu}.} sixteen containers [for holding mind and often breath in the context of this type of yogic practice] (\textit{ādhāra}s) and the five [meditative] spaces (\textit{vyoman}s) through which the yogin progresses on the path to the highest state of Yoga. However, these do not specifically belong to yogic physiology like the channels and \textit{nāḍī}s. Thus, they are not treated in the first Jñānayoga section but dealt with separately during the text.

This first form of Jñānayoga in the \textit{Yogasvarodaya}, like much of its content and even its sequence, is adopted by Rāmacandra in his \textit{Yogatattvabindu}. Surprisingly, he adopts the first form of Jñānayoga under a different name.\footnote{Perhaps, the designation \textit{jñānayoga} in this context is a result of textual corruption, as the second Jñānayoga presented later on in the text lives up to its name much better. However, without further textual evidence, this remains unproven.} Instead of Jñānayoga, Rāmacandra calls it Siddhakuṇḍaliniyoga and Mantrayoga. It is unclear why Rāmacandra made this change. Perhaps Rāmacandra did not want to teach two different forms of Jñānayoga, or he was convinced that Siddhakuṇḍaliniyoga and Mantrayoga were the more appropriate terms for this type of Yoga. Another possibility would be that the quotations of the \textit{Yogasvarodaya} in the \textit{Prāṇatoṣiṇī} are corrupted. However, this is the less likely scenario. A detailed discussion of Siddhakuṇḍalinīyoga and Mantrayoga in Rāmacandra's \textit{Yogatattvabindu} can be found on p.\pageref{siddhayogaintro}.

The Jñānayoga of the second passage\footnote{\textit{Prāṇatoṣiṇī}, Ed. p. 835-837.} is almost identical with Rāmacandra's Jñānayoga. Rāmacandra adopts most of the verses verbatim from the \textit{Yogasvarodaya}. There are minor details that Rāmcandra modifies, but they do not significantly change the concept and aim of Jñānayoga. A slight simplification of the presentation can be observed. 

\subsection{Jñānayoga in the \textit{Yogasiddhāntacandrikā}}
\label{jnanayogaintrocandrika}
Nārāyaṇatīrtha situates his Jñānayoga \footnote{For an earlier brief discussion of Jñānayoga in Nārāyaṇatīrtha's \textit{yogacandrika} see \citeauthor{penna2004}, 2004: 76.} in the context of \citetitle{yogasutra}'s \textit{sūtra} 1.28, which says:
\begin{quote} \textit{taj japas tadarthabhāvanam} || 28 || \end{quote}
\begin{quote} It's low-voice muttering; contemplation of its meaning. \end{quote}

This is the last \textit{sūtra} of an extensive section (1.23 - 1.28) in the \citetitle{yogasutra}\footnote{An entire monograph entitled \citetitle{harimoto2014} is dedicated to this section by \citeauthor{harimoto2014} (2014). It provides an edition, translation and detailed discussion of this critical passage in the \textit{Pātañjalayogaśāstravivaraṇa}.}, which is entirely dedicated to one of the means of attaining \textit{samādhi}, namely \textit{īśvarapraṇidhāna}, devotion to Īśvara, the Supreme Lord.

Īśvara is most aptly represented by the sacred syllable \textit{auṃ}. The above \textit{sūtra} instructs the quiet murmuring of this syllable while contemplating its meaning (\textit{tadarthabhāvanam}) as a practical method of \textit{īśvarapraṇidhāna} to attain the highest state of Yoga, which is called \textit{asaṃprajñātasamādhi}.

In this context, Nārāyaṇatīrtha explains that in this \textit{sūtra}, the term `low-voice muttering' (\textit{japa}) refers to the practice of Mantrayoga. The term 'contemplating its meaning' (\textit{arthabhavana}) refers to Jñānayoga as a form of practice that cultivates discriminating knowledge (see previous paragraph). Furthermore, in this context, Nārāyaṇatīrtha refers to Advaitayoga, also associated with this \textit{sūtra}, which is a form of Yoga characterised by the view of the non-differentiation of the individual self and the supreme self.\footnote{Cf. \textit{Yogasiddhāntacandrikā} Ed. p. 46: 'Furthermore, by the term `\textit{japa}', the practice of Mantrayoga is indicated; by '\textit{arthabhavana},' the knowledge of discrimination, the form of practice [called] Jñānayoga, and Advaitayoga is the form of cultivating non-differentiation. (\textit{kiñca japa ity anena mantrayogaḥ arthabhāvanam ity anena vivekajñānā 'bhyāsarūpo jñānayogaḥ abhedabhāvarūpo 'dvaitayogaś ca saṃgṛhītaḥ} |).}

Nārāyaṇatīrtha, thus, offers two alternatives about the specific performance of the contemplation. Either, while quietly murmuring the \textit{praṇava} syllable, which symbolises Īśvara and his qualities, attention is supposed to be focused on the distinction between consciousness (\textit{puruṣa}) and primordial nature (\textit{prakṛti}) including its effects (\textit{tatkārya}).\footnote{Cf. \textit{Yogasiddhāntacandrikā} Ed. p. 45: `The low-voice muttering of \textit{praṇava} [and] pronunciation according to the rules [along with] the contemplation of the meaning of that \textit{praṇava}, [being associated with] the Supreme Self endowed with inconceivable power and supremacy, is the fixation of the attention with discernment from the individual self and nature with its effects.' (\textit{tasya praṇavasya japaḥ vidhivad uccāraṇaṃ, tadarthasya praṇavārthasya acintyaiśvaryaśaktiyuktasya paramātmano bhāvanaṃ prakṛtitatkāryapuruṣebhyo vivekenānusaṃdhānam}).} This is Nārāyaṇatīrtha's Jñānayoga. Alternatively, one is supposed to reflect on the non-difference between the highest self (\textit{paramātman}) and the individual self (\textit{jīva}).\footnote{Ibid. (Ed. p. 45): `Alternatively, its meaning is the repeated memorization in the mind of the non-distinction between the individual self and the total supreme self.' (\textit{athavā tadarthasya paramātmanaḥ pūrṇasya bhāvanaṃ jīvābhedena punaḥ punaś cetasi niveśanam |}).} This is Nārāyaṇatīrtha's Advaitayoga.

\subsection{Jñānayoga in the \textit{Sarvāṅgayogapradīpikā}}

The Jñānayoga of Sundardās (SYP 4.13-24) is similar to the Jñānayoga of Rāmacandra and the \textit{Yogasvarodaya}. Although Sundardās does not mention a \textit{tattva} system, the reality of Śambhu or the physical effects of this yoga is also about recognising that the universe and the world form a unity.\footnote{See \citeauthor{burger2014sarvangayogapradipika} (2014: 702) for an earlier brief discussion of Sundardās's Jñānayoga in French.} According to Sundardās, the self is the cause, and the whole universe is the effect.\footnote{\citetitle{sarvangayoga} 4.13: `Now understand Jñānayoga. Recognize the cause and effect. The cause is the indivisible soul. The effect is the whole universe.' (\textit{jñāna yoga aba esaiṃ jānaiṃ} | \textit{kāraṇa aru kāraya pahicānaiṃ} | \textit{kāraṇa ātama āhi akhāṃḍā} | \textit{kāraya bhayau sakala brahmaṇḍā} || 13 ||)} To illustrate the relationship of cause and effect between self and universe, Sundardās presents the same metaphor of the seed and the tree as Rāmacandra in \uproman{21}.4-5.\footnote{\citetitle{sarvangayoga} 4.14: `Just as the tree [grows] out of the seed, bringing forth countless branches, leaves, fruits and flowers, in the same way the self is the root of the universe.' (\textit{jyauṃ aṃkuru teṃ taru vistārā} | \textit{bahuta bhāṃti kari nikasī ḍārā} | \textit{śāṣā patra aura pharaphulā} | \textit{yauṃ ātamā viśva kau mūlā} || 14 ||)} The rest of the section consists of different comparisons, which are supposed to illustrate the non-difference between the self and the whole or the universe.\footnote{For example \citetitle{sarvangayoga} 4.20: `Just like various ornaments made of gold, are worn with different names and forms. However, in essence, all become one in the melting pot. In the same way, the self is not separate from the universe.' (\textit{jyauṃ kuñcana ke bhūṣana nānā} | \textit{bhinna bhinna kari nāṃva baṣaṇā} | \textit{gāre sarba eka hi huvā} | \textit{yaiṃ ātamā biśva nahiṃ juvā} || 20 ||)}

%Notes:

%Chapter 15 - Trikāṇḍa-Yoga: Bhakti Surpasses
%Knowledge and Detachment
%(1) Śrī Uddhava said: 'The Vedic literature of Your Lordship, oh Lotus-eyed One,
%that pays attention to the injunctions concerning actions and prohibitions, deals with
%the good and bad sides of karma [akarma and vikarma]. (2) They also discuss the dif-
%ferences within the varṇāśrama system wherein the father may be of a higher [anulo-
%ma] or a lower [pratiloma] class than the mother, they are about heaven and hell and
%expound on the subjects of having possessions, one's age, place and time [see also 4.8:
%54 and *]. (3) How can human beings without Your prohibitive and regulatory words
%concerning final beatitude, tell the difference between virtue and vice [compare 11.19:
%40-45]? (4) The Vedic knowledge emanating from You offers the forefathers, the gods as
%72Uddhava Gītā
%also the human beings a superior eye upon the - not for everyone that evident - meaning
%of life, what would be the goal, and how we may achieve. (5) The difference between
%virtue and vice one can see with the help of Your Vedic knowledge and that insight does
%not arise by itself, but the Vedas also nullify such a difference and thus clearly confuse
%the issue....'
%(6) The Supreme Lord said: 'The three ways of yoga I described in My desire to
%grant human beings the perfection, are the path of philosophy [jñāna], the path of work
%[karma] and the path of devotion [bhakti]; no other means can be found [for one's
%emancipation. See also B.G. contents and trikāṇḍa].


\section{3. Caryāyoga}
\label{caryayogaintro}

Caryāyoga occupies third place in Rāmcandra's list, is absent in the \textit{Yogasvarodaya}, in second place in Nārāyaṇatīrtha, and also absent in Sundardās. However, Sundardās describes a Yoga with the almost homophonic name Carcāyoga. Carcāyoga is subsumed under the main category of Bhaktiyoga together with Mantrayoga und Layayoga. Due to the phonological similarity with Caryāyoga, the concept Carcāyoga will be compared with the concepts of Caryāyoga to determine whether there is a connection between the two concepts or not. 

\subsection{Caryāyoga in the \textit{Yogatattvabindu}}

Rāmacandra keeps the section on Caryāyoga (section \uproman{18}) extremely short, with only eight prose sentences. After characterising the self as 'formless, permanent, immovable and indivisible', Rāmacandra lets the reader know that by stabilising the mind in such a self, the self does not come into contact with sin and merit. When the mind is absorbed into the formless [self], this is Cāryayoga. This is all which Rāmacandra says on this subject. The brevity of the passage and the fact that Rāmacandra's source text, the \textit{Yogasvarodaya}, does not contain this type of Yoga, but Rāmacandra clearly constructs it on the basis of a description of Rājayoga of the \textit{Yogasvarodaya},\footnote{Cf. \textit{Yogatattvabindu} \uproman{18}, p. \pageref{caryayoga}} suggests that Rāmacandra merely wanted to do justice to his list mentioned at the beginning.\footnote{One could argue that Rāmacandra may not have done so, since not all fifteen Yogas announced at the beginning are described in the course of his text anyway. I suspect that this may nevertheless have been his original intention but that Rāmacandra discarded this intention while writing his text, perhaps due to inconsistencies in his source text} It is puzzling why this particular Yoga with this particular description bears the name Caryāyoga. The apparent association of the first four Yogas in Rāmacandra's and \textit{Yogasvarodaya}'s list with the four \textit{pāda}s of the Śaiva Āgamas (\textit{kriyā}-, \textit{jñāna}-, \textit{caryā}- and \textit{yogapāda}) does not offer a convincing solution in this case, as \textit{caryā°} in this context has nothing to do with the original ritual discipline of the śaivite practices, as would be the case in the \textit{caryāpada} of a Śaiva Āgamas. It seems, therefore, unlikely that any Yoga practitioners back then practised a Caryāyoga that corresponds to the brief description of Rāmacandra. 

\subsection{Caryāyoga in the \textit{Yogasvarodaya}}

The term Caryāyoga does not exist in the \textit{Yogasvarodaya} sources available to me, namely \textit{Prāṇatoṣinī} and \textit{Yogakarṇikā}. The term Caryāyoga does not appear in the taxonomy of Yoga categories in the \textit{Yogasvarodaya}\footnote{\textit{Prāṇatoṣiṇī} ed. p. 831.} Although the verses concerning the yogic taxonomy postulate a total of fifteen Yogas, only eight are mentioned. Whether Caryāyoga remained unmentioned is unclear, but its presence in the taxonomies of the \textit{Yogatattvabindu}\footnote{\textit{Yogatattvabindu} I. ll. 1-4.} and the \textit{Yogasiddhāntacandrikā}\footnote{\textit{Yogasiddhāntacandrikā} Ed. p. 2.} certainly makes its affiliation seem possible and quite probable. What the \textit{Yogasvarodaya} originally looked like can only be surmised. Although it almost seems as if the entire \textit{Yogasvarodaya} has been handed down in the \textit{Prāṇatoṣiṇī}, the \textit{Yogakarṇikā} contains several verses of the \textit{Yogasvarodaya} which have not been handed down in the \textit{Prāṇatoṣinī}. \footnote{It is striking that Rāmacandra's prosaisation is based almost exclusively on the verses quoted by the \textit{Prāṇatoṣiṇī}.} For this reason, the text may have been more extensive and could have transmitted a Caryāyoga. The \textit{Yogasvarodaya} was likely the first text to mention the taxonomy of the fifteen yogas.\footnote{See p.\pageref{???} for a genereal description of the \textit{Yogasvarodaya}.} If the association with the four \textit{pāda}s of the Śaiva Āgamas had been made by the author of the \textit{Yogasvarodaya}, then a yoga related to the ritual discipline of śaivite practices would be the most obvious suggestion of what such a Caryāyoga might have looked like. Indeed, in \textit{Yogakarṇikā} 1.23-61 under the heading \textit{dinacaryā} (`daily [ritual] behaviour'), there are detailed descriptions of daily yogic ritual behaviour. It is surprising that for a large part of the first chapter (1.1-168), the source texts of the verses are not given, especially since the rest of the first chapter and all other chapters of the text are largely compilations of quotations consisting of verses from other texts on typical yogic topics. Throughout the \textit{Yogakarṇikā}, larger sections of the \textit{Yogasvarodaya} and other texts are repeatedly quoted with reference. Is it possible that Nāth Aghorānanda, the author of the \textit{Yogakarṇikā}, also drew on verses from the \textit{Yogasvarodaya} here? At least in the second part of the first chapter (1.169-280), 37 verses (1.244-280) have been taken from the \textit{Yogasvarodaya} with reference and demonstrably at least four verses without reference (1.210-213).\footnote{The verses of \textit{Yogakarṇikā} 1.244-280 and 1.210-213 are all also found in the \textit{Prāṇatoṣiṇī}, ed. pp. 832-833 and ed. p. 831}. This question cannot be answered without manuscripts of the \textit{Yogasvarodaya}. However, there may be further verses of the \textit{Yogasvarodaya} within the first 168 verses of the \textit{Yogakarṇikā}. Nonetheless, for the time being, one of the most plausible scenarios is that the original Caryāyoga in the taxonomy of the fifteen Yogas was similar to the contents of the \textit{dinacaryā} section of the \textit{Yogakarṇikā}. This section deals with daily ritual ablutions with mantra recitation, visualisation and meditation (1.23-36) and other ritual acts such as ritual dressing, the application of the sectarian sign (\textit{tilaka}) including tying the hair into a knot (1.38), offerings, and the devotional performance of prostrations in front of one's own \textit{iṣṭadevatā} etc. (1.39-61).               

\subsection{Caryāyoga in the \textit{Yogasiddhāntacandrikā}}

In his \citetitle{yogacandrika}\footnote{\citetitle{yogacandrika}, ed. pp. 2, 52-53, 100-101, 150.} Nārāyaṇatīrtha presents Caryāyoga\footnote{For an earlier brief discussion of Caryāyoga in Nārāyaṇatīrtha's \textit{yogacandrika} see \citeauthor{penna2004}, 2004: 66-67.} in the context of Yogasūtra 1.33:

\begin{quote}
  Due to impurities of the mind like jealousy, etc., preventing the attainment of Yoga, the method of removing them is Caryāyoga - Purity of the mind arises through the cultivation of friendliness, compassion, joy and equanimity in circumstances of happiness, suffering, virtue and vice. \footnote{\citetitle{yogacandrika}, Ed. p. 52 (\textit{tasya cittasyāsūyādimalavato yogāsambhavāt tannirāsopāyaṃ caryāyogam āha- maitrīkaruṇāmuditopekṣāṇāṃ sukhaduḥkhapuṇyāpuṇyaviṣayāṇāṃ bhāvanātaścittaprasādanam} || 33 ||)} 
  \end{quote}
  
Caryāyoga is to cultivate kindness towards those in fortunate circumstances to prevent jealousy. Towards those who are in sorrowful circumstances, compassion is supposed to be cultivated to prevent ill-will. Towards those who act virtuously, one is supposed to cultivate joy to prevent aversion; and towards those who act unvirtuously, one is supposed to cultivate equanimity to prevent anger.\footnote{Cf. Ibid. (\textit{tathā ca sukhiteṣu maitrīṃ sauhārdam īrṣyākāluṣyanivarttakaṃ, duḥkhiṣu karuṇāṃ dayāmasūyākāluṣyanivarttikāṃ, puṇyavṛttiṣu harṣaṃ dveṣanivarttakam, apuṇyaśabditapāpiṣu upekṣām amarṣakāluṣyanivarttikāṃ bhāvayet} |)}        

 With this practice of Caryāyoga, which gradually purifies the mind, the sattvic nature of the mind is brought forth. This leads to a clear and serene mind.\footnote{Cf. \citetitle{yogacandrika}, ed. pp. 52- 53 (\textit{tad evaṃ caryāyogena cittamalanirāsakena mukhyādiṣu yathākramamuktabhāvanārūpeṇa sāttviko dharmo jāyate} | \textit{tena ca śuklena dharmeṇa cittaṃ prasannaṃ bhavati} | \textit{prasāde ca sthitipadaṃ labhate} | \textit{etac ca puṣkalaṃ viraktasyaiva sambhavatīti mukhyacaryāyogo vairāgyameveti saṃkṣepaḥ} || 33 ||)}

 Since the word \textit{caryā°} in this context refers to purposeful behaviour designed to give rise to the sattvic nature, the Caryāyoga of the \textit{Yogasiddhāntacandrikā} can be meaningfully translated as 'Yoga of behaviour'.  

\subsection{Carcāyoga in the \textit{Sarvāṅgayogapradīpikā}}

Within \citetitle{sarvangayoga} (2.40-51, Ed. pp. 96-98), Sundardās describes Cārcāyoga as one of the three subtypes of Bhaktiyoga which is \textit{bhakti} towards unmanifest consciousness (\textit{avyakta puruṣa}) in delightful devotion. \footnote{See \citeauthor{burger2014sarvangayogapradipika} (2014: 694-695) for an earlier brief discussion of Sundardās's Carcāyoga in French}. He extensively describes the unmanifest consciousness (\textit{avyakta puruṣa}) as being formless and eternal and so on (40), as beginningless and endless, and so on (41). Next, Sundardās describes the various layers of creation emanating from \textit{oṃ} (42-45). He says the unmanifest consciousness illuminates every corner of existence (46), being the inner knower of all (47). Then, Sundardās expresses the importance of deep awe towards the infinite, divine, all-knowing and incomprehensible (48-49) unmanifest consciousness.

The entire passage on Carcāyoga is characterised by a discussion and description of the unmanifest consciousness (\textit{avyakta puruṣa}). This aspect is the core of this type of Yoga. Unlimited unmanifested consciousness can be put into limiting words only, and yet the practitioner is confronted with the question of how it is supposed to be defined and determined.\footnote{Cf. \citetitle{sarvangayoga} 2.41ab (\textit{avyakta puruṣa agama apārā} \textit{kaisaiṃ kai kariye nirddhārā} |} And this is precisely the practice of Carcāyoga. The term \textit{carcā°} here refers to `discussing' or `putting into words' and emphasising individual details of unmanifest consciousness to generate deep reverence for the cultivation of Bhaktiyoga, the Yoga of devotional worship of \textit{avyakta puruṣa}. The following verse illustrates this:
\begin{quote}
How to discuss, where to find you, O Lord? You are the inner knower of everything. There is no end to describing creation. Your limit cannot be reached by any means.\footnote{Cf. Ibid. 2.47 (\textit{carcā karaiṃ kahāṃ laga svamī} | \textit{tum saba hī ke antarjāmī} | \textit{sṛṣṭi kahat kachu anta na āvai} | \textit{terā pāra kaiṃna dhaiṃ pāvai} || 47 ||} \end{quote}

Thus, it is clear that no direct conceptual connection exists between the Caryāyogas described above and Carcāyoga. A meaningful explanation for the conspicuous homophony of both terms cannot be offered for the time being.  

\section{4. Haṭhayoga}
\label{hathayogaintro}

Haṭhayoga appears without exception in all complex late medieval yoga taxonomies. In the taxonomies with fifteen Yogas of the \textit{Yogatattvabindu}, the \textit{Yogasvarodaya} and the \textit{Yogasiddhāntacandrikā}, it occupies the fourth position. In the twelvefold taxonomy of Sundardā's \textit{Sarvāṅgayogapradīpikā}, it is in fifth position and, in addition to its own category, is also the superordinate category for the three subsequent Yogas described by Sundardās, namely Rāja-, Lakṣa- and Aṣṭāṅgayoga.

\subsection{Haṭhayoga in the \textit{Yogatattvabindu} and \textit{Yogasvarodaya}}

In section \uproman{19}-\uproman{20} of the \textit{Yogatattvabindu}, two categories of Haṭhayoga are distinguished. Both are based on the explanations of the \textit{Yogasvarodaya}, differ only slightly in formulation, and can, therefore, be considered together.\footnote{See \citetitle{ramatosana}, ed. p. 835 and \citetitle{shabdakalpadruma}, ed. p. 501. These passages contain quotations from the \textit{Yogasvarodaya} of both types of Haṭhayoga. See also \citetitle{yogakarnika} 12.23-26. Here, verses of the second category of Haṭhayoga are reproduced}. Both passages in these two texts are characterized by their brevity. 

The first type of Haṭhayoga described teaches the control of the breath through exhalation (\textit{recaka}), inhalation (\textit{pūraka}) and breath retention (\textit{kumbhaka}) etc. With the term ``etc.'' (\textit{°ādi°}), the text probably refers to other known practices of \textit{Haṭhayoga}. In addition to other breathing exercises, this could also refer to the other known basic building blocks of Haṭhayoga, which have been associated with Haṭhayoga since Svātmarāma's \textit{Haṭhapradīpikā}: \textit{āsana}, \textit{mudrā} and \textit{nādānusandhāna}. At least \textit{āsana} is explicitly mentioned in the \textit{Yogasvarodaya}, but not in the \textit{Yogatattvabindu} (\textit{kṛtvāsanaṃ pavanāśaṃ śarīre rogahārakam}). Both texts then mention the six actions that purify the body (\textit{ṣatkarma}). Then Rāmacandra states that when the full breath dwells within the solar channel (\textit{sūryanāḍi}), the mind becomes immobile. Through the immobility of the mind, bliss arises, and the mind is absorbed into emptiness (\textit{śūnya}). The resulting state leads to the delay of the time of death (\textit{kālaḥ samīpe nāgachati}). The naming of the sun channel is striking in this context. The \textit{Yogasvarodaya} is no concrete help here, as it merely speaks of an unspecified \textit{nāḍī},\footnote{Since the YSv mentions no specific \textit{nāḍī}, it is likely that it is the \textit{nāḍī} \textit{par excellance}, the \textit{suṣūmnā}} in which, triggered by the preceding practice, the fullness of breath is established (\textit{etan nāḍyān tu deveśi vāyupūrṇaṃ pratiṣṭhitam} | \textit{tato mano niścalaṃ syāt tata ānanda eva hi} |). The majority of texts of the Haṭhayoga genre would certainly specify \textit{suṣūmnā}, the central channel, and not the right channel associated with the sun called \textit{piṅgalā}, in the context of the ``immobility of the mind'', a central characteristic of the \textit{samādhi} state, because the occurrence of the yoga state, or \textit{samādhi}, is often associated with the entry of the breath into the central channel. \footnote{This is already evident, for example, in the oldest written testimony of the Haṭhyoga genre, the \textit{Amṛtasiddhi} 26.1-2 (\textit{yo 'sau siddhimayo vāyur madhyamāpadaniścalaḥ} | \textit{tadānandamayaṃ cittam ekarūpaṃ nabhaḥsamam} || 26.1 || \textit{yadānandamayaṃ cittaṃ bāhyakleśāvivarjitam} | \textit{bhavaduḥkhāni saṃhṛtya samādhir jāyate tadā} || 26.2 ||) is the case. \citeauthor{asiddhi} translate: (1) `When Breath is perfected and fixed in the place of the Goddess of the Centre, then consciousness has the nature of bliss, uniform like the sky.' (2) `When consciousness has the nature of bliss, free from external afflictions, then, having the sorrows of existence, Samādhi arises'. This idea, which can be found in this genre from the 11th century at the latest, subsequently permeates the entire genre}. Either the term \textit{sūryanāḍi} is to be understood here as an unfortunate synonym,\footnote{In the sense of being ambiguous and overlapping with the \textit{piṅgalā} channel.} or the text is corrupt.\footnote{A conjecture of \textit{sūryanāḍī} to \textit{śūnyanāḍī} would be obvious. In \textit{Jyotsnā} 4.10, Brahmānanda understands ``the void'' (\textit{śūnya}) as the central channel. In \textit{Haṭhapradīpikā} 3.4, \textit{śūnyapādavī} is a synonym of \textit{suṣumnā}.} A final possibility would be to assume a practice associated with the \textit{piṅgalā} channel. The term \textit{sūryanāḍī} is found in the \textit{Siddhasiddhāntapaddhati}, a text that also served as a model for Rāmacandra.\footnote{Cf. \textit{Siddhasiddhāntapaddhati} 2.5: \textit{pañcamaṃ kaṇṭhacakraṃ caturaṅgulaṃ tatra vāme iḍā candranāḍī dakṣiṇe piṅgalā sūryanāḍī tanmadhye suṣumnāṃ dhyāyet saivānāhatakalā anāhatasiddhir bhavati} |}

The second type of Haṭhayoga in \textit{Yogatattvabindu} instructs the yogin to contemplate a non-specific form (\textit{kiṃcidrūpā}) in the colours white, yellow, blue and red equal to the radiance of ten million suns in one's own body from head to toe (\textit{cintyate}). This is supposed to burn away all diseases of the body and prolong life. In the \textit{Yogasvarodaya}, there is no mention of an unspecific form. However, these colours and the sun's radiance are meant to be contemplated in the area of the tip of the nose (\textit{ākāśe nāsikāgre tu sūryakoṭisamaṃ smaret} | \textit{śvetaṃ raktaṃ tathā pītaṃ kṛṣṇam ity ādirūpataḥ}). |). Rāmacandra and the \textit{Yogasvarodaya} describe the second type of Haṭhayoga so briefly and vaguely that the reader is denied a clearer picture. It should be noted at this point that the formulation is very reminiscent of Bāhyalakṣya's explanations in section \uproman{23}\footnote{Cf. p. \pageref{bahya}}. Interestingly, in Sundardā's \citetitle{sarvangayoga}, Lakṣ(y)ayoga is a subcategory, i.e. a partial practice, of Haṭhayoga. Is this the source for this differentiation? Further parallels to practices of other texts of Haṭhayoga involving coloured or non-coloured light exist but are still too distant to convincingly assign Rāmacandra's second type,\footnote{see p.\pageref{bahyatrans} for the parallel passages} and thus remain enigmatic for the time being.

\subsection{Haṭhayoga in the \textit{Yogasiddhāntacandrikā}}

In the \textit{Yogasiddhāntacandrikā}, the discussion and description of Nārāyaṇatīrthas Haṭhayoga is spread over several \textit{sūtra}s of the first two chapters, the \textit{samādhipāda} (1.34) and the \textit{sādhanapāda} (2.46-52). The commentary by Nārāyaṇatīrtha is particularly extensive and detailed here.\footnote{For an earlier, short discussion of Haṭhyoga in Nārāyaṇatīrtha's \textit{yogacandrika} see \citeauthor{penna2004}, 2004: 76.}

Nārāyaṇatīrtha first locates Haṭhayoga in the context of \textit{sūtra} 1.34. This \textit{sūtra} is one of several options (1.32-40) that can be applied to overcome the distractions described in \textit{sūtra}s 1.30-31, which distract from the state of yoga (\textit{asaṃprajnātasamādhi} or \textit{kaivalya}) sought in Pātañjalayoga:
\begin{quote} \textit{pracchardanavidhāraṇābhyāṃ vā prāṇasya} || 34 || \end{quote}
\begin{quote} Or, through exhaling and restraining of the breath. \end{quote}

This method thus serves to establish a clear mind. This is referred to by Nārāyaṇatīrtha as Haṭhayoga. In his commentary, Nārāyaṇatīrtha explains that the term \textit{pracchardana} means the slow outward emptying of the breath of the abdomen through one of the two nostrils in measured quantities.\footnote{\citetitle{yogacandrika} 1.34 (Ed. p. 53): \textit{kauṣṭhyasya vāyoḥ pracchardanam, ekataranāsāpuṭena mātrāpramāṇena śanaiḥ śanair bāhar niḥsāraṇam |}} The term \textit{vidhārana} is the external continuous breath-holding of exhaled air.\footnote{Ibid. 1.34 (Ed. p. 53): \textit{vidhāraṇaṃ recitasya vāyor bahir eva sthāpanaṃ kumbhakaṃ} |} Furthermore, Nārāyaṇatīrtha specifies this method of breath retention as \textit{recitakumbhaka}. It is the first of a total of seven breathing postures (\textit{saptakumbhaka}) and is considered particularly praiseworthy, as hardly any rules need to be observed for this type. However, this group of seven \textit{kumbhaka}s - \textit{recita, pūrita, śānta, pratyāhāra, uttara, ādhāra} and \textit{sama} - is only mentioned in the second chapter in the context of the fourth limb of the \textit{aṣṭāṅgayoga} called \textit{prāṇāyāma} (2. 49-53) together with another seven of the eight \textit{kumbhaka}s of the \citetitle{hathapradipika2024}.\footnote{Ibid. 1.34 (Ed. p. 53): \textit{tathā cātra pūrakavarjanād recitapūritaśāntapratyāhārottarādhārasamabhedena saptakumbhakeṣu madhye recitakumbhako 'yaṃ prathamābhyāse 'nekaniyamānapekṣatayā praśastaḥ} | \textit{sarvam etad agre prāṇāyāmaprakaraṇe sphuṭī bhaviṣyati} |}

According to Nārāyaṇatīrthas, the mastery of the breath and the mastery of the mind are intrinsically linked. At the same time, \textit{prāṇāyāma} has the power to eradicate all sins, which enables the mind to concentrate and stabilize on a meditative focal point or goal (\textit{lakṣya}).\footnote{\citetitle{yogacandrika} 1.34 (Ed. p. 53): \textit{tad etābhyāṃ prāṇajaye cittajayas tayor avinābhāvāt prāṇāyāmasya sarvapāpanāśakatvāt pāpanivṛttyā ca cittam ekatra lakṣye sthiraṃ bhavati} |}

Finally, Nārāyaṇatīrtha authenticates the linking of \textit{prāṇāyāma} and Haṭhayoga (\textit{prāṇāyāmasya haṭhayogatvam uktaṃ smṛtau}) with the famous verse of \citetitle{yogabija} (148cd-149ab), in which the syllable ``\textit{ha}'' is linked to the sun and the syllable ``\textit{ṭha}'' to the moon. Thus, \textit{haṭha} is understood as the union of sun and moon. \footnote{Ibid. 1.34 (ed. p. 53): \textit{hakāreṇa tu sūryo 'sau ṭhakāreṇendur ucyate} | \textit{sūryācandramasor aikyaṃ haṭha ity abhidhīyate} || The context suggests here, that Nārāyaṇatīrtha associates the sun and moon with the \textit{piṅgalānāḍī} (representing the sun) and \textit{iḍānāḍī} (representing the moon). Their union would then be the inhalation through these channels with subsequent breath holding.}

The next section of the \textit{Yogasiddhāntacandrikā}, which discusses aspects of Haṭhayoga, is only found in the context of the third limb of the \textit{aṣṭāṅgayoga}, which is described beginning with \textit{sūtra} 2.46.

\begin{quote} \textit{itaḥ paraṃ sakalarogādinivṛttidvārā haṭhayogasyopāyam āsanam āha- \\
sthirasukham āsanam} || 46 || \end{quote}
\begin{quote} From here on, postures, being the means of Haṭhayoga, are said to be the gateways to preventing all diseases etc. \\
A comfortable and steady position.
\end{quote}

Nārāyaṇatīrtha then presents various \textit{āsana}s. Of a total of 84 \textit{āsana}s, 38 are described in detail. \citeauthor{birch2018proliferation} observed as early as 2018,\footnote{Cf. \citeauthor{birch2018proliferation} 2018, p. 105, fn. 9. } that Nārāyaṇatīrtha's descriptions of the \textit{āsana} were borrowed from earlier yoga texts, such as the \textit{Haṭhapradīpikā} (which Nārāyanatı̄rtha refers to as \textit{Yogapradīpa}), the \citetitle{vasisthasamhita2017} and the \textit{Dharmaputrikā}. \footnote{A list of the 38 of 84 \textit{āsana}s discussed in detail below can be found in \citetitle{yogacandrika} 2.46 (Ed. p. 107-108): \textit{tac ca padma-siddha-bhadra-vīra-svastika-siṃha-daṇḍa-sopāśraya-paryaṅka-mayūra-kukkuṭa-uttānakukkuṭa-paścimatāna-matsyendrapīṭha-cakra-gomukha-karma-dhanu-mṛgasvastika-arddhacandra-añjalika-pīṭha-vajra-mukta-candra-arddhaprasāritaśava-kapāla-guruḍa-arddhāsana-kamala-krauñcaniṣadana-hastiniṣadana-uṣṭraniṣadanakapiniṣadana-yogāsana-yonyāsana-samasthāna-ādibhedena caturāśītiprakāram} | \textit{eteṣāṃ lakṣaṇāni yogapradīpādāv uktāni} | The detailed descriptions of the 38 \textit{āsana}s can be found immediately following on p. 108-114.}\footnote{\citeauthor{penna2004} (2004: 207-209) has briefly discussed the \textit{āsana}s of the \textit{Yogasiddhāntacandrikā}.}

In 2.47-48, Nārāyaṇatīrtha specifies further details on the execution of the Yoga postures, which are not discussed further here.\footnote{A more detailled sketch of the \textit{prāṇāyāma}-system of Nārāyaṇatīrtha's \textit{Yogasiddhāntacandrikā} can be found in \citeauthor{penna2004} (2004: 209-18).} Far more important for the determination of Nārāyaṇatīrtha's Haṭhayoga is 2.49-51. In addition to a detailed discussion of the three basic elements of \textit{prāṇāyāma} - exhalation (\textit{recaka}), inhalation (\textit{pūraka}) and breath holding (\textit{kumbhaka}) and their specifics in the commentary to 2. 49-50, Nārāyaṇatīrtha then discusses \textit{kevalakumbhaka}, the fourth aspect of \textit{prāṇāyāma}, the overarching goal and ultimate result of breath retention.\footnote{Cf. \citetitle{yogacandrika} 1.34 (Ed. p. 116): `Yājñavalkya declares its characteristic as follows - Having abandoned inhalation and exhalation, that comfortable restraint of breath is breath-control. This indeed is indeed taught as ``isolated retention''.' (\textit{asya ca lakṣaṇaṃ yājñavalkya āha- recakaṃ pūrakaṃ tyaktvā yat sukhaṃ vāyudhāraṇam} | \textit{prāṇāyāmo 'yam ity uktaḥ sa vai kevalakumbhakaḥ} ||}\footnote{See \citetitle{hathapradipika2024} 2.72-80 for the \textit{locus classicus} of all descriptions of \textit{kevalakumbhaka}.} 

This \textit{kevalakumbhaka} is achieved in a lengthy process with gradually more subtle advances through the practice of ordinary \textit{kumbhaka}, which is specified as \textit{sahitakumbhaka}.\footnote{This \textit{kumbhaka} is "accompanied" (\textit{sahita}) because, unlike \textit{kevalakumbhaka}, it is still accompanied by inhalation and exhalation. Cf. \citetitle{hathapradipika2024} 2.73.} Only when the bodily channels have been purified through practice, and the movements of exhalation and inhalation have entirely ceased does \textit{kevalakumbhaka} arise. An appropriate translation is ``isolated breath retention'', as it is isolated from the inhalation and exhalation.\footnote{Cf. \citetitle{yogacandrika} 2.51: \textit{evambhūta ubhayoḥ śvāsapraśvāsayor gativicchedaś caturthaḥ prāṇāyāma ity arthaḥ} | \textit{etena sahitakumbhakābhyāsa evāsyā 'sādhāraṇam} | \textit{yadā nāḍīviśuddhiḥ syād yoginastattvadarśinaḥ} | \textit{tadā vidhvastadoṣasya bhavet kevalasambhavaḥ} ||}

The yogin who masters \textit{kevalakumbhaka} can hold his breath for any length of time.\footnote{Cf. \citetitle{hathapradipika2024} 2.76.} Nārāyaṇatīrtha then quotes seven of the eight \textit{kumbhaka}s\footnote{\citetitle{yogacandrika} 2.51, ed. p. 118-121. The seven \textit{kumbhaka}s mentioned by Nārāyaṇatīrtha are: 1. \textit{sūryabhedana}; 2. \textit{ujjāyī}; 3. \textit{sītkā(ra)}; 4. \textit{śītalī}; 5. \textit{brahmarī}; 6.\textit{mūrchā}; and 7. \textit{bhastrikā}.} of \citetitle{hathapradipika2024} (except \textit{plāvanī}, cf. \citetitle{hathapradipika2024} 2.71).\footnote{Cf. \citetitle{hathapradipika2024} 2.48-71.} Then the other seven \textit{kumbhaka}s already mentioned in the commentary to 1.54 are explained in more detail.\footnote{\textit{Yogasiddhāntacandrikā} 2.51, p. 121: \textit{kumbhaḥ saptavidho jñeyo recitādiprabhedataḥ} | \textit{recitaṃ pūratiḥ śāntaḥ pratyāhārottaro'dharaḥ} || \textit{samaśceti vinirdiṣṭaḥ kumbhakaḥ saptabhedataḥ} \textit{iti eteṣāṃ lakṣaṇāni cāha-} \textit{recitasya bahistambho vāyo recitakumbhakaḥ} \\textit{pūrakeṇa vinā samyag yogo 'yaṃ sukhado nṛṇām} || 1 || \textit{pūritasyodare rodhaḥ paścādrecakasaṃyutaḥ} | \textit{nāḍīśuddhikaraḥ samyak proktaḥ pūritakumbhakaḥ} || 2 || \textit{kāyasyāntarbahir vyāptir yā sa syāc chāntakumbhakaḥ} || 3 || \textit{sthānayorantare rodhaḥ pratyāhārākhyakumbhakaḥ} || 4 || \textit{āpūrayet kramādūrdhvam ūrdhvarodho hṛdādiṣu} || 5 || \textit{uttaraḥ kumbhakaḥ sa syādadho 'dho mūrddhato 'dharaḥ} || 6 || \textit{recanāpūraṇe tyaktvā manasā maruto dhṛtiḥ} | \textit{yā nābhyādpradeśeṣu samaḥ kumbhaḥ prakīrttitaḥ} || 7 ||} The commentary to 2.50 then quotes further explanations from various texts, such as \textit{Yogabhāskara}, \textit{Nandipurāṇa} and \textit{Mārkaṇḍeyapurāṇa} on the subject of \textit{prāṇāyāma}. In addition, the four stages (\textit{avasthā}) of yoga practice - \textit{ārambha}, \textit{ghāṭa}, \textit{paricaya} and \textit{niṣpatti} are introduced,\footnote{See \citetitle{asiddhi} \textit{viveka} 19,21,29 and 31 for the oldest account of the four stages. Also cf. \citetitle{hathapradipika2024} 4.16-25.} etc.\footnote{For example, the yogic dietary guidelines and the dwelling of the yogi based on the explanations of the first chapter of \citetitle{hathapradipika2024}.}

The Haṭhayoga of Nārāyaṇatīrtha thus consists primarily of two of the four main classical categories of Haṭhayoga according to the \citetitle{hathapradipika2024}\footnote{Cf. \citetitle{hathapradipika2024} 1.56.} - \textit{āsana} and \textit{kumbhaka}, which are located in Pātañjalayoga. The third main category of Haṭhayoga after the \citetitle{hathapradipika2024}, namely \textit{mudrā}, is also found in the \textit{Yogasiddhāntacandrikā}. However, surprisingly, the \textit{mudrā}s, together with the \textit{ṣatkarma}s, are only taught in the context of Karmayoga. Surprisingly, because \textit{mudrā} and \textit{ṣaṭkarma} are the elements of Haṭhayoga that form the main distinguishing feature from other Yoga systems. Nārāyaṇatīrtha is not unaware of this. At the end of his section on Karmayoga, he mentions them belonging to Haṭhayoga, but nonetheless decides to present them in the contect of Karmayoga. These will, therefore, only be dealt with in the corresponding sub-chapter of this work. The fourth main category of the \citetitle{hathapradipika2024}, \textit{nādānusandhāna}, is not found in the \textit{Yogasiddhāntacandrikā}.

\subsection{Haṭhayoga in the \textit{Sarvāṅgayogapradīpikā}}

In the \textit{Sarvāṅgayogapradīpikā} (3.1-52), Haṭhayoga is both an individual category (3.1-12) and a superordinate category. In the following, Haṭhayoga is primarily discussed as the individual category. As a superordinate category, it subsumes three other Yogas, namely Rājayoga (3.13-24), Lakṣayoga (3.25-36) and Aṣṭāṅgayoga (3.37-52). These subcategories will be only briefly characterised in this chapter. They are then discussed in detail in the respective chapter according to the order of the list of the fifteen Yogas of the \textit{Yogatattvabindu}.\footnote{A French description of Haṭhayoga in the \textit{Sarvāṅgayogapradīpikā} can be found in \citeauthor{burger2014sarvangayogapradipika} 2014, pp. 701-709.}

Sundardās initially locates Haṭhayoga within the Āditnātha tradition and specifies the union of sun and moon as its definition.\footnote{\citetitle{sarvangayoga} 3.1: \textit{abahi hahūṃ haṭhayoga sunāī} | \textit{ādinātha ke bandaiṃ pāī} | \textit{ravi śaśi doū eka milāvai} | \textit{yāhī teṃ haṭhayoga kahāvai} || 1 ||}

This is followed by describing the ideal environment for Yoga practice, short practice instructions and dietary rules (3.2-8). These are very reminiscent of the explanations in the first chapter of the \textit{Haṭhapradipikā}.\footnote{See \citetitle{hathapradipika2024} 1.57-60.} The chapter concludes with the naming of the six actions (\textit{ṣaṭkarma}s). Due to the lack of details in his descriptions, it is hardly comprehensible to perform the practices without a teacher or other instructive texts. Sundardās could not have conceived his chapter on Haṭhayoga as an instruction manual. Instead, his primary aim must have been to list and characterise it.

The ideal environment for Haṭhayoga is in a well-governed country where justice prevails. Here, the yogin is supposed to build a hut (\textit{maṭhikā}) with a small door and no holes. The yogin shall smear the hut with cow dung for this purpose. A small well is dug into the ground next to the hut.\footnote{Ibid. 3.2-3ab: \textit{prathama sudharma deśa kahuṃ tākai} | \textit{bhalau rājya kachu deṣala na jākai} | \textit{tāhāṃ jāī kai maṭhikā karī} | \textit{alpa dvāra aru chidra su bharaī} || 2 || \textit{lipta karai cahūṃ ora sugandhā} | \textit{kūpa sahita maṭha ihīṃ bidhi baṃdhā} |}\footnote{Cf. \citetitle{hathapradipika2024} 1.12-13.}

The yogin is supposed to sit in the hut, devote himself to Haṭhayoga and regulate the breath.\footnote{\textit{Sarvāṅgayogapradīpikā} 3.3cd: \textit{tāmahiṃ paiṭhi karai abhyāsā | gutu gami haṭha kari jātai svāsā} || 3 ||} Accordingly, for Sundardās, as in all texts with complex Yoga taxonomies without exception, breath cultivation is the central element of Haṭhayoga. In the following, he specifies the practice of Yoga postures (\textit{āsana}).\footnote{\textit{Sarvāṅgayogapradīpikā} 3.5ab: \textit{haṭhi kari āsana sādhaiṃ bhāī} \textit{hatha kari nidrā tajatau jāīī} |} Furthermore, Sundardās recommends ritual washing and god worship in the morning.\footnote{Ibid. 3.7b: \textit{prāta sanāna upāsana koī} | What this might have looked like is described in great detail within the first chapter of the \textit{Yogakarṇikā}.} The diet is supposed to be regulated.\footnote{Ibid. 3.5c: \textit{haṭha hī kari āhāra ghaṭāvai} |} For Sundardās, this means avoiding hot, spicy and sour foods. Specifically mustard, sesame, alcohol, meat, green vegetables, ginger and garlic, shall be avoided, too.\footnote{Ibid. 3.6: \textit{haṭha kari tīkṣaṇa kaṭuka sutyāgai} | \textit{sarasoṃ tila mada māṃsa na māṃgai} | \textit{harita śāka kabahū nahiṃ ṣaī} | \textit{hiṃgu lasanu saba deśa bahāī} || 6 ||} A diet of rice, milk,\footnote{Ibid. 3.7c: \textit{gohūṃ śāli su karai ahārā} |} ghee, honey and gourd vegetables is recommenced. Furthermore, clear water is supposed to be ingested.\footnote{Ibid. 3.8ab: \textit{ṣīra ṣāṃḍa ghṛta madhi puni sāṃnī} \textit{sūṃṭhi paṭola nirmala ati pāṃnī} |} When the haṭhayogin eats in this way, his body is freed from disease.\footnote{Ibid. 3.8cd: \textit{yahu bhojana su karai haṭha yogī} \textit{dina dina kāyā hoī nirogī} || 8 ||}

Verses 3.9-11 mention the six actions (\textit{ṣaṭkarma}s) - \textit{dhauti}, \textit{basti}, \textit{netī}, \textit{trāṭaka}, \textit{naulī} and \textit{kapālabhātī}. They are supposed to to purify the channels,\footnote{Ibid. 3.9b: \textit{nāḍī śuddha hoṃhi mala ṭalai} |} and lead to success.\footnote{Ibid. 3.10c: \textit{ye ṣaṭa karma siddhi ke dātā} |} In the last verse of this section, we learn that the power of Haṭhayoga leads to bliss.\footnote{Ibid. 3.12a: \textit{yā haṭha yoga prabhāva teṃ, pragaṭa hoī ānanda} |}

As already mentioned at the beginning, Sundardās also subsumes Rājayoga (3.13-24), Lakṣayoga (3.25-36) and Aṣṭāṅgayoga (3.37-52) under the superordinate category Haṭhayoga. Sundardā's Rājayoga practice is that what is commonly known as \textit{vajrolīmudrā}.\footnote{The verses do not specify the term, but the practice is identical.} Lakṣ(y)ayoga, a practice found in all complex late medieval taxonomies, is the fixation of the gaze (\textit{dṛṣṭi}) on differently located focal points or objects inside or outside the body. In the context of Aṣṭāṅgayoga, the generally known eight limbs are then discussed individually. Similar to Nārāyaṇatīrtha, characteristic practices of Haṭhayoga such as \textit{āsana}s, \textit{kumbhaka}s, \textit{mudrā}s and \textit{bandha}s are assigned to the individual limbs. A detailed comparative discussion of the subcategories takes place in the following chapters.

\section{5. Karmayoga}
\label{karmayogaintro}

In formal discourse, the term Karmayoga is particularly known from the \citetitle{kaushik1993}\footnote{Cf. for example \citetitle{kaushik1993} 2.47-49, 3.1-7, \& 4.20. Here, Karmayoga is a path (\textit{marga}) to liberation (\textit{mokṣa}) through action (\textit{karma}) without attachment to one's deeds.}. In the four complex late medieval taxonomies of the twelve to fifteen Yogas, it appears in fifth place in the \textit{Yogatattvabindu} and third place in the \textit{Yogasvarodaya} and \textit{Yogasiddhāntacandrikā}. The \textit{Sarvāṅgayogapradīpikā} does not mention Karmayoga. 

\subsection{Karmayoga in the \textit{Yogatattvabindu} and \textit{Yogasvarodaya}}

In both texts, the term Karmayoga is not mentioned, despite its inclusion in the taxonomies. This absence surprises the reader, as the structure of the text, beginning with the list of fifteen Yogas and then treating individual Yogas, raises the expectation that all the subtypes of Yoga mentioned in the list will be treated. It is particularly noteworthy that Kriyāyoga, as the first entry in the list, is also treated first, and the following sections of the text largely follow the order of the list, reinforcing this expectation. However, this expected structure becomes less and less clear as the text progresses. This results in two possible explanations. Either the list merely served to illustrate the diversity of the different categories of Yoga, and it was never the authors' intention to cover all the Yogas, or the transmission of the text has fallen victim to corruption. The analysis of the texts made it clear that Rāmacandra based at least the first half and also large parts of the second half of the text on the \textit{Yogasvarodaya}.\footnote{In the second half of his text, Rāmacandra also frequently uses content and verses from the \citetitle{ssplonavla} and almost without exception follows the structure as given by the quotations from the \textit{Yogasvarodaya} in the \textit{Prāṇatoṣinī}.}
However, we also know that the transmission of the \textit{Prāṇatoṣinī} is by no means complete. Many of the verses of the \textit{Yogasvarodaya} found in the \textit{Prāṇatoṣinī} can also be found in the \textit{Yogakarṇikā}. In addition, the \textit{Yogakarṇikā} contains a non-negligible number of verses that are not found in the \textit{Prāṇatoṣinī} but are nevertheless attributed to the \textit{Yogasvarodaya}.\footnote{Surprisingly, the contents of the verses of the \textit{Yogasvarodaya} cannot be traced in the \textit{Yogakarṇikā} either. Does this mean that \textit{Yogatattvabindu} used the quotations from \textit{Prāṇatoṣinī} as a template? This is impossible, as the \textit{Prāṇatoṣinī} dates from the 19th century. There were probably several recensions of the \textit{Yogasvarodaya}.} This means that the transmission of the \textit{Yogasvarodaya} based only on the verses of the \textit{Prāṇatoṣinī} and the \textit{Yogakarṇikā} cannot possibly be complete, and the original text may also have described the other fifteen Yogas not mentioned in the quotations. The structural analysis of both texts in the context of Karmayoga reveals a strong indication of corruption in the tradition.
This reference is in section \uproman{41}. Like the previous sections, starting with \uproman{32}, this section deals with the microcosmic equivalents of the macrocosm in the yogic body. In particular, it deals with the listing of various contents of the yogic body, such as twenty-seven stars, twelve signs of the zodiac, nine planets, the fluctuation of the Ūrmi, which sets the body in motion, countless deities inhabiting the pores of the arms, celestial ascetics (\textit{divyatapasvin}s) residing in the pores of the back, etc.
Then, the topic changes abruptly. In both the \textit{Yogatattvabindu} and the \textit{Yogasvarodaya}, there is suddenly a passage that describes \textit{mukti} through \textit{karma}, without a corresponding preceding introduction. Rāmacandra, apparently, as so often, prosaises the contents of \textit{Yogasvarodaya}. Therefore, the text's structural problem originates in the \textit{Yogasvarodaya}. The change in content is so abrupt that one or more folios of the copy of an archetype on which the surviving text was ultimately based may have been lost. This section of the text, which concludes the \uproman{41} section, could well be part of an original description of Karmayoga due to the abrupt change of subject. \\

The \textit{Yogasvarodaya} (PT, Ed. p. 843-44) reads:

\begin{quote}
  \textit{samagradarśanān muktaḥ svargabhogañ ca matsukham} | \\
  \textit{tad etac cintayā yāti rogaśokavivarjjitaḥ} ||\\
  \textit{yat karmā karmaṇā śaṅkā manomadhye bhaved bahiḥ\footnote{\textit{bahiḥ} em.] \textit{vahiḥ} YSv (PT).}} |\\
  \textit{tat karmākaraṇaṃ}\footnote{\textit{karmākaraṇaṃ} em.] \textit{karmakaraṇaṃ} YSv (PT).} \textit{muktir ity āha bhagavān śivaḥ} ||
\end{quote}
\begin{quote}
  As a result of complete vision\footnote{It seems very unlikely that this \textit{samagradarśanāt} refers back to the previously mentioned microcosmic contents of the macrocosm. Especially given the following statements about \textit{karma}. What it refers to is unclear.} one is liberated from heavenly pleasures and happiness. Through contemplating that, one reaches freedom from sorrow and disease. Whatever action creates concern within the mind by [considering] the action, externally, the non-execution of that [very] action brings about liberation. Thus says the exalted Śiva.
\end{quote}

The modified prosaisation of this passage in the \textit{Yogatattvabindu} (Section \uproman{41}, Ed. p. \pageref{ascetics}) reads:

\begin{quote}
  \textit{puruṣasya nṛtyadarśanāt} || \textit{gītaśravaṇāt} || \textit{vallabhavastuno darśaṇāt} || \textit{ya ānanda utpadyate saḥ svargalokaḥ kathyate} | \textit{rogapīḍito durjanebhyaḥ puruṣasya yad duḥkhaṃ utpadyate} | \textit{tad bahutaraṃ narakaṃ kathyate} | \textit{atha ca yatkarmakaraṇāt sarveṣāṃ lokānāṃ svamanasi ca śubhaṃ na bharete tat karma bandhanam ity ucyate} | \textit{atha ca yatkarmakaraṇān manomadhye śaṅkā na bhavati tatkarma muktikāraṇam} | 
\end{quote}
\begin{quote}
Whatever bliss is generated as a result of witnessing dance, listening to songs, [and] viewing beloved objects, that [bliss] is called heaven. The suffering which arises for a person afflicted by disease or by evil persons is considered a very great hell. Moreover, as a result of performing actions that do not bring about happiness in all worlds and one’s mind, it is said that this [very] action is binding. Furthermore, from whatever action within the mind, concern does not arise; that action becomes the cause of liberation.
  \end{quote}

It is probably not possible to extrapolate the complete concept from this hypothetical remnant of Karmayoga. However, it is clear that even though it is not specified as Karmayoga, a path to liberation through specific actions (\textit{karma}s) is laid out here. In the \textit{Yogasvarodaya}, all actions are not supposed to cause worry. In the \textit{Yogatattvabindu}, it is the cultivation of all actions that make one happy and the renunciation of actions that lead to sorrow. At the same time, this passage is another reference to Rāmacandra's wealthy and pleasure-oriented audience. There is also a radical contrast to the ``classical'' Karmayoga of the \citetitle{kaushik1993}. The focus is no longer on the non-attachment towards the action but on actions that bring about happiness.

\subsection{Karmayoga in the \textit{Yogasiddhāntacandrikā}}

Nārāyaṇatīrtha situates his Karmayoga\footnote{See \citeauthor{penna2004} 2004, pp. 67-20 for an earlier discussion of Karmayoga in the \textit{Yogasiddhāntacandrikā}.} in the context of his commentary on \textit{sūtra} 2.28:\footnote{Cf. \textit{Yogasiddhāntacandrikā}, ed. pp. 92-98.}

\begin{quote}
  \textit{yogāṅgānuṣṭhānād aśuddhikṣaye jñānadīptir āvivekakhyāteḥ} || 28 ||
\end{quote}
\begin{quote}
As a result of the practice of the limbs of Yoga upon the destruction of impurities, the lamp of knowledge up to the realisation of discrimination arises. 
\end{quote}

This \textit{sūtra} introduces a description of the eight well-known limbs of Pātañjalayoga. Nārāyaṇatīrtha explains that the practice of the eight limbs leads to the realisation of the overarching goal of Yoga, the discriminating knowledge of \textit{puruṣa} and \textit{prakṛti}, thereby removing ignorance (\textit{vidyā}) and manifesting liberation. He then presents Karmayoga as an alternative to attaining the lamp of knowledge:\footnote{This differentiation inevitably awakens the association with the differentiation of the eight-fold yoga according to Yajñavalkya and the Haṭhayoga with \textit{mudrā}s etc. of Kapila already stated in \citetitle{datta2024} in verse 29}

\begin{quote}
\textit{athavā yogāṅgānāṃ dhautīvastītyādiṣaṭkarmaṇāṃ mahāmudrādīnāṃ ca anuṣṭhānād dṛḍhābhyāsāj jñānadīptiḥ} | \textit{jñāyate 'neneti jñānaṃ karaṇavargaḥ} | \textit{tasya dīptiḥ rogādyanabhighātena tejasvitā dṛḍhatā ca, āvivekakhyāteḥ vivekakhyātiparyantaṃ bhavatīty arthaḥ} | \textit{rogādinā jñānasya kuṇṭhabhāvas tu prasiddha eva} | \textit{sa caiteṣv aṅgeṣv anuṣṭhiteṣu rogapratibandhān na bhavatīty arthaḥ} | \textit{tathā ca karaṇadārḍhyadvārā samādhidārḍhyārthārthakarmayogo 'pi prathamato 'nuṣṭheyo rogabhīruṇeti bhāvaḥ} | \textit{sa ca karmayogaḥ ṣaṭkarmarūpo mudrārūpaś ceti dvividho nirūpita ākare yathā} | 
\end{quote}
\begin{quote}
Alternatively, as a result of executing consistent practice of the limbs of yoga, [particularly] of the six actions like Dhautī, Vastī etc. and the great seal etc., the lamp of knowledge arises. By this [word] ``\textit{jñāna} (knowledge)'', the group of sense organs is understood. Its ``\textit{dīpti} (lamp)'' becomes brilliant and robust without damage through diseases, etc. The meaning of [the word] ``\textit{āvivekakhyāteḥ} (up to the realisation of discrimination)'' extends as far as the realisation of discrimination. Through diseases, etc., the state of the inefficiency of the sense organs (\textit{jñāna}) is thus established. Furthermore, the meaning of ``after having practised these limbs'' is [that] there are no obstacles from diseases. And thus, Karmayoga is the means for acquiring resilience of the sense organs for the steadfastness of \textit{samādhi}, which shall be practised first so that one does not become afraid of disease. And that Karmayoga, having the nature of the six actions and having the nature of the seals is discussed twofold accordingly.
\end{quote}

Next, Nārāyaṇatīrtha simply lists the \textit{ṣatkarma}s and nine \textit{mudrā}s: 
\begin{quote}
\textit{dhāutī vastī tathā neti trāṭakaṃ naulikaṃ tathā} |
\textit{kapālabhātī caitāni ṣaṭ karmāṇi pracakṣate} ||
\textit{karmaṣaṭkam idaṃ gopyaṃ dehaśodhanakārakam iti} |
\textit{mahāmudrā mahābandho mahāvedhaś ca khecarī} ||
\textit{śakticālo mūlabandha uḍḍīyānaṃ tataḥ param} |
\textit{jālandharābhidho yogo viparītakṛtis tatheti} ||
\textit{lakṣaṇāni ca tatraivoktāni} |
\end{quote}
\begin{quote}
  Dhautī, Vastī, as well as Neti, Trāṭaka and Nauli,
  and also Kapālabhāti - these six actions are being told.
  This hexade of action is to be kept secret as it produces the purification of the body.
  The great seal, the great lock, the great piercing and Khecarī,
  the stimulation of the goddess, the root lock, Uḍḍīyāṇa [and] thereafter
  [that] Yoga [practice which is] known as Jālandhara as well as the act of inversion.
  The characteristics are described there [in the following]. 
  \end{quote}

After that, Nārāyaṇatirtha presents verses containing instructive descriptions of every practice borrowed from earlier Yoga texts.\footnote{The section on the \textit{ṣaṭkarma}s is based on \textit{Haṭhapradipikā} 2.24-26, whereas the descriptions of the \textit{mudrā}s are primarily taken from the \textit{Yogacintāmanī} (Ed. p. 132 ff).} Even though Nārāyaṇatīrtha situates the \textit{ṣaṭkarma}s and \textit{mudrā}s within his Karmayoga, at the very end of the section on Karmayoga he notes that they are part of the practice of Haṭhayoga.\footnote{Cf. \textit{Yogasiddhāntacadrikā} (Ed. p. 98): \textit{etac ca sarvaṃ yogāṅgānuṣṭhānāditi sūtre sūtritam api haṭhayogāṅgatvena deha siddhamātraphalatvena sākṣādrājayogā 'naṅgatvāt kaṇṭharaveṇa sūtrakṛtā noktam iti mantavyam iti saṃkṣepaḥ} || 28 ||}

\section{6. Layayoga}
\label{layayogaintro}

Layayoga occupies fifth place in the taxonomy of the \textit{Yogatattvabindu} but is not listed in the verses on the fifteen Yogas of the \textit{Yogasvarodaya}. Ultimately, however, the description of Layayoga is missing in both texts. In the taxonomy of the \textit{Yogasiddhāntcandrikā}, Layayoga is in thirteenth place. In Sundardā's \textit{Sarvāṅgayogapradīpikā}, it is in third place and is assigned to the first of three superordinate categories, namely Bhaktiyoga.

\subsection{Layayoga in the \textit{Yogasiddhāntacandrikā}}

Nārāyaṇatīrtha places his discussion of Layayoga\footnote{For an earlier discussion see \citeauthor{penna2004} 2004, pp. 85-89.} in the context of his commentary of \textit{sūtra} 1.41:\footnote{\textit{Yogasiddhāntacandrikā} Ed. p. 64.} 

\begin{quote}
\textit{samprajñātasya viṣayaṃ pradarśayan na samprajñātāpararyāyaṃ layayogam āha}-\\
\textit{kṣīṇavṛtter abhijātasyeva maṇer grahītṛgrahaṇagrāhyeṣu tatsthatadañjanatā samāpattiḥ} || 41 ||
\end{quote}
\begin{quote}
Pointing out the object of [the] \textit{saṃprajñāta}[-type of \textit{samādhi}], it is said that Layayoga is for nothing other than [the] \textit{saṃprajñāta}[-type of \textit{samādhi}] - 

\textit{Samāpatti}, the state of complete absorption of the mind when it is devoid of its mental fluctuations, happens when the mind becomes like a transparent jewel that takes the form of the object placed before it, whether it is the knower, the instrument of knowing or that which is to be known.
\end{quote}

After the previous \textit{sūtra}s introduced various objects that can support the mind in meditation, this \textit{sūtra} now continues the analysis of different stages within the state of meditation, regardless of its object.\footnote{This analysis already began in \textit{Pātañjalayogaśāstra} I.17.} When the \textit{vṛtti}s of the mind fade, the mind becomes more and more like a crystal (\textit{maṇi}). Just as a crystal takes on the colouring (\textit{añjanatā}) of any object placed in front of it, the clear mind focusing on any object also takes on the colouring of that very object. \footnote{\textit{Yogasiddhāntacandrikā} 1.34 (Ed. p. 64): \textit{uparāgeṇa tadākāratāyāṃ dṛṣṭāntam āha- abhijātasyeva maṇer iti} | \textit{nirmalasya sphaṭikāder yathā japākusumādy uparāgeṇa raktādyākāratā tathety arthaḥ} |} With regard to the objects that serve absorption, the \textit{sūtra} specifies here the hierarchical sequence of the knower (\textit{grahītṛ}), the instrument of knowledge (\textit{grahaṇa}) and that what is to be known (\textit{grahyā}). For Nārāyaṇatīrtha, the knower is \textit{puruṣa}. The instrument of knowledge is the sense organs, and what is to be known is the object that can be grasped by the mind.\footnote{Ibid. 1.34 (Ed. p. 64): \textit{kṣīṇavṛtter iti} | \textit{abhyāsavairāgyābhyām apagamavṛttyantarasya cittasya grahītṛgrahaṇagrāhyeṣu, grahītā puruṣaḥ sthūlasūkṣmabhedena, grahaṇaṃ gṛhyate 'rtho 'nenetīndriyam, evaṃ grāhyaṃ ca grahītṛgrahaṇagrāhyāni} |} Depending on which object the mind focuses on, it takes on its colour and nature. The term \textit{samāpatti} refers to the complete identification of the mind with the object of meditation. Nārāyaṇatīrtha (ed. p. 64) then equates the term \textit{samāpatti} with \textit{laya}:

\begin{quote}
\textit{teṣu yā tatsthatadañjanatā tatsthena uparāgeṇa tadañjanatā tanmayatā samyak tadākāratā samāpattiḥ samyagāpattir layaḥ samprajñātalakṣaṇo yogo bhavatīty arthaḥ} |
\end{quote}
\begin{quote}
In those [objects] which are ``coloured by that which resides there'', by colouring, that [state of] colouration, being absorbed in it, thoroughly being in the state of that form, is absorption (\textit{samāpatti}), the total entering into [that] state is Laya, being a Yoga characterized \textit{samprajñāta}. This is the meaning.
\end{quote}

For Nārāyaṇatīrtha, Layayoga is therefore a synonym for the state of \textit{samāpatti} and is attributed to the \textit{samprajñāta} form of \textit{samādhi}, in which the consciousness is still focussed on one of the aforementioned objects. \textit{Samprajñātasamādhi} is also known as `\textit{samādhi} with discrimination', as the meditator retains awareness of the distinction between the meditator, the meditation object and the process of meditation itself. It is therefore a \textit{samādhi} in which there is still a minimal remainder of \textit{vṛtti}s, in contrast to the final \textit{asaṃprajñāta} form of \textit{samādhi} in which the last \textit{vṛtti} also expires and final liberation and \textit{kaivalya} occur.\footnote{See \citetitle{yogastura} 1.17-22 for more detailed explanations of the \textit{samprajñāta} and \textit{asaṃprajñāta} forms of \textit{samādhi}.}

\subsection{Layayoga in the \textit{Sarvāṅgayogapradīpikā}}

For Sundardās, Layayoga (2.28-39) is a subcategory of Bhaktiyoga,\footnote{A description of Layayoga in French can be found in \citeauthor{burger2014sarvangayogapradipika} 2014, pp. 693-94.}\footnote{?????Reference to Bhaktiyoga chapter!} and recognises it as a method for the liberation from the cycle of birth and death.\footnote{Cf. \textit{Sarvāṅgayogapradīpikā} 2.28c: \textit{laya binu janma marana nahīṃ chūṭai} |} Sundardās emphasises that Layayoga is an incomparable method and therefore attaches great importance to it among the Yoga methods he presents.\footnote{Cf. ibid. 2.29a: \textit{laya samāna nahīṃ aura upāī} |} Layayoga dispels all illusion,\footnote{Cf. Ibid. 2.29c: \textit{āvāgamana sakala bhrama bhāgai} || 29 ||} makes one attain the highest state,\footnote{Cf. ibid. 2.30d: \textit{parama sthāna samāvai soī} || 30 ||} dispels anger and difficulties,\footnote{Cf. ibid. 2.32cd: \textit{esī laya jo koī lāvai} | \textit{jonī saṃkaṭa bahuri na āvai} || 32 ||} and makes one equal to Brahman.\footnote{Cf. Ibid. 2.31a: \textit{yaha laya yoga anupa hai karai brahma samāna} |} The main emphasis of the practice is the continuous absorption of the mind into a specific goal, which he defines as Rāma\footnote{Cf. Ibid. 2.29b: \textit{jo jana rahai rāma laya lāī} |} or Hari.\footnote{Cf. Ibid. 2.38ab: \textit{sa saṃprakāra hari sauṃ lavai} | \textit{koī videha parama pada pāvai} |} This absorption is supposed to be continued throughout day and night.\footnote{Cf. ibid. 2.29c: \textit{niśi vāsara esaiṃ lai lāgai} |} To illustrate how exactly this practice is to be carried out, he draws various comparisons. For example, \textit{Sarvāṅgayogapradīpikā} reads 2.35: 

\begin{quote}
\textit{jaisaiṃ gāu jaṃgala kauṃ dhāvai} | \textit{pānī pivai ghāsa cari āvai} |\\ 
\textit{citta rahai bacharā kai pāsā} | \textit{aisī laya lāvai haridāsā} || 2.35 ||
\end{quote}
\begin{quote}
  Just as a cow walks towards the forest, drinks water, and grazes, but its mind remains near the calf, in such a way, Haridāsā practices Laya.
\end{quote}

Another example is \textit{Sarvāṅgayogapradīpikā} 2.35:

\begin{quote}
\textit{jyauṃ jananī gṛha kāja karāī} | \textit{putra piṃghrau pauḍhata bhāī} |\\
\textit{ura apnai taiṃ kṣaṇ na na bisārai} | \textit{aisī laya jana kauṃ nistārai} || 36 ||
\end{quote}
\begin{quote}
Just as a mother does the housework while her son plays or crawls nearby and never for a moment forgets him in her heart, Laya liberates the person who practices it.
\end{quote}

These comparisons illustrate Sundardā's concept of Layayoga. Layayoga is the continuous absorption or centring of the mind on Rāma or Hari while performing the necessary daily activities. The examples of the cow and the mother emphasise that this is supposed to be done in a way that resembles the tireless love and attention of a mother towards her child.

\section{7. Dhyānayoga}
\label{dhyānayogaintro}

Rāmacandra positions Dhyānayoga at the seventh place in his taxonomy of fifteen Yogas. In the \textit{Yogasvarodaya}, Dhyānayoga is to be found at the fifth position. In both cases, Dhyānayoga as a single subcategory of Rājayoga is not discussed explicitly in the remainder of the text. In the \textit{Yogasiddhāntacandrikā}, it is in the fourteenth position. Sundardās, in his taxonomy of the three Yoga tetrads of the \textit{Sarvāṅgayogapradīpikā}, does not list Dhyānayoga at all.

Thus, the only explicit description of Dhyānayoga within the texts of the complex Yoga taxonomies occurs only in the \textit{Yogasiddhāntacandrikā}. However, this description parallels various contents of the \textit{Yogatattvabindu} and \textit{Yogasvarodaya}.   

\subsection{Dhyānayoga in the \textit{Yogasiddhāntacandrikā}}

Nārāyaṇatīrtha situates Dhyānayoga in the context of his comparatively extensive commentary on \textit{sūtra} 1.39:\footnote{Cf. \emph{Yogasiddhāntacandrikā} ed. p. 56-63.}

\begin{quote}
\textit{dhyānayogam āha} - 
\textit{yathā 'bhimatadhyānād vā} || 39 ||
\end{quote}
\begin{quote}
Dhyānayoga, is said to be [the following]:
 Or, as a result of meditation on what one favours.
\end{quote}

Below, Nārāyaṇtīrtha's commentary offers two alternative explanations of Dhyānayoga. The first explanation is presented briefly and reads as follows: 

\begin{quote}
  \textit{yatheti} | \textit{kim bahunā, harirāmādirūpaṃ parameśvaraṃ bāhyaṃ candrasūryādijyotir vā yad eveṣṭaṃ tad eva dhyāyet} | \textit{tasmād api dhyānāl labdhasthitikasya cittasya sādhanāntaraṃ vināpi kevale paramātmani sthitau yogyatā bhavatīty arthaḥ} | \textit{ayam eva dhyānayoga ukto yogagrantheṣu} |\\
  
  \textit{vinā deśādibandhena vṛttir yā 'bhimate sthirā} |\\
  \textit{dhyānayogo bhaved eva cittacāñcalyanāśakaḥ} ||\\
    \textit{ity ādinā} | 
\end{quote}

\begin{quote}
  [Regarding the term] ``yathā'' - Why [say] more? One should meditate on the supreme lord in the form of Hari, Rāma, etc., or on an external light such as the moon, sun, etc. [or] just to what is favored. Because of that, as a result of meditation alone, the stability of the mind is attained without the need for any other means, enabling one to reside in the supreme self. This is the meaning. This very Dhyānayoga is taught in the texts of Yoga; [for example] in quotations such as: \\

  Without being confined by place, etc., the fluctuations of the mind become stable in the preferred [object]. In fact, Dhyānayoga is the destroyer of the fickleness of the mind.\footnote{I am yet to identify the source of this \textit{śloka}.}\\
  
\end{quote}

The first model refers to the meditation of primarily to certain external objects in general, which leads to the reduction of fluctuations in the mind. 
The second model, on the other hand, is described in the following sentences and then explained in detail:

\begin{quote}
\textit{yad vā yathābhimatānāṃ tīrthadevalokavarṇatattvādīnāṃ yathābhimateṣu svadehādiṣu dhyānād bhāvanāviśeṣān manasaḥ sthitir bhavatīty arthaḥ} | \textit{tatra yady api brahmavido brahmamayatvādinā sarvam eva tīrthaṃ pratilomakūpaṃ ca tīrthāni bhavantīti tathāpi yuñjānena cittaśuddhy arthaṃ prathamatas tīrthādikam avaśyaṃ bhāvanīyam} |
\end{quote}
\begin{quote}
 Alternatively, that stability of the mind arises from a specific application of meditation onto favoured [objects] like, for example, sacred sites, deities, worlds, letters, principles, etc., with regard to favoured locations within one's own body. In that case, it is stated, although the knowers of Brahman assert that because of the pervasiveness of Brahman, everything indeed is a sacred place, and even the pores of the skin become places of pilgrimage. Nevertheless, the yogin (\textit{yuñjāna}) who is aiming at the purification of the mind, must inevitably contemplate sacred places, etc. in the beginning [of pracitce].   
  \end{quote}

Nārāyaṇatīrtha differentiates an alternative that is aimed particularly at beginners in meditation practice. Nārāyaṇatīrtha devotes the rest of his commentary on \textit{sūtra} 1.39 to this type of meditation, which is aimed at objects located inside the body. He first specifies \textit{tīrthabhāvanā},\footnote{Cf. \emph{Yogasiddhāntacandrikā} Ed. p. 57-59} the meditation on sacred places, in which the practitioner is supposed to meditate on various sacred places of India in different body parts. He then specifies \textit{devabhāvanā},\footnote{Cf. Ibid. Ed. p. 59.} the meditation of different deities, which are located in body parts, and \textit{lokabhāvanā},\footnote{Cf. Ibid. Ed. p. 59.} the meditation on the worlds in the body and \textit{varṇabhāvanā},\footnote{Cf. Ibid. Ed. p. 59.} the meditation on letters in the body, each placed in one of six \textit{cakra}s.\footnote{Cf. Ibid. Ed. p. 59-61}. Then \textit{tattvabhāvana}, the meditation on the principles, is described.\footnote{Cf. Ibid. Ed. p. 61-63} The commentary concludes by discussing manipulating air currents through the nostrils for beneficial results, such as in heat or cold exposure, intercourse, travelling, etc. A useful summary of the details of this part of Nārāyaṇatīrtha's commentary has already been provided by \citeauthor{penna} (2004: 91-97) and does not need to be repeated here.

\subsection{Dhyāna in the \textit{Yogatattvabindu} and \textit{Yogasvarodaya}}

Dhyānayoga is mentioned in the taxonomies of both texts\footnote{The list of mentions of \textit{dhyāna} is based on the sections of the \textit{Yogatattvabindu}. The corresponding passages of the \textit{Yogasvarodaya} can be taken from the critical apparatus of the present edition of the text.} but is does treated as an individual topic. However, various \textit{dhyāna}s can be found throughout the text. The first mention of \textit{dhyāna} occurs in the context of nine \textit{cakra}s in the sections \uproman{4}-\uproman{12}. Rāmacandra and the unknown author of the \textit{Yogasvarodaya} instruct \textit{dhyāna} on the respective \textit{cakra}, or a \textit{mūrti} located in the respective \textit{cakra}. The scribe-author of manuscript \getsiglum{U2} even adds more precise instructions on the duration of the meditations on the respective \textit{cakra}s. However, as we discover in section \uproman{3}, this meditation practice is attributed to Siddhakuṇḍalinīyoga or Mantrayoga and not to Dhyānayoga. We also encounter the term \textit{dhyāna} in the description of \textit{adholakṣya} in section \uproman{15}, in the second subtype of Haṭhayoga in section \uproman{20}, in the description of \textit{bāhylākṣya} in section \uproman{23}, as well as within \textit{antaralakṣya} in section \uproman{24}. Another mention can be detected within the list and the eight limbs of \textit{aṣṭāṅgayoga} in section \uproman{31}. Here, Rāmacandra states that \textit{dhyāna} will not be discussed, as this has happened many times before.\footnote{\textit{dhyānaṃ ca bahutaraṃ prāg uktaṃ tenātra cocyate} |} In \uproman{32}-\uproman{41} the identity of the external universe with the body is taught. Various contents, such as the fourteen worlds, mountains and rivers, etc., are located in the body, similar to the \textit{Yogasiddhāntacandrikā}. However, Rāmacandra does not specify a concrete reason for listing these physical equivalents of the external universe in the body. The same is true for the parallel passages of \textit{Yogasvarodaya} and \textit{Siddhasiddhāntapaddhati}. In section \uproman{48}, in the context of the divisions of the lotus in the heart, meditation on this heart lotus is precribed. This meditation is supposed to lead to the illumination of the self and enhance vitality. Therefore, I conclude that although Dhyānayoga is not provided with its own section in either text, it is at least implicitly present in both texts and the generic term of meditation (\textit{dhyāna}) is nevertheless a central theme. 

\section{8. Mantrayoga}
\label{dhyānayogaintro}

Mantrayoga occupies the eighth position in the taxonomy of the \textit{Yogatattvabindu}, the sixth position in the \textit{Yogasvarodaya}, the fifth position in the \textit{Yogasiddhāntacandrikā} and is in the second place of the twelve yogas of the \textit{sarvāṅgayogapradīpikā}. Sundardās attributes Mantrayoga to Bhaktiyoga.

\subsection{Mantrayoga in the \textit{Yogatattvabindu} and \textit{Yogasvarodaya}}

Apart from the mention of Mantrayoga in the first verses of the quotations of the \textit{Yogasvarodaya} in the \textit{Prāṇatoṣinī}\footnote{Cf. \emph{Prāṇatoṣinī} ed. p. 831 quoted with reference \textit{yogasvarodaye}.} the quotations we have at hand contain neither a description of Mantrayoga nor a description of a Yoga practice that includes \textit{mantra}s. In the \textit{Yogatattvabindu}, however, the term Mantrayoga appears again in section \uproman{3}:

\begin{quote}
  \textit{idānīṃ rājayogasya bhedāḥ kathyante} | \textit{ke te} | \textit{ekaḥ siddhakuṇḍalinīyogaḥ mantrayogaḥ amū rājayogau kathyete} |
\end{quote}
\begin{quote}
  Now, varieties of Rājayoga are described. Which are these? One is Siddhakuṇḍalinīyoga and one is Mantrayoga. These two Rājayogas are described [in the following].
\end{quote}

This is followed by an explanation of the three primary channels of the yogic body: Iḍā, Piṅgalā and Suṣumnā. The section concludes with the assertion that the practitioner becomes omniscient once knowledge about the central channel is generated. In the following sections (\uproman{4}-\uproman{12}), a system consisting of a total of nine \textit{cakra}s is then described.

This passage is problematic from a text-critical perspective. Rāmancandra is very much orientated towards his textual source, the \textit{Yogasvarodaya}, in terms of structure and content, particularly in the first half of his text and mainly in the second half. However, the \textit{Yogasvarodaya} specifies \textit{jñānayoga} instead of \textit{siddhakuṇḍalinīyoga mantrayogaḥ}. As usual, the remainder of the section is very similar in content to the \textit{Yogasvarodaya}. However, the manuscripts offer no alternatives for the conspicuous passage, so the text must be accepted for now. Another reason is the seemingly strange sentence construction, which is ultimately unsurprising if one knows the rest of the text and can be accepted. Right after the term \textit{mantrayogaḥ}, the reader would have wished for a \textit{ca} (``and''). Only the manuscript \getsiglum{L} omits the term \textit{mantrayogaḥ} but preserves the following dual forms, so this is not a solution either.    

The first \textit{cakra} named \textit{mūlacakra} is provided with the following introduction:

\begin{quote}
  \textit{idānīṃ suṣumṇāyāḥ jñānotpattāv upāyāḥ kathyante} | \textit{ādau caturdalaṃ mūlacakraṃ vartate} | 
\end{quote}
\begin{quote}
  Now, the means for the genesis of knowledge of the central channel is described. At the beginning [of the central channel] exists the four-petalled root-cakra.
  \end{quote}

  On the basis of this description, it can only be assumed that the sections \uproman{4}-\uproman{12} describing the nine \textit{cakra}s are assigned by Rāmacandra to Siddhakuṇḍalinīyoga and Mantrayoga. However, almost all manuscripts, with the exception of the \getsiglum{U2} manuscript, do not allow any conclusions to be drawn in this context about a practice that could be described as Mantrayoga.

  However, the manuscript \getsiglum{U2} contains detailed additional passages that solve the problem and supplement a practice that can be described as Mantrayoga. For each \textit{cakra}, all manuscripts instruct \textit{dhyāna} on the respective \textit{cakra}. Manuscript \getsiglum{U2}, in addition to various additional details, always contains an indication of the duration of the meditation, which is measured in \textit{ajapājapa}s (``The recitations of the non-recited.'').\footnote{The \textit{cakra}s additionally receive the same time indication measured in \textit{ghaṭi}s, \textit{pala}s and \textit{akṣara}s. See \citeauthor{birch2013} 2013: 265, n. 46}. Finally, the additional material in section \uproman{11} of manuscript \getsiglum{U2} makes it clear that the so-called \textit{ajapā mantra} or \textit{haṃsa mantra} must be meant here:\footnote{Probably first taught in the Yoga literature in \citetitle{vivekamartandaolda} 28-30}.
    
  \begin{quote}
    \textit{sakāreṇa bahir yāti hakāreṇa viśet punaḥ} |\\  
    \textit{haṃsaḥ so 'haṃ tato mantraṃ jīvo japati sarvadā} ||
    \end{quote}
  \begin{quote}
With the sound ``sa'', he exhales. With the sound ``ha'', he inhales again: ``I am he, he is I''. Because of that, the embodied soul constantly utters the Mantra.
\end{quote}

The \textit{ajapā mantra} (``unmuttered mantra'') consists of the two syllables \textit{haṃ} and \textit{saḥ} according to the phonological association with the sound of inhalation and exhalation. Because all living beings inhale and exhale, they recite the \textit{ajapā mantra} continuously day and night. At the same time, \textit{haṃsa}, best translated as "swan" or "goose" in English, is a famous and ancient metaphor for the soul travelling through the wheel of Brahman or Saṃsāra.\footnote{See \citetitle{hauschild1927} 1.6 and 3.18.} Sometimes this mantra is also specified as \textit{ajapā gāyatrī}.\footnote{The \textit{ajapā} can be seen as a yogic appropriation of the Vedic \textit{gāyatrīmantra} (\citetitle{rootsofyoga2017} 2017, 134).}  

Manuscript \getsiglum{U2} explains that the total daily number of all silent recitations of the \textit{haṃsa mantra} is 21600.\footnote{The number of total breaths is based on the assumption of an average breath duration of four seconds. Each day has 86400 seconds. If one divides this total number by four, one gets the 21600 breaths of the \textit{ajapā mantra}. \citeauthor{birch2013} (2013, 265, n. 46) argues that this assumption comes from \citetitle{svacchandatantra} 7.54-55. In addition to the \getsiglum{U2} manuscript of \textit{Yogatattvabindu}, this yogic axiom is widely used in Sanskrit Yoga literature. See for example \citeauthor{amaraughaprabodha} 58, Hemacandra's \citetitle{hemacandras} 5.232, \textit{Vivekamārtaṇḍa} 46, \textit{Gheraṇḍasaṃhitā} 5.79, \textit{Dhyānabindūpaniṣad} 62ab-63ab or \citetitle{jogpradipyaka} 913.} The association of the term Mantrayoga with the practice of \textit{haṃsa mantra} is widespread in Sanskrit Yoga literature.\footnote{See e.g. \citetitle{yogabija} 147; \citetitle{shivayogapradipika} 2.26-27 and 2. 29-32 (\citeauthor{shivayogapradipikax} (2023: 205), explains that here, however ``mantra is reframed and interiorised within a \textit{prāṇāyāma} environment, specifically in the form of the \textit{ajapā}, the "unuttered" mantra''); \emph{yogacintamani} (Ed. p. 12); \citetitle{hathatattvakaumudi} 55.28; and \citetitle{yogasikhopanisad} 132.}
%%maybe incorporate ŚYP : (2.34) Having made so ’ham one’s personal mantra—in which the two syllables are expressed as one's self and the Supreme—[the yogin] should take away the two consonants and refashion it as the divine mantra oṃ. Having joined it with the nasal sound (anusvāra), it is the best of all mantras. He who leads it to the brahmanāḍī (i.e. suṣumṇā) is full of bliss, [even if] deprived of the experience of Kuṇḍalinī. He attains release from [all] karma. 

From a text-critical perspective, there is ambivalent evidence regarding the authenticity of the passages under discussion. All manuscripts mention Mantrayoga in the above passage. We must, therefore, assume that Mantrayoga was originally and perhaps even deliberately specified here by Rāmacandra, even if, or precisely because, he reads the source text differently. The fact that only the manuscript \getsiglum{U2} explicitly teaches a Mantrayoga must make one suspicious. This manuscript only contains additional material in the sections \uproman{4}-\uproman{12}. The most likely scenario is that the scribe of the manuscript \getsiglum{U2} made these additions to provide the missing explanations on Mantrayoga.\footnote{The connection between Siddhakuṇḍalinīyoga and Mantrayoga established in \getsiglum{U2} is found in a similar form in \citetitle{sarada} 25.37ab: `The \textit{kuṇḍalī} Śakti abides in the \textit{haṃsaḥ} [and] supports the [individual] Self.' (\textit{bibharti kuṇḍalī śaktir ātmānaṃ haṃsaṃ āśritā} |), see \citeauthor[2011: pp. 218, 228]{saradatilakafull}.} Manuscript \getsiglum{U2} belongs to the \beta group of manuscripts, which often contains poorer readings in a large part of the text than the \alpha group with the oldest manuscript \getsiglum{N1}. This also makes the other scenario seem far less likely at first, namely that \getsiglum{U2}, despite its later dating, transmits a more original text than all other textual witnesses. However, the oldest manuscript \getsiglum{N1} has immense gaps, at least in the last third of the text. On the other hand, manuscript \getsiglum{U2} is complete here, together with some candidates of the \beta-group. Furthermore, only manuscript \getsiglum{U2} preserves the correct variant of the sentence \begin{quote} \textit{bhuktimuktidā śivarūpiṇī suṣumṇānāḍī pravartate} | \textit{asyā jñānotpattau satyāṃ puruṣaḥ sarvajño bhavati} | \end{quote} in section \uproman{3}. Therefore, the additions of \getsiglum{U2} were printed in greyscale in the edition and not relegated to a footnote.

\subsection{Mantrayoga in the \textit{Yogasiddhāntacandrikā}}
\label{mantrayogaintrocandrika}
Nārāyaṇatīrtha locates Mantrayoga, like Jñānayoga before it, in the context of \textit{sūtra} 1.28. This \textit{sūtra} and the corresponding commentary by Nārāyatīrtha have already been discussed in the chapter on Jñānayoga in the \textit{Yogasiddhāntacandrikā} (p.\pageref{jnanayogaintrocandrika} et seqq.) and therefore need not be repeated here.\footnote{For another discussion of Mantrayoga in the \textit{Yogasiddhāntacandrikā} see \citeauthor{penna2004} 2004, pp. 71-76.} Mantrayoga in the \textit{Yogasiddhāntacandrikā} is \textit{japa} (``low-voice muttering'') of \textit{praṇava} (``sacred syllable \textit{auṃ}''), which can be performed in two alternative ways, as Jñānayoga\footnote{I discuss the concept of Jñānayoga in the \textit{Yogasiddhāntacandrikā} on p. \pageref{jnanayogaintrocandrika}.} or Advaitayoga.\footnote{The concept of Advaitayoga in the \textit{Yogasiddhāntacandrikā} I discuss on p.\pageref{advaitayogaintrocandrika}.}      
%\begin{quote}
%\textit{taj japas tadarthabhāvanam} || 28 ||
%\end{quote}
%\begin{quote}
%Its low-voice muttering; contemplation of its meaning. || 28 ||
%\end{quote}
%Dieses \textit{sūtra} gehört zu den \textit{sūtra}s des \textit{Pātañjalayogaśāstra}, welches verschiedene Methoden erläutern \textit{samādhi} zu erlangen. \textit{Sūtra} 1.28 gehört zu einer Gruppe von \textit{sūtra}s (1.23-1.28), welche als ``hingebungsvolle Verehrung des höchsten Gottes'' (\textit{īśvarapraṇidhāna}) bezeichnet wird. Während 1.24-26 das Konzept von \textit{īśvara} erläutern, erklärt 1.27 das dessen Bezeichnung (\textit{vācaka}) die Silbe \textit{auṃ} ist, die hier mit \textit{praṇava} bezeichnet wird. 
%Nārāyaṇatīrtha erklärt zu 1.28, dass sich \textit{taj japas} (``its low-voice muttering'') auf die stille Rezitation von \textit{praṇava} bezieht. Gleichzeitig solle dessen Bedeutung, nämlich \textit{īśvara} kontempliert werden. Hierfür richtet der Übende dieser Kontemplation richtet seinen Geist auf das höchste Selbst, also \textit{īśvara}, welches mit unbegreiflich oberherrlicher Macht versehen ist. Diese Kontemplation führe zur unterscheidenen Erkenntnis zwischen der Urnatur (\textit{prakṛti}, dessen Effekten\footnote{Für eine Übersicht siehe \citeauthor{bryant2009}, 2009, pp. xlvii-liii.} und dem Selbst (\textit{puruṣa}.\footnote{\textit{Yogasiddhāntacandrikā} (Ed. p. 45): \textit{taj japa iti} | \textit{tasya praṇavasya japaḥ vidhivaduccāraṇaṃ, tadarthasya praṇavārthasya acintyaiśvaryaśaktiyuktasya paramātmano bhāvanaṃ prakṛtitatkāryapuruṣebhyo vivekenānusaṃdhānam} |} Insbesondere soll der Übende das eigene Selbst mit dem höchsten Brahman\footnote{Hier explizit als das höchste Selbst (\textit{puruṣa}) gedacht.} identifizieren, so wie unteilbare Natur von einer Qualität von dessen Substanz. Bei der kontemplativen Rezitation von \textit{praṇava} wrid eben dieses in drei Teile geteilt, denen je eine Bedeutung zugewiesen wird. Das \textit{a} (\textit{akāra} steht für das eigene Selbst, das \textit{m} (\textit{makāra}) für Brahman (das höchste Selbst) und \textit{u} (\textit{ukāra}) für Zweifelslosigkeit (\textit{avicikitsita}).\footnote{Ibid. (Ed. p. 45): \textit{tam etam ātmānaṃ ūm̐mati brahmaṇaikīkṛtya brahma vātmanomityekīkṛtya} | \textit{iti śruteḥ} | \textit{vastutas tu} - \textit{akāreṇa mamātmānamanviṣya makāreṇa brahmaṇā 'nusaṃdadhyāt ukāreṇā 'vicikitsitaḥ} | ity ādiśrutes tadarthasya jīvaparamātmanor abhedasyaitanmate dravyaguṇayor ivāvinābhāvarūpasya bhāvena cintanamityarthaḥ | japapūrvakaṃ bhāvanaṃ karttavyam |}                                                    
%
%Am Ende des Kommentares zu 1.28 baut Nārayaṇatīrtha dann Assoziationen mit insgesamt dreien seiner fünfzehn Yogas auf, unter anderem auch das hier im Fokus stehende Mantrayoga:
% \begin{quote}
%\textit{kiñ ca japa ity anena mantrayogaḥ arthabhāvanam ity anena vivekajñānā 'bhyāsarūpo jñānayogaḥ abhedabhāvarūpo 'dvaitayogaś ca saṃgṛhītaḥ} |
%\end{quote}
%\begin{quote}
%Furthermore, by \textit{japa} (``low-voice muttering''), Mantrayoga is implied; by \textit{arthabhāvanam} (``contemplation of its meaning,'') Jñānayoga is implied, which is characterized by the practice of discriminative knowledge; and [additionally] Advaitayoga is implied, which has the nature of non-difference.
%\end{quote}

\subsection{Mantrayoga in the \textit{Sarvāṅgayogapradīpikā}}
\label{mantrayogaintrosarva}

Sundardās introduces his remarks with the question of how the formless and featureless highest reality can be named.\footnote{\textit{Sarvāṅgayogapradīpikā} 2.16cd: \textit{jākai kachū rūpa nahiṃ reṣā} \textit{kauna prakāra jāī so deṣā} || 16 ||} For without giving it a name, one cannot refer to it.\footnote{Ibid. 2.17b: \textit{nāma binā nahiṃ lagai piyārā} |} A personal surrender, a devotion to the highest reality, is the basic prerequisite for Bhaktiyoga, the superordinate category of Sundardā's Mantrayoga. The best, or verbatim the crown of all names for the highest reality, is \textit{rāma}.\footnote{Ibid. 2.19cd: \textit{rāma mantra sabakai siramaurā} \textit{tāhi na koī pūjata aurā} || 19 ||} After verses of praise of the \textit{rāma mantra} Sundardās explains that the \textit{rāma mantra} has to be learnt from the Guru. At the beginning of Mantrayoga practice, one is supposed to recite the \textit{rāma mantra} with the tongue, i.e. audibly.\footnote{Ibid. 2.23cd: \textit{prathama ..vana suni guru kai pāsā} \textit{puni so rasanā karat abhyāsā} || 23 ||} In the course of the practice, the \textit{rāma mantra} is then supposed to be recited mentally, constantly, day and night, in order to unite the practitioner with the omnipresent highest reality:

\begin{quote}
\textit{..pīchai hiradai maiṃ dhārai} | \textit{jihvā rahita maṃtra uccārai} |\\ 
\textit{niśa dina mana tāsauṃ raha lāgau} | \textit{kabahūṃ naiṃka na ṭūṭai dhāgau} || 24 ||\\
\textit{puni tahāṃ pragaṭa hoī raṃkārā} | \textit{āpuhi āpu akhaṇḍita dhārā} |\\
\textit{tana mana bisari jāī tahāṃ soī} | \textit{romahi roma rāma dhuni hoī} || 25 ||\\
\end{quote}
\begin{quote}
(24) Afterwards, retain it [the mantra] in the heart; recite the mantra without the tongue.
Night and day, let your mind stay attached to it; may the thread never break.\\
(25) Then there, the omnipresent one manifests; oneself becomes an unbroken stream.
Body and mind forgotten there, in that state; in every hair, the sound of Rāma resonates.
\end{quote}

Thus, Mantrayoga in \textit{Sarvāṅgayogapradīpikā} is a form of Bhaktiyoga that seeks union with the highest reality in the form of devotional recitation of the \textit{rāma mantra}. 

\section{9. Lakṣyayoga}
\label{laksyayogaintro}

Lakṣyayoga is one of the most voluminous and most important topics\footnote{In the \textit{Śivayogapradīpikā} 1.8, the one who has attained the realisation of Brahman using the (in this case) three \textit{lakṣya}s is called a knower of Rājayoga. In this text, the practice of \textit{lakṣya}s is the primary characteristic practice of Rājayoga. In addition, being free from mental fluctuation through gnosis is specified as the second characteristic practice. (\textit{triṣu laṣyeṣu yo brahmasākṣātkāraṃ gamiṣyati} | \textit{jñāne vātha manovṛttirahito rājayogavit} || 1.8 ||} in the \textit{Yogatattvabindu}.\footnote{Cf. \textit{Yogatattvabindu} sections \uproman{13} (overview of the five \textit{lakṣya}s), \uproman{14} (\textit{adholakṣya}), \uproman{15} (\textit{ūrdhvalakṣya}), \uproman{23} (\textit{bāhyalakṣya}), \uproman{24} (\textit{antaralakṣya}) and \uproman{27} (\textit{madhyalakṣya}) of the \textit{Yogatattvabindu} deal exclusively with the types of Lakṣyayoga.} The concept of this type of Yoga has a complex history of reception, and its origins as a category of specific Yoga techniques can be traced far back into early Tantric texts.\footnote{The yoga practice of \textit{lakṣya}s derives from an ancient Śaiva paradigm. The exact roots of this paradigm are difficult to reconstruct precisely. In many cases, the \textit{lakṣya}s are taught together with a system of six to nine \textit{cakra}s, sixteen \textit{ādhāra}s and five \textit{vyoma}s, \textit{ākāśa}s or \textit{kha}s. In most texts that take up this paradigm, there is a variant of a verse also contained in the \textit{Yogatattvabindu}, which lists the elements just mentioned as essential components of Yoga. See \textit{Yogatattvabindu} section \uproman{28}.1 for the verse and its variants in other contemporary and earlier texts. Perhaps the oldest datable textual evidence for the practice of yogic \textit{lakṣya}s can be found in \textit{Netratantra} 7.1-2, which was composed between 700-850 CE, cf. \citeauthor{sanderson2004} 2004, p. 243. However, here, the \textit{lakṣya}s are only listed and not further explained, so we can assume that this practice is probably older than the \textit{Netratantra} itself. Kṣemarāja, in his \textit{Netroddyota} commentary, further elaborates on the three \textit{lakṣya}s. He briefly states: \textit{trīṇy antarbahirubhayarūpāṇi lakṣyāṇi lakṣaṇīyāni yatra} | \textit{nirāvaraṇarūpatvāt ``khamanantaṃ tu janmākhyaṃ''} \textit{Netratantra} (7.27). `The three foci, internal, external or both, are to be attained, and because they are unobstructed, ``The endless void is called the birth''. Furthermore, the \textit{lakṣya}s are no longer mentioned directly in the text. However, the \textit{Netratantra} in 8.39-44 seems to refer to the techniques of the \textit{lakṣya}s. At this passage of the text, the yogin has already reached \textit{samādhi}. In this state, he is instructed not to direct his meditation towards various foci anymore. The descriptions of the foci negated here sound very similar to the descriptions of the three to five \textit{lakṣya}s of the late medieval texts of the complex Yoga taxonomies. For example, \textit{Netratantra} 8.42 explains: \textit{nāntaḥ śarīrasaṃsthāne na bāhye bhāvayet kvacit} | \textit{nākāśe bandhayel lakṣyaṃ nādho dṛṣṭiṃ niveśayet} || 42 ||. `One should not contemplate any place of the body inside or outside. One should not fix one's attention towards the sky (open space), nor should one direct one's gaze downwards.' Instead, the yogin should abandon everything and focus the mind on the supreme alone and in isolation". Cf. \textit{Netratantra} 8.44cd.

The \textit{Mālinīviyajottaratantra} (12.9) and other linked Tantras (e.g. \textit{Kiraṇatantra} 2.22-23 and \textit{Dīkṣottara} 2.2-3.) also contain a system of \textit{lakṣya}s. In the \textit{Mālinīviyajottaratantra}, there are six \textit{lakṣya}s. These six \textit{lakṣya}s are labelled as follows: 1. emptiness (\textit{vyoman}), 2. body (\textit{vigraha}), 3. drop (\textit{bindu}), 4. phoneme (\textit{arṇa}), 5. world (\textit{bhuvana}) and 6. resonance (\textit{dhvani}). According to \citeauthor{vasudeva2004} (2004: 255), \textit{lakṣyabheda} in \textit{Mālinīviyajottaratantra} denotes `the ultimate destination upon which the Yogin must fix his attention’. These \textit{lakṣya}s are `different manifestations through which Śiva can be approached’. He further states: `To the Yogin engaged in the conquest of realities the \textit{lakṣya}s serve as teleological magnets drawing him towards the sought after rewards’. Despite the same basic concept, the \textit{lakṣya}s of the \textit{Mālinīviyajottaratantra} appear very different at first glance. On closer inspection, however, there are striking parallels with the \textit{lakṣya} systems found in the late medieval texts treated in this chapter. For example, the first \textit{lakṣya} of the \textit{Mālinīviyajottaratantra} 12.10abc is described as follows: \textit{bāhyabhyantarabhedena samuccayakṛtena ca} \textit{trividhaṃ kīrtitaṃ vyoma}. `The void is said to be threefold by the division of external, internal and that arising from accumulation’. \citeauthor{vasudeva2004} (2004: 263) maintains that this elliptical definition can only be explained on the basis of the teachings on the voids of other Śaiva Tantras but notes that none of the systems he consulted show complete congruence with the position of the \textit{Mālinīviyajottaratantra}. Nevertheless, he cites, for example, the passages from \textit{Dīkṣottara} 3.10c-11 and \textit{Svaccandatantra} 4.289 that are particularly interesting for our context, in which an upper emptiness (\textit{ūrdhvaśūnya}), a lower emptiness (\textit{adhaḥśūnya}) and a middle emptiness (\textit{madhyaśūnya}) are distinguished.
  
Taken together, the basic features of the late medieval differentiation of the five \textit{lakṣya}s into \textit{ūrdhva}-, \textit{adho}-, \textit{bāhya}-, \textit{antara}-, and \textit{madhyalakṣya} can already be discerned here. The \textit{lakṣya}s of the \textit{Mālinīviyajottaratantra} are discussed in detail in \citeauthor{vasudeva2004} (2004: 253-293). This rough overview illustrates that different systems of yogic \textit{lakṣya} practices have been circulating in the Śaiva Tantras for a very long time. Over the centuries, the techniques were passed on, copied and reused in the yoga traditions of Haṭha- and Rājayoga. In addition to the four texts analysed in this chapter, different forms of \textit{lakṣya} practice can also be found, for example, in \textit{Vivekamārtaṇḍa}, \textit{Śivayogapradīpikā}, (recensions of the \textit{Haṭhapradīpikā}), \textit{Yogasvarodaya}, \textit{Nityanāthapaddhati}, \textit{Siddhasiddhāntapaddhati}, \textit{Yogacūḍāmaṇyupaniṣad}, \textit{Maṇḍalabrāhmaṇopaniṣat}, \textit{Haṭhatattvakaumudi} and \textit{Haṭhasaṃketacandrikā}.} However, it was not labelled as an independent Yoga category until the texts of the complex late medieval Yoga taxonomies emerged. In the fifteen-fold Yoga taxonomy of \textit{Yogatattvabindu}, Lakṣyayoga is listed in the ninth position. The \textit{Yogasvarodaya} does not mention Lakṣyayoga in its introductory verses. The \textit{Yogasvarodaya} dedicates two verses to listing the fifteen Yogas. Although the verses announce fifteen Yogas, only eight Yogas are specified, probably for metrical reasons. Lakṣyayoga is not among the eight Yogas mentioned but is dealt with in detail throughout the text. In the \textit{Yogasiddhāntacandrikā}, Lakṣyayoga is mentioned in the eighth position\footnote{For an earlier discussion of \textit{Lakṣyayoga} in the \textit{Yogasiddhāntacandrikā}, see \citeauthor{penna2004} 2004, pp. 77-78.} and in the \textit{Sarvāṅgayogapradīpikā} Lakṣayoga\footnote{The terms vary in the literature. The most common term is \textit{lakṣya}, but \textit{lakṣa} or \textit{lakṣana} were also commonly specified.} at the seventh position.\footnote{See \citeauthor{burger2014sarvangayogapradipika} 2014, pp. 697-98 for another discussion of Lakṣayoga in the \textit{Sarvāṅgayogapradīpikā} in French.} For Sundardās, Lakṣayoga is a subcategory of Haṭhayoga alongside Rāja- and Aṣṭāṅgayoga. In contrast to the Yoga categories discussed so far, Lakṣyayoga is conceptually largely congruent in the late medieval texts of the complex Yoga taxonomies and differs only in a few details.

\subsection{Lakṣyayoga in the \textit{Yogatattvabindu}, \textit{Yogasvarodaya} and \textit{Sarvāṅgayogapradīpikā}}

The three texts present Lakṣyayoga as a simple Yoga method right at the beginning of their respective discourses. The descriptions of the texts are very similar. A separate analysis of them separately, as in the previous chapters, would be redundant. The word \textit{lakṣya} means ‘goal’. In the practice of Lakṣyayoga, it refers to goals on which the gaze (\textit{dṛṣṭi}) and the mind are directed, i.e. a ‘focus’ for stabilising the mind on which one constantly meditates. The three texts distinguish five categories from one another, depending on the place to be focussed. The following order\footnote{The order in the \textit{Sarvāṅgayogapradīpikā} is not identical, but as follows: 1. \textit{adho lakṣa}, 2. \textit{ūrddha lakṣa}, 3. \textit{madhya lakṣa}, 4. \textit{bāhya lakṣa} and 5. \textit{aṃtar lakṣa}.} is given in the \textit{Yogatattvabindu} and \textit{Yogasvarodaya}: 1. the upper focus (\textit{ūrdhvalakṣya}), 2. the lower focus (\textit{adholakṣya}), 3. the outer focus (\textit{bāhyalakṣya}), 4. the middle focus (\textit{madhyalakṣya}) and 5. the inner focus (\textit{antar(a)lakṣya}).\footnote{Only in \textit{Yogatattvabindu} is this \textit{lakṣya} is designated as \textit{antaralakṣya}. In all other texts, including the \textit{Haṭhasaṃketacandrikā}, which quotes the \textit{Yogatattvabindu}, the term \textit{antarlakṣya} is used.}\footnote{In the \textit{Yogatattvabindu} section \uproman{13}, in the \textit{Yogasvarodaya} (PT) ed. p. 833-34 and \textit{Sarvāṅgayogapradīpikā} 3.25-36.} Meditation on particular foci produces specific results.

\subsubsection{Ūrdhvalakṣya}
The upper focus (\textit{ūrdhvalakṣya})\footnote{\emph{Yogatattvabindu} \uproman{15}, \emph{Yogasvarodaya} PT p. 834 and \emph{Yogakarṇikā} 2.5.} refers to the fixation of the gaze (\textit{dṛṣṭi}) and the mind (\textit{manas}) on the centre of the sky, or the zenith (\textit{ākāśamadhye}). This results in the unity of the gaze with the splendour of the Supreme God (\textit{parameśvara}). In addition, an object arises in the sky within the practitioner’s scope of vision, an object that was previously unseen.\footnote{Cf. \textit{Yogatattvabindu} \uproman{14} (Ed. p. \pageref{urdhvalaksya}): \textit{etasya lakṣyasya dṛḍhīkaraṇāt parameśvarasya tejasā saha dṛṣṭairkyaṃ bhavati} | \textit{atha cākāśamadhye yaḥ kaścid adṛṣṭaḥ padārtho bhavati} | \textit{sa sādhakasya dṛṣṭigocare bhavati} |} The latter effect is cryptic. The source text, the \textit{Yogasvarodaya}, also does not contribute to clarity in this case, as there is no parallel passage. The \textit{Haṭhasaṃketacandrikā}\footnote{\textit{Haṭhasaṃketacandrikā} 2244 fol. 124v ll. 1-2.} quotes this passage literally, without further explanation. The only clue I found is in the description of \textit{ūrddha lakṣa} in \textit{Sarvāṅgayogapradīpikā} 3.27. The technique described here is identical. Here, the practitioner shall focus the gaze on the sky day and night. Sundardās explains the effect resulting from the practice in similar terms.\footnote{\textit{Sarvāṅgayogapradīpikā} 3.27: \textit{ūrddha lakṣa karai ihīṃ bhāṃtī} | \textit{duṣṭy ākāśa rahai dina rātī} | \textit{bibidh prakāra hoi ujiyārā} | \textit{gopi padāratha dīsahiṃ sārā} || 27 ||} In 3.27cd Sundardās states: `Various kinds of splendour manifest, the essence of the Gopīs’ object of consideration becomes visible’. Due to the striking similarity of the formulations and the fact that Sundardās must have been a contemporary of Rāmacandra, a correlation is probable. Sundardās was a disciple of Dādu Dayāl (1544-1603) and a member of the school named after him, and therefore a Vaiṣṇava, so the phrase ‘the essence of the object of the Gopīs’ consideration’ is probably the essence of Krṣṇa. Gopīs are paradigmatic figures of devotion (\textit{bhakti}) to Kṛṣṇa.\footnote{See e.g. \citetitle{bhagavata} 10.29.} Undoubtedly, the object of contemplation of the Gopīs must be Kṛṣṇa. Since Kṛṣṇa is considered the eighth \textit{avātara} of Viṣṇu, the essence or being of Kṛṣṇa is probably Viṣṇu, who is sometimes called \textit{puruṣottama} or \textit{parameśvara}. Whether the \textit{adṛṣṭaḥ padārthaḥ} of Rāmacandra is the same as the \textit{gopi padāratha} is uncertain, but the parallels to the wording of the \textit{sarvāṅgayogapradīpikā} are striking. Rāmacandra does not seem to favour any sectarian affiliation, and despite the clear Śaiva orientation of the main source text of his compilation, he is remarkably neutral in his formulations. Here, once more, he maintains his neutrality.\\\ 

\subsubsection{Adholakṣya}
The lower focus (\textit{adholakṣya}) of Rāmacandra is the stabilisation of the gaze (\textit{dṛṣṭi}) at a distance of twelve fingers' breadth from the tip of the nose or on the tip of the nose itself. The technique stabilises the \textit{dṛṣṭi}, the breath and prolongs life.\footnote{Cf. \emph{Yogasvarodaya} (PT): \textit{nāsikopari deveśi dvādaśāṅgulamānataḥ} \textit{dṛṣṭiḥ sthirā} (\textit{dṛṣṭisthiran} YK 2.5) \textit{tu karttavyā} (\textit{karttavyam} YK 2.5) \textit{adholakṣam idaṃ bhaja} (\textit{bhajet} YK 2. 5) | \textit{athavā} (\textit{tathā ca} YK 2.5) \textit{nāsikāgre tu sthirā dṛṣṭir iyaṃ bhavet} (\textit{śṛṇu} YK 2. 5) \textit{sthirā dṛṣṭiś cirāyuḥ syāt tathāsau} (\textit{yasya bhavet sthirā dṛṣṭiś cirāyuḥ} YK 2. 6) \textit{sthiradṛṣṭimān} |}\footnote{Rāmacandra, in contrast to \textit{Yogasvarodaya}, notes himself at this point that both options are taught as techniques of external focus (\textit{bāhyalakṣya}). The difference for Rāmacandra appears to be not only the designation but, above all, the subsequent focussing on \textit{śūnya}.} Afterwards, the practitioner is supposed to focus inwardly and outwardly on emptiness (\textit{śūnya}), which leads to freedom from the fear of death (\textit{maraṇatrāsa}).\footnote{Rāmacandra reduces and massively changes his source text. See edition \uproman{15} Ed. p. \pageref{adholaksya}. Rāmacandra's \textit{adholakṣya} on \textit{śūnya} is attributed to \textit{antarlakṣya} in the \textit{Yogasvarodaya}. For a translation of the passage, see the subchapter on \textit{antar(a)lakṣya} on p.\pageref{antarsvayotrans}.} 
\label{samketaadd1}
Sundaradeva, in his \citetitle{hathasamketacandrikachennai},\footnote{The collation of the passages of the \citetitle{hathasamketacandrikachennai} I based on ORI B 220 (f.239 r l.8 - f. 240r l.13), GOML R 3239 (f. 258 l.14 - f. 259 l.10) and HSC 2244 (HSC 2244 f. 124r ll. 5-9 - f. 125r ll. 1-2).} quotes the \textit{Yogatattvabindu} without attribution. He adds the following alternative techniques to his description of \textit{adholakṣya}: 
%\begin{quote}
%  \textit{athavā dṛṣṭir netrayor dvayor netrādhobhāgayor akṣikūṭayos tad adhogallayor ūbhayor upari sthirā kartavyā} | \textit{ekānte vijane dīpam āvarake saṃsthāpya ciraṃ gatvāvalokya stheyaṃ} | \textit{ghaṭīmātraṃ vā ghaṭikārdhaṃ vā tato dīpam ācchādya bhūmau sarvatrāvalokane sarvaṃ śvetanīlapītasphuliṅgakaṇāṃ 'te maṇḍalākāriṇiś ceta jyotiścakrāṇi pañcaṣaṭ vā dṛśyante} | \textit{tataś cāndhakāre dṛśyate} | \textit{dīptamatsarvaṃ svaśarīraṃ dṛśyate bhāsate sarvo 'pi sapradeśo dīptimān sphuṭo dṛśyate} | \textit{etad ārḍye jyotirmayacakrāṃte parameśvarasya tejomūrtir dṛśyate} | \textit{puṃsaḥ paramānandotpattir jāyate} | \textit{svadehavismṛtiś ca saṃbhavati} |
%\end{quote}
\begin{itquote}
  \begin{ekdosis}
    \note[type=witnesses, labelb=_intro1b, labele=_intro1e, nosep]{J = Jodhpur MS. No. 2244; C = Chennai GOML Ms. No. R 3239; C\textsubscript{pc} = Ibid. \textit{post correctionem}; M = Mysore ORI Ms. No. B 220.}
\linelabel{_intro1b}
% ----------------------
% athavā dṛṣṭir netrayor dvayo  netrā 'dhobhāgayor akṣikūṭayos tad adhogallayo rūpayor upari sthirā kartavyā (||) \J
% athavā dṛṣṭi  netrayor dvayor netrādhobhāgayor   akṣikūṭayos tad adhogallayo rūpa     pari sthirā kartavyā  ||  \M
% athavā dṛṣṭi  netrayor dvayor netrādhobhāgayor   akṣikūṭayos tad adhogallayo rūpayor upari sthirā kartavyā      \C
% athavā dṛṣṭi  netrayor dvayor netrādhobhāgayor   akṣikūṭayos tad adhogallayo ūbhayor upari sthirā kartavyā      \Cpc
% ----------------------
  athavā
  \app{\lem[wit={J}]{dṛṣṭir}
    \rdg[wit={C,Cpc,M}]{dṛṣṭi}} netrayor
  \app{\lem[wit={C,Cpc,M}]{dvayor}
    \rdg[wit={J}]{dvayo}}
  \app{\lem[wit={C,Cpc,M}]{netrādhobhāgayor}
    \rdg[wit={J}]{netrā 'dhobhāgayor}}
  akṣikūṭayos tad adhogallayo
  \app{\lem[wit={Cpc}]{ūbhayor}
    \rdg[wit={C,J}]{rūpayor}
    \rdg[wit={M}]{rūpa}}
  \app{\lem[wit={C,Cpc,J}]{upari}
    \rdg[wit={M}]{pari}}
 sthirā kartavyā \normalpipe  
% ----------------------  
% ekāṃte vijane dīpam āvarake saṃsthāpya ciraṃ gatvāvalokyastheyaṃ (||)  \J
% ekāṃte vijane dīpam āvake   saṃsthāpya ciraṃ gatvāvālokyastheyaṃ  ||  \M
% ekānte vijane dīpam āvake   saṃsthāpya ciraṃ gatvāvalokyastheyaṃ --       \C
% ----------------------
 ekānte vijane dīpam
 \app{\lem[wit={J}]{āvarake}
   \rdg[wit={C,Cpc, M}]{āvake}}
saṃsthāpya ciraṃ gatvāvalokyastheyaṃ \normalpipe  
% ---------------------- 
% ghaṭīmātra  vā ghaṭikārdhaṃ vā tato dīpam ācchādya bhūmau sarvatrāvalokane sarvaṃ śvetanīlapīta--sphuliṃgakaṇāṃ 'te maṃḍalākāriṇiś ceta jyotiśrakrāṇi paṃcaṣaṭ vā dṛśyaṃte (||) \J
% ghaṭīmātraṃ vā ghaṭikārdhaṃ vā tato dīpam ācchādya bhūmau sarvatrāvalokane sarvaṃ śvetanīlayoṃta-sphuliṃgakaṇāṃ te  maṃḍalākāriṇiś ceti jyotiśrakrāṇi paṃcaṣaṭ vā dṛśyaṃte  ||  \M
% ghaṭīmātraṃ vā ghaṭikārdhaṃ vā tato dīpam ācchādya bhūmau sarvatrāvalokane sarvaṃ śvetanīlayomta-sphuliṃgakaṇān te  maṇḍalākāriṇiś ceti jyotiścakrāṇi paṃcaṣad vā dṛśyaṃte ||   \C
% ----------------------
\app{\lem[wit={C,Cpc,M}]{ghaṭīmātraṃ}
  \rdg[wit={J}]{ghaṭīmātra}}
vā ghaṭikārdhaṃ vā tato dīpam ācchādya bhūmau sarvatrāvalokane sarvaṃ
śvetanīla\app{\lem[wit={J},alt={°pīta°}]{pīta}
  \rdg[wit={M}]{yoṃta}
  \rdg[wit={C,Cpc}]{yomta}}
sphuliṅgakaṇāṃ 'te maṇḍalākāriṇiś
\app{\lem[wit={C,Cpc,M}]{ceti}
  \rdg[wit={J}]{ceta}} 
 jyotiścakrāṇi pañcaṣad vā dṛśyante \normalpipe  
% ----------------------
% tataś cāṃdhakāre dṛśyate (||) \J
% tataś vāṃdhakāre dṛśyate || \M
% tataś cāndhakāre dṛśyate --- \C
% ----------------------
 tataś
 \app{\lem[wit={C,Cpc,J}]{cāṃdhakāre}
   \rdg[wit={M}]{vāṃdhakāre}}
 dṛśyate \normalpipe  
% ----------------------
% dīptamatsarvaṃ svaśarīraṃ dṛśyate bhāsate sarvo pi sapradeśo dīptimān sphuṭo dṛśyate (||) \J
% dīptamatsarvaṃ svaśarīraṃ dṛśyate bhāsate sarvo pi sapradeśo dīptimān sphuṭo dṛśyate || \M
% dīptimatsarvaṃ svaśarīraṃ dṛśyate bhāsate sarvo pi sapradeśo dīptimān sphuṭo dṛśyate -- \C
% ----------------------
 dīptimatsarvaṃ svaśarīraṃ dṛśyate bhāsate sarvo 'pi sapradeśo dīptimān sphuṭo dṛśyate \normalpipe  
% ----------------------
% etadārḍye jyotirmayacakrāṃte parameśvarasya tejomūrtir dṛśyate || \J
% etadārḍye jyotirmayacakrāṃte parameśvarasya tejomūrtir dṛśyate || \M
% ekadārḍye jyotirmayacakrāṃte parameśvarasya tejomūrtir dṛśyate \C
% ----------------------
 ekadārḍye jyotirmayacakrāṃte parameśvarasya tejomūrtir dṛśyate \normalpipe  
% ---------------------- 
% puṃsaḥ paramānaṃdotpattir jāyate (||) \J
% puṃsaḥ paramānaṃdotpattir jāyate|| \ M
% puṃsaḥ paramānandotpattir jāyate -- \C
% ----------------------
puṃsaḥ paramānandotpattir jāyate \normalpipe  
% ----------------------
% svadehavismṛtiś ca saṃbhavati (||) \J
% svadehavismṛtiś ca saṃbhavati || \M
% svadehavismṛtiś ca saṃbhavati -- \C
% ----------------------
svadehavismṛtiś ca
\app{\lem[wit={C,Cpc,M}]{saṃbhavati}
  \rdg[wit={J}]{saṃbhavati | athavā svanetrayor vartmanīr dakṣahastamadhyamātarjanībhyām akṣikū dehavismṛtiś ca saṃbhavati |}} \normalpipe   
% ----------------------
% athavā svanetrayor vartmanīr dakṣahastamadhyamātarjanībhyām akṣikū dehavismṛtiś ca saṃbhavati (|)
% \om
% \om
% ----------------------
\linelabel{_intro1e}
\end{ekdosis}
\end{itquote}
\begin{quote}
 Alternatively, the gaze should be fixed without wavering on both lower parts of the corners of the two eyes, below the cheekbones. In a lonely place without people, a lamp shall be placed in the darkness and observed for a long time. After one \textit{ghaṭikā} (24 minutes) or half a \textit{ghaṭikā} (12 minutes) [already], cover the lamp and then gaze all around on the ground; one may see all white, blue, and yellow sparkles forming circular patterns, and perhaps even fifty-six such circles of light become visible. As a consequence, one can see in the dark. One’s own body is seen illuminated. Also, the entire place lights up [and] is seen brightly and clearly. In this phase, within the circle of light, the luminous form of the supreme lord is seen. The generation of supreme bliss arises for the person. Forgetting of one’s own body occurs.
\end{quote} 
%\begin{quote}
%  \textit{athavā svanetrayor vartmanīr dakṣahastamadhyamātarjanībhyām akṣikūṭayor adhaḥ kṛtvā akṣivartmani dṛḍhaṃ cālanī ye ghaṭikārdhaṃ cā ghaṭīmātraṃ tata evaṃ kṛte sādhyakasyāgre suśvetajyotiḥ prākāśaḥ prāg bhavatīti} |
%\end{quote}
\begin{itquote}
  \begin{ekdosis}
    \note[type=witnesses, labelb=_intro2b, labele=_intro2e, nosep]{J = Jodhpur MS. No. 2244; C = Chennai GOML Ms. No. R 3239; C\textsubscript{pc} = Ibid. \textit{post correctionem}; M = Mysore ORI Ms. No. B 220.}
\linelabel{_intro2b}
% ----------------------
% athavā  svanetrayor  vartamanīr dakṣahastamadhyamatarjanibhyāṃ akṣikūtvā                    akṣivanmanī dṛdhaṃ cālanī ye ghaṭikārdhaṃ vā ghaṭīmātraṃ tatta evaṃ kṛte sādhyakasyāgre suśvetajyotiḥ prākāśaḥ prāgvad bhavatīti  \J
% athavā  svanetrayor  vartmanā   dakṣahastamadhyamātarjanībhyām ākṣikoṭayor     adhaḥ kṛtvā akṣivartmanī dṛḍhaṃ cālanī ye ghaṭikārdhaṃ vā ghaṭīmātraṃ tata  evaṃ kṛte sādhyakasyāgre suśvītajyotiḥ prākāśaḥ prāg    bhavatīti \M
% athavā  svanetrayor  vartmanā   dakṣahastamadhyamātarjanībhyām akṣikūṭakūṭayor adhaḥ kṛtvā akṣivartmani dṛḍhaṃ cālanī ye ghaṭikārdhaṃ cā ghaṭīmātraṃ tata  evaṃ kṛte sādhyakasyāgre suśvetajyotiḥ prākāśaḥ prāg    bhavatīti \C
% ----------------------
athavā svanetrayor
\app{\lem[wit={J}]{vartamanīr}
  \rdg[wit={C,Cpc,M}]{vartmanā}}
dakṣahastamadhyamātarjanībhyām
\app{\lem[type=emendation, resp=egoscr]{akṣikuṭayor}
  \rdg[wit={M}]{ākṣikoṭayor}
  \rdg[wit={C,Cpc}]{akṣikūṭakūṭayor}
  \rdg[wit={J}]{akṣikūtvā}}
\app{\lem[wit={C,Cpc,M}]{adhaḥ kṛtvā}
  \rdg[wit={J}]{\om}}
\app{\lem[wit={C,Cpc,M}]{akṣivartmanī}
  \rdg[wit={J}]{akṣivanmanī}} 
dṛḍhaṃ cālanī ye ghaṭikārdhaṃ vā ghaṭīmātraṃ tata  evaṃ kṛte sādhyakasyāgre suśvītajyotiḥ prākāśaḥ
\app{\lem[wit={C,Cpc,M}]{prāg}
  \rdg[wit={J}]{prāgvad}}
bhavatīti \normalpipe
\linelabel{_intro2e}
\end{ekdosis}
\end{itquote}

\begin{quote}
Alternatively, having placed the thumb and index finger of the right hand below the edge of the eye socket at the eyelids of the own eyes, and steadily causing to move [the fingers] at the eyelids, either for a half \textit{ghaṭikā} (12 minutes) or for a \textit{ghaṭikā} (24 minutes), as a result of having done this, very highly bright white light becomes visible in front of the practitioner.
\end{quote}

Sundardā's \textit{adho lakṣa} is the simple focusing of the gaze on the tip of the nose, which leads to the stabilisation of breath and mind.\footnote{\textit{Sarvāṅgayogapradīpikā} 2.26: \textit{prathamahīṃ adho lakṣa kauṃ jānaiṃ} | \textit{nāśā agra dṛṣṭi sthira ānaiṃ} | \textit{yātoṃ mana pavanā thira hoī} | \textit{adho lakṣa jo sādhai koī} || 26 ||}\\

\subsubsection{Bāhyalakṣya}
The external focus (\textit{bāhyalaksya})\footnote{\textit{Yogatattvabindu} \uproman{23}; \textit{Yogasvarodaya} (PT Ed. p.837).} is the fixation of the gaze (\textit{dṛṣṭi}) on one of the five gross elements at different distances from the tip of the nose or, in one case, directly on the tip of the nose. The texts present the foci as alternatives. The presentation of the three texts follows the same pattern in every case. They list a specific location, followed by an element (in most cases) and a characteristic, such as an associated colour. A table is the best way to illustrate the spread of the various techniques across the texts.
\begin{landscape}
\begin{longtable}{|m{2cm}|m{1.5cm}|m{2cm}|m{1.7cm}|m{1.7cm}|m{1.7cm}|m{1.7cm}|}
    \caption{Foci of Bāhyalakṣya}\\
    \hline
    \textbf{Location} & \textbf{Element} & \textbf{Characteristic} & \textbf{\textit{Yogatattvabindu}} & \textbf{{Yogasvarodaya}} & \textbf{\textit{Haṭhasaṃketacadrikā}} & \textbf{\textit{Sarvāṅgayogapradīpikā}} \\ 
    \hline
    \endfirsthead
    
    \multicolumn{6}{c}%
    {{\tablename\ \thetable{} -- continued from previous page}} \\
    \hline
    \textbf{Distance} & \textbf{Location} & \textbf{Characteristic} & \textbf{\textit{Yogatattvabindu}} & \textbf{\textit{Yogasvarodaya}} & \textbf{\textit{Haṭhasaṃketacadrikā}} & \textbf{\textit{Sarvāṅgayogapradīpikā}}\\ 
    \hline
    \endhead
    
    \hline
    \multicolumn{6}{|r|}{{Continued on next page}} \\ \hline
    \endfoot
    
    \hline \hline
    \endlastfoot

    Four finger breadths from the nose & Space & Appearing blue, full of splendour & X & X (Element missing) & X (Element = Wind; Characteristic= In the shape of smoke)\footnote{Possibly the text is corrupt and merged the first and second focus.} & X\\ 
    \hline
    Six finger breadths from the nose & Wind & In the shape of smoke & X & X & - & X\\ 
    \hline
    Eight finger breadths from the nose & Fire & Very red & X & X & X & X \\ 
    \hline
    Ten finger breadths from the nose & Water & White, fickle & X & -  & - & X \\ 
    \hline
    Twelve finger breadths from the nose & Earth & Yellow-coloured & X & - & - & X \\ 
    \hline
    At the tip of the nose & Space & Full of fire, shining like ten million suns & X & - & - & - \\ 
    \hline
    Above the space-element & Space & Connected to the sun without the sun (thousand rays) & X & - & - & - \\ 
    \hline
    Seventeen-finger wide distance above the head & Light & Mass of light & X & X & - & - \\ 
    \hline
    In front of the gaze & Earth & Appearing in the colour of molten gold & X & X & - & - \\
    \hline
\end{longtable}
\end{landscape}

 The table shows that the \textit{Yogatattvabindu} contains the greatest variety of foci of the \textit{bāhyalakṣya} category. Sundaradeva does not adopt all the foci in his \textit{Yogasaṃketacandrikā}. However, the text appears rather corrupt, as the text mixes up the first two foci. The \textit{Yogasvarodaya} only contains five of the nine foci in the table. Rāmacandra has added further foci based on the explanations of Bahirlakṣya in the \textit{Siddhasiddhāntapaddhati} 2.28 (ed. 38-40).\footnote{The \textit{Siddhasiddhāntapaddhati} teaches only three instead of five Lakṣyas: \textit{antarlakṣya} (2.26-27); \textit{bahiryalakṣya} (2.28); and \textit{madhyalakṣya} (2.29).} Sundardās describes the first five foci for the five elements in a perfectly analogous fashion.\footnote{Cf. \textit{Sarvāṅgayogapradīpikā} 2.29-31.} In the last verse of his explanation of \textit{bāhya lakṣa}, he explains that there are many more \textit{bāhya lakṣa}s, but they must be revealed by the Guru.\footnote{Cf. Ibid. 2.32: \textit{bāhya lakṣa aur bahuterī} \textit{so jānaṃ jo pāvai serī} | \textit{sataguru kṛpā karai jau kabahī} | \textit{dei batāi chinak maiṃ sabahī} || 32 ||}
The effects attributed to the practice of \textit{bāhyalakṣya} are similar throughout the texts. Regardless of the variant practised, the practice promises rejuvenation, improved health, but moreover an improved social life\footnote{\textit{Yogatattvabindu} \uproman{23}: \textit{samagrāḥ śatravaḥ svapne ‘pi mitratām ayānti} |} and a longer life span etc. 

\subsubsection{Antar(a)lakṣya}
The inner focus (\textit{antar(a)lakṣya}) is a special case, as there are noticeable deviations between Rāmacandra’s \textit{Yogatattvabindu} and the \textit{Yogasvarodaya}. Although Rāmacandra continues to follow the \textit{Yogasvarodaya} in terms of structure and content for the description of his \textit{antar(a)lakṣya}, the passages in the \textit{Yogasvarodaya} are not explicitly attributed to \textit{antaralakṣya}, but are evidently assigned to the preceding \textit{bāhyalakṣya}.\footnote{Cf. \textit{Yogatattvabindu} \uproman{24} and \textit{Yogasvarodaya} (PT Ed. pp. 837-38).} In addition, Rāmacandra simultaneously uses the \textit{Siddhasiddhāntapaddhati} (2.26-27) as a template for this passage, which attributes largely similar practices to the category of \textit{antar(a)lakṣya}. In the \textit{Yogasvarodaya}, there is a separate description of \textit{antarlakṣya}, the core practice of which was already integrated by Rāmacandra in the context of his \textit{adholakṣya}.\footnote{This is the meditation on emptiness (\textit{śūnya}). Cf. \textit{Yogatattvabindu} \uproman{15} and \textit{Yogasvarodaya} (PT Ed. p. 834).} 
The concept of the \textit{antar lakṣa} of Sundardās is essentially identical.

In the \uproman{24} section of the \textit{Yogatattvabindu}, Rāmacandra specifies a total of three alternative \textit{antar(a)lakṣya}s. 
As part of the explanations of the first \textit{antar(a)lakṣya}, Rāmacandra first presents a description of the central channel in the yogic body, which is labelled here as \textit{brahmanāḍī}. It originates from the spine (\textit{brahmadaṇḍa}) and passes through the spine from bottom to top. The central channel extends from the root bulb (\textit{mūlakanda}) to the opening of Brahman (\textit{brahmarandhra}) at the top of the head. It is shaped like the stem of a lotus flower and shines like ten million suns. The practice of \textit{antar(a)lakṣya} consists of meditating on it, which allows the practitioner to acquire supernatural abilities. Just the first of the three techniques appears in the context of \textit{antar lakṣa} in the \textit{Sarvāṅgayogapradīpikā} of Sundardās, albeit in less detail. According to Sundardās, one is supposed to meditate on the central channel known as Brahmanāḍī, which leads to the eight supernatural faculties.\footnote{Cf. \textit{Sarvāṅgayogapradīpikā} 3.33: \textit{aṃtar lakṣa ju sunahuṃ prakāśā} | \textit{brahma nāḍikā karahu abhyāsā} | \textit{aṣṭa siddhi nava niddhi jahāṃlauṃ} | \textit{ṭarahiṃ na kabahūṃ jivai jahāṃ lauṃ} || 33 ||}.
Rāmacandra’s second technique for the practice of \textit{antaralakṣya} is a meditation on a bright light above the forehead, preventing certain diseases.
The third alternative for the practice of \textit{antaralakṣya} is meditation on the very fine red light in the centre between the eyebrows, which causes the yogin to be loved by everyone in the royal court and ensures that no one can take their eyes off him. \footnote{All three techniques of \textit{antar(a)lakṣya} are also specified in the \textit{Yogasvarodaya} (PT Ed. p. 837-28), but still in the context of \textit{bāhyalakṣya}: \textit{mūlakandotthatalato brahmanāḍīsamudbhavā} | \textit{śvetavarṇā brahmarandhraparyantam eva tiṣṭhati} | \textit{eṣā tu brahmarandhrākhyā tanmadhye varttate parā} | \textit{padmatantusamākārā koṭisūryataḍitprabhā} | \textit{calaty ūrddhaṃ mahāmūrttir asya dhyānād bhavec chivaḥ} | \textit{aṇimādy aṣṭasiddhis tu samagreṇa prasīdati} | \textit{lalāṭopari vā dhyātvā candraṃ vā jyotir īśvaram} | \textit{nāśayet kuṣṭharogādīn mahāyuṣmān śivaḥ paraḥ25↩} | \textit{bhruvor madhye’ thavā dhyātvā arkantu teja īśvaram} | \textit{sthiradṛṣṭau rājapūjyo jīvanmuktaḥ śivo yathā} | \textit{ātmānam ātmarūpaṃ hi dhyātvā yo niṣkriyo bhavet} | \textit{nirāśīryatattvo ‘yaṃ itaro na nṛpasthitiḥ} |}  

The \textit{antar(a)lakṣya} of the \textit{Yogasvarodaya},\footnote{\textit{Yogasvarodaya} (PT Ed. p. 824) and \textit{Yogakarṇikā} 2.8-13.} the \textit{Yogatattvabindu}, \textit{Sarvāṅgayogapradīpikā}, and \textit{Siddhasiddhāntapaddhati} differs greatly from the models in \textit{Yogatattvabindu}, \textit{Sarvāṅgayogapradīpikā}, and \textit{Siddhasiddhāntapaddhati}. It is exclusively about meditation on emptiness (\textit{śūnya}): 

\label{antarsvayotrans}
  \begin{quote}
    \begin{ekdosis}
      \note[type=witnesses, labelb=_intro3b, nosep]{PT= \textit{Prāṇatoṣiṇī} quotes \textit{Yogavarodaya} with reference \textit{yogasvarodaye}. YK= \textit{Yogakarṇikā} quotes \textit{Yogavarodaya} with reference \textit{yogasvarodaye}.}
      \linelabel{_intro3b}
    \textit{
  antarlakṣaṃ śṛṇu \app{\lem[wit={PT}, alt={subhru°}]{subhru}
    \rdg[wit={YK}]{śukra°}}digvidigādivarjitam} |\\
\textit{
\app{\lem[wit={YK}]{bāhyabhyantara ākāśaṃ vādhāmantraṃ paraṃ mataṃ}
  \rdg[wit={PT}]{\om}}} ||
  \end{ekdosis}
\end{quote}
\begin{quote}
Listen to the internal focus, oh lovely-browed [Goddess], being devoid of the major and minor directions, etc. The internal and external space is the magical formula against pain, the supreme view.
\end{quote}
\begin{quote}
  \begin{ekdosis}
\textit{calajjāgratsuṣupteṣu bhojaneṣu ca sarvadā} |\\
\textit{sarvāvasthāsu deveśi cittaṃ śūnye niyojayet} ||
  \end{ekdosis}
\end{quote}
\begin{quote}
While walking, waking, sleeping and eating at all times
[and] in all states, oh Goddess, the mind shall be focussed onto emptiness.  
\end{quote}
\begin{quote}
  \begin{ekdosis}
    \textit{karttā kārayitā \app{\lem[wit={YK}]{śūnyaṃ}
        \rdg[wit={PT}]{śunyaḥ}}mūrtimān śūnya īśvaraḥ} |\\
    \textit{harṣaśokaghaṭastho ’yaṃ janmamṛtyū labhet svayam} ||
      \end{ekdosis}
\end{quote}
\begin{quote}
The actor and he who causes to act are void; the form-bearer in the void is the supreme lord.
Situated in a vessel of joy and sorrow, he himself experiences both birth and death. 
\end{quote}
\begin{quote}
  \begin{ekdosis}
\textit{\app{\lem[wit={YK}]{ghaṭasthāṃ}
    \rdg[wit={PT}]{ghaṭasthā}}
\app{\lem[wit={YK}]{cintayen}
  \rdg[wit={PT}]{cintyayor}}
\app{\lem[wit={YK}]{mūrttimitaś}
  \rdg[wit={PT}]{mūrtir hata°}}cintāsvarūpadhṛk} |\\
\textit{viṣayaṃ viṣavad \app{\lem[wit={YK}]{dṛṣṭvā}
    \rdg[wit={PT}]{duṣṭaṃ}} tyaktvā jñātvā tu mārutam} ||
      \end{ekdosis}
    \end{quote}
    \begin{quote}
He shall contemplate [himself as] being situated in a vessel, established as form [and] carrying the nature of thought. 
Having abandoned sense objects as defective like poison, having realized them as consisting of the Maruts, \ldots 
\end{quote}
\begin{quote}
  \begin{ekdosis}
\textit{saṃjñāśūnyamanā bhūtvā puṇyapāpair na lipyate} |\\
\textit{bāhyam ābhyantaraṃ
\app{\lem[wit={PT}]{khaṃ}
  \rdg[wit={YK}]{\om}}
\app{\lem[type=emendation, resp=egoscr]{yad}
  \rdg[wit={YK}]{yad hi}
  \rdg[wit={PT}]{hi}} antarlakṣam iti smṛtam} ||
     \end{ekdosis}
    \end{quote}
\begin{quote}
\ldots having become aware of the emptiness of conception, he is not tainted by merits or sin.
That which is the inner and outer space is taught as the internal focus.
\end{quote}
\begin{quote}
  \begin{ekdosis}
\textit{etad dhyānāt sadā kiñcid duḥkhaṃ na syāc chivo bhavet} |\\
\textit{śūnyan tu saccidānandaṃ niḥśabdaṃ brahmaśabditam} |\\
\textit{saśabdaṃ jñeyam
\app{\lem[wit={PT}]{ākāśam}
  \rdg[wit={YK}]{ākāśa}}iti bhedadvayan tv iha} ||
   \end{ekdosis}
 \end{quote}
    \begin{quote}
Because of this meditation, any kind of suffering will no longer arise [and] one would become Śiva.
Emptiness is being-consciousness-bliss, [and] called the soundless Brahman;
space [on the other hand] is to be understood as with sound. Indeed, this is the twofold distinction in this world.  
\end{quote}

\subsubsection{Madhyalakṣya}

The concept of the central focus (\textit{madhyalakṣya}) is very similar in all three texts. In the \textit{Yogatattvabindu}\footnote{see \textit{Yogatattvabindu} \uproman{27}, Ed. p. \pageref{madhyalaksya}.}, a light is visualised by the mind. The light is supposed to be the size of one's own body. Like a room on fire, this body shall be envisioned as filled with light. The light shall be white, yellow, red, grey or blue. The envisioned light is compared to the light of the sun, lightning or a crescent moon. \textit{Madhyalakṣya} leads to the burning of the impurities of the mind. It also produces the sattvic quality of the mind. The practitioner becomes blissful. Rāmacandra remains very close to his original text regarding the choice of terminology and the content. Thus, there is no significant conceptual difference in comparison with the \textit{madhyalakṣya} of the \textit{Yogasvarodaya}.\footnote{Cf. \textit{Yogasvarodaya} (Ed. p. 839): \textit{idānīṃ madhyalakṣantu kathyate siddhikārakam} | \textit{śvetaṃ raktaṃ tathā pītaṃ dhūmrākārantu nīlabham} | \textit{agnijvālāsamānābhā vidyutpuñjasamaprabhā} | \textit{ādityamaṇḍalākāramathavā candramaṇḍalam} | \textit{jvaladākāśatulyaṃ vā bhāvayed rūpamātmanaḥ} | \textit{etaj jyotirmayaṃ dehaṃ manomadhye tu lakṣayet} | \textit{eteṣāñ ca kṛte lakṣe nānāduḥkhaṃ praṇaśyati} | \textit{manas astu malo yāti mahānando bhavet tataḥ} |} Sundardā's descriptions in the \textit{Sarvāṅgayogapradīpikā} are shorter, but equally similar. The mind is supposed to dwell in its centre and focus on the form of the body. The practice brings about the sattvic quality of the mind. However, Sundardās does not specify any visualisation of a light.\footnote{Cf. \textit{Sarvāṅgayogapradīpikā} 3.28: \textit{madhya lakṣa mana madhya bicārai} | \textit{vapu pramāna koi rūpa nihārai} |\textit{yāte sātvik upajai āī} | \textit{madhya lakṣa jo sādhai bhāīī} ||)}

\subsection{Lakṣyayoga in the \textit{Yogasiddhāntacandrikā}}
\label{laksyayogaintrocandrika}

Nārāyaṇatīrtha neither divides Lakṣyayoga into five,\footnote{As in the \textit{Yogatattvabindu}, the \textit{Yogasvarodaya} or in the \textit{Sarvāṅgayogapradīpikā}. } nor in three subcategories.\footnote{As in the \textit{Siddhasiddhāntapaddhati} or the \citetitle{shivayogapradipika}.} His explanations are of a more general nature. He locates Lakṣyayoga within the framework of his commentary on \textit{Yogasūtra} 1.35.

\begin{quote}
  \textit{lakṣyayogasvarūpam upāyāntaram āha}-\\
  \textit{viṣayavatī vā pravṛttir utpannā manasaḥ sthitinibandhinī} || 35 ||
\end{quote}
\begin{quote}
He states another method having the nature of Lakṣyayoga - \\
Alternatively, activity directed to a sense object, which is generated, causes the stopping of the mind.  
\end{quote}

Nārāyaṇatīrtha explains:

\begin{quote}
  \textit{viṣayavatīti} | \textit{nāsāgrādau cittasya saṃyamarūpāl lakṣyayogād divyagandhādisākṣātkāro bhavati} | \textit{seyaṃ viṣayavatī pravṛttir viśvāsam utpādya parameśvarādāv atisūkṣme manasaḥ sthitiṃ sampādayatīty arthaḥ} | \textit{tathā ca śāstrīyānubhavaviṣaye jāte śraddhayā yogino dhyānādau sthirā bhavatīty ayaṃ lakṣyayogaḥ} |\\

  \textit{yā hi nāsādideśeṣu dṛṣṭiḥ puṃsāṃ sthirā bhavet} |\\
  \textit{sa lakṣyayoga ākhyāto yoge śraddhākaraḥ paraḥ} ||\\
  
\textit{iti smṛter iti} || 35 ||
\end{quote}
\begin{quote}
  [Regarding the term] ``\textit{viṣayavatī}''. As a result of Lakṣyayoga, which has the nature of concentration of the mind (\textit{saṃyama}) on the tip of the nose, etc., a direct perception of divine fragrances and other objects occurs. This activity being directed to sense objects, having produced confidence, causes to generate fixedness of the mind in [something] very subtle, in [something like] the supreme Lord, etc. Such is the meaning. 
  And thus, stability in meditation, etc., arises for the yogin after the sense object from the experience of scripture has been produced with confidence. This is Lakṣyayoga.\\
  
  For indeed, when the gaze of the person becomes steady at places like the tip of the nose, etc., that is called Lakṣyayoga, which in Yoga, is considered the supreme faith-inspiring [practice].\\

  Thus, it is remembered.
  \end{quote}

  Nārāyaṇatīrtha is referring to the \textit{bhāṣya} part of the \textit{Pātañjalayogaśāstra} concerning \textit{sūtra} 1.35.\footnote{\citetitle{yogasutraed} (ed. p. 80): \textit{nāsikāgre dhārayato ‘sya yā divyagandhasaṃvit sā gandhapravṛttiḥ} | \textit{jihvāgre rasasaṃvit} | \textit{tāluni rūpasaṃvit} | \textit{jihvāmadhye sparśasaṃvit} | \textit{jihvāmūle śabdasaṃvid ity etā vṛttaya utpannāś cittaṃ sthitau nibadhnanti}, \textit{saṃśayaṃ vidhamanti}, \textit{samādhiprajñāyāṃ ca dvārībhavantīti} | \textit{etena candrādityagrahamaṇipradīparaśmyādiṣu pravṛttir utpannā viṣayavaty eva veditavyā yady api hi tattacchāstrānumānācāryopadeśair avagatam arthatattvaṃ sadbhūtam eva bhavati} | \textit{eteṣāṃ yathābhūtārthapratipādanasāmarthyāt}, \textit{tathāpi yāvad ekadeśo ‘pi kaścin na svakaraṇasaṃvedyo bhavati tāvat sarvaṃ parokṣam ivāpavargādiṣu sūkṣmeṣv artheṣu na dṛṃ buddhim utpādayati} | \textit{tasmāc chāstrānumānācācāryopadeśopodbalanārtham evāvaśyaṃ kaścid arthaviśeṣaḥ pratyakṣīkartavyaḥ} | \textit{tatra tadupadiṣṭārthaikadeśapratyakṣatve sati sarvaṃ sūkṣmaviṣayam api āpavargāc chraddhīyate} | \textit{etadartham evedaṃ cittaparikarma nirdiśyate} | \textit{aniyatāsu vṛttiṣu tadviṣayāyāṃ vaśīkārasaṃjñāyām upajātāyāṃ samarthaṃ syāt tasya tasyārthasya pratyakṣīkaraṇāyeti} | \textit{tathā ca sati śraddhāvīryasmṛtisamādhayo ‘syāpratibandhena bhaviṣyantīti} |} In the \textit{bhāṣya} part, various foci for meditation and specific effects that arise through concentration on the respective point are listed. Concentration on the tip of the nose creates absolute odour perception. Concentration on the tip of the tongue leads to absolute perception of flavour. Concentration on the palate leads to absolute perception of form. Concentration on the centre of the tongue leads to absolute perception of touch. Concentration on the root of the tongue leads to absolute perception of sound. In addition, the \textit{bhāṣya} lists the moon, sun, planets, jewels and lamps as sensory objects for focussing the mind. The resulting heightened perceptions stabilise the mind, remove doubt and are a gateway to \textit{samādhi}. Furthermore, the \textit{bhāṣya} explains that although the true nature of reality can be revealed through scriptures, inferences or instructions from teachers, these must be experienced personally, through one's own senses, so that the experience is not second-hand. Otherwise doubts occur for the practitioner. However, if these heightened perceptions referred to in this \textit{sūtra} are experienced personally, then faith, trust or confidence (\textit{śraddhā}) in the statements of the scriptures etc., the entire yogic endeavour and especially the possibility of the desired liberation is strengthened.

\section{10. Vāsanāyoga}
\label{vasanayogaintro}

Vāsanāyoga befindet sich in der im \textit{Yogatattvabindu} eingangs präsentierten Taxonomie auf Position zehn. In der \textit{Yogasvarodaya} auf Position acht. Alldings beinhalten beide Texte keine dezidierte Beschreibung von Vāsanāyoga. In der \textit{Yogasiddhāntacandrikā} findet sich Vāsanayoga auf Position zwölf.\footnote{For another discussion of Vāsanāyoga in the \textit{Yogasiddhāntacandrikā} see \citeauthor{penna2004} 2004, pp. 82-85.} Die \textit{Sarvāṅgayogapradīpikā} führt Vāsanayoga nicht auf. Der Begriff \textit{vāsanāyoga} ist in der gesamten Yogaliteratur äußerst selten und taucht nur im Kontext der spätmittelalterlichen Yogataxonomien auf. In den frühen und mittelalterlichen Texten des Yoga findet sich überhaupt nicht. Das Kompositum \textit{vāsanāyoga} taucht an wenigen Stellen in der tantrischen Literatur auf, aber in einem anderen Kontext und nicht als eigenständig zu unterscheidende Yogakategorie.  

Der Begriff \textit{vāsanā} ist ein technischer Begriff, mit dem häufig in der indischen Philosophie, vor allem im Kontext der Konzeption von \textit{karma} operiert wird. Insbesondere im Yoga und Advaita Vedānta spielt er eine tragende Rolle. Auch in der buddhistischen Philosophie wird dieser Begriff in diesem Kontext verwendet. Die Konzeption des Begriffes \textit{vāsanā} kann in der Yogaphilosophie des Pātañjalayoga und Advaita Vedānta, welche im Kontext der hier behandelten Texte kongruiert, wiefolgt charakterisiert werden. \textit{Vāsanā} bezeichnet eine Art des karmischen Abdrucks. In der Kommentarliteratur des \textit{Pātañjalayogaśāstra} ist der Begriff und das Konzept \textit{vāsanā} eng mit Begriff und dem Konzept von \textit{saṃskāra} verknüpft, oft werden beide Begriffe hier sogar synonym verwendet. Ein \textit{saṃskāra} ist ein mentaler Abdruck, der von jeder Handlung (\textit{karma}) im Geist (\textit{citta}) hinterlassen wird. \textit{Saṃskāra}s triggern Gedanken, Erinnerungen und Handlung (\textit{karma}). \textit{Vāsanā} hingegen bezieht sich vor allem auf kummulative innewohnenden Abdrücke, welche einen unterbewussten Einfluss auf die Persönlichkeit und die Handlung der Person ausüben, sozusagen eine aus vergangegen Handlungen verursachte Verhaltenstendenz. \textit{Vāsanā}s sind auch diejenigen \textit{saṃskāra}s, welche einen Einluss auf spätere Wiedergeburten ausüben, bzw. die Konfiguration der Wiedergeburt steuern.\footnote{Cf. \citeauthor{bryant2009} 2009, p. 418.} Jede von einem Subjekt ausgeführte Handlung hinterlässt einen Abdruck oder eine Spur im Karmaspeicher (\textit{karmāśaya}) des Geistes (\textit{citta}). Weil der Geist im Pātañjalayoga Hauptbestandteil des transmigrierenden feinstofflichen Körpers (\textit{sūkṣmaśarīra}) ist, bestimmt die Konfiguration des Karmaspeichers im Geist auch die Art der zukünftigen Wiedergeburt.\footnote{Cf. \textit{Pātañjalayogaśāstra} 4.7-11.} Wortwörtlich betrachtet bedeutet \textit{vāsanā} sogar `Beduftung' oder in diesem Kontext vielmehr `Duftspur'. Die Handlungen hinterlassen metaphorisch gesprochen eine bestimmte Duftnote, eine Duftnote, welche die Person durchdringt und noch lange in zukünftigen Handlungen wieder zum Tragen kommen wird, denn es wird angenommen, dass die Anhäufung dieser gewohnheitsmäßigen Tendenzen die Person zu bestimmten zukünftigen Denk- und Verhaltensmustern prädisponiert. Diese Denk- und Verhaltensmuster können jederzeit, beispielsweise ausgelöst durch Sinnesreize, aktiviert werden. Im Kontext einer meditativen Yogapraxis, die darauf ausgerichtet ist den Zustand namens \textit{samādhi} mittels Konzentration zu erreichen, einen Zustand, der durch einen temporären Stillstand der mentalen Aktivität gekennzeichnet ist, führen die \textit{saṃskāra}s und \textit{vāsanā}s im Geiste des Yogin, bei deren deren Aktivierung, beispielsweise durch Sinnesreize dazu, immer wieder zu neu aufkommender Geistesaktivität und somit zu Ablenkung vom angestrebten Ziel. Sind diese aktiv gelten die meisten als hinderlich für das Endziel der Yogapraxis und sind entweder zu reduzieren oder zumindest inaktiv bzw. latent. Ist der Yogin durch die Yogapraxis frei von aktivierten \textit{saṃskāra}s und \textit{vāsanā}s, kann er hierdurch ausgelöst nicht nur den \textit{samādhi} Zustand erreichen, sondern er wird auch nicht mehr wiedergeboren und ist somit aus dem Geburtenkreislauf (\textit{saṃsāra}) befreit. Es ist wichtig zu betonen, dass es auch sehr positive \textit{saṃskāra}s und \textit{vāsanā}s gibt, welche die Yogapraxis begünstigen, beispielsweise die Angewohnheit einer regelmäßigen Yogapraxis (\textit{yogābhyāsa}), oder zuträgliche Essgewohnheiten. Aber auch alle positiven \textit{saṃskāra}s und \textit{vāsanā}s, dürfen, zumindest für den finalen Yogazustand des \textit{Pātañjalayogaśāstra}, dem \textit{asaṃprajñātasamādhi}, nicht mehr aktiv sein.\footnote{Siehe hierzu \textit{Pātañjalayogaśāstra} 1.18, 1.50-51 und \citeauthor{bryant2009} 2009, p. 70-72 (1.18) und p. 164-68 (1.50-51) für eine Zusammenfassung der klassischen Kommentare.}  

Lesen wir also von einem Vāsanāyoga, erwarten wir natürlicherweise ein Yoga, welches darauf ausgerichtet ist die \textit{vāsanā}s zu reduzieren und hierüber \textit{mokṣa} zu erlangen.

\subsection{Der Begriff \textit{vāsanā} im \textit{Yogatattvabindu} und \textit{Yogasvarodaya}}

Ähnlich wie bereits im Fall des Dhyānayoga, welches beide genannten Texte nicht als separate Kategorie einführen, sich aber dennoch das Konzept von \textit{dhyāna} in beiden Texten extrapolieren lässt, so lassen sich auch trotz der Abwesenheit einer dezidierten Beschreibung von Vāsanayoga, Rückschlüsse über die Verwendung und des Konzeptes von \textit{vāsanā} in beiden Texten ziehen.

Im \textit{Yogatattvabindu} spielt der Begriff bei der Deutung (\textit{nirukti}) des Wortes \textit{avadhūta} eine Rolle. Diese Wortdeutung wird in \uproman{44}.3 und \uproman{44}.4\footnote{Obwohl die meisten Verse und Passagen in \textit{Yogatattvabindu} \uproman{44} der \textit{Siddhasiddhāntapaddhati} entnommen sind, lässt in diesem Fall keine Entsprechung zu den Versen \uproman{44}.3-4 finden. Möglicherweise handelt es sich um Verse von Rāmacandra selbst. Das \textit{Yogasvarodaya} thematisiert den \textit{avadhūta} überhaupt nicht.} wiedergegeben.

\begin{quote}
  \textit{ātmā hy akāro vijñeyo vakāro bhavavāsana} |
  \textit{dhūta tatkaṃpanaṃ proktaṃ so 'vadhūta udāhṛtaḥ} || \uproman{44}.3 ||
\end{quote}

\begin{quote}
  The letter \textit{a} is, in fact, to be known as the self and the letter \textit{va} as the impressions of [mundane] existence; \textit{dhūta} (`has shaking off') is said to be the special weapon; he is called an Avadhūta.
\end{quote}

\begin{quote}
  \textit{akārārtho jīvabhūto vakārārtho 'tha vāsanā} |
  \textit{etad dvayaṃ yaḥ jānati so 'vadhūta udāhṛtaḥ} || \uproman{44}.4 ||
    \end{quote}

    \begin{quote}
      The meaning of the letter \textit{a} the being of the embodied soul, the meaning of the letter \textit{va} then impressions. He who knows this couple is declared to be an Avadhūta.
    \end{quote}
    
    Demnach ist ein Avadhūta dadurch charakterisiert, die verkörperte Seele (\textit{jīva}) und die durch Handlung (\textit{karma}) im Rahmen von Erlebnissen und Erfahrungen hervorgebrachten \textit{vāsana}s (`innewohnende mentale Abdrücke') nicht nur zu kennen, sondern der Avādhūta ist eine verkörperte Seele (\textit{jīva}), die bereits alle \textit{vāsanā}s abgeschüttelt hat und, wie uns die folgenden Verse \uproman{44} 5-10 wissen lassen, ein durch eine Yogapraxis vollendeter Yogin (\textit{siddhayogin}) geworden ist.\\
    
    Darüber hinaus taucht der Begriff \textit{vāsanā} erneut im Kontext von \textit{Yogatattvabindu} Sektion \uproman{52}. Diese Sektion steht in einer thematischen Abfolge von Sektionen, welche metaphysische Konzepte der Kosmogenie ausdifferenzieren. Dies beginnt mit Sektion \uproman{48}, welche mit dem Statement eingeleitet wird, dass folgendes Wissen durch die Vollendung des Yoga entsteht.\footnote{\textit{Yogatattvabindu} \uproman{48}: \textit{idānīṃ yogasiddhar anantaraṃ etādṛśaṃ jñānaṃ utpadyate}.} Daraufhin rollt Rāmacandra eine Kosmogonie aus. Diese basiert auf der Kosmogonie der \textit{Yogasvarodaya} und der \textit{Siddhasiddāntapaddhati}. Er vermischt, reduziert, reorganisiert und vereinfacht jedoch die Inhalte seiner Quelltexte. Die Schöpfung beginnt noch bevor der Schöpfer, der aus \textit{kula} (Śakti) und \textit{akula} (Śiva) zusammengesetzt ist, existierte. Das was vor dem Schöpfer existierte wird als die unmanifestiere (\textit{avyakta}), namenlose (\textit{anāmā}) höchste Realität (\textit{paraṃ tattvaṃ}) bezeichnet. Diese entfaltet sich im Verlaufe der Sektionen \uproman{48} - \uproman{56} in Pentaden, die selbst wiederum je aus fünf Qualitäten entstehen. In Sektion \uproman{52} führt Rāmacandra die nächste Pentade an, die jedoch von ihm, aus nicht nachvollziehbaren Gründen nicht benannt wird. Diese basiert jedoch eindeutig auf den Ausführungen der Pentade zum Thema \textit{vyaktaśakti} der \textit{Siddhasiddhāntapaddhati}.\footnote{Cf. \textit{Siddhasiddhāntapaddhati} 1.54.} Diese Pentade besteht aus Wille (\textit{icchā}), Aktivität (\textit{kriyā}), Illusion (\textit{māyā}), Urnatur (\textit{prakṛti}) und Sprache (\textit{vācā}). Jede Pentade besitzt wiederum fünf Eigenschaften. Der Wille (\textit{icchā}) besteht aus den fünf Eigneschaften -  intense passion (\textit{unmāda}), mentale Abdrücke (\textit{vāsanā}), Wunsch (\textit{vāñchā}), mentaler Zustand (\textit{caitta}) und Verhalten (\textit{ceṣṭā}). Diese Pentade findet sich gleichfalls im \textit{Yogasvarodaya} wieder.\footnote{\textit{Yogasvarodaya} (PT Ed. p. 847).} Weitere Hintergründe zu diesen fünf Eigenschaften präsentiert keiner der Texte. \\

  Die letzte Erwähnung von \textit{vāsanā} befindet sich in Sektion \uproman{57}. Diese Sektion gehört mitunder zu einer der längsten Sektionen des gesamten Textes und ihr wird daher besondere Bedeutung für das gesamte Yogasystem des Rāmacandra beigemessen. Sie trägt den Titel `Majestät des Yoga' (\textit{yogasya māhātmyaṃ}) und betont vehement die Unerlässlichkeit eines Lehreres (\textit{guru}) für die Erlangung der Realität des Yoga (\textit{yogatattva}). Dies sollte allerdings nicht einfach nur irgendein Lehrer sein, sondern ein wahrer Lehrer (\textit{sadguru}):
  \begin{quote}
    \textit{vikalpa etādṛśo yathā samudramadhye mahttarakallolāḍambaraḥ prapañcacāsanā etādṛśī yathodakamadhye mahattaraṅgāḥ} | \textit{tādṛśāt saṃsārārṇavād yo nāvā paraṃ pāraṃ prāpayati} | \textit{sa sadguruḥ kathyate} |
    \end{quote}
  \begin{quote}
    The changing thought is like the roar of waves within the ocean. The manifold mental imprints are like the ripples in the water. He who causes to navigate the boat from such an ocean of \textit{saṃsāra} to the other shore is called a true teacher.
  \end{quote}

  Insgesamt kann zusammenfassend festgestellt werden, dass in der uns vorliegenden Überlieferung des \textit{Yogasvarodaya} der Begriff \textit{vāsanā} nur im Kontext der Kosmogonie auftaucht und Vāsanāyoga zumindest in dieser Überlieferung schlichtweg nicht vorhanden ist. Auch im Rahmen der drei Kontexte in denen \textit{vāsana} im \textit{Yogatattvabindu} genannt wird - \textit{avadhūta}, Kosmogonie und Bedeutung des Lehrers für die Yogapraxis, kann nicht von einem Vāsanāyoga gesprochen werden.

\subsection{Vāsanāyoga in the \textit{Yogasiddhāntacandrikā}}
\label{laksyayogaintrocandrika}  

Die \textit{Yogasiddhāntacandrikā} ist der einzige Text innerhalb der Texte der komplexen spätmittelalterlichen Taxonomien, welcher eine dezidierte Beschreibung eines Vāsanāyogas beinhaltet.

Nārāyaṇatīrtha verortet Vāsanayoga im Rahmen seines Kommentares zu \textit{Yogasūtra} 1.37 und 1.38\footnote{Cf. \textit{Yogasiddhāntacandrikā} Ed. p. 55-56.} und differenziert dementsprechend zweierlei Methoden des Vāsanāyoga.
Zunächst widmen wir uns der ersten:

\begin{quote}
\textit{avāntaravāsanāyogam āha}-
\textit{vītarāgaviṣayaṃ vā cittam} || 37 ||
\end{quote}
\begin{quote}
With regard to [the two different methods of] Vāsanāyoga he states: \\
Or, [the mind becomes stable when it is directed], on a mind that is without the desire for sense objects. 
\end{quote}

Im Kontext des ersten Kapitels des Pātañjalayogaśāstra nennt dieses \textit{sūtra} eine weitere Möglichkeit \textit{samādhi} zu erlangen. Die Möglichkeit zur Stabilisierung des Geistes ist hier die Meditation des Geistes von jemanden, dessen Geist bereits frei vom Verlangen nach Sinnesobjekten ist, etwa über den Geist einer Person, die bekannt dafür ist, diesen Zustand bereits erreicht zu haben. Das kann der eigene realisierte Lehrer sein, dies kann jedoch auch berühmter Yogameister der Vergangenheit sein. Insbesondere sollte der Geist der gewählten Person frei von \textit{vāsanā}s sein. Nārāyaṇatīrtha erklärt:

\begin{quote}
  \textit{vīteti} | \textit{vītarāgaṃ nirvāsanaṃ yat sanakādīnāṃ cittaṃ tadviṣayaṃ tadvibhāvanaparaṃ kuryāt} | \textit{nirvāsanavāsitam antaḥkaraṇaṃ kuryād iti yāvat} | \textit{anenātra yogino mumukṣālābhena vāsanāyogo darśitaḥ} |
\end{quote}
\begin{quote}
[Regarding the term] \textit{vīta} [`without']. On a mind without desire, without sublime impressions, which is like that of Sanaka and others, he shall be fully devoted to that reflection [which has] that [type of mind] as its object. To be precise, the mind shall be free from subliminal impressions. In this case, Vāsanayoga has shown [itself] through the attainment of the yogi's strong desire for liberation. 
\end{quote}

Das wichtigste Merkmal des gewählten Geistes ist die Freiheit von \textit{vāsanā}s. Wenn der richtige Geist als Meditationsobjekt gewählt wurde, äußert sich dies für den Übenden zunächst insbesondere durch ein gesteigertes Verlangen nach Befreiung. Im weiteren Verlauf des Kommentares zu 1.37 erklärt Nārāyaṇatīrtha weiterhin, dass Vāsanayoga vor allem zur Steigerung der sattvischen Geistesqualität führe. Dies wiederum würde auch die Effizienz aller anderen geübten Yogamethoden erhöhen.\footnote{Cf. \textit{Yogasiddhāntacandrikā} (Ed. p. 56) regarding \textit{sūtra} 1.37: \textit{uktañ ca smṛtau} - \textit{sattvāvalambanaṃ yat tad bījaṃ cittaviśodhane} | \textit{bhavet sa vāsanāyogo yogāntaravivarddhakaḥ} || \textit{iti} || `It is said in the Smṛti: That which supports the sattvic constitution is the primary cause for the the purification of the mind, this is the Vāsanāyoga which enhances the other Yogas.' Ich war leider nicht in der Lage die Quelle dieses Verse zu identifizieren.} Der Clue dieser Praxis, dass durch die Meditation über einen Geist der frei von \textit{vāsanā}s ist, dass automatsich mittels dieser Methode auch die eigenen \textit{vāsanā}s ausgelöscht werden.\footnote{Cf. Ibid.: \textit{tejaḥpratibandhajalaśaityavad iti vinaiva sādhanāntaraṃ yogino mokṣasukhaniṣṭhāsambhavāt} | \textit{ayaṃ śubho vāsanāyogo viruddhavāsanānivarttaka iti} || 37 || `As without that which is ``like cold water combined with heat'' is the yogi's inner practice, [for] this auspicious vāsanayoga is that which removes the blocking sublime impressions because from this arises the state of happiness and liberation for the yogi.'}\\

Die zweite Methode wird von Nārāyaṇatīrtha wiefolgt eingeleitet:
\begin{quote}
\textit{vāsanāyogasyāvāntaraṃ bhedam āha}-\\
\textit{svapnanidrājñānālambanaṃ vā} || 38 ||
\end{quote}
\begin{quote}
With regards to the [other] distinction of Vāsanayoga he says:\\ 
Or, [onto] the support of knowledge from dreams and sleep. 
\end{quote}

Nārāyaṇatīrtha erklärt diesbezüglich, dass während des Schlafes im Traum manche Menschen eine Vision der favorisierten Form des Göttlichen haben und andere wiederum Glück durch Schlaf erfahren. In diesem Fall soll man diese Erfahrungen als Meditationsobjekt einsetzen. Diese Methode funktioniere deshlab so gut, weil diese Erfahrungen auf vorherigen sehr sattvischen \textit{vāsanā}s beruhen. Diese Meditation über diese stiegert daher auch die sattvsiche Qualität im Wachzustand und führt somit zur Befreiung.\footnote{Cf. Ibid.: \textit{svapne bhagavato yadrūpaṃ priyam ārādhayann eva prabuddha, evaṃ nidrādau yatsukham anubhūyate tad avalambanaṃ tad vibhāvanaparaṃ cittaṃ kuryāt} | \textit{pūrvavāsanāprāptasattvapradhānam evāntaḥkaraṇaṃ kuryād iti yāvat} || 38 || `With regard to a dream, worshiping the divine in the favored form, similarly, when one is awake, the mind should make the happiness experienced during sleep, etc., the support; that is what should be contemplated. To put it plainly: The mind should indeed cultivate the predominance of purity obtained from previous impressions.'}\\

Somit steht die erste Methode des Vāsanayoga im starken Kontrast zur zweiten Methode des Vāsanayoga. Die erste Methode des Vāsanayoga reduziert \textit{vāsanā}s, indem der Übende den Geist auf einen anderen Geist richtet, der bereits seine \textit{vāsanā}s aufgelöst hat. Die zweite Methode nutzt gezielt sehr positive \textit{vāsanā}s, um die sattvische Qualität zu kultivieren, was ebenfalls einen Weg zu \textit{mokṣa} sein kann.   


\section{11. Śivayoga}
\label{sivayogaintro}

Rāmacandra positioniert Śivayoga an der elften Stelle seiner Taxonomie der fünfzehn Yogas. Die beiden Verse, welche in der \textit{Yogasvarodaya} die fünfzehn Yogas erwähnen, jedoch nur insgesamt acht davon auflisten, führen Śivayoga nicht als eigenständiges Thema ein. Das Śivayoga zu den fehlenden sieben Yogas gehören dürfte ist sehr wahrscheinlich, handelt es sich dabei doch insbesondere im Fall des \textit{Yogasvarodaya} eindeutig um einen Yogatext und ein Yogasystem, dass einem Śaiva Milieu entstammt. Von dieser Warte aus betrachtet, könnte das gesamte hierin präsentierte Yogasystem als Śivayoga betrachten. Tatsächlich existieren frappante Ähnlichkeiten zum Yogasystem der \textit{Śivayogapradīpipā}, die es sich hier lohnt aufzuzeigen. Ähnlich ist es im Fall des \textit{Yogatattvabindu}, weil sich die Lehrinhalte beider Texte kaum unterscheiden. Der Unterschied besteht jedoch darin, dass es Rāmacandra offenbar ein Anliegen war jegliche religiöse und sektarische Affiliation, die in seinen Quelltexten vorlagen, in seinem Rendering der Lehren auszublenden und völlig neutral wiederzugeben. Wenn Rāmacandra von einem Gott spricht, dann ausschließlich von \textit{īśvara}. In der \textit{Sarvāṅgayogapradīpikā} wird Śivayoga nicht erwähnt. Ein Śivayoga wäre im Milieu des Vaiṣṇava \textit{bhakti} eines Sants wie Sundardās auch nicht zu erwarten gewesen.\footnote{Cf. \citeauthor{horstmann2023shrine} 2023, p. 7.} Die einzige Beschreibung eines Śivayoga im Rahmen der Texte der komplexen spätmittelalterlichen Taxonomien findet sich erneut ausschließlich in Nārāyaṇatīrthas \textit{Yogasiddhāntacandrikā}. 

\section{Śivayoga in the \textit{Yogasvarodaya} and \textit{Yogatattvabindu}?}

Das \textit{Yogasvarodaya} und das \textit{Yogatattvabindu} widmen Śivayoga wird als Unterkategorie des Rājayoga keine eigene Sektion, wie dies bei anderen aufgelisteten Unterkategorien des Rājayoga der Fall war. Der Vergleich der Lehrinhalte beider Texte mit der \textit{Śivayogapradīpikā} einem der einflussreichsten Śivayogatexte überhaupt,\footnote{Erst vor kurzem wurde eine kritische Edition im Rahmen einer umfangreichen Dissertatiosstudie von \citeauthor{powell2023} (2023) abgeschlossen. An dieser Stelle möchte ich Dr. Seth \citeauthor{powell2023} danken, dass er mir noch der Veröffentlichung seiner Dissertation, seine Arbeit zur Einsicht zur Verfügung stellte.} wirft jedoch die Frage auf, ob nicht auch das gesamte in beiden Texten präsentierte Yogasystem als Śivayoga aufzufassen wäre, denn bereits Cennasadāśivayogin, der Autor der \textit{Śivayogapradīpikā} setzt Śivayoga und Rājayoga in Vers 1.13 gleich:
\begin{quote}
In reality, there is no difference between Śivayoga and Rājayoga. Yet for those who worship Śiva [a difference] is thus declared, in order to increase wisdom.\footnote{Translated by \citeauthor{powell2023} 2023, p. 315.}\footnote{\textit{Śivayogapradīpikā} 1.13: \textit{na bhedaḥ śivayogasya rājayogasya tattvataḥ} | \textit{śivārcināṃ evam ukto buddeḥ pravṛddhaye} || 13 ||} 
\end{quote}
Eine ähnliche Aussage findet sich ebenfalls in der \citetitle{yogasarasangraha}. Hier werden Rājayoga, Śivayoga, \textit{samādhi} und andere Bezeichnungen für den höchsten soteriologischen Zustand gleichgesetzt.\footnote{\citetitle{yogasarasangraha} p. 60: \textit{rājayogaḥ samādhiś conmanī ca manonmanī} | \textit{śivayogo layastatvaṃ śūnyāśūnyaṃ nirañjanam} || \textit{amanaskaṃ yathā caitannirālambaṃ nirañjanam} | \textit{jīvanmuktiś ca sahajam ity adir hy ekavācakam} ||}. 
Das \textit{Yogasvarodaya} ist ein Text des Rājayoga Genres, der einem Śaiva Milieu entsprungen ist. So heißt es im Text beipsielsweise, dass der Yogin als Kenner des ersten Typus des Jñānayoga den Rang eines Śiva genannten Erlösten erlangt,\footnote{ \textit{Yogasvarodaya} (PT Ed. p. 831): \textit{jñānayogaṃ pravakṣyāmi tajjñānī śivatāṃ vrajet} |}, dass der Yogin durch die Praxis von Haṭhayoga dem Śiva gleich wird,\footnote{Ibid. (PT Ed. p. 835): \textit{śivatulyo mahātmāsau haṭhayogaprasādataḥ} |} oder dass der Yogin als Ergebnis der Praxis des \textit{madhyalakṣya} einer ist, der in der Welt lustwandelt wie Śiva ohne Sünde oder Verdienst,\footnote{Ibid. (PT Ed. p. 839): \textit{śivavad vihared viśve pāpapuṇyavivarjitaḥ} |} Darüber hinaus wird im Abschnitt über \textit{yogamāhātmya} ein wahrer Lehrer (\textit{sadguru}) mit Śiva gleichgesetzt.\footnote{Ibid. (PT Ed. p. 848): \textit{nānāvikalpavibhrāntināśañca kurute tu yaḥ} | \textit{sadguruḥ sa tu vijñeyo na tu vairaprakalpakaḥ} | \textit{ata eva maheśāni sadguruḥ śiva āditaḥ} |} Es finden sich weitere Erwähnungen von Śiva in der \textit{Yogasvarodaya}. Rāmacandra hingegen bedient sich zwar großzügig bei der \textit{Yogasvarodaya} für die Kompilation seines Textes, blendet die śivaitischen Element aber weitestgehend systematisch aus, um religiöse Neutralität zu wahren.\footnote{Nur sehr wenige Passagen des \textit{Yogatattvabindu} verraten die śivaitische Abstammung der Inhalte. In Sektion \uproman{3} the central channel is qualified as being \textit{śivarūpiṇī} (``Śiva-gestaltig'' bzw. ``form of benevolence''). In Sektion \uproman{21}.3 wird der höchste soteriologische Zustand, der durch Jñānayoga hervorgebracht werden kann als \textit{śāmbhavīsattā} (``die zu Śiva gehörige Realität'') bezeichnet und in Sektion \uproman{48}.1 tauchen Śakti und Śiva als \textit{kula} und \textit{akula} in Rāmacandras Ausführungen zur Kosmogonie auf.}   
Die inhaltlichen Parallelen unserer Texte mit der \textit{Śivayogapradīpikā} sind frappant, sodass es im Hinblick auf die Fragestellung dieses Unterkapitles sinnvoll ist an dieser Stelle die Grundzüge dieser Ähnlichkeit darzustellen. Die \textit{Śivayogapradīpikā} von Cennasadāśivayogin wird von \citeauthor{powell2023} auf circa 1400 – 1450 n. u. Z. datiert.\footnote{\citeauthor{powell2023} 2023, p. 157.} Somit befinden wir bis zu rund zweihundert Jahre vor der Abfassung des \textit{Yogatattvabindu} und des \textit{Yogasvarodaya}. 
Im Gegensatz zu der fünfzehnfachen Yogataxonomie unserer Texte greift Cennasadāśivayogin auf das oftmals in der mittelalterlichen Yogaliteratur verwendete Modell von Mantra-, Laya-, Haṭha- und Rājayoga, welche als Unterkategorien von Śivayoga betrachtet werden.\footnote{\textit{Śivayogapradīpikā} 1.3-4: \textit{śivatattvavidāṃ śreṣṭha vakṣyāmi śṛṇu te 'dhūna} | \textit{śivayogaṃ paraṃ guhyam api tvadbhaktigauravāt} || 3 || \textit{mantro layo haṭho rājayogaś ceti caturvidham} | \textit{tam āhuḥ pūrvamunayaḥ siddhāḥ śaṃbhuprabodhitāḥ} || 4 ||} Wie bereits im obigen Zitat von \textit{Śivayogapradīpikā} 1.13 erwähnt, setzt Cennasadāśivayogin Śivayoga mit Rājayoga gleich, wobei darüber hinaus anzumerken ist, dass er Rājayoga in drei Unterkategorien aufteilt, nämlich Sāṅkhyayoga, Tārakayoga und Amanaska Rājayoga.\footnote{Ibid. \textit{Śivayogapradīpikā} 1.10-11: \textit{so 'pi tridhā bhavet sāṅkhyas tārakaś cāmanā iti} | \textit{pañcaviṃśatitattvānāṃ jñānaṃ tat sāṅkhyaṃ ucyate} || 10 || \textit{bahirmudrāparijñānād yogas tāraka ucyate} | \textit{antarmudrāparijñānād amanaska itīritaḥ} || 11 ||} Cennasadāśivayogin bezeichnet sein Sāṅkhyayoga abschließend auch als Jñānayoga.\footnote{Idid. 4.31.} Um seinen Text und dessen Lehren zu strukturieren verwendet Cennasadāśivayogin die acht Glieder des Aṣṭāṅgayoga.\footnote{Ibid. 2.4-5: \textit{śivayogaḥ sādhakānāṃ sādhyas tatsādhanaṃ haṭhaḥ} | \textit{tasmād ādau prayoktavyaṃ haṭhayogam imam śṛṇu} || 4 || \textit{aṅgāny aṣṭau haṭhasyāpi bāhyāny abhyantarāṇi ca} | \textit{yamādihir ato 'ṣṭāṅgair devapūjāṃ samācaret} || 5 ||} Dabei handelt es sich nicht um das Standard-Modell des achtgliedrigen Yoga des \textit{Pātañjalayogaśāstra}, sondern ein spezifisches Modell einer Gruppe von Texten, welche \textit{dhyāna} und \textit{dhāraṇa} vertauschen. Dieses Phänomen findet sich ansonsten nur in \textit{ṣaḍaṅga} oder \textit{pañcāṅga} Yogasystemen.\footnote{See table 10: \textit{Yogāṅgas with Dhyāna before Dhāraṇa} in \citeauthor{powell2023} 2023, p. 166 for an overview.} \citeauthor{powell2023} (2023: 168) erklärt, diese Vertauschung von \textit{dhyāna} und \textit{dhāraṇa} in einem achtgliedrigen System nur in der \textit{Śivayogapradīpikā} gefunden hat. Erst die kritische Edition des \textit{Yogatattvabindu}, insbesondere die Inspektion der ältesten Handschriften konnte zeigen, dass auch andere Texte mit achtgliedrigen Systemen diese Reihenfolge konservieren.\footnote{Siehe Sektion \uproman{31} in der kritischen Edition des \textit{Yogatattvabindu} auf p.\pageref{ashtanga}.} Darüber hinaus findet sich diese vertauschte Reihenfolge auch in der Überlieferung der eng mit der \textit{Śivayogapradīpikā} verknüpften \textit{Siddhasiddhāntapaddhati} in den Handschriften J\textsubscript{1} und J\textsubscript{2}.\footnote{Siehe krititsche Edition der \citetitle{ssplonavla} von \citeauthor{ssplonavla} (2016) zu Sektion 2.32 (Ed. p. 45).} Die Überlieferung der \textit{Yogasvarodaya} benennt zwar ein achtgliedriges Yoga, nennt im Vers der die Glieder auflistet nur \textit{dhāraṇa}, erläutert im Verlauf des Abschnittes allerdings \textit{dhyāna} und belässt \textit{dhāraṇa} unerklärt. Dieses Phänomen deutet auf eine enge rezeptionsgeschichtliche Verknüfung der vier involvierten Texte. Darüber hinaus listet die \textit{Śivayogapradīpikā} alle im Text benannten Yogas zwar nicht in er Taxonomie auf, zusammengenommen ergibt jedoch eine ähnliche Vielfalt wie in den spätmittelalterlichen Texten.\footnote{Insgesamt finden sich in der \textit{Śivayogapradīpikā} zehn Yogakategorien. Das gesamte System ist ein System des 1. Śivayoga, welches in ein System des 2. Aṣṭāṅgayoga eingebettet ist. Hierin werden 3. Mantrayoga, 4. Layayoga, 5. Haṭhayoga und 6. Rājayoga verortet. Letzteres teilt sich wiederum auf in 7. Sāṅkhyayoga = 8. Jñānayoga, 9. Tārakayoga und 10. Amanaska Rājayoga.}   
Im Kontext des vierten Gliedes \textit{prāṇāyāma} differenziert Cennasadāśivayogin drei Arten des \textit{prāṇāyāma}: 1. natürlich (\textit{prākṛta}), 2. modifiziert (\textit{vaikṛta}) und 3. \textit{kevalakumbhaka}, welches sich von selbst entfaltet, mit oder ohne die Praxis der beiden erstgenannten Varianten.\footnote{Cf. \textit{Śivayogapradīpikā} 2.22: \textit{prāṇāyāmas tridhā proktaḥ prākṛto vaikṛtas tathā} | \textit{dvābhyāṃ vinā jṛmbhate 'sau kevalaḥ kumbhakaḥ svayam} || 22 ||} Bei der ersten Variante\footnote{Ibid. 2.29-34} handelt es sich tatsächlich um das \textit{ajapā mantra}, welches auch von Rāmacandra in Sektion \uproman{3} angedeutet, bzw. im Rahmen von der Handschrift \getsiglum{U2} dezidiert im Kontext einer Meditation über die neun \textit{cakra}s instruiert wird. Das Mantrayoga der \textit{Śivayogapradīpikā} wird also dem \textit{prāṇāyāma} untergeordnet.\footnote{Siehe hierzu \citeauthor{powell2023} 2023, p. 205.} Die zweite Variante des des \textit{prāṇāyāma} ist deckungsgleich mit der in \textit{Yogatattvabindu} Sektion \uproman{31}.\footnote{Ibid. 22.4: \textit{āgamoktavidhānena recapūrasvabhāvataḥ} | \textit{yadi prāṇanirodhaḥ syād vaikṛtaḥ sa udītritaḥ} || 24 ||} Im dritten Kapitel der \textit{Śivayogapradīpikā},welches dem fünfen der acht Glieder \textit{dhyāna} gewidmet ist finden wir dann eine ausführliche Beschreibung der auch im \textit{Yogatattvabindu} und \textit{Yogasvarodaya} so zentralen Themen der neun \textit{cakra}s\footnote{Ibid. 3.7-16.} und der sechszehn \textit{ādhāra}s\footnote{Ibid. 3.17-32}. Die Beschreibungen der individuellen Elemente beider Themen sind größtenteils kongruent.  

Neben diversen Ähnlichkeiten gibt es auch signifikante Unterschiede zwischen den Texten. Beispielsweise beinhalten beide Texte Varianten des Jñānayoga (\textit{Śivayogapradīpikā} 4.31 bezeichnet Sāṃkhyayoga als Jñānayoga). Die \textit{Śivayogapradīpikā} lehrt ein System mit insgesamt fünfundzwanzig \textit{tattva}s plus \textit{puruṣa}.\footnote{Siehe \textit{Śivayogapradīpikā} 4.19-31. Außerdem wird System der \textit{tattva}s der \textit{Śivayogapradīpikā} asuführlich von \citeauthor{powell2023} 2023, pp. 239-42 analysiert.} \textit{Yogasvarodaya} und \textit{Yogatattvabindu} lehren ein simpleres System mit nur zehn \textit{tattva}s.\footnote{Cf. \textit{Yogatattvabindu} \uproman{31}.6 und \textit{Yogasvarodaya} (PT Ed. p. 836).} Während Cennasadāśivayogi zunächst eine große Seele (\textit{mahātman}) als eine Seele definiert, die weiß, dass das wahre Selbst (\textit{ātman}) ontologisch von den Evoluten der \textit{prakṛti} zu differenzieren ist,\footnote{\textit{Śivayogapradīpikā} 4.28: \textit{dehatrayaṃ prathitaṣoḍaśadhāvikārān liṅgāni saptadaśadhā navadhā padārthān} | \textit{ātmānām aṣṭavidhayā prakṛtisvabhāvaṃ jñātvā tad anya iti jīvati yo mahātmā} || 28 ||} verkündet er unmittelbar darauf jedoch die Nondualität von \textit{ātman} und \textit{brahman} im Sinne des Advaitavedānta.\footnote{Ibid. 4.29-30: \textit{satyaṃ jñānam anantaṃ yad brahmeti vadati śrutiḥ} | \textit{muktānandasvarūpaṃ ca nanu tat tvam asi sthiram} || 29 || \textit{naitad ahaṃ naidrad ahaṃ ceti yad anyaṃ vibhāvayātmānam} | \textit{so 'haṃ iti so 'ham iti nanu bhāvaya sarvaṃ tvam ātmānam || 30 ||}} \textit{Yogasvarodaya} und \textit{Yogatattvabindu} hingegen lehren einen radikle Non-dualität, die radikale Einheit von Allseele, Individualseele und Schöpfung,\footnote{Siehe \textit{Yogatattvabindu} Sektion \uproman{21}.7 und \textit{Yogasvarodaya} (PT Ed. p. 836).} was stark an Vallabha's Śuddhādvaita erinnert.\footnote{Siehe \citeauthor{glasenapp1949philosophie} 1985, pp. 270–72.}
 Im Rahmen des Tārakayoga im vierten Kapitel der \textit{Śivayogapradīpikā}\footnote{Ibid. 4.32-52.} werden die drei \textit{lakṣya}s \textit{antar}-, \textit{bāhya}- und \textit{madhyalakṣya} gelehrt, wohingegen in \textit{Yogasvarodaya} und \textit{Yogatattvabindu} fünf \textit{lakṣya}s gelehrt werden.
Es existieren weitere Unterschiede, aber der wahrscheinlich zentralste Unterscheid ist, dass alle Lehren in Cennasadāśivayogins \textit{Śivayogapradīpikā} in den rituellen und devotionalen Rahmen der Vīraśaivas eingebettet sind.\footnote{\citeauthor{powell2023} 2023, p. 8.} So definiert Cennasadāśivayogin Śivayoga in Vers 1.15 als:
\begin{quote}
  Śivayoga is five-fold, indeed: gnosis (\textit{jñāna}) comprised of Śiva, devotion (\textit{bhakti}) to Śiva,
meditation (\textit{dhyāna}) comprised of Śiva, Śaiva religious observance (\textit{vrata}), and worship of Śiva
(\textit{arcā}).\footnote{\textit{Śivayogapradīpikā} 1.15: \textit{jñānaṃ śivamayaṃ bhaktiḥ śaivī dhyānaṃ śivātmakam} | \textit{śaivavrataṃ śivārceti śivayogo hi pañcadhā} || 15 || Translation by \citeauthor{powell2023} 2023, p. 315.} 
  \end{quote}
  Trotz der klaren Śaiva Affiliation des \textit{Yogasvarodaya} lassen sich diese Elemente dort nirgends finden. Gleiches gilt für das \textit{Yogatattvabindu}. Selbst das achtgliedrige (\textit{aṣṭāṅga}) Schema wird in diesem Text als rituelle Verehrung von Śiva (\textit{śivapūja}) betrachtet\footnote{Cf. Ibid. 2.1-5.} und \citeauthor{powell2023} (2023) schlussfolgert, dass es eben diese hingebungsvolle und rituelle Ausrichtung macht es zum Śivayoga.\\

  Kann man demnach sagen, dass die Yogasysteme der \textit{Yogasvarodaya} and \textit{Yogatattvabindu} implizit Śivayoga lehren. Diese Frage lässt sich, wie gezeigt wurde nicht ganz eindeutig beantworten. Es ist Fakt, dass auf der Ebene der Lehrinhalte alle drei Texte kaum voneinander zu unterscheiden sind. Inhaltlich gesehen müsste diese Frage tendenziell positiv beantwortet werden. Die stark śivaitische Ausrichtung,\footnote{Das Wort \textit{śiva} wird in der \textit{Śivayogapradīpikā} insgesamt neunungsiebzig Mal erwähnt.} wie sie in der \textit{Śivayogapradīpikā} zu erkennen ist, ist jedoch in der \textit{Yogasvarodaya} und dem \textit{Yogatattvabindu} weitestgehend abwesend und beide Texte ordnen den Śivayoga faktisch dem Rājayoga unter. Der Grad der śivaitische Ausrichtung im \textit{Yogasvarodaya} ist mit zehn Erwähnungen des Wortes \textit{śiva} eher mäßig und im \textit{Yogatattvabindu} fast vollständig erloschen. Aus diesem Blickwinkel heraus muss die Fragestellung dieses Unterkapitels eindeutig negativ beantwortet werden. Nichstdestotrotz, wäre vor dem hier präsentierten Hintergund die mysteriöse Präsenz der Kategorie Śivayoga in den fünfzehnfachen Taxonomien, welche den Śivayoga als Unterkategorie des Rājayoga, zu unserem Leidwesen nicht explizit erläutern, leicht zu erklären. Śivayoga und Rājayoga wären gemäß der eingangs genannten Auffassung von Cennasadāśivayogin inhaltlich weitestgehend Deckungsgleich und somit als Synonyme zu betrachten. Dies würde die Abwesenheit einer gesonderten Widmung einer einzelnen Sektion, welche explizit Śivayoga erläutert völlig überflüssig machen. Es scheint als habe Rāmacandra die Auffassung Cennasadāśivayogin geteilt.        

  Außerdem lassen die frappanten inhaltlichen Ähnlichkeiten, wie etwa die spezielle Reihenfolge der acht Glieder der Aṣṭāṅgayogas, keinen anderen Schluss zu, als dass die \textit{Śivayogapradīpikā} und das \textit{Yogasvarodaya} und hierüber auch das \textit{Yogatattvabindu}, welches verwobenerweise auch auf die \textit{Siddhasiddhāntapaddhati} zurückgreift, einem Text, welcher der \textit{Śivayogapradīpikā} wiederum extrem Nahe steht\footnote{For a discussion of the relationship between the \textit{Śivayogapradīpikā} and \textit{Siddhasiddhāntapaddhati} see \citeauthor{powell2023} 2023, pp. 147-52.} aus dem gleichen intertextuellen Netzwerk entspringen. Die Inklusion des Śivayoga in die 

\section{Śivayoga in the \textit{Yogasiddhāntacandrikā}}



\section{12. Brahmayoga}
\label{sivayogaintro}

\section{13. Advaitayoga}
\label{advaitayogaintro}
\section{14. Siddhayoga}
\label{siddhayogaintro}

XLIV.10 One who is always indifferent, peaceful and immersed in great bliss by
means of Siddhayoga196 is said to be a Siddhayogin

\section{15. Rājayoga}
\label{rajayogaintro}

\section{Other Yogas}

\subsection{Urayoga}
\subsection{Premabhaktiyoga}
\subsection{Aṣṭāṅgayoga}
\subsection{Sāṃkhyayoga}
\subsection{Sahajayoga}
\subsection{Satyayoga}



\end{document}

